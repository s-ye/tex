\documentclass[12pt]{article}
\usepackage[english]{babel}
\usepackage[utf8x]{inputenc}
\usepackage[T1]{fontenc}
\usepackage{listings}
\usepackage{tikz}
\usepackage{/Users/songye03/Desktop/math_tex/style/quiver}
\usepackage{/Users/songye03/Desktop/math_tex/style/scribe}

\begin{document}
Songyu Ye

\today

\hfill

This is a note regarding what I understand about toric origami manifolds in preparation for my meeting with Tara.

\section{Introduction}

The object of study are origami manifolds and in particular toric origami manifolds. These are smooth manifolds equipped with a closed form which
is almost nondegenarate in the sense that the degeneracy is well behaved.

\begin{definition}
    Let $M,\omega$ be a smooth manifold and $\omega$ closed form. We say that a $(M,\omega)$ is 
    a folded symplectic form if $\omega^n$ vanishes transversally on a submanifold $Z$. If this is the case,
    then \red{$Z = (\omega^n)^{-1}(0)$ is necessarily an embedded submanifold of codimension $1$} and is called the fold locus.
\end{definition}

If we are in this case, $i^*\omega$ has a $1$-dimensional kernel at each point of $Z$. This gives us a line field on $Z$ 
called the null foliation.

\begin{definition}
    Foliation
\end{definition}

\begin{definition}
    Further assume that the null foliation is fibrating, i.e. there is a fiber bundle $Z\to B$ over a compact base with orientable $S^1$-fibers.
    Then we say that $(M,\omega)$ is an origami manifold.
\end{definition}

\section{Examples}

\begin{example}
    $\R^{2n}$ can be equipped with the following closed form \begin{align*}
        \omega = x_1 dx_1 \wedge dy_1 + \sum_{i=2}^n dx_i\wedge dy_i
    \end{align*}
    The top power of this form is \begin{align*}
        \omega^n = x_1 dx_1 \wedge dy_1 \wedge dx_2 \wedge dy_2 \wedge \cdots \wedge dx_n \wedge dy_n
    \end{align*} which vanishes along the codimension $1$ submanifold $Z = \{x_1 = 0\}$.
    \red{This is not origami because the null foliation is not fibrating.}
\end{example}

\begin{example}
    Consider the unit sphere $S^{2n}\subset \R^{2n+1}$ with coordinates $(x_1,y_1,\ldots,x_n,y_n,h)$. The form \begin{align*}
        \omega = dx_1 \wedge dy_1 + \dots + dx_n \wedge dy_n = r_1dr_1\wedge d\theta_1 + \dots + r_n dr_n \wedge d\theta_n
    \end{align*} is folded \red{why does the top power of this form vanish?} along $h=0$, a copy of $S^{2n-1}$. The \red{null foliation is the Hopf foliation because} \begin{align*}
        i_{\frac{\partial}{\partial\theta} + \dots + \frac{\partial}{\partial\theta}}\omega = -r_1dr_1 - \dots - r_n dr_n
    \end{align*} vanishes on $Z$. Hence the null fibration is $S^1\to S^{2n-1}\to S^{2n-1}/S^1 = \C\P^{n-1}$.

    \hfill

    In particular this example is origami.
\end{example}

\begin{example}
    This example carries over to $\R\P^{2n} = S^{2n}/Z_2$ with the form \begin{align*}
        \omega = dx_1 \wedge dy_1 + \dots + dx_n \wedge dy_n
    \end{align*} is $\Z_2$ invariant and therefore descends to a form on $\R\P^{2n}$. The fold locus is $[x_1,y_1,\dots,x_n,y_n,0]$
    a copy of $\R\P^{2n-1}$. The null foliation is the $\Z_2$ quotient
    of the Hopf foliation and it looks like $S^1\to \R\P^{2n-1}\to \C\P^{n-1}$.

    \hfill

    In particular this example is origami.
\end{example}

The notion of a moment map depends only on $\omega$ being closed. Therefore, we can 
still make sense of torus actions on origami manifolds which are Hamiltonian and effective.

\begin{example}
    There is a $T^2$ action on $S^4 \subset \C^2\oplus \R$
    rotating the first two coordinates which makes it into a toric origami manifold. 
    The moment map \begin{align*}
        \mu(z_1,z_2,h) = (|z_1|^2, |z_2|^2)
    \end{align*} and the image is two overlapping copies of a triangle.
    \begin{center}
        \includegraphics[scale = .5]{/Users/songye03/Desktop/math_tex/img/s4.png}
    \end{center}
        
    The fold locus is the preimage of the hypontenuse which is a copy of $S^3$.
    The null fibration is happening over the hypotenuse and the fibers are $S^1$.

    \red{Note that if we sort of cut along the fold and take closures,
    we get two copies of $\C\P^2$ glued along a copy of $\C\P^1$. This 
    motivates the classification of toric origami manifolds by templates.}

    \hfill 

    \red{What does the moment image for $\R\P^4$ look like?}
\end{example}

\section{Cutting and radial blowups}
Cutting is analagous to the blowdown map in algebraic geometry. The fold $Z$ is analagous 
to the exceptional divisor. 

\hfill

Example 2.4 demonstrated a very simple example of how we should classify toric origami manifolds.
The idea is to cut along the fold locus and take closures. In this way 
we obtain a collection of toric symplectic manifolds and we need to keep track of
the gluing data. This is the fashion in which origami templates classify toric origami manifolds.

\hfill

Consider the closures of the connected components of $M\backslash Z$ and collapse the fibers of the
null fibration. The result is a disconnected collection of symplectic manifolds $M_i$,
collectively referred to as the symplectic cut space of $M$. We need to 
keep track of which components were connected before we cut. In particular if the circle fibers of $Z_0$ a connected component of $Z$ are orbits
for a circle subgroup of $T$ then $\Phi(Z_0)$ is a facet of the moment polytopes of 
neighboring components of $M\backslash Z$. This is the origami template.

\begin{example}
    The symplectic cut space of a torus is the union of two $\C\P^1$s.
    \begin{center}
        \includegraphics[scale = .5]{/Users/songye03/Desktop/math_tex/img/Screenshot 2024-03-01 at 8.06.53 PM.png}
    \end{center}
    \red{How can we see this on the level of moment images?}

    The moment image/origami template looks like \begin{center}
        \includegraphics[scale = .3]{/Users/songye03/Desktop/math_tex/img/Screenshot 2024-03-01 at 8.23.34 PM.png}
    \end{center}

\end{example}

\begin{definition}
    An origami template is a graph $G = (V,E)$ along with maps \begin{align*}
        \Psi_V:V \to D_n \\
        \Psi_E:E \to E_n
    \end{align*} where $D_n$ is the set of 
    Delzant polytopes in $\R^n$ and $E_n$ is the set of subsets of $\R^n$ which are
    facets in some Delzant polytope. 
\end{definition}

\begin{example}
\red{This example is confusing me because I thought that the Hirzebruch surface 
was a toric symplectic manifold. Why is it folded and why does the moment graph
look like that as opposed to a trapezoid? How does one make sense of the GKM conditions?
}
\begin{center}
        \includegraphics[scale = .5]{/Users/songye03/Desktop/math_tex/img/Screenshot 2024-03-01 at 8.27.26 PM.png}
    \end{center}
\end{example}

To read off the GKM conditions I am supposed to look at the moment image of the $1$-skeleton. 
But now in our situation there are multiple polytopes and gluing data and now I 
am not sure how to read off the GKM conditions.

\begin{example}
    For example, these templates are giving me a lot of different polytopes and gluing data.
    What exactly is moment image here, i.e. the fixed points and the $1$-dimensional orbits?
    \begin{center}
        \includegraphics[scale = .5]{/Users/songye03/Desktop/math_tex/img/Screenshot 2024-03-01 at 8.31.54 PM.png}
    \end{center}
    Also what are these manifolds?
\end{example}

\red{In general what is the procedure to take a template and produce a toric origami manifold?
For example, what does the toric origami manifold corresponding to the template where I have 3 triangles
glued along their hypontenuse look like?}

\section{Questions}
Holm and Ana showed that when the fold is 
coorinentable meaning that it admits an orientable neighborhood, and when the template graph is acyclic,
then the origami manifold has cohomology concentrated in even degree 
and is therefore $GKM$.

\hfill

We are interested in questions regarding rigidity and realization for this
fixed family of toric origami manifolds.

\hfill

Lots of things are known.

The homeomorphism type of toric manifolds is completely determined by
the integral cohomology ring in complex dimensions $1$ and $2$.

If you consider the family of toric symplectic manifolds, then there are also some results. 
For example people know many things about Bott manifolds. $B_n$ is diffeomorphic to $(S^2)^n$ if
they have isomorphic integral homology.

\hfill

Is there any hope of making any sort of statement like above for the family of toric origami manifolds?
For example, if you are a toric origami manifold and you have the same integral cohomology ring as $M$
then are you $M$? It seems like one would have to look in dimension $6$.

\section{After the meeting}
I met with Tara on 3/4 and we discussed the above. She remarked the following.

\hfill

The problem that I am interested in reminds her of the work that she did with Liat Kessler. In particular they were studying circle actions on 4 manifolds, 
in particular they had an $n-1$ torus acting on a $2n$-manifold. They showed rigidity results in those case by exhibiting a diffeomorphism whenever two cohomology rings
had the same generator and relations presentation. In general she remarked that understanding GKM data is often not that good because it exhibits the cohomology ring 
as a \textbf{subring} and it is generally quite difficult to understand the cohomology ring as a subring, it is much better to understand it presented with generators and relations.

\hfill

To this avail, she pointed me toward tools that one uses to identify geometric meaning in the cohomology ring, in particular Chern classes. 
She said that her hope was that perhaps one can visibly read off Chern classes and that this would help determine the rigidity of toric origami manifolds.

\hfill

She pointed me toward several resources in this regard. Her notes from Spring 2012 Bernstein Seminar and this paper by Andrzej Weber.
\end{document}