\documentclass[12pt]{article}
\usepackage[english]{babel}
\usepackage[utf8x]{inputenc}
\usepackage[T1]{fontenc}
\usepackage{listings}
\usepackage{hyperref}
\usepackage{tikz}
\usepackage{/Users/songye03/Desktop/math_tex/style/quiver}
\usepackage{/Users/songye03/Desktop/math_tex/style/scribe}

\begin{document}
Songyu Ye

\today

\hfill

This is a note from my meeting with Allen on March 22, 2024. Recall what we began by looking at.
In particular Allen did not object to anything I wrote down.

\begin{center}
    \includegraphics[scale = .13]{/Users/songye03/Desktop/math_tex/img/IMG_1876.JPG}
\end{center}

\section{Decomposition}
There is some combinatorics to do involving the decomposition of $\Sym^2(V\otimes W)$. In particular
it decomposes into two pieces $\Sym^2(V)\otimes \Sym^2(W)$ and $\Lambda^2(V)\otimes \Lambda^2(W)$
(by dimension count). We understand the decomposition of the first piece $\Sym^2(V)$ by applying Pieri rule
to $V\otimes V$ and then identifiying the fact that only the even pieces land in $\Sym^2(V)$.

\begin{center}
    \includegraphics[scale = .13]{/Users/songye03/Desktop/math_tex/img/IMG_1874.JPG}
\end{center}

Allen mentioned some multiplicity issue to worry about. In particular we need to understnad the multiplicity 
of the pieces which look like $V\otimes V^*$ in the decomposition of $\Sym^2(V\otimes W)$.
One thing we did identify is that things which are not in the antidiagonal 
(i.e. things which look like $V\otimes V^*$) are for sure in the kernel.


\section{Another "more stupid" degeneration}
Allen suggested another more stupid degeneration that we could try that would sort of be a preliminary step.
It amounts to knowing the leading term of the relations and how many there are. It comes, from a 
submonoid of dominant weights, assigning a representation which has not just a $G$ action 
but a larger $G\times T$ action. We equip 
this vector space with the structure of the ring via Schur's lemma and then 
we can try to find $\Spec$ of this ring.

\hfill

Allen points out that what we want to get a hold of is generators for the monoid $(\mu,\lambda)$ with 
$\mu \leq \lambda$ in dominance order. This is the stuff we are looking at in red.






\end{document}