\documentclass[12pt]{article}
\usepackage[english]{babel}
\usepackage[utf8x]{inputenc}
\usepackage[T1]{fontenc}
\usepackage{listings}
\usepackage{tikz}
\usepackage{/Users/songye03/Desktop/math_tex/style/quiver}
\usepackage{/Users/songye03/Desktop/math_tex/style/scribe}

\begin{document}
Songyu Ye

\today

\hfill

This is a note for the talk with Mike Stillman on Feb 22, 2024.

Note: next week we are talking about Cartier divisors, line bundles, and Chern classes.

\section{Introduction}
The goal of intersection theory is to make sense of the interscetion product. Its existence rests on Chow's moving lemma: 

\begin{theorem}[Chow's moving lemma]
	Let $X$ be a smooth projective variety.
	\begin{itemize}
		\item Given $\alpha,\beta \in A(X)$ there exist representing cycles $A$ and $B$ intersecting transversally.
		\item The classes of the interesction of these cycles is independent of $A$ and $B$.
	\end{itemize}
\end{theorem}
It seems that for a long time this theorem was folklore in the algebraic geometry community. According 
to Eisenbud and Harris, the first proof was given by Fulton in 1984.

\hfill

There is an intersection product on $A(X)$:
\begin{theorem}
    If $X$ is a smooth quasiprojective variety, then there exists a unique product on $A(X)$ so that if two subvarieties 
    $A,B$ are generically transverse then $[A]\cdot [B] = [A\cap B]$.
\end{theorem}
If we are in a setting where Chow's moving lemma holds, then the intersection product is characterized by this property. 
If we cannot find generically transverse representatives, then we need to start making sense of multiplicitiies. 

\begin{theorem}
    Let $X$ be a smooth projective variety and $A,B$ subvarieites so that each component $C$
    of the intersection has the right dimension. Then there are positive integers $m_C(A,B)$ so that
    \begin{align*}
        [A]\cdot [B] = \sum_C m_C(A,B)[C].
    \end{align*}
    In case $A$ and $B$ are Cohen–Macaulay at a general point of C,
    then this number is multiplicity of the component of the scheme $A \cap B$ supported on C
\end{theorem}
Serre found a formula for this number as the alternating sum of the dimension of the Tors.
\red{If $A$ and $B$ are not intersecting properly, what should the intersection product of their classes be?}

\section{Examples}
\begin{theorem}
    $A(\P^n) \cong \Z[x]/(x^{n+1})$ where $x$ is the class of a hyperplane. If $X$ is a irreducible 
    subvariety of codimension $r$ and degree $d$, then $[X] = dx^r$.
\end{theorem}

\begin{proof}
    Each $A^i(\P^n) \cong \Z$ as a group because of the following theorem.
    \begin{theorem}
        If a scheme $X$ admits a quasiaffine stratification, then the classes of the closed strata generate $A(X)$.
        If the stratification is affine, then the classes of the strata form a basis.
    \end{theorem}
    \begin{proof}
        Follows from excision.
    \end{proof}
    Write $\P^n = \cup \A^i$ and therefore $A(\P^n)$ 
    is generated by the classes of the closures $\P^i$ of $\A^i$, a copy of $\Z$ for each $i$. None of the
    $\P^i$ are equivalent to zero because it is the pushforward 
    of the class $[\P^i]$ under the inclusion of $\P^i\hookrightarrow \P^n$

    \hfill

    \red{This statement very much reminds me of cellular homology. The CW decomposition of $\C\P^n$ has 
    only even dimensional cells and therefore classes of these cells generate the cohomology ring (the boundary maps go from even to odd)}

    \hfill

    To compute the ring product, note that \begin{itemize}
        \item a generic plane $L$ of codimension $r$ is the transverse 
        intersection of $r$ hyperplanes, so $[L] = x^r$.
        \item If $X$ as above, then $[X]x^{n-r} = [X \cap L_{n-k}] = d[*] = d\zeta^n$.
    \end{itemize}
\end{proof}
\begin{example}
    [Degree of Veronese embedding]
    The Veronese embedding is the map $v:\P^n \to \P^{\binom{n+d}{d}-1}$ given by 
    \begin{align*}
        [x_0:\cdots:x_n] \mapsto [x^I]
    \end{align*}
    where $z^I$ is a monomial of degree $d$ in $n+1$ variables. The image is the image of the 
    map of the complete linear system $|\mc{O}(d)|$. Letting $H$ be a 
    hyperplane in the target, $\inv{v}(H)$ comprises all hypersurfaces of degree $d$.

    \hfill 

    To see the degree, consider the image $\cap H_1 \cap \cdots \cap H_n$ of $n$ hyperplanes in the target.
    The map is $1:1$ and the preimage is the intersection of the preimages of the $H_i$ which is the intersection of $n$ hypersurfaces
    of degree $d$. The degree of the intersection is the product of the degrees of the hypersurfaces, which is $d^n$.
\end{example}

\begin{example}
    [Product of projective spaces]
    $A(\P^r\times \P^s) \cong A(\P^r)\otimes A(\P^s)$. If $\alpha,\beta$ denote pullbacks via projection of hyperplane
    classes on $\P^r$ and $\P^s$, then \begin{align*}
        A(\P^r\times \P^s) = \Z[\alpha,\beta]/(\alpha^{r+1},\beta^{s+1})
    \end{align*} Moreover the class of hypersurface $V(f)$ where $f$ is a form of bidegree $(d,e)$ is $d\alpha + e\beta$.

    \hfill

    There is an affine stratification given by the product of the affine stratifications of the factors
    so the classes $\phi_{a,b} = [\P^{r-a} \times \P^{s-b}]$ generates $A(\P^r\times \P^s)$. We have \begin{align*}
        \phi_{a,b} = \alpha^a\beta^b
    \end{align*} because $\P^{r-a} \times \P^{s-b}$ is the transverse intersection of $a$ hyperplanes in the first factor and $b$ hyperplanes in the second factor.
    Moreover it is clear that $\alpha^{r+1} = 0$ and $\beta^{s+1} = 0$. Therefore, there is an onto map \begin{align*}
        \Z[\alpha,\beta]/(\alpha^{r+1},\beta^{s+1}) \to A(\P^r\times \P^s)
    \end{align*} and it is injective because there is a pairing \begin{align*}
        A^{p+q}(\P^r\times\P^s) \times A^{r+s-p-q}(\P^r\times\P^s) \to \Z
    \end{align*} given by the intersection product which sends $(\alpha^p\beta^q,\alpha^m\beta^n)$ to $\delta_{p+m,r}\delta_{q+n,s}$.

    \hfill

    Finally if $F$ is bihomogeneous of bidegree $(d,e)$, then $F/X_0^dY_0^e$ is a rational function on $\P^r\times\P^s$ and its divisor is $[F] - d\alpha + e\beta$.
\end{example}

\begin{example}
    We can use this to immediately calculate the degree of the Segre embedding $s:\P^r\times \P^s \to \P^{rs+r+s}$ \begin{align*}
        [x_0:\cdots:x_r]\times [y_0:\cdots:y_s] \mapsto [x_iy_j]
    \end{align*}
    We consider the intersection with the image of $r+s$ hyperplanes. We pullback to $\P^r\times\P^s$
    because the map is an embedding and intersect with the preimage of the hyperplanes. The function $X_iY_j$ is bihomogeneous
    of bidegree $(1,1)$ so \begin{align*}
        [S_{r,s}] &= ([\alpha] + [\beta])^{r+s} \\
        &= \binom{r+s}{r}\alpha^r\beta^s
    \end{align*}
\end{example}

\begin{example}
    [Blowup of $\P^n$ at a point]
    Let $B\subset \P^n\times \P^{n-1}$ be the blowup of $\P^n$ at a point $p$. I like to think of
    this as the incidence variety of lines in $\P^n$ and points in $\P^{n-1}$. \begin{align*}
        B = \{(x,L) \in \P^n\times \P^{n-1}\mid x\in L\}
    \end{align*}
    We want to write down the Chow ring of $B$. We need to write down an affine stratification of $B$.

    \hfill

    Pick $\Lambda' \subset \P^n$ hyperplane not containing $p$ and consider $\Lambda \subset B$ the preimage of $\Lambda'$.
    Then $\Lambda \cong \P^{n-1}$ is those $(q,l)$ so that $q\in \Lambda'$ and $l$ is a line through $q$ and $p$.

    \hfill

    Pick $\Gamma_0' \subset \Gamma_1'\dots \subset \Gamma_{n-1}' = \P^{n-1}$ and set
    $\Gamma_k = \pi_l^{-1}(\Gamma_k')\subset B$ where $\pi_l:B\to \P^{n-1}$ is the projection and put 
    $\Lambda_k = \Gamma_{k+1} \cap \Lambda$. The situation looks like the following \begin{center}
        \includegraphics[scale = .1]{/Users/songye03/Desktop/math_tex/img/IMG_1808.JPG}
    \end{center}
    The open strata are all affine. 

    \hfill

    Take $\Gamma_k$ and rip out $\Gamma_{k-1}$ and I have affine space, but with $\P^1$ fiber over each point.
    Once I rip out $\Lambda_k$, I have an $\A^1$ fiber over each point is the product of affine spaces. 
    \red{All bundles over affine spaces are trivial.}

    \hfill

    This shows that the classes $\lambda_k = [\Lambda_k]$ and $\gamma_k = [\Gamma_k]$ generate $A(B)$.

    \hfill

    Now we make heavy use of the moving lemma. Since $\Lambda_k$ is the preimage of $k$-plane in $\P^n$
    not containing $p$ and all such planes are rationally equivalent in $\P^n$, 
    the classes of their pullbacks are all $\lambda_k$.

    \hfill

    Similarly the class of the proper transform (meaning rip out $p$, take preimage, and take closure) 
    of any $k$-plane in $\P^n$ is $\gamma_k$. \red{The preimage of hyperplane $H \subset \P^n$ 
    is the sum of irreducible divisors $E$ and $\tilde H$ where $E$ is the exceptional divisor and $\tilde H$ is the proper transform. 
    $\tilde H$ looks like all the lines (points of $\P^{n-1}$) in $H$.}

    \hfill

    With the representatives we can compute \begin{align*}
        \lambda_k\lambda_l = \lambda_{k+l} \texty {if $k+l\geq n$} \\
        \gamma_k\lambda_l = \lambda{k+l} \texty {if $k+l\geq n$} \\
        \gamma_k\gamma_l = \gamma_{k+l} \texty {if $k+l\geq n+1$}
    \end{align*} Note that if we have $k+l = n$ in the last case, then two general representatives
    do not meet. An example of this is the following \begin{center}
        \includegraphics[scale = .1]{/Users/songye03/Desktop/math_tex/img/IMG_1809.JPG}
    \end{center}
    We see that the pairing \begin{align*}
        \lambda_k \cdot \lambda_{n-k} = \lambda_k \cdot \gamma_{n-k} = 1 \\
        \gamma_k \cdot \gamma_{n-k} = 0
    \end{align*} and therefore the classes $\lambda_0,\dots, \lambda_{n-1},\gamma_1,\dots,\gamma_n$ form a basis for $A(B)$.

    \hfill

    Finally we consider the exceiptional divisor $E$. It has $\dim E = n-1$ and therefore
    we should be able to write it as a combination of $\gamma_{n-1}$ and $\lambda_{n-1}$. 

    \hfill

    $\Lambda_{n-1}'$ is linearly equivalent in $\P^n$ to hyperplane $\Sigma \subset \P^n$
    containing $p$. The pullback of $\Sigma$ is linearly equivalent to $E + D$ some divisor $D$ 
    \red{pullback along the blowup map preserves linear equivalence? but the blowup map 
    is not flat since it doesn't have equidimensional fibers}.

    \hfill

    $D$ projects to a hyperplane in $\P^{n-1}$ and consider $Y$ the preimage of such hyperplane. $Y$ 
    is irredicuble since it is a $\P^1$-bundle over its image and therefore $Y=D$. Therefore $D \sim \gamma_{n-1}$ 
    since they are both linearly equivalent to the preimage of a hyperplane in $\P^{n-1}$.

    \hfill

    We get ahold of the formula \begin{align*}
        E = \lambda_{n-1} - \gamma_{n-1}
    \end{align*} There are relations \begin{align}
        \lambda \cdot E = 0 \texty {since they don't meet} \\
        \gamma_k = \gamma^{n-k}_{n-1} = (\lambda_{n-1} - E)^{n-k} = \lambda_{n-1}^{n-k} + (-1)^{n-k}E^{n-k} \\
        0 = \lambda_{n-1}^n + (-1)^nE^n
    \end{align}
    so there is a map \begin{align*}
        \Z[\lambda_{n-1},E]/(\lambda_{n-1}^n + (-1)^nE^n) \to A(B)
    \end{align*} and it is surjective because $\lambda,e$ generate $A(B)$.
    It is injective injective $A^m(B)$ is a free $\Z$-module of rank $2$ for each $0<m<n$.

    \hfill

    One reason we compute Chow groups is because they are supposed to tell us about 
    the geometry of the variety. Let's compute $E^2$. We see that \begin{align*}
        E^2 &= E\cdot \lambda_{n-1} - E\cdot \gamma_{n-1} \\
        &= [E\cap (\Lambda - \Gamma)] \\
        &= -[E\cap \Gamma_{n-1}] \\
    \end{align*} is minus the class of a hyperplane in $E$.
    \red{How do I make sense of this? What exactly happened in the second equality?}
    \red{In what sense is sum about union and difference is about set difference?
    The class of the union of two irreducible subvarieties is equivalent to the sum 
    of the classes of the subvarieties. This follows from the defintion of rational equivalence. 
    One must be more careful when there are coefficients involved.}
\end{example}

\section{References}
\begin{itemize}
    \item 3264 Chapter 2
\end{itemize}
\end{document}