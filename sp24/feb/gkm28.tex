\documentclass[12pt]{article}
\usepackage[english]{babel}
\usepackage[utf8x]{inputenc}
\usepackage[T1]{fontenc}
\usepackage{listings}
\usepackage{tikz}
\usepackage{/Users/songye03/Desktop/math_tex/style/quiver}
\usepackage{/Users/songye03/Desktop/math_tex/style/scribe}

\begin{document}
Songyu Ye

\today

This is what Allen and I talked about on Feb 7.

\section{Gelfand Cetlin}
Consider the irreducible representation of $\GL(n)$ with highest weight $\lambda \in \Z^n$ where 
$\lambda_1\geq \cdots \geq \lambda_n$. One can consider $V_\lambda$ as a $\GL(n-1)$ representation
and decompose it into irreducible representations of $\GL(n-1)$. Which irreps appear and with what
multiplicity? This is the Gelfand Cetlin problem.

\hfill

The answer is given by the Gelfand Cetlin patterns. In particular, \begin{align*}
    V_\lambda = \bigoplus_{\mu} (V_\mu \otimes \C_{\sum\lambda - \sum\mu})
\end{align*} where the sum is over all $\mu$ such that $\lambda_1\geq \mu_1\geq \lambda_2\geq \mu_2\geq \cdots \geq \lambda_n$.
Remarkable each such $\mu$ shows up and it shows up with multiplicity one.

One can keep going all the way down to $\GL(1)$. The irreps which appear correspond to the 
different ways of filling out a Gelfand Cetlin pattern.

\begin{example}
    The simplest example is when $n=3$. In this case we have the following picture:
    \begin{center}
        \begin{tikzpicture}
            \draw (0,0) -- (4,0);
            \draw (4,0) -- (2,3.46);
            \draw (2,3.46) -- (0,0);
            
            % Nodes along the bottom edge
            \node at (0,0) [below] {$\lambda_1$};
            \node at (2,0) [below] {$\lambda_2$};
            \node at (4,0) [below] {$\lambda_3$};

            % Nodes along the halfway line 
            \node at (1,1.73) [above] {$\mu_1$};
            \node at (3,1.73) [above] {$\mu_2$};

            % Nodes at the top
            \node at (2,3.46) [above] {$\nu_1$};
        \end{tikzpicture}
    \end{center}
    We think about all the possible ways to fill out this diagram with the corresponding
    inequalities. There is a corresponding moment polytope which is the convex hull of the
    patterns and it looks like 
    \begin{center}
    \includegraphics[scale = .6]{img/Screenshot 2024-02-08 at 1.19.40 PM.png}
    \end{center}
    The numbers denote the size of the fiber once you project onto the board. 
    The interesting thing to note about the moment polytope on the left is
    that it represents a singular toric variety. Remember that the condition 
    that we are looking for is that each vertex has the same number of edges
    coming out of it. This is not the case for the moment polytope on the left.

    \hfill

    The moment polytope has the problem that there is a central vertex with 4 edges, 
    it looks like a pyramid corresponding to the singularity $ad - bc = 0$. Another way of understanding 
    the problem is that there are secret conditions, once you know that you are on the left and right face,
    then you must be actually at the vertex itself instead of knowing you are at an edge.
\end{example}

There is something in symplectic geometry known as the Thimm trick which we can use in representation theory to
extend group actions on vector spaces.

\hfill

The input data is a group $K$ acting on our vector space $V$. We can decompose $V$ into 
\red{isotopic components} \begin{align*}
    V = \bigoplus_{\lambda \in \mf{t}_T^*} V_\lambda
\end{align*} Then we get an action of another torus $T$ where $T$ acts on each component $V_\lambda$ by the character
$\lambda$. This is the Thimm trick. This action of $T$ is not related to the action of $K$ but they commute and so we get an action of 
$K\times T$ on $V$. If we repeat this process with $\GL(n)$ acting on $V$, then we are 
supposed to end up with an action of $T^{\binom{n}{2}}$ on $V$. \red{This is the Gelfand Cetlin action?}

\hfill

In algebraic geometry, if we have a group $G$ acting on a ring $R$, then we 
can consider $R$ as a $G$-vector space and again break it up into isotopic components \begin{align*}
    R = \bigoplus_{\lambda \in \mf{t}_T^*} I_\lambda
\end{align*} Then one makes the definition
\begin{align*}
    R_\lambda = \bigoplus_{\rho} I_{\lambda - \rho}
\end{align*} where $\rho$ is sums of the positive roots.
In this way we get that $R_\lambda R_\mu \subseteq R_{\lambda + \mu}$ and so we define \begin{align*}
    \gr R = \bigoplus_{\lambda} (R_\lambda/\sum_\alpha R_{\lambda - \alpha})
\end{align*} Each summand is isomorphic to $I_\lambda$. We get an action of $G\times T$.

\begin{example}
    \red{Vinberg asymptotic cone}
    Consider the action of $\SL(2)\times \SL(2)$ acting on $\SL(2)$ by left and right multiplication.
    We are considering the ring \begin{align*}
        R = \C[\SL(2)] &= \C[a,b,c,d]/(ad - bc - 1) \\
        &= \bigoplus_\lambda V_\lambda \otimes V_{\lambda}^* \\
        &= \bigoplus_n \Sym_n(\C^2) \otimes \Sym_n(\C^2)
    \end{align*} the map $\gr: R\to \C[a,b,c,d]/(ad - bc)$ the degeneration 
    to a singular toric variety. In particular we just forgot the degree $0$ part.
\end{example}

In general we want to consider the situation \begin{align*}
    \Proj \bigoplus_n V_{n\lambda} = \Fl(n) \leftarrow \mf{L}_\lambda
\end{align*} and we are supposed to apply the same deeneration trick \begin{align*}
    \Fl(n)\to \gr_{n-1}\Fl(n)\to \gr_{n-2}\gr_{n-1}\Fl(n) \to \dots \to 
\end{align*} and eventually we are supposed to end up with the toric variety
of the Gelfand Cetlin polytope correspodnign to $\lambda$.

\hfill

$\Proj \bigoplus_n V_{n\lambda}$ is supposed to be able to be presented as some polynomial ring,
i.e. we are supposed to have a surjection $\Sym(V_\lambda) \twoheadrightarrow \Proj \bigoplus_n V_{n\lambda}$
and in fact it is true that the kernel is generated in degree $2$.

\hfill

The question is we have an action $\PGL(n)\times\PGL(n)$ on $\bar\PGL(n)\hookrightarrow\Gr(\dim \mf g, \mf g\oplus\mf g)$
and we want to use these tricks to degenerate $\bar\PGL(n)$ to a toric variety with the action of $T^{\binom{n+1}{2}}\times T^{\binom{n+1}{2}}$.

\hfill

The question is what is the polytope.
\end{document}