\documentclass[12pt]{article}
\usepackage[english]{babel}
\usepackage[utf8x]{inputenc}
\usepackage[T1]{fontenc}
\usepackage{listings}
\usepackage{tikz}
\usepackage{/Users/songye03/Desktop/math_tex/style/quiver}
\usepackage{/Users/songye03/Desktop/math_tex/style/scribe}



\begin{document}
Songyu Ye

\today

\hfill

These are remarks about intersection theory. 

For a long time, algebraic geometers had ad hoc procedures of making sense of intersection multiplicities.
It is intuitive that one should count some points with multiplicity in order to get the right answer,
but Fulton and MacPherson really laid out groundwork for making sense of these things in the 1970s.

\section{Introduction}
If $A,B\subset X$ are subvarieties their intersection product should reflect geometry.
Let's consider two extreme cases.

\hfill

If $A$ and $B$ intersect properly, meaning that $\codim A + \codim B = \codim (A\cap B)$, then we have 
\begin{align*}
    A\cdot B = \sum_{\text{irreducible components}} \text{intersection multiplicity}\cdot C
\end{align*}
\red{This is why one should care about stratifications. What are the irreducible components of $A\cap B$?
Well, if you have a stratification, you can just look at the poset}

\hfill

In the other case if $A = B$ then one has the self intersection formula, which tells us that $A\cdot A$ is the 
top Chern class of the normal bundle of $A$ in $X$.

\hfill

In general we want formulas for the intersection  product $A\cdot B$ as rational equivalence classes of cycles
on $A\cap B$. This is the subject of Fulton's \textit{Schubert Varieties and Degeneracy Loci}.

\section{The hard work}
\begin{definition}
    If $f,g\in K[x,y]$ are plane curves $F,G$ the intersection scheme $Z$ is the subscheme of $\A^2$ corresponding to the ideal 
    $\langle f,g\rangle$. The intersection multiplicity of $Z$ at a point $p$ is the dimension of the local ring $\O_{Z,p}$
    as a $K$-vector space.
\end{definition}

Let $V\subset X$ be a subvariety of codimension $1$. Then $A = O_{V,X}$ is a local ring of dimension $1$.

\hfill

Recall that $O_{V,X}$ is the set of equivalence classes of pairs $(U,f)$ where $U$ is an open set in $X$ with 
$V\cap U = \emptyset$ and $f:U\to K$ is a regular function. 

\hfill

Recall that a regular function on an affine variety is a map $f:X\to K$ which is locally a quotient of polynomials.
A regular function on an abstract variety is the result of gluing regular functions on affine open sets.

\hfill

Recall that the field of rational functions on $X$ is the set of equivalence classes of pairs $(U,f)$ where $U$ is an open set in $X$
and $f:U\to K$ is a regular function.

\hfill

Consider $A$. It is a local ring with maximal ideal $m = \set{[f]\st f\vert_V = 0}$.
It is a discrete valuation ring if $V$ is a smooth subvariety of $X$. The prime ideals of $A$ correspond to 
the irreducible subvarieties of $X$ containing $V$.

\hfill

For a given $r\in A$ we define the order of vanishinging of $r$ along $V$ as \begin{align*}
    \ord_V(r) = \ell_A(A/r)
\end{align*} the length of $A/r$ as an $A$-module, this is defined as $1 + $ 
the length of a maximal chain of submodules.

When $A$ is regular (i.e. $V$ is smooth) then $A$ is a discrete valuation ring and $\ord_V$ agrees
with the discrete valuation. 

\hfill

In particular recall the following, see Hartshorne.
\begin{itemize}
    \item When $A$ is a dimension $1$ local ring, it is a $DVR$ if and only if it is regular 
    \item Regular is defined as $\dim_k m/m^2 = \dim A$.
    \item Affine $Y$ is nonsingular at $p$ if and only if $\cO_{Y,p}$ is a regular local ring.
\end{itemize}

For any rational function $r\in K(X)^*$ we define its order by 
writing it as a fraction $f/g$ for $f,g\in A$ (\red{one can always do this}) defining \begin{align*}
    \ord_V(r) = \ord_V(f) - \ord_V(g)
\end{align*}
In particular we are interested in $\ord$ to be a homomorphism \begin{align*}
    \ord(rs) = \ord(r) + \ord(s)
\end{align*} for $r,s\in K(X)^*$.
For a fixed $r\in K(X)^*$ there are only finitely many codimension $1$ subvarieties $V$ so that 
$\ord_V(r)\neq 0$ (\red{See Fulton appendix}).

\section{Cycles}
Let $X$ be a variety. A $K$-cycle on $X$ is a formal $\Z$-sum \begin{align*}
    Z = \sum n_i Z_i
\end{align*} where $Z_i\subset X$ is a subvariety of dimension $k$. For any $k+1$-dimensional 
variety $W$ of $X$, any $r\in K(W)^*$ we have a divisor (\red{homotopy}) \begin{align*}
    [\div(r)] = \sum_{\text{codim }1} \ord_V(r) [V]
\end{align*} Then we say that a $K$-cycle $\alpha$ is \textbf{rationally equivalent} to $0$ if there is a
there exist finitely many $k+1$-dimensional subvarieties $W_i$ and rational functions $r_i\in K(W_i)^*$ so that
\begin{align*}
    \alpha = \sum_i [\div(r_i)]
\end{align*}
The $k$th Chow group of $X$ is the group of $K$-cycles modulo rational equivalence and 
is denoted $A_k(X)$. We can put the Chow groups together to form the Chow ring $A_*(X)$.

\hfill

\red{Here come some deep results which I don't understand yet.}

\hfill

If $f:X\to Y$ is proper then subvarieties $V\subset X$ are sent to subvarieties $W = f(V)\subset Y$.
There is an induced embedding $K(W)\hookrightarrow K(V)$ which is a finite field extension if $W$ has the same 
dimension as $V$ (in this case $W,V$ have the same transcendence degree). Put \begin{align*}
    \deg(V/W) = [K(V):K(W)] \text{ if } \dim V = \dim W
\end{align*} and $0$ otherwise and define the pushfoward of a proper map \begin{align*}
    f_*:Z_k(X)\to Z_k(Y) \\
    f_*[V] = \deg(V/W)[W]
\end{align*}
Then the fact is that $f_*$ preserves rationally equivalent to $0$.

\section{Another persective}
The Chow group of $X$ is the group of cycles of $X$ modulo rational equivalence. Another way of thinking about 
rational equivalence is the following.

\begin{definition}
    $Z_1,Z_2$ are rationally equivalent if there exist rationally parametrized family of cycles 
    interpolating between them, i.e. a cycle on $\P^1\times X$ so that the fibers over $t_1,t_2$ are 
    $Z_1,Z_2$. The group of cycles rationally equivalent to $0$ is then generated by the differences \begin{align*}
        [\Phi \cap {t_0}\times X] - [\Phi\cap {t_1}\times X]
    \end{align*} for any subvariety $\Phi\subset \P^1\times X$ not contained in a fiber.
\end{definition}

\begin{remark}
    What makes Chow groups usefeul is that under good conditions, the rational
    equivalence class of $A\cap B$ depends only on the rational equivalence classes of $A$ and $B$.
\end{remark}

$A(X)$ is a ring under the intersection product. 
\begin{theorem}
    Let $X$ be a smooth quasi-projective variety. Then there exists a unique 
    product structure on $A(X)$ so that if $A$ and $B$ are generically transverse,
    then \begin{align*}
        [A]\cdot [B] = [A\cap B]
    \end{align*}
    This makes $A(X) = \bigoplus A^c(X)$ into an associative ring graded by codimension.
\end{theorem}

\subsection{Via divisors}
If $X$ affine, $f\in \cO_X$ nonzero, then the irredducible components of the subscheme cut out by $f$ 
are all of codim $1$ (Krull's Principal Ideal Theorem) so to this subscheme we associate a cycle $[\div(f)]$.

\hfill

If $X$ is not affine we can pick $U\subset X$ open affine $K(X) = K(U)$. This is 
because rational functions are just regular functions on an open subset, and
$V\subset X$ open and $f=g$ on $U$ open. Then we use the fact that if two functions 
agree on an open set then they are equal on $X$ wherever they are defined.

\hfill

In particular we get a divisor $\div(f\vert_U)$ on each $U$ and they agree on overlaps, 
so we get $\div(\alpha)$ on $X$ itself.

\hfill

These two notions agree [Fulton 1984]

\begin{example}
    2 points on a curve are birationally equivalent if and only if the curve is 
    birational to $\P^1$.
\end{example}

\section{Yet another perspective}
Ravi Vakil has an intersection theory course website which is very good for intuition
and examples. In particular he points out where the difficult parts lie.

\hfill

\subsection{Chow groups}
\begin{itemize}
    \item Two points on $\P^1$ are defined to be rationally equivalent
    \item If $\pi: X\to Y$ flat, then there is a pullbaack. If $\dim X - \dim Y = d$
    then the map $\pi^*:H_n(Y)\to H_{n+d}(X)$.
    \item If $\pi:X\to Y$ is proper (which implies that the image of closed is closed)
    then we have a pushforward $\pi_*:H_n(X)\to H_n(Y)$.
\end{itemize}

What should multiplicity of scheme theoretic intersection be?

\begin{example}
    [Two planes in $\P^4$] Put coordinates $a,b,c,d,e$ and consider two planes
    $X_1 = \langle a,b\rangle$ and $X_2 = \langle c,d\rangle$.
    Then $X_1\cup X_2 = \langle a,b\rangle\cup \langle c,d\rangle = \langle a,b,c,d\rangle$
    should have degree two ("number of points of intersection with arbitrary linear subspace")
    
    \hfill

    How should we count this? Let $P = \langle a- c,b-d\rangle$ be a third plane. Then 
    $P$ meets each of $X_1,X_2$ in a point as long as $\dim X\cap P = 0$. But if $P$ passes through
    $X_1\cap X_2$ then we have trouble. Let's see why.

    \hfill

    Work on affine open chart $e = 1$. The ring in question is $k[a,b,c,d]$. The planes 
    $X_1$ and $X_2$ correspond to ideals $\langle a,b\rangle$ and $\langle c,d\rangle$. 
    Then we consider the ring \begin{align*}
        k[a,b,c,d]/\langle a,b,c,d\rangle \cup \langle a-c, b-d\rangle \cong k[a,b]/\langle ab,a^2,b^2\rangle
    \end{align*} The problem is that this ring is $3$-diml as a vector space over $k$. What's the fix?

    \hfill

    The current formula says that we should look at \begin{align*}
        \dim_k R/I_1\otimes R/I_2
    \end{align*} But Serre says that $\otimes$ is not exact. It is only right exact and there is cohomology
\begin{align*}
    &\rightarrow \Tor^2(M,A) \rightarrow \Tor^2(M,B) \rightarrow \Tor^2(M,C) \\
    &\rightarrow \Tor^1(M,A) \rightarrow \Tor^1(M,B) \rightarrow \Tor^1(M,C) \\
    &\rightarrow M\otimes A \rightarrow M\otimes B \rightarrow M\otimes C \rightarrow 0 \\
\end{align*}
Instead Serre says you should look at the Tor groups.
\begin{align*}
    \chi = \sum (-1)^i \dim \Tor_i(R/I_1,R/I_2)
\end{align*} and this is the right answer.

\hfill

Cohen Macauley is about higher Tor groups vanishing.
\end{example}

A common thing to do is to localize at the generic point of a subvariety $X$ of a scheme $Y$.
What this means is that we consider rational functions on $Y$ defined on a dense open subset of $X$,
of dimension $ = \dim Y - \dim X$.

\hfill

Recall that the points of $Y$ correspond to irreducible subvariety of $Y$, and the closed 
points correspond to the "old points" of $Y$. 

\hfill

Moreover points of $\Spec O_{X,Y}$ correspond to prime ideals correspond to irreducible subvarieties
of $Y$ containing $X$. The maximal ideal corresponds to $X$ itself.

\begin{example}
    What is the order of vanishing of $y/x$ at $0$ in $y^2 = x^3$?
    What about $y^2 = x^3 + x^2$?
    Note that these guys are singular.
\end{example}

\section{References}
\begin{itemize}
    \item Fulton's \textit{Intersection Theory}
    \item Eisenbud Harris \textit{3264 and all that}
    \item Ravi Vakil's Course Website
\end{itemize}
\end{document}