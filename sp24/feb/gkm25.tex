\documentclass[12pt]{article}
\usepackage[english]{babel}
\usepackage[utf8x]{inputenc}
\usepackage[T1]{fontenc}
\usepackage{listings}
\usepackage{tikz}
\usepackage{/Users/songye03/Desktop/math_tex/style/quiver}
\usepackage{/Users/songye03/Desktop/math_tex/style/scribe}



\begin{document}
Songyu Ye

\today

\hfill

This is my plan for research this week.

\section{Tools for solving the problem}
\subsection{Toric degeneration}
\begin{definition}
	A toric degeneration of a projective variety $V$ is a flat morphism $\phi: X\to \A^1$ which
	trivializes away from the fiber over $0$ in $\A^1$ \begin{align*}
		X\setminus \phi^{-1}(0) \cong V\times \A^1\setminus 0
	\end{align*}
\end{definition}
We should really consider toric degenerations which are $T$-equivariant meaning that 
the action on the central fiber is an extension of the torus action on $X$.

\hfill

One place toric degenrations come from is valuations.

\hfill

\red{Why should I care?}

\subsection{Intersection theory}
The matter at hand is to expand the class of a subvariety in the geometric basis for cohomology.
We know that the cohomology ring of the flag variety has a geometric basis given by the Schubert classes.
There is this classical result which sort of characterizes the geometry of this basis.

\begin{theorem}
	Let $X = G/B$ be the flag variety. Then
	\begin{enumerate}
		\item The bases $\{[X_w]\}$ and $\{[X^v = X_{w_0v}]\}$ are dual to each other for the Poincare pairing.
		\item For any subvariety $Y\subset X$ there is an expansion \begin{align*}
			      [Y] = \sum_{w\in W}a^w(Y) [X_w]
		      \end{align*} where $a^w(Y) = \langle [Y], [X^w]\rangle = \#(Y\cap gX^w)$ for general $g\in G$.
		\item There are nonnegative structure constants for the expansion \begin{align*}
			      [X_v^{w_0w}]= [X_v][X^w] = \sum_{u\in W}c_{vw}^u[X^u]
		      \end{align*}
	\end{enumerate}
	Moreover the map $f:G/B\to G/P$ induces a map on cohomology which sends any Schubert class $[X_{wP}]$ to
	the Schubert class $[X_{ww_{0,P}}]$ where $w\in W^P$.
\end{theorem}

Recall that $W^P$ parametrizes the coset space $W/W_P$ where $W_P = S_n/S_{d_1}\times \cdots \times S_{d_k}$
where $G/P = \Fl(d_1,\dots,d_k)$ where the indexing of the flag variety denotes the step. In particular
$\Fl(1,\dots,1)$ is the full flag variety. The elements of $W^P$ are the minimal length coset representatives,
the unique permutation so that $w(1)<w(2)<\cdots<w(d_1)$, $w(d_1+1)<w(d_1+2)<\cdots<w(d_1+d_2)$, and so on.

\hfill

It looks like that this theorem should tell us how to expand the class of any subvariety.
This is not the case because we do not even know how to expand $[X_v^{w_0w}]$ in the Schubert basis.
This is the object of Schubert calculus and it is one of the oldest problems in algebraic combinatorics.

\hfill

For the wonderful compactification, Brion tells us about his geometric basis.

\begin{theorem}
    $A_{T\times T}^*(X)$ has a basis given by the classes \begin{align*}
        X(\omega,\tau) = [\overline{B\times B^-(\omega,\tau)z_\Phi}]
    \end{align*} where $\Phi = \set{\alpha\in\Sigma \st \tau(\alpha)\in R^+}$ and $z_\Phi$ is the basepoint of the $G\times G$ orbit corresponding
    to $\Phi$. The restriction to $G\times B^-\times G/B$ is equal to \begin{align*}
        (D_\omega \otimes D_\tau) \prod_{\alpha\in\Sigma\backslash\Phi}c^{T\times T}(\alpha,-\alpha)\sum_{w\in W_P(\Phi)}[\overline{B^-wB}/B\times \overline{Bw_{0,P}B}/B]
    \end{align*}
\end{theorem}
where $c^{T\times T}:X^*(T\times T)\to A_{T\times T}^*(G/B\times G/B)$ is the characteristic map.

So the $T\times T$-equivariant cohomology ring of the wonderful compactification
has a geometric basis given by certain $B\times B^-$ orbit closures, in particular the ones which
contain $T\times T$ fixed point.

\hfill

Our hope is to expand the class of a given $B\times B^-$ orbit closure intersect $B^-\times B$ orbit closure
in terms of the geometric basis. There is a subproblem of expanding the class of a 
general $B\times B^-$ orbit closure in terms of the geometric basis. Of course this should be very hard
but we hope to reduce the problem to the Schubert calculus for $G/B$.

\subsection{GKM}
A thing that one might try is to write down Brion's basis explicitly in terms of GKM data.
Then one can try to write down what an arbitrary $B\times B^-$ orbit closure intersect $B^-\times B$ orbit closure
looks like in terms of GKM data. If we can do this, then we can try to expand these classes 
on the GKM side.

\hfill

I don't even know how to do this for $\overline{\PGL(2)}$. We should talk to Tara about the computation.

\subsection{Basis of anticanonical divisors}
The canonical bundle of a nonsingular algebraic variety $V$ 
is the line bundle $\Omega^n = \omega$, the $n$th exterior power of the cotangent bundle $\Omega$ on $V$.
In particular over the complex numbers it is the determinant bundle of the holomorphic cotangent
bundle $\Omega$ on $V$. \red{Bundles have classes}

\hfill

Therefore the canonical bundle defines a divisor class, denoted the \red{canonical class}.

\begin{definition}
    An anticanonical divisor is any divisor class $-K$ with $K$ canonical.
\end{definition}

\hfill

One thing that we worried about is that the boundary divisors of the Richardson stratification
are anticanonical in the Richardson variety. \red{Why does this matter?}

\end{document}