\documentclass[12pt]{article}
\usepackage[english]{babel}
\usepackage[utf8x]{inputenc}
\usepackage[T1]{fontenc}
\usepackage{listings}
\usepackage{tikz}
\usepackage{/Users/songye03/Desktop/math_tex/style/quiver}
\usepackage{/Users/songye03/Desktop/math_tex/style/scribe}

\begin{document}
Songyu Ye

\today

\hfill

This is what Allen and I talked about on 2/23/2024.

\section{Introduction}

The relevant guy is a torus bundle. This is because at each point $p$ of $X$ we have direct sum of line bundles with a $T$ action, the fiber looks like
a Laurent polynomial ring.
Now that we know that the one guy is quasiaffine (read the proof) 
we know that the guy in question is. 

\hfill

Now that we know the guy is quasiaffine, the thing to do is study the ring of functions and take its Spec.

\hfill

The guy down below, we want to find a presentation for him i.e. generators and relations. Allen said that there are some papers in the literature that compute those dimensions.
\red{This is the next thing to look at.}

\hfill

Recall that the guy in question is our substitute for $G/N$. There are 3 tori acting on the guy in question, two coming from $G\times G$ and the other coming from the $X^*(T)$
What is the connection between the guy Allen awnted to know is quasiaffine and the story of degenerating $G/B$? The point is that the guy who is quasiaffine, this means that he sits
as an open subset of affine space, we can think about degenerating this guy to a toric variety, and then dividing out by the torus action as we did in the story of $G/N$.

\hfill

There was a remark about wall crossings and picking up higher cohomology. The example was the blowup of $\P^2$ and then the flag variety. The numbers correspond to lattice points in 
the polytope correspond to the dimensions of those weights when considering the $H^0$ and $H^1$ as a representation.
Allen said that he thiks this doesn't happen in the sotry of the flag variety.

\section{Pictures}
\begin{center}
    \includegraphics[scale = .2]{/Users/songye03/Desktop/math_tex/img/D9C50CBE-1C0E-4384-AA3F-6E74F156B914_1_102_o.jpeg}
\end{center}
Side remark about the blowup of $\P^2$ at a point. The data of a line bundle on the blowup is the lengths $a,b$ as depicted.
If you think about taking tensor products of $a$ and $-b$ what you find is that you get the triangle shown above. The numbers in 
the polytope correspond to the dimensions of the weights when considering the $H^0$ and $H^1$ as a representation. If you are a toric variety, these numbers are 
always $1$. In the second picture, you can consider $G/B$ for $G = \GL(3)$ and we get numbers which are not $1$ because we are not a toric variety. 
However in the degernation to the toric variety of the Gel'fand-Tsetlin polytope, the numbers are $1$ again, but the picture is $3$ dimensinoal. The map induces a reverse map
on the weight lattices which collapses that polytope into the numbers that we see in the 2-dimensional picture.

\hfill

If we think about letting $a$ negative, then now there are no global sections, and instead all the lattice points correspond to $H^1$. The point from going from 
$H^0$ to $H^1$ is thinking about the two different maps $\tilde{\P^2} \to \P^2$ and $\tilde{\P^2} \to \P^1$. The second map drops dimension and 
so we pick up a higher cohomology group.
\begin{center}
    \includegraphics[scale = .12]{/Users/songye03/Desktop/math_tex/img/F668F5EA-445C-4824-B8EB-E976D93BCCDD_1_102_o.jpeg}
\end{center}

\begin{center}
    \includegraphics[scale = .15]{/Users/songye03/Desktop/math_tex/img/A930D276-8AB2-4386-803B-2FBDB63FEC1E_1_102_o.jpeg}
\end{center}

\begin{center}
    \includegraphics[scale = .2]{/Users/songye03/Desktop/math_tex/img/97550720-7FCC-4E43-8155-DE683F6A1776_1_102_o.jpeg}
\end{center}
The remark about $\Z^N$ vs $\N^N$ is that it shouldn't matter.

\hfill 

There is a natural basis for that ring which we have written down in terms of the dominant Weyl chamber. 
We know that the NefPic is generated by the fundamental weights i.e. the Schubert divisors, the construction which does that, rips out the unstable points (which
Allen says ends up being the same guy as what we have) and then mods out by the torus action, we recover $X$.

\hfill

This is parallel to the story of if you have $O(1)$ on $\Proj R$ and you consider $S = \bigoplus \Gamma(\Proj R, O(n))$ then this new graded ring you have 
$\Proj S = \Proj R$. 
\end{document}