\documentclass[12pt]{article}
\usepackage[english]{babel}
\usepackage[utf8x]{inputenc}
\usepackage[T1]{fontenc}
\usepackage{listings}
\usepackage{bookmark}
\usepackage{tikz}
\usepackage{/Users/songye03/Desktop/math_tex/style/quiver}
\usepackage{/Users/songye03/Desktop/math_tex/style/scribe}
\usepackage{fancyhdr}

\begin{document}


\lhead{Songyu Ye}
\rhead{\today}
\cfoot{\thepage}

\title{Rep theory of real reductive groups}

\author{Songyu Ye}
\date{\today}
\maketitle


\begin{abstract}
Notes for a talk for Dan. Basic introduction to real reductive groups, Harish-Chandra modules, and Kazhdan-Lusztig-Vogan polynomials.
Perhaps details about Langlands correspondence. 
We will write from Trapa's slideshow, Milicic's notes, and "Unitary Representations of Real Reductive Groups" by Trapa, Vogan et al.
\end{abstract}

\tableofcontents

\section{Introduction}
We are interested in complex representations of real reductive groups.
If you have such a group $G$ which acts smoothly on a space $X$, then one gets an (infinite-dimensional!) 
representation of $G$ on the space of (whatever adjective you like) functions $X \to \C$. Starting with Gelfand,
the general idea is to decompose Hilbert spaces of functions on
$X$, and reassemble this information to make deductions about
$X$. The holy grail is to understand the irreducible unitary representations of $G_\R$.


\hfill

Harish-Chandra succeeded in decomposing $L^2(G_\R)$ into something called tempered representations. So inside the unitary 
dual of $G_\R$ there is a subset of representations called tempered representations. \begin{align*}
    \hat{G}_\R^{temp} \subset \hat{G}_\R^{unitary}
\end{align*} More generally we can look at congruence subgroups $\Gamma$ of $G_\R$ and
look at the representations which appear in $L^2(\Gamma \backslash G_\R)$. We call this packet of representations the 
\red{automorphic dual} of $G_\R$.

\hfill

Harish-Chandra found something larger than the unitary dual which is more tractable, called \red{Harish-Chandra modules}.
Given an irreducible unitary representation $\pi$ of $G_\R$, we can differentiate at the identity and 
complexify to get a representation of the complexified Lie algebra $\mf g_\C$, in particular of the enveloping algebra $U(\mf g_\C)$.

\hfill

However some information is lost! For this reason we must keep track of the restriction of $\pi$ to a maximal compact subgroup $K_\R$ of $G_\R$.

\begin{enumerate}
    \item \red{The Langlands classification and Langlands parameters are about classifying the Harish-Chandra modules of $G_\R$. Can you
    please explain this story to me, and how it relates to the calculations carried out in the Atlas project?}
    \item \red{Category $\cO$ and quivers}
\end{enumerate}

\section{After meeting}
We have to read chapter 9 and 10 from Helgason.

\hfill

Recall the original statement that I first became interested in. 
Let $G$ be a complex reductive group and pick a Borel subgroup $B$.

\begin{theorem}
    The category of $B$-equivariant regular 
    holonomic $D$-modules on $G/B$ is equivalent to category $\cO_0$.
\end{theorem}

Dan puts this in a larger perspective, in the context of the 
representation theory of real reductive groups. In particular let 
$G_\R$ be a real reductive group, and $K_\R$ a maximal compact subgroup. 
Let $\mf g = \Lie(G_\R)$.

\begin{definition}
    A Harish-Chandra module is a finitely generated $U(\mf g)$-module $M$ which carries a compatible action of $K_\R$.
    The action of $K_\R$ is compatible with the action of $U(\mf g)$ in the sense that 
    \begin{align*}
        k \cdot (u \cdot m) = u \cdot (k \cdot m)
    \end{align*} for all $k \in K_\R$, $u \in U(\mf g)$, and $m \in M$.

    Moreover, the action must be \red{locally finite $K$-action}, in 
    the sense that every $m \in M$ is contained in a finite dimensional $K$-stable subspace of $M$.

    We also insist on \red{finite dimensional $K$-types}, meaning that the module $M$ decomposes
    as a direct sum of finite dimensional ireducble representations.
\end{definition}

\begin{definition}
    A Harish-Chandra module is called \red{admissible} if we further insist
    that the isotopic $K$ components appear with finite multiplicity.
\end{definition}

Here comes a really deep theorem.

\begin{theorem}
    There is an equivalence of categories \begin{align*}
        K_\R-\Mod \cong \text{admissible $\mf g,K_\R$ modules with trivial central character}
    \end{align*}
\end{theorem} Let the right hand category be denoted by $\text{Admiss}$.
The point is that simple objects 
in $\text{Admiss}$
can be realized as local systems on orbit 
closures of $K_\R$-orbits in $G/B$. \red{Note that 
$K_\R$ acts on $G/B$ with finitely many orbits! This is the 
secret sauce in some sense.}

\hfill

Then the Kazhdan-Lusztig polynomials, whose coefficients
are the dimensions of the appropriate intersection 
cohomology groups, when evaluated at $1$ give the
multiplicities of the simple objects in $\text{Admiss}$ appearing
in some other representation. This is because the other representation carries a filtration, 
and each graded piece of the K-L polynomial gives the multiplicities
appearing in the corresponding graded piece of the filtration, so 
the signed count gives the total multiplicity.

\begin{remark}
    Dan said that this part of the story is quite elementary. The hard 
    part which was state of the art at the time, was actually computing
    these intersection cohomology groups and finding out that they 
    are dimensions of $\Ext$ groups between the simples. This 
    somehow involved Weil conjectures, using finite field geometry 
    to get stuff about complex geometry.
\end{remark}

\begin{remark}
    Another point Dan made was about this notion of direct sum and direct
    integral, these ideas used to write down what the 
    "simple" modules should be. In particular, the point is that for 
    a real reductive group like $\R$, the $L^2$-space is a direct integral
    of irreducible unitary representations. This 
    is the subject of Fourier analysis.
\end{remark}

The real reductive group case covers the complex case, because
given a complex reductive group $G$ with a Borel subgroup $B$,
we can consider the real group $G\times G$ with an action of the 
maximal compact subgroup $K = B \times B$. Then 
we just interpret the left hand side as $\cD$-modules on $G/B$, and 
the right hand side as Category $\cO_0$.

\hfill

I will finish with a closing remark from Dan. 
"Harish-Chandra devoted his life to finding an explicit 
formula for characters of irreducible unitary representations
of real reductive groups. We use his theorems every day."



\end{document}