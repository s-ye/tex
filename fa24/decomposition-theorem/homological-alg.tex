What follows comes from \cite{griffiths-harris} and \cite{bott-tu}.
\subsection{First look}
Recall that a short exact sequence of chain complexes \begin{align*}
    0 \to A^* \to B^* \to C^* \to 0
\end{align*} induces a long exact sequence in cohomology \begin{align*}
    \cdots \to H^i(A) \to H^i(B) \to H^i(C) \to H^{i+1}(A) \to \cdots
\end{align*} We can view this as a piece of cohomological data coming from the 
two step filtration of the total complex $B^*$ by $0 \subset A^* \subset B^*$.
This is the beginning of a spectral sequence.

\hfill

The filtration of the total complex descends to a filtration on 
cohomology. In particular each page $E^r$ of the spectral sequence is
filtered. There is a very special filtration on the $H^*(X,\C)$ of a Kahler manifold
coming from the Cech to de Rham spectral known as the \textbf{Hodge filtration}.
\begin{definition}
    A \textbf{spectral sequence} is a sequence $(E^r,d^r)$ of bigraded abelian groups \begin{align*}
        E^r = \bigoplus_{p,q} E^{p,q}_r
    \end{align*} for $r\geq 0$ and differentials \begin{align*}
        d_r: E^{p,q}_r \to E^{p+r,q-r+1}_r
    \end{align*} so that $d_r^2 = 0$ and $H^*(E^r,d^r) = E_{r+1}$.
\end{definition}
\subsection{Spectral sequence of a filtered complex}
\begin{proposition}
    Every filtered complex $K^* = F^0K^* \supset F^1K^* \supset \cdots$ gives rise 
    to a spectral sequence with \begin{align*}
        E_0^{p,q} &= F^pK^{p+q}/F^{p+1}K^{p+q} \\
        E_1^{p,q} &= H^{p+q}(F^pK^*/F^{p+1}K^*) = H^{p+q}(\Gr^p K^*) \\
        E_\infty^{p,q} &= \Gr^p(H^{p+q}(K^*))
    \end{align*}
\end{proposition}
\begin{proof}
    We have the differential inherited from $K^*$ \begin{align*}
        d_0&: E_0^{p,q} \to E_0^{p,q+1} \\
        d_0&: F^pK^{p+q}/F^{p+1}K^{p+q} \to F^pK^{p+q+1}/F^{p+1}K^{p+q+1}
    \end{align*} \red{Note that 
    the differential only increases the cohomological degree of $K^*$ by 1.} We compute 
    the first page \begin{align*}
        E_1^{p,q} &= \ker d_0 / \im d_0 \\
        &= \set{a\in F^pK^{p+q} \st da \in F^pK^{p+q+1}} / d(F^pK^{p+q-1}) + F^{p+1}K^{p+q} \\
        &= H^{p+q}(F^pK^*/F^{p+1}K^*) \\
        &= H^{p+q}(\Gr^p K^*)
    \end{align*} The next differential \begin{align*}
        d_1: E_1^{p,q} \to E_1^{p+1,q} \\
        a \mapsto da
    \end{align*} is well-defined because \begin{align*}
        da \in \set{b\in F^{p+1}K^{p+q+1} \st db \in F^{p+2}K^{p+q+2}}
    \end{align*} One can compute the kernel and image of $d_1$ to get the second page,
    to find that \begin{align*}
        E_2^{p,q} = \set{a \in F^pK^{p+q} \st da \in F^{p+2}K^{p+q+1}} / d(F^{p-1}K^{p+q-1}) + F^{p+1}K^{p+q}
    \end{align*} Take denominator as written intersect the numerator. The general pattern is \begin{align*}
        E_r^{p,q} = \set{a \in F^pK^{p+q} \st da \in F^{p+r}K^{p+q+1}} / d(F^{p-r+1}K^{p+q-1}) + F^{p+1}K^{p+q}
    \end{align*} and then one can check that \begin{align*}
        H^*(E^r,d^r) = E^{r+1}
    \end{align*} For $r$ sufficiently large we get that \begin{align*}
        E_\infty^{p,q} &= \set{a\in F^pK^{p+q} \st da = 0} / d(K^{p+q-1}) + F^{p+1}K^{p+q} \\
        &= \Gr^p(H^{p+q}(K^*))
    \end{align*} as desired.
\end{proof}
\subsection{Spectral sequence of a double complex}
Let $K^{*,*}$ be a double complex with differentials \begin{align*}
    d: K^{p,q} \to K^{p+1,q} \text{ and } \delta: K^{p,q} \to K^{p,q+1}
\end{align*} so that \begin{align*}
    d^2 = \delta^2 = 0 \text{ and } d\delta + \delta d = 0
\end{align*} We form the associated single compex $(K,D)$ with \begin{align*}
    K^* = \bigoplus_{p+q=n} K^{p,q} \text{ and } D = d + \delta
\end{align*}
There are two filtrations on $(K^*,D)$, by rows and columns \begin{align*}
    F'^pK^n &= \bigoplus_{i\leq p} K^{i,n-i} \\
    F''^qK^n &= \bigoplus_{i\leq q} K^{n-i,i}
\end{align*} and there are two associated spectral sequences, both abuting to 
$H^*(K,D)$. The corresponding $E_2$ pages turn out to be \begin{align*}
    E'^2_{p,q} &= H^p_d(H^q_\delta(K^{*,*})) \\
    E''^2_{p,q} &= H^q_\delta(H^p_d(K^{*,*}))
\end{align*}
\begin{example}
    If you consider the complex of real-valued forms on a manifold \begin{align*}
        0 \to \Omega^0 \to \Omega^1 \to \cdots \to \Omega^n \to 0
    \end{align*} and filter it by the limit of all of the "good covers" (intersection 
    of two opens is contractible) of the manifold, 
    then the associated spectral sequence is the Cech to de Rham spectral sequence and
    realizes the de Rham isomorphism \begin{align*}
        H^*(X,\R) \cong H^*_{dR}(X)
    \end{align*}
\end{example}
Alternatively we can think about the de Rham isomorphism like this. Recall the 
following lemma:

\begin{lemma}
    A map of complexes of sheaves which is a quasi-isomorphism
    (i.e. induces an isomorphism on all cohomology sheaves)
    induces an isomorphism on hypercohomology.
\end{lemma} The point is is that this map will induce an
isomorphism on the associated spectral sequences. 
\begin{example}
    Again we have the real-valued de-Rham complex \begin{align*}
        0 \to \Omega^0 \to \Omega^1 \to \cdots \to \Omega^n \to 0
    \end{align*} and also the trivial complex $\R^*$ \begin{align*}
        \R \to 0 \to 0
    \end{align*} The \red{$d$-Poincare lemma} shows that these 
    two complexes have the same cohomology sheaves and therefore inclusion 
    is a quasiisomorphism. Therefore the lemma shows they have 
    the same cohomology. Now consider the two 
    spectral sequences at hand. We have \begin{align*}
        (E_{\R^*}^{'p,q})_2 = H^p(M,\R) \text{ if $q=0$} \text{ and } 0 \text{ otherwise}
    \end{align*} so the spectral sequence is trivial and \begin{align*}
        H^p(M,\R) = \cH^p(M,\R^*)
    \end{align*} On the other hand we compute $H^q(M,\Omega^*)= 0$ if $q>0$ by 
    partition of unity and 
    \begin{align*}
        E_{\Omega^*}^{''p,q} = H^p_{dR}(M) \text{ if $q=0$} \text{ and } 0 \text{ otherwise}
    \end{align*} and so putting everything together we find that \begin{align*}
        H^p(M,\R) \cong H^p_{dR}(M)
    \end{align*}
\end{example}
\subsection{Leray Spectral Sequence}
Let $f:X\to Y$ be a continuous map of topological spaces. Recall 
that the $q$-th direct image sheaf $R^q_f\cF$ is the sheaf associated to the presheaf \begin{align*}
    U \mapsto H^q(f^{-1}(U),\cF)
\end{align*}

The Leray spectral sequence is a spectral sequence with \begin{align*}
    E_2^{p,q} = H^p(Y,R^q_f\cF) \\
    E_\infty \to H^*(X,\cF)
\end{align*}

This spectral sequence is particularly special when $f:E\to B$ is a fiber bundle
with fiber $F$. In this case, for the constant sheaf $\Q$ on $E$ we have by
the Kunneth formula \begin{align*}
    H^q(\pi^{-1}(U),\Q) \cong H^*(F,\Q)
\end{align*} which suggests \begin{align*}
    R^q_f\Q \cong H^*(F,\Q)
\end{align*} is a constant sheaf on $B$. However this is not so 
because we have to account for the monodromy action of $\pi_1(B)$ on the fibers.

\hfill

\begin{example}
    We consider the derivation of the Leray spectral sequence for de Rham cohomology
    in the case of a fiber bundle.
    \red{I don't understand this example} but the relevant fact is that \begin{align*}
        E_2^{p,q} = H^p_{DR}(B,H^q_{DR}(F,\R))
    \end{align*} where we can interpret the right hand side as 
    the the de Rham cohomology of $B$ with coefficients in the 
    local system $H^q_{DR}(F,\R)$.
\end{example}

\section{Appendix B: Homological algebra}
What follows comes from \cite{goresky}.
\subsection{The four functors $f_*,f^*,f_!,f^!$}
There is a nice geometrical way to think about these functors which requires
introducing the leaf space or espace etale of a presheaf.

\begin{definition}
    Given a presheaf $\cF$ on a topological space $X$, the \textbf{espace etale} $L\cF$
    is the disjount union of the stalks of $\cF$ \begin{align*}
        L\cF = \bigsqcup_{x\in X} \cF_x \to X
    \end{align*} equipped with a topology that is discrete on each $S_x$ and that makes $\pi$ into 
    a local homeomorphism.
\end{definition}

Given the leaf space, we have the sheafification of $\cF$ given by $U \mapsto \Gamma(U,L\cF)$
and the presheaf $\cF$ is a sheaf if and only if it is isomorphic to its sheafification.

\begin{definition}
    Given a continuous map $f:X\to Y$ and a sheaf $\cF$ on $X$ and $\cG$ sheaf on $Y$,
    we define the \textbf{direct image}
    $f_*\cF$ on $Y$ by \begin{align*}
        f_*\cF(U) = \Gamma(f^{-1}(U),\cF)
    \end{align*} and the \textbf{inverse image} $f^*\cG$ on $X$ by \begin{align*}
        f^*\cG(U) = \lim_{V\supset f(U)} \cG(V)
    \end{align*} We also have the \textbf{pushforward with proper support $f_!$ }
    with sections $\Gamma(U,f_!\cF)$ consisting of all sections $s\in \Gamma(f^{-1}(U),\cF$
    so that the mapping $\text{closure}(x\in U \st s(x)\neq 0) \to U$ is proper. 
    In particular if $X\to Y$ is the inclusion of a subspace then $f_!\cF(U)$
    is those sections $s\in \Gamma(U\cap X,\cF)$ with compact support.
\end{definition}
\begin{lemma}
    When $Y$ is locally compact and $X \to Y$ is the inclusion then $f_!\cF = f_*\cF$.
\end{lemma}
\subsection{Adjunction}
Let $f:X\to Y$ be a continuous map of topological spaces and $\cF$ a sheaf on $X$ and $\cG$
a sheaf on $Y$. Then there are natural maps \begin{align*}
    f^*f_*\cF\to \cF \text{ and } \cG \to f_*f^*\cG
\end{align*} To see this for the first one, consider
an open set $U\subset X$. Then \begin{align*}
    \Gamma(U,f^*f_*\cF) = \lim_{V\supset f(U)} \Gamma(V,f_*\cF) = \lim_{V\supset f(U)} \Gamma(f^{-1}(V),\cF)
\end{align*} has a natural map to $\Gamma(U,\cF)$, compatible with restriction.
For the second one, we have $V\subset Y$ open. If $t$ is a section of $L\cG$ over $V$,
then the pullback by $f$ is a section of the leaf space of $f^*\cG$, over the set $f^{-1}(V)$.
Therefore we have a map \begin{align*}
    \Gamma(V,\cG) \to \Gamma(f^{-1}(V),f^*\cG) = \Gamma(V,f_*f^*\cG)
\end{align*} compatible with restriction.
Therefore we have the following adjunction \begin{theorem}
    There are natural isomorphisms \begin{align*}
        \Hom(f^*\cG,\cF) \cong \Hom(\cG,f_*\cF)
    \end{align*}
\end{theorem}
\begin{remark}
    One might ask if there is an adjoint to the functor $f_!$. The answer is a very subtle 
    question and introduces Verdier duality.
\end{remark}

\subsection{Cohomology}
\begin{definition}
    A sheaf $\cF$ is \textbf{injective} if for every $0 \to A \to B$, 
    any morphism $A\to \cF$ can be extended to a morphism $B\to \cF$. That is the diagram \begin{center}
        \begin{tikzcd}
            0 \arrow[r] & A \arrow[r] \arrow[d] & B \arrow[dl,dashed] \\
            & \cF
        \end{tikzcd}
    \end{center} commutes.
    An \textbf{injective resolution} of a sheaf $\cF$ is an exact sequence \begin{align*}
        0 \to \cF \to I^0 \to I^1 \to \cdots
    \end{align*} where each $I^i$ is injective sheaf.
    The \textbf{sheaf cohomology groups} $H^*(X,\cF)$ is the cohomology of the complex \begin{align*}
        0 \to \Gamma(X,I^0) \to \Gamma(X,I^1) \to \cdots
    \end{align*} Any other injective resolution of $\cF$ will give the same cohomology groups
    up to canonical isomorphism. Given a complex of sheaves $\cF^*$ we can also take the
    cohomology sheaves $\cH^*(\cF^*) := \ker d/\im d$. \red{The stalk of the cohomology
    sheaf coincides with the cohomology of the stalks.}
    A morphism $f:\cF\to \cG$ of sheaves is a \textbf{quasi-isomorphism} if it induces
    an isomorphism on all cohomology sheaves. \red{A quasi-isomorphism induces an isomorphism
    on cohomology groups.}
\end{definition} 

\begin{definition}
    The \textbf{mapping cone} $C(\phi)$ of a map $\phi:\cF^*\to \cG^*$ of complexes of sheaves
    is the total complex of the double complex corresponding to $\phi$. We often write
    this as a "magic triangle" \begin{center}
        \begin{tikzcd}
            \cF^* \arrow[r,"\phi"] & \cG^* \arrow[r] & \cC(\phi) \arrow[r] & \cF^*[1]
        \end{tikzcd}
    \end{center}
\end{definition}

Mapping cones are interesting because they give rise to long exact sequences in cohomology.
\begin{lemma}
    If $\phi$ is injective then there is a natural quasiisomorphism $\coker\phi\cong C(\phi)$.
    If $\phi$ is surjective then there is a natural quasiisomorphism $C(\phi)\cong \ker\phi[1]$.
    Moreover there is a long exact sequence on cohomology \begin{align*}
        \cdots \to \cH^{r-1}(\cG^*) \to \cH^{r-1}(C(\phi)) \to \cH^r(\cF^*) \to \cH^r(\cG^*) \to \cH^r(C(\phi)) \to \cdots
    \end{align*}
\end{lemma}

\begin{lemma}
    Let $C^**$ be a first quadrant double complex with exact rows and so that the zeroth
    horizontal arrows $d_h^{0q}$ are injections (which is the same as saying we 
    can add an extra zero to the left end of each row). Then the total complex
    $T$ is exact.
\end{lemma}

\begin{proof}
    We will check that the total complex is exact at $T^2$. Let $x = x_{02} + x_{11} + x_{20} \in T^2$
    and suppose that $dx = 0$. Then \begin{align*}
        d_v x_{02} &= 0 \\
        d_vx_{11} + d_hx_{02} &= 0 \\
        d_vx_{20} - d_hx_{11} &= 0 \\
        d_hx_{20} &= 0
    \end{align*} and since the bottom row is exact, we have $x_{20} = d_hy_{10}$ for some $y_{10}$.
    Consider $x'_{11} = x_{11} - d_vy_{10}$. Then \begin{align*}
        d_hx'_{11} = d_hx_{11} - d_hd_vy_{10} = d_hx_{11} - d_vd_hy_{10} = d_hx_{11} - d_vx_{20} = 0
    \end{align*} Since the row is exact, we have $x'_{11} = d_hy_{01}$ for some $y_{01}$,
    and the argument continues. Check that $d(y_{01} + y_{10}) = x$. The 
    argument is the same in the general case. 
\end{proof}

\begin{corollary}
    Let $C^{**}$ be a first quadrant double complex with exact rows. Let $T^*$
    be the total complex and $A^r = \ker(d^{0,r}:C^{0,r}\to C^{1,r})$ be the subcomplex of the zeroth 
    column with vertical differential. Then the inclusion $A^* \to T^*$ is a quasiisomorphism.
\end{corollary}

\begin{proof}
    Augment $C^{**}$ by putting the complex $A^*$ in the $-1$th column. Then the total 
    complex $S^*$ of the augmented complex is precisely the mapping cone of 
    the inclusion $A^* \to T^*$. Therefore there is a long exact sequence in cohomology \begin{align*}
        \cdots \to \cH^{r-1}(T^*) \to \cH^{r-1}(S^*) \to \cH^r(A^*) \to \cH^r(T^*) \to \cdots
    \end{align*} and by the previous lemma, $S^*$ is exact. Therefore the inclusion $A^* \to T^*$
    is a quasiisomorphism.
\end{proof}

\begin{remark}
    What have we done? Given a complex of sheaves $A^*$, take an injective resolution of each
    sheaf and stack them up to get a double complex. It's not so obvious you can do this
    but Goresky assures me we can. Then what we are saying is that 
    we can take the total complex and the resulting map $A^* \to T^*$ is a quasiisomorphism.
\end{remark} Thus we end up with the following definition which works in any abelian category.

\begin{definition}
    An \textbf{injective resolution} of a complex $A^*$ is a quasi-isomorphism $A^*\to T^*$
    where each $T^r$ is an injective object. The cohomology $H^r(X,A^*)$ is the cohomology
    of the complex of global sections of any injective resolution of $A^*$.
\end{definition}

\begin{example}
    Consider the resolution of a single sheaf $S$. It is a quasi-isomorphism \begin{center}
        \begin{tikzcd}
            0 \arrow[r] & S \arrow[r] \arrow[d] & 0 \arrow[r] & 0 \arrow[r] & \cdots \\
            0 \arrow[r] & I_0 \arrow[r] & I_1 \arrow[r] & I_2 \arrow[r] & \cdots
        \end{tikzcd}
    \end{center}
    Recall the Poincare lemma which says that closed forms on a smooth manifold are locally, when restricted to 
    a Euclidean ball, exact. The statement about sheaves is that we have a resolution of the 
    constant sheaf by the de Rham complex \begin{center}
        \begin{tikzcd}
            0 \arrow[r] & \R \arrow[r] \arrow[d] & 0 \arrow[r] & \cdots \\
            0 \arrow[r] & \Omega^0 \arrow[r] & \Omega^1 \arrow[r] & \cdots
        \end{tikzcd}
    \end{center} which one again proves the de Rham isomorphism $H^r(M,\R) \cong H^r_{dR}(M)$.
\end{example}

\begin{definition}
    The Cech cohomology of a sheaf $\cF$ on a topological space $X$ is defined for 
    an open cover $\cU$ of $X$ by the complex \begin{align*}
        0 \to \cF(X) \to \prod_{U\in \cU} \cF(U) \to \prod_{U,V\in \cU} \cF(U\cap V) \to \cdots
    \end{align*} The cohomology of this complex is the \textbf{Cech cohomology} $\hat H^*(X,\cF)$.
\end{definition}

\begin{theorem}
    Suppose $\cF$ is a sheaf on $X$ and $\cU$ is an open cover of $X$ so that \begin{align*}
        H^r(U_J,\cF) = 0 \text{ for all $r>0$ and all finite intersections $U_J$}
    \end{align*} Then there is a canonical isomorphism
    \begin{align*}
        H^*(X,\cF) \cong \hat H^*(X,\cF)
    \end{align*}
\end{theorem}

The proof of this theorem is via first sheafifying the Cech complex which is functorial, picking an injective resolution
of $\cF$ and combining the two to get a double complex. The cohomology of this total complex is equal to 
both the LHS and RHS of the theorem.

\subsection{Derived categories and derived functors}
The motto is: derive by taking injective resolutions.
\subsubsection{First construction}
\begin{definition}
    Two morphisms of complexes are \textbf{homotopic} if there is a chain of maps
    $h^r:A^r\to B^{r-1}$ so that $d_Bh + hd_A = d_A$. Let $[A^*,B^*]$ be the set of homotopy classes
    of morphisms of complexes.
     Define the complex of abelian groups \begin{align}
        \Hom^n(A^*,B^*) = \prod_r \Hom(A^r,B^{r+n})
    \end{align} with differential $d(f) = d_Bf - (-1)^nf d_A$. 
\end{definition}
We have the following \begin{align*}
    H^n(\Hom^*(A^*,B^*)) = [A^*,B^*[n]]
\end{align*}

\begin{definition}

    The \textbf{bounded homotopy category }$K^b(X)$ of complexes of sheaves on a topological space $X$
    is the category whose objects are bounded complexes of sheaves 
    and whose morphisms are homotopy classes of maps.

    \hfill

    The \textbf{bounded derived category} $D^b(X)$ is homotopy category of injective sheaves. Its objects
    are bounded complexes of injective sheaves and its morphisms are homotopy classes of maps.
\end{definition}

\begin{remark}
    In topology, homotopic maps induce the same map on cohomology. In the derived category, the same is true.
    However, the following lemmas will show that once you restrict your objects to injective complexes,
    every quasi-isomorphism between injective complexes is in fact a homotopy equivalence.
    Therefore, this interprets the statement "inverting quasi-isomorphisms" as being the same as 
    restricting to the injective objects.
\end{remark}

This lemma is where the work happens and then everything else follows.
\begin{lemma}
    Let $C^*$ be a bounded complex of sheaves and suppose that all cohomology sheaves are zero. 
    Let $J^*$ be injective. Then any map $C^*\to J^*$ is homotopic to zero.
\end{lemma}
The proof is quite standard, but we get a lot out of it: 

\begin{corollary}
The following hold:
\begin{enumerate}
    \item Suppose $J^*$ is injective and has no cohomology. Then $J$ 
    is homotopy equivalent to the zero complex.
    \item Let $\phi:X^*\to Y^*$ be a quasi-isomorphism of bounded complexes of injective sheaves. 
    Then $\phi$ is a homotopy equivalence.
    \item Let $A^*\to I^*$ and $B^*\to J^*$ be injective resolutions of complexes $A^*$ 
    and $B^*$ and $f:A^*\to B^*$ a map of complexes. There is a lift 
    $\tilde f:I^*\to J^*$ unique up to homotopy.
    \item $f:A^*\to B^*$ a quasiisomorphism induces an isomorphism on hypercohomology.
\end{enumerate} 

\begin{remark}
    Injective objects are like CW complexes, and these facts are basically about 
    the homotopy theory of CW complexes. The first statement is Whitehead's theorem.
\end{remark}
There is a functor $K^b(X) \to D^b(X)$ which sends a complex of sheaves 
to its "canonical" Godemont injective resolution. If $A^* \to B^*$
is a quasi-isomorphism of complexes of sheaves, then it becomes 
an isomorphism in the derived category.

\begin{definition}
    The \textbf{right derived functor} $RT: D^b(X) \to D^b(Y)$ of a functor 
    $T:\Sh(X)\to \cC$ is defined by replacing a complex $A^*$ 
    by its injective resolution $I^*$ and applying $T$ to get $TI^*$.
\end{definition}

\begin{remark}
    We emphasize that the hypercohomology of $R\Hom$ is exactly the group of 
    homomorphisms in the derived category: \begin{align*}
        H^0(R\Hom(A^*,B^*)) = \Hom_{D^b(X)}(A^*,B^*)
    \end{align*}
\end{remark}

\subsubsection{Second construction}
The derived category can also be interpreted as a quotient 
category of the category of complexes of sheaves, by inverting
quasi-isomorphisms. Let $E^b(X)$ be the category of bounded
complexes of sheaves on $X$ with a morphism $A^*\to B^*$ being the data of a diagram \begin{center}
\begin{tikzcd}
	& {C^*} \\
	{A^*} && {B^*}
	\arrow["qi", from=1-2, to=2-1]
	\arrow[from=1-2, to=2-3]
\end{tikzcd}
\end{center} up to the equivalence relation between two morphisms $C^*_1$ and $C^*_2$ if there is 
a diagram like \begin{center}
    \begin{tikzcd}
        & {C_1} \\
        A & {C_3} & B \\
        & {C_2}
        \arrow[from=1-2, to=2-1]
        \arrow[from=1-2, to=2-3]
        \arrow[from=2-2, to=1-2]
        \arrow["qi"', from=2-2, to=2-1]
        \arrow[from=2-2, to=2-3]
        \arrow[from=2-2, to=3-2]
        \arrow[from=3-2, to=2-1]
        \arrow[from=3-2, to=2-3]
    \end{tikzcd}
\end{center}
and then there is a natural functor $D^b(X) \to E^b(X)$ which is an equivalence of categories. 

\begin{definition}
    A functor $T:C\to D$ between abelian categories is \textbf{exact}
    if it preserves exact sequences and \textbf{left exact} if it
    preserves kernels. An object $X$ is \textbf{$T$-acyclic} if $H^r(TX) = 0$
    for all $r>0$.
\end{definition}
The advantage of $T$-acyclic objects is that one can use them in place of injective objects
when computing derived functors. 

\begin{lemma}
    Let $T$ be left exact and $A^* \to X^*$ a quasi-isomorphism of complexes. 
    If $X^r$ is $T$-acyclic for all $r>0$ then $RT(A^*)$ can be computed by
    the complex $TX^*$.
\end{lemma}

\begin{proof}
    The proof follows from the fact that if the rows of a double complex are exact, then 
    the total complex is exact. The condition that $X^r$ is $T$-acyclic for all $r>0$ is about
    the rows of the double complex being exact once you apply $T$ to an injective resolution.
\end{proof}
\begin{remark}
    Fine, flabby, soft sheaves are all $\Gamma$-acyclic. Therefore, 
    sheaf cohomology may be computed with any of those resolutions.
\end{remark}
\end{corollary}


\subsection{Verdier duality}
\subsubsection{Borel-Moore homology}
Borel and Moore defined a sheaf $C^*_{BM}$ whose presheaf of sections is the 
"locally finite r dimensional chians in $U$". The cohomology of this sheaf is called the 
\textbf{Borel-Moore homology} $H^*_{BM}(X)$. Moreover,
the compactly supported cohomology of this sheaf is the same as the singular homology of $X$.
\begin{align*}
    H^{-i}_c(X,C^*_{BM}) = H_i(X)
\end{align*} and the stalk cohomology is the local homology of $X$ at $x$. \begin{align*}
    H^{-r}_x(X,C^*_{BM}) = H_r(X,X-x)
\end{align*}

\subsubsection{The dual of a complex}
Borel and Moore also gave a way to define the dual $\D(S^*)$ of a complex of sheaves $S^*$.
Unfortunately, the double dual of $S^*$ is not $S^*$. Later
Verdier discovered another way to interpret the BM dual sheaf theoretically:
\begin{align*}
    \D(S^*) = R\Hom^*(S^*,\D^*)
\end{align*} where $\D^*$ is a universal sheaf called the dualizing complex. Then 
Verdier showed that there is a canonical quasiisomorphism
in the derived category \begin{align*}
    S^* \to \D(\D(S^*))
\end{align*} thereby restoring double duality. When we are talking about $R=\Z$-modules,
the dualizing complex is precisely $\D^* =  R\Hom^*(\Z,\D^*) = \D(\Z)$ is the 
Borel-Moore dual of the constant sheaf, so it is the sheaf of chains.

\begin{definition}
    If $f:X\to Y$ is a continuous map and $S^*$ is a complex of sheaves on $Y$
    define $f^!(S^*) = \D_X(f^*(\D_Y(S^*)))$.
\end{definition}

\begin{definition}
    Let $i:Z\to X$ closed subspace and $j:U\to X$ open complement. 
    If $S$ is a sheaf on $X$ define $i^!(S)$ to be the restriction to $Z$ of the 
    presheaf with sectinos supported in $Z$, that is \begin{align*}
        i^!(S) = i^*(S^Z) \text{ and } S^Z(V) = \set{s\in S(V) \st \supp s \subset Z}
    \end{align*} Thus if $W\subset Z$ is open then \begin{align*}
        i^!(S)(W) = \lim_{V\supset W} S(V)
    \end{align*}
\end{definition}

\begin{definition}
    A triangle of morphisms
    \begin{align*}
        A^* \to B^* \to C^* \to A^*[1]
    \end{align*} in $D^b(X)$ is \textbf{distinguished} if it is homotopy equivalent to a mapping cone 
    $C(\phi)$ of a morphism $\phi:A^*\to B^*$. This means that there should be maps between 
    the objects in the two triangles so that the squares commute up to homotopy.
\end{definition}

\begin{theorem}
    [Verdier duality] Let $f:X\to Y$ be a 
    stratified mapping between Whitney stratified spaces. Let $A^*,
    B^*,C^*$ be constructible sheaves of abelian groups of $X,Y,Y$ respectively. Then 
    $f^*,f^!,Rf_*,Rf_!$ take distinguished triangles to distinguished triangles. There
    are canonical isomorphisms in $D^b_c(X)$ as follows.

    \begin{enumerate}
        \item $\D\D(A^*) \cong A^*$
        \item $\D^*_X \cong f^!(\D^*_Y)$
        \item $f^!(A^*) \cong \D_X(f^*(\D_Y(A^*)))$
        \item $Rf_!(A^*) \cong \D_Y(Rf_*(\D_X(A^*)))$ so $f^!$ is the dual of $f^*$
        and $Rf_!$ is the dual of $Rf_*$.
        \item $f^!R\Hom^*(B^*,C^*) \cong R\Hom^*(f^*B^*,f^!C^*)$
        \item $Rf_*(R\Hom^*(A^*,f^!B^*)) \cong R\Hom^*(Rf_!A^*,B^*)$ is the statement of Verdier
        duality. This says that $Rf_!$ and $f^!$ are adjoint just as $Rf_*$ and $f^*$ are adjoint.
        \item $Rf_*R\Hom^*(f^*B^*,A^*) \cong R\Hom^*(B^*,Rf_!A^*)$
        \item $Rf_!(R\Hom^*(A^*,f^!B^*)) \cong R\Hom^*(Rf_!A^*,B^*)$
        \item If $f:X\to Y$ is an open inclusion then $f^!B^* \cong f^*B^*$ 
        \item If $f:X\to Y$ closed inclusion then $Rf_!A^* \cong Rf_*A^*$.
        \item If $f:X\to Y$ inclusion of oriented submanifold and $B^*$ is 
        cohomologically locally constant on $Y$ then $f^!B^* \cong f^*B^*[\dim Y - \dim X]$.
    \end{enumerate}
\end{theorem}