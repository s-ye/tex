The language of perverse sheaves and D-modules are critical 
in modern methods of geometric representation theory. 
For example, applications of perverse sheaf methods played an essential
 role in the proof of the Kazhdan-Lusztig conjecture
 by Beilinson-Bernstein and Brylinski-Kashiwara
, Lusztig’s construction of canonical bases in quantum groups, and the work of Beilinson-Drinfeld on 
the Geometric Langlands conjecture. 
Historically, these tools have also proved very effective in the study of a 
central object in geometric representation theory called the affine Grassmannian. 

\subsection{Springer correspondence}
\subsubsection{Springer resolution}
The Springer correspondence is a way of realizing 
the irreducible representations of the Weyl group in a geometric way. In particular 
Springer realizes the group algebra \begin{align*}
    \Q[W] \cong H^{BM}_{2\dim \tilde N}{\tilde N\times_N \tilde N}
\end{align*} where the right hand side 
has a canonical basis given by the irreducible components.

\hfill

Let $G/B$ be the flag variety for a connected reductive algebraic group. 
The Lie algebras are $\mf g$ and $\mf b$. If $x\in G$ and $xB\inv{x} = B$ 
then $x\in B$ and so we can identify $G/B$ with the set of subgroups of $G$
which are conjugate to $B$, or equivalently the set of all subalgebras of $\mf g$
that are conjugate to $\mf b$, that is \textbf{the variety of all Borel subalgebras} of $\mf g$.

\hfill

Let $\cN\subset \mf g$ be 
the cone of nilpotent elements and let \begin{align*}
    \tilde \cN = \set{(x,\mf b)\in \cN\times G/B \st x\in \mf b}
\end{align*}

\begin{lemma}
    The projection $\tilde \cN \to G/B$ gives an isomorphism of bundles $\tilde \cN \cong T^*G/B$
\end{lemma}

\begin{proof}
    The tangent space to $G/B$ at the identity is $\mf g/\mf b$ so its dual space is \begin{align*}
        T^*_I(G/B) = \set{\phi:\mf g \to \C \st \phi(\mf b) = 0}
    \end{align*} The Killing form $\mf g \times \mf g \to \C$ given by
    $x,y\mapsto \tr(\ad x \circ \ad y)$ is nondegenerate and we pair $\mf g$ with $\mf g^*$ 
    to get \begin{align*}
        T^*_I(G/B) = \set{x\in \mf g \st \langle x,\mf b\rangle = 0} = \mf n
    \end{align*} is the nilradical of $\mf b$. So for each Borel subgroup $A\subset G$, 
    the cotangent space $T^*_{A}(G/B)$ is the nilradical of $\text{Lie}(A)$, is exactly 
    the fiber of $\tilde \cN$ over $A$.
\end{proof}

The \textbf{Springer resolution} is the map $\pi:\tilde \cN \to \cN$
 given by projection.
Therefore $\tilde \cN$ carries a natural holomorphic symplectic form (i.e. a 2,0 form)
and the Springer resolution map is semismall. One constructs an action of $W$ on 
$\pi_*\Q_{\tilde \cN}[\dim \tilde \cN]$ and then extends it to an algebra homomorphism 
$\Q[W] \to \End_{\cD_{\cN}}(\pi_*\Q_{\tilde \cN}[\dim \tilde \cN])$
which is isomorphic to the desired BM homology group. To construct the initial action,
one observes that there is a special $W$-fibration and then one pushes this action 
of $W$ on the fiber around. 

\hfill

There is the Chevalley map $q:\mf g\to \mf t/W$ which sends a 
matrix to the roots of its characterstic polynomial. Consider $\mf t_{rs}$
the regular semisimple elements of $\mf t$, obtained by ripping out the 
root hyperplanes, and consider a dominant chamber $\mf t_{rs}/W$.
Consider $\mf g_{rs} = q^{-1}(\mf t_{rs}/W)$ and 
\begin{align*}
    \tilde{\mf g}  &= \set{(x,\mf b)\in \mf g\times G/B \st x\in \mf b} \\
    \tilde{\mf g_{rs}} &= \pi^{-1}(g_{rs})
\end{align*} 
Then the map $\tilde{\mf g_{rs}} \to \mf g_{rs}$ is a $W$-fibration and 
the map $\tilde g \to \mf g$ is small. Associated to the $W$-covering is 
a local system \begin{align*}
    L = \pi'_*\Q_{\tilde g_{rs}}
\end{align*} where $\pi'$ is the restriction of $\pi$ to $\tilde g_{rs}$.
Then they extend to an action of $W$ on intersection cohomology and pushforward perverse
sheaves along the small map.

\begin{remark}
    The image of the root hyperplanes is the discriminant variety of all polynomials
    with multiple roots. The complement of the union of root hyperplanes is 
    the \textbf{configuration space of $n$ ordered points} in $\C$ with $\pi_1 = $ the \textbf{
        colored braid group
    } The complement of the image is the configuration space on $n$ unordered points and
    has $\pi_1$ = \textbf{braid group}.
\end{remark}

\subsubsection{Algebra of correspondences}
There is a general construction described in \cite{ginzburg} 
which he advertises as a method of geometrically constructing 
representations of finite dimensional algebras. The idea is to
introduce the convolution product on Borel Moore homology, which is 
supposed to generalize the convolution of functions. 

\begin{definition}
    Let $M_1,M_2,M_3$ connected oriented smooth manifolds and let $Z_{12}\subset M_1\times M_2$
    and $Z_{23}\subset M_2\times M_3$ be closed subsets. The \textbf{composition} of
    $Z_{12}$ and $Z_{23}$ is the set \begin{align*}
        Z_{12}\circ Z_{23} = \set{(x_1,x_3)\in M_1\times M_3 \st \exists x_2\in M_2 \text{ s.t. } (x_1,x_2)\in Z_{12} \text{ and } (x_2,x_3)\in Z_{23}}
    \end{align*}
    The \textbf{convolution in Borel-Moore homology} generalizes 
    this to cycles and is defined as follows:
    \begin{align*}
        H^{BM}_i(Z_{12}) \times H^{BM}_j(Z_{23}) \to H^{BM}_{i+j}(Z_{12}\circ Z_{23})
        c_1,c_2 \mapsto c_1 * c_2
    \end{align*} where 
    \begin{align*}
        c_1 * c_2 := \pi_{13*}(\pi_{12}^*c_1 \cap \pi_{23}^*c_2)
    \end{align*}
\end{definition}

The convolution product is associative. Now let $\mu:M\to N$ a proper map 
of complex varieties and consider $M_1 = M_2 = M_3 = M$ and $Z = Z_{12} = Z_{23} 
= M\times_N M$. Then we get convolution maps \begin{align*}
    H^{BM}_*(Z) \times H^{BM}_*(Z) \to H^{BM}_{*}(Z)
\end{align*} The convolution product is not graded, but it does preserve the 
middle dimension. If $\dim_C M = n$ then the convolution product
\begin{align*}
    H^{BM}_{2n}(Z) \times H^{BM}_{2n}(Z) \to H^{BM}_{2n}(Z)
\end{align*} and we call this \textbf{middle dimensional subalgebra} $H(Z)$.

\subsubsection{Sheaf theory applied to the convolution algebra}
The convolution product makes $H^{BM}_*(Z)$ into an algebra. 
It turns out that we can express the algebra structure as 
the Ext algebra of a particular generator. Let $\cC_M$ be the constant 
perverse sheaf on $M$ i.e. $\cC_M = \C_M[\dim M]$ extended along irreducible
components.

\begin{proposition}
    There is a (not necessarily grading preserving) natural algebra 
    isomorphism \begin{align*}
        H^{BM}_*(Z) \to \Ext^*_{D^bN}(\mu_*\cC_M,\mu_*\cC_M)
    \end{align*} 
\end{proposition}

Assume that $\mu:M\to N$ is productive and that $N$ is stratified
 so that the restriction maps are all locally trivial topological fibrations.
 We can study the convolution algebra by analyzing the pushforward
 $\mu_*\cC_M$ of the constant sheaf on $M$. Applying the 
 decomposition theorem we find that \begin{align*}
    H^{BM}_*(Z) &\cong \bigoplus_{k\in \Z} \Ext^k_{D^bN}(\mu_*\cC_M,\mu_*\cC_M) \\
    &= \bigoplus_{i,j,k\in\Z,\phi,\psi} \Hom_\C(L_\phi(i),L_\psi(j))\otimes \Ext^k_{D^bN}(IC_\phi[i],IC_\psi[j]) \\
    &= \bigoplus_{i,j,k\in\Z,\phi,\psi} \Hom_\C(L_\phi(i),L_\psi(j))\otimes \Ext^k_{D^bN}(IC_\phi,IC_\psi) \text{ reindexing}
 \end{align*}
where $L_\psi$ is the multiplicity space of the decomposition of $\mu_*\cC_M$ into irreducible 
IC sheaves.
\red{Now it is a fact that $\Ext^k_{D^bN}(IC_\phi,IC_\psi)$ vanishes if $k<0$}
Also $\Hom(IC_\phi,IC_\psi)$ is nonzero only if $\phi = \psi$. Therefore 
we find that \begin{align*}
    H^{BM}_*(Z) \cong \bigoplus_\phi \End_C(L_\phi) \oplus \big(\bigoplus_{k>0,\phi,\psi} \Hom_\C{L_\phi,L_\psi}\otimes
    \Ext^k_{D^bN}(IC_\phi,IC_\psi)\big)
\end{align*} The first summand is semisimple and the second $H^{BM}_*(Z)^+$ is nilpotent because
it is concentrated in degrees $k>0$. Moreover, this nilpotent ideal is the radical 
of our algebra because 
\begin{align*}
    H^{BM}_*(Z) / H^{BM}_*(Z)^+ = \bigoplus_\phi \End_C(L_\phi)
\end{align*} is semisimple. The composition \begin{align*}
    H^{BM}_*(Z) \to H^{BM}_*(Z) / H^{BM}_*(Z)^+ \cong \bigoplus_\phi \End_C(L_\phi) \twoheadrightarrow \End_C(L_\phi)
\end{align*} yields an irreducible representation of the algebra $H^{BM}_*(Z)$ on the vector space $L_\psi$.
\begin{theorem}
    The nonzero members of the collection $\set{L_\phi}$ are the 
    irreducible representations of the algebra $H^{BM}_*(Z)$.
\end{theorem}

\subsubsection{Semi-small maps}
When $\mu$ is semismall then the previous calculation becomes nicer 
since the shifts go away. In particular, we have the following: 
\begin{theorem}
    \begin{enumerate}
        \item Let $\cC_M$ be the constant perverse sheaf on $M$. If $\mu$ is semismall
        then $\mu_*\cC_M$ is a perverse sheaf and we have a decomposition without shifts \begin{align*}
            \mu_*\cC_M = \bigoplus_{N_\phi,\chi_\phi} L_\phi \otimes IC_\phi
        \end{align*} where $\phi = (N_\phi,\chi_\phi)$ is a pair of a stratum
        a local system on the stratum. Furthermore, $H(Z)$ is a
        subalgebra of $H^{BM}_*(Z)$ and \begin{align*}
            H(Z) \cong \bigoplus_\phi \End_C(L_\phi)
        \end{align*}
        \item Let $H(M_x)$ denote the top 
        Borel-Moore homology of the fiber $\mu^{-1}(x)$. For any 
        stratum $N_\alpha$, $x\mapsto H(M_x)$ is a local system on $N_\alpha$.
        If $x\in N_\phi,\chi_\phi$ then \red{the corresponding 
        multiplicity space is given by the isotopy invariants of the
        top cohomology of the fiber}. \begin{align*}
            L_\phi = H(M_x)^{\pi_1(N_\alpha,x)}
        \end{align*}
    \end{enumerate}
\end{theorem}
\subsubsection{Applying the machinery to the Springer resolution}
In this section we explain the following setup and theorem.
Set $Z = \tilde \cN\times_\cN \tilde \cN$ the Steinberg variety.
If $x\in \cN$ then $M_x$ is formed by pairs $(x,\mf b)$ where $\mf b$ 
runs over the subset $\cB_x$ of $x$ invariant Borel subalgebras. Equivalently 
any element $x\in \mf g$ induces a vector field on $G/B$ and $\cB_x$ is the
zeros of this vector field.

\hfill 

Let $G(x)$ be the centralizer of $x$ in $G$, $A(x) = G(x)/G(x)_0$ the isotopy
group acting on the connected components. Let $A(x)^*$ denote the set of isomorphism classes
of $A(x)$-representations occuring in the BM homology groups $H_{top}(M_x)$.
The main techincal result known as "Springer construction of Weyl groups"
is the following:

\begin{theorem} [Geometric Construction of $W$]
    \begin{enumerate}
        \item $H(Z) \cong \C[W]$
        \item The collection $\{H(M_x)_\phi\}$ as $(x,\phi)$ runs over $G$ conjugacy
        classes of points in $\cN$ and $\phi\in A(x)^*$ is a complete set of irreducible
        representations of $W$.
    \end{enumerate}
\end{theorem}

\subsubsection{Fourier transform}
We introduce the main tool of the construction, the Fourier transform
on perverse sheaves (or D-modules). 

\subsection{Schubert varieties and Kazhdan Lustzig polynomials}
Let $G$ be an algebraic group. The Kazhdan Lustzig polynomials are a family of polynomials defined for
two Weyl group elements $v,w$ in a Weyl group $W$ with a system of 
generators $S$. 

\begin{definition}
    The \textbf{Hecke algebra} $\cH$ is the algebra of $B$-bi-invariant 
    functions of $G(\F_q)$.
    The algebra structure is given by normalized convolution.
    It has a basis consisting of functions \[\phi_w = \id_{BwB} \text{ for } w\in W\]
\end{definition}

\hfill

\begin{lemma}
    If $s\in S$ is a simple reflection and $w\in W$ then we have the following relations
    \begin{align*}
        \phi_w * \phi_{w'} &= \phi_{ww'} \text{ if } l(ww') = l(w) + l(w') \\
        \phi_s * \phi_s &= (q-1)\phi_s + q\phi_1 \\
        \phi_s * \phi_w &= (q-1)\phi_w + q\phi_{sw} \text{ if } l(ws) = l(w) + 1
    \end{align*}
\end{lemma}

We have the "standard" description of Hecke algebra
\begin{proposition}
$\cH$ is the free $\Z[q,\inv q]$-module with basis $\set{\phi_w}_{w\in W}$ and relations
\begin{align*}
    \phi_s\phi_w = \phi_{sw} \text{ if } l(sw) = l(w) + 1 \\
    (\phi_s - q)(\phi_s + 1) = 0
\end{align*}   
If $q = 1$, then this is the group algebra $\Z[W]$. 
\end{proposition}
The convention is to use $q^{\half}$ and $q^{-\half}$ as the formal parameters.
Each $\phi_w$ is invertible and the algebra admits an involution \begin{align*}
    i(q^{\half}) = q^{-\half} \text{ and } i(\phi_w) = \phi_{\inv w}^{-1}
\end{align*}

\begin{theorem}
    [Kazhdan-Lusztig] For each $w$ there is a unique $c_w\in \cH$ and 
    a uniquely determined polynomial $P_{yw}$ with $y\leq w$ so that $i(c_w) = c_w$ and
    $P_{ww} = 1$ and $P_{yw}(q)$ has degree less than $\half(l(w) - l(y)-1)$ when $y<w$ and 
    \begin{align*}
        c_w = q^{-l(w)/2}\sum_{y\leq w} P_{yw}(q)\phi_y
    \end{align*}
\end{theorem}
Kazhdan and Lusztig conjectured that the coefficients 
of the polynomials $P_{yw}(q)$ are nonnegative integers and that in the 
Grothendieck group of Verma modules \begin{align*}
    [L_w] = \sum_{y\leq w} (-1)^{l(w)-l(y)}P_{yw}(1)[M_y]
\end{align*}
The second conjecture became known as the \textbf{Kazhdan-Lustzig conjecture} and 
was proven by Beilinson-Bernstein and Brylinski-Kashiwara independently. The
interpretation of $c_w$ and $P_{yw}$ in terms of intersection cohomology was critical 
to the proof.
\begin{theorem}
    Set for any $v\leq w$ the number $h^i(\bar X_w)_v = \dim \cH^i(\IC_{\bar X_w})_v$.
    Then \begin{align*}
        P_{v,w}(q) = \sum_i h^i(\bar X_w)_v q^i
    \end{align*} is a polynomial in $q$ with nonnegative integer coefficients.
\end{theorem}

\subsection{Geometric Satake Isomorphism}
\begin{theorem}
    The category $\cP_{\cO}$ is equivalent to the category of
    representations of the Langlands dual group $^L G$, as categories with tensor and fiber structures.

\end{theorem}

There is a bilinear functor $\star:\cP_{\cO}\times \cP_{\cO} \to \cP_{\cO}$ with compatible
commutativity, associativity restraints, and a fiber functor which respects the tensor product.
In fact, one defines the desired functor geometrically so that $H$, the operation
of taking cohomology, is the fiber functor. Therefore 
the equivalence is between \begin{align*}
    (\cP_{\cO},\star,H) \cong (\Rep(^L G), \otimes, \text{Forget})
\end{align*}
The $G(\cO)$ orbits of the affine Grassmannian are indexed by dominant coweights $\lambda$.
A shadow of the geometric Satake isomorphism is the statement that the 
intersection cohomology of the $\lambda$-orbit is the irreducible representation of $^L G$
with highest weight $\lambda$.
