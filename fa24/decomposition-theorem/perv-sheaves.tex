\subsection{Precise definitions}
The category of perverse sheaves is defined by relaxing 
the support and cosuppoort conditions for the IC sheaf by one.

\begin{definition}
    Let $W$ be a $n$-dimensional Whitney stratified space. A \textbf{middle perversity 
    perverse sheaf} on $W$ is a complex of sheaves $A^\bullet$ in the bounded 
    constructible derived category $D^b_c(W)$ such that if $S$ is a stratum of 
    dimension $d$, $A^*$ satisfies the support and cosupport conditions
    \begin{align*}
        H^i(j^*_SA^*) = 0 \text{ for all } i > -d/2\\
        H^i(j^!_SA^*) = 0 \text{ for all } i < -d/2
    \end{align*}
\end{definition}

\begin{definition}
    A \textbf{perversity} is a function on dimension $p:\Z_{\geq 0}\to \Z_{\geq 0}$
     such that $p(0) = 0$ \begin{align*}
        p(d) \geq p(d+1) \geq p(d) - 1
     \end{align*} Middle perversity is the perversity $p(d) = -d/2$. The 
     category of perverse sheaves with perversity $p$ is those objects in $D^b_c(W)$
     for which \begin{align*}
        H^i(j^*_SA^*) = 0 \text{ for all } i > p(\dim S)\\
        H^i(j^!_SA^*) = 0 \text{ for all } i < p(\dim S)
    \end{align*}
\end{definition}
Each perversity involves its own shift: for a space $W$ of dimension $n$ the stalk
cohomology of the IC sheaf in the top stratum is nonzero in degree $p(n)$.

\begin{remark}
    One can check that perversity $0$ corresponds precisely to the 
    category of constructible sheaves (not complexes!, just sheaves).
\end{remark}

\begin{theorem}
    The category of middle perverse sheaves forms an abelian
    subcategory of $D^b_c(X)$ that is preserved by Verdier duality.
\end{theorem}

\begin{theorem}
    If $W$ is an algebraic variety, the simple objects 
    are the shifted IC sheaves with irreducible local coefficients of irreducible 
    subvarieties.
\end{theorem}

\begin{remark}
    To a $\cD$-module there is a corresponding sheaf of solutions which 
    is constructible. Beilinson, Bernstein, Brylinski, and Kashiwara showed that
    each Verma modlue can be associated to a certain 
    \red{holonomic $\cD$-module with regular singularities} whose sheaf of solutions 
    turns out to be the IC sheaf. However,
    the category of $\cD$-modules is abelian whereas the constructible
    derived category is not, so it was conjectured that there is some abelian subcategory of the 
    category which "receives the solution sheaves". This is precisely
    the category of perverse sheaves with middle perversity!
\end{remark}

\subsection{Examples}
\begin{example}
There are some special examples where we have conmbinatorial descriptions of the
middle perverse sheaves on an algebraic variety with respect to a
stratification. \begin{enumerate}
    \item $\C^n$ with respect to hyperplane arrangements
    \item Square matrices with respect to the rank stratification
    \item The flag variety with respect to the Schubert stratification
\end{enumerate}
In particular the most simple example is $\C, \{0\}$. The category of perverse
sheaves is equivalent to the category of representations of the quiver \begin{center}
    \begin{tikzcd}
        \bullet \ar[r, bend left, "\alpha"] & \bullet \ar[l, bend left, "\beta"]
    \end{tikzcd}
\end{center} for which $I-\alpha\beta$ and $I-\beta\alpha$ are invertible.
\end{example}

\begin{example}
    Stratify $\P^1$ with a single zero dimensional stratum $N$ the north pole. The
    support diagram for middle perversity sheaves is 
    \begin{center}
        \begin{tabular}{|c|c|c|}
            \hline
            i $\backslash$ codim & 0 & 2\\
            \hline
            2 & c & c \\
            \hline
            1 &  & cx \\
            \hline
            0 & x & x\\
            \hline
        \end{tabular}
    \end{center}
    The columns index codimension of the strata and the rows index the cohomological
    degree (the convention is $0$ to $n$ as opposed to $-n/2$ to $n/2$). x denotes 
    that there may be nontrivial stalk cohomology sheaves supported along strata
    of that given codimension. c denotes that the same thing but with compactly supported
    cohomology stalks.

    \hfill 

    The category of perverse sheaves is equivalent to the category of representations 
    of the quiver \begin{center}
        \begin{tikzcd}
            \bullet \ar[r, bend left, "\alpha"] & \bullet \ar[l, bend left, "\beta"]
        \end{tikzcd} 
    \end{center} for $\alpha\beta = \beta\alpha = I$.
    \red{Somehow I want to relate this to the category of perverse sheaves on $\C, \{0\}$.
    Does a stratified map descend to a relationship between these categories?}
    In particular there are 5 such indecomposable representations and consequently,
    5 simple perverse sheaves. They are 
    $\Q_N[-1],\Q_{\P^1},j_!\Q_U, j_*\Q_U$ with the support diagrams
    \begin{center}
        \begin{tabular}{|c|c|c|}
            \hline
            i $\backslash$ codim & 0 & 2\\
            \hline
            2 &  &  \\
            \hline
            1 &  & cx \\
            \hline
            0 &  & \\
            \hline
        \end{tabular}
        \begin{tabular}{|c|c|c|}
            \hline
            i $\backslash$ codim & 0 & 2\\
            \hline
            2 & c & c \\
            \hline
            1 &  &  \\
            \hline
            0 & x & x\\
            \hline
        \end{tabular}
        \begin{tabular}{|c|c|c|}
            \hline
            i $\backslash$ codim & 0 & 2\\
            \hline
            2 &c  & c \\
            \hline
            1 &  &  c\\
            \hline
            0 & x & \\
            \hline
        \end{tabular}
        \begin{tabular}{|c|c|c|}
            \hline
            i $\backslash$ codim & 0 & 2\\
            \hline
            2 &  c&  \\
            \hline
            1 &  &  x\\
            \hline
            0 & x & x\\
            \hline
        \end{tabular}
    \end{center}
    and the last one which is not an IC sheaf. It is gotten by taking a closed disk and putting the constant sheaf on the interior
    and the 0 sheaf on the boundary, except for a point, and then pushing 
    this sheaf forward along the map $D^2\to S^2$ which collapses the boundary. It has
    stalk cohomology and compactly supported stalk cohomology in degree $1$. Verdier duality
    interchanges these conditions and so this sheaf is self-dual. The first two sheaves
    are self-dual as well and the last two are dual to each other. Verdier duality can be seen
    by reflecting the support diagram across the horizontal axis and swapping $x$ and $c$.
    \red{Does the support diagram of an IC sheaf determine the sheaf up to isomorphism?}
\end{example}

\subsection{t-structures and perverse sheaves}
\subsubsection{Motivation: perversity 0 t-structure}
Let $W$ be a stratified space. The category of perversity 0 complexes 
of sheaves is equivalent to the category of 
constructible ordinary sheaves $\Sh_c(W)$. 

\begin{definition}
    Let $A^*$ be a complex of sheaves on $W$. 
    We have \textbf{truncation functors} $\tau_{\leq r} $ and $ \tau_{\geq r}$ defined by
    \begin{align*}
        A^* = A^{r-1} \to A^r \to A^{r+1} \to \cdots\\
        \tau_{\leq r}A^* := A^{r-1} \to \ker(d^r) \to 0 \to \cdots\\
        \tau_{\geq r}A^* := 0 \to \coker(d^{r-1}) \to A^r \to A^{r+1} \to \cdots
    \end{align*}
\end{definition}

Then there is a short exact sequence $0\to \tau_{\leq 0}A^* \to A^* \to \tau_{\geq 1}A^* \to 0$
and the cohomology sheaf of $A^*$ is given by \begin{align*}
    H^i(A^*) = \tau_{\leq i}\tau_{\geq i}A^*
\end{align*} In particular we have the following theorem 
which is the version which we can generalize. \begin{theorem}
    The cohomology functor $H^r: D^b_c(W) \to \Sh_c(W)$ is given
    by the composition of the trunctation functors.The functor $H^0$ restricts to an equivalence
    of categories between $\Sh_c(W)$ and the full subcategory of 
    $D^b_c(W)$ consisting of complexes $A^*$ such that $H^i(A^*) = 0$ for $i\neq 0$.
    This category is Noetherian and Artinian and its simple objects
    are the sheaves $j_!(\cE)$ where $\cE$ is a simple local system on a single 
    stratum $j:X\to W$.
\end{theorem}

\subsubsection{General t-structures}
Fix a perversity $p$ and let $\cP(W)$ be the category of 
perverse sheaves with perversity $p$ on $W$. Then \begin{proposition}
    There are truncation functors \begin{align*}
        ^p\tau_{\leq r} : D^b_c(W) \to D^b_c(W)\\
        ^p\tau_{\geq r} : D^b_c(W) \to D^b_c(W)
    \end{align*} which take distinguished triangles to exact seuqences and 
    satisfy \begin{align*}
        ^p\tau_{\leq r}(A^*) = (^p\tau_{\leq 0}(A^*)[r])[-r]\\
    \end{align*}
\end{proposition}

\begin{definition}
    We have the \textbf{perverse cohomology}
    \begin{align*}
        ^p H^r(A^*) = ^p\tau_{\leq r}^p\tau_{\geq r}A^*
    \end{align*}
\end{definition}

\begin{theorem}
    The perverse cohomology functor $^pH^r: D^b_c(W) \to \cP(W)$ is given
    by the composition of the truncation functors. 
    The functor $^pH^0$ restricts to an equivalence
    of categories between $\cP(W)$ and the full subcategory of 
    $D^b_c(W)$ consisting of complexes $A^*$ such that $^pH^i(A^*) = 0$ for $i\neq 0$.
    This category is Noetherian and Artinian and its simple objects
    are the sheaves $Rj_*(\IC_p^*(\cE))$ where 
    $\cE$ is a simple local system on a single stratum $X$ and $j:\bar X\to W$.
\end{theorem}

\begin{remark}
    More generally, a $t$-structure on a 
    triangulated category is a pair of strictly full subcategories $\cD^{\geq 0}$
    and $\cD_{\leq 0}$ satisfying technical
    conditions as above. In this case, the heart $P = \cD^{\geq 0}\cap \cD_{\leq 0}$ 
    is an abelian full subcategory.
\end{remark}

\subsection{Algebraic varieties and the decomposition theorem}
\subsubsection{Intersection cohomology}
Intersection cohomology enjoys many of the same remarkable properties as ordinary
cohomology for algebraic varieties. In particular, the Lefschetz hyperplane theorem,
the hard Lefschetz theorem, the Lefschetz decomposition theorem, and the 
Hodge structure all hold for intersection cohomology.

\begin{theorem}
    Let $W\subset \C\P^N$ projective algebraic variety of dimension $n$. 
    Let $L^j$ be a generic linear subspace of codimension $j$ in $\C\P^N$.
    If $L^j$ is transverse to each stratum of a Whitney stratification of $W$,
    then there are natural restriction maps \begin{align*}
        IH^r(W) \to IH^r(W\cap L^j) \\
        IH^r(W\cap L^j) \to IH^{r+2j}(W)
    \end{align*} If $L^1$ is transverse to $W$ then 
    the restriction map is an isomorphism for $r\leq n-2$ and an injection for $r = n-1$.
    If $j\geq 1$ and $L^j$ transverse to $W$ then the composition 
    \[L^j:IH^{n-j}(W) \to IH^{n-j}(W\cap L^j) \to IH^{n+j}(W)\] is an isomorphism.
\end{theorem}

\begin{theorem}
    We have the Leftschetz decomposition which says that \begin{align*}
        IH^r = \bigoplus_{j\geq 0} L^jIP^{r-2j}
    \end{align*} where the primitive cohomology 
    $IP^i\subset IH^i$ is the kernel of $\cdot L^{n-i+1}$. Poincare duality \begin{align*}
        IH^{n+r}(W,\Q) \cong \Hom(IH^{n-r}(W,\Q),\Q)
    \end{align*} and the Lefschetz isomorphism $L^r: IH^{n-r}(W) \to IH^{n+r}(W)$ 
    induce a nondegenerate bilinear pairing on $IH^{n-r}(W)$. The Lefschetz decomposition is orthogonal 
    with respect to this pairing and its signature is described by the 
    \red{Hodge-Riemann bilinear relations}.
\end{theorem}

\subsubsection{Decomposition theorem}
The decomposition theorem was first formulated and proved by Beilinson, Bernstein, 
and Deligne. It is one of the deepest theorems we have about the topology of algebraic maps.
Let $f:X\to Y$ be a proper map of algebraic varieties. The decomposition theorems says that 
$Rf_*IC_X$ breaks into a direct sum of intersection complexes of subvarieties of $Y$
with coefficients in various local systems, with shifts.

\begin{theorem}
    [Decomposition theorem] Let $f:X\to Y$ be a proper map of algebraic varieties.
    \begin{enumerate}
        \item $Rf_*IC^*_X = \bigoplus_i ^p \cH^i(Rf_*IC^*_X)[-i]$ (
            this says that the push forward sheaf is a direct sum of perverse sheaves,
            in particular its own perverse cohomology sheaves
        )
        \item Each summand is a semisimple perverse sheaf. In particular it is 
        a direct sum of shifted IC sheaves with local coefficients on irreducible
        subvarieties of $Y$, and it enjoys all of the above properties of intersection
        cohomology.
        \item Relative hard-Lefschetz theorem: if $\eta$ is the class of a hyperplane of 
        $X$ then for all $r$ the map $\eta^r:^p\cH^{-r}(Rf_*IC^*_X) \to ^p\cH^{r}(Rf_*IC^*_X)$
        is an isomorphism.
    \end{enumerate}
\end{theorem}

\subsubsection{Example: square pyramid}
The square pyramid does not correspond to a smooth toric variety because it is singular 
at the tip where the four faces meet. However there is a resolution of singularities
$\pi:\tilde Y\to Y$ given by corner chopping. We consider the decomposition theorem for this 
map. 

\hfill 

The map is an isomorphism except at the singular point whose fiber $\pi^{-1}(y) = \P^1\times 
\P^1$. The stalk cohomology of the pushforward $R\pi_*\Q_{\tilde Y}$ is $\Q,0,\Q\oplus\Q,0,\Q$ 
\red{is equal to the cohomology of the fiber}. We consider the support diagram 
\begin{center}
    \begin{tabular}{|c|c|c|c|c|c|}
        \hline
        i $\backslash$ codim & 0 & 2 & 4 & 6 & $H^*(\pi^{-1}(y))$ \\
        \hline
        6 & c & c & c & c & \\      
        \hline
        5 &  &  & c & c &  \\
        \hline
        4 &  & & & c & $\Q$ \\
        \hline
        3 &  &  &  & 0 & 0 \\
        \hline
        2 &  &  &  & x & $\Q\oplus\Q$ \\
        \hline
        1 &  &  &  & x & x \\
        \hline
        0 & x & x & x & x & $\Q$ \\
        \hline
    \end{tabular}
\end{center}
The support condition at the bottom of the table says $\Q[3]$ is part of the IC sheaf.
The $\Q[-1]$ at the top of the table is not part of the IC sheaf. By duality, neither is one of 
the copies of $\Q$ in the middle. Finally, we can show that the other copy of $\Q$ in the middle
is part of the IC sheaf. These two terms constitute the primitive cohomology of the fiber.



