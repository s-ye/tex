\documentclass[12pt]{article}
\usepackage[english]{babel}
\usepackage[utf8x]{inputenc}
\usepackage[T1]{fontenc}
\usepackage{scribe}
\usepackage{listings}
\usepackage{quiver}
\usepackage{tikz}



\begin{document}

Songyu Ye 

sy459@cornell.edu

\today
\section{Sept 18 - First look at toric varieties}

We consider the following polytopes and their associated toric varieties.
\begin{example}
    \[\begin{tikzcd}
	\bullet_a & \bullet_b
	\arrow[no head, from=1-1, to=1-2]
\end{tikzcd}\]
The associated toric variety is $\Proj \C[a,b] \cong \P^1$. However there is more data in this diagram. In particular, there is a corresponding map of line bundles that is part of the data.
\[\begin{tikzcd}
	{\text{id}^*(\mathcal O(1)) \cong \mathcal O(1)} & {\mathcal O(1)} \\
	{\mathbb{P}^1} & {\mathbb{P}^1}
	\arrow[from=1-2, to=2-2]
	\arrow["{\text{id}}", from=2-1, to=2-2]
	\arrow[from=1-1, to=1-2]
	\arrow[from=1-1, to=2-1]
\end{tikzcd}\] In particular this comes from the map of rings $\C[x,y]\to\C[x,y]$.
\end{example}

\begin{example}
    \[\begin{tikzcd}
	\bullet_x & \bullet_y & \bullet_z
	\arrow[no head, from=1-1, to=1-2]
	\arrow[no head, from=1-2, to=1-3]
\end{tikzcd}\]
The associated toric variety is also $\Proj \C[x,y,z]/\langle xz - y^2\rangle \cong \P^1$. However, there is also the data of how $\P^1$ embeds into ambient space. In this case, we have the ring map $\C[x,y,z]\to\C[x,y,z]/\langle xz - y^2\rangle$ which gives us an embedding $\P^1\to\P^2$ in which $\P^1\subset \P^2$ as a conic.

\hfill

Pulling back the line bundle $\mc O(1)_{\P^2}$ along the inclusion we get 
\[\begin{tikzcd}
	{??} & {\mathcal O(1)_{\mathbb{P}^2}} \\
	{\mathbb{P}^1} & {\mathbb{P}^2}
	\arrow[from=1-2, to=2-2]
	\arrow["{\text{id}}", from=2-1, to=2-2]
	\arrow[from=1-1, to=1-2]
	\arrow[from=1-1, to=2-1]
\end{tikzcd}\] As it turns out, we should be getting $\mc O(2)_{\P^1}$. This of course has to do with the fact that we are embedding $\P_1\subset \P^2$ as a conic. In particular, we can do some dimension counting (of some vector space of monomials, I think $\Sym^k(V)$) and a complete classification of the line bundles over projective space (there are not that many).
\end{example}

\begin{fact}
In general, we have that \begin{align*}
    \Gamma(\P V,O(k))\cong\Sym^k(V)
\end{align*} where $\Sym^k(V)$ is $V^{\otimes k}$ modded out by action of the symmetric group $S^k$. In particular, if we choose a basis $x_1,\dots,x_n$ for $V$, we can identify a basis of $\Sym^k(V)$ with monomials of degree $k$ (the symmetry allows me to commute the tensor powers past each other and write it as a monomial). Thus $\dim\Sym^k(V) = \binom{n+k-1}{k}$.
\end{fact}

\begin{example}
In Example 1.2, we have $\dim V = 2$ and $k = 2$. There are 3 monomials of degree 2 in variables $a$ and $b$, namely $a^2,ab,b^2$. Somehow the number 3 was important in this example, which is how we knew it was supposed to be $\mc O(2)_{\P^1}$
\end{example}


\begin{example}
    \[\begin{tikzcd}
	& \bullet & \bullet & \bullet \\
	\bullet & \bullet & \bullet
	\arrow[no head, from=2-1, to=2-2]
	\arrow[no head, from=2-2, to=2-3]
	\arrow[no head, from=2-3, to=1-4]
	\arrow[no head, from=2-1, to=1-2]
	\arrow[no head, from=1-2, to=1-3]
	\arrow[no head, from=1-3, to=1-4]
\end{tikzcd}\]
This toric variety is the product of the two previous examples, in particular it is $\P^1\times\P^1\hookrightarrow\P^5$ (there are six lattice points). On the level of rings there is a map \begin{align*}
    \C[a,b,c,d,e,f]\to \C[a,b,c,d,e,f] / \langle \text{parallelograms}\rangle
\end{align*}
We can get the line bundle over this toric variety as follows. Pullback the line bundles along the projection maps.
\[\begin{tikzcd}
	& {\pi_1^*(\mathcal{L}_1)} && {\pi_2^*(\mathcal{L}_2)} \\
	{\mathcal{L}_1} && {X\times Y} && {\mathcal{L}_2} \\
	& X && Y
	\arrow[from=2-3, to=3-2]
	\arrow[from=2-3, to=3-4]
	\arrow[from=2-1, to=3-2]
	\arrow[from=2-5, to=3-4]
	\arrow[from=1-4, to=2-3]
	\arrow[from=1-4, to=2-5]
	\arrow[from=1-2, to=2-1]
	\arrow[from=1-2, to=2-3]
\end{tikzcd}\]
Then form $L_1 \boxtimes L_2= \pi_1^*(\mathcal{L}_1) \otimes\pi_2^*(\mathcal{L}_2) $. Recall that the tensor product of two vector bundles is the fiberwise tensor product of vector spaces.
\end{example}
\begin{fact}
    The faces of the polytope are in correspondence with the $T$-orbits.
\end{fact}
\red{Serre twisting sheaf, tautological line bundle, Fubini-Study metric, Delzant Theorem, Spherical Variety}
\red{algebraic vector bundle vs. geometric vector bundle, cocycle condition and transition maps}
\red{Why do we care about classification of vector bundles over a variety? What does their classification tell us about the variety? There are topological $K$-groups which concern the classification of vector bundles which are interesting to algebraic geometers?}
\section{Sept 25 - toric varieties}
We will probably follow Fulton's \textit{Toric Varieties}
\red{want to ask Allen about Hirzebruch surfaces}
\section{Oct 2 - Blowups}
\begin{theorem}
If $S = \Spec k[x_1,\dots,x_n]/I$ and $R = Spec k[x_1,\dots,x_n]/J$ subschemes of affine space and $I\leq J$, i.e. we have $J = I + \langle g_1,\dots,g_n \rangle$ then there exists affine space $V,W$ and $e\subset V\times W$ so that $R = S \cap (0\times W)$.
\end{theorem}
\begin{proof}
\red{include graphic}    
\end{proof}
If we have $J\hookrightarrow A \twoheadrightarrow B$ so $B\cong A/J$ we get a map $\Spec B\hookrightarrow \Spec A$. We can consider the descending chain $A\geq J\geq J^2\dots \geq J^n\geq \dots$ and form \begin{align*}
    A\oplus tJ\oplus t^2J^2\dots \subset A[t]
\end{align*} Taking $\Proj$ of the left hand side (prime homogeneous ideals) gives us the \textbf{blowup of $\Spec A$ along $\Spec A/J$}. There is a map to $\Spec A$ called the blowdown map where I take a prime homogeneous ideal and intersect with $A$ to get a prime ideal of $A$. 

\section{Oct 16 - Toric Surfaces}
We start with the following correspondence:\begin{theorem}
    There is a correspondence between smooth toric surfaces (2 complex dimensional) $M^4$ and "smooth" polygons $P\subset\R^2$. In particular, the vertices of $P$ are anywhere, the edges of $P$ are in lattice directions, and for each vertex, the primitive normal vectors (or equivalently primative tangent vectors) to each adjacent edge span $\Z^2$.
\end{theorem}

\hfill

Given a smooth polygon $P$ as above we can build our toric surface as follows. Form the topological space $T^2\times P/\sim$ where $(s,p)\sim(t,q)$ if $p=q\in\partial P$ and $s^{-1}t\in T_F$ for $F$ vertex or edge, where $T_F = T^2$ if $F$ is a vertex and $T_F =$ image of $\R v_i$ under the map $\R^2\to\R^2/\Z^2\cong T^2$ if $F$ is an edge and $v$ is the normal vector to $F$, in particular $T_F$ is a circle.

\hfill

It is not clear at all that what you get is a complex manifold or a variety. One can also do this construction using geometric invariant theory.

\begin{example}
    If $P = I$ the interval then $M_P = S^2$.
\end{example}

Detour into the land of low-dimensional topology for a second. We want to compute the intersection form of this 4-manifold. \red{I guess algebraic geometers also care about this but I learned this stuff from Tara} We do this as follows. We have an obvious map $\pi:M_P\to P$ which is well-defined since we only identified points with $p = q$.

\begin{theorem}
    The composition $f:M_P\to P\to\R$ is Morse, where $P\to \R$ is projection onto any affine subspace which is not parallel to the normal vectors $v_i$. In particular, the critical points of $F$ are precisely $\pi^{-1}(\text{vertices of $P$})$ and the index is twice the \red{the number of edges pointing down}. In particular we have exactly one c.p of index $0$, $n-2$ c.p of index $2$, one c.p of index $4$. Moreover we can compute the self-intersection numbers of the spheres on the boundary via \begin{align*}
        d_i = e(v(S^2_i)) = \det[v_{i-1},v_{i+1}]
    \end{align*} where $e$ is for Euler class. \red{There is some relevant theorem that ?-bundles over ? are completely determined by their Euler class, whatever that is}
\end{theorem}
\red{Wow there is a lot of content in this theorem. Hopefully Tara can unpack some of this.}

Now let us thing about $H^2(M_P)$. We claim that $[S^2_i]$ for $i=1,\dots,n-2$ is a basis for $H^2(M)$. \red{I just realized I don't feel really comfortable about computations with homology or the idea of Poincare duality or homology classes of embedded surfaces, I should probably reconcile this with Nikhil at some point. } Then with respect to this basis we have that the intersection \begin{align*}
    Q_m = \begin{bmm}
        d_1 & 1 & \\
        1 & d_2 & 1 \\
         & 1 & d_3 & 1 \\
         &  & 1 & d_4 & \dots \\
        \vdots
    \end{bmm}
\end{align*} We get $1$ in the super/sub diagonal because adjacent spheres intersect in precisely one point, the point in the fiber over the vertex in which the two edges intersect. The diagonal is precisely the self-intersection numbers of the spheres.
\begin{example}
    A right triangle with legs $1,1$ corresponds to $\C\P^2$. If I apply a corner chop operation ("toric blowup") to one of the two corners not opposite the hypotenuse, then this gives me the blowup $\tilde{\C\P^2}$. It is an example of Hirzebruch surface \red{can add some stuff from Fulton Hirzebruch surfaces} and in particular it is an $S^2$-bundle over $S^2$.
\end{example}

\begin{example}
This example, we took a first look at K3-surfaces. Start with $T^4 = (S^1)^4$ and consider the involution $\sigma:T^4\to T^4$ defined on each on $S^1\subset \C$ by complex conjugation. Then $T^4/\sigma$ is not a manifold, but has 16 $\Z/2$-singularities. We say that $\T^4/\sigma$ is an orbifold.
\end{example}


After speaking with Tara, I spoke with Allen. Let's see what we can get out of what he said.

Consider the following symplectic point of view. We have $(T^*A,\omega)$ is naturally a symplectic manifold \red{um check this in Lee}. Now let $A$ be a group. Then \begin{align*}
    T^*A = A\times T^*_eA = A \times \mf A^*
\end{align*} the dual of the lie algebra. Now let $A = T^n$ and we see that \begin{align*}
    T^*T^n =  T^n\times\mf t = T^n \times \R^n
\end{align*} and we have the inclusion $T^n\times P \hookrightarrow T^n \times \R^n$. Now we can compute \begin{align*}
    T_{(t,p)}(T^*A) &= T_tT\oplus T_p\mf t \\
    &\cong t\oplus t^*
\end{align*} and one can put a symplectic pairing on this space. Then the point is that the restriction of this symplectic form may be degenerate once we restrict $T\times P \hookrightarrow T\times \R^n$. The solution? We collapse the vectors which are null with respect to the sympletic form on the manifold with corners. Note that we can actually do this construction with any manifold with corners, but most of the tijme the resulting quotient space will fair basic nice properties, e.g. Hausdorff.

\begin{example}
    This is a nonexample. One can consider a pyramid with a square base $\subset \R^3$ and the point is that this $P$ is not smooth because it has 4 edges meeting at the top which is too many (the resulting quotient manifold will not have a neighborhood homeomorphic to $U\subset\R^4$).
\end{example}

\red{Make sure to ask about toric blowups. Allen had those diagrams up with the Hopf fibration}
\red{"Kirby moves"}

\section{CoxLittleSheanck}
Let $F\subset\C^n$ be a closed algebraic subvariety. Which should be the basic functions on V? Since we are doing algebraic geometry, we should consider the polynomials on $\C$. There are two ways to go from $\C^n$ to $V$. One can consider those functions that are restrictions of polynomials or those functions that are locally restrictions of polynomials. Fortunately these two notions agree. Such a function is called regular on V.

A rational function $f$ is called regular at $f\in V$ if there are polynomials $g$ and $h$ such that $f = g/h$ and $h(v) \neq 0$. A rational function is called regular on V if it is regular at each point. One can see that a regular rational function is a regular function.

Elements of $\mc O(V)$ are functions that are locally quotients of polynomials, e.g functions from $V$ to $k$ that are locally rational. What this equality in the case where $\mc O(V)$ is affine is saying is that functions that are locally rational on an affine variety are globally polynomials.

Let $F\subset\C^n$ be an algebraic variety and $v\in V$. $V$is said to be normal at $v\in V$ if every rational function bounded in some neighborhood of $v$ is regular at $v$. $V$ is called normal if it is normal at every point. In particular if $V$ is normal at $v$, then a rational function is regular at $v$ iff it is continuous at $v$. \red{Wow this seems like some really deep complex geometry stuff. Why is this the same as saying that we have this integrally closed condition?}

Riemann's extension theorem says that $\C$ is normal. From this it follows that smooth points are normal in all dimensions.

\begin{definition}
    A normal variety is one whose local rings are integrally closed in their fraction fields.
\end{definition}
\begin{fact}
    A variety is normal if it is covered by open affine $V$ which are normal. This is because in general, we have for all $U\subset V$ open \begin{align*}
        \mc O_{U,p} = \mc O_{V,p}
    \end{align*}
\end{fact}
\begin{fact}
    An affine variety $V$ is normal if and only if $\Gamma(V)$ is normal.
\end{fact}
\begin{proof}
    We have that \begin{align*}
        \mc O_X(X) = \bigcap_{p\in X}\mc O_{X,p}
    \end{align*} and the intersection of integrally closed rings will be integrally closed. Conversely, if $\Gamma(V)$ is normal, then so is each localization $\Gamma(V)_S$. These are commutative algebra statements that need to be checked.
\end{proof}

\begin{fact}
    An irreducible affine variety $V$ has a normalization $V' = \Spec\C[V]'$ where \begin{align*}
        \C[V]' = \set{\alpha\in\C(V)\st\alpha \text{integral over $\C[V]$}}
    \end{align*} the integral closure of the coordinate ring of $V$. One can show that this is integrally closed as the name would suggest and moreover finitely generated as a $\C$-algebra and hence $V'$ is a normal affine variety, called the normalization of $V$. There is a canonical map $V'\to V$ induced by the inclusion of rings $V'\to V$ which is a birational isomorphism. Moreover, the normalization satisfies the universal property that any other normal variety $Z$ and dominant map $Z\to V$ factors uniquely through $V'\to V$.
\end{fact}

\begin{proposition}
    A smooth irreducible affine variety $V$ is normal.
\end{proposition}
\begin{proof}
    Need to see that $\mc O_{V,p}$ is normal whenever $p$ is smooth. This comes from commutative algebra: $p$ smooth is equivalent to $\mc O_{V,p}$ is a regular local ring, every regular local ring is a UFD, and every UFD is normal.  
\end{proof}

\section{Meeting 11/6}
We talked about Hironaka's result on the resultion of singularities. The statement is as follows (one can look in Kollar's book or this AMS article https://www.ams.org/journals/bull/2003-40-03/S0273-0979-03-00982-0/S0273-0979-03-00982-0.pdf for a quick discussion): 

\begin{theorem}
    Let $X$ be a reduced scheme over a field of characteristic zero. We have \begin{align*}
        X_{sing}\hookrightarrow X \hookleftarrow X_{reg}
    \end{align*} Then there exists a regular scheme $\tilde{X}$ and a birational proper map $\tilde{X}\to X$ so that $X_{reg}\hookrightarrow \tilde{X}$ is an open immersion and moreover the pullback $X_{sing}\times_X \tilde{X}$ is a simple (intersection is connected) normal crossing (union of smooth things where the intersections look like hyperplane) divisor (codimension 1 subvariety)
\end{theorem}

\begin{definition}
    For $X$ scheme and $P\in X$, we have the Zariski cotangent space at $P$ which we define to be $m_p/m_p^2$ where $m_p\subset \mc O_{X,P}$ is the maximal ideal of the local ring $O_{X,P}$. Morally we are looking at germs of functions which vanish at $P$ and then modding out by $m_p^2$ to kill off any of the nonlinear terms.
\end{definition}

\begin{example}
    Consider a blue $\Z^2$ sitting inside black $\Z^2$ so that the quotient is of order 2. Then in the picture, I have a normal scheme with 2 tori acting, the smaller lattice acts faithfully and the bigger lattice corresponds to a 2:1 map $T_2\to T_1$. In particular we have \begin{align*}
        0\to blue \to black\to \{1,-1\}\to 0
    \end{align*} and we can take $\Hom(-,\C^*)$ we get \begin{align*}
        0\to \{\pm 1\}^* \to T \to T/\{\pm 1\}^*\to 0
    \end{align*} sequence of tori.
\end{example}

\begin{definition}
    Normalization is a universal map $\tilde X\to X$ which is finite and biractional (meaning that the fibers are finite and the map is proper). For toric varieties what this means is to throw in all of the lattice points which are differences of guys in your semigroup. This corresponds to thinking about the integral closure.

    \hfill

    We also have the seminormalization which fits in between the two guys \begin{align*}
        \tilde X\to X_{semi}\to X 
    \end{align*} which we ask that hte map $X_{semi}\to X$ is further a bijection.
\end{definition}

\section{Meeting toward the end of the semester}
There are a package of theorems about localization in equivariant cohomology which I care about: For "good spaces" (e.g. reductive groups acting on algebraic varieties) we have \begin{enumerate}
    \item the map $H^*_GX\to H^*X$ is a surjection
    \item the map $H^*_GX\to H^*_GX^T$ is an injection
    \item the integration formula when $X^G $finite\begin{align*}
        \int_X\alpha = \sum_{p\in X^G}
    \end{align*}
\end{enumerate}

\begin{example}
    We have the polytopal pictures for the Hirzebruch surface $F_1$, these pictures live inside $\mf t^*$ which allow us to give a generators-relations presentation of the surface \begin{align*}
        H^*_T(F_1) = \Z[\upsilon_1,\dots,\upsilon]/\ideal{\upsilon_1\upsilon_3,\upsilon_2,\upsilon_4}
    \end{align*} The presentation of this ring looks different if we instead insist on thinking of $H_T^*(F_1)$ as an algebra over $H_T^*(*)$ in which case we would have some new variables $X,Y$ so that $X = \upsilon_2 - \upsilon_4$ and $Y = \upsilon_1 - \upsilon_3 - \upsilon_2$, in particular you should think of $X$, $Y$ as directions in $\mf t^*$ which are telling us coefficients on $\upsilon_i$ via dotting.
\end{example}

\begin{example}
    Let's think about a simpler example. We have $\C\P^1$ which has moment polytope given by an interval, whose endpoints we label $a$ and $b$. The localization theorem tells us that \begin{align*}
        H_T^*(\P^1) = \Z[a,b]/\ideal{ab}
    \end{align*} with $\deg a = \deg b = 2$ since $a$ and $b$ do not touch. Moreover that this ring injects into $H_T^*(\P^{1,T}) = H_T^* \oplus H_T^* = \Z[y]\oplus\Z[y]$. The injection sends $a$ to $(y,0)$ and $b$ to $(0,-y)$ based on the weights on the action and degree consideration. 

    \hfill
    
    \red{We remarked earlier that it is not quite correct to think of this cohomology theory as having $\Z$-coefficients, in particular we should think of our ring as an algebra over $H^*_T$}

    \hfill

    In this case there is an obvious diagonal map $H_T^*\to H_T^*(\P^{1,T}) \cong H_T^* \oplus H_T^*$ and this map factors through $H_T^*(\P^1)$ if and only if $Y$ maps to something in $H_T^*(\P^1)$ which then maps to $(Y,Y)$ we quickly see that the only option is $Y\mapsto a-b$. Then we get the presentation of the cohomology \begin{align}
        H_T^*(\P^1) \cong H_T^*[a,b]/\ideal{ab,a-b-y}
    \end{align}

\end{example}

\begin{example}
    Returning to the Hirzebruch surface example, we see from cofrner chopping that $F_1$ is the blowup of $\P^2$ and if we blowup the vertex point $c$, we have this pretty picture \begin{align*}
        F_1 = \set{(p,l) \st p,c\in l} 
    \end{align*} and this maps to $\P^1$ by projection. In the polytopal picture, you can visualize this map as collapsing the trapezoid onto the bottom base.
\end{example}

\begin{example}
    Recall that $\Sym(\mf t^*) \cong H_T^*$ and the isomorphism comes from I will take a weight $\lambda \in \mf t^* \mapsto c_1(L)$ where $L$ is the line bundle over a point corresponding to the 1-dimensional representation in which $T$ acts on $\C$ by $\lambda$. More generally, this is a really stupid instance (I think where $G = B = T$) of Borel Weil Theorem which according to Allen is about the following commutative square. % https://q.uiver.app/#q=WzAsNCxbMSwwLCJLX0cocHQpIl0sWzAsMCwiS19HKEcvQikiXSxbMCwxLCJcXG1hdGhmcmFre3R9XioiXSxbMSwxLCJIX0cocHQpIl0sWzEsMF0sWzAsM10sWzIsM10sWzIsMV1d
\[\begin{tikzcd}
	{K_G(G/B)} & {K_G(pt)} \\
	{\mathfrak{t}^*} & {H_G(pt)}
	\arrow[from=1-1, to=1-2]
	\arrow[from=1-2, to=2-2]
	\arrow[from=2-1, to=2-2]
	\arrow[from=2-1, to=1-1]
\end{tikzcd}\] 
\end{example} where the left arrow and the top arrow are corresponding to Borel-Weil.

\begin{example}
    At the end we started talking about loop spaces of compact group, such as $SU(2)$ as in the HHH paper. \red{In particular it is a sympletic manifold and there is an $S^1$ action given by acting by free loops and then recentering. Unfortunately it is not Hamiltonian but there is something we can do ...? } Namely we can look at the universal central extension for whic h

    \hfill

    Inthe HHH paper we are thinking about the problem of developing a GKM theory for the homogenous spaces of loop groups. Tara obtains these very nice pictures of the moment polytope in the HHH paper. In particular, if we go on the heuristic that cohomology classes are about finite codimension submanifolds and homology classes are about finite dimensional manifolds, then we can start thinking about \red{the pictures in the case that we are thinking about submanifolds which are neither finite dimension or codimension. This is something that Allen said he didn't understand the computations for. What is an example of what you were thinking about?}
\end{example}

\begin{example}
    The moment polytope for $\C\P^\infty$ is a ray going in the positive real direction. There is an $S^1$ action on $\P(\bigoplus_{n\in\N}\C_n)$. Cap product makes $H^*_{S^1}(\C\P^\infty)$ into a module over $H_*^{S^1}(\C\P^\infty)$. Inside the equivariant homology we have the class of $\P\C_3$ which we can expand in point classes to get $(0,0,0,1,0,\dots)$. Now consider the homology class of $\P(\C_3\oplus\C_4)$. How do we expand this in a basis? We need to invert some elements, and then we get that it expands as $(0,0,0,1/L, 1/R,0,\dots)$ where $L$ is a left arrow and $R$ is a right arrow.
\end{example}
\begin{center}
    \includegraphics[angle = 270,scale = .1]{img/IMG_0892.jpeg}
\end{center}

\section{Theory of principal G bundles and characteristic classes}

\section{Another note for Allen}
Last time we talked some more about subvarieties of infinite dimensional GKM spaces 
which are neither finite dimension nor codimension. He described the process by which 
one can slice up the root lattice (imagine a broomstick) into ways which correspond to 
infinite dimension and codimension subvarieties, and then thinking about GKM pictures 
for these subvarieties.

He mentioned his most successful student Joel Kamnitzer who graduated from UC Berkeley in 2005.
In particular to take a look at some of his work. However I could not find any trace of what
he was talking about at the dinner.

\section{Meeting 12/4}
Recall some definitions:

\begin{definition}
    Let $X = \Spec R$ an affine scheme. Given $M$ an $R$-module, there is an obvious way of defining
    a sheaf of $\mc O_X$-modules on $X$ $\tilde M$ via \begin{align*}
        \tilde M(D(f)) = M[\inv{f}]
    \end{align*} We say that a sheaf $\mc F$ of $\mc O_X$-modules is \textbf{quasicoherent} if $X$ admits an open affine cover
    by $\Spec U_i$ so that $\mc F\vert_{U_i}\cong \tilde{M_i}$ for some $M_i$ a module over the corresponding ring of functions.
    \red{This is the local triviality condition, why should it be equivalent to the thing that one asks for about topological vector bundles?}
    We say that $\cF$ is \textbf{coherent} if $M_i$ is finitely generated over $U_i$. We say that $\cF$ is \textbf{free} if it is isomorphic to the some number of copies of the 
    structure sheaf. We say that $\cF$ is \textbf{locally free} if $X$ can be covered by open sets for which $\cF\vert_U$ is free.
    If $X$ is connected, then we can talk about the rank of a locally free sheaf. An \textbf{invertible sheaf} is a locally free sheaf of rank 1.
    \end{definition}

    We then have the following proposition which tells us that quasicoherent sheaves over affine schemes are 
    precisely those which are globally of the form $\tilde M$ for $M$ a module over $\cO_X$.

    \begin{proposition}
        [Hartshorne 5.4] Let $X$ be a scheme. Then an $\cO_X$ module $\cF$ is 
        quasicoherent if and only if for every affine open $U = \Spec R$, $\cF\vert_U \cong \tilde M$ for some $R$-module $M$.
        For $X$ Noetherian, $\cF$ is coherent if and only if it is quasicoherent and $\cF\vert_U$ is finitely generated over $R$ for every affine open $U = \Spec R$.
    \end{proposition}

    We can now consider the same question for projective schemes $X = \Proj R$. The affine story tells us that quasicoherent sheaves
    should roughly be about graded modules. Recall the proj construction associated to a graded ring $R = \bigoplus_{d\geq 0} R_d$. 
    We can form the set of homogeneous prime ideals $\mf p\subset R$ which do not contain the irrelevant ideal $R_+ = \bigoplus_{d>0} R_d$.
    We can topologize this set by declaring a basis of open sets indexed by $f$ homogeneous ideals to be $D(f) = \set{\mf p\in \Proj R \st \mf p \not\supset f}$ 
    Moreover $\Proj R$ has a structure sheaf $O_{\Proj R}$ defined by 
    \begin{align*}
        O_{\Proj R}(D(f)) = R[\inv{f}]_0
    \end{align*} 
    where the subscript $0$ denotes the degree 0 part of the localization. 

    \begin{proposition}
        [Hartshorne 2.5] Let $R$ be a graded ring. 
        \begin{enumerate}
            \item For any $\mf p \in \Proj R$, the stalk $\cO_p$ is isomorphic to the local ring $R_{(\mf p)}$.
            \item For any homogeneous $f\in R_+$, the open set $D_+(f)$ is isomorphic to $\Spec R_{(f)}$, the degree 0 part of the localization $R_f$.
            \item $\Proj R$ is a scheme.
        \end{enumerate}
    \end{proposition}

    \begin{example}
        Consider $R = k[x_0,\dots,x_n]$ with its usual grading. Then the closed points are $\Proj R$ are 
        in correspondence with lines through the origin. We have an open cover of $\Proj R$ by $D(x_i)$ 
        and we have that $O_{\Proj R}(D(x_i)) = R[\inv{x_i}]_0 = k[x_0/x_i,\dots,x_n/x_i]_0 = k[x_0/x_i,\dots,x_n/x_i]$.
    \end{example} 

    Now we can try to play the same game. 

    \begin{definition}
        If $M$ a graded $R$-module, then we can define a sheaf $\tilde M$ on $\Proj R$ by 
        \begin{align*}
            \tilde M(D(f)) = M[\inv{f}]_0
        \end{align*} 
        We can check that this is a sheaf of $\cO_{\Proj R}$-modules.
        \red{grothendieck topology}
    \end{definition}

    \begin{example}
        Let $R = k[x_0,\dots,x_r]$ with its usual grading.
        Let $M$ be a graded module over $R$ and let $M(n)$ denote the graded module with $M(n)_d = M_{n+d}$. \red{there is a convention here}
        Consider the sheaf $\tilde R(n)$ on projective space. Then we have that $\tilde R(0) = \cO_{\Proj R}$ and 
        \begin{align*}
            \tilde R(n) = \text{sheaf whose value of $D(x_i)$ is $R[\inv{x_i}]_n$}
        \end{align*} 
        but observe that $x_i$ is an isomorphism between the degree $n$ part of $R$ and the degree $n+1$ part of $R$.
        Hence $\tilde R(n)$ is locally isomorphic to the structure sheaf $\cO_{\Proj R}$. In particular, we have that $\tilde R(n)$ is a line bundle.
        
        \hfill

        We can see that $\tilde R(n)$ is not isomorphic to $\tilde R(0)$ by looking at the global sections of the line bundle (at least for $n$ positive).
        A global section when restricted to the open sets $D(x_i)$ and $D(x_j)$ determines a degree $n$ element of the rings $k[x_0,\dots,x_r]_{x_i}$ and $k[x_0,\dots,x_r]_{x_j}$.
        If $i\neq j$ then this means that $f$ has no poles in $x_i$ or $x_j$ and therefore global sections must be degree $n$ polynomials in $x_0,\dots,x_r$.
        We finish off the classification by noting that 
        \begin{align*}
            \tilde{M(n)} &\cong \tilde{M}(n) := \tilde{M}\otimes_{\cO_{\Proj R}} \tilde{R}(n) \\
            \tilde{R(m)} \otimes \tilde{R(n)} &\cong \tilde{R}(m+n)
        \end{align*} 
        and so we see that we get a least a $\Z$-s worth of line bundles on $\P^n$.
    \end{example}

    We began by considering the correspondence between line bundles on $X$, maps to projective space $X\to \P V$, 
    and divisors on $X$. 

    Given a point $x\in X$ and $h$ a line bundle on $X$, we can eat global section of the line bundle 
    \begin{align*}
        F:X\to\P\Gamma(X;h)^* \\
        \sigma \mapsto [\sigma\mapsto \sigma\vert_x\in h_x]
    \end{align*} 
    Note that this map is not defined when all sections vanish. 

    \begin{example}
        If you consider line bundles on $\P^1$, a topologist would say to compute Chern classes and then
        one sees that there is one isomorphism class for each integer. An algebraic geometer would agree. 
        
        \hfill
        
        In particular, a line bundle on $\P^1$ is the data of line bundles on its two affine charts $\Spec k[x]$ and $\Spec k[1/x]$, along with a gluing map on the
        intersection of the affine charts, which is a torus $\Spec k[x,x^{-1}]$. Over each of the affine charts, a line bundle (aka an invertible sheaf)
        is trivial in the sense that it is simply $\tilde M$ for $M$ some $k[x]$-module.

    \end{example}
    \begin{example}
        Consider $\Sym^n(\C^2)$ \red{degree $n$ rational normal curve} and pick some random linear map to $V$ some $4$-dimensional vector space.
        There is an induced map $\P\Sym^n(\C^2)\to \P V$ and we can ask about the pullback of the tautological line bundle on $\P V$. Moreover 
        we can consider $\P^1\to \Sym^n(\C^2)$ the Veronese embedding and ask about the pullback of the tautological line bundle on $\P^1$.
        The point is is that the map of sections is not onto. There are sections of $\cO(n)\to\P^1$ (\red{fact: $\cO(n)$ is the line bundle
        you get pulling back along the composition $\P^1\to \P\Sym^n(\C^2)\to V$}) which do not extend to global sections of $\cO(1)\to\P^3$.
    \end{example}
    

    Recall the map $F$ we considered above. \begin{enumerate}
        \item If defined everywhere, then this gives an embedding $X\hookrightarrow \P\Gamma(X;h)^*$ \red{"by a complete linear system"}. We say that $h$ is \textbf{very ample}.
        \item If defined everywhere, then we can pullback the tautological line bundle on projective space along 
        the induced map $X\hookrightarrow \P\Gamma(X;h)^*$ to get a line bundle on $X$, and this will be the same line bundle we started with.
    \end{enumerate}

    \subsection{Divisors} We have maps
    \begin{align*}
        \set{\text{subschemes locally defined by one equation, not zero divisor}} \hookrightarrow \set{\text{subschemes of pure codimension 1}} \\
        \to \set{\Z-\text{linear combinations of irreducible codim 1 subvarieties}} \\
        \text{denote by} A\to B\to C
    \end{align*} where the second map $f$ is you take multiplicities. \red{What do these sets and maps look like for $\P^1$?}
    $A$ has a map to invertible sheaves given by taking rational functions $p/f$ where $f$ is the local equation defining the subscheme. \red{ideal sheaf?}
    $A$ is about Cartier divisors and $B\to C$ is about Weil divisors.

    \begin{example}
        On $\P^1$ $A$ has two guys mapping to isomorphic line bundles. This is because the came from sections of the same line bundle. 
    \end{example}
    \begin{example}
        Consider the toric variety $\C[a,b,c]/\ideal{b^2-ac}$ and consider the ideal $\ideal{a,b}$. This is a subscheme of codimension 1, but it is not irreducible.
        This guy is Weil and not Cartier. Consider sheaves of functions which are allowed to blow up along this ideal. \red{"rank 1 sheaf associated to Weil divisor"}
    \end{example}

    \subsection{Toric stuff}
    If I have $C\hookrightarrow X$ a smooth curve and a line bundle on $X$ I can restrict the line bundle to $C$ and I am supposed to get an integer 
    by the following means. I have a line bundle $\in H_2$ and its Chern class $\in H^2$ and I can \red{integrate the Chern class over the line bundle}
    and I get an integer, the degree of the line bundle.

    \hfill

    Now say I have a line bunlde on a complex toric manifold. Explicitly lets think about $\P^2$ and its blowup.
    We can consider the polytopal corner chopping picture. \begin{enumerate}
        \item Sympletic geometer: care about the area of the new $S^2$ that you get.
        \item AG: care about the degree of the line bundle on the new $\P^1$.
    \end{enumerate}

    \begin{example}
        Consider the map % https://q.uiver.app/#q=WzAsMjUsWzAsMCwiXFxidWxsZXQiXSxbMCwxLCJcXGJ1bGxldCJdLFswLDIsIlxcYnVsbGV0Il0sWzAsMywiXFxidWxsZXQiXSxbMCw0LCJcXGJ1bGxldCJdLFsxLDQsIlxcYnVsbGV0Il0sWzIsNCwiXFxidWxsZXQiXSxbMyw0LCJcXGJ1bGxldCJdLFs0LDQsIlxcYnVsbGV0Il0sWzMsMywiXFxidWxsZXQiXSxbMiwzLCJcXGJ1bGxldCJdLFsxLDMsIlxcYnVsbGV0Il0sWzEsMiwiXFxidWxsZXQiXSxbMiwyLCJcXGJ1bGxldCJdLFsxLDEsIlxcYnVsbGV0Il0sWzYsMCwiXFxidWxsZXQiXSxbNiwxLCJcXGJ1bGxldCJdLFs2LDIsIlxcYnVsbGV0Il0sWzYsMywiXFxidWxsZXQiXSxbNiw0LCJcXGJ1bGxldCJdLFs3LDQsIlxcYnVsbGV0Il0sWzgsNCwiXFxidWxsZXQiXSxbOCwzLCJcXGJ1bGxldCJdLFs4LDIsIlxcYnVsbGV0Il0sWzcsMSwiXFxidWxsZXQiXSxbMCw0LCIiLDAseyJzdHlsZSI6eyJoZWFkIjp7Im5hbWUiOiJub25lIn19fV0sWzQsOCwiIiwwLHsic3R5bGUiOnsiaGVhZCI6eyJuYW1lIjoibm9uZSJ9fX1dLFs4LDAsIiIsMCx7InN0eWxlIjp7ImhlYWQiOnsibmFtZSI6Im5vbmUifX19XSxbMTUsMTksIiIsMCx7InN0eWxlIjp7ImhlYWQiOnsibmFtZSI6Im5vbmUifX19XSxbMTksMjEsIiIsMCx7InN0eWxlIjp7ImhlYWQiOnsibmFtZSI6Im5vbmUifX19XSxbMjEsMjMsIiIsMCx7InN0eWxlIjp7ImhlYWQiOnsibmFtZSI6Im5vbmUifX19XSxbMjMsMTUsIiIsMCx7InN0eWxlIjp7ImhlYWQiOnsibmFtZSI6Im5vbmUifX19XV0=
        \[\begin{tikzcd}
            \bullet &&&&&& \bullet \\
            \bullet & \bullet &&&&& \bullet & \bullet \\
            \bullet & \bullet & \bullet &&&& \bullet && \bullet \\
            \bullet & \bullet & \bullet & \bullet &&& \bullet && \bullet \\
            \bullet & \bullet & \bullet & \bullet & \bullet && \bullet & \bullet & \bullet
            \arrow[no head, from=1-1, to=5-1]
            \arrow[no head, from=5-1, to=5-5]
            \arrow[no head, from=5-5, to=1-1]
            \arrow[no head, from=1-7, to=5-7]
            \arrow[no head, from=5-7, to=5-9]
            \arrow[no head, from=5-9, to=3-9]
            \arrow[no head, from=3-9, to=1-7]
        \end{tikzcd}\]
    \end{example}
    The guy on the left is $\cO(4)\to \P^2$ and the guy on the right is the blowup of $\P^2$ at a point. Look at the new edge that was introduced. It lies on 
    3 lattice points. This is telling us that there is a line bundle over that $\P^1$ with a 3-dimensional space of sections, i.e so it is $\cO(2)$.
    \end{document}
