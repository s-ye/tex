%%% template.tex
%%% This is a template for making up an AMS-LaTeX file
%%% Version of February 12, 2011
%%%---------------------------------------------------------
%%% The following command chooses the default 10 point type.
%%% To choose 12 point, change it to
%%% \documentclass[12pt]{amsart}
\documentclass{amsart}
\usepackage{cite}
%%% The following command loads the amsrefs package, which will be
%%% used to create the bibliography:
\usepackage[lite]{amsrefs}
\usepackage{/Users/songye03/Desktop/math_tex/style/macros}

\usepackage{hyperref}
\hypersetup{
    colorlinks,
    citecolor=black,
    filecolor=black,
    linkcolor=black,
    urlcolor=black
}

\usepackage{parskip} % Automatically respects blank lines
\setlength{\parskip}{1em} % Adds more space between paragraphs
\setlength{\parindent}{0pt} % Removes paragraph indentation

%%% The following command defines the standard names for all of the
%%% special symbols in the AMSfonts package, listed in
%%% http://www.ctan.org/tex-archive/info/symbols/math/symbols.pdf
\usepackage{amssymb}
\usepackage{tikz-cd}
\usetikzlibrary{decorations.pathmorphing}
\usepackage{float}

\usepackage{parskip} % Automatically respects blank lines
\setlength{\parskip}{1em} % Adds more space between paragraphs
\setlength{\parindent}{0pt} % Removes paragraph indentation

%%% The following commands allow you to use \Xy-pic to draw
%%% commutative diagrams.  (You can omit the second line if you want
%%% the default style of the nodes to be \textstyle.)
\usepackage[all,cmtip]{xy}
\let\objectstyle=\displaystyle

%%% If you'll be importing any graphics, uncomment the following
%%% line.  (Note: The spelling is correct; the package graphicx.sty is
%%% the updated version of the older graphics.sty.)
% \usepackage{graphicx}



%%% This part of the file (after the \documentclass command,
%%% but before the \begin{document}) is called the ``preamble''.
%%% This is where we put our macro definitions.

%%% Comment out (or delete) any of these that you don't want to use.


%%%-------------------------------------------------------------------
%%%-------------------------------------------------------------------
%%% The Theorem environments:
%%%
%%%
%%% The following commands set it up so that:
%%% 
%%% All Theorems, Corollaries, Lemmas, Propositions, Definitions,
%%% Remarks, Examples, Notations, and Terminologies  will be numbered
%%% in a single sequence, and the numbering will be within each
%%% section.  Displayed equations will be numbered in the same
%%% sequence. 
%%% 
%%% 
%%% Theorems, Propositions, Lemmas, and Corollaries will have the most
%%% formal typesetting.
%%% 
%%% Definitions will have the next level of formality.
%%% 
%%% Remarks, Examples, Notations, and Terminologies will be the least
%%% formal.
%%% 
%%% Theorem:
%%% \begin{thm}
%%% 
%%% \end{thm}
%%% 
%%% Corollary:
%%% \begin{cor}
%%% 
%%% \end{cor}
%%% 
%%% Lemma:
%%% \begin{lem}
%%% 
%%% \end{lem}
%%% 
%%% Proposition:
%%% \begin{prop}
%%% 
%%% \end{prop}
%%% 
%%% Definition:
%%% \begin{defn}
%%% 
%%% \end{defn}
%%% 
%%% Remark:
%%% \begin{rem}
%%% 
%%% \end{rem}
%%% 
%%% Example:
%%% \begin{ex}
%%% 
%%% \end{ex}
%%% 
%%% Notation:
%%% \begin{notation}
%%% 
%%% \end{notation}
%%% 
%%% Terminology:
%%% \begin{terminology}
%%% 
%%% \end{terminology}
%%% 
%%%       Theorem environments

% The following causes equations to be numbered within sections
\numberwithin{equation}{section}

% We'll use the equation counter for all our theorem environments, so
% that everything will be numbered in the same sequence.

%       Theorem environments

\theoremstyle{plain} %% This is the default, anyway
\newtheorem{theorem}[equation]{Theorem}
\newtheorem{cor}[equation]{Corollary}
\newtheorem{lemma}[equation]{Lemma}
\newtheorem{proposition}[equation]{Proposition}

\theoremstyle{definition}
\newtheorem{definition}[equation]{Definition}

\theoremstyle{remark}
\newtheorem{remark}[equation]{Remark}
\newtheorem{example}[equation]{Example}
\newtheorem{notation}[equation]{Notation}
\newtheorem{terminology}[equation]{Terminology}

%%%-------------------------------------------------------------------
%%%-------------------------------------------------------------------
%%%-------------------------------------------------------------------
%%%-------------------------------------------------------------------
%%%-------------------------------------------------------------------
%%%-------------------------------------------------------------------
%%%-------------------------------------------------------------------
\begin{document}

%%% In the title, use a double backslash "\\" to show a linebreak:
%%% Use one of the following two forms:
%%% \title{Text of the title}
%%% or
%%% \title[Short form for the running head]{Text of the title}
\title{Moment maps and equivariant cohomology in toric geometry}


%%% If there are multiple authors, they're described one at a time:
%%% First author: \author{} \address{} \curraddr{} \email{} \thanks{}
%%% Second author: \author{} \address{} \curraddr{} \email{} \thanks{}
%%% Third author: \author{} \address{} \curraddr{} \email{} \thanks{}
\author{Songyu Ye} \email{sye1@berkeley.edu}

%%% In the address, show linebreaks with double backslashes:
\address{}

%%% Current address is optional.
% \curraddr{}

%%% Email address is optional.
% \email{}


%%% If there's a second author:
% \author{}
% \address{}
% \curraddr{}
% \email{}


%%% To have the current date inserted, use \date{\today}:
\date{\today}

%%% To include an abstract, uncomment the following two lines and type
%%% the abstract in between them:
\begin{abstract}
Abstract.
\end{abstract}


\maketitle
\tableofcontents

%%% To include a table of contents, uncomment the following line:
% \tableofcontents


%%%-------------------------------------------------------------------
%%%-------------------------------------------------------------------
%%% Start the body of the paper here!  E.G., maybe use:
%%% \section{Introduction}
%%% \label{sec:intro}

%%% For a numbered display, use
%%% \begin{equation}
%%%   \label{something}
%%%   The display goes here
%%% \end{equation}
%%% and you can refer to it as \eqref{something}.

%%% For an unnumbered display, use
% \begin{equation*}

% \end{equation*}

%%% To import a graphics file, you must have said
%%% \usepackage{graphicx}
%%% in the preamble (i.e., before the \begin{document}).
%%% Putting it into a figure environment enables it to float to the
%%% next page if there isn't enough room for it on the current page.
%%% The \label command must come after the \caption command.
% \begin{figure}[h]
%   \includegraphics{filename}
%   \caption{Some caption}
%   \label{somelabel}
% \end{figure}

\section{Introduction}
% \input{} adds the contents of the file in the curly braces to the
% current file.  It's a good way to keep the file organized.
\bibliography{refs}{}
% need refs.bib file
\bibliographystyle{plain}

\end{document}