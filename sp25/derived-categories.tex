\documentclass[12pt]{article}
\usepackage[english]{babel}
\usepackage[utf8x]{inputenc}
\usepackage[T1]{fontenc}
\usepackage{listings}
\usepackage{bookmark}
\usepackage{tikz}
\usepackage{/Users/songye03/Desktop/math_tex/style/quiver}
\usepackage{/Users/songye03/Desktop/math_tex/style/scribe}
\usepackage{fancyhdr}

\usepackage{parskip} % Automatically respects blank lines
\setlength{\parskip}{1em} % Adds more space between paragraphs
\setlength{\parindent}{0pt} % Removes paragraph indentation


\begin{document}

\lhead{Songyu Ye}
\rhead{\today}
\cfoot{\thepage}

\title{Coherent sheaves and exceptional collections}

\author{Songyu Ye}
\date{\today}
\maketitle

\begin{abstract}
    Coherent sheaves, vector bundles, and exceptional collections in algebraic geometry.
\end{abstract}

\tableofcontents



\section{Preliminaries}
\subsection{Schemes}
\begin{definition}[Closed and Non-closed Points]
    Let $X = \operatorname{Spec}(A)$ be an affine scheme.
    \begin{enumerate}
        \item A point $p \in X$ is called a \textit{closed point} if the corresponding prime ideal $\mathfrak{p}$ is a maximal ideal of $A$.
        \item A point $p \in X$ is called a \textit{non-closed point} if the corresponding prime ideal $\mathfrak{p}$ is not maximal.
        \item A \textit{generic point} of an irreducible component of $X$ corresponds to a minimal prime ideal of $A$.
    \end{enumerate}
\end{definition}

\begin{example}
    Consider $X = \operatorname{Spec}(\mathbb{C}[x,y])$, the affine plane over $\mathbb{C}$.
    \begin{enumerate}
        \item Closed points correspond to maximal ideals of the form $(x-a, y-b)$ for $a, b \in \mathbb{C}$. These are the familiar points $(a,b)$ in the complex plane.
        \item Prime ideals like $(x-1)$ correspond to non-closed points. Geometrically, this represents the "generic point" of the vertical line $x=1$.
        \item The prime ideal $(0)$ corresponds to the generic point of the entire plane.
    \end{enumerate}
\end{example}

\begin{remark}
    For a scheme over a field $k$:
    \begin{enumerate}
        \item If $k$ is algebraically closed (like $\mathbb{C}$), the closed points of $\operatorname{Spec}(k[x_1,\ldots,x_n])$ correspond exactly to the $n$-tuples $(a_1,\ldots,a_n) \in k^n$.
        \item If $k$ is not algebraically closed (like $\mathbb{Q}$), there are additional closed points. For example, in $\operatorname{Spec}(\mathbb{Q}[x])$, the ideal $(x^2 + 1)$ is maximal and corresponds to a closed point, even though it does not correspond to a rational value of $x$.
    \end{enumerate}
\end{remark}

\begin{proposition}
    Let $X$ be a scheme of finite type over a field $k$. Then:
    \begin{enumerate}
        \item The closed points of $X$ are dense in $X$ (Zariski topology).
        \item If $X$ is irreducible, it has a unique generic point.
        \item The closure of any point $p \in X$ consists of $p$ and all the specializations of $p$.
    \end{enumerate}
\end{proposition}

\begin{definition}[Stalk of the Structure Sheaf]
    Let $X$ be a scheme and $p \in X$ a point. The \textit{stalk} of the structure sheaf $\mathcal{O}_X$ at $p$, denoted $\mathcal{O}_{X,p}$, is defined as the direct limit:
    \[
        \mathcal{O}_{X,p} = \varinjlim_{U \ni p} \mathcal{O}_X(U)
    \]
    where the limit is taken over all open sets $U$ containing the point $p$.
\end{definition}

\begin{proposition}
    Let $X = \operatorname{Spec}(A)$ be an affine scheme and $p \in X$ the point corresponding to a prime ideal $\mathfrak{p} \subset A$. Then:
    \[
        \mathcal{O}_{X,p} \cong A_{\mathfrak{p}}
    \]
    where $A_{\mathfrak{p}}$ is the localization of $A$ at the prime ideal $\mathfrak{p}$.
\end{proposition}

\begin{remark}
    The stalk $\mathcal{O}_{X,p}$ is always a local ring. Its unique maximal ideal, denoted $\mathfrak{m}_p$, consists of germs of functions that vanish at $p$.
\end{remark}

\begin{example}
    Let $X = \operatorname{Spec}(\mathbb{C}[x,y])$ and $p$ the origin (corresponding to the maximal ideal $(x,y)$). Then:
    \[
        \mathcal{O}_{X,p} \cong \mathbb{C}[x,y]_{(x,y)}
    \]
    This is the ring of rational functions in $x$ and $y$ that are defined at the origin.
\end{example}

\begin{example}
    Let $X = \operatorname{Spec}(\mathbb{C}[x,y]/(xy))$, a union of two coordinate axes, and $p$ the origin. Then:
    \[
        \mathcal{O}_{X,p} \cong \mathbb{C}[x,y]_{(x,y)}/(xy)
    \]
    This local ring has zero divisors, reflecting the fact that $p$ is a singular point of $X$.
\end{example}

\begin{definition}[Residue Field]
    Let $X$ be a scheme and $p \in X$ a point. The \textit{residue field} at $p$, denoted $\kappa(p)$, is defined as:
    \[
        \kappa(p) = \mathcal{O}_{X,p}/\mathfrak{m}_p
    \]
    where $\mathfrak{m}_p$ is the maximal ideal of the local ring $\mathcal{O}_{X,p}$.
\end{definition}

\begin{proposition}
    Let $X = \operatorname{Spec}(A)$ be an affine scheme and $p \in X$ the point corresponding to a prime ideal $\mathfrak{p} \subset A$. Then:
    \[
        \kappa(p) \cong \operatorname{Frac}(A/\mathfrak{p})
    \]
    the fraction field of the domain $A/\mathfrak{p}$.
\end{proposition}

\begin{remark}
    For a closed point $p$ corresponding to a maximal ideal $\mathfrak{m}$, we have $\kappa(p) \cong A/\mathfrak{m}$, which is already a field.
\end{remark}

\begin{example}
    Let $X = \operatorname{Spec}(\mathbb{C}[x,y])$.
    \begin{enumerate}
        \item For the closed point $p$ corresponding to the maximal ideal $(x-a, y-b)$, the residue field is:
              \[
                  \kappa(p) \cong \mathbb{C}[x,y]/(x-a, y-b) \cong \mathbb{C}
              \]

        \item For the non-closed point $q$ corresponding to the prime ideal $(x-a)$, the residue field is:
              \[
                  \kappa(q) \cong \operatorname{Frac}(\mathbb{C}[x,y]/(x-a)) \cong \mathbb{C}(y)
              \]
              the field of rational functions in one variable.

        \item For the generic point $\eta$ corresponding to the prime ideal $(0)$, the residue field is:
              \[
                  \kappa(\eta) \cong \operatorname{Frac}(\mathbb{C}[x,y]) \cong \mathbb{C}(x,y)
              \]
              the field of rational functions in two variables.
    \end{enumerate}
\end{example}

\begin{example}
    Let $X = \operatorname{Spec}(\mathbb{Q}[x])$.
    \begin{enumerate}
        \item For the closed point $p$ corresponding to the maximal ideal $(x-a)$ where $a \in \mathbb{Q}$, the residue field is:
              \[
                  \kappa(p) \cong \mathbb{Q}[x]/(x-a) \cong \mathbb{Q}
              \]

        \item For the closed point $q$ corresponding to the maximal ideal $(x^2 + 1)$, the residue field is:
              \[
                  \kappa(q) \cong \mathbb{Q}[x]/(x^2 + 1) \cong \mathbb{Q}(i)
              \]
              which is a degree 2 extension of $\mathbb{Q}$.

        \item For the generic point $\eta$ corresponding to the prime ideal $(0)$, the residue field is:
              \[
                  \kappa(\eta) \cong \operatorname{Frac}(\mathbb{Q}[x]) \cong \mathbb{Q}(x)
              \]
    \end{enumerate}
\end{example}


\begin{definition}[Geometric Point]
    A \textit{geometric point} of a scheme $X$ is a morphism $\operatorname{Spec}(K) \to X$, where $K$ is an algebraically closed field.
\end{definition}

\begin{remark}
    A geometric point can be thought of as a scheme-theoretic point together with an embedding of its residue field into an algebraically closed field.
\end{remark}

\begin{proposition}
    Let $X$ be a scheme over a field $k$. If $k$ is algebraically closed, then every closed point of $X$ naturally gives rise to a geometric point. If $k$ is not algebraically closed, this is not generally true.
\end{proposition}

\begin{example}
    For $X = \operatorname{Spec}(\mathbb{Q}[x])$, the closed point corresponding to $(x^2 + 1)$ has residue field $\mathbb{Q}(i)$. This gives two distinct geometric points when we consider embeddings of $\mathbb{Q}(i)$ into $\mathbb{C}$ (corresponding to $i$ and $-i$).
\end{example}

\subsection{Commutative Algebra}

\begin{definition}
    [Support of a module]
    Let $A$ be a ring and $M$ an $A$-module. The \textit{support} of $M$, denoted $\operatorname{Supp}(M)$, is the set of prime ideals
    \begin{align*}
        \operatorname{Supp}(M) & = \{\mathfrak{p} \in \operatorname{Spec}(A) \mid M_{\mathfrak{p}} \neq 0\}
    \end{align*}
\end{definition}

\begin{definition}
    [Annihilator of a module]
    Let $A$ be a ring and $M$ an $A$-module. The \textit{annihilator} of $M$, denoted $\operatorname{Ann}(M)$, is the ideal of elements
    \begin{align*}
        \operatorname{Ann}(M) & = \{a \in A \mid a \cdot m = 0 \text{ for all } m \in M\} 
    \end{align*}
\end{definition}

\begin{proposition}
    Let $A$ be a ring and $M$ an $A$-module. Then \[\operatorname{Supp}(M) = \operatorname{V}(\operatorname{Ann}(M)) = \{\mathfrak{p} \in \operatorname{Spec}(A) \mid \operatorname{Ann}(M) \subset \mathfrak{p}\}\] In particular, the support of $M$ is a closed subset of $\operatorname{Spec}(A)$.

\end{proposition}

\subsection{Sheaves}

\begin{definition}[Quasi-Coherent Sheaf]
    Let $X$ be a scheme. A sheaf $\mathcal{F}$ of $\mathcal{O}_X$-modules is called \textit{quasi-coherent} if for every open subset $U \subset X$, there exists a covering $\{U_i\}$ of $U$ and a family of $\mathcal{O}_{U_i}$-modules $\mathcal{F}_i$ such that for each $i$, there exists an isomorphism $\mathcal{F}|_{U_i} \cong \mathcal{F}_i$.
\end{definition}

\begin{definition}[Coherent Sheaf]
    Let $X$ be a scheme. A quasi-coherent sheaf $\mathcal{F}$ of $\mathcal{O}_X$-modules is called \textit{coherent} if:
    \begin{enumerate}
        \item $\mathcal{F}$ is of finite type, i.e., for every open subset $U \subset X$, there exists a surjection $\mathcal{O}_U^{\oplus n} \to \mathcal{F}|_U \to 0$ for some integer $n$.
        \item For any open set $U \subset X$ and any morphism $\varphi: \mathcal{O}_U^{\oplus n} \to \mathcal{F}|_U$ of $\mathcal{O}_U$-modules, the kernel $\ker \varphi$ is of finite type.
    \end{enumerate}
\end{definition}

\begin{definition}[Support of a Sheaf]
    Let $X$ be a scheme and $\mathcal{F}$ a sheaf of $\mathcal{O}_X$-modules. The \textit{support} of $\mathcal{F}$, denoted $\operatorname{Supp}(\mathcal{F})$, is the set of points $x \in X$ where the stalk $\mathcal{F}_x$ is non-zero:
    \[
        \operatorname{Supp}(\mathcal{F}) = \{x \in X \mid \mathcal{F}_x \neq 0\}
    \]
\end{definition}

\begin{proposition}
    For a coherent sheaf $\mathcal{F}$ on a scheme $X$:
    \begin{enumerate}
        \item $\operatorname{Supp}(\mathcal{F})$ is a closed subset of $X$.
        \item If $X$ is Noetherian, then $\operatorname{Supp}(\mathcal{F})$ equals the set of points where the fiber $\mathcal{F}_x \otimes_{\mathcal{O}_{X,x}} \kappa(x)$ is non-zero.
        \item For an affine scheme $X = \operatorname{Spec}(A)$ and $\mathcal{F} = \widetilde{M}$ corresponding to an $A$-module $M$, the support of $\mathcal{F}$ corresponds to $\{\mathfrak{p} \in \operatorname{Spec}(A) \mid M_{\mathfrak{p}} \neq 0\}$.
    \end{enumerate}
\end{proposition}

\begin{remark}
    On a noetherian scheme, a sheaf of $\mathcal{O}_X$-modules is coherent if and only if it is of finite type.
\end{remark}

\begin{definition}[Vector Bundle]
    A \textit{vector bundle} of rank $r$ on a scheme $X$ is a coherent sheaf $\mathcal{E}$ on $X$ that is locally free of rank $r$, i.e., for every point $x \in X$, there exists an open neighborhood $U$ of $x$ such that $\mathcal{E}|_U \cong \mathcal{O}_U^{\oplus r}$.
\end{definition}

\begin{definition}[Torsion Sheaf]
    A coherent sheaf $\mathcal{F}$ on a scheme $X$ is called a \textit{torsion sheaf} if its support is a proper closed subset of $X$. Equivalently, for any open affine subset $\Spec(A) \subset X$, the corresponding $A$-module $\Gamma(\Spec(A), \mathcal{F})$ is a torsion $A$-module.
\end{definition}


\begin{definition}[Points of a Scheme]
    Let $X = \operatorname{Spec}(A)$ be an affine scheme. The points of $X$ are in one-to-one correspondence with the prime ideals of $A$. Given a prime ideal $\mathfrak{p} \subset A$, we denote the corresponding point by $p_{\mathfrak{p}}$, or simply $p$ when the context is clear.
\end{definition}


\section{Examples of Non-Vector Bundle Coherent Sheaves}
\begin{example}[Skyscraper Sheaf]
    Let $X$ be a scheme and $p \in X$ a point. The skyscraper sheaf $\mathcal{O}_p$ is a coherent sheaf defined as:
    \[
        \mathcal{O}_p(U) = \begin{cases}
            \kappa(p) & \text{if } p \in U     \\
            0         & \text{if } p \not\in U
        \end{cases}
    \]
    The residue field $\kappa(p)$ is a module over several rings. In particular, we can see that it is coherent because it is generated by a single element over the ring at hand.
    \begin{itemize}
        \item It's an $\mathcal{O}_{X,p}$-module via the natural quotient map $\mathcal{O}_{X,p} \to \mathcal{O}_{X,p}/\mathfrak{m}_p$
              \begin{itemize}
                  \item Any function germ in $\mathcal{O}_{X,p}$ acts on elements of $\kappa(p)$
                  \item Elements in $\mathfrak{m}_p$ act by zero
                  \item Elements outside $\mathfrak{m}_p$ act as non-zero scalars
              \end{itemize}

        \item It's an $\mathcal{O}_X(U)$-module for any open set $U$ containing $p$
              \begin{itemize}
                  \item The action is via the composition $\mathcal{O}_X(U) \to \mathcal{O}_{X,p} \to \kappa(p)$
                  \item This allows functions defined on $U$ to act on the residue field
              \end{itemize}

        \item For affine opens $U = \operatorname{Spec}(A)$ containing $p$, it's an $A$-module
              \begin{itemize}
                  \item If $p$ corresponds to the prime ideal $\mathfrak{p} \subset A$
                  \item Then $\kappa(p) \cong A_{\mathfrak{p}}/\mathfrak{p}A_{\mathfrak{p}} \cong \operatorname{Frac}(A/\mathfrak{p})$
                  \item The action is via $A \to A/\mathfrak{p} \to \operatorname{Frac}(A/\mathfrak{p})$
              \end{itemize}
    \end{itemize}

    It is not a vector bundle because:
    \begin{itemize}
        \item It fails to be locally free at all points. It is a torsion sheaf: any function vanishing at $p$ annihilates the entire sheaf.
        \item Its support is just the single point $\{p\}$, whereas vector bundles have support equal to $X$.
    \end{itemize}
\end{example}


\begin{example}[Ideal Sheaf of a Subvariety]
    Let $L \subset \mathbb{P}^n$ be a line with ideal sheaf $\mathcal{I}_L$. This is a coherent sheaf that fails to be a vector bundle because:
    \begin{itemize}
        \item Its rank is not constant: $\operatorname{rank}(\mathcal{I}_L) = 1$ on $\mathbb{P}^n \setminus L$ but $\operatorname{rank}(\mathcal{I}_L) = 0$ along $L$.
        \item The dimension of $(\mathcal{I}_L)_p \otimes \kappa(p)$ varies: it equals 1 for $p \not\in L$ (as the stalk $(\mathcal{I}_L)_p \cong \mathcal{O}_{\mathbb{P}^n,p}$) but equals 0 for $p \in L$ (as all functions in the ideal vanish at points on $L$).
    \end{itemize}
    The exact sequence $0 \to \mathcal{I}_L \to \mathcal{O}_{\mathbb{P}^n} \to \mathcal{O}_L \to 0$ illustrates this behavior.
\end{example}

\begin{example}[Tangent Sheaf of a Singular Variety]
    For a singular variety $X$, the tangent sheaf $\mathcal{T}_X$ is coherent but not a vector bundle because:
    \begin{itemize}
        \item At smooth points $x \in X$, the sheaf is locally free of rank $\dim X$.
        \item At singular points, the stalk $(\mathcal{T}_X)_x$ fails to be a free $\mathcal{O}_{X,x}$-module.
        \item For example, on a nodal curve, the tangent sheaf at the node has torsion.
    \end{itemize}
\end{example}

\begin{example}[Structure Sheaf of a Singular Variety]
    Let $X$ be a singular variety with structure sheaf $\mathcal{O}_X$. Though $\mathcal{O}_X$ is always coherent, it fails to be locally free at singular points:
    \begin{itemize}
        \item At a singular point $p \in X$, the stalk $\mathcal{O}_{X,p}$ is not a regular local ring.
        \item For instance, if $X = \{xy = 0\} \subset \mathbb{A}^2$, then at the origin, $\mathcal{O}_{X,(0,0)} \cong k[x,y]/(xy)$, which is not a free module over itself.
    \end{itemize}
\end{example}

\section{Exceptional Collections in Derived Categories}

\begin{definition}[Exceptional Object]
    An object $E$ in a derived category $D^b(X)$ is called \textit{exceptional} if:
    \begin{enumerate}
        \item $\operatorname{Hom}(E, E) \cong k$ (the base field)
        \item $\operatorname{Hom}(E, E[n]) = 0$ for all $n \neq 0$
    \end{enumerate}
\end{definition}

\begin{definition}[Exceptional Collection]
    An \textit{exceptional collection} in $D^b(X)$ is an ordered sequence of exceptional objects $\{E_1, E_2, \ldots, E_n\}$ such that:
    \[
        \operatorname{Hom}(E_j, E_i[m]) = 0 \quad \text{for all } j > i \text{ and all } m \in \mathbb{Z}
    \]
\end{definition}

\begin{definition}[Full Exceptional Collection]
    An exceptional collection $\{E_1, E_2, \ldots, E_n\}$ in $D^b(X)$ is called \textit{full} if the objects generate the derived category. Formally, this means the smallest triangulated subcategory of $D^b(X)$ containing the collection and closed under direct sums and direct summands is $D^b(X)$ itself.

    Equivalently, for any object $Y \in D^b(X)$, if $\operatorname{Hom}(E_i[m], Y) = 0$ for all $i=1,2,\ldots,n$ and all $m \in \mathbb{Z}$, then $Y \cong 0$.
\end{definition}

\begin{definition}[Strong Exceptional Collection]
    An exceptional collection $\{E_1, E_2, \ldots, E_n\}$ is called \textit{strong} if:
    \[
        \operatorname{Hom}(E_i, E_j[m]) = 0 \quad \text{for all } i, j \text{ and all } m \neq 0
    \]
\end{definition}

\section{Semiorthogonal Decompositions}

\begin{definition}[Semiorthogonal Decomposition]
    A \textit{semiorthogonal decomposition} of a triangulated category $\mathcal{T}$ is a sequence of full triangulated subcategories $\mathcal{A}_1, \mathcal{A}_2, \ldots, \mathcal{A}_n$ such that:
    \begin{enumerate}
        \item For any objects $A_i \in \mathcal{A}_i$ and $A_j \in \mathcal{A}_j$ with $i > j$, we have $\operatorname{Hom}(A_i, A_j) = 0$.
        \item For any object $T \in \mathcal{T}$, there exists a unique sequence of morphisms:
              \[
                  0 = T_n \to T_{n-1} \to \cdots \to T_1 \to T_0 = T
              \]
              such that the cone of each morphism $T_i \to T_{i-1}$ lies in $\mathcal{A}_i$ for $i=1,2,\ldots,n$.
    \end{enumerate}
    We denote this by $\mathcal{T} = \langle \mathcal{A}_1, \mathcal{A}_2, \ldots, \mathcal{A}_n \rangle$.
\end{definition}

\begin{proposition}
    A full exceptional collection $\{E_1, E_2, \ldots, E_n\}$ in $D^b(X)$ gives rise to a semiorthogonal decomposition:
    \[
        D^b(X) = \langle \mathcal{A}_1, \mathcal{A}_2, \ldots, \mathcal{A}_n \rangle
    \]
    where $\mathcal{A}_i$ is the triangulated subcategory generated by $E_i$.
\end{proposition}

\section{The Splitting Problem and Beilinson's Exceptional Collection}

\subsection{The Splitting Problem}

The splitting problem in algebraic geometry asks: When is a vector bundle on a variety isomorphic to a direct sum of line bundles?

\begin{theorem}[Grothendieck]
    Every vector bundle on $\mathbb{P}^1$ splits as a direct sum of line bundles:
    \[
        \mathcal{E} \cong \bigoplus_{i=1}^r \mathcal{O}(a_i)
    \]
    for some integers $a_1, a_2, \ldots, a_r$.
\end{theorem}

However, for projective spaces of higher dimension, the situation is different:

\begin{theorem}
    For $n \geq 2$, there exist vector bundles on $\mathbb{P}^n$ that do not split as direct sums of line bundles.
\end{theorem}

\begin{example}
    The tangent bundle $T\mathbb{P}^n$ is a non-split vector bundle on $\mathbb{P}^n$ for $n \geq 2$. It fits into the Euler sequence:
    \[
        0 \to \mathcal{O} \to \mathcal{O}(1)^{n+1} \to T\mathbb{P}^n \to 0
    \]
    This sequence is non-split, as it represents a non-zero element in the group $\operatorname{Ext}^1(T\mathbb{P}^n, \mathcal{O})$.
\end{example}

\subsection{Beilinson's Exceptional Collection}

\begin{theorem}[Beilinson]
    The collection $\{\mathcal{O}, \mathcal{O}(1), \mathcal{O}(2), \ldots, \mathcal{O}(n)\}$ is a full exceptional collection in $D^b(\mathbb{P}^n)$.
\end{theorem}


This has several important consequences:
\begin{enumerate}
    \item Every coherent sheaf (or complex of coherent sheaves) on $\mathbb{P}^n$ can be reconstructed from its "cohomological information" with respect to this collection.

    \item The Grothendieck group $K_0(\mathbb{P}^n)$ is a free abelian group of rank $n+1$ with basis given by the classes $[\mathcal{O}], [\mathcal{O}(1)], \ldots, [\mathcal{O}(n)]$. This means that for any coherent sheaf $\mathcal{F}$ on $\mathbb{P}^n$, its class in $K_0(\mathbb{P}^n)$ can be written uniquely as a integer linear combination of these classes:
          \[
              [\mathcal{F}] = a_0[\mathcal{O}] + a_1[\mathcal{O}(1)] + \cdots + a_n[\mathcal{O}(n)]
          \]
\end{enumerate}


\begin{remark}
    The fact that any coherent sheaf can be written as a linear combination of $[\mathcal{O}(i)]$ in $K_0(\mathbb{P}^n)$ does \textit{not} imply that every vector bundle on $\mathbb{P}^n$ splits as a direct sum of line bundles.


    In particular the tangent bundle $T\mathbb{P}^n$ has class:
    \[
        [T\mathbb{P}^n] = (n+1)[\mathcal{O}(1)] - [\mathcal{O}]
    \]
    in $K_0(\mathbb{P}^n)$, but this does not mean $T\mathbb{P}^n \cong \mathcal{O}(1)^{\oplus (n+1)} \oplus \mathcal{O}^{\oplus (-1)}$, which is not even meaningful for a negative exponent.

    The obstruction to a vector bundle splitting is measured by extension groups $\operatorname{Ext}^1$, which precisely capture the non-splitting of exact sequences.
\end{remark}

\end{document}