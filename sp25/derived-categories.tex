\documentclass[12pt]{article}
\usepackage[english]{babel}
\usepackage[utf8x]{inputenc}
\usepackage[T1]{fontenc}
\usepackage{listings}
\usepackage{bookmark}
\usepackage{tikz}
\usepackage{/Users/songye03/Desktop/math_tex/style/quiver}
\usepackage{/Users/songye03/Desktop/math_tex/style/scribe}
\usepackage{fancyhdr}

\usepackage{parskip} % Automatically respects blank lines
\setlength{\parskip}{1em} % Adds more space between paragraphs
\setlength{\parindent}{0pt} % Removes paragraph indentation
\DeclareMathOperator{\Ass}{Ass}

\begin{document}

\lhead{Songyu Ye}
\rhead{\today}
\cfoot{\thepage}

\title{Coherent sheaves and exceptional collections}
\author{Songyu Ye}
\date{\today}

\maketitle

\begin{abstract}
    Coherent sheaves, vector bundles, and exceptional collections in algebraic geometry.
\end{abstract}

\tableofcontents



\section{Preliminaries}
\subsection{Schemes}
\begin{definition}[Closed and Non-closed Points]
    Let $X = \operatorname{Spec}(A)$ be an affine scheme.
    \begin{enumerate}
        \item A point $p \in X$ is called a \textit{closed point} if the corresponding prime ideal $\mathfrak{p}$ is a maximal ideal of $A$.
        \item A point $p \in X$ is called a \textit{non-closed point} if the corresponding prime ideal $\mathfrak{p}$ is not maximal.
        \item A \textit{generic point} of an irreducible component of $X$ corresponds to a minimal prime ideal of $A$.
    \end{enumerate}
\end{definition}

\begin{example}
    Consider $X = \operatorname{Spec}(\mathbb{C}[x,y])$, the affine plane over $\mathbb{C}$.
    \begin{enumerate}
        \item Closed points correspond to maximal ideals of the form $(x-a, y-b)$ for $a, b \in \mathbb{C}$. These are the familiar points $(a,b)$ in the complex plane.
        \item Prime ideals like $(x-1)$ correspond to non-closed points. Geometrically, this represents the "generic point" of the vertical line $x=1$.
        \item The prime ideal $(0)$ corresponds to the generic point of the entire plane.
    \end{enumerate}
\end{example}

\begin{remark}
    For a scheme over a field $k$:
    \begin{enumerate}
        \item If $k$ is algebraically closed (like $\mathbb{C}$), the closed points of $\operatorname{Spec}(k[x_1,\ldots,x_n])$ correspond exactly to the $n$-tuples $(a_1,\ldots,a_n) \in k^n$.
        \item If $k$ is not algebraically closed (like $\mathbb{Q}$), there are additional closed points. For example, in $\operatorname{Spec}(\mathbb{Q}[x])$, the ideal $(x^2 + 1)$ is maximal and corresponds to a closed point, even though it does not correspond to a rational value of $x$.
    \end{enumerate}
\end{remark}

\begin{proposition}
    Let $X$ be a scheme of finite type over a field $k$. Then:
    \begin{enumerate}
        \item The closed points of $X$ are dense in $X$ (Zariski topology).
        \item If $X$ is irreducible, it has a unique generic point.
        \item The closure of any point $p \in X$ consists of $p$ and all the specializations of $p$.
    \end{enumerate}
\end{proposition}

\begin{definition}[Stalk of the Structure Sheaf]
    Let $X$ be a scheme and $p \in X$ a point. The \textit{stalk} of the structure sheaf $\mathcal{O}_X$ at $p$, denoted $\mathcal{O}_{X,p}$, is defined as the direct limit:
    \[
        \mathcal{O}_{X,p} = \varinjlim_{U \ni p} \mathcal{O}_X(U)
    \]
    where the limit is taken over all open sets $U$ containing the point $p$.
\end{definition}

\begin{proposition}
    Let $X = \operatorname{Spec}(A)$ be an affine scheme and $p \in X$ the point corresponding to a prime ideal $\mathfrak{p} \subset A$. Then:
    \[
        \mathcal{O}_{X,p} \cong A_{\mathfrak{p}}
    \]
    where $A_{\mathfrak{p}}$ is the localization of $A$ at the prime ideal $\mathfrak{p}$.
\end{proposition}

\begin{remark}
    The stalk $\mathcal{O}_{X,p}$ is always a local ring. Its unique maximal ideal, denoted $\mathfrak{m}_p$, consists of germs of functions that vanish at $p$.
\end{remark}

\begin{example}
    Let $X = \operatorname{Spec}(\mathbb{C}[x,y])$ and $p$ the origin (corresponding to the maximal ideal $(x,y)$). Then:
    \[
        \mathcal{O}_{X,p} \cong \mathbb{C}[x,y]_{(x,y)}
    \]
    This is the ring of rational functions in $x$ and $y$ that are defined at the origin.
\end{example}

\begin{example}
    Let $X = \operatorname{Spec}(\mathbb{C}[x,y]/(xy))$, a union of two coordinate axes, and $p$ the origin. Then:
    \[
        \mathcal{O}_{X,p} \cong \mathbb{C}[x,y]_{(x,y)}/(xy)
    \]
    This local ring has zero divisors, reflecting the fact that $p$ is a singular point of $X$.
\end{example}

\begin{definition}[Residue Field]
    Let $X$ be a scheme and $p \in X$ a point. The \textit{residue field} at $p$, denoted $\kappa(p)$, is defined as:
    \[
        \kappa(p) = \mathcal{O}_{X,p}/\mathfrak{m}_p
    \]
    where $\mathfrak{m}_p$ is the maximal ideal of the local ring $\mathcal{O}_{X,p}$.
\end{definition}

\begin{proposition}
    Let $X = \operatorname{Spec}(A)$ be an affine scheme and $p \in X$ the point corresponding to a prime ideal $\mathfrak{p} \subset A$. Then:
    \[
        \kappa(p) \cong \operatorname{Frac}(A/\mathfrak{p})
    \]
    the fraction field of the domain $A/\mathfrak{p}$.
\end{proposition}

\begin{remark}
    For a closed point $p$ corresponding to a maximal ideal $\mathfrak{m}$, we have $\kappa(p) \cong A/\mathfrak{m}$, which is already a field.
\end{remark}

\begin{example}
    Let $X = \operatorname{Spec}(\mathbb{C}[x,y])$.
    \begin{enumerate}
        \item For the closed point $p$ corresponding to the maximal ideal $(x-a, y-b)$, the residue field is:
              \[
                  \kappa(p) \cong \mathbb{C}[x,y]/(x-a, y-b) \cong \mathbb{C}
              \]

        \item For the non-closed point $q$ corresponding to the prime ideal $(x-a)$, the residue field is:
              \[
                  \kappa(q) \cong \operatorname{Frac}(\mathbb{C}[x,y]/(x-a)) \cong \mathbb{C}(y)
              \]
              the field of rational functions in one variable.

        \item For the generic point $\eta$ corresponding to the prime ideal $(0)$, the residue field is:
              \[
                  \kappa(\eta) \cong \operatorname{Frac}(\mathbb{C}[x,y]) \cong \mathbb{C}(x,y)
              \]
              the field of rational functions in two variables.
    \end{enumerate}
\end{example}

\begin{example}
    Let $X = \operatorname{Spec}(\mathbb{Q}[x])$.
    \begin{enumerate}
        \item For the closed point $p$ corresponding to the maximal ideal $(x-a)$ where $a \in \mathbb{Q}$, the residue field is:
              \[
                  \kappa(p) \cong \mathbb{Q}[x]/(x-a) \cong \mathbb{Q}
              \]

        \item For the closed point $q$ corresponding to the maximal ideal $(x^2 + 1)$, the residue field is:
              \[
                  \kappa(q) \cong \mathbb{Q}[x]/(x^2 + 1) \cong \mathbb{Q}(i)
              \]
              which is a degree 2 extension of $\mathbb{Q}$.

        \item For the generic point $\eta$ corresponding to the prime ideal $(0)$, the residue field is:
              \[
                  \kappa(\eta) \cong \operatorname{Frac}(\mathbb{Q}[x]) \cong \mathbb{Q}(x)
              \]
    \end{enumerate}
\end{example}


\begin{definition}[Geometric Point]
    A \textit{geometric point} of a scheme $X$ is a morphism $\operatorname{Spec}(K) \to X$, where $K$ is an algebraically closed field.
\end{definition}

\begin{remark}
    A geometric point can be thought of as a scheme-theoretic point together with an embedding of its residue field into an algebraically closed field.
\end{remark}

\begin{proposition}
    Let $X$ be a scheme over a field $k$. If $k$ is algebraically closed, then every closed point of $X$ naturally gives rise to a geometric point. If $k$ is not algebraically closed, this is not generally true.
\end{proposition}

\begin{example}
    For $X = \operatorname{Spec}(\mathbb{Q}[x])$, the closed point corresponding to $(x^2 + 1)$ has residue field $\mathbb{Q}(i)$. This gives two distinct geometric points when we consider embeddings of $\mathbb{Q}(i)$ into $\mathbb{C}$ (corresponding to $i$ and $-i$).
\end{example}

\subsection{Commutative Algebra}
\subsubsection{Associated Primes}
\begin{definition}
    [Support of a module]
    Let $A$ be a ring and $M$ an $A$-module. The \textit{support} of $M$, denoted $\operatorname{Supp}(M)$, is the set of prime ideals
    \begin{align*}
        \operatorname{Supp}(M) & = \{\mathfrak{p} \in \operatorname{Spec}(A) \mid M_{\mathfrak{p}} \neq 0\}
    \end{align*}
\end{definition}

\begin{definition}
    [Annihilator of a module]
    Let $A$ be a ring and $M$ an $A$-module. The \textit{annihilator} of $M$, denoted $\operatorname{Ann}(M)$, is the ideal of elements
    \begin{align*}
        \operatorname{Ann}(M) & = \{a \in A \mid a \cdot m = 0 \text{ for all } m \in M\}
    \end{align*}
\end{definition}

\begin{proposition}
    Let $A$ be a ring and $M$ an $A$-module. Then \[\operatorname{Supp}(M) = \operatorname{V}(\operatorname{Ann}(M)) = \{\mathfrak{p} \in \operatorname{Spec}(A) \mid \operatorname{Ann}(M) \subset \mathfrak{p}\}\] In particular, the support of $M$ is a closed subset of $\operatorname{Spec}(A)$.
\end{proposition}

\begin{proof}
    First we show that if $\Ann(M) \subset \mathfrak{p}$, then $M_{\mathfrak{p}} \neq 0$. Annihilators behave well with respect to localization. \begin{align*}
        \Ann_{A_{\mathfrak{p}}}(M_{\mathfrak{p}}) & = \{a/s \in A_{\mathfrak{p}} \mid a/s \cdot m/s = 0 \text{ for all } m/s \in M_{\mathfrak{p}}\} \\
                                                  & = \{a/s \in A_{\mathfrak{p}} \mid a \cdot m = 0 \text{ for all } m \in M\}                      \\
                                                  & = \Ann(M)A_{\mathfrak{p}} \subset \mathfrak{p}A_{\mathfrak{p}}
    \end{align*} In particular, $\Ann_{A_{\mathfrak{p}}}(M_{\mathfrak{p}})$ is proper. If $M_{\mathfrak{p}} = 0$, then $\Ann_{A_{\mathfrak{p}}}(M_{\mathfrak{p}}) = A_{\mathfrak{p}}$ would be the whole ring.

    Conversely, if $\Ann_A(M) \not\subset \mathfrak{p}$, then $M_{\mathfrak{p}} = 0$. There exists some $a \in \Ann_A(M)$ such that $a \not\in \mathfrak{p}$. When we localize at $\mathfrak{p}$, $a$ becomes invertible since $a \not\in \mathfrak{p}$. Consider any element $m/s\in M_{\mathfrak{p}}$. We have $a \cdot m = 0$ in $M$, so $a \cdot m/s = 0$ in $M_{\mathfrak{p}}$. But $a$ is invertible, so $m/s = 0$ in $M_{\mathfrak{p}}$. Therefore $M_{\mathfrak{p}} = 0$.
\end{proof}

\begin{remark}
    There is a fundamental dictionary between commutative algebra and algebraic geometry:
    \begin{align*}
        \mathfrak{p} \in \Supp(M) \iff M_{\mathfrak{p}} \neq 0 \iff \Ann(M) \subset \mathfrak{p} \iff \mathfrak{p} \in \operatorname{V}(\Ann(M))
    \end{align*}
\end{remark}

\begin{definition}
    [Associated Primes] Let $A$ be a ring and $M$ an $A$-module. An associated prime of $M$ is a prime ideal of the form $\operatorname{Ann}(m)$ for some $m \in M$.
\end{definition}


\begin{theorem}[Associated Primes of a Module]\label{thm:associated_primes}
    If $I$ is an ideal in a Noetherian ring $R$ and $I$ is contained in all associated primes of a finitely generated module $M$, then $I^n M = 0$ for some positive integer $n$.
\end{theorem}

\begin{proof}
    \begin{enumerate}
        \item The associated primes of $M$ are finite in number: $\operatorname{Ass}(M) = \{{\mathfrak p}_1, {\mathfrak p}_2, \ldots {\mathfrak p}_r\}$

        \item The radical of the annihilator equals the intersection of all associated primes:
              \[\sqrt{\operatorname{Ann}(M)} = {\mathfrak p}_1 \cap {\mathfrak p}_2 \cap \ldots \cap {\mathfrak p}_r\]
    \end{enumerate}
    If $I \subseteq {\mathfrak p}_i$ for all $i$, then $I \subseteq \sqrt{\operatorname{Ann}(M)}$. This means every element $f \in I$ satisfies $f^{n_f} \in \operatorname{Ann}(M)$ for some $n_f$. Since $I$ is finitely generated, we can find a uniform $N$ such that $I^N \subseteq \operatorname{Ann}(M)$. Therefore, $I^N M = 0$.
\end{proof}

To prove the first two statements, we need to turn to the theory of primary decompositions.
\subsubsection{Primary decompositions}
Primary decomposition generalizes the unique factorization of integers into prime powers. In a ring, it allows us to express ideals as intersections of simpler ideals called primary ideals.

\begin{definition}[Prime, Primary Ideal]
    An ideal $\mathfrak{p}$ in a ring $R$ is prime if:
    \begin{enumerate}[label=(\roman*)]
        \item $\mathfrak{p} \neq R$
        \item For any $a, b \in R$, if $ab \in \mathfrak{p}$, then either $a \in \mathfrak{p}$ or $b \in \mathfrak{p}$
    \end{enumerate}


    An ideal $\mathfrak{q}$ in a ring $R$ is primary if:
    \begin{enumerate}[label=(\roman*)]
        \item $\mathfrak{q} \neq R$
        \item For any $a, b \in R$, if $ab \in \mathfrak{q}$ and $a \not\in \mathfrak{q}$, then $b^n \in \mathfrak{q}$ for some positive integer $n$
    \end{enumerate}
\end{definition}

Every primary ideal $\mathfrak{q}$ has an associated prime ideal $\mathfrak{p} = \sqrt{\mathfrak{q}}$ (the radical of $\mathfrak{q}$). We say $\mathfrak{q}$ is $\mathfrak{p}$-primary.

\begin{theorem}[Primary Decomposition Theorem]
    In a Noetherian ring $R$, every ideal $I$ can be expressed as a finite intersection of primary ideals:
    \[
        I = \mathfrak{q}_1 \cap \mathfrak{q}_2 \cap \cdots \cap \mathfrak{q}_n
    \]
    where each $\mathfrak{q}_i$ is a primary ideal with associated prime $\mathfrak{p}_i = \sqrt{\mathfrak{q}_i}$.
\end{theorem}

\begin{definition}[Minimal Primary Decomposition]
    A primary decomposition is called minimal if:
    \begin{enumerate}[label=(\roman*)]
        \item The associated primes $\mathfrak{p}_i = \sqrt{\mathfrak{q}_i}$ are all distinct
        \item No $\mathfrak{q}_i$ can be removed from the intersection without changing the ideal $I$
    \end{enumerate}
\end{definition}

\begin{proposition}
    In a minimal primary decomposition, the set of associated primes $\{\mathfrak{p}_1, \mathfrak{p}_2, \ldots, \mathfrak{p}_n\}$ is uniquely determined by the ideal $I$.
\end{proposition}
The theory extends naturally to modules over Noetherian rings.
\begin{theorem}
    For a submodule $N$ of a module $M$ over a Noetherian ring $R$, we can find a primary decomposition:
    \[
        N = N_1 \cap N_2 \cap \cdots \cap N_n
    \]
    where each $N_i$ is a $\mathfrak{p}_i$-primary submodule of $M$. This means $M/N_i$ has the property that for each $r \in R$ and $m + N_i \in M/N_i$, if $r(m + N_i) = 0 + N_i$ and $r \not\in \mathfrak{p}_i$, then $(m + N_i)^n = 0 + N_i$ for some $n > 0$.
\end{theorem}

When considering the zero submodule $(0)$ in $M$, and its minimal primary decomposition \[(0) = N_1 \cap N_2 \cap \cdots \cap N_n\] the associated primes of this decomposition are exactly the associated primes of $M$.

\begin{theorem}
    In a Noetherian ring, the set of associated primes of a finitely generated module $M$ is finite.
\end{theorem}

\begin{proof}
    The finiteness follows directly from:
    \begin{enumerate}[label=(\arabic*)]
        \item The Noetherian property ensures every ideal has a finite primary decomposition
        \item The associated primes of a module $M$ correspond to the primes in the primary decomposition of the zero submodule $(0)$ in $M$
        \item Therefore, $\Ass(M) = \{\mathfrak{p}_1, \mathfrak{p}_2, \ldots, \mathfrak{p}_r\}$ is finite
    \end{enumerate}
\end{proof}

\begin{theorem}
    For a finitely generated module $M$ over a Noetherian ring $R$ with associated primes $\Ass(M) = \{\mathfrak{p}_1, \mathfrak{p}_2, \ldots, \mathfrak{p}_r\}$, the following equality holds:
    \[
        \sqrt{\Ann(M)} = \mathfrak{p}_1 \cap \mathfrak{p}_2 \cap \cdots \cap \mathfrak{p}_r
    \]
\end{theorem}

\begin{proof}
    This equality follows from:
    \begin{enumerate}[label=(\arabic*)]
        \item The primary decomposition of $(0)$ in $M$ as $(0) = N_1 \cap N_2 \cap \cdots \cap N_r$ where each $N_i$ is $\mathfrak{p}_i$-primary

        \item An element $a \in R$ annihilates $M$ completely if and only if $aM \subseteq N_1 \cap N_2 \cap \cdots \cap N_r = (0)$

        \item If $a \in \mathfrak{p}_1 \cap \mathfrak{p}_2 \cap \cdots \cap \mathfrak{p}_r$, then some power of $a$ lies in each $\mathfrak{q}_i$ (the primary ideals associated with the $N_i$), which means some power of $a$ annihilates $M$, so $a \in \sqrt{\Ann(M)}$

        \item Conversely, if $a \in \sqrt{\Ann(M)}$, then some power of $a$ annihilates $M$, which means $a$ belongs to each $\mathfrak{p}_i$
    \end{enumerate}

    This establishes the equality $\sqrt{\Ann(M)} = \mathfrak{p}_1 \cap \mathfrak{p}_2 \cap \cdots \cap \mathfrak{p}_r$.
\end{proof}


\subsection{Sheaves}

\begin{definition}[Quasi-Coherent Sheaf]
    Let $X$ be a scheme. A sheaf $\mathcal{F}$ of $\mathcal{O}_X$-modules is called \textit{quasi-coherent} if for every open subset $U \subset X$, there exists a covering $\{U_i\}$ of $U$ and a family of $\mathcal{O}_{U_i}$-modules $\mathcal{F}_i$ such that for each $i$, there exists an isomorphism $\mathcal{F}|_{U_i} \cong \mathcal{F}_i$.
\end{definition}

\begin{definition}[Coherent Sheaf]
    Let $X$ be a scheme. A quasi-coherent sheaf $\mathcal{F}$ of $\mathcal{O}_X$-modules is called \textit{coherent} if:
    \begin{enumerate}
        \item $\mathcal{F}$ is of finite type, i.e., for every open subset $U \subset X$, there exists a surjection $\mathcal{O}_U^{\oplus n} \to \mathcal{F}|_U \to 0$ for some integer $n$.
        \item For any open set $U \subset X$ and any morphism $\varphi: \mathcal{O}_U^{\oplus n} \to \mathcal{F}|_U$ of $\mathcal{O}_U$-modules, the kernel $\ker \varphi$ is of finite type.
    \end{enumerate}
\end{definition}

\begin{definition}[Support of a Sheaf]
    Let $X$ be a scheme and $\mathcal{F}$ a sheaf of $\mathcal{O}_X$-modules. The \textit{support} of $\mathcal{F}$, denoted $\operatorname{Supp}(\mathcal{F})$, is the set of points $x \in X$ where the stalk $\mathcal{F}_x$ is non-zero:
    \[
        \operatorname{Supp}(\mathcal{F}) = \{x \in X \mid \mathcal{F}_x \neq 0\}
    \]
\end{definition}

\begin{proposition}
    For a coherent sheaf $\mathcal{F}$ on a scheme $X$:
    \begin{enumerate}
        \item $\operatorname{Supp}(\mathcal{F})$ is a closed subset of $X$.
        \item If $X$ is Noetherian, then $\operatorname{Supp}(\mathcal{F})$ equals the set of points where the fiber $\mathcal{F}_x \otimes_{\mathcal{O}_{X,x}} \kappa(x)$ is non-zero.
        \item For an affine scheme $X = \operatorname{Spec}(A)$ and $\mathcal{F} = \widetilde{M}$ corresponding to an $A$-module $M$, the support of $\mathcal{F}$ corresponds to $\{\mathfrak{p} \in \operatorname{Spec}(A) \mid M_{\mathfrak{p}} \neq 0\}$.
    \end{enumerate}
\end{proposition}

\begin{remark}
    On a noetherian scheme, a sheaf of $\mathcal{O}_X$-modules is coherent if and only if it is of finite type.
\end{remark}

\begin{definition}[Vector Bundle]
    A \textit{vector bundle} of rank $r$ on a scheme $X$ is a coherent sheaf $\mathcal{E}$ on $X$ that is locally free of rank $r$, i.e., for every point $x \in X$, there exists an open neighborhood $U$ of $x$ such that $\mathcal{E}|_U \cong \mathcal{O}_U^{\oplus r}$.
\end{definition}

\begin{definition}[Torsion Sheaf]
    A coherent sheaf $\mathcal{F}$ on a scheme $X$ is called a \textit{torsion sheaf} if for any open affine subset $\Spec(A) \subset X$, the corresponding $A$-module $\Gamma(\Spec(A), \mathcal{F})$ is a torsion $A$-module. If $X$ is locally noetherian, then equivalently $\cF$ is a torsion sheaf if its support is a proper closed subset of $X$.
\end{definition}

\begin{proof}
    Suppose $\mathcal{F}$ is a coherent sheaf on $X$ with support that is a proper closed subset of $X$. Let $\text{Spec}(A) \subset X$ be an affine open subset. We want to show that $\Gamma(\text{Spec}(A), \mathcal{F})$ is a torsion $A$-module. Since $\text{Supp}(\mathcal{F})$ is a proper closed subset of $X$, its intersection with $\text{Spec}(A)$ is a proper closed subset of $\text{Spec}(A)$. This means there exists a non-zero ideal $I \subset A$ such that $V(I)$ contains $\text{Supp}(\mathcal{F}) \cap \text{Spec}(A)$.

    Algebraically, this means $I$ is contained in all prime ideals $P \in \text{Supp}(\mathcal{F}) \cap \text{Spec}(A)$. The associated primes of $\Gamma(\text{Spec}(A), \mathcal{F})$ are contained in $\text{Supp}(\mathcal{F}) \cap \text{Spec}(A)$, so $I$ is contained in all associated primes of $\Gamma(\text{Spec}(A), \mathcal{F})$. Since $A$ is Noetherian ($X$ has to be locally noetherian), and $I$ is contained in all associated primes of the finitely generated $A$-module $\Gamma(\text{Spec}(A), \mathcal{F})$, by Theorem \ref{thm:associated_primes}, there exists some $n > 0$ such that:
    \[I^n \cdot \Gamma(\text{Spec}(A), \mathcal{F}) = 0\]
    This shows that $\Gamma(\text{Spec}(A), \mathcal{F})$ is a torsion $A$-module.


    Conversely, suppose for every affine open $\text{Spec}(A) \subset X$, the $A$-module $\Gamma(\text{Spec}(A), \mathcal{F})$ is a torsion module. For each $m \in \Gamma(\text{Spec}(A), \mathcal{F})$, there exists a non-zero element $a_m \in A$ such that $a_m \cdot m = 0$. Let $I$ be the ideal generated by all such $a_m$ for all $m \in \Gamma(\text{Spec}(A), \mathcal{F})$. By construction, $I$ is non-zero and $I \cdot \Gamma(\text{Spec}(A), \mathcal{F}) = 0$. This means $V(I)$ contains $\text{Supp}(\mathcal{F}) \cap \text{Spec}(A)$. Since $I$ is non-zero, $V(I)$ is a proper closed subset of $\text{Spec}(A)$. Therefore, the support of $\mathcal{F}$ restricted to $\text{Spec}(A)$ is a proper closed subset. Since this holds for all affine opens covering $X$, we conclude that $\text{Supp}(\mathcal{F})$ is a proper closed subset of $X$, making $\mathcal{F}$ a torsion sheaf.
\end{proof}


\begin{definition}[Points of a Scheme]
    Let $X = \operatorname{Spec}(A)$ be an affine scheme. The points of $X$ are in one-to-one correspondence with the prime ideals of $A$. Given a prime ideal $\mathfrak{p} \subset A$, we denote the corresponding point by $p_{\mathfrak{p}}$, or simply $p$ when the context is clear.
\end{definition}

To unpack this example, we recall how to work with the tangent sheaf of an affine variety.
\begin{definition}
Let $X$ be a variety over a field $k$. The \textbf{tangent sheaf} of $X$, denoted $\mathcal{T}_X$, is defined as the dual of the sheaf of differentials:
\begin{equation}
\mathcal{T}_X = \mathcal{H}om_{\mathcal{O}_X}(\Omega^1_X, \mathcal{O}_X)
\end{equation}
\end{definition}

For an affine variety $X = \text{Spec}(R)$ where $R = k[x_1, \ldots, x_n]/I$, the tangent sheaf has a concrete description in terms of derivations.

\begin{proposition}
Let $X = \text{Spec}(R)$ be an affine variety. Then the global sections of the tangent sheaf correspond to the $R$-module of $k$-derivations of $R$:
\begin{equation}
\Gamma(X, \mathcal{T}_X) \cong \text{Der}_k(R, R)
\end{equation}
\end{proposition}

\begin{definition}
A $k$-derivation on a $k$-algebra $R$ is a $k$-linear map $D: R \to R$ satisfying the Leibniz rule:
\begin{equation}
D(fg) = fD(g) + gD(f) \quad \text{for all } f,g \in R
\end{equation}
\end{definition}


\begin{proposition}
The tangent sheaf $\mathcal{T}_X$ of any variety $X$ is a coherent sheaf of $\mathcal{O}_X$-modules.
\end{proposition}

\begin{proposition}
For a smooth variety $X$, the tangent sheaf $\mathcal{T}_X$ is locally free of rank equal to the dimension of $X$.
\end{proposition}

\subsection{Dual Perspective: The Cotangent Sheaf}

For completeness, we briefly discuss the dual perspective in terms of the cotangent sheaf.

\begin{definition}
The \textbf{cotangent sheaf} or \textbf{sheaf of differentials} $\Omega^1_X$ of a variety $X$ over $k$ is the sheaf associated to the presheaf $U \mapsto \Omega^1_{\mathcal{O}_X(U)/k}$, where $\Omega^1_{A/k}$ denotes the module of Kähler differentials of a $k$-algebra $A$.
\end{definition}

\begin{proposition}
For an affine variety $X = \text{Spec}(R)$ where $R = k[x_1, \ldots, x_n]/I$, the module of Kähler differentials can be described as:
\begin{equation}
\Omega^1_{R/k} \cong \frac{R \cdot dx_1 \oplus \cdots \oplus R \cdot dx_n}{\{df \mid f \in I\}}
\end{equation}
\end{proposition}



\section{Examples of Non-Vector Bundle Coherent Sheaves}
\subsection{Skyscraper Sheaf}
\begin{example}[Skyscraper Sheaf]
    Let $X$ be a scheme and $p \in X$ a point. The skyscraper sheaf $\mathcal{O}_p$ is a coherent sheaf defined as:
    \[
        \mathcal{O}_p(U) = \begin{cases}
            \kappa(p) & \text{if } p \in U     \\
            0         & \text{if } p \not\in U
        \end{cases}
    \]
    The residue field $\kappa(p)$ is a module over several rings. In particular, we can see that it is coherent because it is generated by a single element over the ring at hand.
    \begin{itemize}
        \item $\mathcal{O}_{X,p}$-module via the natural quotient map $\mathcal{O}_{X,p} \to \mathcal{O}_{X,p}/\mathfrak{m}_p$
        \item $\mathcal{O}_X(U)$-module for any open set $U$ containing $p$
        \item For affine opens $U = \operatorname{Spec}(A)$ containing $p$, it's an $A$-module
              \begin{itemize}
                  \item If $p$ corresponds to the prime ideal $\mathfrak{p} \subset A$
                  \item Then $\kappa(p) \cong A_{\mathfrak{p}}/\mathfrak{p}A_{\mathfrak{p}} \cong \operatorname{Frac}(A/\mathfrak{p})$
                  \item The action is via $A \to A/\mathfrak{p} \to \operatorname{Frac}(A/\mathfrak{p})$
              \end{itemize}
    \end{itemize}

    It is not a vector bundle because:
    \begin{itemize}
        \item It fails to be locally free at all points. It is a torsion sheaf: any function vanishing at $p$ annihilates the entire sheaf.
        \item Its support is just the single point $\{p\}$, whereas vector bundles have support equal to $X$.
    \end{itemize}
\end{example}



\subsection{Tangent sheaf of nodal curve}

\begin{example}[Tangent Sheaf of a Singular Variety]
    For a singular variety $X$, the tangent sheaf $\mathcal{T}_X$ is coherent but not a vector bundle because:
    \begin{itemize}
        \item At smooth points $x \in X$, the sheaf is locally free of rank $\dim X$.
        \item At singular points, the stalk $(\mathcal{T}_X)_x$ fails to be a free $\mathcal{O}_{X,x}$-module.
        \item For example, on a nodal curve, the tangent sheaf at the node has torsion.
    \end{itemize}
\end{example}
Consider the nodal curve $X = \Spec R = \Spec k[x,y]/(y^2-x^2)$. The curve $X$ has a singularity (node) at the origin, where the two branches (given by $y = x$ and $y = -x$) intersect. We work with derivations on the quotient ring $R = k[x,y]/(y^2-x^2)$.

\begin{lemma}
Any $k$-derivation $D$ on the polynomial ring $k[x,y]$ can be written uniquely as
\begin{equation}
D = a\frac{\partial}{\partial x} + b\frac{\partial}{\partial y}
\end{equation}
for some $a,b \in k[x,y]$, where $\frac{\partial}{\partial x}$ and $\frac{\partial}{\partial y}$ are the standard partial derivatives.
\end{lemma}

\begin{proof}
A derivation $D$ on $k[x,y]$ is completely determined by its values on the generators $x$ and $y$. Setting $a = D(x)$ and $b = D(y)$, we can verify that $D = a\frac{\partial}{\partial x} + b\frac{\partial}{\partial y}$ by the Leibniz rule and linearity of $D$.
\end{proof}

For a derivation on $k[x,y]$ to induce a well-defined derivation on $R = k[x,y]/(y^2-x^2)$, it must preserve the ideal $(y^2-x^2)$. This constrains the possible derivations.

\begin{proposition}
Let $\bar{x}$ and $\bar{y}$ denote the images of $x$ and $y$ in the quotient ring $R = k[x,y]/(y^2-x^2)$. A $k$-derivation $D$ on $R$ is determined by its values $D(\bar{x}) = \bar{a}$ and $D(\bar{y}) = \bar{b}$, where $\bar{a}, \bar{b} \in R$ are subject to the constraint:
\begin{equation}
\bar{y} \cdot \bar{b} = \bar{x} \cdot \bar{a}
\end{equation}
\end{proposition}

\begin{proof}
Since $\bar{y}^2 = \bar{x}^2$ in $R$, applying the derivation $D$ to both sides and using the Leibniz rule, we get:
\begin{align}
D(\bar{y}^2) &= D(\bar{x}^2)\\
2\bar{y} \cdot D(\bar{y}) &= 2\bar{x} \cdot D(\bar{x})\\
2\bar{y} \cdot \bar{b} &= 2\bar{x} \cdot \bar{a}
\end{align}
Simplifying, we obtain $\bar{y} \cdot \bar{b} = \bar{x} \cdot \bar{a}$.
\end{proof}

\begin{theorem}
The tangent sheaf $\mathcal{T}_X$ of the nodal curve $X$ defined by $y^2 = x^2$ can be described as:
\begin{equation}
\mathcal{T}_X \cong \{(\bar{a},\bar{b}) \in R \times R \mid \bar{y} \cdot \bar{b} = \bar{x} \cdot \bar{a}\}
\end{equation}
where each pair $(\bar{a},\bar{b})$ corresponds to the derivation $D = \bar{a}\frac{\partial}{\partial \bar{x}} + \bar{b}\frac{\partial}{\partial \bar{y}}$.
\end{theorem}

\begin{remark}
This description shows that $\mathcal{T}_X$ is a submodule of the free $R$-module $R \times R$, defined by a single linear constraint.
\end{remark}
To show that the tangent sheaf is not locally free at the node, we need to identify torsion elements.

\begin{proposition}
The tangent sheaf $\mathcal{T}_X$ has torsion at the node, proving it is not a free $R$-module.
\end{proposition}

\begin{proof}
Consider the element $v = (\bar{y}+\bar{x}, \bar{y}+\bar{x}) \in \mathcal{T}_X$. This corresponds to the derivation $(\bar{y}+\bar{x})\frac{\partial}{\partial \bar{x}} + (\bar{y}+\bar{x})\frac{\partial}{\partial \bar{y}}$.

First, let's verify that $v \in \mathcal{T}_X$ by checking the constraint:
\begin{align}
\bar{y}(\bar{y}+\bar{x}) &= \bar{x}(\bar{y}+\bar{x})\\
\bar{y}^2 + \bar{x}\bar{y} &= \bar{x}\bar{y} + \bar{x}^2
\end{align}

Using the relation $\bar{y}^2 = \bar{x}^2$, this equation is satisfied.

Now, let's multiply $v$ by the element $f = \bar{y}-\bar{x} \in R$:

\begin{align}
f \cdot v &= (\bar{y}-\bar{x}) \cdot (\bar{y}+\bar{x}, \bar{y}+\bar{x})\\
&= ((\bar{y}-\bar{x})(\bar{y}+\bar{x}), (\bar{y}-\bar{x})(\bar{y}+\bar{x}))\\
&= (\bar{y}^2-\bar{x}^2, \bar{y}^2-\bar{x}^2)
\end{align}

Since $\bar{y}^2 = \bar{x}^2$ in $R$, we have $f \cdot v = (0,0)$. This demonstrates that $v$ is a non-zero element of the tangent sheaf that is annihilated by the non-zero element $f$, confirming that $\mathcal{T}_X$ has torsion at the node.
\end{proof}

\begin{remark}[Geometric Interpretation]
The element $v = (\bar{y}+\bar{x}, \bar{y}+\bar{x})$ corresponds to a derivation that acts as the standard tangent vector along the component $y = x$ but vanishes along the component $y = -x$. The element $f = \bar{y}-\bar{x}$ that annihilates this derivation is precisely the function that defines the component where the derivation doesn't vanish.

This reflects the fact that at the node, we have two distinct tangent directions (corresponding to the two branches of the curve), and a vector field can act along one branch while vanishing along the other. Such vector fields cannot be part of a free basis for the tangent sheaf, as they are annihilated by functions that distinguish between the branches.
\end{remark}

\section{Exceptional Collections in Derived Categories}

\begin{definition}[Exceptional Object]
    An object $E$ in a derived category $D^b(X)$ is called \textit{exceptional} if:
    \begin{enumerate}
        \item $\operatorname{Hom}(E, E) \cong k$ (the base field)
        \item $\operatorname{Hom}(E, E[n]) = 0$ for all $n \neq 0$
    \end{enumerate}
\end{definition}

\begin{definition}[Exceptional Collection]
    An \textit{exceptional collection} in $D^b(X)$ is an ordered sequence of exceptional objects $\{E_1, E_2, \ldots, E_n\}$ such that:
    \[
        \operatorname{Hom}(E_j, E_i[m]) = 0 \quad \text{for all } j > i \text{ and all } m \in \mathbb{Z}
    \]
\end{definition}

\begin{definition}[Full Exceptional Collection]
    An exceptional collection $\{E_1, E_2, \ldots, E_n\}$ in $D^b(X)$ is called \textit{full} if the objects generate the derived category. Formally, this means the smallest triangulated subcategory of $D^b(X)$ containing the collection and closed under direct sums and direct summands is $D^b(X)$ itself.

    Equivalently, for any object $Y \in D^b(X)$, if $\operatorname{Hom}(E_i[m], Y) = 0$ for all $i=1,2,\ldots,n$ and all $m \in \mathbb{Z}$, then $Y \cong 0$.
\end{definition}

\begin{definition}[Strong Exceptional Collection]
    An exceptional collection $\{E_1, E_2, \ldots, E_n\}$ is called \textit{strong} if:
    \[
        \operatorname{Hom}(E_i, E_j[m]) = 0 \quad \text{for all } i, j \text{ and all } m \neq 0
    \]
\end{definition}

\section{Semiorthogonal Decompositions}

\begin{definition}[Semiorthogonal Decomposition]
    A \textit{semiorthogonal decomposition} of a triangulated category $\mathcal{T}$ is a sequence of full triangulated subcategories $\mathcal{A}_1, \mathcal{A}_2, \ldots, \mathcal{A}_n$ such that:
    \begin{enumerate}
        \item For any objects $A_i \in \mathcal{A}_i$ and $A_j \in \mathcal{A}_j$ with $i > j$, we have $\operatorname{Hom}(A_i, A_j) = 0$.
        \item For any object $T \in \mathcal{T}$, there exists a unique sequence of morphisms:
              \[
                  0 = T_n \to T_{n-1} \to \cdots \to T_1 \to T_0 = T
              \]
              such that the cone of each morphism $T_i \to T_{i-1}$ lies in $\mathcal{A}_i$ for $i=1,2,\ldots,n$.
    \end{enumerate}
    We denote this by $\mathcal{T} = \langle \mathcal{A}_1, \mathcal{A}_2, \ldots, \mathcal{A}_n \rangle$.
\end{definition}

\begin{proposition}
    A full exceptional collection $\{E_1, E_2, \ldots, E_n\}$ in $D^b(X)$ gives rise to a semiorthogonal decomposition:
    \[
        D^b(X) = \langle \mathcal{A}_1, \mathcal{A}_2, \ldots, \mathcal{A}_n \rangle
    \]
    where $\mathcal{A}_i$ is the triangulated subcategory generated by $E_i$.
\end{proposition}

\section{The Splitting Problem and Beilinson's Exceptional Collection}

\subsection{The Splitting Problem}

The splitting problem in algebraic geometry asks: When is a vector bundle on a variety isomorphic to a direct sum of line bundles?

\begin{theorem}[Grothendieck]
    Every vector bundle on $\mathbb{P}^1$ splits as a direct sum of line bundles:
    \[
        \mathcal{E} \cong \bigoplus_{i=1}^r \mathcal{O}(a_i)
    \]
    for some integers $a_1, a_2, \ldots, a_r$.
\end{theorem}

However, for projective spaces of higher dimension, the situation is different:

\begin{theorem}
    For $n \geq 2$, there exist vector bundles on $\mathbb{P}^n$ that do not split as direct sums of line bundles.
\end{theorem}

\begin{example}
    The tangent bundle $T\mathbb{P}^n$ is a non-split vector bundle on $\mathbb{P}^n$ for $n \geq 2$. It fits into the Euler sequence:
    \[
        0 \to \mathcal{O} \to \mathcal{O}(1)^{n+1} \to T\mathbb{P}^n \to 0
    \]
    This sequence is non-split, as it represents a non-zero element in the group $\operatorname{Ext}^1(T\mathbb{P}^n, \mathcal{O})$.
\end{example}

\subsection{Beilinson's Exceptional Collection}

\begin{theorem}[Beilinson]
    The collection $\{\mathcal{O}, \mathcal{O}(1), \mathcal{O}(2), \ldots, \mathcal{O}(n)\}$ is a full exceptional collection in $D^b(\mathbb{P}^n)$.
\end{theorem}


This has several important consequences:
\begin{enumerate}
    \item Every coherent sheaf (or complex of coherent sheaves) on $\mathbb{P}^n$ can be reconstructed from its "cohomological information" with respect to this collection.

    \item The Grothendieck group $K_0(\mathbb{P}^n)$ is a free abelian group of rank $n+1$ with basis given by the classes $[\mathcal{O}], [\mathcal{O}(1)], \ldots, [\mathcal{O}(n)]$. This means that for any coherent sheaf $\mathcal{F}$ on $\mathbb{P}^n$, its class in $K_0(\mathbb{P}^n)$ can be written uniquely as a integer linear combination of these classes:
          \[
              [\mathcal{F}] = a_0[\mathcal{O}] + a_1[\mathcal{O}(1)] + \cdots + a_n[\mathcal{O}(n)]
          \]
\end{enumerate}


\begin{remark}
    The fact that any coherent sheaf can be written as a linear combination of $[\mathcal{O}(i)]$ in $K_0(\mathbb{P}^n)$ does \textit{not} imply that every vector bundle on $\mathbb{P}^n$ splits as a direct sum of line bundles.


    In particular the tangent bundle $T\mathbb{P}^n$ has class:
    \[
        [T\mathbb{P}^n] = (n+1)[\mathcal{O}(1)] - [\mathcal{O}]
    \]
    in $K_0(\mathbb{P}^n)$, but this does not mean $T\mathbb{P}^n \cong \mathcal{O}(1)^{\oplus (n+1)} \oplus \mathcal{O}^{\oplus (-1)}$, which is not even meaningful for a negative exponent.

    The obstruction to a vector bundle splitting is measured by extension groups $\operatorname{Ext}^1$, which precisely capture the non-splitting of exact sequences.
\end{remark}

\end{document}