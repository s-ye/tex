\documentclass[12pt]{article}
\usepackage[english]{babel}
\usepackage[utf8x]{inputenc}
\usepackage[T1]{fontenc}
\usepackage{listings}
\usepackage{bookmark}
\usepackage{tikz}
\usepackage{/Users/songye03/Desktop/math_tex/style/quiver}
\usepackage{/Users/songye03/Desktop/math_tex/style/scribe}
\usepackage{fancyhdr}

\usepackage{parskip} % Automatically respects blank lines
\setlength{\parskip}{1em} % Adds more space between paragraphs
\setlength{\parindent}{0pt} % Removes paragraph indentation

\begin{document}


\lhead{Songyu Ye}
\rhead{\today}
\cfoot{\thepage}

\title{Title}

\author{Songyu Ye}
\date{\today}
\maketitle


\begin{abstract}
	Abstract
\end{abstract}

\tableofcontents

\section{Vector bundles and connections}
This section follows definitions from \cite{hotta_takeuchi_tanisaki_d_modules_2008}.
\begin{lemma}
	Let \( M \) be an \( \mathcal{O}_X \)-module. Giving a left \( \mathcal{D}_X \)-module structure on \( M \) extending the \( \mathcal{O}_X \)-module structure is equivalent to giving a \( \mathbb{C} \)-linear morphism
	\[
		\nabla : \Theta_X \rightarrow \mathcal{E}nd_{\mathbb{C}}(M) \quad (\theta \mapsto \nabla_\theta),
	\]
	satisfying the following conditions:
	\begin{enumerate}
		\item
		      $\nabla_{f\theta}(s) = f\nabla_{\theta}(s) \quad (f \in \mathcal{O}_X, \theta \in \Theta_X, s \in M)$

		\item
		      $\nabla_{\theta}(fs) = \theta(f)s + f\nabla_{\theta}(s) \quad (f \in \mathcal{O}_X, \theta \in \Theta_X, s \in M)$
		\item
		      $\nabla_{[\theta_1,\theta_2]}(s) = [\nabla_{\theta_1}, \nabla_{\theta_2}](s) \quad (\theta_1, \theta_2 \in \Theta_X, s \in M)$
	\end{enumerate}
	In terms of \(\nabla\) the left \( \mathcal{D}_X \)-module structure on \( M \) is given by
	\[
		\theta_s = \nabla_{\theta}(s) \quad (\theta \in \Theta, s \in M).
	\]
\end{lemma}


The condition (3) above is called the \emph{integrability condition} on \( M \).

For a locally free left \( \mathcal{O}_X \)-module \( M \) of finite rank, a \( \mathbb{C} \)-linear morphism \(\nabla : \Theta_X \rightarrow \mathcal{E}nd_{\mathbb{C}}(M) \) satisfying the conditions (1), (2) is usually called a \emph{connection} (of the corresponding vector bundle). If it also satisfies the condition (3), it is called an \emph{integrable (or flat) connection}. Hence we may regard a (left) \( \mathcal{D}_X \)-module as an integrable connection of an \( \mathcal{O}_X \)-module which is not necessarily locally free of finite rank.

\begin{definition}
	We say that a \( \mathcal{D}_X \)-module \( M \) is an \emph{integrable connection} if it is locally free of finite rank over \( \mathcal{O}_X \).
\end{definition}

\begin{definition}
	A \emph{local system} on \( X \) is a locally free $\C_X$-module of finite rank.
\end{definition}

\section{Principal bundles and connections}

\begin{definition}
	A principal \( G \)-bundle on \( X \) is a fiber bundle \( P \) on \( X \) with a right action of \( G \) on \( P \) such that the action is free and transitive on each fiber of \( P \) and the projection map \( \pi : P \rightarrow X \) is \( G \)-equivariant.
\end{definition}

\begin{definition}
	Let $V = \ker{\pi: TP \to TX}$ be the vertical bundle of $P$. An \emph{Ehresmann connection} on \( P \) is a smooth subbundle \( H \) of \( TP \), called the horizontal bundle of the connection, which is complementary to \( V \), in the sense that it defines a direct sum decomposition \( TE = H \oplus V \).
\end{definition}

The \emph{fundamental vector field} \( X^\# \) at a point \( p \in P \) is defined as:
\[
	X^\#_p = \left. \frac{d}{dt} \right|_{t=0} (p \cdot \exp(tX)),
\]
where:
\begin{itemize}
	\item \( \exp: \mathfrak{g} \to G \) is the exponential map,
	\item \( \exp(tX) \) is the one-parameter subgroup of \( G \) generated by \( X \),
	\item \( p \cdot \exp(tX) \) is the action of \( \exp(tX) \) on \( p \).
\end{itemize}

In other words, \( X^\# \) is the tangent vector to the curve \( t \mapsto p \cdot \exp(tX) \) at \( t = 0 \). The fundamental vector field \( X^\# \) is vertical, meaning it is tangent to the fibers of \( P \). This is because the action of \( G \) preserves fibers, so \( X^\# \) lies in the kernel of the differential \( d\pi: TP \to TM \), where \( \pi: P \to M \) is the projection.  The map \( X \mapsto X^\# \) is a Lie algebra homomorphism. Moreover, for \( g \in G \), the pushforward of \( X^\# \) by the right action \( R_g \) is:
\[
	(R_g)_* X^\# = (\text{Ad}_{g^{-1}} X)^\#,
\]
where \( \text{Ad} \) is the adjoint action of \( G \) on \( \mathfrak{g} \).


\begin{definition}
	A \emph{principal connection} is a \(\mathfrak{g}\)-valued 1-form \(\omega \in \Omega^1(P, \mathfrak{g})\) (where \(\mathfrak{g}\) is the Lie algebra of \(G\)) satisfying the following conditions:

	\begin{enumerate}
		\item For all \(X \in \mathfrak{g}\),
		      \[
			      \omega(X^\#) = X,
		      \]
		      where \(X^\#\) is the \textbf{fundamental vector field} on \(P\) generated by \(X\).


		\item For all \(g \in G\),
		      \[
			      R_g^* \omega = \text{Ad}_{g^{-1}} \circ \omega,
		      \]
		      where:
		      \begin{itemize}
			      \item \(R_g: P \to P\) is the right action of \(g\) on \(P\),
			      \item \(R_g^* \omega\) is the pullback of \(\omega\) by \(R_g\),
			      \item \(\text{Ad}_{g^{-1}}: \mathfrak{g} \to \mathfrak{g}\) is the adjoint action of \(G\) on \(\mathfrak{g}\).
		      \end{itemize}
	\end{enumerate}
\end{definition}

\begin{definition}
    The curvature of a $\mf{g}$-valued 1-form $\omega$ on a principal $G$-bundle $P$ is the $\mf{g}$-valued 2-form $F = d\omega + \frac{1}{2}[\omega, \omega]$, where $d\omega$ is the exterior derivative of $\omega$ and $[\omega, \omega]$ is the Lie bracket of $\omega$ with itself. We say that $\omega$ is \emph{flat, or integrable} if $F = 0$.
\end{definition}

\begin{remark}
    We can also write the curvature as the failure of commutativity of the covariant derivative:
    \[
        F(X, Y) = [\nabla_X, \nabla_Y] - \nabla_{[X, Y]}.
    \]
\end{remark}



\section{Associated bundle construction}
Given a principal $G$-bundle $P$ where $G$ acts freely transitively on the fibers, and a representation 
$G\to GL(V)$, we can form an associated vector bundle $E = P\times_G V$ by taking the quotient of $P\times V$ by the action of $G$ given by 
\[
g\cdot (p,v) = (p\cdot g^{-1}, g\cdot v).
\]
In other words, if $P$ has transition functions $U_{\alpha\beta} : U_\alpha\cap U_\beta \to G$, then $E$ has transition functions $U_{\alpha\beta} : U_\alpha\cap U_\beta \to GL(V)$. Conversely, if $E$ is a vector bundle with structure group $G$ via a representation $\rho$, we can recover the principal $G$-bundle $P$ as the frame bundle of $E$. 

The notion of a linear connection on $E$ coincides precisely with the notion of a principal connection on $P$. Likewise does the notion of integrability of a connection.

\section{Local systems}
Let $G^{\text{discrete}}$ be $G$ with the discrete topology. Then local systems are precisely $G^{\text{discrete}}$-bundles on $X$. Other interpretations of local systems include:
\begin{enumerate}
    \item A local system is given by a covering of open sets, transition functions
     \( \gamma_{ij} : U_i \cap U_j \to G \) which are locally constant and satisfy the 1-cocycle condition \( g_{ij} g_{jk} g_{ki} = id \). Equivalence of local systems is given by a common refinement of two coverings and a family of maps to \( G \) which conjugate one system of transition functions to the other.

    \item Suppose that \( X \) is connected. Then equivalence classes of local systems are in one-to-one correspondence with the equivalence classes of homomorphisms of the fundamental group \( \pi_1(X) \) to \( G \). Let $M$ be an integrable connection of rank $m$. Then the sheaf of horizontal sections \begin{align*}
        M^{\nabla} = \{s\in M : \nabla_X s = 0 \text{ for all } X\in \Theta_X\}
    \end{align*} is a locally free $\C_X$-module of rank m. This is how one thinks about local systems as locally constant sheaves of vector spaces.
    
    Moreover, by considering the parallel transport of sections of $M$ along paths, we can define a representation of the fundamental group $\pi_1(X)$ on the germ of horizontal sections, which is a vector space of dimension $m$. This representation is called the monodromy representation of $M$. 
\end{enumerate}

\section{Nonabelian Hodge theory}

Nonabelian Hodge theory on a smooth projective complex curve $X$, as formulated by Simpson, studies three different moduli problems for bundles for a complex reductive group $G$:
\begin{enumerate}
    \item [deRham] $\mathcal{C}onn_G(X)$: the moduli stack of flat G-connnections on X
    \item [Dolbeaut] $\mathcal{H}iggs_G(X)$: the moduli stack of G-Higgs bundles on X
    \item [Betti] $\mathcal{L}oc_G(X)$: the moduli stack of G-local systems on X
\end{enumerate}

\section{Geometric Langlands Program}
The geometric Langlands program provides a nonabelian, global and categorical form of harmonic analysis. We fix a complex reductive group $G$ and study the moduli stack $\Bun_G(X)$ of $G$-bundles on $X$. This stack comes equipped with a large commutative symmetry algebra: for any point $x\in X $we have a family of correspondences acting on $\Bun_G(X)$ by modifying $G$-bundles at $x$. The goal of the geometric Langlands program is to simultaneously diagonalize the action of Hecke correspondences on suitable categories of sheaves on $\Bun_G(X)$. 

One can ask to label the common eigensheaves (Hecke eigensheaves) by their eigenvalues (Langlands
parameters), or more ambitiously, to construct a Fourier transform identifying categories of sheaves
with dual categories of sheaves on the space of Langlands parameters.
The kernels for Hecke modifications are bi-equivariant sheaves on the loop group \( G(K) \), \( K = \mathbb{C}((t)) \), with respect to the arc subgroup \( G(O) \), \( O = \mathbb{C}[[t]] \). The underlying double cosets are in bijection with irreducible representations of the Langlands dual group.

\bibliography{refs}{}
% need refs.bib file
\bibliographystyle{plain}

\end{document}