\documentclass[12pt]{article}
\usepackage[english]{babel}
\usepackage[utf8x]{inputenc}
\usepackage[T1]{fontenc}
\usepackage{listings}
\usepackage{bookmark}
\usepackage{tikz}
\usepackage{/Users/songye03/Desktop/math_tex/style/quiver}
\usepackage{/Users/songye03/Desktop/math_tex/style/scribe}
\usepackage{fancyhdr}

\usepackage{parskip} % Automatically respects blank lines
\setlength{\parskip}{1em} % Adds more space between paragraphs
\setlength{\parindent}{0pt} % Removes paragraph indentation

\begin{document}


\lhead{Songyu Ye}
\rhead{\today}
\cfoot{\thepage}

\title{Equivariant Cohomology}

\author{Songyu Ye}
\date{\today}
\maketitle


\begin{abstract}
    Consider a compact Lie group $G$ acting on a space $X$. We can form the equivariant cohomology $H^*_G(X)$, which is a module over the equivariant cohomology ring $H^*_G(\text{pt})$. In this note, we will discuss some examples of equivariant cohomology, including the equivariant cohomology of Grassmannians and flag varieties. We will also discuss the relationship between equivariant cohomology and the representation theory of $G$.
\end{abstract}

\tableofcontents

\section{Examples of Equivariant Cohomology}
Consider $G = GL(V)$, for an $n$-dimensional vector space $V$. This has its standard representation on $V$ itself, so there are Chern classes $c_i^G(V) \in H_G^{2i}(\text{pt}) = \Lambda_G^{2i}$.

\begin{proposition}
    We have
    \[\Lambda_G = \mathbb{Z}[c_1, \ldots, c_n]\]
    where $c_i = c_i^G(V)$.
\end{proposition}

In particular,
\[\Lambda_{GL(V)} \text{ is a polynomial ring generated by the Chern classes of the standard representation.}\]

The result follows from the computation of the cohomology of Grassmannians and the fact that $BGL(V)$ can be identified with the infinite Grassmannian of $n$-planes in $\C^\infty$. Let $\mathbb{S} \to \Gr(n, \mathbb{C}^\infty)$ be the tautological bundle. Then we have
\begin{lemma}
    We have
    \[H^*\Gr(n, \mathbb{C}^m) = \mathbb{Z}[c_1(\mathbb{S}), \ldots, c_n(\mathbb{S})]/
        (R_{m-n+1}, \ldots, R_m),\]
    where $R_k$ is a relation of degree $k$.
\end{lemma}

\begin{proposition}
    We have\begin{align*}
        \Lambda_T & = H^*(BT) \cong H^*((\C\P^\infty)^n) \cong \Z[t_1, \ldots, t_n]
    \end{align*} where $t_i = c_1^T(L_i)$ is the first Chern class of the tautological line bundle on the $i$th factor of $\C\P^\infty$.
\end{proposition}
We can give a representation theoretic interpretation of this ring as well. Recall for each character of $T$, we can form a corresponding line bundle on $BT$ whose first Chern class is the corresponding character. Then we have the following result.

\begin{proposition} \label{prop:symmetric}
    Consider the inclusion $T \hookrightarrow GL_n$. The corresponding pullback map on cohomology gives
    \[\Lambda_{GL_n} = \mathbb{Z}[c_1,\ldots,c_n] \to \mathbb{Z}[t_1,\ldots,t_n] = \Lambda_T,\]
    defined by $c_i \mapsto e_i(t_1,\ldots,t_n)$, so $\Lambda_{GL_n}$ embeds in $\Lambda_T$ as the ring of symmetric polynomials.
\end{proposition}


\begin{remark}
    The inclusion $\Lambda_{GL_n} \hookrightarrow \Lambda_T$ is a manifestation of the \emph{splitting principle}. In particular, the elementary symmetric polynomials $e_i$ represent taking a $G$-module, decomposing it into characters, and then applying the formula for the Chern classes of a sum of line bundles. \begin{align*}
        e_i(c_1^T(L_1), \ldots, c_1^T(L_n)) & = c_i^T(L_1 \oplus \cdots \oplus L_n)
    \end{align*}
\end{remark}


\begin{example}
    If $H\subset G$ is a closed subgroup acting on $X$, then $G$ acts on $G\times_H X$ and there is a canonical isomorphism \begin{align*}
        H^*_G(G\times_H X) & \cong H^*_H(X)
    \end{align*}
    This is because the left hand side is the cohomology \begin{align*}
        H^*_G(G\times_H X) \cong H^*_G(EG \times_G (G\times_H X)) \cong H^*_G(EG \times_H X)
    \end{align*} and the right hand side is the cohomology \begin{align*}
        H^*_H(X)\cong H^*(EH \times_H X)
    \end{align*} and note that there is a canonical identification of $EG \cong EH$ since $EG$ is contractible and carries a free $H$-action.
\end{example}
\begin{example}[$G$-equivariant cohomology of $G/B$]
    Taking $X = *$ in the above example and $H = B$, we have \begin{align*}
        H^*_G(G/B) & \cong H^*_B(*)
    \end{align*} Note that $B$ can be decomposed as a semidirect product:
    \[B = T \ltimes U\]
    where $U$ is the unipotent radical of $B$ (consisting of the strictly upper triangular matrices in the classical case). In particular $U$ is isomorphic (as a variety) to an affine space $\mathbb{A}^N$ for some $N$ and therefore $B$ deformation retracts onto $T$.

    In particular, we have that $BB \cong BT$ and therefore we can identify \begin{align*}
        H^*_G(G/B) & \cong H^*_B(*) \cong H^*_T(*)
    \end{align*}
\end{example}
Another way of understanding this identification is given in the following proposition.
\begin{proposition}
    We have
    \[H^*_G Fl(V) = \Lambda_G[x_1,\ldots,x_n]/(e_i(x) - c_i)_{i=1,\ldots,n},\]
    where $c_i = c_i^G(V)$. A basis over $\Lambda_G$ is given by
    \[\{x_1^{m_1}\cdots x_n^{m_n} \mid 0 \leq m_i \leq n-i\},\]
    so $H^*_G Fl(V)$ has rank $n!$ as a $\Lambda_G$-module.
\end{proposition}
There is a map \begin{align*}
    \Lambda_G[x_1,\ldots,x_n] & \to \Lambda_T \\
\end{align*} which on the coefficient ring, equals the pullback on cohomology of the inclusion $T \hookrightarrow G$, and on the variables, sends $x_i \mapsto t_i$. This map is surjective, and the kernel is generated by the relations $e_i(x) - c_i$. This precisely gives the identification \begin{align*}
    \Lambda_G[x_1,\ldots,x_n]/(e_i(x) - c_i) & \cong \Lambda_T
\end{align*}
Generalizing Proposition \ref{prop:symmetric}, we have the following result which is due to Borel and can be found in Atiyah-Bott.

\begin{theorem}
    Let $G$ be a compact Lie group and $T$ a maximal torus, $W = N(T)/T$ the Weyl group. For any $G$-topological space $X$, we have
    \begin{align*}
        H^*_G(X) & \cong H^*_T(X)^W
    \end{align*}
\end{theorem}

\begin{theorem}[$T$-equivariant cohomology of $G/B$]
    We have
    \begin{align*}
        H^*_T(G/B) & \cong \text{Sym}(T^*) \otimes_{\text{Sym}(T^*)^W} \text{Sym}(T^*)
    \end{align*} In particular, for $\GL_n(\mathbb{C})$, we have
    \begin{align*}
        H^*_T(Fl(\mathbb{C}^n)) & \cong \mathbb{Q}[x_1,\ldots,x_n,y_1,\ldots,y_n] \Big/ \langle p(x) = p(y) \text{ $\forall$ symmetric polynomials $p$}\rangle
    \end{align*}
\end{theorem}
\begin{proof}
    \begin{align*}
        H^*_T(G/B) & \cong H^*_B(G/B)                                                  & \text{undividing by the free $B$-action}                     \\
                   & \cong H^*_{B \times B}(G)                                         & \text{undividing $G \times G$ by the free $G_\Delta$-action} \\
                   & \cong H^*_G(G/B \times G/B)                                       & \text{dividing by the now-free $B \times B$-action}          \\
                   & \cong H^*_G(G/B) \otimes_{H^*_G(pt)} H^*_G(G/B)                   & \text{the equivariant Künneth theorem}                       \\
                   & \cong H^*_{G \times B}(G) \otimes_{H^*_G(pt)} H^*_{G \times B}(G) & \text{undividing by the $B \times B$-action}                 \\
                   & \cong H^*_B(pt) \otimes_{H^*_G(pt)} H^*_B(pt)                     & \text{dividing by the free $G \times G$-action}              \\
                   & \cong H^*_T(pt) \otimes_{H^*_G(pt)} H^*_T(pt)                     & \text{since $B/T$ is contractible}                           \\
                   & \cong \text{Sym}(T^*) \otimes_{\text{Sym}(T^*)^W} \text{Sym}(T^*)
    \end{align*}
I learned this proof from Allen Knutson.
\end{proof}

\begin{definition}[Double Schubert Polynomials]
    The double Schubert polynomials $\mathfrak{S}_w(x,y)$ for $w \in S_n$ (the symmetric group on $n$ letters) are polynomials in two sets of variables $x = (x_1, x_2, \ldots)$ and $y = (y_1, y_2, \ldots)$ defined recursively as follows:

    \begin{enumerate}
        \item For the longest permutation $w_0 = n, n-1, \ldots, 1$ in $S_n$, we define:
              \[ \mathfrak{S}_{w_0}(x,y) = \prod_{1 \leq i < j \leq n} (x_i - y_j) \]

        \item For any $w \in S_n$ and a simple transposition $s_i = (i, i+1)$ such that $\ell(ws_i) < \ell(w)$ (where $\ell$ denotes the length function), we define:
              \[ \mathfrak{S}_w(x,y) = \partial_i(\mathfrak{S}_{ws_i}(x,y)) \]
              where $\partial_i$ is the divided difference operator given by:
              \[ \partial_i(f) = \frac{f(x_1, \ldots, x_i, x_{i+1}, \ldots) - f(x_1, \ldots, x_{i+1}, x_i, \ldots)}{x_i - x_{i+1}} \]
    \end{enumerate}
\end{definition}

\end{document}