\documentclass[12pt]{article}
\usepackage[english]{babel}
\usepackage[utf8x]{inputenc}
\usepackage[T1]{fontenc}
\usepackage{listings}
\usepackage{bookmark}
\usepackage{tikz}
\usepackage{/Users/songye03/Desktop/math_tex/style/quiver}
\usepackage{/Users/songye03/Desktop/math_tex/style/scribe}
\usepackage{fancyhdr}

\usepackage{parskip} % Automatically respects blank lines
\setlength{\parskip}{1em} % Adds more space between paragraphs
\setlength{\parindent}{0pt} % Removes paragraph indentation

\begin{document}


\lhead{Songyu Ye}
\rhead{\today}
\cfoot{\thepage}

\title{Gelfand-Tsetlin Patterns for $\overline{PGL(3)}$}

\author{Songyu Ye}
\date{\today}
\maketitle

In $\sl(3)$, we have the simple roots \begin{align*}
    \alpha_1 &= \epsilon_1 - \epsilon_2, \\
    \alpha_2 &= \epsilon_2 - \epsilon_3.
\end{align*} subject to $\epsilon_1 + \epsilon_2 + \epsilon_3 = 0$. This Lie algebra has a Cartan matrix \begin{align*}
    A = \begin{pmatrix}
        2 & -1 \\
        -1 & 2
    \end{pmatrix}.
\end{align*} The fundamental weights are \begin{align*}
    \omega_1 &= \frac{2}{3} \alpha_1 + \frac{1}{3} \alpha_2 \\
    \omega_2 &= \frac{1}{3} \alpha_1 + \frac{2}{3} \alpha_2.
\end{align*} In particular the root lattice $\Lambda_r$ sits inside the weight lattice $\Lambda_w$ as a sublattice of index 3. \begin{align*}
    \Lambda_r &= \set{a \omega_1 + b \omega_2 \mid a \equiv b \mod 3} 
\end{align*}
The weight lattice for the adjoint group $\PGL(3)$ is just the root lattice $\Lambda_r$. This follows from the fact that if $G_{sc}$ is the simply connected group whose Lie algebra is $\mf g$ then $\Lambda_w / \Lambda_r \cong Z(G_{sc})$.

Now recall that we are interested in the lattice \begin{align*}
    L = \set{(\mu,\lambda,\Gamma_1,\Gamma_2)}
\end{align*} where $\mu$ is a dominant weight, $\lambda$ is any weight, $\mu \leq \lambda$, and $\Gamma_1, \Gamma_2$ are Gelfand-Tsetlin patterns for the dominant weight $\mu$. 

\begin{remark}
    We copied this down wrong back in the spring of 2024. When we asked Shiliang about the noncomputability stuff, we had asked that $\lambda$ be dominant as well. This is not the case. In particular, unlike for the flag variety $G/B$, some line bundles on the wonderful compactification that correspond to non-dominant weights have global sections.
\end{remark}

So now we want to take a Hilbert basis for this monoid. I have the following list which I understand when I project to the $\mu,\lambda$. \begin{enumerate}
    \item $(3\omega_1, 3\omega_1, -, -)$
    \item $(3\omega_2, 3\omega_2, -, -)$
    \item $(\omega_1 + \omega_2, \omega_1 + \omega_2, -, -)$
    \item $(2\omega_1 - \omega_2, 0, -, -)$
    \item $(\omega_1 - 2\omega_2, 0, -, -)$
\end{enumerate}
The first three are about the fact that every dominant weight is a sum of those three. The last two are about the fact that $\lambda$ need not be dominant. My question that I wasn't sure about is what I do for the patterns.

\begin{itemize}
    \item Patterns for $3\omega_1$. The inequalities "look like" $3 \geq x \geq 0$ but I feel like I should only be able to take $x = 0$ and $x=3$. Is this the case?
    \item I am also confused (moreso than the first point) about the patterns for $\omega_1 + \omega_2$. So this weight looks like the equivalence class $(2,1,0)$ and to fill in the pattern $2\geq x \geq 1 \geq y \geq 0$, I feel like there should only be three ways, corresponding to the possible values of the difference $x-y$. But I am not sure.
\end{itemize}
I guess I am looking for an answer to the following problem: You defined Gelfand-Tsetlin inequalities in terms of the numbers in the partition. Is there a good way to write down the inequalities using fundamental weights? 


\end{document}