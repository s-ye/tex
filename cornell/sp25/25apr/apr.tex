\documentclass[12pt]{article}
\usepackage[english]{babel}
\usepackage[utf8x]{inputenc}
\usepackage[T1]{fontenc}
\usepackage{listings}
\usepackage{bookmark}
\usepackage{tikz}
\usepackage{/Users/songye03/Desktop/math_tex/style/quiver}
\usepackage{/Users/songye03/Desktop/math_tex/style/scribe}
\usepackage{fancyhdr}

\usepackage{parskip} % Automatically respects blank lines
\setlength{\parskip}{1em} % Adds more space between paragraphs
\setlength{\parindent}{0pt} % Removes paragraph indentation

\begin{document}


\lhead{Songyu Ye}
\rhead{\today}
\cfoot{\thepage}

\title{April 2025 notes}

\author{Songyu Ye}
\date{\today}
\maketitle


\begin{abstract}
    I enjoyed the end of my senior year in Ithaca. I had a lot of fun reconnecting with friends and hanging out. I also had a lot of fun doing math.
\end{abstract}

\tableofcontents

\section{Apr 7}
\subsection{Hilbert scheme}
Recall that the Hilbert polynomial of a subscheme $X$ of $\mathbb{P}^r$ is a numerical polynomial characterized by the equations
\[
    P_X(m) = h^0(X, \mathcal{O}_X(m))
\]
for all sufficiently large $m$. If $X$ has degree $d$ and dimension $s$, then the leading term of $P_X(m)$ is $dm^s/s!$. This shows both that $P_X$ captures the main numerical invariants of $X$, and that fixing it yields a set of subschemes of reasonable size. Moreover, if a proper connected family $\mathcal{X} \to B$ of such subschemes is flat, then the Hilbert polynomials of all fibers of $\mathcal{X}$ are equal, and, if $B$ is reduced, then the converse also holds. Thus, fixing $P_X$ also forces the fibers of the families we're considering to vary nicely.

Intuitively, the Hilbert scheme $\mathcal{H}_{P,r}$ parameterizes subschemes $X$ of $\mathbb{P}^r$ with fixed Hilbert polynomial $P_X$ equal to $P$. More formally, it's a fine moduli space for the functor $\mathsf{Hilb}_{P,r}$ whose value on $B$ is the set of proper \emph{flat} families
\[
    \begin{tikzcd}
        \mathcal{X} \arrow[hookrightarrow, "i"]{r} \arrow[dr, "\varphi"'] & \mathbb{P}^r \times B \arrow[d, "\pi_B"] \arrow[r, "\pi_{\mathbb{P}^r}"] & \mathbb{P}^r \\
        & B
    \end{tikzcd}
\]
with $\mathcal{X}$ having Hilbert polynomial $P$.

\begin{theorem}[Grothendieck]
    The functor $\mathsf{Hilb}_{P,r}$ is representable by a projective scheme $\mathcal{H}_{P,r}$.
\end{theorem}

Let $S = \C[x_0, \ldots, x_r]$ and let $Q(m) = \cO_r(m) - P(m)$ be the dimension of the degree $m$ piece of the ideal of $X$, for large $m$.
\begin{lemma}
    For every $P$, there is an $m_0$ such that if $m \geq m_0$ and $X$ is a subscheme of $\mathbb{P}^r$ with Hilbert polynomial $P$, then:

    \begin{enumerate}
        \item $I(X)_m$ is generated by global sections and $I(X)_{\geq m}$ is generated by $I(X)_m$ as an $S$-module.

        \item $h^i(X, I_X(m)) = h^i(X, \mathcal{O}_X(m)) = 0$ for all $i > 0$.

        \item $\dim(I(X)_m) = Q(m)$, $h^0(X, \mathcal{O}_X(m)) = P(m)$ and the restriction map
              \[
                  r_{X,m} : S_m \rightarrow H^0(X, \mathcal{O}_X(m))
              \]
              is surjective.
    \end{enumerate}
\end{lemma}


The key idea of the construction is that the lemma allows us to associate to \emph{every} subscheme $X$ with Hilbert polynomial $P$ the point $[X]$ of the Grassmannian
\[
    \mathcal{G} = \mathbb{G}(P(m), \mathcal{O}_r(m))
\]
determined by $r_{X,m}$. More formally again, if $\varphi : \mathcal{X} \to B$ is any family as in (1.8), then from the sheafification of the restriction maps
\[
    (\pi_{\mathbb{P}})^*(\mathcal{O}_{\mathbb{P}^r}(m)) \longrightarrow (\pi_{\mathbb{P}})^*(\mathcal{O}_{\mathbb{P}^r}(m) \otimes \mathcal{O}_\mathcal{X}) \longrightarrow 0
\]
we get a second surjective restriction map
\[
    \begin{tikzcd}
        (\pi_B)_*(\pi_{\mathbb{P}})^* \mathcal{O}_{\mathbb{P}^r}(m) \arrow[r] \arrow[d, equal] &
        (\pi_B)_*(\pi_{\mathbb{P}})^*\left( \mathcal{O}_{\mathbb{P}^r}(m) \otimes \mathcal{O}_{\mathcal{X}} \right) \arrow[r] & 0 \\
        \mathcal{O}_B \otimes S_m &
    \end{tikzcd}
\]

The middle factor is a locally free sheaf of rank $P(m)$ on $B$ and yields a map $\Psi(\varphi) : B \to \mathcal{G}$. Since these maps are functorial in $B$, we have a natural transformation $\Psi$ to the functor of points of some subscheme $\mathcal{H} = \mathcal{H}_{P,r}$ of $\mathcal{G}$.

It remains to identify $\mathcal{H}$ and to show it represents the functor $\mathsf{Hilb}_{P,r}$. The key to doing so is provided by the universal subbundle $\mathcal{F}$ whose fiber over $[X]$ is $I(X)_m$ and the multiplication maps
\[
    \times_k : \mathcal{F} \otimes S_k \to S_{k+m}.
\]

\begin{lemma}
    The conditions that $\operatorname{rank}(\times_k) \leq Q(m+k)$ for all $k \geq 0$ define a determinantal subscheme $\mathcal{H}$ of $\mathcal{G}$, and a morphism $\psi : B \to \mathcal{G}$ arises by applying the construction above to a family $\varphi : \mathcal{X} \to B$ (i.e., $\psi = \Psi(\varphi)$) if and only if $\psi$ factors through this subscheme $\mathcal{H}$.
\end{lemma}

\begin{remark}
    A point of the Hilbert scheme corresponds to an ideal $J \subset S$ such that $S/J$ has Hilbert polynomial $P$. The lemmas are saying that among all subspaces of $S_m$ for sufficiently large $m$, the ones that correspond to ideals of $S$ are precisely those which under multiplication by $S_k$ land in the subspace of $S_{k+m}$ and have the correct rank. This is expressed in a rank condition which very clearly exhibits the Hilbert scheme as a closed subscheme of the Grassmannian.
\end{remark}
The existence of the following Hilbert schemes can be deduced from very similar arguments.

\begin{definition}[Hilbert schemes of subschemes]
    Given a subscheme $Z$ of $\mathbb{P}^r$, we can define a closed subscheme $\mathcal{H}^Z_{P,r}$ of $\mathcal{H}_{P,r}$ parameterizing subschemes of $Z$ that are closed in $\mathbb{P}^r$ and have Hilbert polynomial $P$.
\end{definition}

\begin{definition}[Hilbert schemes of maps]
    If $X \subset \mathbb{P}^r$ and $Y \subset \mathbb{P}^s$, there is a Hilbert scheme $\mathcal{H}_{X,Y,d}$ parameterizing polynomial maps $f : X \to Y$ of degree at most $d$. This variant is most easily constructed as a subscheme of the Hilbert scheme of subschemes of $X \times Y$ in $\mathbb{P}^r \times \mathbb{P}^s$ using the Hilbert points of the graphs of the maps $f$.
\end{definition}

\begin{definition}[Hilbert schemes of projective bundles]
    From a $\mathbb{P}^r$ bundle $\mathcal{P}$ over $Z$, we can construct a Hilbert scheme $\mathcal{H}_{P,\mathcal{P}/Z}$ parameterizing subschemes of $\mathcal{P}$ whose fibers over $Z$ all have Hilbert polynomial $P$.
\end{definition}

\begin{definition}[Relative Hilbert schemes]
    Given a projective morphism $\pi : \mathcal{X} \to Z \times \mathbb{P}^r \to Z$, we have a relative Hilbert scheme $\mathcal{H}$ parameterizing subschemes of the fibers of $\pi$. Explicitly, $\mathcal{H}$ represents the functor that associates to a scheme $B$ the set of subschemes $\mathcal{Y} \subset B \times \mathbb{P}^r$ and morphisms $\alpha : B \to Z$ such that $\mathcal{Y}$ is flat over $B$ with Hilbert polynomial $P$ and
    \[
        \mathcal{Y} \subset B \times_{Z} \mathcal{X}.
    \]
\end{definition}

\subsection{Tangent space of the Hilbert scheme}
Recall that the tangent space to any scheme $X$ at a closed point $p$ is the set of maps $\operatorname{Spec}(\C[\epsilon]/(\epsilon^2)) \to X$ that send the closed point of $\operatorname{Spec}(\C[\epsilon]/(\epsilon^2))$ to $p$. Let $\bI = \Spec \C[\epsilon]/(\epsilon^2)$ and more generally let $\bI^n = \Spec \C[\epsilon_1, \ldots, \epsilon_n]/(\epsilon_1,\dots,\epsilon_n)^2$. The tangent space of a scheme $X$ at a point $p$ is the set of morphisms $\bI \to X$ that send the closed point of $\bI$ to $p$. This is equivalent to the set of morphisms $\bI^n \to X$ that send the closed point of $\bI^n$ to $p$.

Given two tangent vectors, one forms their sum by taking the fiber product over the closed point $\bI \times \bI$ which can then be identified with $\bI^2$. Then the map $\bI \to \bI^2 \to X$ where the first map is the diagonal (on the level of rings, it sends each $\epsilon_i$ to $\epsilon$) is precisely the sum.

If $\cH$ is a Hilbert scheme and $[X] \in \cH$ is a closed subscheme then a tangent vector to $\cH$ at $[X]$ corresponds to, by the universal property, a flat family $\cX \to \bI$ of subschemes of $\P^r \times \bI$ whose fiber over $0 \in \bI$ is $X$. This family is known as a \textbf{first order deformation } of $X$.

Recall that $\cH$ is a closed subscheme of the Grassmannian $\cG$ of $P(m)$-dimensional quotients of $S_m$. Any tangent vector to $\cG$ at the point $[Q]$ corresponding to the quotient $Q$ of $S_m$ by the subspace $L$ of codimension $P(m)$ can be identified with a $\C$-linear map $\phi: L \to S_m/L$. If $\tilde\phi: L \to S_m$ lifts $\phi$ then the collection \begin{align*}
    \set{f + \epsilon \tilde\phi(f) \st f \in I(X)_m} \subset S_m
\end{align*} gives a first order deformation of $X$. In particular, we can view the collection $\set{f + \epsilon \tilde\phi(f) \st f \in I(X)_m}$ as polynomial functions on $\P^r\times \bI$ defining a subscheme $\cX$ of $\P^r \times \bI$. Then such a tangent vector to $\cG$ will lie in the Zariski tangent space to $\cH$ if and only if $\cX$ is flat over $\bI$. This is equivalent to saying that apriori $\C$-linear map $I(X)_m \to S_m/I(X)_m$ actually extends to a homomorphism of $S$-modules
\[
    \phi: I(X)_{l\geq m} \to S_m/I(X)_{l\geq m}.
\]
This map $\phi$ in fact determines a map of coherent sheaves $\cI \to \cO_{\P^r}/\cI$ and by $S$-linearity the kernel of such a map must contain $\cI^2$. Therefore, we see that a tangent vector to $\cH$ at $[X]$ corresponds to an element of $\Hom(\cI/\cI^2,\cO_X)$. When $X$ is smooth, this is precisely the normal sheaf $\cN_{X/\P^r}$ of $X$ in $\P^r$. When $X$ is singular, we define this to be the normal sheaf of $X$. With this convention, the Zariski tangent space to the Hilbert scheme at a point $X$ is the space of global sections of the normal sheaf of $X$: \begin{align*}
    T_{[X]}\cH \cong H^0(X,\cN_{X/\P^r}).
\end{align*}

\subsection{Vector bundles, locally free sheaves, and a classical example}
Recall that a locally free sheaf $\cF$ on a scheme $X$ is a sheaf of $\cO_X$-modules such that for every point $x \in X$, there exists an open neighborhood $U$ of $x$ such that $\cF|_U \cong \cO_U^{\oplus n}$ for some integer $n$. The rank of the locally free sheaf is the integer $n$.

If $\cF$ is a locally free sheaf, then there exists an affine open cover $\Spec R_i$ of $X$ such that $\cF|_{\Spec R_i} $ is a finitely generated projective $R_i$-module.

\begin{proposition}
    Let $X = \Spec R$ be an affine scheme, and let $\cF = \tilde M$ be a quasicoherent sheaf. Then $\cF$ is locally free sheaf on $X$ of finite rank if and only if $M$ is a finitely generated projective $R$-module. In this case, the rank of $\cF$ is the rank of $M$ as an $R$-module.
\end{proposition}

\begin{theorem}
    A finitely generated module $M$ over a commutative ring $R$ is projective if and only if it is locally free, i.e. for every $\mathfrak{p} \in \Spec R$, the localization $M_{\mathfrak{p}}$ is a free $R_{\mathfrak{p}}$-module.
\end{theorem}

\begin{proof}
    Suppose first that $M$ is finitely generated and projective. Then there exists an $R$-module $N$ and an integer $n$ such that
    $$
        M \oplus N \cong R^n.
    $$
    For any prime ideal $\mathfrak{p} \in \operatorname{Spec}(R)$, localizing gives
    $$
        M_{\mathfrak{p}} \oplus N_{\mathfrak{p}} \cong (R^n)_{\mathfrak{p}} \cong R_{\mathfrak{p}}^n.
    $$
    Since $R_{\mathfrak{p}}$ is local and every finitely generated projective module over a local ring is free (by Nakayama's lemma, for example), it follows that $M_{\mathfrak{p}}$ is free for every prime $\mathfrak{p}$. Thus, $M$ is locally free.

    Conversely, suppose that for every prime ideal $\mathfrak{p}$, the localization $M_{\mathfrak{p}}$ is a free $R_{\mathfrak{p}}$-module. Since $M$ is finitely generated, there exists a surjection
    $$
        \pi \colon R^N \to M
    $$
    for some integer $N$. Localizing at each prime $\mathfrak{p}$, we obtain
    $$
        \pi_{\mathfrak{p}} \colon R_{\mathfrak{p}}^N \to M_{\mathfrak{p}}.
    $$
    By assumption, $M_{\mathfrak{p}}$ is free, so each localized surjection splits; that is, there exists an $R_{\mathfrak{p}}$-module homomorphism
    $$
        s_{\mathfrak{p}} \colon M_{\mathfrak{p}} \to R_{\mathfrak{p}}^N
    $$
    with $\pi_{\mathfrak{p}} \circ s_{\mathfrak{p}} = \operatorname{id}_{M_{\mathfrak{p}}}$. These local splittings can be patched together (using the fact that $M$ is finitely generated) to produce a global splitting
    $$
        s \colon M \to R^N
    $$
    with $\pi \circ s = \operatorname{id}_M$. Hence, $M$ is isomorphic to a direct summand of the free module $R^N$, which implies that $M$ is projective.
\end{proof}

\begin{lemma}
    
Let $R$ be a commutative ring and let $I$ be an ideal contained in the Jacobson radical of $R$. If $M$ is a finitely generated $R$-module such that
$$
M = I M,
$$
then $M = 0$.

\end{lemma}

\begin{proof}

Let $m_1,\dots,m_n$ be generators for $M$. Since $M = I M$, for each $i$ there exist elements $a_{i1}, a_{i2}, \dots, a_{in} \in I$ such that
$$
m_i = \sum_{j=1}^n a_{ij} m_j.
$$
We can write these relations in matrix form as
$$
\begin{pmatrix}
m_1 \\ m_2 \\ \vdots \\ m_n
\end{pmatrix}
=
\begin{pmatrix}
a_{11} & a_{12} & \cdots & a_{1n} \\
a_{21} & a_{22} & \cdots & a_{2n} \\
\vdots & \vdots & \ddots & \vdots \\
a_{n1} & a_{n2} & \cdots & a_{nn}
\end{pmatrix}
\begin{pmatrix}
m_1 \\ m_2 \\ \vdots \\ m_n
\end{pmatrix}.
$$
Denote this $n \times n$ matrix by $A$. Then the above equation is equivalent to
$$
\Bigl(I - A\Bigr)
\begin{pmatrix}
m_1 \\ m_2 \\ \vdots \\ m_n
\end{pmatrix}
= 0,
$$
where $I$ is the $n\times n$ identity matrix.

Since every entry $a_{ij}$ belongs to the ideal $I$, and $I$ is contained in the Jacobson radical of $R$, each diagonal element $1 - a_{ii}$ is a unit in $R$. In fact, it follows that the matrix $I - A$ is invertible over $R$. Multiplying both sides of the equation by $(I-A)^{-1}$, we deduce that
$$
\begin{pmatrix}
m_1 \\ m_2 \\ \vdots \\ m_n
\end{pmatrix} = 0.
$$
Thus, $m_1 = m_2 = \cdots = m_n = 0$, so $M = 0$. 
\end{proof}

\begin{corollary}
    Every finitely generated projective module over a local ring is free. 
\end{corollary}

\begin{proof}
Let $(R,\mathfrak{m})$ be a local ring and let $P$ be a finitely generated projective $R$-module. Since $P$ is projective, it is a direct summand of some free module. In other words, there exists an $R$-module $Q$ and an integer $n$ such that
$$
P \oplus Q \cong R^n.
$$

Reducing modulo $\mathfrak{m}$, we obtain an isomorphism
$$
P/\mathfrak{m}P \oplus Q/\mathfrak{m}Q \cong (R/\mathfrak{m})^n.
$$
In particular, the minimal number of generators of $P$ is given by the $R/\mathfrak{m}$-vector space dimension
$$
r := \dim_{R/\mathfrak{m}}(P/\mathfrak{m}P).
$$

Now, because $P$ is finitely generated, there exists a surjective $R$-module homomorphism
$$
\pi \colon R^r \twoheadrightarrow P.
$$
Reducing $\pi$ modulo $\mathfrak{m}$ gives a surjective linear map
$$
\overline{\pi}\colon (R/\mathfrak{m})^r \to P/\mathfrak{m}P.
$$
Since the number $r$ is minimal, $\overline{\pi}$ must be an isomorphism.

Let $K = \ker \pi$. Then we have the exact sequence
$$
0 \to K \to R^r \xrightarrow{\pi} P \to 0.
$$
After reducing modulo $\mathfrak{m}$, this yields
$$
K/\mathfrak{m}K \to (R/\mathfrak{m})^r \xrightarrow{\overline{\pi}} P/\mathfrak{m}P \to 0.
$$
Since $\overline{\pi}$ is an isomorphism, it follows that
$$
K/\mathfrak{m}K = 0.
$$

Now, applying Nakayama’s Lemma (which states that if a finitely generated $R$-module $M$ satisfies $M = \mathfrak{m}M$, then $M = 0$), we deduce that $K = 0$. Therefore, the map $\pi$ is an isomorphism, and hence
$$
P \cong R^r.
$$

This proves that every finitely generated projective module over a local ring is free.
\end{proof}

\begin{remark}
    If $\cF$ is a coherent sheaf, $\cF_x/\mathfrak{m}_x \cF_x$ is a finite dimensional vector space over the residue field $\kappa(x) = \cO_{X,x}/\mathfrak{m}_x$. The vector space $\cF_x/\mathfrak{m}_x \cF_x$ is called the \textbf{fiber} of $\cF$ at $x$. The rank of $\cF$ at $x$ is the dimension of this vector space. If $\cF$ is locally free, then the rank is constant on $X$.

Nakayama's lemma implies that every basis of the fiber lifts to a minimal generating set of $\cF_x$, which is a module over the local ring $\cO_{X,x}$. 
\end{remark}

\begin{example}
    We will show that the sheaf associated to the ideal $I = \langle x,y \rangle$ is not a vector bundle on $\A^2$. In particular we will see that its fibers are not of constant rank.

    In particular the rank of $\tilde I_x$ is different for the closed points $\ideal{x-a,y-b}$ and $\ideal{x,y}$.

    Recall that there is an exact sequence \begin{align*}
        R^2 \to R \to k(\ideal{x,y}) = \text{Frac}(R/\ideal{x,y}) \to 0
    \end{align*} where the first map is given by multiplication by $x,y$ and the second map is the canonical evaluation map taking an element $r\in R$ to its value in the residue field at the closed point $\ideal{x,y}$.

We can apply $\otimes_R I$ to this sequence to obtain the following exact sequence (note that tensor product is right exact):
\begin{align*}
    I^2 \to I \to k(\ideal{x,y}) \otimes_R I \to 0
\end{align*}
We can identify the rightmost term as, writing $\cF = \tilde I$\begin{align*}
\cF_{\ideal{x,y}} &= k(\ideal{x,y}) \otimes_{R_{\ideal{x,y}}} I_{\ideal{x,y}} \\
&= k(\ideal{x,y}) \otimes_{R_{\ideal{x,y}}} R_{\ideal{x,y}} \otimes_R I \\
&= k(\ideal {x,y}) \otimes_R I \\
&= I/\ideal{xI,yI} \\
&= k\cdot x \oplus k\cdot y 
\end{align*} is a 2-dimensional vector space over $k$.

We can do the same for the other closed point $\ideal{x-a,y-b}$:
\begin{align*}
    \cF_{\ideal{x-a,y-b}} &= k(\ideal{x-a,y-b}) \otimes_{R_{\ideal{x-a,y-b}}} I_{\ideal{x-a,y-b}} \\
    &= k(\ideal{x-a,y-b}) \otimes_{R_{\ideal{x-a,y-b}}} R_{\ideal{x-a,y-b}} \otimes_R I \\
    &= k(\ideal {x-a,y-b}) \otimes_R I \\
    &= I/\ideal{(x-a)I,(y-b)I} 
\end{align*}
Now say that $a$ is nonzero and $b$ is nonzero. Then this is a $3$-dimensional vector space over $k$ because we have the relation $x^2 = ax$ and $y^2 = bx$ (in particular $xy\not\in \ideal{(x-a)I,(y-b)I}$) with basis elements $x,xy,y$.
\end{example}

\section{Apr 9}
\subsection{Primes and Dedekind Domains, Separability and Unramified Extensions}
\begin{definition}
    An algebraic extension of fields $K/F$ is called separable if every element of $K$ is the root of a separable polynomial over $F$. A field extension $K/F$ is called purely inseparable if every element of $K$ is the root of a polynomial of the form $x^{p^n} - a$ for some $a \in F$ and some integer $n \geq 0$. This is equivalent to saying that the minimal polynomial of every element of $K$ over $F$ is not separable.
\end{definition}

\begin{example}
    Every algebraic extension of a field of characteristic $0$ is separable. 
    For finite fields, the only algebraic extensions are of the form $F_{p^n}/F_p$ for some integer $n \geq 0$, and these are all separable.
\end{example}

\begin{definition}
    A morphism $\pi : X \to Y$ is \emph{smooth of relative dimension $n$} provided that it is locally of finite presentation and flat of relative dimension $n$, and $\Omega_{X/Y}$ is locally free of rank $n$.
    
    A morphism $\pi : X \to Y$ is \emph{\'etale} provided that it is locally of finite presentation and flat, and $\Omega_{X/Y} = 0$.
    
    A morphism $\pi : X \to Y$ is \emph{unramified} provided that it is locally of finite presentation, and $\Omega_{X/Y} = 0$.
    \end{definition}

\begin{example}
    Let $K/F$ be a finite separable extension of fields. Then the map $\Spec K \to \Spec F$ is etale.
\end{example}

In particular, separability is about the fact that the module of relative Kahler differentials is zero because the differential of a separable polynomial satisfies $(f,df) = 1$ so we are quotienting by relations that end up being the whole module.

We have a finite extension of fields so the morphism is indeed locally of finite presentation. Finally, every finite module over a field is a vector space and hence flat.

\subsection{Absolute values, valuations, and places}

\begin{definition}
An absolute value on a field $K$ is a function
\[
|\cdot| : K \to \mathbb{R}_{\geq 0}
\]
satisfying:
\begin{enumerate}
    \item $|x| = 0 \iff x = 0$,
    \item $|xy| = |x||y|$,
    \item $|x + y| \leq |x| + |y|$ (triangle inequality).
\end{enumerate}

We say it is \textbf{Nonarchimedean} if it satisfies the strong triangle inequality:
    \[
    |x + y| \leq \max(|x|, |y|),
    \]
and \textbf{Archimedean} otherwise (like the usual absolute value on $\mathbb{Q}$ or $\mathbb{R}$).

\end{definition}


\begin{definition}
A valuation on a field $K$ is a map
\[
v: K^\times \to \Gamma
\]
into a totally ordered abelian group $\Gamma$, satisfying:
\begin{itemize}
    \item $v(xy) = v(x) + v(y)$,
    \item $v(x + y) \geq \min\{v(x), v(y)\}$,
\end{itemize}
and we set $v(0) = \infty$.

In the case of a discrete valuation, $\Gamma = \mathbb{Z}$. For example, the $p$-adic valuation:
\[
v_p\left( \frac{a}{b} \right) = \text{exponent of } p \text{ in the prime factorization of } a - b.
\]

\end{definition}
We can go between valuations and absolute values using an exponential:

Given a valuation $v: K^\times \to \mathbb{Z}$, define: $|x| := c^{-v(x)} \quad \text{for some } c > 1$. This gives a nonarchimedean absolute value. For example, with $v_p(x)$, take $c = p$, and you get the $p$-adic absolute value: $|x|_p := p^{-v_p(x)}$. 

In particular, nonarchimedean absolute values are precisely those that arise from valuations into $\mathbb{R}_{\geq 0}$. On the other hand, archimedean absolute values correspond to embeddings into $\mathbb{R}$ or $\mathbb{C}$, from which the absolute value is obtained by taking the usual absolute value on $\mathbb{R}$ or $\mathbb{C}$.

We say that two absolute values are equivalent if for all $x \in K$, $|x|_1 < 1$ if and only if $|x|_2 < 1$. The set of equivalence classes of absolute values on $K$ is called the set of places of $K$, denoted $\mathcal{P}_K$. The nonarchimedean places are those that arise from nonarchimedean absolute values, i.e. discrete valuations. 

\subsection{Adeles and Ideles}
\begin{definition}
The adeles of a global field $K$ are defined as the restricted product of the completions $K_v$ of $K$ at all places $v \in \mathcal{P}_K$, with respect to the valuation rings $\mathcal{O}_v$ for nonarchimedean places. Explicitly:
\[
\mathbb{A}_K = \left\{ (x_v) \in \prod_{v \in \mathcal{P}_K} K_v \mid x_v \in \mathcal{O}_v \text{ for almost all nonarchimedean } v \right\}.
\]

The ideles of $K$ are the units of the adele ring:
\[
\mathbb{I}_K = \mathbb{A}_K^\times = \left\{ (x_v) \in \prod_{v \in \mathcal{P}_K} K_v^\times \mid x_v \in \mathcal{O}_v^\times \text{ for almost all nonarchimedean } v \right\}.
\]
\end{definition}
\begin{example}
For $K = \mathbb{Q}$, the adeles are:
\[
\mathbb{A}_\mathbb{Q} = \left\{ (x_\infty, x_2, x_3, \dots) \in \mathbb{R} \times \prod_{p \text{ prime}} \mathbb{Q}_p \mid x_p \in \mathbb{Z}_p \text{ for almost all } p \right\}.
\]
The ideles are:
\[
\mathbb{I}_\mathbb{Q} = \mathbb{A}_\mathbb{Q}^\times = \left\{ (x_\infty, x_2, x_3, \dots) \in \mathbb{R}^\times \times \prod_{p \text{ prime}} \mathbb{Q}_p^\times \mid x_p \in \mathbb{Z}_p^\times \text{ for almost all } p \right\}.
\]
\end{example}

The adele ring $\mathbb{A}_K$ is a topological ring, and the ideles $\mathbb{I}_K$ form a topological group under multiplication. The field $K$ embeds diagonally into $\mathbb{A}_K$ and $\mathbb{I}_K$ via:
\[
x \mapsto (x, x, \dots).
\]


\begin{definition}
Let $K$ be a field, and let $L/K$ be a finite field extension.
We say that $L/K$ is unramified if:
\begin{enumerate}
    \item $L/K$ is separable, and
    \item The extension of valuations at a given nonarchimedean place is unramified.
\end{enumerate}
\end{definition}



To make this precise, we need a valuation on $K$, say given by a discrete valuation ring (DVR) $\mathcal{O}_K \subset K$, with maximal ideal $\mathfrak{m}_K$ and residue field $k = \mathcal{O}_K/\mathfrak{m}_K$.
Then the valuation extends to $L$, and we can talk about the integral closure $\mathcal{O}_L$ of $\mathcal{O}_K$ in $L$.

\begin{definition}[Field-theoretic]
We say that the extension $L/K$ is unramified at $\mathfrak{m}_K$ if:
\begin{itemize}
    \item $\mathcal{O}_L$ is a finite étale $\mathcal{O}_K$-algebra (i.e., it is finite, flat, and unramified as an $\mathcal{O}_K$-algebra), and
    \item the residue field extension $\mathcal{O}_L/\mathfrak{m}_L \, / \, \mathcal{O}_K/\mathfrak{m}_K$ is separable and of degree equal to the field extension, i.e., no ``defect''.
\end{itemize}
\end{definition}

This is equivalent to:
\begin{itemize}
    \item No ramification of the valuation (valuation extends uniquely),
    \item Unchanged value group, and
    \item Unramified residue field extension.
\end{itemize}

So, unramified roughly means: the extension doesn't introduce any infinitesimal thickening, and behaves nicely in both the value group and the residue field.



Let $L$ be a number field and $\mathcal{R}_L$ be the ring of integers of $L$. Then given
any prime ideal $(p)$ in $\mathbb{Z}$, we get a unique primary decomposition $(p)\mathcal{R}_L =
\beta_1^{e_1} \cdots \beta_r^{e_r}$.

Note that the ideals $\beta_k$ are precisely the prime ideals that lie over $(p)$, i.e.
the prime ideals such that $\beta_k \cap \mathcal{R}_L = (p)$.

The special properties were just that $\mathbb{Z}$ was a Dedekind domain and $\mathbb{Q}$ is
its fraction field, $L$ a finite extension of $\mathbb{Q}$ and $\mathcal{R}_L$ the integral closure of $\mathbb{Z}$
in $L$.

Let $D$ be any Dedekind domain with fraction field $K$. Let $L/K$ be a
finite separable field extension and $\mathcal{R}$ the integral closure of $D$ in $L$. Then $\mathcal{R}$ is again a Dedekind domain, and so given any prime $\mathfrak{p} \subset D$, the ideal $\mathfrak{p}\mathcal{R}$ has a primary decomposition $\beta_1^{e_1} \cdots \beta_r^{e_r}$. The situation is the same as above when $D = \mathbb{Z}$ and $K = \mathbb{Q}$.

The number $e_k$ is called the \textbf{ramification index} of the prime $\beta_k$.

Ramified means that the prime $\beta_k$ is not a product of distinct primes, i.e. $e_k > 1$.

Let $D$ be a Dedekind domain, the main example in mind is the ring of integers $\mathcal{R}_L$ of a number field $L$. Then the localization of $D$ at a prime ideal $\mathfrak{p}$ is a discrete valuation ring (DVR) with maximal ideal $\mathfrak{p}D_\mathfrak{p}$. The global object, $\mathcal{R}_L$, is recovered by intersecting all of the local rings \begin{align*}
    \mathcal{R}_L = \bigcap_{\mathfrak{p} \subset D} \mathcal{R}_{L,\mathfrak{p}}.
\end{align*}
In essence, the valuation ring (elements with norm $ \leq 1$ ) is the local analog of the global ring of integers $\mathcal{O}_K$ in number theory. Each prime ideal $\mathfrak{p} \subset \mathcal{O}_K$ gives a valuation, and the corresponding valuation ring is $\mathcal{O}{K,\mathfrak{p}}$, the localization of $\mathcal{O}_K$ at $\mathfrak{p}$.

Given a finite seperable field extension $L/K$, we can extend the valuation $v$ on $K$ to a valuation $w$ on $L$. This process involves understanding the following:

\begin{itemize}
    \item Does $v$ extend to a valuation $w$ on $L$?
    \item What does the extended valuation $w$ look like?
    \item What are the associated valuation ring $\mathcal{O}_L$, residue field $k_L$, and value group?
\end{itemize}

The extension of the valuation introduces three invariants:
\begin{itemize}
    \item \textbf{Ramification index:} $e := w(\pi_K)$, where $\pi_K$ is a uniformizer of $\mathcal{O}_K$.
    \item \textbf{Residue field degree:} $f := [k_L : k_K]$, where $k_K = \mathcal{O}_K / \mathfrak{m}_K$ and $k_L = \mathcal{O}_L / \mathfrak{m}_L$ are the residue fields.
    \item \textbf{Number of distinct extensions:} $g$, the number of distinct valuations $w$ of $L$ lying over $v$.
\end{itemize}

These invariants satisfy the inequality:
\[
[L : K] \geq efg,
\]
with equality holding if the extension is well-behaved (e.g., separable and without defect).



Let $K$ be a field with a discrete valuation $v$, and suppose $L/K$ is a finite extension. We say that $L/K$ is unramified at $v$ if:
\begin{itemize}
    \item $L/K$ is separable, and
    \item The extension of valuation rings $\mathcal{O}_K \subset \mathcal{O}_L$ is unramified, which means:
    \begin{itemize}
        \item the ramification index is $e = 1$,
        \item the residue field extension is separable, and
        \item the residue degree $f = [k_L : k_K]$ satisfies $[L : K] = f$.
    \end{itemize}
\end{itemize}

Equivalently, $L/K$ is unramified if:
\begin{itemize}
    \item The valuation $v$ extends uniquely to $L$,
    \item The extension does not introduce infinitesimal or wild behavior,
    \item The Kähler differentials $\Omega_{\mathcal{O}_L / \mathcal{O}_K} = 0$.
\end{itemize}

\section{Apr 11}
\subsection{Nodal curves and dualizing sheaves}
\begin{definition}
    If $\Sigma$ is a nodal curve, its normalization $\tilde \Sigma$ is the Riemann surface obtained by ungluing its nodes. Specifically, the normalization is defined as the unique integral normal scheme $\tilde \Sigma$ with a birational morphism $\tilde \Sigma \to \Sigma$. Locally around a node $\Spec k[x,y]/(xy)$, the normalization is given by the map \begin{align*}
        \Spec k[x,y]/(x) \coprod \Spec k[x,y]/(y) \to \Spec k[x,y]/(xy)
    \end{align*}
In general, locally the normalization map looks like \begin{align*}
    \Spec(\prod R/(f_i,g)) \to \Spec(R/(f_1\cdot f_2 \cdots f_n, g))
\end{align*}
The inverse image of the node points are known as the branch points of the normalization.
\end{definition}

\begin{definition}
    The arithmetic genus of a curve defined as $p_a(\Sigma) = 1 - \chi(\mathcal{O}_\Sigma)$. This coincides with the genus of the surface one obtains by smoothing the nodes of the curve. 

    The geometric genus of a curve is defined as the arithmetic genus of its normalization. It is denoted by $p_g(\Sigma) = p_a(\tilde \Sigma)$.
\end{definition}

A genus $g$ non-singular curve has a $g$ dimensional space of holomorphic differentials. 
\begin{definition}
    Define a differential on a nodal curve $\Sigma$ to be a meromorphic differential $\omega$ on each component so that \begin{enumerate}
        \item $\omega$ is holomorphic away from the nodes,
        \item $\omega$ has a pole of order at most $1$ at each node, and
        \item the residues of $\omega$ at the two node-branches corresponding to a given node add to zero.
    \end{enumerate}
\end{definition}
\begin{definition}
    
The \textbf{dualizing sheaf} $\omega_\Sigma$ of a nodal curve $\Sigma$ is defined as the sheaf of meromorphic differentials on $\Sigma$. The dualizing sheaf is a line bundle on $\Sigma$.

\end{definition}

On a smooth curve, the sheaf of differentials $\Omega^1_C$ is a line bundle, and the canonical sheaf $\omega_C$ canonically identifies with $\Omega^1_C$. But for a singular curve, $\Omega^1_C$ is not locally free at the nodes (it can have torsion).

\begin{example}
    Consider the nodal curve $C = \Spec k[x,y]/(xy)$, which has a single node at the origin. We can compute the module of differentials $\Omega^1_C$ as follows:
    \begin{align*}
        \Omega^1_C &= \frac{\mathcal{O}_C dx + \mathcal{O}_C dy}{\mathcal{O}_C d(xy)} \\
        &= \frac{\mathcal{O}_C dx + \mathcal{O}_C dy}{\mathcal{O}_C (y dx + x dy)}.
    \end{align*}
    This module is not locally free at the node, as it has torsion. In fact we have the relation \begin{align*}
        x \cdot (xdy) = x^2 dy = -xy dx = 0
    \end{align*}
    So $\Omega^1_C$ is not locally free and cannot be a line bundle.
\end{example}


Instead, one works with the dualizing sheaf $\omega_C$, which is a line bundle on nodal curves, and plays the role of "canonical sheaf" in the singular setting. So when we talk about "canonical divisors" or "differentials" on a nodal curve, we really mean sections of the dualizing sheaf. 

Let $X$ be a smooth projective variety over a field $k$, and let $\mathcal{F}$ be a locally free sheaf (e.g., a vector bundle) on $X$. Then we have:
\[
\operatorname{Ext}^i_X(\mathcal{F}, \omega_X) \cong H^i(X, \omega_X \otimes \mathcal{F}^\vee).
\]
In particular, Ext can be computed as cohomology of internal Hom sheaves when one of the arguments is locally free. Serre duality is the statement that there is a coherent dualizing sheaf $\omega_X$ and perfect pairing \begin{align*}
    \operatorname{Ext}^i_X(\mathcal{F}, \omega_X) \times H^{n-i}(X, \mathcal{F}) \to H^n(X, \omega_X) \cong k
\end{align*} giving rise to
\begin{align*}
    \operatorname{Ext}^i_X(\mathcal{F}, \omega_X) &\cong H^{n-i}(X, \mathcal{F}) \\
\end{align*}

\begin{definition}
A complex $\omega_X^\bullet\in D^b_{\operatorname{coh}}(X)$ is a \emph{dualizing complex} if the functor
\[
\mathbf R\!\mathcal Hom_X(-,\omega_X^\bullet):D^b_{\operatorname{coh}}(X)\longrightarrow D^b_{\operatorname{coh}}(X)
\]
induces a duality on $D^b_{\operatorname{coh}}(X)$, i.e.\ the canonical map
\[
A\longrightarrow \mathbf R\!\mathcal Hom_X(\mathbf R\!\mathcal Hom_X(A,\omega_X^\bullet),\omega_X^\bullet)
\]
is an isomorphism for all $A\in D^b_{\operatorname{coh}}(X)$.
\end{definition}

\begin{example}
If $X$ is smooth of pure dimension $n$, then $\omega_X^\bullet\simeq\omega_X[n]$, where $\omega_X=\Omega^n_{X/k}$ is the canonical line bundle and $[n]$ denotes a cohomological shift by $n$ to the left.
\end{example}

Let $f:X\to Y$ be a proper morphism of noetherian schemes. \begin{theorem}[Grothendieck--Verdier Duality]
There exists a right adjoint
\[
f^{!}:D^{+}_{\operatorname{qc}}(Y)\longrightarrow D^{+}_{\operatorname{qc}}(X)
\]
to the derived push‑forward $\mathbf R f_*$.  Restricted to $D^b_{\operatorname{coh}}$ we have an adjunction
\[
\operatorname{Hom}_Y(\mathbf R f_*A,B)\;\cong\;\operatorname{Hom}_X(A,f^{!}B),
\qquad
A\in D^b_{\operatorname{coh}}(X),\;B\in D^b_{\operatorname{coh}}(Y).
\]
\end{theorem}

When $Y=\operatorname{Spec}k$ the object $f^{!}k$ is a dualizing complex on $X$.

\subsection{Behavior under smooth morphisms}
If $f$ is smooth of relative dimension $d$, then
\[
f^{!}B\simeq f^*B\otimes\Omega^d_{X/Y}[d].
\]
In particular, for the structure morphism $X\to\operatorname{Spec}k$ with $X$ smooth of dimension $n$ we recover $f^{!}k\simeq\omega_X[n]$.

Applying the adjunction to $f:X\to\operatorname{Spec}k$ yields a counit
\[
\epsilon:\mathbf R f_*f^{!}k\longrightarrow k.
\]
For smooth $X$ this becomes
\[
\epsilon: \mathbf R\Gamma(X,\omega_X[n])\longrightarrow k,
\]
and taking $H^0$ gives a $k$‑linear map
\[
\operatorname{tr}_X:H^{n}(X,\omega_X)\longrightarrow k,
\]
called the \emph{trace} or \emph{integration} map.

Locally, $\operatorname{tr}_X$ can be expressed as a sum of Grothendieck residues:
\[
\operatorname{tr}_X(\alpha)=\sum_{x\in X}\operatorname{Res}_{x}(\alpha),\qquad\alpha\in H^{n}(X,\omega_X).
\]
where we interpret $\operatorname{Res}_{x}(\alpha)$ as the residue of the meromorphic differential $\alpha$ at the point $x$. Locally we can write $\alpha$ as a meromorphic differential on the local ring $\mathcal{O}_{X,x}$, and the residue is computed as the coefficient of $dz/z$ in the Laurent series expansion of $\alpha$ around $x$.

One recovers the classical statement of Serre Ext-Hom duality for smooth projective varieties over an algebraically closed field $k$. Take $f:X\to\operatorname{Spec}k$ and set $B=k$ in the adjunction:
\[
\mathbf R\Gamma(X,\mathcal F)\;\cong\;\mathbf R f_*\mathcal F
\quad\Longrightarrow\quad
\mathbf R\!\mathcal Hom_k(\mathbf R\Gamma(X,\mathcal F),k)
\;\cong\;
\mathbf R\!\mathcal Hom_X(\mathcal F,f^{!}k).
\]
For smooth $X$, substitute $f^{!}k=\omega_X[n]$ and compare cohomology in degree $n-i$ to obtain the claimed isomorphisms.
\begin{theorem}[Serre Duality]
Let $X$ be a smooth projective variety of dimension $n$ over an algebraically closed field $k$ and $\mathcal F$ a coherent sheaf on $X$.  There is a functorial perfect pairing
\[
H^{i}(X,\mathcal F)\times H^{n-i}(X,\mathcal F^{\vee}\otimes\omega_X)\;\longrightarrow\;k,
\]
given by the cup product followed by $\operatorname{tr}_X$.  Consequently
\[
H^{i}(X,\mathcal F)\;\cong\; \operatorname{Hom}_k\!\bigl(H^{n-i}(X,\mathcal F^{\vee}\otimes\omega_X),k\bigr).
\]
\end{theorem}

So what we are saying that if $\pi:\tilde \Sigma \to \Sigma$ is the normalization, then the dualizing sheaf $\omega_\Sigma$ can be identified $\pi^*\omega_{\Sigma} \subset \omega_{\widetilde{\Sigma}}(p + q)$, where $p$ and $q$ are the two points in the normalization that correspond to the node.
\begin{align*} 
    \omega_\Sigma = \{ \eta \in \omega_{\widetilde{\Sigma}}(p + q) \mid \operatorname{res}_p(\eta) + \operatorname{res}_q(\eta) = 0 \}
\end{align*}

Choose analytic coordinates so that a node looks like
\[
\operatorname{Spec}\Bigl(\mathbb{C}[x,y]/(xy)\Bigr).
\]
On the normalization this is two discs with coordinates $x$ and $y$.
The Kähler differentials on the smooth pieces give meromorphic 1‑forms
$\frac{dx}{x} \quad \text{and} \quad \frac{-dy}{y}$
whose residues at the origins are $+1$ and $-1$.
The requirement ``residues add to 0'' is exactly the relation
$\frac{dx}{x} = -\frac{dy}{y}
\quad \text{in } \nu_{*}\bigl(\Omega^{1}_{\widetilde{\Sigma}}(q^{+} + q^{-})\bigr).
$

\subsection{Moduli of Riemann surfaces}
Riemann surfaces of genus $g>1$ have a well-behaved moduli space $\mathcal{M}_g$. it is the nonsingular DM-stack (think orbifold) of nonsingular genus $g$ curves up to isomorphism. Being smooth, we can talk about its dimension, and it turns out that the dimension of $\mathcal{M}_g$ is $3g-3$.

In order to introduce the Deligne-Mumford compactification of $\mathcal{M}_g$, we need to introduce the notion of stable curves. Stability is equivalent to finite automorphism group.
\begin{definition}
    A \textbf{stable curve} is a connected nodal curve so that \begin{enumerate}
        \item every irreducible component of geometric genus zero has at least three node branches,
        \item every irreducible component of geometric genus one has at least one node branch. 
    \end{enumerate}
\end{definition}

If $E \simeq \mathbb{P}^1$ carries fewer than three special points (node branches or marked points), then
    \[
    \operatorname{Aut}(\mathbb{P}^1, \Sigma_E) \supseteq
    \begin{cases}
        \mathrm{PGL}_2(k) & |\Sigma_E| = 0, \\[4pt]
        \text{affine line} & |\Sigma_E| = 1, \\[4pt]
        \text{diagonal torus} & |\Sigma_E| = 2,
    \end{cases}
    \]
    all of which are positive-dimensional. So the whole curve would acquire an infinite automorphism group, which prevents the existence of a separated, Deligne–Mumford moduli space.

A smooth genus-1 curve $E$ has a 1-dimensional automorphism group $E$ acting by translations. Fixing one branch (or marked point) kills that translation freedom, leaving only the finite subgroup of automorphisms fixing the chosen point. Hence “at least one branch” plays for genus 1 the same role that “three branches” plays for genus 0.

For components of genus $\geq 2$, no condition is necessary—their automorphism groups are already finite by Hurwitz's theorem.

\begin{theorem}
    [Hurwitz] The automorphism group of a smooth projective curve of genus $g \geq 2$ is at most $84(g-1)$.
\end{theorem}

$\mathcal{M}_g$ is not compact because we can visualize degenerations where the limit is nodal. In fact there are degenerations in which the limit curve is no longer stable (as it can pick up a component with only one node-branch). However, it turns out that one can replace the limit of the degenerating family with a stable curve, and any such family has exactly one stable curve as a limit. This is the idea behind the Deligne-Mumford compactification of $\mathcal{M}_g$.

\subsection{Compactification of $\mathcal{M}_g$}

\begin{definition}
    An $n$-pointed curve is a nodal curve with $n$ distinct labelled nonsingular points. A special point of a component of a pointed curve is a point on the normalization of the component that is either a node-branch or the preimage of a marked point. 

    A pointed curve is stable if the following conditions hold:
    \begin{enumerate}
        \item Every irreducible component of geometric genus zero has at least three special points.
        \item Every irreducible component of geometric genus one has at least one special point.
    \end{enumerate}
\end{definition}
The set of isomorphism classes of stable $n$-pointed curves of genus $g$ is denoted $\overline{M}_{g,n}$. It is irreducible and has as an open set the moduli space of smooth curves $\mathcal{M}_{g,n}$. Also \begin{align*}
    \dim \overline{M}_{g,n} = 3g - 3 + n
\end{align*}
There is a forgetful morphism whenever $n_1 \geq n_2$ and $2g - 2 + n_2 > 0$: \begin{align*}
    \pi_{n_1,n_2} : \overline{M}_{g,n_1} \to \overline{M}_{g,n_2}
\end{align*} which can be identified with the universal curve when $n_1 = n_2 + 1$.


\subsection{Moduli spaces of stable maps}
\begin{definition}
    Let $X$ be a smooth projective variety. A morphism $f$ from a pointed nodal curve to $X$ is a \textbf{stable map }if every genus zero contracted component of $\Sigma$ (where contracted means that the image of the component is a point) has at least three special points, and every genus one contracted component has at least one special point.
\end{definition}

\begin{definition}
    A stable map represenents a homology class $\beta \in H_2(X,\Z)$ if $f_*[C] = \beta$ in $H_2(X,\Z)$. The moduli space of stable maps to $X$ is denoted $\overline{M}_{g,n}(X,\beta)$, and it parametrizes stable maps from $n$-pointed curves of genus $g$ to $X$ representing the homology class $\beta$.
\end{definition}

The moduli space $\overline{M}_{g,n}(X,\beta)$ is a Deligne–Mumford stack.

\subsection{Geometric Langlands Correspondence: Nick Rozenblum}
Weil's Rosetta Stone:
\begin{enumerate}
    \item Number theory: \begin{itemize}
        \item Number field $k$
        \item Reciprocity laws (e.g. Gauss's quadratic reciprocity law for if $p$ is a square mod $q$). Have to be careful about how to generalize this since the question fails for cubes, in particular some extension fails to be abelian. Quadratic reciprocity is about ramification at those two primes, but in the case considered by GLC, we are really talking about the global unramified case.
        \item Gives rise to abelian reciprocity laws "class field theory", the upshout is that we understand $\Gal(k)^{ab}$
    \end{itemize}
    \item Geometry over $\F_q$: \begin{itemize}
        \item algebraic curves over $\F_q$:
    \end{itemize}
    \item Complex geoemtry\begin{itemize}
        \item compact Riemann surface
    \end{itemize}
\end{enumerate}
Studying $\Gal(k)^{\ab}$ is about studying $1$-dimensional representatinos of $\Gal(k)$. The idea of Langlands is to study higher dimensional representations of $\Gal(k)$. Really, representations with $G$ reductive \begin{align*}
    \Gal(k) \to G
\end{align*} are talking about representations respecting some additional structure which is important (symplectic form, etc). Reductive groups are classified the same as compact Lie groups, by combinatorial data (which carries natural duality and gives another reductive group $G^\vee$). It is a very strange passage, which is not group theoretic or geometric. Apriori, the groups don't have much to do with each other.
\begin{example}
    $\GL_n$ is self dual. The dual of $\SL_n$ is $\PGL_n$. The dual of $\SO(2n+1)^\vee \cong \Sp(2n)$.
\end{example}
There is an important automorphism of a root system known as the Cartan involution.
Langlands proposed that representations $\Gal(k) \to G^\vee$ should correspond to automorphic forms for $G$.

Rather than talking about automorphic forms for number fields, we can give an interpretation just for smooth projective curves $X$ over $\F_q$, also due to Weil. In this context $G$-automorphic forms is the same thing as functions $\Bun_G(X)(\F_q)$ the moduli stack of principal $G$-bundles. This is a finite groupoid, this moduli stack is not finite type, so we should take compactly supported functions on $\pi_0$.

In number theory, there is an interpretation in terms of an adelic double quotient, and Weil noticed that passing to this quotient is the same as looking at the $\F_q$ points. The description that makes sense for number fields \begin{align*}
    \text{functions on } G(K) \backslash G(\A) / G(\cO)
\end{align*} where $\A$ is the adeles and $\cO$ is the integral adeles. The Langlands correspondense should be compatible with the Hecke action. If $x\in X(\F_q)$, there is the Satake isomorphism which implies \begin{align*}
    \Rep(G^\vee) = K_0(\Rep(G^\vee)) \text{ acts on } \Autom_G
\end{align*} one for each rational point. Explicitly this action compes from the Hecke stack/correspondense \begin{align*}
    \Hecke_G(X) \to \Bun_G(X) \times \Bun_G(X)
\end{align*} which is a groupoid of correspondenses because we have composition. Moreover, the functions on the Hecke stack carry an algebra structure given by convolution. Satake says that there is an algebra map \begin{align*}
    K_0(\Rep(G^\vee)) \to \text{functions on } \Hecke_G(X)
\end{align*} which can be identified with automorphisms of the space of automorphic forms.


The idea is to simultaneously diagonalize this action. Common eigenvalues are parametrized by representations $\pi_1^{\text{arith}} \to G^\vee$ (the arithmetic fundamental group). 

How to produce automorphic functions? Categorification. 

The natural object to consider is $\ell$-adic sheaves on $\Bun_G$ which one should think of as continuously varying families of vector spaces with respect to geometry over finite fields.

\begin{remark}
    [Interlude - functions sheaves correspondence] Let $Y/\F_q$ variety and consider the variety $\bar Y = Y \times_{\F_q} \bar \F_q$ is a variety over $\bar\F_q$ given by taking the coefficients and regarding them as over $\bar \F_q$. This variety carries an action of Frobenius on $\bar Y$ and the fixed points of the action of Frobenius is precisely the $\F_q$ points of the variety. \begin{align*}
        \bar{Y}^{\text{Frob}} = Y(\F_q)
    \end{align*}
\end{remark}
Given an $\ell$-adic sheaf on $Y$, suppose that $y\in Y(\F_q) = \bar{Y}^{\text{Frob}}$, then $\cF \cong \Frob_*\cF$ and on stalks $\alpha_y: \cF_y \cong \cF_{\Frob(y)}$ = $\cF_y$. So given an $\ell$-adic $\cF$ we get a function $f_{\mathcal{F}}$ on all of the rational points given by \begin{align*}
    f_{\mathcal{F}}(y) = \tr(\alpha_y)
\end{align*}
Given a representation $\rho: \pi_1^{\text{arith}} \to G^\vee$, we want to construct an $\ell$-adic sheaf on $\Bun_G$. This is what Drinfeld did for $\GL_2$, constructed automorphic forms by constructing $\ell$-adic sheaves. Not so clear what an automorphic form is for a Riemann surface, but the above goal makes sense for complex geometry.

\begin{remark}
    What is an $\ell$-adic sheaf? I should get an $\Q_\ell$-vector space for every etale open, and instead of $\Q_\ell$ vector spaces, I really want a complex. 
\end{remark}

Now let $X$ be a compact Riemann surface. Consider a representation $\pi_1 X \to G^\vee$ construct a sheaf on stack $\Bun_G(X)$. But now by sheaf, we really mean $D$-module, which is about specifying functions by the linear differential equations which they satisfy. 

Over the complex numbers, the new feature that we have is that the representations $\pi_1: X\to G^\vee$ fit into a moduli space, a finite type algebraic stack. Two distinct versions, the Betti version (which can be presentated as an affine variety quotient a group, ala taking generators and relations of $\pi_1 X$) and the de Rham version (parametrizing $G$-bundles with a flat connection) $LS_{G^\vee}$. 

They have the same points but their stack structure is different because they fit differently in families. For example the Betti version has global functions where as the de Rham version has no global functions. 

Conjecture: Beilinson and Drinfeld suggested that the entire category of $\cD$-modules on $\Bun_G$ are the same thing as ind-coherent sheaves on $LS_{G^\vee}$ which are nilpotent, $\cong \QCoh(LS_{G^\vee})$.

For $\sigma \in LS_{G^\vee}$ there is a corresponding skyscraper sheaf $\cF_\sigma$ which goes to the $\cD$-module side and lands in "eigensheaves."

\section{Apr 13}

\subsection{The Mittag-Leffler Problem}

Let $S$ be a Riemann surface, not necessarily compact, $p$ a point of $S$ with local coordinate $z$ centered at $p$. A \emph{principal part} at $p$ is the polar part $\sum_{k=1}^{n} a_k z^{-k}$ of a Laurent series. If $\mathcal{O}_p$ is the local ring of holomorphic functions around $p$, $\mathcal{M}_p$ the field of meromorphic functions around $p$, a principal part is just an element of the quotient group $\mathcal{M}_p/\mathcal{O}_p$. The \emph{Mittag-Leffler question} is, given a discrete set $\{p_n\}$ of points in $S$ and a principal part at $p_n$ for each $n$, does there exist a meromorphic function $f$ on $S$, holomorphic outside $\{p_n\}$, whose principal part at each $p_n$ is the one specified? The question is clearly trivial locally, and so the problem is one of passage from local to global data. Here are two approaches, both of which lead to cohomology theories.

\textbf{\v{C}ech.} Take a covering $\underline{U} = \{U_\alpha\}$ of $S$ by open sets such that each $U_\alpha$ contains at most one point $p_n$, and let $f_\alpha$ be a meromorphic function on $U_\alpha$ solving the problem in $U_\alpha$. Set
\[
f_{\alpha\beta} = f_\alpha - f_\beta \in \mathcal{O}(U_\alpha \cap U_\beta).
\]

In $U_\alpha \cap U_\beta \cap U_\gamma$, we have
\[
f_{\alpha\beta} + f_{\beta\gamma} + f_{\gamma\alpha} = 0.
\]

Solving the problem globally is equivalent to finding $\{g_\alpha \in \mathcal{O}(U_\alpha)\}$ such that
\[
f_{\alpha\beta} = g_\beta - g_\alpha \quad \text{in} \quad U_\alpha \cap U_\beta;
\]

given such $g_\alpha$, $f = f_\alpha + g_\alpha$ is a globally defined function satisfying the conditions, and conversely. In the \v{C}ech theory,
\begin{align*}
\{\{f_{\alpha\beta}\} : f_{\alpha\beta} + f_{\beta\gamma} + f_{\gamma\alpha} = 0\} &= Z^1(\underline{U}, \mathcal{O}) \\
\{\{f_{\alpha\beta}\} : f_{\alpha\beta} = g_\beta - g_\alpha, \text{ some } \{g_\alpha\}\} &= \delta C^0(\underline{U}, \mathcal{O})
\end{align*}

and the \emph{first \v{C}ech cohomology group}
\[
H^1(\underline{U}, \mathcal{O}) = \frac{Z^1(\underline{U}, \mathcal{O})}{B^1(\underline{U}, \mathcal{O})}
\]
is the obstruction to solving the problem in general.

\textbf{Dolbeault.} As before, take $f_\alpha$ to be a local solution in $U_\alpha$ and let $\rho_\alpha$ be a bump function, 1 in a neighborhood of $p_n \in U_\alpha$ and having compact support contained in $U_\alpha$. Then
\[
\varphi = \sum_\alpha \bar{\partial}(\rho_\alpha f_\alpha)
\]
is a $\bar{\partial}$-closed $C^\infty$ (0,1)-form on $S$ ($\varphi \equiv 0$ in a neighborhood of $p_n$). If $\varphi = \bar{\partial}\eta$ for $\eta \in C^\infty(S)$, then the function
\[
f = \sum_\alpha \rho_\alpha f_\alpha - \eta
\]
satisfies the conditions of the problem. In particular, $f - \sum_\alpha \rho_\alpha f_\alpha$ is everywhere holomorphic because we are exactly cancelling out the poles. Thus the obstruction to solving the problem is in $H_{\bar{\partial}}^{0,1}(S)$.

\subsection{Residues}
First we recall some important facts. Let $X$ be a compact Riemann surface.
\begin{theorem}[Global residue theorem]\label{thm:global-residue}
    Let $\omega$ be a meromorphic differential (meromorphic $1$--form) on $X$ whose poles form a finite set $P \subset X$.  For each $p\in P$ choose a local holomorphic coordinate $z$ centred at $p$ and write the Laurent expansion
    \[
      \omega\;=\;\Bigl(\tfrac{a_{-1}}{z}+a_0+a_1z+\cdots\Bigr)\,dz\quad\text{on }\{|z|<\varepsilon\}.
    \]
    The coefficient $\operatorname{Res}_{p}(\omega):=a_{-1}$ is called the \emph{residue of $\omega$ at $p$}.  Then
    \[
      \boxed{\displaystyle \sum_{p\in P} \operatorname{Res}_{p}(\omega)=0 }.
    \]
    \end{theorem}

\subsection{Analytic interpretation}

    Let $p\in P$.  In a coordinate neighbourhood $(U,z)$ centred at $p$ we have $\omega = f(z)\,dz$ with $f$ meromorphic.  The residue is $\operatorname{Res}_{p}(\omega)=\operatorname{Res}_{z=0}f$.
    

    Choose $\varepsilon>0$ small enough that the closed discs $D_p:=\{|z|\le\varepsilon\}\subset U$ are pairwise disjoint and contain no other poles.  Define
    \[
      X_{\varepsilon}:= X\setminus \bigcup_{p\in P} D_p^{\circ},\qquad \partial X_{\varepsilon}=\bigcup_{p\in P} \partial D_p.
    \]
    Each boundary component $\partial D_p$ is oriented positively with respect to $X$.
    

    Because $\omega$ is \emph{holomorphic} on the manifold with boundary $X_{\varepsilon}$ we have $d\omega=0$ there.  Stokes' theorem gives
    \[
      0 = \int_{X_{\varepsilon}} d\omega = \int_{\partial X_{\varepsilon}} \omega = \sum_{p\in P} \int_{\partial D_p} \omega.
    \]
    In local coordinates
    \[
      \int_{\partial D_p}\!\omega = \int_{|z|=\varepsilon} \Bigl(\tfrac{a_{-1}}{z}+\text{holomorphic}\Bigr)dz = 2\pi i\,a_{-1} = 2\pi i\,\operatorname{Res}_{p}(\omega).
    \]
    \subsection{Sheaf--cohomological interpretation}
    Let $D=\sum_{p\in P}n_p\,[p]$ be an effective divisor with $n_p\ge1$ so that $K_X(D)$ contains all meromorphic differentials having poles bounded by~$D$.  There is an exact sequence of sheaves
    \begin{equation}\label{eq:res-seq}
      0\;\longrightarrow\;K_X\;\longrightarrow\;K_X(D)\;\xrightarrow{\;\rho\;}\;\bigoplus_{p\in P}\mathbb C_{p}\;\longrightarrow 0,
    \end{equation}
    where $\rho$ is the \emph{residue map} sending a meromorphic differential to the collection of its residues.
    
    Taking global sections of~\eqref{eq:res-seq} gives a long exact sequence
    \[
      0 \longrightarrow H^{0}(X,K_X) \longrightarrow H^{0}(X,K_X(D)) \xrightarrow{\;\rho\;} \bigoplus_{p\in P}\mathbb C \xrightarrow{\;\delta\;} H^{1}(X,K_X) \longrightarrow \cdots.
    \]
    By Serre duality $H^{1}(X,K_X) \cong H^{0}(X,\mathcal O_X)^{*}$.  Since $X$ is compact and connected, $H^{0}(X,\mathcal O_X)=\mathbb C\cdot1$.  Under this identification the connecting map $\delta$ evaluates a vector $(c_p)_{p\in P}$ by
    \[
      \delta\bigl((c_p)_{p}\bigr)(1)\;=\;\sum_{p\in P} c_p.
    \]
    Thus $\delta$ is precisely the \emph{sum of residues}. Therefore, by exactness at $\bigoplus_{p\in P}\mathbb C$ the image of $\rho$ is the kernel of $\delta$, i.e. any meromorphic differential $\omega$ satisfies $\delta(\rho(\omega))=0$, i.e.
    \[
      \sum_{p\in P}\operatorname{Res}_{p}(\omega)=0.
    \]
    \begin{remark}
    Cohomologically, the theorem reflects the fact that the canonical bundle $K_X$ has degree $2g-2$ and hence admits no non-trivial divisors of degree $1$; the residues form such a divisor of degree~$0$, forcing the only relation among them.
    \end{remark}
    
    %=====================================================================
    \subsection{Examples and further remarks}
    \begin{example}[Logarithmic derivative]
    Let $f$ be a non--constant meromorphic function on $X$.  Its logarithmic derivative $\omega = d\!\log f = f'/f\,dz$ has simple poles at the zeros and poles of~$f$ with residues equal to their multiplicities (positive for zeros, negative for poles).  The global residue theorem recovers the classical identity
    \[
      \sum_{a:\,f(a)=0} \operatorname{ord}_{a}(f) \;=\; \sum_{b:\,f(b)=\infty} \operatorname{ord}_{b}(f),
    \]
    expressing that the divisor of a meromorphic function has degree~$0$.
    \end{example}
    


\begin{proposition}
    The dimension of the space $\mathcal{L}(K)$ of holomorphic $1$-forms is exactly the genus $g$. Moreover, there are no common zeros of the canonical line bundle $\mathcal{L}(K)$, so there is a canonical map $X \to \mathbb{P}^{g-1}$ associated to the linear series of all $1$-forms.
    \end{proposition}


\subsection{Dolbeault's theorem} \label{sec:dolbeault}

    Dolbeault's theorem is a complex analog of de Rham's theorem. It asserts that the Dolbeault cohomology is isomorphic to the sheaf cohomology of the sheaf of holomorphic differential forms. Specifically,
    \[
    H^{p,q}(M) \cong H^q(M, \Omega^p)
    \]
    where $\Omega^p$ is the sheaf of holomorphic $p$ forms on $M$.
    
    A version of the Dolbeault theorem also holds for Dolbeault cohomology with coefficients in a holomorphic vector bundle $E$. Namely one has an isomorphism
    \[
    H^{p,q}(M, E) \cong H^q(M, \Omega^p \otimes E).
    \]
This interpretation is important for our forthcoming discussion of Serre duality on Riemann surfaces.

\subsection{Serre duality}

Serre duality on a compact Riemann surface X of genus g is a canonical isomorphism of complex vector spaces:
\[H^1(X, K_X) \cong H^0(X, \mathcal{O}_X)^{\ast}\]
More generally, Serre duality states that for any line bundle L on a compact Riemann surface X,

\[H^1(X, L) \cong H^0(X, K_X \otimes L^{-1})^{\ast}\]

In particular there is a perfect pairing $H^1(X,K_X)\times H^0(X,\mathcal{O}_X)\longrightarrow\mathbb{C}$ given by \begin{align*}
    H^1(X,K_X)\times H^0(X,\mathcal{O}_X) \to H^1(X, K_X) \to \C
\end{align*} where the second map is integration of the holomorphic $1$-form over the Riemann surfaec

Elements of $H^1(X, K_X)$ can be concretely represented by Čech 1-cocycles of sections of $K_X$. Given an open cover $\{U_\alpha\}$, a class in $H^1(X, K_X)$ is represented by a collection of holomorphic 1-forms defined on intersections $U_{\alpha\beta} = U_\alpha \cap U_\beta$, satisfying a cocycle condition. Such a cocycle measures how a collection of locally defined holomorphic 1-forms fails to glue to a global form.

Elements of $H^0(X, \mathcal{O}_X)$ are just global holomorphic functions. On a compact connected Riemann surface, these are simply constants, since any global holomorphic function is constant. Thus:
\[
H^0(X, \mathcal{O}_X) \cong \mathbb{C}.
\]
Concretely, the Serre duality pairing is given by: 
\[
([\{\omega_{\alpha\beta}\}], f) \mapsto \text{“integral of } f \cdot [\{\omega_{\alpha\beta}\}] \text{”}.
\]
This duality is defined abstractly by viewing cohomology groups as derived functors, and then using a duality statement at the level of sheaves. Concretely, though, the isomorphism is realized through a perfect pairing given explicitly by integration and residues.

What does it mean to integrate a Cech cocycle? This is where we pass to the Dolbeault interpretation. The subtlety is that Čech cocycles themselves represent cohomological obstructions. They measure precisely how local objects fail to glue globally. The integration pairing (e.g., Serre duality) is never literally performed directly on intersections, but instead we have to pass to representatives on a suitable open cover, extend with bump functions, and then integrate. The integration is performed on the global object, which is a holomorphic differential form.

We have a Čech cocycle $[\{\omega_{\alpha\beta}\}] \in H^1(X,K_X)$ representing an element in the sheaf cohomology group.
Explicitly, this cocycle is given by a family of holomorphic 1-forms:
\begin{itemize}
    \item $\omega_{\alpha\beta}$ is defined holomorphically on the intersection $U_{\alpha\beta}=U_{\alpha}\cap U_{\beta}$.
    \item The cocycle condition is:
    \[
    \omega_{\alpha\beta} + \omega_{\beta\gamma} + \omega_{\gamma\alpha}=0\quad\text{on}\quad U_{\alpha}\cap U_{\beta}\cap U_{\gamma}.
    \]
\end{itemize}

To get a Dolbeault representative (a global smooth $(1,1)$-form), we proceed as follows:
\begin{itemize}
    \item Choose local smooth sections $\omega_\alpha\in C^\infty(U_\alpha,K_X)$ so that:
    \[
    \omega_{\alpha\beta}=\omega_\beta-\omega_\alpha\quad\text{on}\quad U_{\alpha\beta}.
    \]
    This is always possible precisely because the cocycle condition holds (we’ll clarify this explicitly below).
    \item Choose a partition of unity $\{\rho_\alpha\}$ subordinate to the cover $\{U_\alpha\}$. Each $\rho_\alpha$ is smooth, supported inside $U_\alpha$, and satisfies:
    \[
    \sum_\alpha \rho_\alpha\equiv 1\quad\text{globally on } X.
    \]
\end{itemize}

Then define the form:
\[
\eta:=\sum_\alpha \bar{\partial}\rho_\alpha\wedge \omega_\alpha.
\]
This is the form that gets integrated in the Serre duality pairing with $H^0(X,\mathcal{O}_X)$. Expliclty we get \begin{align*}
    ([\{\omega_{\alpha\beta}\}], f) &= \int_X f\cdot \eta
\end{align*}
A different choice of local smooth sections $\omega_\alpha'$ satisfing the same cocycle data will differ from $\omega_\alpha$ by a global holomorphic differential $h$. If we study what happens to $\eta$ under this change, we find:
$$\eta{\prime} = \sum_\alpha \bar{\partial}\rho_\alpha \wedge (\omega_\alpha + h) = \eta + \sum_\alpha \bar{\partial}\rho_\alpha \wedge h$$
but since $h$ is a global holomorphic differential we can in fact factor it out of the sum:
$$\eta{\prime} = \eta + \left(\sum_\alpha \bar{\partial}\rho_\alpha\right)\wedge h$$
but then we see that the last term vanishes because $\sum_\alpha \bar{\partial}\rho_\alpha\equiv 0$ globally on $X$. This shows that the form $\eta$ is independent of the choice of local sections $\omega_\alpha$ and depends only on the Čech cocycle $[\{\omega_{\alpha\beta}\}]$.

This construction gives a global smooth $(1,1)$-form $\eta$ whose class in the Dolbeault cohomology group $H^{1,1}(X)$ corresponds to the corresponding Čech cocycle in the group $H^1(X,K_X)$.

\subsection{Discussion of the Riemann-Roch theorem}
    
    Given a divisor $D$, we're (as before) interested in the space $\mathcal{L}(D)$ of rational functions with the desired poles. Suppose for simplicity $D$ is a sum of distinct points, $D = p_1 + p_2 + \cdots + p_d$. The vector space $\mathcal{L}(D)$ consists of meromorphic functions that have at most a simple pole at the points $p_i$ but are otherwise regular. How might you define such a function?
    
    Choose a local coordinate $z_i$ around $p_i$; given a function $f \in \mathcal{L}(D)$, we can write it near $p_i$ as $\frac{a_i}{z_i} + f_0$ for some holomorphic $f_0$ (defined in a neighborhood of $p_i$), because $f$ is only allowed to have simple poles. This \textbf{polar part} $\frac{a_i}{z_i}$ says a lot about $f$. In fact, $f$ is determined \textbf{up to addition of scalars} by specifying these polar parts $\{a_i\}$, again because there are no nonconstant holomorphic functions on $X$. In other words, there is a natural map
    \[
    \mathcal{L}(D) \to \mathbb{C}^d,
    \]
    whose kernel consists of the constant functions. We get in particular,
    \[
    \ell(D) \leq 1 + d.
    \]    
    If you want to describe $\mathcal{L}(D)$ (and in particular its dimension), you want to know the image. This raises the question: \red{Given $a_1, \ldots, a_d$, when is there a global meromorphic $f : X \to \mathbb{P}^1$ with polar part $\frac{a_i}{z_i}$ at $p_i$, and holomorphic elsewhere?}
    
    In other words, we need to find \emph{constraints} on the $\{a_i\}$ for them to form a family of polar parts of a function in $\mathcal{L}(D)$. Here's the point: if $f \in \mathcal{L}(D)$, and $\omega$ is a \emph{holomorphic $1$-form}, we can consider the meromorphic differential $f\omega$. This has potentially simple poles at the $\{p_i\}$, but is holomorphic elsewhere. In particular, the sum of the residues is zero.
    
    So, if $\omega = g_i(z_i)dz_i$ locally, near $p_i$ (using the local coordinate $z_i$ around $p_i$), then the residue of $f\omega$ at $p_i$ is $a_i g_i(p_i)$. It follows that if the $\{a_i\}$ arise as a system of polar parts, then we want $\sum a_i g_i(p_i) = 0$. Thus the image of the map $\mathcal{L}(D) \to \mathbb{C}^d$ is contained in the \emph{orthogonal complement} of the space of holomorphic differentials (where each $\omega$ maps to $(g_i(p_i)) \in \mathbb{C}^d$ as before). We get a total of $g$ linear conditions on the $\{a_i\}$, over a family of holomorphic differentials, suggesting that
\begin{align*}
\ell(D)\;\le\;1+\deg D-\bigl(g-\ell(K-D)\bigr)
\end{align*}
Explicitly, we have the following maps:
\begin{align*}
\Phi_D:\;L(D)\;\longrightarrow\;\mathbf{C}^{\,d},
\qquad
f\;\longmapsto\;(a_1,\dots,a_d),
\end{align*}
where $a_i$ is the coefficient of the $\tfrac{1}{z_i}$ term of $f$ at $p_i$.
\begin{align*}
\Psi_D:\;H^0(X,K)\;\longrightarrow\;(\mathbf{C}^{\,d})^{\ast},
\qquad
\omega\;=\;g_i(z_i)\,dz_i\;\longmapsto\;
\bigl(v\mapsto\sum_{i=1}^{d}v_i\,g_i(p_i)\bigr).
\end{align*}
The map $\Psi_D$ simply records the values $\bigl(g_i(p_i)\bigr)$ of the holomorphic differential at the chosen points, so its rank is the number of linearly independent conditions it imposes on a polar-part vector. A holomorphic differential $\omega$ lies in $\ker\Psi_D$ iff it vanishes at every point $p_i$.
That means $\omega\in H^0\bigl(X,K(-D)\bigr)$, so $\dim\ker\Psi_D=\ell(K-D)$.
Since $\dim H^0(X,K)=g$, we have
\begin{align*}
\operatorname{dim}\operatorname{im}\Psi_D
\;=\;
g-\ell(K-D).
\end{align*}
Those are exactly the independent linear relations that the residues
$\sum_i a_i g_i(p_i)=0 $ impose on $ (a_i).$
Because every relation coming from \(\Psi_D\) cuts down the image of \(\Phi_D\) by at most one dimension,
\begin{align*}
\dim\operatorname{Im}\Phi_D
\;\le\;
d-\bigl(g-\ell(K-D)\bigr).
\end{align*}
Finally, \(\ker\Phi_D\) consists of the constant functions, so
\(\dim\ker\Phi_D=1\). Therefore
\begin{align*}
\ell(D)
=\dim L(D)
=\dim\ker\Phi_D+\dim\operatorname{Im}\Phi_D
\;\le\;
1+d-\bigl(g-\ell(K-D)\bigr),
\end{align*}
However, a 1-form is the dual of a vector field. So the degree of a 1-form is just minus the degree of the corresponding vector field, and that is the topological Euler characteristic by the Poincare-Hopf theorem.In particular, the degree of a 1-form is the opposite of the topological Euler characteristic, i.e. $2g - 2$. So the degree of $K-D$ is equal to $2g-2-d$. Letting $D = K - D$ gives \begin{align*}
    \ell(K - D) \leq 1 + (2g - 2 - d) - (g - \ell(D)) 
\end{align*} Adding these two inequalities gives \begin{align*}
    \ell(D) + \ell(K - D) \leq \ell(D) + \ell(K - D) 
\end{align*} and we can conclude that both of the inequalities that we added are in fact equalities. Therefore, we can conclude that \begin{align*}
    \ell(D) = 1 + d - g + \ell(K - D)
\end{align*} 

\section{Apr 19}
Let $f: X \to Y$ be a nonconstant holomorphic map between two Riemann surfaces. Thism implies that the map $f$ has finite degree $d$ and over all but finitely many points of $Y$ will have preimage of size $d$. There will be a finite number of ramification points in $X$ where $f$ will fail to be a local isomorphism. 
\begin{definition}

Given $p \in X$ mapping to $q \in Y$, we can choose local coordinates near $p, q$ such that $f$ looks like $f : z \mapsto z^m$; that is, it looks like a standard $m$-fold cover of a disk by a disk. In this case, we say that the \textbf{ramification index} is $m - 1$. When $m \geq 2$, then there is \textbf{ramification} at $p$. We denote $m - 1$ by $\nu_p(f)$.

\end{definition}

\begin{definition}
    
Let $R$ be the \textbf{ramification divisor} on $X$. This is $\sum_{p\in X} \nu_p(f)p$. The \textbf{branch divisor} $B$ is the image of $R$ under $f$ \[B = \sum_{q\in Y} \left(\sum_{p\in f^{-1}(q)} \nu_p(f)\right) q\]

In particular, if we write $B = \sum n_q q$, then the cardinality of $f^{-1}(q)$ is $d-n_q$. When you have ramification, you lose points in the fiber (so there will be less than $d$).

\end{definition}

Consider $X$ with all the points in the branch divisor removed. Take $q_1, \ldots, q_\delta$ be the points appearing in the branch locus. We consider $X' = X - f^{-1}(\{q_1, \ldots, q_\delta\})$, which maps by a \textbf{covering space map} to $Y' = Y - \{q_1, \ldots, q_\delta\}$. So $X' \to Y'$ is a $d$-sheeted covering space map. In particular, $\chi(X') = d\chi(Y')$. We know these Euler characteristics. So $\chi(Y') = \chi(Y) - \delta = 2 - 2h - \delta$. Similarly, $\chi(X') = \chi(X) - \sum_i (d - n_q)$ for $n_q = |f^{-1}(q)|$, and $\chi(X) = 2 - 2g$. We have just used the fact that removing points decreases the Euler characteristic accordingly.
\begin{theorem}[Riemann-Hurwitz]
    Let $f : X \to Y$ be a nonconstant holomorphic map of degree $d$ between compact Riemann surfaces. Let $g$ and $h$ be the genera of $X$ and $Y$, respectively. Then
    $$2 - 2g = d(2 - 2h) - \deg(R)$$
    where $\deg(R)$ is the degree of the ramification divisor.
\end{theorem}

\begin{remark}
    One interpretation of Riemann Hurwitz is that the Euler characteristic $\chi(X)$ upstairs is exactly what it would be if $X \to Y$ were a covering space—that is, $d\chi(Y)$—with correction terms for the ramification. 
\end{remark}

Another formulation in terms of differentials. Suppose $f : X \to Y$ is a map. Let $\omega$ be \textit{any} meromorphic differential on $Y$. For simplicity, let's assume that $\div(\omega)$ is supported away from the branch points. So there are no zeros or poles of $\omega$ on these branch points. What's the divisor of $f^*\omega$? But if $\omega$ has a pole or a zero or downstairs, then pulling back gives it a pole or zero of the \textit{same degree} upstairs except at the branch points. At the branch points, if $f$ looks locally like $z \mapsto z^m$, then the differential $\omega$ looks locally like $dz$, so the pull-back $(m - 1)z^{m-1}dz$ vanishes to degree $m - 1$. That is:

\begin{theorem}[Riemann -Hurwitz for differentials]
     Let $f : X \to Y$ be a nonconstant holomorphic map of degree $d$ between compact Riemann surfaces. Let $\omega$ be a meromorphic differential on $Y$. Then
\begin{align*}
    \div(f^*\omega) &= f^*\div(\omega) + R
\end{align*} 
where $R$ is the ramification divisor of $f$.
\end{theorem}

\subsection{Maps to projective space}
Recall that a linear series on $X$ is a pair of line bundle $\cL$ of degree $d$ and a vector subspace $V \subset H^0(X, \cL)$ of dimension $r$. The linear series is written $g^r_d$. We say that $V$ is basepointfree if the global sections $v\in V$ do not simultaneously vanish at any point of $X$. A base-point free linear series defines a regular map $X \to \mathbb{P}^r$. In general, a linear system will have common zeros.

Note that a global section $\sigma \in H^0(\mathcal{L})$ is determined, up to scalar, by its divisor (or divisor of zeros). The reason is simply that any two sections differ by a meromorphic function, and the only meromorphic functions on $X$ with no zeros or poles are the constants. So, by associating to each section the divisor, we get a \textit{family} of effective divisors $D$ of degree $d$ on $X$, parametrized by the projectivization $\mathbb{P}(V)$. In other words, a $g_d^r$ corresponds to a family of effective divisors of degree $d$, parametrized by $\mathbb{P}^r$. Very often, when people talk about linear series, they mean a family of divisors in this sense, and then denote the family by a $\mathcal{D}$.

If $(\mathcal{L}, V)$ is a linear series on $X$ without base points, then as before we get a map $\phi : X \to \mathbb{P}(V^\vee) = \mathbb{P}^r$. We can do this as follows. If $E$ is any effective divisor, we let $V(-E)$ be the set of sections in $V$ whose divisors are at least $E$. So, to give the map $X \to \mathbb{P}(V^\vee)$, we send each point $p \in X$ to the hyperplane $V(-p) \subset V$, and consider that as a line in $\mathbb{P}(V^\vee)$.

More concretely, if $\mathcal{L} = \mathcal{O}_X(D)$, then we can think of $H^0(\mathcal{L})$ as consisting of meromorphic functions with suitable restrictions on poles (or zeros). Then the map $X \to \mathbb{P}(H^0(\mathcal{L})^\vee)$ can be thought of by choosing a basis $f_0, \ldots, f_r \in V$ of $V$ of meromorphic functions, and then considering the homogeneous vector
$$x \in X \mapsto [f_0(x), \ldots, f_r(x)].$$

\begin{proposition}
The map $\phi : X \to \mathbb{P}^r$ associated to a linear series is characterized, up to automorphisms of $\mathbb{P}^r$, by the property that $\phi^{-1}(H)$ are exactly the divisors in the linear series $\mathcal{D}$ on $X$.

\end{proposition}

This is true either in algebraic geometry or in complex geometry, in \textit{any dimension}. We don't just have to work with Riemann surfaces (though we can't use the same language of divisors).

We want a condition that the map $\phi : X \to \mathbb{P}^r$ associated to a linear series without base-points be an imbedding.

\begin{proposition}
    $\phi$ is an imbedding if and only if: \begin{enumerate}
\item \textit{For all pairs $p, q \in X$, the subspace $V(-p - q)$ of sections (of the line bundle, contained in $V$) vanishing at both $p, q$ has dimension $\dim V - 2$. This is equivalent to $V(-p) \neq V(-q)$, since both have codimension one in $V$.}

\item \textit{$V(-2p)$ is properly contained in $V(-p)$ for each $p \in X$.}
\end{enumerate}
\end{proposition}

\begin{lemma}[Separating points and tangent vectors] \label{lemma:separating-points-tangent}
    Let $k$ be an algebraically closed field. Let $X$ be a proper $k$-scheme. Let $\mathcal{L}$ be an invertible $\mathcal{O}_X$-module. Let $V \subset H^0(X, \mathcal{L})$ be a $k$-vector space. If
    \begin{enumerate}
    \item for every pair of distinct closed points $x,y \in X$ there is a section $s \in V$ which vanishes at $x$ but not at $y$, and
    \item for every closed point $x \in X$ and nonzero tangent vector $\theta \in T_{X/k,x}$, there exists a section $s \in V$ which vanishes at $x$ but whose pullback by $\theta$ is nonzero,
    \end{enumerate}
    then $\mathcal{L}$ is very ample and the canonical morphism $\varphi_{\mathcal{L},V} : X \to \mathbb{P}(V)$ is a closed immersion.
    \end{lemma}
In particular, we can interpret the statement that $V(-p) \neq V(-q)$ as saying that the linear series $\mathcal{D}$ separates points, as each point $p$ is separated from $q$ by a section of the linear series. The second condition says that the linear series separates points and tangent vectors, as it separates the tangent vector $\theta$ at $p$ from the point $p$ itself.

Also, recall that an embedding is characterized by the property that the differential of the map is injective. Since the source of the map is a curve, the differential is injective precisely when it is nonzero. 

Choose a local coordinate $t$ on $X$ with $t(p) = 0$. Then near $p$, each section $s_i$ looks like:
\[
s_i(t) = a_i + b_i t + \ldots.
\]

Then $\phi(t) = [s_0(t) : \ldots : s_r(t)] = [a_0 + b_0 t + \ldots : \ldots : a_r + b_r t + \ldots].$

Taking the derivative at $t = 0$ gives the velocity vector of the path $\phi(t)$ in $\mathbb{P}^r$, which is:
\[
(b_0 : b_1 : \ldots : b_r).
\]
This is exactly the image of the tangent vector $\partial / \partial t \in T_p X$ under $d\phi_p$.

So we interpret this as:
\[
d\phi_p\left(\frac{\partial}{\partial t}\right) = \left(b_0 : b_1 : \ldots : b_r\right) \in T_{\phi(p)} \mathbb{P}^r.
\]
In coordinate free language, the map $d\phi_p$ is the one which takes a tangent vector $\theta \in T_p X$ and send it to the global sections obtained by differentiating the sections $\phi(p)$ along the tangent vector $\theta$. In particular, asking for this differential to be nonzero amounts to asking that these sections do not all vanish at $p$. In particular, the existence of a section which separates tangent vectors is equivalent to the condition that the differential is nonzero.


\begin{corollary}
If $\mathcal{L}$ has degree $d \geq 2g + 2$ and $V$ is the complete linear series $H^0(\mathcal{L})$, then $(\mathcal{L}, V)$ defines an imbedding of $X$ in projective space.
\end{corollary}

\begin{proof}
Let $D$ be the associated divisor. Riemann-Roch states that $\ell(D) = d - g + 1 + \ell(K - D)$, and likewise
$$\ell(D - p - q) = d - 2 - g + 1 + \ell(K - D + p + q).$$

However, $\ell(K - D) = \ell(K - D + p + q) = 0$ because these two divisors have degree $< 0$ (as $\deg K = 2g - 2$ and $\deg D$ is large). As a result, the claim follows: $\ell(D - p - q) = \ell(D) - 2$.

One can see more from this.
\end{proof}


\begin{example}
    If $\mathcal{L}$ is a line bundle of degree $d < 2g + 2$, when would it fail to be an imbedding? The only way it would fail would be if $\ell(K - D + p + q)$ admitted a nonzero section, and since $K - D + p + q$ has degree zero, this would happen precisely when $K - D + p + q$ was principal for some $p, q$.
\end{example}


Consider the canonical class of our curve $X$. Let's say $g \geq 2$. If we take $\mathcal{L}$ the canonical class, we get a map $\phi_K : X \to \mathbb{P}^{g-1}$ given by the canonical divisor. This will be an imbedding precisely when
$$\ell(K - p - q) = \ell(K) - 2,$$
and the first term, by Riemann-Roch, is
$$2g - 4 - g + 1 + \ell(p + q) = g - 3 + \ell(p + q).$$

Of course, $\ell(p + q) \geq 1$ (because of the constant function), but the problem would be if this happened to be two-dimensional. If there is a nonconstant meromorphic function with just two poles $p, q$ on $X$, then the canonical divisor fails to be an imbedding. 

\begin{corollary}
    The map $\phi_K : X \to \mathbb{P}^{g-1}$ from the canonical divisor is an imbedding unless there exists a divisor $D = p + q$ on $X$ with $\ell(D) = 2$. In other words, unless there exists a nonconstant meromorphic function of degree two on $X$.
\end{corollary}

As a result of the last section, we say:

\begin{definition}
    We say that a compact Riemann surface $X$ is \textbf{hyperelliptic} if (equivalently) $X$ has a global meromorphic function of degree 2, or if $X$ is expressible as a 2-sheeted (branched) cover of $\mathbb{P}^1$.
\end{definition}

So for non-hyperelliptic curves of genus $\geq 2$, the canonical divisor induces an imbedding if and only if $X$ is \textit{not} hyperelliptic. We have now shown that there are non-hyperelliptic curves, but we will see this soon enough. Once you get to genus 3, \textit{most} curves are non-hyperelliptic. One question we'll raise later is what degree one needs in general to express a curve as a branched cover of the sphere.

For a non-hyperelliptic curve, we call $\phi_K(X) \subset \mathbb{P}^{g-1}$ the \textbf{canonical model} of $X$. The whole point is to understand the connection between the abstract Riemann surface $X$ and concrete subvarieties of projective varieties in $\mathbb{P}^n$. There are \textit{lots} of ways of imbedding a Riemann surface in projective space. This is a canonical one for a non-hyperelliptic curve of genus $\geq 2$.

\begin{theorem}[Geometric Riemann Roch]
    Let $X$ be nonhyperelliptic of genus at least two. Then the canonical map $\phi_K : X \to \mathbb{P}^{g-1}$ is an imbedding of degree $2g - 2$. Consider a divisor $D = \sum_{i=1}^d p_i$ on $X$ of distinct points. Then $r(D):= \ell(D) - 1$ is the number of linear relations on the points $p_1,\dots,p_d$ (you can interpret these points as living in $\mathbb{P}^{g-1}$).
\end{theorem}


\subsection{Differentials on smooth plane curves}
Let $C \subset \mathbb{P}^2$ be a \textit{smooth plane curve}, of degree $d$ (and genus\footnote{Recall that we did this by computing the degree of the canonical divisor in terms of the topological genus, by the Poincaré-Hopf theorem (for instance). In this case, we showed that the canonical bundle $K_C$ was $(d-3)L$ for $L$ a line in $\mathbb{P}^2$: the degree was thus $d(d-3)$, which was $2g-2$, and let us compute $g$.} given by the usual formula $\binom{d-1}{2}$). Choose an affine open in $\mathbb{P}^2$, so say a complement of a line. Let's take the line at $\infty$, which we'll call $L_\infty$. We have $\mathbb{A}^2 = \mathbb{P}^2 - L_\infty$ (the usual plane before they projectivized it). We can choose affine coordinates $(x,y)$ on this affine plane $\mathbb{A}^2 \subset \mathbb{P}^2$, and let the corresponding homogeneous coordinates on $\mathbb{P}^2$ be $\{X, Y, Z\}$. (The line $L_\infty$ is thus given by $Z = 0$.) We're going to single out the vertical point at $\infty$, which is $[0, 1, 0]$.

We now want to make a couple of assumptions about the coordinate plane:

\begin{enumerate}
\item $L_\infty$ intersects the curve $C$ transversely, in $d$ distinct points.

\item $C$ does not contain the vertical point at $\infty$, $[0, 1, 0]$.
\end{enumerate}

We can achieve this by rotating the axes.

In $\mathbb{A}^2$, let us say that the curve is the zero locus of an inhomogeneous polynomial $f$ in $x, y$ (so the projective curve is the zero locus of the homogenization). Let's now write \textit{one} holomorphic differential on the curve. The standard way to do this is to pick a \textit{meromorphic} differential, and then to kill the poles by multiplying by a meromorphic function. So, consider the meromorphic differential $dx$: this is a perfectly good meromorphic differential, which is even \textit{holomorphic} in the finite plane (but not at $\infty$). To say this in another way, if we \textit{project} from the point $[0, 1, 0] \notin C$ (in the affine picture: just the vertical projection, or projection to the $x$-coordinate), we get a map
$$\pi : C \to \mathbb{P}^1_x$$
which expresses $C$ as a $d$-sheeted cover of $\mathbb{P}^1$, unramified at $\infty$ (because we assumed there were $d$ points in $L_\infty$). The differential $dx$ on $C$ is the pull-back of the differential $dx$ on $\mathbb{P}^1$ (which is meromorphic).

So, there's our differential: $dx$. What's its divisor? In the finite plane, there are no poles; where are the zeros? The zeros are the points where $\pi$ is ramified. In fact, this is equivalently where $\frac{\partial f}{\partial y} = 0$. To see this, note that
$$\left(\frac{\partial f}{\partial x}dx + \frac{\partial f}{\partial y}dy\right)\bigg|_C = 0,$$

because it is $df$ and $f = 0$ on $C$. Consequently, if $\frac{\partial f}{\partial y}$ vanishes at a point, \textit{smoothness} of the curve (that is, the partials of $f$ have no common zeros), $dx$ must vanish on $C$ to the \textit{same order} as $\frac{\partial f}{\partial y}$.

The key result of this discussion is that:
\begin{proposition}
    In the finite plane, $dx$ is holomorphic, and its divisor is the divisor of the function $\frac{\partial f}{\partial y}|_C$.
\end{proposition}
```latex
At infinity, we have $C \to \mathbb{P}^1$ as a $d$-sheeted cover. Well, the claim is that $dx$ has a pole of order 2 at $\infty$: this is because $dx$ on $\mathbb{P}^1$ has a pole of order 2 at $\infty$ (easy to check), and $\pi$ is unramified above the points at $\infty$. So, if $D_\infty$ is the intersection $L_\infty \cap C$, the points at $\infty$, then the differential $dx$ has double poles at each point in $D_\infty$.


In particular, we find:
$$\operatorname{div}(dx) = \operatorname{div}\left(\frac{\partial f}{\partial y}\right)\bigg|_{\mathbb{A}^2} - 2D_\infty.$$

Now, we have to divide by a function with poles at $\infty$ to get rid of the poles of $dx$. Consider a polynomial of degree 2 in $x, y$: there will be a pole at $\infty$. If we divide by this, we can get rid of the poles at $\infty$ of $dx$. Of course, we have to be careful: this division will introduce \textit{new poles} where this polynomial by which we divide vanishes. I.e., we're looking for a holomorphic differential of the form $dx/h$ where $h \in \mathbb{C}[x, y]$ has degree at least two—so has no poles at $\infty$—but we have to worry about the zeros of $P$. If we can arrange the zeros of $P$ so they occur at pre-existing zeros of $dx$, then we won't have a problem: the zeros will just cancel.

Let's consider $P(x,y) = \frac{\partial f}{\partial y}$; let's assume this is degree $\geq 2$. Given the expression for $\operatorname{div}(dx)$, we see that the zeros of $P$ are precisely at the zeros of $dx$. Consequently, we have obtained a holomorphic differential on the curve:

\begin{proposition}
    Suppose the degree is at least 3. Then $\omega = \frac{dx}{\frac{\partial f}{\partial y}}$ is a holomorphic differential on the curve $C$. Moreover, $\omega$ is nonzero on the affine part of $C$, i.e. $\mathbb{A}^2 \cap C$. There will be zeros of order $d - 3$ along the divisor at $\infty$.    
\end{proposition}


Note that dividing by $P$ precisely eliminated the zeros in the affine part. The divisor at $\infty$ was obtained by considering the pole of $P = \frac{\partial f}{\partial y}$ at $\infty$. Note also that since the differential $\operatorname{div}(\omega)$ is $(d - 3)D_\infty$, we have obtained again that the canonical divisor is $d(d - 3)$, as expected by the Riemann-Roch theorem and the genus formula.

How can we write down other holomorphic differentials? We don't have to worry about poles except at $\infty$; we just have to multiply $\omega$ as above by \textit{any} rational function with poles at most $d - 3$ at $\infty$. These things are known as polynomials of degree at most $d - 3$. In other words, if $g(x,y)$ is a polynomial of degree at most $d - 3$, then $g\omega$ is a global holomorphic differential. This space of $\{g\omega\}$ is a \textit{vector space} of differentials, and the dimension is $\binom{d-1}{2}$. As a result, we have expected a $g$-dimensional vector space of holomorphic differentials, and that's all.

\begin{corollary}
    The holomorphic differentials on $C$ are of the form $g\omega$ for $\omega$ as above and $g \in \mathbb{C}[x,y]$ a polynomial of degree $\leq d - 3$.
\end{corollary}

We've written down all the differentials on a plane curve. This is a special case: \textit{most} curves are not smooth plane curves. The second case that we will later consider is the normalization of a nodal plane curve. Since any curve can be realized in this way, we'll be done.

\section{Apr 20}
\subsection{Dualizing sheaf on a nodal curve}
Let $C$ be a nodal curve and let pair of points \( q_i \) and \( r_i \) of \( \widetilde{C} \) lying over each node \( p_i \). The dualizing sheaf \( \omega_C \) may be defined as a subsheaf of the pushforward of the sheaf of rational differentials on \( \widetilde{C} \): it’s the sheaf associating to each open \( U \subset C \) the space of rational one-forms \( \eta \) on \( \nu^{-1}(U) \subset \widetilde{C} \) having at worst simple poles at the pairs of points \( q_i \) and \( r_i \) of \( \widetilde{C} \) lying over each node \( p_i \in U \) of \( C \), and such that for each such pair of points
\[
\operatorname{Res}_{q_i}(\eta) + \operatorname{Res}_{r_i}(\eta) = 0.
\]
There is a short exact sequence of sheaves \begin{align*}
    0 \to \omega_C \to \nu^*\omega_{\widetilde{C}} \to \bigoplus_p \C \to 0
\end{align*} where nearby a node, the second map takes a pair of differential forms on the different branches and sends them to the sum of their residues at the node. This short exact sequences identifies the dualizing sheaf of the nodal curve with the kernel of this residue map, and can be thought of as Serre dual to the exact sequence on functions \begin{align*}
    0 \to \mathcal{O}_C \to \nu^*\mathcal{O}_{\widetilde{C}} \to \bigoplus_p \C \to 0
\end{align*}
where the second map takes a pair of functiosn and sends them to the difference of their values at the node. The dual of this sequence is the one we just wrote down is obtained by applying $\Hom_C(-, \omega_C)$ to the sequence on functions, and the object $\omega_C$ represents the duality functor $\Ext ^1_C(-, \omega_C)$.

For a local model, we have the formal completion of the local ring at a node, which is a local ring of the form \(R = \C[[x,y]]/(xy)\). We then pass to the normalization, and in particular the formal completions of the local rings at the points \(q_i\) and \(r_i\) and so we get \begin{align*}
    R &\to \widetilde R = \C[[x]] \oplus \C[[y]]  \\
    f &\mapsto (f(x,0), f(0,y))
\end{align*}
One should be able to see presentations for $\omega_R$ and $\omega_{\widetilde{R}}$ but I don't know how to talk about the residue of an element in $k[[x]] dx$. Shouldn't it just be zero? 


The dualizing sheaf gives a characterization of moduli stable curves amongst all nodal ones.
\begin{definition}
    A curve is \textbf{stable} if it is connected, nodal, and every irreducible component of geoemtric genus zero has at least three node branches, and every irreducible component of geometric genus one has at least one node branch. 
\end{definition}
\begin{exercise}
    Let \( C \) be a stable curve of genus \( g \).
    \begin{enumerate}
        \item Show that \( H^0(C, \omega_C^{\otimes n}) = (2n - 1)(g - 1) \), and that for \( n \geq 2 \), \( H^1(C, \omega_C^{\otimes n}) = 0 \).
        \item Show that for \( n \geq 3 \), \( \omega_C^{\otimes n} \) is very ample on \( C \).
    \end{enumerate}
    \end{exercise}
    
    \begin{exercise}
    Let \( C \) be a complete connected nodal curve.
    \begin{enumerate}
        \item Show that, for \( n \geq 3 \), the sheaf \( \omega_C^{\otimes n} \) is very ample if and only if \( C \) is moduli stable. Hence, \( \omega_C \) is ample on a complete connected nodal curve \( C \) if and only if \( C \) is moduli stable.
        \item Similarly, let \( p_1, \ldots, p_n \in C \) be distinct smooth points of \( C \). Show that \( (C; p_1, \ldots, p_n) \) is a stable \( n \)-pointed curve if and only if the line bundle
        \[
        \omega_C\left( \sum_{i=1}^n p_i \right)
        \]
        is ample.
    \end{enumerate}
    \end{exercise}

\subsection{Automorphisms}

Our definition of a stable curve requires that it have only finitely many automorphisms. To verify the local description of \(\mathcal{M}_g\) given in the previous chapter (as smooth away from loci of curves with automorphisms at which it has quotient singularities), we'll need the slightly stronger assertion that the scheme-theoretic automorphism group of a stable curve is finite and \emph{reduced}. This is our next goal.

To start with, we need to make precise the scheme structure on the automorphism group. This turns out to be a bit involved. We will just sketch the ideas here and refer you to Section 1 of Deligne Mumford's original for more details. We also simplify by working over a point rather than a more general base. The first step is to define, for any two stable curves \(C\) and \(D\), an isomorphism functor \(\operatorname{Isom}(C, D)\) whose value on a scheme \(S'\) is the set of \(S'\)-isomorphisms between \(C \times S'\) and \(D \times S'\). Any such isomorphism must identify the relative dualizing sheaves of \(C\) and \(D\) and hence all powers of these sheaves. This leads to a representation of the functor \(\operatorname{Isom}(C, D)\) by a subscheme of a suitable projective linear group.

\medskip

More precisely, fix \(g \geq 2\) and an integer \(n \geq 3\). Define integers \(r\) and \(d\) in terms of these by
\begin{align*}
r + 1 &= (2n - 1)(g - 1), \\
d &= 2n(g - 1)
\end{align*}

Let \(\mathcal{H} = \mathcal{H}_{d, g, r}\), and let \([C]\) and \([D]\) be the points of \(\mathcal{H}\) determined by \(C\) and \(D\). Define a map \(\mu : \operatorname{PGL}(r+1) \to \mathcal{H} \times \mathcal{H}\) by \(\mu(\alpha) = (\alpha \cdot [C], [D])\), and let
\[
I(C, D) = \mu^{-1}(\Delta) \subset \operatorname{PGL}(r+1),
\]
where \(\Delta\) is the diagonal in \(\mathcal{H} \times \mathcal{H}\). The scheme \(\operatorname{Isom}(C, D)\) turns out to represent \(\operatorname{Isom}(C, D)\), although we won't verify this here.

To define the automorphism group of a stable curve \(C\), we just take \(D = C\) in the foregoing. This amounts to identifying \(\operatorname{Aut}(C)\) with the stabilizer in \(\operatorname{PGL}(r + 1)\) of \([C] \in \mathcal{H}\). The assertion we're after is then:

\begin{lemma}
\(\operatorname{Aut}(C) = \operatorname{stab}_{\operatorname{PGL}(r + 1)}([C])\) \emph{is reduced}.
\end{lemma}

If not, there would be a nonzero \(\mathbb{I}\)-valued point of \(\operatorname{Aut}(C)\) lying over the identity, or equivalently, a nonzero regular vector field on \(C\). The following exercise rules this out. Such a vector field would correspond to a regular vector field on the normalization \(\widetilde{C}\) vanishing at all points lying over the nodes of \(C\). Show that such a vector field must be identically 0 on every component of \(\widetilde{C}\).

\begin{proposition}
On a stable curve \(C\), there is no nonzero, everywhere regular vector field. Equivalently, \(\operatorname{Ext}^0(\Omega_C, \mathcal{O}_C) = \{0\}\).
\end{proposition}

\section{Apr 20 - 26}
I am trying this new system where I take notes on paper during the week and then type them up on the weekend. 
\subsection{Dualizing sheaf on a nodal curve}
Consider the dualizing sheaf $\omega_C$ of a nodal curve $C$. We have a short exact sequence \begin{align*}
    0 \to \omega_C \to \nu^*\omega_{\widetilde{C}} \to \bigoplus_p \C \to 0
\end{align*} where near $p$ the map $\omega_C \to \nu^*\omega_{\widetilde{C}}$ looks like the inclusion of rational differentials with at worst simple poles at the nodes, satisfying the condition that the sum of the residues at the nodes is zero. This sequence can be interpreted as the Serre dual of the sequence on functions \begin{align*}
    0 \to \mathcal{O}_C \to \nu^*\mathcal{O}_{\widetilde{C}} \to \bigoplus_p \C \to 0
\end{align*} where the second map takes a pair of functions and sends them to the difference of their values at the node. We are interpreting the dualizing sheaf in this context as the object representing the duality functor $\Ext^1_C(-, \omega_C)$ for coherent sheaves on $C$.

In particular, $\omega_C$ is the locally free sheaf on $C$ so that for every coherent sheaf $\cF$ on $C$, we have a natural isomorphism \begin{align*}
    \Hom_C(\cF, \omega_C) \cong \Ext^1_C(\cF, \mathcal{O}_C)^\vee
\end{align*}

Let's take an explicit local model for a nodal curve. We can consider the local ring at a node $\C[x,y]/(xy)_{\mathfrak{m}}$ where $\mathfrak{m}$ is the maximal ideal at the node. In particular we are considering rational functions $f(x,y)/g(x,y)$ where $f,g \in \C[x,y]/(xy)$ and $g$ does not vanish at the node. But the better object is the formal completion of the local ring at a node, which is a local ring of the form \(R = \C[[x,y]]/(xy)\). This ring is obtained by completing $\C[x,y]/(xy)$ at the maximal ideal $\mathfrak{m}$, in particular we are taking a limit \begin{align*}
    R &= \lim \C[x,y]/(xy) /\mathfrak{m}^n \\
    &= \C[[x,y]]/(xy)
\end{align*}
The normalization map is given by \begin{align*}
    R &\to \widetilde R = \C[[x]] \oplus \C[[y]]  \\
    f &\mapsto (f(x,0), f(0,y))
\end{align*} and the image is clearly those pairs $f(x),g(y)$ so that $f(0) = g(0) = 0$. 
\begin{proposition}
    Let $\cF$ be a coherent sheaf on a nodal curve $C$. Then \begin{align*}
        \Hom(\cF, \omega_C) &= \Ext^1(\cF, A)^\vee \\
    \end{align*}
\end{proposition}

The first step in establishing something like this is to see that it is enough to check this fact for $\cF = \cO_{C,p}/\mathfrak{m}_p$ the skyscraper sheaf at a node $p$. This is not quite immediate. For smooth curves $C$, coherent sheaves look very simple. Every coherent sheaf is an extension \begin{align*}
    0 \to \cF_{tor} \to \cF \to \cF_{free} \to 0
\end{align*} where $\cF_{\text{tor}}$ is the torsion part and $\cF_{\text{free}}$ is the free part. This follows from the fact that we can check the claim at stalks, so we are thinking about finitely generated modules over the local ring $\mathcal{O}_{C,p}$. Since $C$ is a smooth curve, $\cO_{C,p}$ is a regular local ring, in particular a DVR, in particular a PID and so the claim follows from the structure theorem for finitely generated modules over a PID.

For a nodal curve, the situation is more complicated. The local ring at a node is not a regular local ring, and so we cannot use the structure theorem. Moreover, one can have modules over $R$ which are not torsion free and not free, for example the ideal $\mathfrak{m} = (x,y)$ is torsion free as a module over $R$ but not free (an ideal of a local ring is free if and only if it is principal).

In general, I believe that we can still reduce the problem to the case of skyscraper sheaves at nodes. The idea is to pass to devissage a la Harder-Narasimhan filtration. In any case, we will proceed by checking the claim for skyscraper sheaves at nodes. In particular we need to see that \begin{align*}
    \Hom_A(A/\mathfrak{m}, \omega_C) &\cong \Ext^1(A/\mathfrak{m}, \mathcal{O}_C)^\vee \\
\end{align*}
We can clearly identify the left hand side with $\omega_C/\mathfrak{m}\omega_C$. This is a $1$-dimensional $\C$-vector space as follows. Note that $\omega_C$ is generated as an $\mathcal{O}_C$-module by the single element $(dx/x,-dy/y)$, i.e. every element of $\omega_C$ is of the form $f(x,y)(dx/x, -dy/y)$ for some $f \in R$. We are modding out by those elements which have $f(0,0) = 0$. Therefore it is clear that we can identify $\omega_C/\mathfrak{m}\omega_C$ with the $1$-dimensional vector space of elements of the form $a(dx/x, -dy/y)$ where $a \in \C$, where $a$ is the constant term of $f$.

As for the other term we have the Koszul resolution of $A/\mathfrak{m}$ given by \begin{align*}
    0 \to A \to A^2 \to A \to A/\mathfrak{m} \to 0
\end{align*} where the first map is given by $\begin{pmatrix}
    x \\
    y
\end{pmatrix}$ (in general the Jacobian matrix). The second map is given by the matrix 
    $\begin{pmatrix}
        y & x \\
    \end{pmatrix}$
and applying $\Hom(-, A)$ and taking cohomology identifies the group \begin{align*}
    \Ext^1(A/\mathfrak{m}, A) &= \ker\left(A^2 \to A\right)/\operatorname{im}\left(A \to A^2\right) \\
\end{align*} The kernel is those $(a,b) \in A^2$ so that $ax + by = 0$ and reasoning about this equation us that $a\in y\C[[y]]$ and $b \in x\C[[x]]$. The image is $(rx,ry)$ for $r \in A$ which can be identified with pairs $a,b$ so that $a \in y\C[[y]]$ and $b \in x\C[[x]]$ with the same linear coefficient. Thuse this vector space is also $1$-dimensional, and there is a pairing \begin{align*}
    \omega_C/\mathfrak{m}\omega_C \times \Ext^1(A/\mathfrak{m}, A) &\to \C \\
    (c(dx/x, -dy/y), (a,b)) &\mapsto c \cdot (\text{linear coeff of } a)
\end{align*}

\subsection{Deformation theory}
A deformation of a smooth variety $X$ with base $(Y,y)$ is a proper flat morphism $\phi: \mathcal{X} \to Y$ with an isomorphism $\psi: X \to \phi^{-1}(y)$. Equivalently we have a pullback square 
\begin{center}
    \begin{tikzcd}
        X \arrow[r, "\psi"] \arrow[d, "\phi"'] & \mathcal{X} \arrow[d, "\phi"] \\
        \Spec \C = y \arrow[r, "i"] & Y
    \end{tikzcd}
\end{center}
$\Spec \C \to Y$ is a geometric point of $Y$. It has a map $\cO_{Y,y} \to \C$ factoring through the residue field \begin{align*}
    \cO_{Y,y} \to \cO_{Y,y}/\mathfrak{m}_y \cong \C
\end{align*} giving a way of "evaluating functions in $\C$ at that point. A first order deformation is a deformation of $X$ over the dual numbers $I = \C[t]/(t^2)$ with basepoint $0$. 
\subsection{Coverings and torsors}
Ultimately we need to construct abelian-by-finite families as moduli of some torsors.
\begin{definition}
    Let $X$ be a scheme. A finite etale covering of $X$ is a finite etale morphism $f: Y \to X$. In particular:
    \begin{itemize}
        \item $f$ is finite, meaning that locally the map $\Spec B \to \Spec A$ is given by a finite $A$-algebra $B$.
        \item $f$ is etale, meaning that the map is flat and unramified.
        \item $f$ is flat, meaning that on the level of stalks the map $\mathcal{O}_{X,x} \to \mathcal{O}_{Y,y}$ is flat.
        \item $f$ is unramified, meaning that $f$ is locally of finite type and the sheaf of relative differentials $\Omega_{Y/X}$ is zero. 
        \end{itemize}
\end{definition}
Note that there is a notion of being $G$-unramified per SGA which by definition means that $f$ is locally of finite presentation and $\Omega_{Y/X} = 0$. For locally noetherian target, the notions correspond.
\begin{example}
    $X \coprod X \to X$ is a finite etale covering. 
    If $X = \Gm$ over $k$  and $\Char k$ does not divide $n$, then the map $\mu_n: \Gm \to \Gm$ given by $x \mapsto x^n$ is a finite etale covering. In particular, the map $\mu_n$ is a finite etale covering of degree $n$. 

    If $X/K$ is a variety and $L/K$ is a finite separable extension, then the base change $$X_L = X \times_{\Spec K} \Spec L$$ is a finite etale covering of $X$. Recall that these are in correspondence with finite sets with a $\pi_1(X)$-action. 
\end{example}
Let $X$ be of finite type over $\C$. Assumptions of this flavor come from the GAGA theorem. If $Z\to X$ is a finite etale covering over $\C$, then $Z^{\an}$ is a finite covering of $X^{\an}$ in the topological sense. SGA1 gives a converse: 

\begin{theorem}
    Every topological covering space of $X^{\an}$ with finite fibers is $Z^{\an}$ for some finite etale covering $Z \to X$.
\end{theorem} Given $Z\to X$ a finite etale cover, there is an action of $\pi_1(X,x)$ on $Z_x$ by the homotopy lifting property.

\begin{definition}
    Let $G$ be a discrete group. A $G$-torsor on $X$ is a surjective map $\pi:Z\to X$ with a right action $\rho: G \times Z \to Z$ such that $\pi\rho = \pi\proj_Z$ and the map $(\id,\rho)$ is an isomorphism.
\end{definition} 
This notion is equivalent to that of a principal $G$-bundle, except the language of principal bundles is more common in complex geometry and usually suggests that the group $G$ is a Lie group. The notion of a $G$-torsor is more common in number theory and suggests that $G$ is discrete or perhaps profinite.

\begin{proposition}
    If $X$ is a nice connected space, there is an isomorphism between the isomorphism classes of $G$-torsors (connected) with outer homomorphisms $\Hom(\pi_1(X,x), G) / \Inn(G)$ (surjective).
\end{proposition}
Now let $X$ be a variety over a number field $K$. The correspondence gives us a correspondence between $\Z/n\Z$-torsors over $X_{\overline K}$ and $\Z/n\Z$-torsors over $X^{\an}_\C$, the latter which we can identify as \begin{align*}
    \Hom(\pi_1(X^{\an}, x), \Z/n\Z) &\cong H^1(X^{\an}, \Z/n\Z) \text{ no inner aut since abelian}\\
    &\cong H^1_{\text{\'et}}(X_{\overline K}, \Z/n\Z) 
\end{align*}
If $X$ is a curve, let $J$ be its Jacobian. Then we also have the identifications \begin{align*}
    \Het{1}{X_{\overline K}}{\Z/n\Z} &\cong \Het{1}{J_{\overline K}}{\Z/n\Z} \\
    &\cong \Hom(J[n](\overline{K}), \Z/n\Z) \\
\end{align*}
All of these groups classify $\Z/n\Z$-torsors over $X_{\overline K}$, and the last group is the group of $n$-torsion points of the Jacobian. 
\begin{example}
    [An example in which etale cohomology is not computed using GAGA like theorems]
    If $X$ is a curve, then \begin{align*}
        \Het{2}{X_{\overline K}}{\Z/n\Z} &\cong \Z/n\Z(-1)
    \end{align*} where the right hand side carries something called the Tate twist. 
\end{example}
Also note that $X$ is one-dimensional yet has nontrivial etale cohomology in degree above the dimenison. This is a general phenomenon and it turns out that the ranks of these groups equal to topological Betti numbers of the analytification $X^{\an}$. 
\subsection{Separated and proper morphisms}
Given a morphism $X\to Y$ the diagonal of this morphism is the morphism $\Delta: X \to X \times_Y X$ given by $\delta(x) = (x,x)$. Properties of the diagonal: \begin{enumerate}
    \item If $X$ and $Y$ are affine, then $\Delta$ is a closed immersion. This is because locally, the map looks like $R \otimes_S R \to R$ which is surjective.
    \item In general, $\Delta$ is a locally closed immersion, i.e. a closed immersion followed by an open immersion.
\end{enumerate} Recall that $i:Z\to X$ is a closed immersion if for all $U\subset X$ affine open $U = \Spec R$, then $i^{-1}(U) = \Spec R/I$ for some ideal $I\subset R$. 
\begin{definition}
    A morpihsm $f:X\to Y$ is separated if $\Delta$ is a closed immersion. 
\end{definition}
A nice fact is that a locally closed immersion is a closed immersion if and only if its image is a closed set. Oftentimes, this lets just check that a map is a closed immersion by checking that the image is closed, this is reducing a scheme theoretic problem to a topological one.
\begin{example}
    Two copies of $\mathbb{A}^1$ glued along $\A^1 - \{0\}$ is not separated. A monomorphism is separated. This implies that a locally closed subsceme of a separated scheme is separated. $\P^n$ is separated.
\end{example}
Here are some important properties of separated morphisms:
\begin{enumerate}
    \item If $X\to Y$ is separated and $\sigma$ is any section, then $\sigma$ is separated. In general, $\sigma$ is only a locally closed immersion.
    \item The intersection of affine opens in a separated scheme is affine.
    \item The locus where two morphisms agree is closed if $Y$ is separated.
    \item If two morphisms from a reduced scheme to a separated scheme agree on an open dense subset, then they agree everywhere. This makes the set of rational morphisms, meaning morphisms defined on a dense open subset, a well defined notion.
\end{enumerate}
\begin{example}
    In a category with fiber products, $f:X\to Y$ is a monomorphism if and only if $X \cong X\times_Y X$. This is true by the functor of points approach. 
\end{example}
Given a locally closed immersion $i:X\to Y$ of schemes over a base $S$, if the structure map $Y\to S$ is separated, then so is $X\to S$. If we show that $Y = \P^n$ is separated over $\Spec \Z$, then every quasiprojective scheme over $\Spec \Z$ is separated. 
\begin{theorem}
    $\P^n\to \Spec \Z$ is separated.
\end{theorem}
\begin{proof}
    There is a commutative diagram involving the diagonal morphism, the segre embedding, and the second veronese which looks like \begin{center}
        \begin{tikzcd}
            \P^n \arrow[r, "\Delta"] \arrow[d, "v_2"'] & \P^n \times_{\Spec \Z} \P^n \arrow[d, "\text{segre}"] \\
            \P^{\binom{n+1}{2} - 1} \arrow[r, "\text{linear}"] & \P^{(n+1)^2 - 1}
        \end{tikzcd}
    \end{center}
where the bottom composition is obviously a closed immersion. Therefore the image of $\Delta$ is a closed subset, and being a locally closed immersion, $\Delta$ is therefore a closed immersion.
\end{proof}
\begin{definition}
    A morphism is proper if it is separated, finite type, and universally closed.
\end{definition} The latter means that for any $Z\to Y$, the base change is a closed morphism. This property is closed under composition, base change, and is local on the target. 
\begin{example}
    A finite morphism is proper. Integral morphisms are closed, and integral + finite type = finite. In particular, a closed immersion is proper.
\end{example}
The main example of proper morphisms to keep in mind is the following.
\begin{theorem}
    [Fundamental theorem of elimination theory] $\P^n \to \Spec \Z$ is proper.
\end{theorem}
\begin{example}
    Fix a ring $A$. Two polynomials $f,g \in A[x]$ define a closed subset $Z \subset \P^1_A$ given by homogenizing. The image of $Z$ in $\Spec A$ is the set of coefficients $a_i,b_j$ for which there exists a common root of $f$ and $g$. This set is cut out by a single polynomial in the $a_i,b_j$ and is known as the resultant. In particular, we have eliminated the quantifier "there exists" and reduced it to some polynomial equation. This is the main idea of elimination theory.
\end{example}

\begin{definition}
    
We say that $f:X\to Y$ satisfies the existence/uniqueness property of the valuative criterion if for any discrete valuation ring $R$ with fraction field $K$ and any diagram of solid arrows there exists a  dotted arrow / any such is unique $\delta: \Spec R \to X$ such that the diagram commutes. In other words, we have a commutative diagram of the form
\begin{center}
    \begin{tikzcd}
        \Spec K \arrow[r, "\alpha"] \arrow[d, "\beta"'] & X \arrow[d, "f"] \\
        \Spec R \arrow[r, "\gamma"] \arrow[ru, "\delta", dashed] & Y
    \end{tikzcd}
\end{center}
\end{definition}
\begin{theorem}
    If $f:X\to Y$ is a finite type morphism of locally noetherian schemes, then $F$ is separated/proper if and only if it satisfies the existence/existence and uniqueness property of the valuative criterion.
\end{theorem}

\subsection{\(p\)-adic Fields and Good Reduction}
\begin{definition}
        A \textbf{local field} is a field that is complete with respect to a discrete valuation and has a finite residue field. 
\end{definition}

There are two main types of local fields:
        \begin{enumerate}
            \item Archimedean local fields: These are \(\mathbb{R}\) and \(\mathbb{C}\), which are complete with respect to the usual absolute value.
            \item Non-archimedean local fields: These are fields complete with respect to a discrete valuation, with a finite residue field. Examples include:
            \begin{itemize}
                \item \(\mathbb{Q}_p\), the field of \(p\)-adic numbers.
                \item Finite extensions of \(\mathbb{Q}_p\).
                \item The field of Laurent series \(\mathbb{F}_q((t))\) over a finite field \(\mathbb{F}_q\).
            \end{itemize}
        \end{enumerate}

\begin{definition}
    A \textbf{\(p\)-adic field} is a finite extension \(K\) of the field of \(p\)-adic numbers \(\mathbb{Q}_p\). It is a complete discrete valuation field with residue field \(k = \mathcal{O}_K / \mathfrak{m}_K\), where:
    \begin{itemize}
        \item \(\mathcal{O}_K\) is the ring of integers of \(K\),
        \item \(\mathfrak{m}_K\) is the maximal ideal of \(\mathcal{O}_K\),
        \item \(k\) is a finite field of characteristic \(p\).
    \end{itemize}
    The valuation \(v: K^\times \to \mathbb{Z}\) extends the \(p\)-adic valuation on \(\mathbb{Q}_p\), and the absolute value \(|\cdot|_p = p^{-v(\cdot)}\) makes \(K\) a locally compact field.
\end{definition}

\begin{remark}
    Equivalently, a \(p\)-adic field is a field that is complete with respect to a discrete valuation (extension of the p-adic valuation) and whose residue field is finite. In particuular, $p$-adic fields are local fields of characteristic $0$ and whose valuation specifically comes from the $p$-adic valuation.
\end{remark}


\begin{definition}
    Let \( K \) be a \( p \)-adic field with ring of integers \( \mathcal{O}_K \), maximal ideal \( \mathfrak{m}_K \), and residue field \( k = \mathcal{O}_K / \mathfrak{m}_K \). A smooth proper variety \( X/K \) is said to have \textbf{good reduction} if there exists a smooth proper scheme \( \mathcal{X} \) over \( \mathcal{O}_K \) such that:
    \[
    \mathcal{X}_\eta := \mathcal{X} \times_{\mathcal{O}_K} \operatorname{Spec}(K) \cong X.
    \]
    Here, \( \mathcal{X}_\eta \) is called the \emph{generic fiber}, and \( \mathcal{X}_s := \mathcal{X} \times_{\mathcal{O}_K} \operatorname{Spec}(k) \) is called the \emph{special fiber}.
\end{definition}

\begin{remark}
    The scheme \(\operatorname{Spec}(\mathcal{O}_K)\) has two points:
\begin{itemize}
    \item The generic point corresponds to the zero ideal \((0)\), its residue field is \(\operatorname{Frac}(\mathcal{O}_K) = K = \mathbb{Q}_p\).
    \item The closed point corresponds to the maximal ideal $\mathfrak{m}_K$, its residue field is \(\mathcal{O}_K / (p) = k = \mathbb{F}_p\).
\end{itemize}
    Intuitively, good reduction means that \( X \) can be extended to a smooth proper family \( \mathcal{X} \) over \( \mathcal{O}_K \), where the special fiber \( \mathcal{X}_s \) is also smooth.
\end{remark}

\begin{example}
    Let \( K = \mathbb{Q}_p \), \( \mathcal{O}_K = \mathbb{Z}_p \), and \( k = \mathbb{F}_p \). Consider an elliptic curve \( E \) over \( \mathbb{Q}_p \) given by a Weierstrass equation:
    \[
    y^2 = x^3 + ax + b, \quad a, b \in \mathbb{Q}_p.
    \]
    If there exists a model over \( \mathbb{Z}_p \) (i.e., \( a, b \in \mathbb{Z}_p \)) such that the reduction modulo \( p \) defines a smooth curve over \( \mathbb{F}_p \), then \( E \) has good reduction.
\end{example}

\section{Apr 30}
\subsection{Finite type morphisms}
How to think about finite type morphisms. One advantage of them is that the image of a constructible set is constructible. In particular, a quasicompact open immersion is finite type.
\begin{example}
    Consider $R = k[x,y]_{(x,y)}$ the local ring at the origin. \begin{align*}
        k[x,y]_{(x,y)} &= \set{\frac{f}{g} \in k(x,y) \mid g(0,0) \neq 0} \\
    \end{align*} Consider the map \begin{align*}
        \Spec R &\to \A^2
    \end{align*}  What are the points of $\Spec R$? On $\Spec R$ we have open sets of the form \begin{align*}
        D(f) &= \set{\mf p \in \Spec R \mid f\not \in \mf p }
    \end{align*}
    It's an element of the intersection of the open sets \begin{align*}
        \bigcap D(g)
    \end{align*} over $g(0,0) \neq 0$. We are thinking about germs of rings of functions near the origin.
    This map is quasicompact but not finite type, in particular not an open immersion. If it were finite type, this would imply that $k[x,y]_{(x,y)}$ is a finitely generated $k[x,y]$-algebra, which is not true because of denominators.

    In general, $R \subset \hat R$ the completion of $R$ at the maximal ideal, with inclusion given by Taylor expansion.  \begin{align*}
        \Spec k[[x,y]] &\to \Spec R = \bigcap_{g(0,0)\neq 0} D(g) \to \A^2
    \end{align*}
    Chevalley's theorem must fail. Let's ponder the image of the second map. The morphism of schemes
    \[
    \phi: \operatorname{Spec}(R) \to \operatorname{Spec}(k[x, y]) = \mathbb{A}^2
    \]
    induced by the localization map \( k[x, y] \to R \) sends a prime ideal \( \mathfrak{p} \subset R \) to its contraction \( \mathfrak{p} \cap k[x, y] \subset k[x, y] \).
    
    The image of \( \phi \) consists of those prime ideals of \( k[x, y] \) that are contained in the maximal ideal \( \mathfrak{m} = (x, y) \). That is,
    \[
    \operatorname{Im}(\phi) = \{ \mathfrak{p} \in \operatorname{Spec}(k[x, y]) \mid \mathfrak{p} \subseteq (x, y) \}.
    \]
    
    Geometrically, this set corresponds to the collection of points of \( \mathbb{A}^2 \) whose closures contain the origin. Recall that explicitly, to take the closure of a point $\mf p$ in $\Spec R$ means to take the closed set of all prime ideals of $R$ containing $\mf p$. 
    This includes:
    \begin{itemize}
        \item the origin itself (the closed point corresponding to \( (x, y) \));
        \item the generic points of irreducible subvarieties (such as curves) passing through the origin;
        \item the generic point of \( \mathbb{A}^2 \) (corresponding to the zero ideal \( (0) \)).
    \end{itemize}
    In particular the image is not constructible. Recall that a constructible set is, by definition, a finite union of locally closed sets. Moreover, $\Spec R \to \A^2$ is a monomorphism but $\Spec \hat R \to \A^2$ is not a monomorphism. In particular, consider the local ring at the origin of a nodal cubic in $\P^2$. The formal completion separates the generic point of the local ring at the origin into two generic points of the two irreducible components of the branches of the node.
\end{example}

\subsection{Rational points, base change, and Weil restriction}
\begin{definition}
    Let $X$ be a scheme over a field $k$. The set of \textbf{$k$-rational }points of $X$, denoted $X(k)$, is defined as:
    \[
    X(k) := \operatorname{Hom}_{\text{Sch}/k}(\operatorname{Spec} k, X).
    \]
    More generally, for any $k$-algebra $R$, the set of $R$-points of $X$, denoted $X(R)$, is defined as:
    \[
    X(R) := \operatorname{Hom}_{\text{Sch}/k}(\operatorname{Spec} R, X).
    \]
\end{definition}

\begin{definition}
    Let \( L/K \) be a field extension. The \textbf{base change} of a variety \( X \) over \( K \) to \( L \), denoted \( X_L \), is defined as:
    \[
    X_L := X \times_{\Spec K} \Spec L.
    \]
    This construction allows us to study the variety \( X \) over the larger field \( L \).
    The set of \( L \)-points of \( X_L \) is given by:
    \[
    X_L(L) = \operatorname{Hom}_{\text{Sch}/L}(\operatorname{Spec} L, X_L).
    \] is canonically identified with the $L$-points of $X$.
\end{definition}

\begin{example}
    Let $L/K$ be an extension of fields and $X/K$ be a variety, Then we have the pullback square
    \begin{center}
        \begin{tikzcd}
            X(L) \arrow[r] \arrow[d] & X(K) \arrow[d] \\
            \Spec L \arrow[r] & \Spec K
        \end{tikzcd}
    \end{center}
    where if $A$ is a local model of $X$, then $A \otimes_K L$ is a local model of $X_L$. In particular, the set of $L$-points is the set of $K$-points of the base change. There is a map from the $K$-points of $X$ to the $L$-points of $X_L$ given by precomposing with $\Spec L \to \Spec K$.
    \begin{align*}
        X(K) &\to X_L(L) \\
        x &\mapsto x_L
    \end{align*} Let $S$ be a subset of the $K$-points of $X$. The image of $S$ in $X_L(L)$ is the set of $L$-points, denoted $S_L$.
    \begin{proposition}
        The Zariski closure of $S_L$ in $X_L$ is the base change of the Zariski closure of $S$ in $X$. 
    \end{proposition}
    Let $\Spec A/I \to \Spec A$ the Zariski closure of $S$ in $X$. Let $\Spec A_L/J \to Spec A_L$ be a local model for the Zariski closure of $S_L$ in $X_L$. We wish to show that $I_L = J$. Certainly $J \supset I_L$ because the polynomials in $I$ which witness the $K$-rational points of $X$ in $S$ will also vanish when you tensor up to $L$. Conversely, if $j\in A \otimes_K L$ vanishes at $S_L$, when we see that $j\in I_L$. Specifically, \begin{align*}
        0 = j(p) = \sum_{i=1}^n a_i(p) \otimes l_i
    \end{align*} for some finite sum, and after choosing a (maybe infinite basis) for $L$ over $K$ and collecting terms, we see that each of the finitely many coefficients $a_i(p)$ are in $I$. Therefore $j \in I_L$ and we have shown that $I_L = J$.
\end{example}

\begin{definition}
    Let $L/K$ a finite separable extension and $Y/L$ quasiprojective variety. The \textbf{Weil restriction} of $Y$ to $K$, denoted $\operatorname{Res}_{L/K} Y$, is the scheme over $K$ defined by the following universal property:
    \begin{align*}
        \operatorname{Res}_{L/K} Y(K) &\cong Y(L) \\
        \operatorname{Res}_{L/K} Y(R) &\cong Y(L \otimes_K R)
    \end{align*}
    for any $K$-algebra $R$. In other words, the set of $K$-points of the Weil restriction is isomorphic to the set of $L$-points of $Y$, and the set of $R$-points of the Weil restriction is isomorphic to the set of $L \otimes_K R$-points of $Y$.
\end{definition}

It is a fact that $\dim \operatorname{Res}_{L/K} Y = [L:K] \cdot \dim Y$ and that if $Y$ is quasiprojective over $K$ then $\operatorname{Res}_{L/K} Y$ is quasiprojective over $K$. \begin{example}
    If $Y = \A^1_L$ then $\operatorname{Res}_{L/K} Y = \A^n_K$ because giving an $L$ point of $\A^1_L$ is the same as giving $n$ $K$-points of $\A^1_K$. 
\end{example} On the level of $K$-algebras, the Weil restriction is given by picking a $K$-basis of $L$ and rewriting the $L$-algebra as a $K$-algebra. 

\begin{proposition}
    Weil restriction is right adjoint to base change. In other words, if $X/K$ is a variety and $Y/L$ is a variety, then we have the following isomorphism:
    \begin{align*}
        \operatorname{Hom}_{\text{Sch}/K}(X, \operatorname{Res}_{L/K} Y) &\cong \operatorname{Hom}_{\text{Sch}/L}(X_L, Y) 
    \end{align*}
\end{proposition}

\begin{proposition}
    \red{This is an exercise that I didn't solve.} Let \( L/K \) be a finite separable extension of fields, let \( Y/L \) be a quasiprojective variety, and let \( \operatorname{Res}^L_K Y \) be its Weil restriction to \( K \). Let \( T \subseteq Y(L) \) be a subset, let \( Z_2 \) be its Zariski-closure in \( Y \), and let \( Z_1 \) be its Zariski-closure in \( \operatorname{Res}^L_K Y \). Show that
    \[
    \dim_L(Z_2) \leq \dim_K(Z_1) \leq [L:K] \cdot \dim_L(Z_2).
    \]
\end{proposition}

The following example was relevant for the numerics of the period maps argument in the Lawrence Venkatesh paper.
\begin{example}
    [Dimension of the Lagrangian grassmannian] We will show that the Lagrangian grassmannian $LG(V)$ is a smooth projective variety of dimension $n(n-1)/2$ where $V$ is a symplectic vector space of dimension $2n$. In particular, there is an identification \begin{align*}
        T_P(LG(V)) &\cong \Sym^2(P^*)
    \end{align*} via deformation theory. Recall that in the ordinary Grassmannian one has \begin{align*}
        T_P(G(V,n)) &\cong \Hom(P,V/P)
    \end{align*} and we can use the form specified by $V$ to identify $V/P$ with $P^*$. Then an element $$\phi \in \Hom(P,P^*) \cong P^* \otimes P^*$$ lies in $T_P(LG(V))$ if and only if \begin{align*}
        \phi(x)(y) = \phi(y)(x) \text{ for all } x,y \in P
    \end{align*} so in particular $\phi$ is symmetric. This gives the identification with $\Sym^2(P^*)$.
\end{example} To argue this last point, we can model the tangent space $T_P(LG(V))$ by tangent vectors $\phi$ which are isotropic to first order. Let $\widetilde{\phi}:P \to V$ lift $\phi$. There is a deformed subspace \begin{align*}
    L_\varepsilon &= \set{v + \varepsilon \widetilde{\phi}(v) \mid v \in P} \subset V \otimes k[\varepsilon] / \varepsilon^2 
\end{align*} so the isotropic condition is \begin{align*}
    \omega(v + \varepsilon \widetilde{\phi}(v), w + \varepsilon \widetilde{\phi}(w)) &= 0 
\end{align*} for all $x,y \in P$. Inspecting the first order term gives \begin{align*}
    \omega(\widetilde{\phi}(v), w) + \omega(v, \widetilde{\phi}(w)) &= 0 
\end{align*}

Using the pairing write $\varphi(x)(y) = \omega(\tilde{\varphi}(x), y) $ and $ \varphi(y)(x) = \omega(\tilde{\varphi}(y), x) = -\omega(x, \tilde{\varphi}(y)).$ so the vanishing of the first order term is equivalent to the condition that $\phi$ is symmetric.
\end{document}