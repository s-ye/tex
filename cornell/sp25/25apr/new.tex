\documentclass[12pt]{article}
\usepackage[utf8]{inputenc}
\usepackage{amsmath, amsthm, amssymb, mathrsfs}
\usepackage{hyperref}
\usepackage{tikz-cd}

\title{Explaining the Identification \( H^1_{\text{\et}}(X, \mathbb{Z}_p) \cong T_p J(-1) \)}
\author{}
\date{}

\begin{document}

\maketitle

Let \( X \) be a smooth projective curve over an algebraically closed field \( k \) of characteristic different from \( p \), and let \( J = \operatorname{Pic}^0(X) \cong \operatorname{Jac}(X) \) be its Jacobian.

We aim to explain the canonical isomorphism:
\[ 
H^1_{\text{\et}}(X, \mathbb{Z}_p) \cong T_p J(-1)
\]
in a way that avoids defining one side in terms of the other.

\section*{1. Étale Cohomology and the Kummer Sequence}
Consider the Kummer exact sequence of étale sheaves:
\[
1 \to \mu_{p^r} \to \mathcal{O}_X^* \xrightarrow{(\cdot)^{p^r}} \mathcal{O}_X^* \to 1
\]
This induces a long exact sequence in étale cohomology:
\[
H^0(X, \mathcal{O}_X^*) \xrightarrow{(\cdot)^{p^r}} H^0(X, \mathcal{O}_X^*) \to H^1(X, \mu_{p^r}) \to H^1(X, \mathcal{O}_X^*) \xrightarrow{(\cdot)^{p^r}} H^1(X, \mathcal{O}_X^*)
\]
If \( k \) is algebraically closed and \( X \) is proper and connected, then \( H^0(X, \mathcal{O}_X^*) = k^* \), and the map \( (\cdot)^{p^r} \colon k^* \to k^* \) is surjective. Hence, the connecting homomorphism
\[
\delta \colon H^1(X, \mu_{p^r}) \to \operatorname{Pic}(X)
\]
is injective with image equal to \( \operatorname{Pic}(X)[p^r] \), giving an isomorphism:
\[
H^1(X, \mu_{p^r}) \cong \operatorname{Pic}(X)[p^r] \cong J[p^r](k)
\]

\section*{2. Passing to the Limit}
Taking the inverse limit over \( r \), we obtain:
\[
\varprojlim_r H^1(X, \mu_{p^r}) \cong \varprojlim_r J[p^r](k) = T_p J
\]
This gives an isomorphism:
\[
H^1(X, \mathbb{Z}_p(1)) := \varprojlim_r H^1(X, \mu_{p^r}) \cong T_p J
\]
Note: here we define \( \mathbb{Z}_p(1) := \varprojlim_r \mu_{p^r} \), and view this as a sheaf or Galois module with the natural action of \( \mathrm{Gal}(\overline{k}/k) \) (if applicable).

\section*{3. The Tate Twist}
The Tate twist \( (-1) \) corresponds to tensoring with the sheaf:
\[
\mathbb{Z}_p(-1) := \underline{\mathrm{Hom}}(\mathbb{Z}_p(1), \mathbb{Z}_p)
\]
This is the dual Galois module to \( \mathbb{Z}_p(1) \). Therefore, we get:
\[
H^1(X, \mathbb{Z}_p) \cong H^1(X, \mathbb{Z}_p(1)) \otimes \mathbb{Z}_p(-1)
\]
Combining with the earlier identification:
\[
H^1(X, \mathbb{Z}_p) \cong T_p J(-1)
\]

\section*{4. Conclusion}
We have constructed the isomorphism:
\[
H^1_{\text{et}}(X, \mathbb{Z}_p) \cong T_p J(-1)
\]
by:
\begin{itemize}
  \item Using the Kummer sequence to identify \( H^1(X, \mu_{p^r}) \cong \operatorname{Pic}(X)[p^r] \)
  \item Taking inverse limits to define \( H^1(X, \mathbb{Z}_p(1)) \cong T_p J \)
  \item Interpreting the Tate twist as dualizing the cyclotomic Galois module
\end{itemize}
This route avoids any circularity and builds the identification from geometric input.

\end{document}
