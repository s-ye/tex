Let $G$ act smoothly on a smooth complex algerbaic variety $X$. Let $E$ be a $G$-equivariant vector bundle on $X$.
There is a linearized action of $\mf g$ on the space of sections of $E$. First we will explain this when $E = TX$ is the tangent bundle.

\begin{theorem}
	There is a map $\exp:\mf g \to G$ which takes $X = [\gamma] \in \mf g$ to $\gamma(1) \in G$ where $\gamma:\R \to G$ is the unique one-parameter subgroup of $G$ with tangent vector $X$ at $1$.
\end{theorem}
Therefore, given an element $X \in \mf g$, we can define a vector field $X_X$ on $X$ by
how it acts on a function $f\in C^\infty(X)$:
\begin{align*}
	X_X(f)(p) = \frac{d}{dt} f(\exp(tX)p)|_{t=0} = \lim_{h\to 0} \frac{f(\exp(hX)p) - f(p)}{h}
\end{align*}
Another way of thinking about it is the map $\gamma(t)\cdot p:\R\to X$ is a path in $X$ whose tangent vector at time $0$ is precisely $X_X$.
This gives $\mf g$ acting on vector fields via bracket with $X_X$.

\hfill

Now let $E$ be a $G$-equivariant vector bundle on $X$. Given
$X\in \mf g$ we can define a section of $E$ by
\begin{align*}
	E_p \ni X_E(p) = \lim_{h\to 0} \frac{E_{\exp(hX)p} - E_p}{h}
\end{align*} but we crucially need to know how to transport $E_p$ to $E_{\exp(hX)p}$.
This is where the notion of a connection comes in. Basically we have a canonical choice of
connection because there is an action of $G$ on the total space of $E$ compatible with the projection to $X$.

\hfill

Morally an action of $G$ on $X$ is a map $G\to \Aut(X)$ which we can differentiate to get a map
$\mf g \to \Der(\cO_X)$. Vector bundles on $X$ correspond to $\cO_X$-modules, i.e. maps $\cO_X
	\to \Gamma(X,E)$, and so given $X\in \mf g$ we get a map $\cO_X\to \cO_X \to \Gamma(X,E)$, i.e.
we have an automorphism of the total space of $E$ which gets an an action on the sections.

\hfill

\textbf{What is the appropraite analog for the universal enveloping algebra for derivations of $\cO_X$?}
The answer is the ring of differential operators $\cD_X$ on $X$.

\subsection{Distributions}
Ultimately our goal is to define the pushforward of a $\cD$-module along a map of varieties.
The trouble with this is that we like to think of the
ring of differential operators as acting on functions, but functions don't pushforward. The correct
thing to let the ring of differential operators act on is distributions. In fact,
this parity is precisely the distinction between the left (functions) and right (distributions) $\cD$-modules.


\begin{example}
	\red{This is uncertain.}
	"Distributions are things which one can integrate functions against.
	Therefore, to specify a distribution, it suffices to specify how it integrates functions. For us,
	algebraic distributions will be the sections of top exterior power of the cotangent bundle."

	\hfill

	Another thing that you might hear is that distributions
	on a smooth manifold are smooth subbundles of $TM$. This makes sense because if you want to integrate
	a function $f$ over a manifold $M$, you integrate the $n$-form $f\,dx_1\wedge \cdots \wedge dx_n$ over
	the $n$-manifold. In fact, all of the $n$-forms on an oriented $n$-manifold look like this.
	"The top exterior power of the cotangent bundle is trivial on an oriented manifold." This gives an
	explanation why $\Omega_X$ is called the "canonical bundle" of $X$.

	\hfill

	There is a corresponding subbundle of $TX$ of corank $1$ given by the tangent bundle of the
	hypersurface $f=0$. This is the "delta function" supported on the hypersurface $f=0$ and is the
	corresponding smooth subbundle alluded to above.
\end{example}

\subsection{Pushforward}
Left $\cD$-modules are about functions. Right $\cD$-modules are about distributions.
We can go from functions to distributions by tensoring with the top exterior power of the cotangent bundle
(or its dual).

\hfill

Let $X\to Y$ be a closed embedding of smooth varieties. We want to define the pushforward
of a right $\cD_X$-module to a right $\cD_Y$-module. For simplicity we
introduce an object called the transfer module.

\begin{definition}
	The \red{transfer $\cD_X$-module} is defined by \begin{align*}
		\cD_{X\to Y} = f^*\cD_Y = \cO_X\otimes_{f^{-1}\cO_Y} \cD_Y
	\end{align*}
\end{definition}
where the $m_i$ are local coordinates, and the $\partial_i$ are the corresponding derivations.
As an $\cO_X$-module, it coincides with the standard pullback of $\cD_Y$ as an $\cO_Y$-module.
It carries a left $\cD_X$-module structure by the rule \begin{align*}
	\theta \cdot (a\otimes f^{-1}(m)) = \theta(a)\otimes f^{-1}(m) + a\sum_i \theta(f^{-1}(m_i)\otimes \partial_i m)
\end{align*} and a right $f^{-1}\cD_Y$-module structure as well.

\red{We need to define the pushforward for right $\cD_X$ modules because distributions push forward.}
\begin{definition}
	There is also a \red{transfer $\inv{f}(\cD_Y)$-$\cD_X$ bimodule going
		the other way} $\cD_{Y \leftarrow X}$ defined as \begin{align*}
		\cD_{Y\leftarrow X} = \Omega_X\otimes_{\cO_X} \otimes_{f^{-1}\cO_Y} \inv{f}\Omega_Y^{-1}
	\end{align*}
\end{definition}
This allows us to define the pushforward of a left $\cD_X$-module to a left $\cD_Y$-module.
\begin{definition}
	For \red{pushforward of left $\cD$-modules }we define \begin{align*}
		f_*M = f_*(\cD_{Y\leftarrow X}\otimes_{\cD_X} M)
	\end{align*}
\end{definition}
\red{Pushforward of $\cD$-modules is left exact (maybe?)}


\begin{definition}
	Let $f:X\to Y$ be a smooth map of smooth varieties.
	Let $M$ be a right $\cD_X$-module. The \textbf{pushforward} $f_*M$ is defined to be the tensor product
	$M\otimes_{\cD_X} \cD_{X\to Y}$. It is a right $\cD_Y$-module.
\end{definition}

\subsection{Pullback}

\begin{definition}
	To define the pullback we start with the pullback of $\cO$-modules. \begin{align*}
		f^*N = \cO_X\otimes_{f^{-1}\cO_Y} N
	\end{align*} It turns out to carry a left $\cD_X$-module structure in the same way as the transfer module.
	By the associatitivity of tensor product,
	we can also write it as \begin{align*}
		f_*N = \cD_{X\to Y}\otimes_{\cD_Y} N
	\end{align*} and therefore we have a right exact functor \begin{align*}
		f^*:\cD_Y\Mod \to \cD_X\Mod
	\end{align*}
\end{definition}

\subsection{Correspondence between left and right $\cD$-modules}
A differential operator $P(x,\partial) = \sum_{\alpha} a_\alpha(x)\partial^\alpha$ has a formal adjoint
$P^t(x,\partial)$ which is defined by \begin{align*}
	P^t(x,\partial) = \sum_{\alpha} (-\partial)^\alpha a_\alpha(x)
\end{align*} which satisfies $(PQ)^t = Q^tP^t$. The left action of differential operators on
functions is given by differentiation. The right action of diffrential operators distributions is given by
acting by the formal adjoint \begin{align*}
	(a(p)dx_1 \wedge \cdots \wedge dx_n)\cdot X := (P^t(X)a(p))(dx_1\wedge \cdots \wedge dx_n)
\end{align*}

\section{Deriving things}
\subsection{$\cD$-affineness}
\begin{definition}
	$X$ is \textbf{$\cD$-affine} if every quasicoherent $\cD_X$-module is the
	"twiddleification" of a $\Gamma(X,\cD_X)$-module.
\end{definition}
Every affine variety is $\cD$-affine. Projective spaces and
the flag variety for a reductive group are $\cD$-affine.

\begin{remark}
	Spaces which are $\cD$-affine are not necessarily $\cD^{op}$-affine. For example,
	\begin{align*}
		\Gamma(\P^1,\cD_{\P^1})      & = U(\mf{sl}_2)/Z(U(\mf{sl}_2)) \text{\red{ would like more explicitly}} \\
		\Gamma(\P^1,\cD_{\P^1}^{op}) & = \Gamma(\P^1,\Omega_{\P^1}) = 0
	\end{align*} since the anticanonical sheaf of $\P^1$ is negative.
\end{remark}

\begin{proposition}
	\begin{enumerate}
		\item $\cD_X$ is a coherent sheaf of rings.
		\item $\cD_X$-modules are coherent if and ony if they are quasicoherent over $\cO_X$ and
		      finitely generated over $\cD_X$.
		\item $\cD_X$-modules are coherent if and only if they are integrable connections (need
		      locally freeness) in which case taking $\Spec$ gets us a vector bundle with integrable
		      connection over our variety $X$.
	\end{enumerate}
\end{proposition}

\subsection{Derived pullback}
Recall that $f^*:\cD_Y\Mod \to \cD_X\Mod$ is a right exact functor. Therefore,
homological algebra tells we get left derived functors $Lf^*:D(\cD_Y\Mod) \to D(\cD_X\Mod)$.
Assume that our varieties are quasiprojective. Then this guarentees that
the following story works:

\begin{proposition}
	Let $M$ be a quasicoherent $\cD_X$-module. Then there
	exist two resolutions
	\begin{align*}
		\cdots \to F_1 \to F_0 \to M \to 0 \\
		0 \to P_n \to \cdots \to P_0 \to M \to 0
	\end{align*} where the $F_i$ are locally free and the $P_i$ are locally projective.
\end{proposition}
The existence of finite projective resolutions tells us that $Lf^*$ restricts to the bounded
derived category $D^b(\cD_Y\Mod) \to D^b(\cD_X\Mod)$.


\begin{proposition}
	$Lf^*$ moreover preserves quasicoherent cohomology sheaves. It does not
	preserve coherent cohomology sheaves because you can pick up infinite rank.
\end{proposition}

\begin{definition}
	The shifted derived pullback is defined by \begin{align*}
		Lf^\dagger M = Lf^*[\dim X - \dim Y]
	\end{align*} is more natural in the Riemann Hilbert correspondence.
\end{definition}

\begin{proposition}
	Let $f:X\to Y$ be a smooth map of smooth varieties and $M\in \Mod(\cD_Y)$. Then
	$Lf^*M$ has no higher cohomology. If $M$ is coherent, then the nonzero part (at most 1)
	$Lf^*M$ is coherent.
\end{proposition}

Recall that a map $i:X\to Y$ of varieties
makes $\cO_X$ into a $i^{-1}\cO_Y$-module and that
the inverse image functor in the category of $\cO_Y$-modules is defined by \begin{align*}
	i^*\cG := i^{-1}\cG \otimes_{i^{-1}\cO_Y} \cO_X
\end{align*} because $i^{-1}\cG$ is a sheaf of $i^{-1}\cO_Y$-modules and
we have a ring homomorphism $i^{-1}\cO_Y \to \cO_X$ coming from the adjoint to $i_*\cO_X \to \cO_Y$.
In particular, if we have $f:\Spec A \to \Spec B$ then $f^*M = M\otimes_B A$.

\begin{definition}
	Define the inverse image functor for $\cD$-modules by \begin{align*}
		i^\natural : \Mod(\cD_Y) \to \Mod(\cD_X)
	\end{align*} by the formula \begin{align*}
		i^\natural M = \Hom_{i^{-1}\cD_Y}(\cD_{Y\leftarrow X},\inv{i}M)
	\end{align*}
\end{definition}
Recall the shifted inverse image functor $i^\dagger  = Li^*[\dim X - \dim Y]$.

\begin{proposition}
	$i^\natural$ is left exact. For closed embeddings $i$ we have $i^\dagger M\cong Ri^\natural M$.
\end{proposition}

\subsection{Derived pushforward}
There are functors \begin{align*}
	D^b(\cD_X) \to D^b(f^{-1}\cD_Y) \to D^b(\cD_Y\Mod)
\end{align*}
which take $M \mapsto \cD_{Y\leftarrow X}\otimes_{\cD_X}^L M$ and $N\mapsto Rf_*N$ respectively.
Their composition is \textbf{the derived directed image }denoted \begin{align*}
	\int_f M: D^b(\cD_X) \to D^b(\cD_Y)
\end{align*} We state some of the properties of this functor.
\begin{proposition}
	\hfill
	\begin{enumerate}
		\item $\int_f$ preserves quasicoherence, and if $f$ is proper, it preserves coherence.
		\item $\int_f = Rj^*$ if $j$ open.
		\item The direct image is locally exact. In particular if $i:X\to Y$ closed and $M\in\Mod(\cD_X)$ then
		      $\int_f^i := H^i(\int_f M)$ is zero for all $i\neq 0$ and $\int_f^0$ is quasicoherent if $M$ is.
		\item $i^\natural$ and $\int_f$ are adjoints.
	\end{enumerate}
\end{proposition}

\subsection{Kashiwara's equivalence}
For closed embedding $X\to Y$ we saw that $\int_f^0$ is exact and $\int_f^0 M$ is a $\cD_Y$-module
supported on $X$. \red{Let the category of such $\cD_Y$-modules be denoted $\Mod^X(\cD_Y)$.}

\begin{theorem}
	[Kashiwara's equivalence] The functor $\int_f^0$ induces an equivalence of categories \begin{align*}
		\Mod_{qc,c}(\cD_X) \to \Mod^X_{qc,c}(\cD_Y)
	\end{align*}
	with quasiinverse $i^\natural$.
\end{theorem}

\subsection{Base change theorem}
The following story really captures the essence of derived categories. Let \begin{align*}
	Z \xrightarrow{i} X \xleftarrow{j} U
\end{align*} be complements and $F$ an injective sheaf of $X$. There is a short exact sequence \begin{align*}
	0 \to \Gamma_Z F \to F \to j_*j^{-1}F \to 0
\end{align*} which gives a distinguished triangle in the derived category for every $F^\bullet \in D^b(\C_X)$
\begin{align*}
	R\Gamma_Z(F^\bullet) \to F^\bullet \to Rj_*j^{-1}F^\bullet \to +1
\end{align*}
\red{We have the following abstract nonsense.}
\begin{proposition}
	\hfill
	\begin{enumerate}
		\item For $M^\bullet \in D^b_{qc}(D_X)$ we have a canonical distinguished triangle \begin{align*}
			      R\Gamma_Z(M^\bullet) \to M^\bullet \to \int_j j^\dagger M^\bullet \to +1
		      \end{align*}
		\item $Z$ smooth and $M\in D^b_{qc}(D_U)$ then \begin{align*}
			      i^\dagger\int_j M = 0
		      \end{align*}
		\item Z smooth and $M^\bullet\in D^b_{qc}(D_X)$ then \begin{align*}
			      R\Gamma_Z(M^\bullet) = \int_i i^\dagger M
		      \end{align*}
	\end{enumerate}
\end{proposition}
\section{Coherent $\cD$-modules}
Coherent $\cD$-modules can be studied with the geometry of $T^*X$.
\subsection{Characterstic cycles}
If $M$ is a coherent $\cD_X$ module, choose a good filtration on it and then
$\gr^F M$ is a coherent $\pi^*\cO_{T^*X}$-module.

\begin{definition}
	Let $\tilde{\gr^F M} = \gr^F M\otimes_{\pi^{-1}\cO_{T^*X}} \cO_{T^*X}$. The support of this module
	is $\Ch(M)$ \red{the characterstic variety of $M$}. It does not depend on the choice of good filtration.
\end{definition}

\begin{remark}
	We can also take cycle invariants, which also do not depend on the choice of good filtration. Explicitly,
	let $C$ be a irreducible component of the characteristic variety of $M$. Then the cycle invariant comes from taking the
	multiplicity of the component $C$ as follows.

	\hfill

	Pick an open set $U$ so that $\overline{C\cap U} = C$. \red{$U$ is a rational coordinate chart.} $C\cap U$ is closed
	in $U$ so take its defining ideal $\mf p$. Then we
	have a local ring $\cO_U(U)_{\mf p}$ and a module over this local ring $M_{\mf p}$.
	They are the stalks of $ \cO_X$ and $M$ at the generic point of $C$ respectively. By commutative algebra,
	the length of $M_{\mf p}$ as a module over $\cO_U(U)_{\mf p}$ is finite and defined as the multiplicity of $C$.
\end{remark}

\begin{example}
	When $M$ is an integrable connection, the following are equvialent:
	\begin{enumerate}
		\item $M$ is an integrable connection.
		\item $M$ is coherent over $\cO_X$
		\item $\Ch(M) = $ zero section of $T^*X$.
	\end{enumerate}
\end{example}

\subsection{Holonomic $\cD$-modules}
\begin{definition}
	A $\cD_X$-module $M$ is \textbf{holonomic} if $\dim \Ch(M) = \dim X$ is as large as possible.
\end{definition}
Character varietyies of holonomic $\cD$-modules are $\C^*$ invariant Lagrangian subvarieties of $T^*X$.
\begin{example}
	Integrable connections are holonomic.
\end{example}
\begin{example}
	Let $i:X\to Y$ be closed smooth subvariety. Let \begin{align*}
		B_{X\vert Y} := \int_i^0 \cO_X \in \Mod^X_{qc}(D_Y)
	\end{align*}
	Then $B_{X\vert Y}$ is holonomic and $\Ch(B_{X\vert Y}) = T^*_XY$ the conormal bundle of $X$ in $Y$.
	When $X = *$ we can apply Kashiwara equivalence to get that \begin{align*}
		\Mod_{qc}^*(D_Y) \cong \Vect_\C \\
		B_{*\vert Y} \iff \C
	\end{align*} which implies that objects in the left hand side are direct sums of $B_{*\vert Y}$.
\end{example}

\subsection{Duality functors}
We begin with a motivating example. Let $X = \C$
and $M = \cD_X/\cD_XP$ where $P\neq 0$. There is a short exact sequence \begin{align*}
	0 \to \cD_X \xrightarrow{P} \cD_X \to M \to 0
\end{align*} which gives \begin{align*}
	0 \to \Hom_{\cD_X}(M,\cD_X) \to D_X \xrightarrow{P} D_X \to \Ext^1_{\cD_X}(M,\cD_X) \to 0
\end{align*} and so we see that \begin{align*}
	\Ext^0(M,\cD_X) = \Hom_{\cD_X}(M,\cD_X) = 0 \\
	\Ext^1(M,\cD_X) = D_X/PD_X
\end{align*} Sidechanging $\Ext^1$ gives us the left $\cD$-module $D_X/P^*D_X$.

\red{$\Ext^1$ is more suited to be called the dual than $\Ext^0$}. In general,
if $n = \dim X$ and $M$ is a holonomic $\cD_X$-module, then \begin{align*}
	\Ext^n_{\cD_X}(M,\cD_X)
\end{align*} is the only nonzero $\Ext$ group and the left $\cD_X$-module \begin{align*}
	\D M := \Ext^n_{\cD_X}(M,\cD_X) \otimes \Omega_X^{-1}
\end{align*} is also holonomic. For a non-holonomic $M$, the other $\Ext$ groups are generally nonzero
and therefore $\D$ is defined in the derived category:

\begin{definition} We define the \red{duality functors}
	\begin{align*}
		\D: D^-(\cD_X) \to D^+(\cD_X)^{op} \\
		\D M := R\Hom_{\cD_X}(M,\cD_X) \otimes \Omega_X^{-1}[\dim X]
	\end{align*}
\end{definition}
\begin{example}
	We have the example $H^k(\D \cD_X) = D_X\otimes \Omega_X^{-1}$ if $k = -\dim X$ and $0$ otherwise.
\end{example}
\begin{example}
	$ \D B_{X\vert Y} = B_{Y\vert X}$
\end{example}
\begin{example}
	$\D M = \Hom_{\cO_X}(M,\cO_X)$ if $M$ is an integrable connection.
\end{example}
\begin{proposition}
	If $X$ is a smooth variety and $M$ is a coherent $\cD_X$-module, then \begin{align*}
		\Ch(M) = \bigcup_{d_X\leq i\geq 0} \Ch(\Ext^i_{\cD_X}(M,\cD_X)\otimes_{\cO_X}\Omega_X^{-1})
	\end{align*} In particular $\Ch(M) = \Ch(\D M)$ if $M$ is holonomic.
\end{proposition}

\section{Holonomic $\cD$-modules}
Holonomic $\cD$-modules enjoy very nice properties. We learned about all of the
previous functors and constructions in order to construct examples of holonomic $\cD$-modules.

\subsection{Basic results}
\begin{proposition}
	\begin{enumerate}
		\item
	\end{enumerate}
	An exact sequence of coherent $\cD_X$-modules \begin{align*}
		0 \to M' \to M \to M'' \to 0
	\end{align*} implies that $M$ is holonomic if and only if $M$ and $M''$ are holonomic.
\end{proposition}

\begin{proposition}
	Let $M$ be a holonomic $\cD_X$-module. Then there
	exists an open $U\subset X$ for which $M|_U$ is coherent over $\cO_U$.
	In other words, $M$ is generically a integrable connection.
\end{proposition}

\begin{definition}
	Let $D^b_h(\cD_X)$ be the full subcategory of $D^b(\cD_X)$ consisting of complexes with
	holonomic cohomology sheaves.
\end{definition}

\begin{remark}
	Beilinson proved that you can consider the category with holonomic $\cD$-modules as objects and the answer is the same.
	\begin{align*}
		D^b_h(\cD_X) \cong D^b(\Mod_h(\cD_X))
	\end{align*}
\end{remark}

\begin{proposition}
	Duality preserves holonomicity. In particular,
	\begin{align*}
		\D_X:\Mod_h(\cD_X) \to \Mod_h(\cD_X)^{op} \\
		\D_X: D^b_h(\cD_X) \to D^b_h(\cD_X)^{op}
	\end{align*} are equivalences of categories.
\end{proposition}

Recall for morphism $f:X\to Y$ of smooth varieties we have the \red{direct and inverse image} functors \begin{align*}
	\int_f: D^b_{qc}(D_X) \to D^b_{qc}(D_Y) \\
	f^\dagger: D^b_{qc}(D_Y) \to D^b_{qc}(D_X)
\end{align*}

Moreover, we have proper pushforward and smooth pullback of coherent $\cD$-modules (but not for general maps!)
Remarkably however, holonomicity is preserved by these functors for general maps.

\begin{theorem}
	Let $f:X\to Y$ be a morphism of smooth varieties. Then \begin{align*}
		\int_f: D^b_h(D_X) \to D^b_h(D_Y) \\
		f^\dagger: D^b_h(D_Y) \to D^b_h(D_X)
	\end{align*}
\end{theorem}

\begin{proof}
	It is enough to prove that $\int_f$ preserves holonomicity when $f$ is the projection $\C^n\to \C^{n-1}$. We will explain this
	in a bunch of steps because it is not obvious at all.

	\hfill

	First, it's true for closed embeddings. In particular if $i:X\to Y$ is a closed embedding, then
	\begin{align*}
		M^\bullet\in D^b_h(\cD_X) \iff \int_i M^\bullet \in D^b_h(\cD_Y)
	\end{align*} This is not hard. Then we factor $f$ as a composition of a closed embedding and a projection by using
	its graph. Now we just have to show the claim for projections. Reduce to local arguments and
	then we see that it is enough to prove the claim for $\C^n\to \C^{n-1}$.

	\hfill

	Finally the claim about pushforwards implies the claim about pullbacks generally.
\end{proof}

It remains to prove that the pushforward along the projection $\C^n\to \C^{n-1}$ preserves holonomicity.

\subsection{Holonomicity of modules over the Weyl algebra}
Let $D_n$ be the Weyl algebra in $n$ variables.
\begin{definition}
	Let $N$ be a $D_n$-module. The \red{Fourier transform} of $N$ is the $D_n$-module $\hat N$ which is the same abelian group,
	but the generators $x_i,\partial_i \in D_n$ act differently.
	\begin{align*}
		x_i \cdot f = -\partial_i f \\
		\partial_i \cdot f = x_i f
	\end{align*} is an involution.
\end{definition}
\begin{proposition}
	Let $p:\C^n\to \C^{n-1}$ be the projection and $i:\C^{n-1}\to \C^n$ the zero section. Then \begin{align*}
		H^k\int_p M \cong H^k Li^*\hat{M}
	\end{align*}
	for any $k$ and any $D_n$-module $M$.
\end{proposition}

\begin{proposition}\label{prop:fourier_transform_holonomic}
	A coherent $D_n$ module is holonomic if and only if its Fourier transform is.
\end{proposition}

\begin{proposition}\label{prop:holonomicity_pushforward}
	Let $j:\C^*\times\C^n\to\C^{n+1}$ the inclusion. If $M$ is a holonomic $D_n$-module then so is \begin{align*}
		H^0\int_j j^\dagger M
	\end{align*}
\end{proposition}
These propositions, which we will prove next, immediately imply the theorem. The distinguished triangle
\begin{align*}
	\int_i i^\dagger M \to M \to \int_j j^\dagger M \to +1
\end{align*} gives an exact sequence \begin{align*}
	0 \to H^0\int_i i^\dagger M \to M \to H^0\int_j j^\dagger M \to H^1\int_i i^\dagger M \to 0
\end{align*} and note that $H^0\int_j j^\dagger M$ is holonomic by the proposition. Therefore, $i^\dagger M$ is holonomic
because $H^0$ and $H^1$ are (holonomicity behaves really well under extension).

\subsection{Hilbert series}
This section is devoted to the proof of the above propositions.
Fix the Bernstein filtration on $D_n$ (filtered by degree of symbols) and we define a notion of a \red{good filtration} on a $D_n$-module $M$ with
respect to the Bernstein filtration.

\hfill

Any finitely generated $D_n$ module has a good filtration with respect to the Bernstein filtration.
\red{The advantage is that for a good filtration on $M$ with respect to the Bernstein filtration,
	every $F^iM$ is finite dimensional over $\C$.} Therefore we can apply results on Hilbert polynomials to the associated graded $\gr^B D_n$
module.

\begin{proposition}
	The function \begin{align*}
		i \mapsto \dim_\C F^iM
	\end{align*} is eventually polynomial and will be denoted $\chi(M,F)$.
	The highest degree term is $d$ and the leading coefficient is $m/d!$. The integers $d$ and $m$
	do not depend on the choice of the good filtration $F$.
\end{proposition}
\red{$d$ is the dimension and $m$ is the multiplicity.}

$d$ is also the dimension of the characterstic variety of $M$ in this case.
\begin{proposition}
	If you find an integer $c$ so that \begin{align*}
		\dim_C F^iM \leq c/n! i^n + O(i^{n-1})
	\end{align*} then $M$ is holonomic and $m(M)\leq c$.
\end{proposition}

\begin{proof}
	[Proof of Proposition \ref{prop:fourier_transform_holonomic}]
	$N$ and $\hat N$ have the same Hilbert polynomial.
\end{proof}

\begin{proof}
	[Proof of Proposition \ref{prop:holonomicity_pushforward}]
	Let $N = \Gamma(\C^n,M)$ and note that $\Gamma(\C^n,H^0\int_j j^\dagger M)$ is
	isomorphic to the localization $N_{x_i} = \C[x,x_i^{-1}]\otimes_{\C[x]} N$.
	Hence it is sufficient to show that the localization is
	holonomic. Kashiwara does this by pickiing a filtration on $N_{x_i}$ and then
	finding an integer $c$ as in the proposition above.
\end{proof}

\subsection{Exceptional functors}
Let $f:X\to Y$ be a morphism of smooth algebraic varieties.
\begin{definition}
	We have the \red{exceptional pushforward and pullback} \begin{align*}
		\int_{f!} := \D_Y\circ \int_f \circ \D_X : D^b_h(\cD_X) \to D^b_h(\cD_Y) \\
		f^\star := \D_X\circ f^\dagger \circ \D_Y : D^b_h(\cD_Y) \to D^b_h(\cD_X)
	\end{align*}
\end{definition}

\begin{theorem}
	There exists a morphism of functors \begin{align*}
		\int_{f!}\to \int_f: D^b_h(\cD_X) \to D^b_h(\cD_Y)
	\end{align*} which is an isomorphism if $f$ is proper. \red{This is some compactly supported thing
		in the spirit of Borel Moore homology and Poincare duality.}
\end{theorem}

\subsection{Stratifications and minimal extensions}
\begin{theorem}
	The following conditions on $M^\bullet \in D^b_c(\cD_X)$ are equivalent:
	\begin{enumerate}
		\item $M^\bullet$ is holonomic.
		\item There eexists a decreasing sequence \begin{align*}
			      X = X_0 \supset X_1 \supset \cdots \supset X_m \subset X_{m+1} = \emptyset
		      \end{align*} of closed subsets of $X$ so that $X_r\setminus X_{r+1}$ is smooth and
		      all of the cohomology sheaves $H^k(i_r^\dagger M^\bullet)$ are integrable connections,
		      where $i_r:X_r\setminus X_{r+1}\to X$ is the inclusion.
		\item For any $x\in X$ all of the cohomology groups $H^k(i_x^\dagger M^\bullet)$ are finite dimensional where
		      $i_x:\{x\}\to X$ is the inclusion.
	\end{enumerate}
\end{theorem}
\red{ $M$ coherent is called simple if it contains no coherent $\cD$-submodules other than $M$ or $0$. }Since
the holonomic $\cD$-modulse form an Artinian category, we have a Jordan Holder series for each coherent $M$.
We will construct simple holonomic $\cD$-modules from integrable connections on locally
closed subvarieties using functors introduced in the previous section, and all of them are of this type.
This construction corresponds under RH correspondence to the minimal DGM extension
in the category of perverse sheaves.

\begin{definition}
	Let $f:X\to Y$ be the inclusion of a locally closed smooth subvariety
	and assume that $f$ is an affine map. Recall that there is a map
	\begin{align*}
		\int_{f!}M \to \int_f M
	\end{align*} which is an isomorphism if $f$ is proper.
	We call the image of this map the \red{minimal extension} $L(Y,M)$ of $M$.
\end{definition}

\begin{theorem}
	\hfill
	\begin{enumerate}
		\item  Let $Y$ be a locally closed smooth connected subvariety of $X$
		      so that $i:Y\to X$ is an affine map. Let $M$ be a
		      simple holonomic $\cD_Y$-module. Then $L(Y,M)$ is a simple holonomic
		      $\cD_X$-module and it is the unique simple submodule of $\int_i M$.
		\item Any simple holonomic $\cD_X$-module is isomorphic to $L(Y,M)$ for some
		      locally closed smooth connected subvariety $Y$ of $X$ and some simple holonomic $\cD_Y$-module $M$.
		\item Two such pairs $(Y,M)$ and $(Y',M')$ give isomorphic simple holonomic $\cD_X$-modules if and only if
		      $\overline Y = \overline Y'$ and $M\cong M'$ on an open dense subset of $Y\cap Y'$.
	\end{enumerate}
\end{theorem}

\section{Analytic $\cD$-modules}
Let $f:X\to Y$ be a morphism of complex manifolds. All the previous theory can be carried out in the analytic category.
\subsection{Solution complex and de Rham functors}
For $M^\bullet \in D^b(\cD_X)$ we have the \red{solution complex} \begin{align*}
	\Sol(M^\bullet) = R\Hom_{\cD_X}(M^\bullet,\cO_X)
\end{align*} which gives a functor \begin{align*}
	D^b(\cD_X) \to D^b(\C_X)^{op}
\end{align*}

We have the \red{de Rham complex} \begin{align*}
	\DR(M^\bullet) = \cO_X\otimes_{\cD_X} M^\bullet
\end{align*} which gives a functor \begin{align*}
	D^b(\cD_X) \to D^b(\C_X)
\end{align*} The motivation for introducing the solutino complex
comes from linear PDEs. For a coherent $\cD_X$-module $M$, the sheaf $\Hom_{\cD_X}(M,\cO_X)$ is the sheaf of solutions
to the system of linear PDEs defined by $M$.

\begin{proposition}
	For $M^\bullet \in D^b_c(\cD_X)$ we have \begin{align*}
		\DR_X(M^\bullet) \cong \Sol_X(\D_X M^\bullet)[\dim X]
	\end{align*}
\end{proposition}
The two complexes are related via duality but the functor $\DR_X$ has
the advantage that it can be computer using a resoplution of the right
$\cD_X$-module $\Omega_X$. We have a locally free resolution
\begin{align*}
	0 \to \Omega_X^0 \otimes_{\cO_X} \cD_X \to \dots \to \Omega_X^{\dim X} \otimes_{\cO_X} \cD_X \to \Omega_X \to 0
\end{align*} For $M\in\Mod(\cD_X)$ the object
$\DR_X(M)[-\dim X]$ is represented in the derived category by the complex \begin{align*}
	\Omega_X^0\otimes_{\cO_X} M \to \dots \to \Omega_X^n\otimes_{\cO_X} M
\end{align*} has differential \begin{align*}
	d^p: \Omega_X^p\otimes_{\cO_X} M \to \Omega_X^{p+1}\otimes_{\cO_X} M \\
	d^p(\omega\otimes s) = d\omega\otimes s + \sum_i dx_i \wedge \omega \otimes \partial_i s
\end{align*} where $x_i,\partial_i$ are local coordinates on $X$.

\subsection{Integrable connections and local systems}
Suppose $M$ is an integrable connection of rank $m$. Consider
the $0$th cohomology sheaf $L$ of the solution complex $\Sol(M)$.
Then $L$ coincides with the kernel of the sheaf homomorphism \begin{align*}
	d^0 = \nabla: M = \Omega_X^0 \otimes_{\cO_X} M  \to \Omega_X^1 \otimes_{\cO_X} M
\end{align*} which is
the sheaf of horizontal sections of $M$. \begin{align*}
	M^{\nabla} =  \set{s\in M \st \nabla s = 0} = \set{s\in M \st \Theta_X s = 0}
\end{align*} It is a locally free $\C_X$ module of rank $m$ by the classical Frobenius theorem.

\begin{definition}
	A locally free $\C_X$ module of finite rank is a \textbf{local system} on $X$.
	Let $\Loc(X)$ be the category of local systems on $X$.
\end{definition}

Using the local system $L$, we can define an integrable connection $M$
by $M = \cO_X\otimes_{\C_X} L$. I still have to tell you what the operator is: \begin{align*}
	\nabla = d\otimes \id_L : M = \cO_X\otimes_{\C_X} L \to \Omega_X^1\otimes_{\cO_X} \cO_X\otimes_{\C_X} L = \Omega_X^1\otimes_{\cO_X} M
\end{align*} Passing to $\Omega_X^\bullet$ with similarly defined differentials
in the tensor product, we study
the higher cohomology groups $H^i(\Omega_X^\bullet\otimes_{\cO_X} M)$. All of the higher ones
vanish by the \red{holomorphic Poincare lemma} and therefore we get an isomorpism \begin{align*}
	\Omega_X^\bullet\otimes_{\cO_X} M \cong M^\nabla
\end{align*}

This gives a functor \begin{align*}
	H^{-\dim X}(\DR_X(\cdot)): \Conn(X) \to \Loc(X)
\end{align*} which is an equivalence of categories.

\begin{remark}
	Given an integrable connection $M$, pick coordinates locally. Then $\nabla$
	looks like the standard derivative operator, and so $\nabla s = 0$ is the condition
	that $s$ is constant function. Locally there is a vector space of constant functions
	of dimension $\rank m$. This is precisely the correspondence
	between integrable connections and local systems. $\nabla$ is flat meaning it
	has no curvature and therefore, the transport of sections along paths depends
	only on the homotopy class of the path. Therefore, you can study how sections
	of $M$ vary along homotopy classes of loops in $X$.

	\hfill

	This is called the \red{monodromy representation} of $M$.\begin{align*}
		\pi_1(X) \to \Aut(L)
	\end{align*}
\end{remark}
\subsection{Kashiwara constructibility theorem}
\begin{theorem}[Kashiwara]
	The solution complex $\Sol_X(M)$ of a holonomic $\cD_X$-module $M$
	is a construtable complex, i.e. all of its cohomology sheaves are
	constructible.
\end{theorem}


\section{Meromorphic connections}
We begin a study of meromorphic connections and Deligne's Riemann-Hilbert correspondence.

\subsection{Meromorphic connections}
We start from the classical theory of ODEs. We are considering open
neighborhoods of $0\in \C$ as a complex manifold. Let \begin{align*}
	\cO = (\cO_\C)_0 = \text{ convergent power series at } 0 = \C\{x\} \\
	K = \text{ quotient field of } \cO = \C\{x\}[x^{-1}]
\end{align*}

Let $A(x)\in M_n(K)$ and consider the system of ODEs \begin{align*}
	\frac{d}{dx}\hat{u}(x) = A(x)\hat{u}(x)
\end{align*} where $\hat{u}(x)\in M_n(K)$ is a column vector. Given an invertible
$T\in \GL_n(K)$ we can change the basis of $A$ and rewrite the system as \begin{align*}
	\frac{d}{dx}\hat{v}(x) = (T^{-1}A(x)T - T^{-1}\frac{d}{dx}T)\hat{v}(x)
\end{align*} and so we say that
two systems are equivalent if they are related by a such a transformation.

\hfill

\red{Let $\tilde K$ denote the ring of possibly multivalued
	holomorphic functions defined on a punctured disk $B_\varepsilon^*$ for sufficiently small $\varepsilon$.}
We say that $\hat{u}$ is a solution of the system if $\hat{u} \in \tilde K^n$.

\begin{definition}
	Let $M$ be a finite dimensional $K$-vector space with a $\C$-linear map
	$\nabla: M\to M$. Then $M,\nabla$ is a \textbf{meromorphic connection} if
	\begin{align*}
		\nabla(fu) = \frac{df}{dx}u + f\nabla u
	\end{align*} for all $f\in K$ and $u\in M$. It is enough to check that
	this holds for $f\in \cO$.
	A morphism of meromorphic connections is a $K$-linear map $M\to N$ commuting with $\nabla$.
\end{definition}
Meromorphic connections naturally form an abelian category. Given
a meromorphic connection $M,\nabla$ the vector spcae $M$ is a left $(\cD_\C)_0$-module
by the action $\partial u = \nabla u$. Note that $\nabla$ also extends uniquely to an element of
$\End_\C(\tilde K \otimes_K M)$. We say that $u\in \tilde K\otimes_K M$ is a \red{
	horizontal section} if $\nabla u = 0$.

\hfill

Let $M$ be a meromorphic connection and choose a $K$-basis $e_1,\dots,e_n$ of $M$. Then
the change of basis matrix with respect to $\nabla$ is: \begin{align*}
	\nabla e_j = \sum_i A_{ij}e_i
\end{align*} \red{"the connection matrix"}
The condition $\nabla u = 0$ for $u = \sum_i u_i e_i \in \tilde K \otimes_K M$ is equivalent
to the system of ODEs \begin{align*}
	\frac{d}{dx}u = A(x)u
\end{align*} for the connection matrix $A$.
\subsection{Regular singularities}
Set $S^\varepsilon_{a,b} = \set{x\in \C\st |x| < \varepsilon, a < \theta < b}$
an open angular sector of the universal cover of $\C^*$. We say that
a function $f\in \tilde K$ has moderate growth (also called Nilsson class) at $x=0$
if it satisfies the following condition.

\begin{center}
	For every $a,b$ and $\varepsilon$ so that $f$ is defined on $S^\varepsilon_{a,b}$, \\
	there is a constant $C>0$ and $N>>0$ so that $|f(x)|\leq C|x|^{-N}$ for all $x\in S^\varepsilon_{a,b}$.
\end{center}

\red{Denote such functions by $\tilde K_{\text{mod}}$.
}
\subsubsection{Important example}
The system of ODEs \begin{align*}
	\frac{d}{dx}u = \frac{A}{x}u
\end{align*} for $A\in M_n(K)$ has a set of solutions $u\in\tilde K^n$
which is in fact a $\C$-vector space of dimension $n$.
We can take $n$ linearly indepdent solutions $u_1,\dots,u_n$ and call the matrix
$S = (u_1,\dots,u_n)$ the \red{fundamental matrix solution} of the system. \red{Since
	the analytic continuation of $S(x)$ along a cirlce around $0$ is again a solution matrix}
there exists some $G\in \GL_n(\C)$ so that $S(x) = S(0)G$ for all $x\in \C^*$. $G$
is called the \red{monodromy matrix} of the system.

\begin{theorem}
	The following three conditions on the ODE system are equivalent:
	\begin{enumerate}
		\item The system is equivalent to a system like \begin{align*}
			      \frac{d}{dx}u = \frac{A(x)}{x}u
		      \end{align*} for $A(x)\in M_n(\cO)$.
		\item The system is equivalent to a system like \begin{align*}
			      \frac{d}{dx}u = \frac{A}{x}u
		      \end{align*} with $\C$-entries.
		\item All solutions in $\tilde K^n$ belong to $\tilde K_{\text{mod}}^n$.
	\end{enumerate}
\end{theorem}

We expand on what it means for the analytic continuation to introduce monodromy.

\begin{example}
	The most basic example is
	\begin{align*}
		A = 1/x\begin{pmatrix}
			       1 & 0 \\
			       0 & 1
		       \end{pmatrix}
	\end{align*}
	The solutions are $u_1 = x$ and $u_2 = x$ and so the fundamental matrix solution is
	\begin{align*}
		S(x) = \begin{pmatrix}
			       x & 0 \\
			       0 & x
		       \end{pmatrix}
	\end{align*} The monodromy matrix is $G = \begin{pmatrix}
			1 & 0 \\
			0 & 1
		\end{pmatrix}$ because nothing happens when you go around $0$. See the
	next example for a more interesting case with the details that I didn't write here.
\end{example}

\begin{example}
	Consider taking \begin{align*}
		A(x) = 1/x\begin{pmatrix}
			          1 & 1 \\
			          0 & 1
		          \end{pmatrix}
	\end{align*} The off-diagonal entries introduce $\log$ terms into the fundamental
	matrix solution and therefore we pick up monodromy. In particular, we can solve
	this ODE with seperation of variables to get \begin{align*}
		u_1 = C_1x + C_2x\log x \\
		u_2 = C_2x
	\end{align*} so we have a fundamental solution matrix \begin{align*}
		S(x) = \begin{pmatrix}
			       x & x\log x \\
			       0 & x
		       \end{pmatrix}
	\end{align*} Now we will be very explicit about the analytic continuation. Write
	$x = |x|e^{i\theta}$ and let $\theta$ vary from $0$ to $2\pi$. We have in general \begin{align*}
		S(x) = \begin{pmatrix}
			       |x|e^{i\theta} & |x|e^{i\theta}\log |x| + i\theta |x|e^{i\theta} \\
			       0              & |x|e^{i\theta}
		       \end{pmatrix}
	\end{align*}
	Going from $0$ to $2\pi$, the solution matrix goes from \begin{align*}
		S(x) = \begin{pmatrix}
			       x & x\log x \\
			       0 & x
		       \end{pmatrix}
	\end{align*} to \begin{align*}
		S(x) = \begin{pmatrix}
			       x & x\log x + 2\pi ix \\
			       0 & x
		       \end{pmatrix}
	\end{align*} which tells us the monodromy matrix is \begin{align*}
		G = \begin{pmatrix}
			    1 & 2\pi i \\
			    0 & 1
		    \end{pmatrix}
	\end{align*}
\end{example}

\subsubsection{Definition of regularity}
Let $\theta = x\nabla = x\partial$.

\begin{definition}
	We say that a meromorphic connection $M,\nabla$ is \red{regular singular} at $x=0$ if
	there exists a finitely generated $\cO$-submodule $L\subset M$ which is stable under $\theta$
	and generates $M$ over $K$. Such an $L$ is called an $\cO$-lattice.
\end{definition}

\begin{lemma}
	Any $\cO$-lattice $L$ is a free $\cO$-module of rank $\dim_K M$.
\end{lemma} In particular, a meromorphic connection is regular if
and only if there exists a $K$-basis $e_1,\dots,e_n$ of $M$ so that
the associated ODE is of the form \begin{align*}
	\frac{d}{dx}u = \frac{A(x)}{x}u
\end{align*} for $A(x)\in M_n(\cO)$. Equivalently
if \begin{align*}
    \frac{d}{dx}u = \frac{A}{x}u
\end{align*} for $A\in M_n(\C)$.

\begin{remark}
    Recall that the affine Grassmannian can be thought of 
    the space of lattices where $c\O = k[x_1,\dots,x_n]$ and 
    $K \C((x_1,\dots,x_n))$. In this case we are thinking about 
    regular singularities for the divisor which is the union of 
    all coordinate hyperplanes.

    \hfill

    Note that the set of regular weights in $\mf t^*$
    is precisely corresponding to ripping out the union of coordinate
    hyperplanes in $\mf t^*$.
\end{remark}

\begin{proposition}
	A meromorphic connection is regular iff all of its horizontal
	sections are in $\tilde K_{\text{mod}} \otimes_K M$.
\end{proposition}

\begin{proposition}
	For a meromorphic connection $M,\nabla$ at $x=0$, the following are equivalent:
	\begin{enumerate}
		\item $M,\nabla$ is regular.
		\item For any $u\in M$ there exists $L\ni u$ finitely generated over $\cO$
		      so that $\theta L\subset L$.
		\item For any $u\in M$ there exists a polynomial \begin{align*}
			      F(t) = t^m + a_{m-1}t^{m-1} + \cdots + a_0
		      \end{align*} so that $F(\theta)u = 0$.
	\end{enumerate}
\end{proposition}

\begin{remark}
	Regularity is a notion which comes from the theory of linear ODEs. In particular
	if we consider a second-order linear differential equation of the form \begin{align*}
		\frac{d^2}{dx^2}u + p(x)\frac{d}{dx}u + q(x)u = 0
	\end{align*} then the point $x=0$ is called a regular singular point if $xp(x)$ and $x^2q(x)$
	are holomorphic at $0$. The solutions of the ODE are then well-behaved
	and can be written as power series or generalized series involving logarithms.

	\hfill

	A regular $\cD$-module is then a generalization of this, which includes the $\cD$-modules
	which arise from linear ODEs with regular singularities.
\end{remark}

\begin{proposition}
	An exact sequence of meromorphic connections \begin{align*}
		0 \to M' \to M \to M'' \to 0
	\end{align*} implies that $M$ is regular if and only if $M'$ and $M''$ are regular.
	If $M,N$ are regular, then so is $M\otimes_K N$ and $\Hom(M,N)$.
\end{proposition}
\subsection{For a general algebraic curve}
The ring $\cO = \C\{x\}$ is replaced by the stalk $\cO_{C,p}$
where $C$ is a smooth algebraic curve and $p\in C$. The field $K$ is replaced by the
quotient field of $\cO_{C,p}$. $\cO_p$ is a DVR and in particular a PID.

\begin{definition}
	\hfill
	\begin{enumerate}
		\item Let $M$ be a finite dimensional $K_{C,p}$-vector space with a $\C$-linear map
		      \begin{align*}
			      \nabla: M\to \Omega^1_{C,p}\otimes_{\cO_{C,p}} M \cong (K_{C,p}\otimes_{\cO_{C,p}} \Omega^1_{C,p})\otimes_{K_{C,p}} M
		      \end{align*}
		      The pair $M,\nabla$ is a \red{algebraic meromorphic connection} at $p\in C$ if \begin{align*}
			      \nabla(fu) = df\otimes u + f\nabla u \quad (\text{ for all } f\in K_{C,p}, u\in M)
		      \end{align*}
		\item A morphism is a $K_{C,p}$-linear map $\phi:M\to N$ so that $\nabla\circ\phi = (\id\otimes\phi)\circ\nabla$.
	\end{enumerate}
\end{definition}
Algebraic meromorphic connections at $p\in C$ naturally form an abelian category.
Choose a local coordinate $x\in \cO_{C,p}$ so that $K_{C,p} = \cO_{C,p}[x^{-1}]$.
Then we can identify $\Omega^1_{C,p}$ with $\cO_{C,p}$ via $f \iff f dx$
and an algebraic meromorphic connection at $p\in C$ is a finite dimensional $K_{C,p}$-vector space
endowed with a $\C$-linear map \begin{align*}
	\nabla: M\to M
\end{align*} which satisfies \begin{align*}
	\nabla(fu) = \frac{df}{dx}u + f\nabla u \quad (\text{ for all } f\in K_{C,p}, u\in M)
\end{align*}


\begin{definition}
	An algebraic meromorphic connection $M,\nabla$ at $p\in C$ is \red{regular} if
	there exists a finitely generated $\cO_{C,p}$-submodule $L\subset M$ which is stable
	under $\theta = x\nabla$ for any (and all) local coordinate $x \in \cO_{C,p}$ at $p$.
	$L$ is called an $\cO_{C,p}$-lattice.
\end{definition}
This definition is about being regular at a point $p\in C$. We now globalize this definition.
Let $M$ be an integrable connection on an algerbaic curve $C$. Take
a smooth completion $j:C\to \overline C$, unique up to isomorphism because
$C$ is a curve. $j$ is an affine map so pushforward is exact and we have $j_*M$.
Since $M$ is locally free over $\cO_C$ it is free on a nontrivial Zariski open $U = C\setminus V$
where $V$ is finitely many points.
Hence $j_(M)\vert_{\overline C \setminus V}$ is a locally free $j_*\cO_C\vert_{\overline C\setminus V}$-module.
In particular $j_*M$ is locally free over $j_*\cO_C$. Let $p\in \overline C \setminus C$ and then
the stalk $(j_*M)_p$ is a $K_{\overline C,p}$-vector space. The stalk is a $\cD_{\overline C,p}$-module
and is naturally endowed with a structure of an algebraic meromorphic connection at
$p\in \overline C$ via  $\nabla(m) = dx \otimes \partial m$ where $x$ is a local coordinate at $p$.
We call this $\cD_{\overline C}$-module \red{the algebraic meromorphic extension of $M$.}
\begin{definition}
	Let $M$ be an integrable connection on a smooth algebraic curve $C$.
	For a boundary point $p\in \overline{C} \setminus C$ we say that $M$ has
	regular singularity at $p$ if the algerbaic meromorphic connection $(j_*M)_p$ is regular.

	\hfill

	An integrable connection $M$ on $C$ is \red{regular} if it has regular singularities at all boundary points.
\end{definition}

Recall that a holonomic $\cD$-module is generically an integrable connection. In dimension $1$
the converse is also true. Therefore we define regularity for holonomic $\cD$-modules
specifically for $C$ a smooth algebraic curve.

\begin{definition}
	A holonomic $\cD$-module $M$ on a smooth algebraic curve $C$ is \red{regular} if
	there is an open dense $C_0$ so that $M\vert_{C_0}$ is a regular integrable connection.
	An object $M^\bullet \in D^b_h(\cD_C)$ is regular if all of its cohomology sheaves are regular.
\end{definition}

\begin{lemma}
	Let $f:C\to C'$ dominant (i.e. the image is dense) and $M$ a holonomic $\cD_C$-module.
	Then $f^\dagger M$ is regular if and only if $M$ is regular.
	Then $\int_f N$ is regular if $N$ is regular.
\end{lemma}


\subsection{Deligne's (classical) Riemann-Hilbert correspondence}
\subsubsection{Regularity in general}
Let $X$ be a complex manifold and $D$ divisor. Let $\cO_X(D)$ be the sheaf of meromorphic functions
on $X$ which are holomorpihc on $Y = X\setminus D$ and have poles along $D$.
Let $h\in \cO_X$ be a local defining equation for $D$ so that $\cO_X(D) = \cO_X(h^{-1})$
is a coherent sheaf of rings.

\begin{definition}
	Let $M$ be a coherent $\cO_X(D)$-module with a $\C$-linear map \begin{align*}
		\nabla: M\to \Omega^1_X\otimes_{\cO_X} M
	\end{align*} so that \begin{align*}
		\nabla(fu) = df\otimes u + f\nabla u \quad (\text{ for all } f\in \cO_X(D), u\in M)
		[\nabla_\theta, \nabla_\theta'] = [\nabla_\theta, \nabla_{\theta'}] \quad (\text{ for all } \theta,\theta'\in \Theta_X)
	\end{align*}
	The pair $M,\nabla$ is a \red{meromorphic connection} on $X$ with poles along $D$.
\end{definition}
Let $\Conn(X,D)$ be the category of meromorphic connections on $X$ with poles along $D$.
Set $B = $ open unit disk in $\C$. For a morphism $i:B\to X$ so that $i^{-1}D = \set{0}$,
the stalk $(i^*M)_0$ is a meromorphic connection in one variable as studied in the $\A^1$ case.

\begin{definition}
	A meromorphic connection $M,\nabla$ on $X$ with poles along $D$ is \red{regular} if
	$(i^*M)_0$ is regular for all $i:B\to X$ so that $i^{-1}D = \set{0}$.
\end{definition}
Let $\Conn^{reg}(X,D)$ be the category of regular meromorphic connections along $D$.

\begin{definition}
	A meromorphic connection on $X$ along $D$ is \red{effective}
	if it is generated as an $\cO_X(D)$-module by a coherent $\cO_X$-submodule.
\end{definition}

Any regular mromorphic connection is effective.

\subsubsection{Logarithmic poles}
Let $D$ be a normal crossing divisor. Say that $D$ is locally defined
by $x_1\cdots x_r = 0$ in local coordinates $x_1,\dots,x_n$. Let $p\in D$
and denote $D_k$ the irreducible components of $D$ defined by $x_k = 0$.

\hfill

Let $M$ be a meromorphic connection on $X$ with poles along $D$.
We also assume that there exists a holomorpihc vector bundle $L$ on $X$ so that
$M = L\otimes_{\cO_X} \cO_X(D)$ as a $\cO_X(D)$-module. Therefore, in an approapriate
basis $e_1,\dots,e_n$ for $L$ we have \begin{align*}
	\nabla e_i = \sum_j A^k_{ij}e_j \otimes dx_k
\end{align*}
where $A^k_{ij} \in \cO_X(D) = \cO_X[x_1^{-1},\dots,x_n^{-1}]$. Then \red{we
	further assume that the functions $x_k A_{ij}$ are holomorphic}. In this case
we say that $M$ along $D$ has a logarithmic pole with respect to the lattice $L$ at $p$.
Inf this is the case at all points of $D$ then we say that $M$ has logarithmic poles along $D$
with respect to $L$.

\hfill

Let $M$ be a meromorphic connection on $X$ with poles along $D$, wiht logarithmic poles
with respect to $L$ along $D$. Take basis $e_i$ of $L$ and consider the $A^k$ matrices from before. Let $B^k = x_k A^k$.
By hypothesis $B^k$ are holomorphic and so we consider its restriction $B^k\vert_{D_k} \in M_n(\cO_{D_k})$.
Then $B^k\vert_{D_k}$ defines a canonical section $\Res^L_{D_k}\nabla$ of the vector bundle $\End_{\cO_{D_k}}(L\vert_{D_k})$.
This is called the \red{residue} of $\nabla$ along $D_k$ with respect to $L$.

\hfill

We care about logarithmic poles from a regularity standpoint.

\begin{proposition}
	Let $M$ be a meromorphic connection which has a logarithmic pole along $D$ wiht
	repsect to a lattice $L$. Then $M$ is regular.
\end{proposition}

\begin{remark}
	What is the difference between a integrable connection and a meromorphic connection?
	It is worth considering this quesiton because in the next section, we will be extending
	integrable connections along divisors at the cost of
	them becoming meromorphic connections.

	\hfill

	Recall that integrable connections are about parallel transport of sections of a vector bundle.
	The requirement that the connection is flat means that the transport of sections along paths
	is independent of the path. This is in fact a hallmark. Given an integrable connection, its
	horizontal sections are the solutions to a system of linear ODEs. The corresponding system of
	ODEs is nonsingular. For a meromorphic connection, the corresponding system of ODEs is
	allowed to have singularities. Regularity is about the singularities being "finite dimensional"
	in some sense.

\end{remark}

\begin{remark}
	\textbf{Meromorphic Connections: Away from the Divisor}
	\begin{itemize}
		\item Away from the divisor, the meromorphic connection behaves similarly to a \textit{regular connection}. That is, it smoothly defines how the sections of the bundle evolve as you move along different directions on the manifold.

		\item For a vector bundle \( E \) over a complex manifold \( X \) with coordinates \( z_1, z_2, \dots, z_n \), the connection locally looks like:
		      \[
			      \nabla = d + A(z),
		      \]
		      where \( A(z) \) is a matrix of holomorphic functions, and \( d \) is the exterior derivative that captures how sections change infinitesimally.

		\item In this region, the connection is \textit{holomorphic} and behaves like a flat or integrable connection if the curvature vanishes.
	\end{itemize}

	\textbf{Meromorphic Connections: Near the Divisor}
	\begin{itemize}
		\item \textit{Geometrically}, near a divisor (which is typically defined by \( D = \{ z_1 = 0 \} \) locally), the connection takes on a different character. The presence of a \textit{pole} along the divisor means that sections of the bundle can exhibit singular behavior as you approach \( D \). However, this singularity is controlled or regular in the case of a regular meromorphic connection.

		\item The local form of a meromorphic connection near a divisor might look like:
		      \[
			      \nabla = d + \frac{A(z)}{z_1} dz_1 + B(z) dz_2 + \cdots,
		      \]
		      where \( A(z) \) is a matrix of holomorphic functions. The term \( \frac{A(z)}{z_1} dz_1 \) indicates the presence of a \textit{pole} along \( z_1 = 0 \) (the divisor \( D \)).
	\end{itemize}
\end{remark}

\subsubsection{What does ODEs and $\cD$-module language really mean?}
Recall that earlier I told you the following theorem:

\begin{theorem}
	Holonomic $\cD$-modules are generically integrable connections.
\end{theorem}
What does this really mean? Recall that given a linear ODE \begin{align*}
	P(x,\partial) u = 0
\end{align*} we can form the corresponding $\cD$-module $M = \cD/\cD P$.

\hfill

We say that a linear ODE is holonomic if
its solution space is finite dimensional.
The point is that this means that the ODE is highly constrained and
therefore so should the corresponding characteristic variety.

\hfill

In the context of linear ODEs, a regular singular point is defined as a point where the ODE has specific controlled behavior. For a second-order linear ODE of the form:

\[
	y'' + p(x)y' + q(x)y = 0,
\]

a point \( x_0 \) is a regular singular point if the coefficients \( p(x) \) and \( q(x) \) have certain pole behaviors:

\begin{itemize}
	\item The function \( p(x) \) can have a pole of order at most 1 at \( x_0 \).
	\item The function \( q(x) \) can have a pole of order at most 2 at \( x_0 \).
\end{itemize}

Recall that we defined regularity as follows:
A meromorphic connection \( \nabla \) on a sheaf \( \mathcal{M} \) is
said to be regular at \( x = 0 \) if there exists a
finitely generated \( \mathcal{O} \)-submodule \( L \) of \( \mathcal{M} \) such that
$L$ is stable under the action of the Euler operator $\theta$ and 
generates \( \mathcal{M} \) over the field \( K \) (the field of fractions of \( \mathcal{O} \)).
\red{Therefore, being regular is about having solutions exhibit controlled growth near singular points.}

\hfill

Reminder! We also proved an equivalent criterion for regularity
in the context of meromorphic connections over $\C$:

\begin{theorem}
	A meromorphic connection $M,\nabla$ is regular
	if and only if there exists a finite $K$-basis $e_i$ for $M$ the associated
	ODE is of the form \begin{align*}
		\frac{d}{dx}u = \frac{A}{x}u
	\end{align*} for $A(x)\in M_n(\C)$.
\end{theorem}

\begin{remark}
	\red{The following is speculation.} From the above theorem and our characterization
	of regular holonomic ODEs, I suspect that all of the regular
	holonomic $\cD$-modules on $\A^1_\C$ are of the form \begin{align*}
		\cD/\cD (x\partial - \lambda)
	\end{align*}
\end{remark}

\begin{remark}
	An ODE system corresponds to an integrable connection if it is flat and has
	no monodromy. \red{This is different than a meromorphic connection!}
	A meromorphic connection can have singularities without specific restrictions.

\end{remark}

\subsubsection{Extending integrable connections along normal crossing divisors}
\begin{theorem}
    Let $D$ be a normal crossing divisor on a complex manifold $X$.
    Let $Y = X\setminus D$ and fix a section $\tau:\C/\Z\to \C$ of the 
    projection $\C\to \C/\Z$. Let $M$ be a integrable connection on $Y$.
    There exists an extension $L_\tau$ of $M$ as a vector bundle on $X$ 
    satisfying the following conditions: \begin{itemize}
        \item The $\C$-linear morphism $\nabla_M:M \to \Omega^1_Y\otimes M$
        can be uniquely extended to a $\C$-linear morphism 
        $\nabla:\cO_X(D) \otimes_{\cO_X} L_\tau \to \Omega^1_X\otimes_{\cO_X} L_\tau$
        so that $(\cO_X(D)\otimes_{\cO_X} L_\tau,\nabla)$ is a meromorphic connection
        with a logarithmic pole along $D$ with respect to $L_\tau$.
        \item For any irreducible component $D'$ of $D$ the eigenvalues of the residue
        $\Res_{D'}^{L_\tau}\nabla$ of $\cO_X(D)\otimes_{\cO_X} L_\tau$ along $D'$
        are contained in $\tau(\C/\Z) \subset \C$.
    \end{itemize}
Moreover $L_\tau$ is unique up to isomorphism.
\end{theorem}

Now we want to extend to a general divisor $D$. We do this 
and I skip a lot of things and then we get the main result:

\begin{theorem}
    [Deligne] Let $X$ be a complex manifold and $D$ a divisor. Let $Y = X\setminus D$. THen 
    the restriction functor \begin{align*}
        \Conn^{reg}(X,D) \to \Conn^(Y) \cong \Loc(Y)
    \end{align*} is an equivalence of categories.
\end{theorem}

For a general algebraic variety we have the following definition:
\begin{definition}
    An integrable connection $M$ on $X$ is called \red{regular} if for every morphism 
    $i_C:C\to X$ from a smooth algebraic curve $C$ the pullback $i_C^*M$ is regular.
\end{definition}

\begin{theorem}
    [Deligne] Let $X$ be a smooth algebraic variety. Then the functor \begin{align*}
        M \mapsto M^{an}
    \end{align*} is an equivalence of categories \begin{align*}
        \Conn^{reg}(X) \to \Conn(X^{an})
    \end{align*}
\end{theorem}
It turns out that when $X$ is a smooth projective variety \begin{align*}
    \Conn(X) \cong \Conn(X^{an})
\end{align*}
These statements are all about Serre GAGA.

\section{The Riemann-Hilbert correspondence}
\subsection{Regularity in full generality}
\begin{definition}
    Let $X$ be a smooth algebraic variety. A holonomic $\cD$-module $M$ on $X$ is called
    \red{regular} if any composition factor of $M$ is isomorphic to the 
    minimal extension $L(Y,N)$ of some regular integrable connnection $N$ 
    on a locally closed smooth subvariety $Y\subset X$ so that the inclusion 
    $Y\hookrightarrow X$ is affine.

    \hfill

    We denote $\Mod_{rh}(\cD_X)$ the full subcategory of $\Mod_h(\cD_X)$
    consisting of regular holonomic $\cD$-modules. We denote $D^b_{rh}(\cD_X)$
    the full subcategory of $D^b_h(\cD_X)$ consisting of complexes with regular
    holonomic cohomology sheaves.
\end{definition}

This agrees with our definition for curves. Any simple holonomic $\cD_C$-module
is of the form $L(Y,N)$ where $Y$ is a single point or a connected nonempty open subset
of $C$ and $N$ is a regular integrable connection on $Y$.

\hfill

We come to some big results.
\begin{theorem}
    Let $X$ smooth algebraic variety.
    \begin{enumerate}
        \item The duality functor $\D_X$ preserves $D^b_{rh}(\cD_X)$.
        \item Let $f:X\to Y$ be a morphism of smooth algebraic varieties.
        Then $\int_f,\int_{f^!},f^\dagger, f^\star$ preserve $D^b_{rh}(\cD_X)$.
    \end{enumerate}
\end{theorem}

\begin{theorem}
    [Curve testing criterion] Let $X$ be a smooth algebraic variety. The 
    following about $M^\bullet \in D^b_h(\cD_X)$ are equivalent:
    \begin{enumerate}
        \item $M^\bullet$ is regular.
        \item For any locally closed embedding $i_C:C\to X$ from a smooth algebraic curve $C$
        the pullback $i_C^\dagger M^\bullet \in D^b_h(\cD_C)$ is regular.
        \item For any morphism $k:C\to X$ from a smooth algebraic curve $C$ 
        $k^\dagger M^\bullet \in D^b_h(\cD_C)$ is regular.
    \end{enumerate}
\end{theorem}

\subsection{The Riemann-Hilbert correspondence}
Recall that for a smooth algebraic variety $X$ we have the duality functors \begin{align*}
    \D_X: D^b_h(\cD_X) \to D^b_h(\cD_X)^{op} \\
    \D_X(M^\bullet) = R\Hom_{\cD_X}(M^\bullet,\cD_X) \\
    \textbf{D}_X: D^b_c(X) \to D^b_c(X)^{op}  \\
    \textbf{D}_X(M^\bullet) = R\Hom_{\C_X}(M^\bullet,\omega_X^\bullet)
\end{align*} where $D^b_c(X)$ is the bounded derived category
 of constructible sheaves on $X$, \textbf{D} is Verdier duality.

All of the functors $\D_X,\int_f,\int_{f^!},f^\dagger,f^\star$ preserve regular holonomic $\cD$-modules.
We also know that \begin{align*}
    \textbf{D}_X \DR_X(M^\bullet) = \DR_X(\D_X(M^\bullet)) \quad \text{ for } M^\bullet \in D^b_{h}(\cD_X)
\end{align*}
i.e. DR commutes with duality. It turns out that it also commutes with the inverse and 
direct image functors. 

\begin{theorem}
    Let $f:X\to Y$ be a morphism of smooth algebraic varieties. Then
    Then \begin{align*}
        \DR_Y \circ \int_f \cong Rf_* \circ \DR_X:D^b_{rh}(\cD_X) \to D^b_{c}(Y) \\
        \DR_Y \circ \int_{f^!} \cong Rf_! \circ \DR_X:D^b_{rh}(\cD_X) \to D^b_{c}(Y) \\
        \DR_X \circ f^\dagger \cong f^! \circ \DR_Y:D^b_{rh}(\cD_Y) \to D^b_{c}(X) \\
        \DR_X \circ f^\star \cong f^{-1} \circ \DR_Y:D^b_{rh}(\cD_Y) \to D^b_{c}(X)
    \end{align*}
\end{theorem}

\begin{theorem}
    [Riemann-Hilbert correspondence] Let $X$ be a smooth algebraic variety.
    The DR functor is an equivalence of categories \begin{align*}
        \DR_X:D^b_{rh}(\cD_X) \to D^b_c(X)
    \end{align*}
\end{theorem}
The solution functor \begin{align*}
    \Sol_X:D^b_{rh}(X) \to D^b_c(X)^{op}
\end{align*} is an equivalence of categories as well.
Passing to the full subcategory $\Mod_{rh}(\cD_X)$ we get 
perverse sheaves which receive the image of the DR functor.

\begin{theorem}
    The DR functor induces an equivalence of categories \begin{align*}
        \DR_X:\Mod_{rh}(\cD_X) \to Perv(\C_X)
    \end{align*}
\end{theorem}

\begin{remark}
    Let $Y \subset X$ algebraic subvariety and consider a local 
    system $L$ on $U^{an}$ for open dense subset $U$ of the 
    regular part of $Y$. Then 
    we can associate to it an intersection complex $IC_X(L)^{\bullet}$
    on $X$ which is a perverse sheaf which is irreducible and whose
    support is contained in $Y$. Consider the regular integrable 
    connection $M$ on $U$ which corresponds to $L$.
    Then \begin{align*}
        \DR_X L(U,M) \cong \IC_L
    \end{align*}
\end{remark}



