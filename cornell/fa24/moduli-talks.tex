\documentclass[12pt]{article}
\usepackage[english]{babel}
\usepackage[utf8x]{inputenc}
\usepackage[T1]{fontenc}
\usepackage{listings}
\usepackage{bookmark}
\usepackage{tikz}
\usepackage{/Users/songye03/Desktop/math_tex/style/quiver}
\usepackage{/Users/songye03/Desktop/math_tex/style/scribe}
\usepackage{fancyhdr}


\begin{document}


\lhead{Songyu Ye}
\rhead{\today}
\cfoot{\thepage}

\title{Moduli in November}

\author{Songyu Ye}
\date{\today}
\maketitle


\begin{abstract}
Recently there were two moduli talks at Cornell. 
The first talk was given by Andrés Ibáñez Núñez and the second talk was given by Rachel Webb.
I am speaking with Rachel on Friday. She talked a lot of about root stacks.
\end{abstract}

\tableofcontents

\section{Motivic enumerative invariants of algebraic stacks}
\subsection{Introduction}
\red{A stack can be thought of points with groups.} Just as a scheme has a point
for each subvariety of a given variety, we can do the same 'equivariantly'.
A stack has a point for each subvariety of a given variety, but the points
also know the stabilizer groups of the subvarieties.

\hfill

A lot of my intuition for stacks comes from experts speaking 
about them with authority. That is how much of this note will be structured.
We are also going to need to supplement with references. 

\begin{definition}[Classifying groupoid]
    If \( G \) is a group, the \textit{classifying groupoid} \( BG \) of \( G \) is defined as the category with one object \( \star \) such that \( \operatorname{Aut}(\star) = \operatorname{Mor}(\star, \star) = G \).
    \end{definition}
\begin{example}
    $B-=0{G_m}$ is the classifying stack of the multiplicative group and it satisfies \begin{align*}
        H_i^*(B_{G_m}) = \Q[t] \quad \text{where} \quad \deg t = 2 \\
        \chi(B_{G_m}) = \infty
    \end{align*} It is a 1-dimensional stack. Compare this to $\C\P^\infty$ which is the
    topological classifying space of the multiplicative group. In particular it classifies
    homotopy classes of principal $\C^*$-bundles on $X$. The groupoid of 
    line bundles on $X$ is equivalent to the groupoid of $\C^*$-bundles on $X$.
    via the correspondence $P \mapsto P \times_{\C^*} \C$ and 
    opposite correspondence $L \mapsto L \backslash \{0\} \to X$.
\end{example}
\subsection{Euler characteristics}
His next point was the Euler characteristics are motivic invariants.
Consider the ZXU setup with \( X \) a smooth variety and 
\( Z \) a closed subvariety, and \( U \) the open complement.

Then Euler characteristics are additive \begin{align*}
    \chi(X) = \chi(U) + \chi(Z)
\end{align*}
\red{and if you take additivity along these setups as a universal property, then you get for each algebraic 
stack, a ring of motives, an example of which is the Grothendieck ring of varieties.}

In particular, let $\mathcal{X}$ be an algebraic stack and consider the ring \begin{align*}
    M(\mathcal{X}) = \bigoplus \Q [Y\to \mathcal{X}]/\sim
\end{align*} where $Y\to \mathcal{X}$ is a representable map of stacks and 
we mod out by the relation generated by the relation \begin{align*}
    [Y\to \mathcal{X}] = [Y'\to \mathcal{X}] + [Y''\to \mathcal{X}]
\end{align*} for $Y'YY''$ in the $ZXU$ setup. Then there is a map \begin{align*}
    \int_\chi : M(\mathcal{X}) \to M(\text{pt})[\bL^{-1},(\bL^k-1)^{-1}]
\end{align*} where $\bL$ is the Tate motive $[\A^1\to *]$ and $k\in\Z$.
\begin{example}
    \[[\GL_n] = \prod_{i=1}^n \bL^n - \bL^i\] 
    and the right hand side is a motive over the point that does not depend on $X/\GL_n$ 
    as a quotient stack. He also had this formula \begin{align*}
        \int_{B\Gm} B\Gm = \frac{1}{\bL - 1}
    \end{align*} which is not defined at $\bL = 1$, which incarnates the fact that 
    the Euler characteristic of $\A^1$ is $1$.
\end{example}
\subsection{Motivic Hall algebra}
If $A$ is an abelian category, let $M(A)$ be the moduli stack of objects in $A$.
Then there are maps $\Ext_A = \set{0\to A \to B \to C \to 0} \to M(A) \times M(A)$
and $\Ext_A \to M(A)$ called $p$ and $q$ respectively. Then we can define the
motivic Hall product $* = q_*p^*$ on $M(M(A))$, the ring associated 
to the algebraic stack $M(A)$.

\subsection{The stack $\A^1/\Gm$}
\red{Let $\theta = \A^1/\Gm$ be the stacky line.} Then \begin{align*}
    \Hom(\theta,\Vect) = \text{Vector bundles on } \theta = \set{\text{Vector spaces with filtration}}
\end{align*} In particular, you can pick a trivialization and look at the $t\to\infty$ 
action of $\Gm$ and you get some power of $k$ corresponding to the action of $\Gm$,
which induces a filtration (independent of the trivialization). This was explained to me 
by Allen and is written down in [Knutson-Sharpe 97].

\hfill

Thus there are stacks \begin{align*}
    \text{Filt}(X) := \Hom(\theta,X) \quad \text{and} \quad \Gr(X) := \Hom(B\Gm,X)
\end{align*}

\subsection{More examples}
\begin{example}
    $\C^2/\GL_2$ was an example that he did.
\end{example}
\subsection{References}
\begin{thebibliography}{9}
    \bibitem{alper} Jarod Alper, \textit{Notes on Stacks and Moduli}, \url{https://sites.math.washington.edu/~jarod/moduli.pdf}
\end{thebibliography}

\section{Abelian Orbicurves with Smooth Coarse Space}

\end{document}