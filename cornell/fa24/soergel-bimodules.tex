\documentclass[12pt]{article}
\usepackage[english]{babel}
\usepackage[utf8x]{inputenc}
\usepackage[T1]{fontenc}
\usepackage{listings}
\usepackage{bookmark}
\usepackage{tikz}
\usepackage{/Users/songye03/Desktop/math_tex/style/quiver}
\usepackage{/Users/songye03/Desktop/math_tex/style/scribe}
\usepackage{fancyhdr}

\begin{document}


\lhead{Songyu Ye}
\rhead{\today}
\cfoot{\thepage}

\title{Soergel bimodules}

\author{Songyu Ye}
\date{\today}
\maketitle


\begin{abstract}
Notes on Soergel bimodules. See "Soergel Calculus" by Elias and Williamson and notes about 
perverse sheaves on the flag variety by Riche.
\end{abstract}

\tableofcontents
\section{Hecke algebra}
Recall Bruhat decomposition. Let $G$ be a reductive group over $\C$ 
with Borel subgroup $B$ and maximal torus $T$. 
Let $W$ be the Weyl group of $G$. Then $G$ has a Bruhat decomposition
\begin{align*}
G = \coprod_{w\in W} BwB
\end{align*} Orbit closures $\overline{BwB}$ are called Schubert varieties.

\hfill

The Hecke algebra is the algebra of $B\times B$-invariant functions on $G$ under
normalized convolution. It therefore has a basis $\{T_w\}_{w\in W}$ where $T_w$ is the 
characteristic function of $BwB$. In 1979, Kazhdan and Lusztig defined another basis $\{C_w\}_{w\in W}$
which enjoys many remarkable positivity properties.

\hfill 

According to Grothendieck's function-sheaf dictionary, the Hecke 
algebra should be categorified by some category of $B$-biinvariant sheaves on $G$.
The \textbf{Hecke category} is the additive subcategory of semisimple complexes \begin{align*}
    \cH \subset D^b_{B\times B}(G)
\end{align*} The objects of $\cH$ are direct sums of 
the various intersection complexes $IC(\overline{BwB})$. These are the simple equivariant
intersection cohomology complexes. There is a monoidal structure on $D^b_{B\times B}(G)$
 given by convolution which preserves $\cH$. The Grothendieck group $[\cH]$ of $\cH$ is an algebra 
 over $\Z[v^{\pm 1}]$ where $v$ acts by shift. 
 
 \begin{theorem}
    $[\cH]$ is isomorphic to the Hecke algebra. The simple objects
    of $\cH$ are the intersection complexes supported on Schubert varieties. The map takes 
    $IC(\overline{BwB})$ to the Kazhdan Lusztig basis element $C_w$. Explicitly, we have
    the Kazhdan-Lusztig polynomial $P_{x,y}(q)$ defined by
    \begin{align*}
        P_{v,w}(q) = \sum_i (-1)^i \dim \cH^i(IC_w)_v q^i
    \end{align*} and the Kazhdan-Lusztig basis element $C_w$ can be written in the standard
     basis as \begin{align*}
            C_w = q^{-\ell(w)/2}\sum_{v\leq w} P_{v,w}(q) T_v
     \end{align*}
 \end{theorem}

\section{Soergel bimodules}
Recall that a Coexter group is a group which has a presentation of the form
\begin{align*}
    \langle s_1,\ldots,s_n | s_i^2 = 1, (s_is_j)^{m_{ij}} = 1\rangle
\end{align*} where $m_{ij}\geq 2$ are symmetric. Weyl groups are Coxeter groups, but 
so are affine Weyl groups and other cool groups.

\hfill

The hypercohomology of any complex in $D^b_{B\times B}(G)$ naturally 
carries an action of $H^*_{B\times B} = R \otimes_C R$ where $R = \Rep(T)$. Since 
$R$ is commutative, this extends to make the hypercohomology a bimodule over $R$.
Therefor we get a functor from $\mathbb{H}^\cdot: D^b_{B\times B}(G) \to R\text{-bimod}$.

\begin{theorem}
    [Soergel] This functor is fully faithful and monoidal.
\end{theorem} Therefore the Hecke category is equivalent to its essential image.
We can also calculate that \begin{align*}
    \mathbb{H}^\cdot(IC(\overline{BsB})) = \cM_s := R \otimes_{R^s} R
\end{align*} where $R^s$ is the $s$-invariant subring of $R$ ($W$ acts by permuting the variables)
and $s$ is a simple reflection in $W$ (soon to be generalized to an arbitrary Coxeter group).

\hfill

Soergel proves that the essential image is the smallest full additive monoidal Karoubian subcategory of $R\text{-bimod}$ containing
the $\cM_s$. This is the category of \textbf{Soergel bimodules}. This story works 
for any Coexter system. Specifically for Weyl groups, Soergel's "categorification theorem" follows from 
the decomposition theorem applied to Bott Samelson resolutions of Schubert varieties.
It is suprising that Soergel makes it work for any Coxeter group because there is 
no underlying geometric context.

\section{Thoughts}
I don't understand precisely Soergel bimodules are related to category $\cO$ 
and perverse sheaves on the flag variety.
\begin{enumerate}
    \item Is it true that the Hecke category can also be described as the (maybe $B$-equivariant)
    perverse sheaves on the flag variety with respect to the Schubert stratification?
    \item If so, can I find a quiver and relations for which the Hecke category of a general 
    Coxeter group is the category of representations of the path algebra of that quiver modulo those relations?
    \item One reason why people are interested in finding these 
    geometric categorifications of these algebras is because the category of perverse sheaves enjoys
    very nice properities. In particular the simple objects carry Hodge structures, satisfy Hard Lefschetz, 
    and have Verdier duals. People are interested in making sense of these properties purely in the context
    of representation theory.
\end{enumerate}

\section{Soergel modules from bimodules}
Going from Soergel bimodules to Soergel modules morally is the same as 
going from $H^*_B(G/B)$ to $H^*(G/B)$. 
\end{document}