\documentclass[12pt]{article}
\usepackage[english]{babel}
\usepackage[utf8x]{inputenc}
\usepackage[T1]{fontenc}
\usepackage{listings}
\usepackage{bookmark}
\usepackage{tikz}
\usepackage{/Users/songye03/Desktop/math_tex/style/quiver}
\usepackage{/Users/songye03/Desktop/math_tex/style/scribe}
\usepackage{fancyhdr}

\begin{document}


\lhead{Songyu Ye}
\rhead{\today}
\cfoot{\thepage}

\title{Nearby Vanishing Cycles and Gluing Perverse Sheaves}

\author{Songyu Ye}
\date{\today}
\maketitle


\begin{abstract}
These are a second set of notes on perverse sheaves, specifically taking a look at the nearby and vanishing cycles functors, gluing perverse sheaves, and 
perverse sheaves on $(\C,0)$ and square matrices with the rank stratification. 
We are following \cite{beilinson}, \cite{reich}, and \cite{braden}. There
is a connection with representation theory which I am interested in 
exploring about Soergel bimodules and quiver represntations.

\end{abstract}
\tableofcontents

\section{Goals}
We want to understand the following result of Beilinson:

\begin{proposition}
Let $X$ be a small disk around $0$ in $\A^1$. Then the category $\Perv(X)$ 
of perverse sheaves on $X$ with singularities at $0$ only is equivalent to the 
category of representations of the quiver \begin{center}
\begin{tikzcd}
V_1 \ar[r, shift left, "v"] & V_2 \ar[l, shift left, "u"]
\end{tikzcd}
\end{center} with the relations that $I - uv$ and $I - vu$ are invertible.
\end{proposition}
Let $j:Z \hookrightarrow X$ be a closed inclusion and $i:U \hookrightarrow X$ 
the open complement and $f:X\to \A^1$. Beilinson shows that the category $\Perv(X)$ is equivalent to
the category $\Perv_f(Z,U)$ of "gluing data of perverse sheaves on $Z$ and $U$".

\hfill

Let $X \subset \A^1$ be a complex disk around $0$. 
Then $\Perv(Z) = \Perv(\set{0}) = \Vect$ and $\Perv{U}$ is vector spaces 
with an automorphism (monodromy), and the \textbf{unipotent nearby cycles functor} $\psi_f$ 
for the map $f:X \subset A^1$ takes a perverse sheaf on $U$ to a perverse sheaf on $Z$.
Beilinson computes this functor to be the one which takes $(V,T)$ to $(W,T)$
where $W$ is the maximal subspace of $V$ on which $T$ acts nilpotently.

\hfill

Therefore, the category $\Perv_f(Z,U)$ is equivalent to the category of data
$V_0', V_1', \phi, u, v$ where $V_0', V_1'$ are vector spaces, 
$\phi$ an automorphism of $V_1$ and $u,v$ are maps \begin{align*}
    v:V_0' \xleftrightarrow{} \psi(V_1', I -\phi):u
\end{align*} so that $vu = I - \phi$. Then the equivalence is given by \begin{align*}
    V_0' &= \psi_f(V_0,uv) \\
    V_1' &= V_1 \\
    \phi &= I - vu\\
    u &= u\\
    v &= v
\end{align*}
The invertibility of $I - vu$, $I-uv$ is about the image
of the functor being those vector spaces for which the prescribed maps are
acting \textbf{maximally} unipotently.

\section{Nearby Cycles}
Let $f:X \to \A^1$ so that $Z = f^{-1}(0)$.
Given $i,j$ the open and closed inclusions of $Z,U$ into $X$, we have 
the \textbf{nearby cycles} functor $R\psi_f:\Perv(U) \to \Perv(Z)$ defined as follows. 
Let $u:\tilde \Gm \to \Gm$ be the universal cover and let $\tilde U = U \times_{\Gm} \tilde \Gm$.
Then $R\psi_f$ is the derived functor \begin{align*}
    R\psi_f = R(i^*j^*v_*v^*)
\end{align*} Note that $v$ is not an algebraic map, but
nonetheless Deligne proved that the nearby cycles functor preserves
constructibility.

\begin{lemma}
The nearby cycles functor $R\psi_f$ decomposes 
$R\psi_f = R\psi_f^{un} \oplus R\psi_f^{\geq 1}$ where for any choice of generator $t$
of $\pi_1\Gm$, we have $1-t$ acts nilpotently on $R\psi_f^{un}(A^*_U)$
for any complex $A^*_U$ and is an automorphism of $R\psi_f^{\geq 1}(A^*_U)$.
\end{lemma}

The unipotent piece is called the \textbf{unipotent nearby cycles} functor.
One can show that $R\psi_f^{un}[-1]$ acts on perverse sheaves 
and we denote this functor $\Psi_f^{un}$.
 
\section{Vanishing Cycles}

\section{Cellular perverse sheaves and the representation theory of category $\cO$}
\subsection{Category $\cO$ and quivers}
The story starts with the BGG Category $\cO$ and Soergel's results about the 
endomorphism ring of projective modules in $\cO$. In particular, we have the following 
phenomenom which we will try to relate to what comes next.
\subsection{Cellular perverse sheaves}
Suppose $K$ is a finite simiplicial complex, which we will 
identify with its geometric realization $|K|$. Given a perversity $p$, 
there are two types of integers, the $*$ for which $k\in *$ if $p(k) = p(k-1)$ and 
the $!$ for which $k\in !$ if $p(k) = p(k-1) -1$. MacPherson defines the \textbf{
perverse dimension
} of a $d$-simplex $\sigma$ to be \begin{align*}
    \delta(\sigma) = -p(d) \text{ if } \sigma \in * \text{ and } \delta(\sigma) = -p(d)-d \text{ if } \sigma \in !
\end{align*}

\begin{definition}
    A \textbf{cellular perverse sheaf} $S$ on the simplicial complex $K$ is a 
    rule which assigns to each simplex $\sigma$ a vector space $S_{\sigma}$ and
    "attaching homomorphisms" $s_{\sigma,\tau}:S_{\sigma} \to S_{\tau}$ whenever $\sigma \iff \tau$
    and $\delta(\sigma) = \delta(\tau)$, so that the resulting sequence \begin{align*}
        \xrightarrow[d]{} \bigoplus_{\delta(\sigma) = r} S_{\sigma} \xrightarrow[d]{} \bigoplus_{\delta(\sigma) = r-1} S_{\sigma} \xrightarrow[d]{} \cdots
    \end{align*} Equivalently, this is saying that whenever $\delta(\sigma) = r +1$ and $\delta(\tau) = r-1$ then 
    we have \begin{align*}
        \sum_{\delta(\theta) = r, \sigma \iff \theta \iff \tau} s_{\sigma,\theta} \circ s_{\theta,\tau} = 0
    \end{align*}
\end{definition}
There is a cohomology functor $T:\Perv(K) \to \Perv^{\Delta}(K)$ which is an equivalence 
between the categories of perverse sheaves on $K$ with repsect to the triangulation $K$
and cellular perverse sheaves on $K$.

\hfill

Given a simplicial complex, there is a quiver whose vertices are the simplices of $K$ and whose
arrows are the elementary relations $\sigma > \tau$ with $\delta(\sigma) = \delta(\tau) +1$.
Form the corresponding path algebra $F$ of the quiver and consider the two-sided ideal $J$ 
generated by the elements \begin{align*}
    \sum_{\delta(\theta) = r, \sigma \iff \theta \iff \tau} s_{\sigma,\theta} \circ s_{\theta,\tau}
\end{align*}

\begin{theorem}
    The category $\Perv(K)$ with respect to the triangulation of $K$ is equivalent to the
    category of modules over the ring $F/J$.
\end{theorem}

\subsection{Vybornov's theorem}
Classically, the problem of computing an explicit quiver for $\cO_{0}$ was posed by Gelfand
and solved by Vybornov. Vybornov constructs a sequence of "IC modules" which are computing 
perverse sheaves, but then he proves that $d^2 = 0$ and 
this complex imposes a bunch of relations on the quiver, of the above shape.
\section{DG categories}
Recall that the bounded derived construcible category is a 
triangulated category and carries truncation functors, the 
heart of the t-structure being the perverse sheaves which form an abelian category.

\hfill

Because of some abstract nonsense about triangulated categories,
there is an equivalence of categories between the heart of a t-structure, i.e.
the perverse sheaves, and the differential graded category of the 
endomorphisms of a particular "generator."

\hfill

Because of this, you can always consider the category of perverse sheaves
on a space wrt a stratification as the category of finite length "dg" modules
over a "dg" quiver, and this "dg" quiver is an ordinary quiver precisely when the 
nonzero cohomology vanishes. Yuri referred to this as 
these sort of Bondal-Kapranov type results. See \cite{bondal-kapranov}.

\hfill

Many of the calculations for quivers, i.e. hyperplane arrangements and
rank stratifications, can be thought of as doing hard work to understand this 
endomorphism algebra of a generator. \red{There is also another 
endomorphism algebra I've been thinking about which is the endomorphism algebra
of the antidominant projective in category $\cO$.}

\hfill

Yuri also talked about the Artin representation, his attempt to generalize it, 
exotic solutions to the coCartesian Yang-Baxter equation \red{maybe work on this and
which ones extend to the Gelfand MacPherson Vilonen action}, its 
connection to Ng's result, and why he thinks there should be a more general 
result about the homotopy theory of spaces. Finally we also 
thought about the perverse sheaves on the knot complement in $\R^3$ \red{open question,
what's happening in $S^3$, its different topologically 
because $\pi_2$ is nontrivial in the $S^3$ knot 
complement}. 

\section{References}
\bibliography{refs}{}
% need refs.bib file
\bibliographystyle{plain}

\end{document}