\documentclass[12pt]{article}
\usepackage[english]{babel}
\usepackage[utf8x]{inputenc}
\usepackage[T1]{fontenc}
\usepackage{listings}
\usepackage{tikz}
\makeatletter
\def\input@path{{../../style/}}
\makeatother

\usepackage{../../style/quiver}
\makeatletter
\def\input@path{{../../style/}}
\makeatother

\usepackage{../../style/scribe}

\begin{document}
Songyu Ye

\today

\hfill

This will be a comprehensive writeup of the topics from geometric representation theory which I have studied this year.
We will attempt to cover the following topics:
\begin{itemize}
    \item Borel-Weil-Bott theorem
    \item The solution of the Kazhdan-Lusztig conjecture
    \item Affine Grassmannians
    \item The geometric Satake equivalence
\end{itemize}
In order to understand many of these results, we need to master many of the basic tools. This includes
\begin{itemize}
    \item Cohomology of algebraic varieties
    \item Derived categories
    \item The theory of perverse sheaves
    \item The theory of D-modules
    \item Intersection Cohomology
\end{itemize}

\end{document}