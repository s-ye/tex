\documentclass[12pt]{article}
\usepackage[english]{babel}
\usepackage[utf8x]{inputenc}
\usepackage[T1]{fontenc}
\usepackage{listings}
\usepackage{tikz}
\usepackage{/Users/songye03/Desktop/math_tex/style/quiver}
\usepackage{/Users/songye03/Desktop/math_tex/style/scribe}

\begin{document}
Songyu Ye

\today

\hfill

Heegard Floer homology:

\begin{definition}
    A \textbf{Heegard splitting} of a closed oriented 3 
    manifold $Y$ is a decomposition of $Y$ into two
    handlebodies $H_1$ and $H_2$ such that their
    intersection is a surface $\Sigma$. The \textbf{genus}
    of the Heegard splitting is the genus of $\Sigma$.
\end{definition}

\begin{example}
    The 3-sphere $S^3$ has a Heegard splitting of genus 0,
    where $H_1$ and $H_2$ are both 3-balls. $S^3$ also 
    has a Heegard splitting of genus 1, where $H^1$ is a tubular
    neighborhood of the $z$-axis and the point at infinity (thinking of $S^3$
    as the compactification of $\R^3$) and $H^2$ is the complement of $H^1$.
\end{example}


\begin{theorem}
    Every closed orientable 3-manifold has a Heegard splitting.
\end{theorem}

\begin{proof}
    Triangulations.
\end{proof}


\end{document}