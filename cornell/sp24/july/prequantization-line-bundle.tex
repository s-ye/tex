\documentclass[12pt]{article}
\usepackage[english]{babel}
\usepackage[utf8x]{inputenc}
\usepackage[T1]{fontenc}
\usepackage{listings}
\usepackage{tikz}
\usepackage{/Users/songye03/Desktop/math_tex/style/quiver}
\usepackage{/Users/songye03/Desktop/math_tex/style/scribe}

\begin{document}
Songyu Ye

\today

\hfill

We want to understand the following statement.

\begin{theorem}
    Let $(M,\omega)$ be a symplectic manifold. Suppose that $[\omega]$ is integral. Then 
there exists a "prequantization" line bundle $\cL \to M$ with $c_1(\cL) = [\omega]$ and a Hermitian connection $\alpha$
whose corresponding curvature form is $\omega$. Moreover $\cL$ is unique up to isomorphism 
(but still requires a particular choice). 
\end{theorem}

\section{Vector bundles and connections}
\subsection{Parallel transport}
One way to think about a connection is to consider parallel transport. You want to 
be able to differentiate sections of a vector bundle along paths. When we are dealing with functions,
we can form the directional derivative  \begin{align*}
    ds(x)X = \lim_{t\to 0}\frac{s(\gamma(t)) - s(\gamma(0))}{t}
\end{align*} for any smooth path $\gamma$ representing the tangent vector $X\in T_xM$ 
and this expression gives us a linear map $ds(x):T_xM\to E_x$.

\hfill 

However if $E$ is a nontrivial bundle, then this expression does not make sense because the
 summands live in different fibers. There is unfortunately no natural way to 
 compare vectors in different fibers. Therefore we need to introduce additional structure to 
 be able to compare these fibers.


\hfill

For a general vector bundle $E\to M$, we want to associate, to a path $\gamma$ in $M$, a smooth family of 
parallel transport isomorphisms $P^t_\gamma: E_{\gamma(0)}\to E_{\gamma(t)}$ such that \begin{itemize}
    \item $P^0_\gamma = \id$
    \item $P^t_{\gamma_1\cdot\gamma_2} = P^t_{\gamma_2}\circ P^t_{\gamma_1}$
\end{itemize} for any paths $\gamma_1,\gamma_2$ and $t\in\R$.

\hfill

Such a choice would allow us to define the directional ("covariant") derivative of a section $s$ along a path $\gamma$ 
as before. We shuold require that \begin{itemize}
    \item The directional derivative depends only on $s$ and $X\in T_xM$, not the particular 
    choice of $\gamma$.
    \item The map $\nabla s(x):T_xM\to E_x$ is $\C$-linear.
\end{itemize}
This will give us the richest notion of a connection on a vector bundle.
\subsection{Connections}
In this section, we consider $M$ a real manifold and $\pi:E\to M$ complex vector bundle.
Let $\cA^i(E) = \Omega^i(M) \otimes E$ denote the sheaf of smooth $i$-forms with values in $E$.
\begin{definition}
    A \textbf{connection} on $E$ 
    is a $\C$-linear map of sheaves
    $\nabla:\cA^0(E)\to\cA^1(E)$ satisfying the Leibniz rule
    \begin{align*}
        \nabla(fs) = df\otimes s + f\nabla s
    \end{align*}
\end{definition}
We can interpret this definition in the sense of parallel transport. Given 
a section $s\in\cA^0(E)$, we can differentiate it along a path $\gamma$ to get another 
section of $E$, i.e. $\nabla:\Gamma(E)\to \Gamma(\Hom(TM,E))$.
\begin{theorem}
    The space of all connections $\cA(E)$ is an affine space modelled on $\cA^1(\End E)$. In particular \begin{itemize}
        \item $\cA(E)$ is nonempty
        \item For any two connections $\nabla_1,\nabla_2$ the difference $\nabla_1-\nabla_2$ 
        is a global section of $\cA^1(\End E)$.
        \item $(\nabla + a)s := \nabla s + as$ is a connection whenever $\nabla$ is a connection and 
        $a\in\cA^1(\End E)$.
    \end{itemize}
\end{theorem}
The idea of a connection generalizes the exterior differential to sections of general vector bundles. 
However, a connection need not satisfy $\nabla^2 = 0$ in general. The obstruction for a connection
define a differential is measured by its curvature. We explain this now.

\subsection{Curvature}

A connection $\nabla:\cA^0(E)\to \cA^1(E)$ induces "differentials" \begin{align*}
    \nabla:\cA^i(E)\to\cA^{i+1}(E)
\end{align*} given by the formula \begin{align*}
    \nabla(\alpha\otimes s) = d\alpha\otimes s + (-1)^i\alpha\wedge\nabla s
\end{align*}

\begin{definition}
    The \textbf{curvature} $F_\nabla$ of a connection $\nabla$ is the composition \begin{align*}
        F_\nabla := \nabla^2:\cA^0(E)\to\cA^2(E)
    \end{align*} In particular $F_\nabla$ is a global section of $\cA^2(\End E)$. This is because 
    the curvature homomorphism is $\cA^0$-linear.
\end{definition}

\begin{example}
    Consider the connections on the trivial bundle $M\times \C^r$. If $\nabla = d$ is the
    trivial connection then $F_\nabla = 0$. 

    \hfill 

    Any other connection is of the form $\nabla = d + A$ where $A$ is a matrix of 1-forms. For a section
    $s$ we compute \begin{align*}
        F_\nabla(s) &= (d+A)(d+A)(s) \\
        &= d^2s + dAs + Ads + AAs \\
        &= d(A)s + A\wedge As
    \end{align*} and therefore \begin{align*}
        F_\nabla = dA + A\wedge A
    \end{align*}
    For line bundles we get that $F_\nabla = dA$ is an ordinary 2-form.
\end{example}

\section{Toric symplectic manifolds}
\subsection{Motivation}
Recall the statement we were originally interested in: 

\begin{theorem}
Let $(M,\omega)$ be a symplectic manifold. Suppose that $[\omega]$ is integral. Then 
there exists a "prequantization" line bundle $\cL \to M$ with $c_1(\cL) = [\omega]$ and a Hermitian connection $\alpha$
whose corresponding curvature form is $\omega$. Moreover $\cL$ is unique up to isomorphism 
(but still requires a particular choice). 
\end{theorem}

Now suppose I have a toric symplectic manifold. Delzant's theorem tells us that these are classified
up to equivariant symplectomorphism by smooth integral lattice polytopes. 

\hfill

The data of a polytope determines a line bundle on our toric manifold, which is the prequantization line bundle.
General theory tells us how to get ahold of this line bundle abstractly. It tells us to look at the
integral cohomology class of the symplectic form and then take a line bundle whose first Chern class is this
integral cohomology class (there is also a connection whose curvature is the symplectic form).

\subsection{Construction}
However there is a way to get ahold of this line bundle more concretely using toric geometry. Begin with a
Delzant polytope $\Delta$ in $\R^n$ with $N$ facets and $v_1,\dots,v_N$ normal vectors to the facets. Then 
$\Delta$ is cut out by inequalities \begin{align*}
    \langle v_i,x\rangle \geq \lambda_i
\end{align*} where we assume all the $\lambda_j$ are integers ("$M$ is prequantizable"). There are maps \begin{align*}
    \pi:\R^N\to\R^n
\end{align*} and the induced map \begin{align*}
    \pi: \R^N/\Z^N\to\R^n/\Z^n
\end{align*} of tori, which give rise to the following short exact sequences. \begin{align*}
    1\to K \to \T^N \to \T^n \to 1
\end{align*} \begin{align*}
    0\to k \to \R^N\to \R^n\to 0
\end{align*} Complexifying, we get \begin{align*}
    1\to K_\C \to T_\C^N \to T_\C^n \to 1 
\end{align*} Let $F_i$ denote the facets of $\Delta$ and for any $z = (z_1,\dots,z_N)\in\C^n$ let $F_z := \cap_{z_i = 0}F_i$.
Consider the set \begin{align*}
    U = \{z\in\C^n: F_z \neq \emptyset\}
\end{align*} Then the quotient $M = U/K_\C$ is a manifold with an action of $T_\C^N/K_\C = T_\C^n$. It 
is a toric variety because this construction is secretly taking a projective GIT quotient.

\begin{remark}
    There is a surjective map from $X^{ss}$ to $X//G$. Two points in $X^{ss}$ lie in the same fiber of this map if and only if the closures of their 
    $G$-orbits intersect. In this case, the $K_\C$ orbits are closed. See Proudfoot's notes on toric varieties for more.
\end{remark}

\begin{proposition}
    The line bundle $\cL = U \times_{K_C} \C$ where $K_\C$ acts on $\C$ with 
    weight $\nu = L(-\lambda)$ is a prequantization line bundle for $M = U/K_\C$. Note that $\nu \in k^*$ is the 
    dual Lie algebra of $K_\C$.
\end{proposition}

See Hamilton.

\begin{theorem}
    With the setup above, we have \begin{align*}
        \dim H^0(M,\cL) = \#(\text{integer points in }\Delta)
    \end{align*}
\end{theorem}

\begin{proof}
    A holomorphic section of $\cL$ over $M$ corresponds to a $K_\C$-equivariant holomorphic function $f:U\to \C$. 
    Such $f$ extends to all of $\C^N$ because of Hartogs theorem (A holomorphic function on $\C^N$ for $N>1$ canont have 
    an isolated singularity and therefore cannot have a singularity on a submanifold of codimension $\geq 2$).

    \hfill

    Write such a function as its Taylor series so that \begin{align*}
        f = \sum_{\alpha\in\N^n} c_\alpha z^\alpha
    \end{align*} Consider the equivariance one term at a time. 
    Thinking about the monomial $f(z) = z^I$ we 
    see that \begin{align*}
        f(k\cdot z) &= f(i(k)\cdot z) = (i(k)\cdot z)^I = i(k)^Iz^I = k^{i^*(I)}z^I \\
        k\cdot f(z) &= k^{\nu}z^I
    \end{align*} and therefore a basis for the space of equivariant functions $f:U\to \C$ 
    is given by the monomials corresponding to lattice points in $\Delta$.
\end{proof}

\section{Origami}
Meeting with Tara
\begin{itemize}
    \item When does a toric origami manifold admit the structure of a complex manifold?
    In this case, we suspect that it should also come with the structure of a complex 
    algebraic variety. 
\end{itemize}

\begin{remark}
    When does a locally free sheaf on a variety correspond to a vector bundle with flat connection?
\end{remark}

\section{References}
\begin{itemize}
    \item THE QUANTIZATION OF A TORIC MANIFOLD IS GIVEN BY THE INTEGER LATTICE POINTS IN THE MOMENT
    POLYTOPE MARK D. HAMILTON
    \item Proudfoot's notes on toric varieties
\end{itemize}
\end{document}