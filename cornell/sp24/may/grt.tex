\documentclass[12pt]{article}
\usepackage[english]{babel}
\usepackage[utf8x]{inputenc}
\usepackage[T1]{fontenc}
\usepackage{listings}
\usepackage{tikz}
\usepackage{/Users/songye03/Desktop/math_tex/style/quiver}
\usepackage{/Users/songye03/Desktop/math_tex/style/scribe}

\begin{document}
Songyu Ye

\today

This is a note from my discussion with Allen on May 10, 2024. Today we 
discussed some various things from GRT.

\section{Affine Grassmannian}
Let $S$ be the standard lattice.
Recall we have the affine Grassmannian $\Gr_G = G(\C((t)))/G(\C[t])$ and a
stratification by open cells \begin{align*}
    \Gr^\lambda
\end{align*} consisting of those $\lambda$s for which our nilpotent operator \begin{align*}
    t\vert_{t^{-b}S/L}
\end{align*} has Jordan form $\lambda + b(1, 1, \ldots, 1)$.

\begin{theorem}
    [Lusztig] The intersection cohomology $IH^*(\Gr^\lambda)$ is isomorphic to $V_\lambda$
    as a representation of $G^\vee$.
\end{theorem}

\begin{example}
    When $G = G^\vee = \GL(n)$ and $\lambda = \omega_k$ we end up with the 
    classical statement \begin{align*}
        IH^*(\Gr^{\omega_k}) &= H^*(\Gr(k,n)) \\
        H^*(\Gr(k,n)) &\cong V_{\omega_k}
    \end{align*}
\end{example}
\section{Tannakian reconstruction}

\section{Intersection cohomology}
Some motivation is coming from analysis. In analysis we have $L^1$ functions and
$L^\infty$ functions and the latter in fact forms a ring, the former forming a 
module over this ring. One should think of this as $L^1$ is about 
homology and $L^\infty$ is about cohomology. Moreover $L^1$ functions have a pushforward to a
point, called integrate the function. 

\hfill

Now consider the role of $L^2$ functions. They form an inner product space and establish
Poincare duality, where we can pair $L^1$ functions with $L^\infty$ functions.
This is the intersection cohomology with "middle perversity".

\hfill

\begin{remark}
    Historically there are other perversities but the middle perversity gets the most mileage.
\end{remark}

Algebraic varieties can be stratified by smooth manifolds. The intersection cohomology
is defined in terms of a particular stratification, with the property that if I take 
two stratifications and take their refinement, I will get the same answer.
The way one can think of it is as singular chains where the simplices are transverse
to the stratification.

\begin{example}
    Consider the toric variety corresponding to the moment polytope the Pyramid of Gyza tilting
    on one side about to tip over. Picking a one-parameter subgroup corresponds to 
    projeccting the moment polytope onto the height. Consider the attracting sets of this one parameter subgroup.

    \hfill

    Bialynicki-Birula decomposition says that the attracting set forms a basis for the
    equivariant cohomology of the toric variety. However we have this singular point, whose 
    attracting set is reducible, it is the union of two copies of $\P^2$ along $\P^1$. 
    Indeed the homology of the toric variety is five dimensional, corresponding to the 
    five attracting sets. However the intersection cohomology is six dimensional, corresponding
    to breaking up the reducible attracting set into two pieces.
\end{example}
\end{document}