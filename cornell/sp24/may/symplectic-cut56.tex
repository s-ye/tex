\documentclass[12pt]{article}
\usepackage[english]{babel}
\usepackage[utf8x]{inputenc}
\usepackage[T1]{fontenc}
\usepackage{listings}
\usepackage{tikz}
\usepackage{/Users/songye03/Desktop/math_tex/style/quiver}
\usepackage{/Users/songye03/Desktop/math_tex/style/scribe}

\begin{document}
Songyu Ye

\today

This is a note from my discussion with Tara on May 6, 2024 about symplectic cut
and a way to prove Danilov's presentation for the equivariant cohomology of 
a toric variety.

\section{Symplectic cut}

Suppose one has a symplectic manifold $M$ with an $S^1$ Hamiltonian action.
Consider a regular value $\alpha$ of the $S^1$ moment map $\mu$.
Then the stablizer of a point on $\mu^{-1}(\alpha)$ is necessarily finite. 
Assume that the $S^1$ action is free on $\mu^{-1}(\alpha)$. Then the magic
is that the quotient $\mu^{-1}(\alpha)/S^1$ is a smooth symplectic manifold.

Moreover one can consider the pieces of the symplectic cut \begin{align*}
    M^+ &= \mu^{-1}((\alpha, \infty)) \\ 
    M^- &= \mu^{-1}((-\infty, \alpha)) 
\end{align*} which are non compact symplectic manifolds.
Each of the pieces $M^+ \cup \mu^{-1}(\alpha)$ and $ M^- \cup \mu^{-1}(\alpha)$ are manifolds
with boundary and collapsing the $S^1$ fibers of the boundary gives symplectic manifolds.

\begin{example}
    $S^1$ acts on $\C^2$ via the anti-diagonal action. The symplectic cut corresponds to collapsing
    the fibers of the $\partial D^4 = S^3$ along the Hopf fibration and the resulting symplectic cut
    is $\C \P^2$. On the level of moment maps, this corresponds to taking the first
    quadrant of $\R^2$ and slicing along $x + y = c$ and taking the compact part.

    \hfill

    The point is that any other hyperplane results in something which is not 
    a smooth manifold. The fixed points at the corners have nontrivial stabilizers.
\end{example}

\section{Equivariant cohomology of toric varieties}
Recall Danilov's theorem: \begin{theorem}
    Let $X = X(P)$ be a toric variety associated to a polytope $P$.
    Then the equivariant cohomology $H_T^*(X)$ is isomorphic to the Stanley Reisner ring of $P$
    which we will write with the following flabby presentation: \begin{align*}
        H_T^*(X) = \C[x_f]/\cI
    \end{align*} where $f$ runs over all faces of $P$ and $\cI$ is the ideal generated by
    \begin{align*}
        \prod x_f = \sum_{\text{face } F \in \cap f} x_F
    \end{align*}
\end{theorem}
The hope is to apply excision to the fold to prove this theorem in 
the case that we have a template of two polytopes glued along a face.
\end{document}