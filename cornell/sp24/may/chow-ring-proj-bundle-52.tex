\documentclass[12pt]{article}
\usepackage[english]{babel}
\usepackage[utf8x]{inputenc}
\usepackage[T1]{fontenc}
\usepackage{listings}
\usepackage{tikz}
\usepackage{/Users/songye03/Desktop/math_tex/style/quiver}
\usepackage{/Users/songye03/Desktop/math_tex/style/scribe}

\begin{document}
Songyu Ye

\today

This is a note for my talk with Mike Stillman on 5/2/2024. We are talking about the Chow ring of projective bundles and some
applications thereof.

\section{Introduction}
\begin{definition}
	A projective bundle on a scheme $X$ is a fiber bundle $\pi:\cE\to X$ where every $x\in X$ has a
	Zariski open set for which $\pi^{-1}(U)\cong U\times \P^n$ for some $n$.
\end{definition}

To every vector bundle over $X$ we can associate a projective bundle by
"projectivizing the fiber". Formally we define \begin{align*}
	\P\cE := \Proj_X(\Sym \cE^*)
\end{align*}
to be the projective bundle associated to $\cE$. The projective bundle $\P\cE$
comes with a tautological line bundle $\cO_{\P\cE}(-1) \subset \pi^*\cE$, whose fiber over $x\in \P\cE$
are the points $v\in \cE_x$ such that $v\in x$.

\hfill

$\P\cE$ comes with the following universal property: Consider
any other projective bundle $p:Y\to X$. Then maps $Y\to \P\cE$ are in bijection
with line subbundles $\cL\subset p^*\cE$. Given a map $f:Y\to \P\cE$, the corresponding
line subbundle is the pullback of $\cO_{\P\cE}(-1)$ along $f$. Conversely,
given a line subbundle $\cL\subset p^*\cE$, the corresponding map $Y\to \P\cE$ is the
map $y\mapsto [\cL_y] \in \P\cE$ is the fiber $\cL_y\subset \cE_{p(y)}$.

\section{Defining projective bundles}
One can ask the question if every projective bundle arises as the projectivization of a vector bundle.
The answer is yes. Moreover, one cannot recover $\cE$ from $\P\cE$ as a vector bundle,
one can only recover it up to twist by a line bundle. We prove these claims now.

\hfill

Consider the bundle $\cE^*$ and its pullback $\pi^*\cE^*$. There is a surjection
$\pi^*\cE^*\to \cO_{\P\cE}(1)$ coming from dualizing the inclusion of tautological
bundle. Concretely this map over a fiber $(x,\zeta) \in \P\cE$ takes a linear form on
$\cE_x$ to the restriction of that form to $\zeta\subset \cE_x$. Therefore,
global sections $\sigma$ of $\cE^*$ give rise to sections $\tilde\sigma$
of $\cO_{\P\cE}(1)$. The total space of $\cO_{\P\cE}(1)$ is over each $(x,\zeta)\in \P\cE$
are the linear functionals on $\zeta$.

\hfill

Say $\cE^*$ has a global section $\sigma$. We consider the zero locus of $\tilde\sigma$.
Over $\P\cE$ the zero locus will meet generic fiber in a hyperplane, except for when
$\sigma$ is zero. In this case, the zero locus will meet the whole fiber.

\hfill

This allows us to characterize projective bundles as follows:

\begin{theorem}
	Let $\pi: Y\to X$ smooth morphism of projective schemes whose
	scheme-theoretic fibers are all $\cong \P^r$. TFAE:
	\begin{enumerate}
		\item $Y$ is the projective bundle associated to a vector bundle $\cE$ over $X$.
		\item $Y$ is a projective bundle over $X$.
		\item There is a line bundle $\cL$ on $Y$ whose restriction to each fiber
		      is isomorphic to $\cO_{\P^r}(1)$.
		\item There is a Cartier divisor $D\subset Y$ intersecting a general fiber
		      of $\pi$ in a hyperplane.
	\end{enumerate}
\end{theorem}

\begin{proof}
	(1)$\Rightarrow$(2),(3) is clear.

	\hfill

	(3)$\iff$(4) is true. Given a divisor $D$, the corresponding line bundle restricted to
	a general fiber is $\cO_{\P^r}(1)$ and then by flatness the same is true for all fibers.
	Conversely, given a line bundle $\cL$ as in part (c) we can tensor with the
	pullback of an ample line bundle from $X$ to get a nonzero global section of $\cL$,
	whose zero locus is the desired divisor.

	\hfill

	(2)$\Rightarrow$(4): Trivialize over $U$ and choose a hyperplane $H\subset \P^r$ and take
	$D$ the closure of $U\times H$ in $U\times \P^r$.

	\hfill

	(3)$\Rightarrow$(1):
\end{proof}

\section{Chow ring of projective bundles}
We now compute the Chow ring of a projective bundle. The statement is as follows:

\begin{theorem}
	Let $\cE$ rank $r+1$ vector bundle on smooth projective scheme $X$ and let
	$\zeta = c_1(\cO_{\P\cE}(1)) \in A^1(X)$. Then the Chow ring of $\P\cE$ is
	\begin{align*}
		A(\P\cE) = \Z[\zeta]/(\zeta^{r+1} + c_1(\cE)\zeta^r + \cdots + c_r(\cE))
	\end{align*}
	In particular the pullback map $\pi^*: A^*(X) \to A^*(\P\cE)$ is injective and
	there is an isomorphism of groups $A(X)^{r+1} \cong A(\P\cE)$ via \begin{align*}
		(a_0,\ldots,a_r) \mapsto \pi^*(a_0)\zeta^r + \cdots + \pi^*(a_r)
	\end{align*}
\end{theorem}

The fact that the pullback map is injective rests on the following claim which I don't
quite understand:

\begin{lemma}
	We have \begin{align*}
		\pi_*(\zeta^i\pi^*(a)) = \begin{cases}
			                         a & i = r \\
			                         0 & i < r
		                         \end{cases}
	\end{align*}
\end{lemma}

One argues this using the push-pull/projection formula, which I am not sure
why is true:

\begin{theorem}
	For $f:X\to Y$ a morphism of ringed spaces and $\cF$ a sheaf of $\cO_X$-modules,
	$\cG$ locally free of finite rank, there is a natural isomorphism \begin{align*}
		f_*(\cF\otimes f^*\cG) \cong f_*(\cF)\otimes \cG
	\end{align*}
\end{theorem}

\begin{remark}
	What is this theorem really saying?
	One interpretation that sort of made sense to me from reading online is
	in terms of differential forms and thinking about the pushforward of a form as
	integration along the fiber.
\end{remark}

With the projection formula we conclude that \begin{align*}
	\pi_*(\zeta^i\pi^*(a)) = \pi_*(\zeta^i)a
\end{align*} and $\pi_*(\zeta^i) = 0$ for $i < r$ because of dimension reasons.
In particular the pushforward of a cycle along a proper map is a cycle
of the same dimension. This is by definition because Fulton defines \begin{align*}
	\pi_*([V]) := \deg(V/W)[W]
\end{align*} where $V$ closed subvariety, $W = f(V)$ and $\deg(V/W)$ is the degree of the
field extension, which is put to zero if $\dim W \neq \dim V$.

\begin{remark}
	The projection formula is about sheaves, but here we are talking about the Chow ring.
	Why should the way that pushforward is defined for sheaves agree with the
	way that pushforward is defined for cycles?\end{remark}

Finally we have $\pi_*(\zeta^r) = 1$. It is a multiple $m$ of the fundamental class of $X$.
Consider $\eta$ the class of a point in $X$. It pulls back to the class of a fiber
$[\P\cE_x]$. Intersecting both sides with $\eta$ we get that \begin{align*}
	m = \deg(\pi_*(\zeta^r)\cdot \eta) = \deg(\zeta^r\cdot \pi^*(\eta)) = 1
\end{align*} since the restriction of $\zeta$ to a fiber is the class of a hyperplane.

\hfill

The rest of the proof is quite techincal. First one shows that subvarieties of $\P\cE$ can 
be expanded in terms of the $\zeta^i$. This is similar to the computation for 
$\P^n$ where we can expand a subvariety in terms of the hyperplane class.
Then one shows that $\zeta^{r+1}$ can be 
expanded in terms of the lower powers of $\zeta$ and there can only be one relation
because of the additive presentation of the Chow ring. Finally one writes down the relation 
by looking at Chern classes of the exact sequence \begin{align*}
    0\to \cO_{\P\cE}(-1) \to \pi^*\cE \to \cQ \to 0
\end{align*} \red{The hard part is the first step and to be honest,
I omit the proof because I don't quite understand it.}

\section{Example: rational normal scrolls}

\begin{claim}
    Every vector bundle on $\P^1$ is a direct sum of line bundles.
\end{claim}

\begin{remark}
    I quite like the proof of this fact.
\end{remark}

\begin{proof}
    An exact sequence of vector bundles \begin{align*}
        0 \to \cE \to \cF \xrightarrow{\alpha} \cG \to 0
    \end{align*} splits iff $\alpha$ has a right section. This is the case iff
    the following sequence is exact \begin{align*}
        0 \to \Hom(\cG,\cE) \to \Hom(\cG,\cF) \to \Hom(\cG,\cG) \to 0
    \end{align*}
    and this is the case if and only if $H^1(\Hom(\cG,\cE)) = H^1(\cG^*\otimes \cE) = 0$.

    Suppose $\cE$ is a rank 2 vector bundle on $\P^1$ of degree $d$.
    By Riemann Roch we have \begin{align*}
        h^0(\cE) \geq d + 2
    \end{align*}
    so there exists a nonzero global section of $\cE$ vanishing at $m\geq d/2$ points
    of $\P^1$. 

    Therefore $\cO_{\P^1}(m) \subset \cE$ and we have \begin{align*}
        0 \to \cO_{\P^1}(m) \to \cE \to \cO_{\P^1}(m-d) \to 0
    \end{align*} and since $2m - d\geq 0$ we get that $H^1(\cO_{\P^1}(2m-d)) = 0$
    and the sequence splits.

    Inductively we have \begin{align*}
        0 \to \cO_{\P^1}(m) \to \cE \to \cF \to 0
    \end{align*} where $m$ is the maximal degree
    of a subline bundle and where $\cF$ splits as a direct sum of line bundles $\cL_i
    \cong \cO_{\P^1}(m_i)$ with $m_i \leq m$. Thus the sequence splits.
\end{proof}

\end{document}