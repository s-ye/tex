\documentclass[12pt]{article}
\usepackage[english]{babel}
\usepackage[utf8x]{inputenc}
\usepackage[T1]{fontenc}
\usepackage{listings}
\usepackage{tikz}
\usepackage{/Users/songye03/Desktop/math_tex/style/quiver}
\usepackage{/Users/songye03/Desktop/math_tex/style/scribe}

\begin{document}
Songyu Ye

These are notes from my meeting with Allen on April 19.
Today it was kinda cooked. We should try to talk about something else.

\section{Hilbert basis for the Kostka cone}
It seems kinda doomed to look for a basis of the Kostka cone.
In particular, there's a paper by Gao and Yong that says that
such a decision problem is NP-hard. So we should probably try to say something else.

\hfill

Allen said he's run into this type of problem before. The workaround to understand some type of partial 
result is to try to find $\Z$-generators for the $\Z$-cone. This amounts to 
looking at a localization of the variety, in particular some open torus. Stuff about cluster algebras here.

\hfill

He also told me about his previous work with Terry Tao, honeycombs and hives and $\GL_n$ invariants of $\GL_n/N^3$. In that situation the cone was also nasty. 
Allen is hoping that the the fluff up with the patterns makes the cone better.

\section{New things to think about}
I think I would like to learn about something else:

\begin{itemize}
    \item $T$-equivariant motives of the flag variety
    \item $\A^1$-homotopy theory in enumertive geometry: presents 
    a unification of theories like $K$, $H$, Chow, cobordism
    \item Schubert calculus for $\A^1$-cohomology theories
    \item Motivic Segre Classes with PZJ and positivity for those classes
    \item Positivity for affine flag varieties and 
    semi-infinite flag varieties
\end{itemize}

\end{document}