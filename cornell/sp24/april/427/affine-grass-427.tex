\documentclass[12pt]{article}
\usepackage[english]{babel}
\usepackage[utf8x]{inputenc}
\usepackage[T1]{fontenc}
\usepackage{listings}
\usepackage{tikz}
\usepackage{/Users/songye03/Desktop/math_tex/style/quiver}
\usepackage{/Users/songye03/Desktop/math_tex/style/scribe}

\begin{document}
Songyu Ye
\today

\hfill

This is a note on the new project I am beginning with Allen because it seems like the previous project
has died.

\section{Motivation for the affine Grassmannian}
Before we start we consider a motivating example. Consider the space 
$\GL(n,\Q)/\GL(n,\Z)$, which we realize as the homogenous space of lattices $A$ in $\Q^n$ for which $A\cap \Z^n$
is finite index inside both $A$ and $\Z^n$. This amounts to the condition that $\Q \otimes_Z A \cong \Q^n$.
This comes equipped with a map to $\Q^*/\Z^* \cong \Q^+$ given by the determinant, which sends a lattice to
its volume.

\begin{definition}
    The affine Grassmannian is the space $\Gr_G = \GL_n(\C((t)))/\GL_n(\C[[t]])$ where
    $\C((t))$ is the field of formal Laurent series (where we only allow
    finitely many negative terms) and $\C[[t]]$ is the ring of formal power series.
\end{definition}

\begin{center}
    \includegraphics[scale = .1]{/Users/songye03/Desktop/math_tex/img/IMG_2046.JPG}
\end{center}
We realize this space as the space of "lattices" $A$ in $\C((t))^n$ for which $A$ is a $C[[t]]$-submodule
of $\C((t))^n$ and $A/A\cap \C[[t]]^n$ and $\C[[t]]^n/A\cap \C[[t]]^n$ are a finite dimensional $\C$-vector spaces.

\hfill

This space comes equipped with a map to $\C((t))^*/\C[[t]]^* \cong \Z$ given by the order, which sends a coset
to the order of its determinant, equal to the difference of the two indices. The order is a discrete invariant
and since the affine Grassmannian is a homogenous space, we see that it is disconnected with components corresponding to 
different orders. Moreover we can draw pictures of elements in the affine Grassmannian 
by drawing an individual lattice by the orders of its basis elements, with everything to the right 
shaded because of the proerpty of being a $C[[t]]$-submodule.

\hfill

In particular we see that there is a filtration by $a,b$ so that the set of lattices \begin{align}
    z^a \C[[t]]^n \leq A \leq z^b \C[[t]]^n
\end{align} is a finite dimensional variety. In particular although the affine Grassmannian is 
infinite dimensional, it is a union of finite dimensional varieties. In particular any given 
calculation is finite dimensional.

\begin{remark}
    The affine Grassmannian is an example of something called an \red{ind-scheme},
    an object which is a filtered colimit of schemes where the 
    transition maps are closed embeddings. Suppose you wanted to tell me
    a vector bundle on an ind-scheme. Morally this should be a vector
    bundle on each of the finite dimensional pieces, which is compatible
    with the transition maps. 
\end{remark}

\section{Deformation retract of the loop group}
Consider $K$ a maximal compact subgroup of a complex reductive group $G$ (i.e. $U(n)$ in $GL(n,\C)$).
Consider $\Omega K$ the space of smooth loops in $K$ based at the identity. \begin{align}
    \Omega K = \Map_{\cdot}(S^1,K)
\end{align} which is a homogenous space for the space of free loops $\Map(S^1,K)$ by the action of
the constant loops. The loop space $\Omega K$ is a topological group with the pointwise multiplication
of loops. It has an action of $K$ by conjugation.

\begin{remark}
    Allen says although it is both a group and a space, in this context we want to largely
    think about it as a space.
\end{remark}

One can try to apply Morse theory to $\Omega K$ to try to understand its topology. This is 
in fact what Bott did and let to the discovery of Morse-Bott theory. Recall for any 
Riemannian manifold $M$, there is a map $LM \to \R_{\geq 0}$ given by \begin{align}
    \gamma \mapsto \int_{S^1} \|\gamma'(t)\|^2 dt
\end{align} called the \red{action functional}. This is a Morse function on $LM$ for generic
choice of metric. Recall in traditional Morse theory, isolated nondegenerate critical points
yield a cell decomposition of the manifold. However, when we apply this recipe to $\Omega K$ 
we find that \begin{align}
    (\Omega K)_{\text{crit}} = \Hom_{\text{gp}}(S^1,K)
\end{align} Moreover given such a group homomorphism, one can conjugate it by $K$ to lie in $T$
the maximal torus of $K$. Thus we see that the critical points of the action functional are
\begin{align*}
    \Hom_{\text{gp}}(S^1,K) = \coprod_{\text{dominant coweights $\lambda$}} K \cdot t^\lambda
\end{align*} where $t^\lambda$ is the loop corresponding to the coweight $\lambda$ and 
moreover one can show that \begin{align*}
    K \cdot t^\lambda \cong G/P(\lambda)
\end{align*} so in particular we see that the critical points are not isolated.
Hence we need to use Morse-Bott theory to understand the topology of $\Omega K$.

\section{Morse Bott theory}
The point of Morse-Bott theory is that when the critical points of a Morse function are not isolated,
but rather form a submanifold, one can still say something about the topology of the manifold.
In particular we get a decomposition of the manifold into vector bundles over the critical submanifolds.

\begin{example}
    Consider the torus sitting as a donut on the table. The height function now has two critical circles
    which are the top and bottom of the donut. Then we see that the torus admits a decomposition as line bundles 
    over those two circles with the fibers identified appropriately.
\end{example}

\begin{example}
    Consider the torus balancing on its side, the standard height function has 4 critical points and indeed 
    recovers the decomposition of the torus as a cell complex. Consider if we 
    pinch the top of the torus toward infinity. Then when we run the Morse machine, we only get $0$-cell,
    and $2$ 1-cells. The point is we only get something homotopic to that which we started with.
\end{example}
Running the machine, we get that the Morse-Bott cell of the critical submanifold $K \cdot t^\lambda$ is equal to \begin{align*}
    G(\C((t)))\cdot t^\lambda \twoheadrightarrow G\cdot t^\lambda \cong G/P(\lambda)
\end{align*} The closure of this guy we will call the \red{affine Schubert cell} $\Gr^\lambda$.
\begin{center}
    \includegraphics[scale = .1]{/Users/songye03/Desktop/math_tex/img/IMG_2047.JPG}
\end{center}
In this way we see that the affine Grassmannian $\Gr_G$ is a deformation retract of the loop group $\Omega K$.

\hfill

We can draw the various $G/P(\lambda)$ with the jellyfish picture, where downwards is about which 
cells are in the closure of which other cells.

\section{Langlands dual group}
As in the finite dimensional case there is a stratificaaiton \begin{align*}
    \Gr^\lambda = \coprod_{\mu \leq \lambda} \Gr^\mu_o
\end{align*} dominance order. Studying the $T$-fixed points we find that \begin{align*}
    \text{weights in $V_\lambda$ for $\prescript{L}{}{G}$}= (\Gr^\lambda)^T = \coprod_{\mu \leq \lambda} \big((\Gr^\mu_o)^T = W\cdot t^\mu\big)
\end{align*}

\begin{example}
    For representations of $\SL(3)$ we have a dominant weight $\lambda$. Apply the Weyl group, take 
    the convex hull, and all of the weights in the interior are of course weights of the representation.
    What the above is saying is that the weights of the interior are those corresponding to the Weyl group 
    orbits through all of the $\mu$ for which $\mu \leq \lambda$ (in the Langlands dual group setting).
\end{example}

However one thing that one also wants to know is the dimension of the weight space $\mu$. For that we have the theorem: \begin{theorem}
    [Mirkovic-Vilonen] The dimension of the weight space $\mu$ is equal 
    to the number of components of the attracting set of $t^\mu$ in $\Gr^\lambda$.
\end{theorem}

\begin{example}
    Take the pyramid of Gaza and tilt it on one of its base points, so that the square base is 
    facing away from you. \red{??}
\end{example}
\section{Things that have to be learned}
\begin{enumerate}
    \item Morse theory and Morse-Bott theory
    \item Affine Grassmannian
\end{enumerate}
\documentclass[12pt]{article}
\usepackage[english]{babel}
\usepackage[utf8x]{inputenc}
\usepackage[T1]{fontenc}
\usepackage{listings}
\usepackage{tikz}
\usepackage{/Users/songye03/Desktop/math_tex/style/quiver}
\usepackage{/Users/songye03/Desktop/math_tex/style/scribe}

\begin{document}
Songyu Ye
\today

\hfill

This is a note on the new project I am beginning with Allen because it seems like the previous project
has died.

\section{Motivation for the affine Grassmannian}
Before we start we consider a motivating example. Consider the space 
$\GL(n,\Q)/\GL(n,\Z)$, which we realize as the homogenous space of lattices $A$ in $\Q^n$ for which $A\cap \Z^n$
is finite index inside both $A$ and $\Z^n$. This amounts to the condition that $\Q \otimes_Z A \cong \Q^n$.
This comes equipped with a map to $\Q^*/\Z^* \cong \Q^+$ given by the determinant, which sends a lattice to
its volume.

\begin{definition}
    The affine Grassmannian is the space $\Gr_G = \GL_n(\C((t)))/\GL_n(\C[[t]])$ where
    $\C((t))$ is the field of formal Laurent series (where we only allow
    finitely many negative terms) and $\C[[t]]$ is the ring of formal power series.
\end{definition}

\begin{center}
    \includegraphics[scale = .1]{/Users/songye03/Desktop/math_tex/img/IMG_2046.JPG}
\end{center}
We realize this space as the space of "lattices" $A$ in $\C((t))^n$ for which $A$ is a $C[[t]]$-submodule
of $\C((t))^n$ and $A/A\cap \C[[t]]^n$ and $\C[[t]]^n/A\cap \C[[t]]^n$ are a finite dimensional $\C$-vector spaces.

\hfill

This space comes equipped with a map to $\C((t))^*/\C[[t]]^* \cong \Z$ given by the order, which sends a coset
to the order of its determinant, equal to the difference of the two indices. The order is a discrete invariant
and since the affine Grassmannian is a homogenous space, we see that it is disconnected with components corresponding to 
different orders. Moreover we can draw pictures of elements in the affine Grassmannian 
by drawing an individual lattice by the orders of its basis elements, with everything to the right 
shaded because of the proerpty of being a $C[[t]]$-submodule.

\hfill

In particular we see that there is a filtration by $a,b$ so that the set of lattices \begin{align}
    z^a \C[[t]]^n \leq A \leq z^b \C[[t]]^n
\end{align} is a finite dimensional variety. In particular although the affine Grassmannian is 
infinite dimensional, it is a union of finite dimensional varieties. In particular any given 
calculation is finite dimensional.

\begin{remark}
    The affine Grassmannian is an example of something called an \red{ind-scheme},
    an object which is a filtered colimit of schemes where the 
    transition maps are closed embeddings. Suppose you wanted to tell me
    a vector bundle on an ind-scheme. Morally this should be a vector
    bundle on each of the finite dimensional pieces, which is compatible
    with the transition maps. 
\end{remark}

\section{Deformation retract of the loop group}
Consider $K$ a maximal compact subgroup of a complex reductive group $G$ (i.e. $U(n)$ in $GL(n,\C)$).
Consider $\Omega K$ the space of smooth loops in $K$ based at the identity. \begin{align}
    \Omega K = \Map_{\cdot}(S^1,K)
\end{align} which is a homogenous space for the space of free loops $\Map(S^1,K)$ by the action of
the constant loops. The loop space $\Omega K$ is a topological group with the pointwise multiplication
of loops. It has an action of $K$ by conjugation.

\begin{remark}
    Allen says although it is both a group and a space, in this context we want to largely
    think about it as a space.
\end{remark}

One can try to apply Morse theory to $\Omega K$ to try to understand its topology. This is 
in fact what Bott did and let to the discovery of Morse-Bott theory. Recall for any 
Riemannian manifold $M$, there is a map $LM \to \R_{\geq 0}$ given by \begin{align}
    \gamma \mapsto \int_{S^1} \|\gamma'(t)\|^2 dt
\end{align} called the \red{action functional}. This is a Morse function on $LM$ for generic
choice of metric. Recall in traditional Morse theory, isolated nondegenerate critical points
yield a cell decomposition of the manifold. However, when we apply this recipe to $\Omega K$ 
we find that \begin{align}
    (\Omega K)_{\text{crit}} = \Hom_{\text{gp}}(S^1,K)
\end{align} Moreover given such a group homomorphism, one can conjugate it by $K$ to lie in $T$
the maximal torus of $K$. Thus we see that the critical points of the action functional are
\begin{align*}
    \Hom_{\text{gp}}(S^1,K) = \coprod_{\text{dominant coweights $\lambda$}} K \cdot t^\lambda
\end{align*} where $t^\lambda$ is the loop corresponding to the coweight $\lambda$ and 
moreover one can show that \begin{align*}
    K \cdot t^\lambda \cong G/P(\lambda)
\end{align*} so in particular we see that the critical points are not isolated.
Hence we need to use Morse-Bott theory to understand the topology of $\Omega K$.

\section{Morse Bott theory}
The point of Morse-Bott theory is that when the critical points of a Morse function are not isolated,
but rather form a submanifold, one can still say something about the topology of the manifold.
In particular we get a decomposition of the manifold into vector bundles over the critical submanifolds.

\begin{example}
    Consider the torus sitting as a donut on the table. The height function now has two critical circles
    which are the top and bottom of the donut. Then we see that the torus admits a decomposition as line bundles 
    over those two circles with the fibers identified appropriately.
\end{example}

\begin{example}
    Consider the torus balancing on its side, the standard height function has 4 critical points and indeed 
    recovers the decomposition of the torus as a cell complex. Consider if we 
    pinch the top of the torus toward infinity. Then when we run the Morse machine, we only get $0$-cell,
    and $2$ 1-cells. The point is we only get something homotopic to that which we started with.
\end{example}
Running the machine, we get that the Morse-Bott cell of the critical submanifold $K \cdot t^\lambda$ is equal to \begin{align*}
    G(\C((t)))\cdot t^\lambda \twoheadrightarrow G\cdot t^\lambda \cong G/P(\lambda)
\end{align*} The closure of this guy we will call the \red{affine Schubert cell} $\Gr^\lambda$.
\begin{center}
    \includegraphics[scale = .1]{/Users/songye03/Desktop/math_tex/img/IMG_2047.JPG}
\end{center}
In this way we see that the affine Grassmannian $\Gr_G$ is a deformation retract of the loop group $\Omega K$.

\hfill

We can draw the various $G/P(\lambda)$ with the jellyfish picture, where downwards is about which 
cells are in the closure of which other cells.

\section{Langlands dual group}
As in the finite dimensional case there is a stratificaaiton \begin{align*}
    \Gr^\lambda = \coprod_{\mu \leq \lambda} \Gr^\mu_o
\end{align*} dominance order. Studying the $T$-fixed points we find that \begin{align*}
    \text{weights in $V_\lambda$ for $\prescript{L}{}{G}$}= (\Gr^\lambda)^T = \coprod_{\mu \leq \lambda} \big((\Gr^\mu_o)^T = W\cdot t^\mu\big)
\end{align*}

\begin{example}
    For representations of $\SL(3)$ we have a dominant weight $\lambda$. Apply the Weyl group, take 
    the convex hull, and all of the weights in the interior are of course weights of the representation.
    What the above is saying is that the weights of the interior are those corresponding to the Weyl group 
    orbits through all of the $\mu$ for which $\mu \leq \lambda$ (in the Langlands dual group setting).
\end{example}

However one thing that one also wants to know is the dimension of the weight space $\mu$. For that we have the theorem: \begin{theorem}
    [Mirkovic-Vilonen] The dimension of the weight space $\mu$ is equal 
    to the number of components of the attracting set of $t^\mu$ in $\Gr^\lambda$.
\end{theorem}

\begin{example}
    Take the pyramid of Gaza and tilt it on one of its base points, so that the square base is 
    facing away from you. \red{??}
\end{example}
\section{Things that have to be learned}
\begin{enumerate}
    \item Morse theory and Morse-Bott theory
    \item Affine Grassmannian
\end{enumerate}
\end{document}