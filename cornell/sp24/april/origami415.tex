\documentclass[12pt]{article}
\usepackage[english]{babel}
\usepackage[utf8x]{inputenc}
\usepackage[T1]{fontenc}
\usepackage{listings}
\usepackage{tikz}
\usepackage{/Users/songye03/Desktop/math_tex/style/quiver}
\usepackage{/Users/songye03/Desktop/math_tex/style/scribe}

\begin{document}
Songyu Ye

These are notes from my meeting with Tara on April 15. 

\section{Blowing down -1 curves}
Consider the symplectic toric 4 manifold given by the following polytopes, a square and 
a trapezoid identified along the straight vertical edge. 

\hfill

Computing intersection forms along the T-invariant curves shows us that there are 2 curves
with self intersection $\pm 1$. The question is can we blow down these curves?

\hfill 

By choosing an orientation we shuold really try to blow down the -1 curve at the top.
I tried drawing moment polytopes that would do so, but one really is unable to make it work.
It works if I am allowed to scale the sympletic form on the manifold.

\section{Maps between toric origami manifolds}
$T$-equivariant maps between toric varieties correspond to maps on the fans.
However, I really want to keep track of the polytopes (i.e. the data of the symplectic form)

\begin{remark}
    \begin{enumerate}
        \item Is there a polytopal characterization of when there is a map between two
        toric symplectic manifolds?
        \item Are maps between toric origami manifolds assembled from maps 
        between the corresponding symplectic cut spaces?
        \item If I have a $T$-invariant $-1$ curve on a toric origami manifold, how can I 
        see them on the symplectic cut space?
    \end{enumerate}
\end{remark}


\section{References}
\begin{enumerate}
    \item Cannas Da Silva, Symplectic Origami https://arxiv.org/pdf/0909.4065.pdf Radial blowup
\end{enumerate}
\end{document}