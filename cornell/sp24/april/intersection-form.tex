\documentclass[12pt]{article}
\usepackage[english]{babel}
\usepackage[utf8x]{inputenc}
\usepackage[T1]{fontenc}
\usepackage{listings}
\usepackage{tikz}
\usepackage{/Users/songye03/Desktop/math_tex/style/quiver}
\usepackage{/Users/songye03/Desktop/math_tex/style/scribe}

\begin{document}
Songyu Ye

This is a note on the intersection form of 4 manifolds and
in particular on complex projective surfaces.

\section{Introduction}
The intersection form of a compact oriented topological
4-manifold $X$ is an invariant that effectively captures the
cohomological information of $X$.

\hfill

Restrict to the case where $X$ is closed and simply connected. In this case,
$H^0$ and $H^4$ are $\Z$, $H^{\text{odd}}$ is $0$ because $\pi_1(X)$ is trivial and
Poincare duality. Moreover $H^2$ is free abelian because
\begin{align*}
	H_2(X) \cong H^2(X) \cong \Hom_{\Z}(H_2(X), \Z)
\end{align*}

\begin{theorem}
	[Freedman] If $X$ is a simply connected smooth 4 manifold, then
	its homeomorphism type is determined by its intersection form.
\end{theorem}

\section{Intersection form}
The intersection form is a bilinear form on $H^2(X)$ (or $H_2(X)$ via
Poincare duality) defined by \begin{align*}
	Q_X(\alpha, \beta) = \alpha \cdot \beta \cap [X]
\end{align*}

When $X$ is a smooth manifold, $Q_X$ has the following geometric interpretation.
We say that a class $\alpha \in H_2(X)$ is represented by a
embedded surface $S$ if $\alpha$ is the image of the fundamental class
under the inclusion map $i: S \to X$.

\hfill

The integer $Q_X(\alpha, \beta)$ is the intersection number of representatives
for the homology classes $\alpha$ and $\beta$.

\begin{remark}
	In particular whenever $X$ is closed,
	oriented, smooth 4 manifold, then every element of $H_2(X)$ is represented by an embedded
	surface.
    
    \hfill
    
    This is because we can identify elements of $H_2(X)$ with isomorphism classes
	of line bundles on $X$ with structure group $U(1)$ (call this group $\cL(X)$). In particular we know that \begin{align*}
		H_2(X) = [X,K(\Z,2)]
	\end{align*} and \begin{align*}
		\cL(X) = [X,BU(1)]
	\end{align*} and both of these spaces $K(\Z,2)$ and $BU(1)$ are homotopy
	equivalent to $\C\P^\infty$.

    \hfill

    Then if $\alpha$ is a homology class, let $\cL_\alpha$ the corresponding line bundle
    and then the representing submanifold is precisely the zero locus of a generic section
    of $\cL_\alpha$.
\end{remark}

To understand the intersection number, choose the
representatives so that the intersections are transverse. 
Observe that the orientations on the
representatives of $\alpha$ and $\beta$ induce an orientation on their intersection,
so we can assign a sign to each point of thier intersection. In particular
concatenate positive bsaes for the tangent spaces at the intersection points
to get an orientation for intersection.




\section{Surfaces in projective space}

\end{document}