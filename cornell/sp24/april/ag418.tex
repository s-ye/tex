\documentclass[12pt]{article}
\usepackage[english]{babel}
\usepackage[utf8x]{inputenc}
\usepackage[T1]{fontenc}
\usepackage{listings}
\usepackage{tikz}
\makeatletter
\def\input@path{{../../style/}}
\makeatother

\usepackage{../../style/quiver}
\makeatletter
\def\input@path{{../../style/}}
\makeatother

\usepackage{../../style/scribe}

\begin{document}
Songyu Ye

These are notes from my meeting with Mike on April 18, 2024.
We are talking about 27 lines on a cubic surface and the geometry
of the Fano scheme.

\section{Enumerative problem}
Given a cubic surface $X \subset \P^3$, we can ask how many lines
are contained in $X$. The answer is gotten by computing the
degree of a particular Chern class. If you believe that
that number is the answer, then the difficulty lies in
interpreting this number. What geometry does this number
really mean?

\section{Fano scheme}
We study the following object which parametrizes $k$-planes
on a hypersurface $X$ in $\P^n$. Let $f$ be a degree $d$ polynomial and
$X = V(f)$. The Fano scheme $F_k(X) \subset G(k,n)$
is the scheme that parametrizes
$k$-planes contained in $X$. It has a scheme structure because
on an open chart of the Grassmannian $G(k,n)$ we have coordinates
given by the row span of the matrix \begin{align*}
	\begin{bmatrix}
		I & A
	\end{bmatrix}
\end{align*} where $A$ is a $n-k \times k+1$ matrix. Using these coordinates,
we can identify each $[L] \in G(k,n)$ with $\P^k$, say with coordinates
$s_0, \ldots, s_k$. Then on the above open set $U$,
the Fano scheme corresponding to a hypersurface $V(f)$ is cut
out by the vanishing of the coefficients of the pullback of $f$ to $U$. In particular
there will be $\binom{k+d}{d}$ equations cutting out the Fano scheme locally. This is
because we are asking that all coefficients on monomials of degree $d$ in $k+1$ variables
vanish.

\section{The Chern class computation}
Recall the very nice thing about Chern classes that we like.

\begin{proposition}
    The top Chern class of a vector bundle $\cE$ is equal to the class of $V(\sigma)$
    where $\sigma$ is any global section of $\cE$.
\end{proposition}

The following proposition relates the class 
of the Fano scheme to the Chern class of a particular vector bundle.

\begin{proposition}
    Consider the tautological vector bundle $\cS$ on $G(k,\P V)$. Let $g$
    be a degree $d$ form on $\P V$. Then $g$ gives rise to a global section $\sigma_g$ of 
    the vector bundle $\Sym^d \cS^*$ whose zero locus is $F_k(V(g))$. In particular,
    the class of the Fano scheme $F_k(V(g))$ is equal to the top ( $= \binom{k+d}{k}$) Chern class
    of $\Sym^d \cS^*$ \red{whenever $F_k(V(g))$ has the right dimension. What goes wrong?}.
\end{proposition}

Now consider the following computation. Recall for $G(1,3)$ we have \begin{align*}
    c(\cS^*) = 1 + \sigma_1 + \sigma_{1,1}
\end{align*} We proved this using degeneraci loci. In particular global sections of
$\cS^*$ are linear forms and their degeneracy loci correspond to incidence conditions. 
Which ones? Precisely those ones above.

\hfill

$\Sym^3\cS^*$ has rank $4$ since $\cS^*$ has rank $2$, so therefore we are after 
$c_4(\Sym^3\cS^*)$. We can compute this by the splitting principle. We have \begin{align*}
    c(\cS^*) = 1 + \sigma_1 + \sigma_{1,1} = (1+\alpha)(1+\beta)
\end{align*} from which it follows that \begin{align*}
    c_4(\Sym^3\cS^*) &= (1+3\alpha)(1 + 2\alpha + \beta)(1 + \alpha + 2\beta)(1 + 3\beta) \\
    &= 9\alpha\beta(2(\alpha + \beta)^2 + \alpha\beta) \\
    &= 9\sigma_{1,1}(2\sigma_1^2 + \sigma_{1,1}) \\
    &= 27 \sigma_{2,2}^2 
\end{align*} and $\sigma_{2,2}$ is the class of a point.
This computation really makes clear why one would care about Schubert calculus 
and the cohomology of Grassmannians. It wasn't always so clear to me before.

\begin{remark}
    What did we just compute above? We showed that on a cubic surface $X$, 
    there are 27 lines provided one counts them with multiplicity.
\end{remark}
The next step is to show that $F_1(X)$ is dimension $0$ and reduced. Then 
we can say that the 27 lines are distinct. This is a nontrivial fact and it miraculously
follows from a bunch of hard work to understand the geometry of the Fano scheme.

\section{The geometry of the Fano scheme}
Enumerative questions become geometric questions about the parameter space. The
following computation will show that $F_1(X)$ is dimension $0$ and reduced.

\begin{proposition}
    Let $X\subset \P^n$ smooth and let $L$ be a $k$-plane on $X$. Then 
    $T_{[L]}F_k(X) = H^0(N_{L/X})$.
\end{proposition} In particular $\dim F_k(X) \leq h^0(N_{L/X})$ with 
equality if $F_k(X)$ is smooth at $[L]$. We will prove this in the next section but
first we will finish the example of the cubic surface.

\hfill

Recall the adjunction formula (written as line bundles) which says that for $Y\subset X$ smooth 
divisor of a smooth variety \begin{align*}
    \omega_Y = \omega_X \otimes \cL = \omega_X \otimes \cO_X(Y)\vert_Y
\end{align*} where $\cL$ is the line bundle associated to $Y$.

\hfill

Recall Riemann Roch: For a line bundle $\cL$ on a curve $X$ of genus $g$ we have 
\begin{align*}
    h^0(\cL) - h^0(K_X \otimes \cL^{-1}) = \deg \cL + 1 - g
\end{align*} This is something which I have blacked boxed but 
it should quickly follow from Serre duality and the adjunction formula.

\hfill

Letting $\cL = \cO_X$ gives that $h^0(X,K) = g$. Letting $\cL = K$ gives $\deg K = 2g-2$.
Putting everything together and considering the case of $C\subset S$ smooth genus $g$ 
curve on a smooth surface, we get that \begin{align*}
    g = \frac{C^2 + K_S\cdot C}{2} + 1
\end{align*}

\hfill

Further let $C = L$ line on degree $d$ surface $S$. 
Use adjunction for $S\subset \P^3$ to compute \begin{align*}
    K_S = K_{\P^3} + S\vert_S = -4\zeta + d\zeta
\end{align*} We finally get \begin{align*}
    0 = \frac{L^2 + d - 4}{2} +1
\end{align*} from which it follows that $L^2 = 2 - d$. When $d = 3$ we get that $L^2 = -1$.

Finally, recall that \red{the following things are true}: Let $C\subset X$ 
be a smooth curve on a smooth surface.
\begin{itemize}
    \item The self intersection $C^2$ is the degree of the normal bundle $N_{C/X}$. 
    This is a matter of definition but it feels like still there should be something
    to be explained here.
    \item More generally if $Y\subset X$ is a Cartier divisor on a variety of 
    dimension $k$ the class $[Y^2] = c_1(N_{Y/X})$ in $A_{k-2}(Y)$.
\end{itemize}
If you believe this then returning to our previous discussion
we have that $N_{L/X}$ is a line bundle of negative degree so $h^0(N_{L/X}) = 0$.
Therefore $F_1(X)$ is dimension $0$ and reduced.

\section{Proof of the proposition}
We will identify both sides with deformations.

\begin{definition}
    A deformation of a closed subscheme $Y\subset X$ over $T$ with a 
    distinguished point $\Spec k \in T$ is a flat family $\cY\subset X\times T$
    whose fiber over $\Spec k$ is $Y$. We say $\cY$ is first order if $T = T_m = 
    \Spec k[\epsilon_1,\ldots,\epsilon_m]/(\epsilon_1^2,\ldots,\epsilon_m^2)$
    for some $m$. 
\end{definition}
 
By the universal property of the Hilbert scheme we can identify maps $T_m\to H_X$ sending $0$ to 
$[Y]$ with first order deformations of $Y$ in $X$ over $T_m$. Taking $m=1$ for any scheme $Z$,
we can identify $\Hom_{z}(T_1,Z)$ with the Zariski tangent space $T_{[z]}Z$. Affinely we are considering
$\Hom(k[X], k[\epsilon]/(\epsilon^2))$ where $k[X]$ is the coordinate ring. Such maps look like 
$h(f) = h_0(f) + \epsilon h_1(f)$ where $h_0$ is the map $f\mapsto f(z)$ and $h_1$ is the map which 
corresponds to a derivation. In conclusion, there are corrrespondences between 
flat families of $X$ over $T_m$ with fiber $Y$ and $\Hom_[Y](T_m,X)$ and when $m=1$
this gives isomorphims between $T_{[Y]}H_X$ and $H^0(N_{Y/X})$.

\begin{remark}
    The word "flat" is about geometry, at least when the base is reduced. It tells us that all fibers 
    have the same discrete invariants in particular Hilbert polynomial. It means that
    all the fibers look roughly the same. A nonexample is the blowdown map $\tilde\P^2\to \P^2$.
\end{remark}


On the level of modules, the normal bundle is about 
$\Hom_{R/I}(I/I^2,R/I)$ where $I$ is the ideal of $Y$ in $X$. Note that maps $I\to R/I$ are
the same as maps $I/I^2\to R/I$ because $I^2$ is in the kernel. 
This globalizes to 
$\cN_{Y/X} = \Hom_{\cO_Y}(\cI_Y/\cI_Y^2,\cO_Y)$. Mike did this very nice example where
we took $I = \ideal{x^3 + y^3 + z^3 + w^3}$ and $J = \ideal{x+y,z+w}$, thinking about lines on a
cubic surface, and we showed that there are no first order deformations of $J$ in $I$,
by looking at $\Hom_{I/J}(J/J^2,I/J)$. In particular the tangent space is the normal bundle 
is zero dimensional. \red{I might have the details wrong here but the computation
in Macaulay2 was very nice and explicit.}
\end{document}

