\documentclass[12pt]{article}
\usepackage[english]{babel}
\usepackage[utf8x]{inputenc}
\usepackage[T1]{fontenc}
\usepackage{listings}
\usepackage{tikz}
\usepackage{/Users/songye03/Desktop/math_tex/style/quiver}
\usepackage{/Users/songye03/Desktop/math_tex/style/scribe}



\begin{document}
Songyu Ye

\today
\section{Canonical generators for equivariant cohomology}
I'm trying to understand the following idea. If I have a $T$-invariant subvariety $Y\subset X$ then
this defines a class $[Y]$ in the $T$-equivariant cohomology $H_T^*(X)$. The idea is that we can then
view them inside the ring $H_T^*(X^T)$ as lists of polynomials indexed by the $T$-fixed points.

\hfill

The ring $H_T^*(X^T)$ has canonical generators also indexed by the $T$-fixed points given by
the Bialynicki-Birula decomposition. The idea is that every fixed point $p$ gives rise to an attracting set
$X_p$ which is a $T$-invariant subvariety, and then we can pullback the class $[X_p]$ to $X^T$ and this gives
us a canonical generator.

\hfill

This decomposition also gives us a partial order on the fixed points,
where we say that $v\leq w$ if $X_v\subset \overline{X_w}$. For the flag variety in $\GL(n)$ with the torus action of
conjugation, the fixed points are indexed by the Weyl group $S^n$ and the partial order is given by the Bruhat order.

\hfill

There is this related idea of looking at the $B$-orbits on $X$. It seems that there is a
relationship between the $B$-orbits and the attracting sets of the $T$-fixed points. They are both defining
elements in the $T$ equivariant cohomology rings of $X$ and there are results saying that they both form a basis.

\red{Should they be the same? In fact why is one even studying the $B$-orbits? We are
	interested in the $T$-equivariant cohomology ring, so why are we looking at the $B$-orbits.
	Because $B$ deformation retracts onto $T$?}

\hfill

Tymoczko describes an algorithm for realizing canonical generators for $H_T^*(X)$
in the ring $H_T^*(X^T)$. The algorithm is as follows. Recall the partial order observed above.

\hfill

Fix $v\in X^T$. I will describe the canonical generator $S_v$ associated to $v$. For all $u<v$ the generator restricts
to $0$. Now consider $w\in X^T$ so that every $u<w$ is known. Then $S_v\vert_w$ is the minimal degree guy that
satisfies all the relations, in particular it should be zero whenever possible (except for when $w = v$ in which
case we insist that it is not zero).

\hfill

The claim is that the output of this algorithm is the pullback of the class $[X_v]$ to $X^T$. This output is
unique up to a scalar whenever the moment graph can be drawn in the plane so that hte vertex $v$ lies above its neighbor $u$ if and only if $v$ has more
downward edges than $u$.

\hfill

One question is what does it even mean to restrict a class of the cell decomposition
to a fixed point? If $X_v$ is a cell (one of the basis elements for the $T$-equivariant cohomology ring) then
when I restrict it to $w$ I should get a polynomial in the weights of $T$. What data does this even mean?

\section{Example}
Consider the $T\times T$ action on $\overline{\PGL(2)} \cong \P^3$. There are two $G\times G$ orbits,
the open one and the closed one corresponding to taking no simple roots or taking all of the simple roots.

\hfill

Now we can ask about the $B\times B^-$ orbits in each. In the open orbit $\cong \PGL(2)$ this is precisely
the Bruhat decomposition. \begin{align*}
	\PGL(2) = B\begin{bmatrix}
		           * & 0 \\
		           0 & *
	           \end{bmatrix}B^- \cup B\begin{bmatrix}
		                                  0 & * \\
		                                  * & 0
	                                  \end{bmatrix}B^-
\end{align*}
In the closed orbit $\cong \P^1\times \P^1$, recall that every projective $2\times 2$ matrix of rank $1$ is just a row
vector times a column vector (projectively) \begin{align*}
	M = r\cdot c
\end{align*} and so thinking about the $B\times B^-$ orbits \begin{align*}
	BMB^- = (Br)(cB^-)
\end{align*} and so there are four $B\times B^-$-orbits, the product of the situations when \begin{align*}
	r = \begin{bmatrix}
		    0 & 1
	    \end{bmatrix}, \begin{bmatrix}
		                   1 & *
	                   \end{bmatrix} \quad \text{ and } \quad c = \begin{bmatrix}
		                                                              0 \\ 1
	                                                              \end{bmatrix}, \begin{bmatrix}
		                                                                             1 \\ *
	                                                                             \end{bmatrix}
\end{align*}
So there are four orbits which I will denote \begin{align*}
	{\infty\times\infty},{\infty\times \A^1}, {\A^1\times \infty}, {\A^1\times \A^1}
\end{align*} Previously I said that there were maps \begin{align*}
	H_T^*(X)\to H_T^*(\P^1\times \P^1)\to H_T^*(X^{T\times T})
\end{align*} and we know that the second map is an injection. The first map
is supposed to correspond to take the $B\times B^-$ orbits in the open orbit, take their closures,
and see which $B\times B^-$ orbits in the closed orbit appear. If you are a $B\times B^-$ orbit in the
closed orbit, then the first map is the identity on the class of your closure. But then the first map is not
injective as it is supposed to be. \red{I've gone wrong somewhere.} There are more $B\times B^-$ orbits in $X$ than
there are $T\times T$ fixed points. Is it just the case that the $B\times B^-$ orbits do not form a basis,
but perhaps they are only generators.

\hfill

Let's continue with the calculation by writing down generators in the GKM ring. What are the attracting sets?

\hfill

We can just use the algorithm: The GKM graph with labels looks like \begin{center}
	\begin{tikzpicture}
		\begin{tikzpicture}[scale=2]
			\draw (0,0) rectangle (4,4); % Draw the square

			% Label the edges
			\node at (2,4) [above] {$t_2$};
			\node at (2,0) [below] {$t_2$};
			\node at (0,2) [left] {$t_1$};
			\node at (4,2) [right] {$t_1$};

			\draw (0,0) -- (4,4) node[midway, above, pos=0.7] {$t_1 - t_2$}; % Draw the first diagonal and label it
			\draw (0,4) -- (4,0) node[midway, below, pos=0.7] {$t_1 + t_2$}; % Draw the second diagonal and label it

			\node at (0,0) [below left] {$\begin{bmatrix}
						0 & 0 \\
						* & 0 \\
					\end{bmatrix}$};
			\node at (4,4) [above right] {$\begin{bmatrix}
						0 & * \\
						0 & 0 \\
					\end{bmatrix}$};
			\node at (0,4) [above left] {$\begin{bmatrix}
						* & 0 \\
						0 & 0 \\
					\end{bmatrix}$};
			\node at (4,0) [below right] {$\begin{bmatrix}
						0 & 0 \\
						0 & * \\
					\end{bmatrix}$};
		\end{tikzpicture}
	\end{tikzpicture}
\end{center}
The partial ordering on these guys looks like \begin{center}
	\begin{tikzpicture}[rotate=-45]
		\draw (0,0) rectangle (2,2);
		\node at (0,0) [below left] {$\begin{bmatrix}
					0 & 0 \\
					* & 0 \\
				\end{bmatrix}$};
		\node at (2,0) [below right] {$\begin{bmatrix}
					0 & 0 \\
					0 & * \\
				\end{bmatrix}$};
		\node at (2,2) [above right] {$\begin{bmatrix}
					0 & * \\
					0 & 0 \\
				\end{bmatrix}$};
		\node at (0,2) [above left] {$\begin{bmatrix}
					* & 0 \\
					0 & 0 \\
				\end{bmatrix}$};
	\end{tikzpicture}
\end{center}
The algorithm says that the canonical generators are
\begin{align*}
	X_{\begin{bmatrix}
		   * & 0 \\
		   0 & 0 \\
	   \end{bmatrix}} & = \begin{tikzpicture}[scale = 1/2]
		                      \draw (0,0) rectangle (2,2);
		                      \node at (0,0) [below left] {};
		                      \node at (2,0) [below right] {};
		                      \node at (2,2) [above right] {};
		                      \node at (0,2) [above left] {$t_1t_2(t_1 + t_2)$};
	                      \end{tikzpicture} \\
	X_{\begin{bmatrix}
				   0 & * \\
				   0 & 0 \\
			   \end{bmatrix}} & = \begin{tikzpicture}[scale = 1/2]
		                      \draw (0,0) rectangle (2,2);
		                      \node at (0,0) [below left] {};
		                      \node at (2,0) [below right] {};
		                      \node at (2,2) [above right] {$t_1(t_1 - t_2)$};
		                      \node at (0,2) [above left] {$t_1(t_1 + t_2)$};
	                      \end{tikzpicture}   \\
	X_{\begin{bmatrix}
				   0 & 0 \\
				   * & 0
			   \end{bmatrix}} & = \begin{tikzpicture}[scale = 1/2]
		                      \draw (0,0) rectangle (2,2);
		                      \node at (0,0) [below left] {$t_2$};
		                      \node at (2,0) [below right] {$0$};
		                      \node at (2,2) [above right] {$t_1$};
		                      \node at (0,2) [above left] {$t_1 + t_2$};
	                      \end{tikzpicture}   \\
	X_{\begin{bmatrix}
				   0 & 0 \\
				   0 & * \\
			   \end{bmatrix}
	}                  & = \begin{tikzpicture}[scale = 1/2]
		                       \draw (0,0) rectangle (2,2);
		                       \node at (0,0) [below left] {$1$};
		                       \node at (2,0) [below right] {$1$};
		                       \node at (2,2) [above right] {$1$};
		                       \node at (0,2) [above left] {$1$};
	                       \end{tikzpicture}                 \\
\end{align*}
It felt like I had to make ad hoc choices along the way. For example it feels like rows 2 and 3 are symmetric in Bruhat order 
so they should either both be degree 2 or both be degree 1. However we know the Betti numbers of $\C\P^3$.
\red{waht should the betti numbers of other wonderful compactifications be? Schubert calculus seems
to be really stupid in the case that all of the Betti numbers are 1}
\hfill

To continue this calculation we can think about what is happening with the $B\times B^-$ orbits. 

\section{$B$ orbits and $T$ attracting sets}
Indeed there is a corresponense that I was concerned about. First an example.

\begin{example}[Allen]
    \begin{center}
        \includegraphics[scale = .5]{/Users/songye03/Desktop/math_tex/img/Screenshot 2024-01-23 at 2.13.19 AM.png}
    \end{center}
    The point is that in this case the $B$ orbits on $\GL(n)/B$ are precisely in correspondense with the fixed points
    of the torus action on the flag variety, and moreover the $B$ orbits are the attracting sets of the fixed points.
\end{example}
   
In general, we appeal to more facts about the Bialynicki-Birula decomposition.

\hfill

Recall the setup. We have a connected reductive algebraic group $G$ with a maximal torus $T$ and Borel $B$, fixed.
Let $X$ be smooth projective variety with an action of $G$ so $X$ has finitely many $G$-orbits.

\hfill

For a one parameter subgroup $\lambda: \C^*\to T$ we can consider the attracting set for $y\in X^T$ \begin{align*}
    X^\lambda(y) = \{x\in X: \lim_{t\to 0} \lambda(t)\cdot x = y\}
\end{align*} 

\hfill

If we choose $\lambda$ in the interior of the Weyl chamber, then $X_\lambda(y)$ is $B$
-stable. For sufficiently general $\lambda$, then $X^{\C^*} = X^T$. This is because we can think of $X$
as sitting in some projective space $\P(V)$ and then $T$ acts on $V$ via characters \begin{align*}
    V = \bigoplus_{\chi\in X^*(T)} V_\chi
\end{align*}. Now let $\lambda$ sufficiently general so that $\langle \lambda, \chi\rangle \neq 0$ for all $\chi$
which appear. Therefore $\lambda$ and $T$ induce the same eigendecomposition of $V$ and so a point $x\in \P(V)$
is fixed by one if and only if it is fixed by the other.

We say that a $G$-variety is spherical if it is normal and some Borel $B$ has an open dense orbit.
We say that a spherical $G$-variety with open $G$-orbit $X^0_G$ is toroidal if the closure of every 
$B$-stable divisor in $X^0_G$ contains no $G$-orbit.

\hfill

Complete symmetric spaces and toric varieties are toroidal.
\begin{theorem}
    Let $X$ be a toroidal complete $G$-variety. Then the intersection of any cell $X^\lambda(y)$ with any
    $G$-orbit is empty or a single $B$-orbit.
\end{theorem}

\section{Our story}
We have the torus $T\times T$ acting on the wonderful compactification $\overline \PGL(2)$. The fixed points are 
indexed by elements of $S_2\times S_2$ and they tell us how to decompose the wonderful compactification
into cells. The attracting sets are the $B\times B^-$ orbits. However there are attacting sets
which do not contain $T\times T$ fixed points. \red{So it is not true that the map }
\begin{align*}
    H_T^*(\overline \PGL(2))\to H_T^*(\P^1\times \P^1)
\end{align*} \red{is injective.}

\section{Loop group of $\SU(2)$}
Is there a hope of doing Schubert calculus in the loop group of $\SU(2)$? 
What indexes the fixed points? What are the attracting sets? Is there a corresponding Bialynicki-Birula decomposition?

\hfill

There is a corresponding notion of Bruhat decomposition and Schubert varieties in Kac Moody groups.
People have also studied the closure patterns of these decompositions. Have people thought
about the equivariant cohomology of these things? One can start with the paper of Billig and Dyer.

\section{Affine flag varieties and their Schubert calculus}
A good place to start thinking about these questions is the paper of Braden and MacPherson. We should read this 
and be prepared to talk with Tara on Thursday about it.

\section{References}
\begin{itemize}
    \item https://mathoverflow.net/questions/358450/bialynicki-birula-decompositions-and-fixed-points
    \item https://mathoverflow.net/questions/284894/bialynicki-birula-decomposition-and-moment-polytopes-graphs
\end{itemize}
\end{document}



