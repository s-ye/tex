\documentclass[12pt]{article}
\usepackage[english]{babel}
\usepackage[utf8x]{inputenc}
\usepackage[T1]{fontenc}
\usepackage{listings}
\usepackage{tikz}
\usepackage{/Users/songye03/Desktop/math_tex/style/quiver}
\usepackage{/Users/songye03/Desktop/math_tex/style/scribe}

\begin{document}
Songyu Ye

\today

\hfill

This is what Allen and I talked about on 2/16/2024.

\hfill

There is the work of Kogan and Miller which was later followed up by the work of Miller 
and Sturmfels. In general, we are continuing to think about the story of 
toric generations of wonderful compactifications of semisimple adjoint groups.

\section{Flag variety}

We need to start with previous work on the degeneration
of the flag variety to the toric variety corresponding to the Gelfand-Cetlin polytope.


\subsection{Grobner}
The reference for this section is Miller and Sturmfels 2005.

\hfill

SAGBI stands for subalgebra analogue to Gröbner bases for ideals. Recall 
the associated graded construction which one can apply to a ring with a given filtration.
If we have \begin{align*}
    R = I_0 \supset I_1 \supset \cdots
\end{align*} then we can form the associated graded ring \begin{align*}
    \gr R = \bigoplus_{i \geq 0} I_i/I_{i+1}.
\end{align*} and if $I$ is a homogeneous ideal in a graded 
polynomial ring, then $\gr I\subset \gr R$
is an ideal. The point is that in general, when one applies the construction 
to a prime ideal in a polynomial ring, $\gr I$ is generally prime. 
\subsection{SAGBI}
One way to think about fixing the story is to consider instead subalgebras
of our polynomial ring. Now we are thinking about when $\gr I$ is Noetherian.

\hfill

We don't have many examples of when this happens, but we do have an important one
which is in the story of the flag variety.

\subsection{Flag variety}
We have the following diagram \begin{center}
    \begin{tikzcd}
        \GL_n \arrow[d,"/B"] & \rightarrow & M_n \arrow[d,"/B"]\\
        \GL_n/B & \GL_n/N \arrow[l, "/T"] \arrow[r, "i"] & M_n//N := \Spec \C[x_{ij}]^{1\times N}
    \end{tikzcd}
\end{center}
The first fundamental theorem of invariant theory tells us what 
$\C[x_{ij}]^{1\times N}$ is. In particular we are consider rightward column operations except we are not 
allowed to do the operation of adding a multiple of one column to another or scale columns.

Thus, for $\lambda \in \binom{[n]}{k}$, we have the invariant \begin{align*}
    p_\lambda = \det(\text{first $k$ columns and $\lambda$ rows})
\end{align*}
The first fundamental theorem tells us that these $p_\lambda$ in fact generate all of the
invariants. There are quadratic relations which are called the Plucker relations.

Then we have a degeneration which takes $p_\lambda$ to its diagonal term, and this turns out
to be a toric degeneration. The relations in the degernation can be written very nicely \begin{align*}
    p_\lambda p_\mu - p_{\lambda \cup \mu}p_{\lambda \cap \mu} = 0
\end{align*}

Allen remarks that the key ingredient in this story 
is the $T$-bundle $G/B \to G/N$. This is in fact a product of circle bundles, 
from the generators of the nef cone. Now we are ready to think about the
wonderful compactification.
\section{Wonderful compactification}
The anagalous thing we should think about is the bundle \begin{align*}
    \bigoplus \text{line bundles corresponding to edges of the nef cone} \to \bar G
\end{align*}
\begin{center}
    \includegraphics[scale = .5]{/Users/songye03/Desktop/math_tex/img/Screenshot 2024-02-16 at 5.53.52 PM.png}
\end{center}

\section{References}
\begin{itemize}
    \item THE TOTAL COORDINATE RING OF A WONDERFUL VARIETY
    MICHEL BRION
    \item RIGIDITY OF WONDERFUL GROUP COMPACTIFICATIONS UNDER
    FANO DEFORMATION
    BAOHUA FU AND QIFENG LI
\end{itemize}
\end{document}