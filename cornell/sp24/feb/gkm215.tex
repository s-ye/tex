\documentclass[12pt]{article}
\usepackage[english]{babel}
\usepackage[utf8x]{inputenc}
\usepackage[T1]{fontenc}
\usepackage{listings}
\usepackage{tikz}
\makeatletter
\def\input@path{{../../style/}}
\makeatother

\usepackage{../../style/quiver}
\makeatletter
\def\input@path{{../../style/}}
\makeatother

\usepackage{../../style/scribe}

\begin{document}
Songyu Ye

\today

\hfill

This is my understanding from looking at the literature and thinking about what 
Allen said last time. Perhaps it is also worth mentioning what Tara told me about.

\hfill

\section{Direction}
Recall that we are interested in a toric degeneration of the wonderful compactification of a 
semisimple adjoint group $G$.

\hfill

To think about this, it would be best to consider previous work and see how it can be applied to our case.
\section{Previous Work}
Starting with the paper of Kogan and Miller in 2005:

Recent developments have come from degenerations of certain varieties (such as flag varieties and Schubert varieties)
or closely realted affine varieties (such as matrix Schubert varieties) to toric varieties.

\hfill

The approach is about finding an appropriate family $\mc{F}$ so that $\mc{F}$
degenerates our variety of interest and to combinatorially identify all components occuring in the degenerate limit.

\hfill

The example Allen was talking about is the degeneration of the flag variety to the toric variety corresponding
to the Gelfand-Cetlin polytope.

\hfill

This was first carried out by Gonciulea and Lakshmibai. They constructed a flat family whose general fiber is 
$R = \bigoplus_a H^0(G.Q, L^a)$, where $Q = \bigcap P_{k_i}$ is a parabolic and $L^a = L_{k_1}^{a_1} \otimes \cdots \otimes L_{k_n}^{a_n}$
and $L_{k_i}$ are the ample generators of $\Pic(G/P_{k_i})$. The special fiber is $R_{\mf L}$ the algebra associated 
to a finite distributive lattice $\mf L$, and hence necessarily toric.

\hfill

Previous work by Sturmfels using SAGBI theory proved the degeneration of the 
Grassmannian to a toric variety. Conca-Herzog-Valla proved the degeneration of 
certain normal scrolls to toric varieties. We also have it for the Bott Samelson scheme of $G/B$.

\subsection{The work of Kogan and Miller}


\section{References}
\begin{itemize}
    \item TORIC DEGENERATION OF SCHUBERT VARIETIES AND GELFAND-CETLIN POLYTOPES Kogan Miller
    \item DEGENERATIONS OF FLAG AND SCHUBERT VARIETIES TO TORIC VARIETIES Gonciulea Lakshmibai
    \item Knutson Miller 03
    \item B. Sturmfels, Gri~bner bases and convex polytopes, Lecture notes
    from the New Mexico State University, preprint (1995). 
    \item M. Grossberg and Y. Karshon, Bott towers, complete integrability,
    and extended character of representations, Duke Math. Journal 76
    (1994), 23-59. 
\end{itemize}
\end{document}