\documentclass[12pt]{article}
\usepackage[english]{babel}
\usepackage[utf8x]{inputenc}
\usepackage[T1]{fontenc}
\usepackage{listings}
\usepackage{tikz}
\makeatletter
\def\input@path{{../../style/}}
\makeatother

\usepackage{../../style/quiver}
\makeatletter
\def\input@path{{../../style/}}
\makeatother

\usepackage{../../style/scribe}

\begin{document}
Songyu Ye 
\today

\hfill

This is what Allen and I talked about on Feb 9, 2024.
I told him that I was having a hard time being productive this week and 
he toned it down a notch. We talked about a bunch of examples today.

The place I should look for most of this stuff is in his paper Frobenius Splittings etc.

\section{Stratifications by anticanonical divisors}
\begin{example}
    $\P^n$
\end{example}
\begin{example}
    Toric varieties
\end{example}
\begin{example}
    Flag varieties
\end{example}

\begin{example}
    Wonderful compactification of a semisimple adjoint group.
\end{example}

The point is that the faces of the Gelfand-Cetlin polytope are indexed by pipe dreams. 
There is a way of from a pipe dream reading off a permutation and this is supposed to tell us 
something about where pieces of the stratification go to in the degeneration.

\hfill

Remember the whole point is that we want to expand strata of the degeneration in terms of
geometric basis for the cohomology of the wonderful compactification.

\section{Goals}
\begin{itemize}
    \item Intersection theory with Mike Stillman
    \item Toric varieties
    \item Tara 
    \item Allen Frobenius Splitting paper
    \item Ritvik example, read the Riemann Hurwitz theorem and Riemann Roch theorem and 
    write down maps of elliptic curves into projective space
\end{itemize}

\end{document}