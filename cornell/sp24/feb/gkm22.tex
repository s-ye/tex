\documentclass[12pt]{article}
\usepackage[english]{babel}
\usepackage[utf8x]{inputenc}
\usepackage[T1]{fontenc}
\usepackage{listings}
\usepackage{tikz}
\usepackage{/Users/songye03/Desktop/math_tex/style/quiver}
\usepackage{/Users/songye03/Desktop/math_tex/style/scribe}



\begin{document}
Songyu Ye

\today

This is what Allen and I talked about on Feb 2.

\section{Flag variety}
In the flag variety we have a distinguished basis for cohomology, the Schubert basis. They form a stratification but
a better one is the Richardson stratification which is given by \begin{align*}
    X_w^v =  X_w \cap W^v
\end{align*} the intersection of the Schubert variety $B_-wB/B$ with the opposite Schubert variety $BwB/B$. 
One can think about the boundary components of the stratification, for Schubert varieties this is precisely
\begin{align*}
    \partial X_w = \bigcup X_s \st X_v \subsetneq X_w = X_w - X_w^o
\end{align*} and one can make the analagous definition for the Richardson stratification.

\hfill

The good thing about the Richardson stratification is that it is the boundary divisors $\partial X_w^v$ are \red{anticanonical} in $X_w^v$.
There was remark made about the top exterior power of the contangent bundle on a smooth projective variety.

\hfill 

There was another remark about the adjunction theorem which tells us that if "$\partial X = D \cup E$" where $D$ is some irreducible component
and $E$ is the rest, then $D\cap E$ is anticanonical in $E$.

\hfill

\begin{example}
    Consider full flags in $\C^3$. There is a toric degeneration of $\Fl(3)$ to a toric variety and there are pieces and
    if you follow the pieces through the degeneration, then you figure out that some of the pieces break up and 
    are become reducible.

    \red{See the picture below. I honestly can't tell you what is going on}

    \begin{center}
        \includegraphics[scale = .2]{/Users/songye03/Desktop/math_tex/img/DBD1E5FE-1CDD-43D2-996D-4237AF35E42E_1_102_o.jpeg}
    \end{center}
\end{example}
The Richardson varieties no longer form a basis for the cohomology in particular there are too many of them. 
It is the object of Schubert calculus to expand them. This is hard so we will take it as a black box, and
we will try to obtain results in terms of this result.


\section{For the wonderful compactification}
One thing that Allen expressed interest in is a toric degeneration for $\bar G$.

\hfill

Parallel to the Schubert basis is Brion's basis which is certain $B\times B_-$ orbit closures indexed
by $W\times W$. The analogy of the Richardson stratification is the anticanonical stratification,
which is given by $B\times B_-$ orbit closures intersected with $B_-\times B$ orbit closures.

\hfill

The matter of Schubert calculus for $\bar G$ concerns the question of expanding the anticanonical basis
in terms of the Brion basis. This is a hard question because it is necessarily a superset of 
Schubert calculus for $G/B$. The hope is to reduce the problem to the Schubert calculus for $G/B$.

\hfill

But in between these two stratifications is a third guy which is just all of the $B\times B_-$ orbit closures.
These are indexed by $P,W,W^P$ by Springer and then later by Chen-Dyer, and they live as a 
subset of the Weyl group of something called \red{Nakajima Dynkin diagrams, see the photo above} The elements correspond
to choices of
 \begin{align*}
    W\times\text{ some funny stuff which commutes with everything except its neighbors} \times W
 \end{align*}


\hfill

The anticanonical stratification is indexed by elements of a particular subset of the Weyl group of two Nakajima Dynkin
diagrams glued together along the teeth.

\hfill

There's the theorem of He-K-Lu (unpublished) which tells us how to take a $B\times B_-$ orbit closure and localize it
to a fixed point corresponding to $W\times W$. It is a formula which invokes the AJS/Billey formula, something
which we know how to compute. Allen will send me the paper at some point.

\hfill

\begin{example}
    We have the following anticanonical stratification for $\bar G$.
    \begin{center}
        \includegraphics[scale = .5]{/Users/songye03/Desktop/math_tex/img/Screenshot 2024-02-02 at 2.04.23 PM.png}
    \end{center}
    The stratification in this case is really stupid because all of the strata are cut out by linear equations, except for the
    $\P^1\times \P^1$.
\end{example}
\red{Ask Allen about the Gelfant Setlin degeneration}

\section{References}
\begin{itemize}
    \item He-Lu indexed the anticanonical stratification
    \item Springer and then Chen-Dyer indexed the $B\times B_-$ orbit closures
    \item Brion has the basis for the cohomology
\end{itemize}

\end{document}



