\documentclass[12pt]{article}
\usepackage[english]{babel}
\usepackage[utf8x]{inputenc}
\usepackage[T1]{fontenc}
\usepackage{listings}
\usepackage{tikz}
\usepackage{/Users/songye03/Desktop/math_tex/style/quiver}
\usepackage{/Users/songye03/Desktop/math_tex/style/scribe}

\begin{document}
Songyu Ye

\today

\hfill

This is a note regarding the talk I gave at the flag variety reading group on Feb 16, 2024.
Most of this talk comes from 2.3 and 3.1 of Cox, Little, Schenck's book. There is a little stuff I borrowed
from Proudfoot toric varieties.

\section{Last time}
Fix $V$ vector space and $V_\Z$ lattice.

Last time we considered the following correspondence:

\hfill

The correspondence between rational convex polytopes which are smooth and simple: \begin{itemize}
	\item simple meaning the number of edges at each vertex $= \dim V$
	\item smooth meaning the outward pointing edge vectors form a $\Z$-basis for $V_\Z$ at each point
\end{itemize} and smooth toric varieties with a particular embedding into projective space.

\hfill

I want to emphasize that the polytope specifically specifies the data of the embedding into projective space.

\section{Today}
Let's demonstrate this with an example.

\begin{example}
	Consider the ratinoal normal scroll $S_{a,b}$ where we will take $a,b = 2,4$.
	The lattice polytope and fan look like the following (image was lost but comes from CLS).
	This defines a map $\C^* \times \C^* \to \P^{4+2+1}$ by the following \begin{align*}
		(s,t)\mapsto [1,s,s^2,t,st,s^2t,s^3t,s^4t]
	\end{align*} and the rational normal scroll in question is the closure of the image of this map. We can write this down
	by looking at the map $\C^*\times \P^1 \to \P^7$ \begin{align*}
		(s,\mu,\lambda)\mapsto [\lambda,s\lambda, s^2\lambda, \mu, s\mu, s^2\mu, s^3\mu, s^4\mu]
	\end{align*} When $\lambda = 0$ we get the rational normal curve $[0,0,0,1,s,s^2,s^3,s^4]$, and when $\mu = 0$
	we get the rational normal curve $[1,s,s^2,0,0,0,0,0]$. In between is some rational homotopy.

	\hfill

	Note that the polytope tells us how to embed the variety into projective space, but we do not get this information
	from the fan. In particular there are many polytopes which give the same fan, corresponding to different choices of
	$a,b$ so that $b-a = 2$ in this case.

	\hfill

	Note that this toric variety is in fact not smooth at the top right vertex of the polytope.
\end{example}

\begin{example}
	Consider the polytope and fan of the weighted projective space $\P(1,1,2)$ (image was lost but comes from CLS).
	It is not smooth by looking at the top vertex of the triangle. The polytope describes the map \begin{align*}
		\C^*\times\C^* \mapsto \P^3 \\
		(s,t) \mapsto [1,s,s^2,t]
	\end{align*}
	Take closure. $\P(1,1,2) = V(x_0x_2 - x_1^2) \subset \P^3$ is not smooth at $[0,0,0,1]$.
\end{example}
\section{Gluing data}
First I will describe how one generally assembles abstract varieties from affine ones,
and then I will describe how fans contain this data.

\begin{example}
	Put together $\P^n$ from $n+1$ copies of $\A^n$ via gluing. Morally these guys represent the $n+1$
	charts of $\P^n$ obtained by putting $x_i = 1$ for each $i$.

	\hfill

	I have $n+1$ copies of affine space \begin{align*}
		U_i = \Spec \C[x_0/x_i,\dots,\hat{x_i/x_i}, \dots,x_n/x_i]
	\end{align*} and for each $i,j$ I have to identify their intersections \begin{align*}
		\Spec \C[x_0/x_i,\dots,\hat{x_i/x_i}, \dots,x_n/x_i][(x_j/x_i)^{-1}] & \to \Spec [x_0/x_j,\dots,\hat{x_j/x_j}, \dots,x_n/x_j][(x_i/x_j)^{-1}] \\
		(x_j/x_i)^{-1}                                                       & \mapsto (x_i/x_j)^{-1}                                                 \\
		x_k/x_i                                                              & \mapsto (x_k/x_j)/(x_i/x_j)
	\end{align*}
\end{example}
This example motivates the following definition of how to glue an abstract variety from affine ones.

\hfill

The data of an abstract variety is \begin{itemize}
	\item a collection of affine varieties $V_\alpha$
	\item open sets $V_{\beta\alpha} \subset V_\alpha$
	\item isomorphisms $g_{\beta\alpha}: V_{\beta\alpha} \to V_{\alpha\beta}$ for each $\alpha,\beta$
\end{itemize} and the maps satisfy the following conditions \begin{itemize}
	\item $g_{\beta\alpha} = g_{\alpha\beta}^{-1}$
	\item $g_{\beta\alpha}(V_{\beta\alpha}\cap V_{\gamma\alpha}) = V_{\alpha\beta}\cap V_{\gamma\beta}$
	\item $g_{\gamma\alpha} = g_{\gamma\beta}\circ g_{\beta\alpha}$ on $V_{\beta\alpha}\cap V_{\gamma\alpha}$
\end{itemize} Then the abstract variety is the quotient of the disjoint union of the $V_\alpha$ by the equivalence relation \begin{align*}
	X             & = \coprod V_\alpha/\sim                              \\
	x\in V_\alpha & \sim y\in V_\beta \text{ if } x = g_{\beta\alpha}(y)
\end{align*} and $X$ is given the quotient topology.

\begin{remark}
	Note that $X$ has an affine open cover $U_\alpha = \set{[a]\in X \st a\in V_\alpha} \cong V_\alpha$.
\end{remark}
\begin{example}
	[Blowup of $\A^2$ at the origin] The data is \begin{align*}
		V_0 = \Spec \C[u,v]            \\
		V_1 = \Spec \C[w,z]            \\
		V_{01} = \Spec \C[w,z,\inv{z}] \\
		V_{10} = \Spec \C[u,v,\inv{v}] \\
		g_{01}:V_{10} \to V_{01} \text{ is given by } u\mapsto wz, v\mapsto \inv{z}
	\end{align*}
	Then we can identify $X \cong V(x_0Y - x_1X)\subset \P^1\times \A^2$ with coordinates $[x_0,x_1]$ and $(X,Y)$. One can see this
	by taking two affine charts of $X$ via $x_0 = 1$ and $x_1 = 1$ and then checking that hte induced gluing map is the same as the one
	we have written down.
\end{example}

\begin{example}
	The lattice hexagon (image was lost but comes from CLS) defines for us a smooth toric variety. Which one is it?

	\hfill

	Using our understanding of \red{corner chopping corresponding to blowups. Why? } we can see that this variety is the
	blowup of $\P^1 \times \P^1$ at two different points.  In particular if we blow up $\P^1\times \P^1$ with coordinates $[x_0;x_1],[y_0;y_1]$ at $(0,0)$ and $(\infty,\infty)$
	then we get the toric variety corresponding to the hexagon. One way of writing this down is looking at the four affine charts corresponding to
	setting each of the pairs of coordinates to be $1$. On the affine chart $x_0 = y_0 = 1$ we have the blowup of $\A^2$ at the origin
	we replace $\C[x_1,y_1] \to \C[x_1,y_1,X_1,Y_1]/\ideal{x_1Y_1 - y_1X_1}$. On the affine chart $x_1 = y_1 = 1$ we do the same thing
	and then we glue them together.

	\hfill

	\red{Note that I told you about a way of covering it with four affine charts. The fan is telling me six.}

	\hfill

	\red{Also one can embed this into $\P^3$ as the set of determinant zero $2\times 2$ matrices. Is there a clear polytopal picture?}

	We can also embed this into $\P^6$. Label each of the lattice points of the hexagon (including the interior)
	by $a,b,c,d,e,f,g$ and then start writing down all of the relations they satisfy. In this case they are generated in
	degree $2$ (the problem of claiming this for general smooth polytopes has stood open for about fourty years now) and so
	this embeds into $\P^6$.
\end{example}

\section{Fans}
Fans contain gluing data. I'll state the formal things (which I don't understand too well) and then do an example.

\begin{definition}
	A fan $\Sigma$ is a collection of cones in $N_\R$ such that \begin{itemize}
		\item each $\sigma$ is a strongly convex (i.e. $0$ is a face) rational (recall the lattice)
		      polyhedral cone (looks like $\set{\sum r_iv_i \st r_i\geq 0}$ for some $v_i\in N$)
		\item each face of a cone in $\Sigma$ is also in $\Sigma$
		\item the intersection of any two cones in $\Sigma$ is a face of each

	\end{itemize}
\end{definition}

Recall each cone $\sigma\in \Sigma$ gives us an affine toric variety $U_\sigma = \Spec \C[\sigma^\vee\cap M]$ where we defined \begin{align*}
    \sigma^\vee = \set{m\in M_\R \st \langle m,n\rangle \geq 0 \text{ for all } n\in \sigma}
\end{align*}
Let $\tau\subset \sigma$ face. Then we can write $\tau = \sigma \cap H_m$ where $m\in \sigma^\vee$
and $H_m = \set{u\in N_\R \st \langle u,m\rangle = 0}$ is
the hyperplane defined by $m$. The equality \begin{align*}
    S_\tau = S_\sigma + \Z(-m)
\end{align*} implies $\C[S_\tau] = C[S_\sigma]_{\chi^m}$ where $\chi^m$ is the character of $T$ given by $m$.

\hfill

If $\tau = \sigma_1 \cap \sigma_2$ then $\sigma_1 \cap H_m = \tau = \sigma_2 \cap H_{-m}$ for some 
$m\in \sigma_1^\vee\cap \sigma_2^\vee \cap M$. So we get \red{gluing data} \begin{align*}
    U_{\sigma_1} \supset (U_{\sigma_1})_{\chi^m} = U_\tau = (U_{\sigma_2})_{\chi^{-m}} \subset U_{\sigma_2}
\end{align*}

\hfill

We will demonstrate what's going on with an example.

\begin{example}
    Consider the fan of the simplex (image was lost but it comes from CLS)
    \red{Look at the outward edge vectors}
    We have the affine open charts 
    \begin{align*}
        U_{\sigma_0} = \Spec \C[x,y] \\
        U_{\sigma_1} = \Spec \C[y,\inv{xy}] \\
        U_{\sigma_2} = \Spec \C[\inv{y},x\inv{y}]
    \end{align*} and the gluing data \red{shared edge vector, pointing outward}
    \begin{align*}
        \C[x,y]_{x} \cong \C[x^{-1},\inv{x}y]_{\inv{x}} \\
        \C[x,y]_{y} \cong \C[y^{-1},x\inv{y}]_{\inv{y}}\\
        \C[\inv{x},\inv{x}y]_{\inv{x}y} \cong  \Spec \C[\inv{y},x\inv{y}]_{x\inv{y}}
    \end{align*}
\end{example}

\section{Remark}
I'm beginning to understand the wonderful thing about toric geometry. You get coordinates for everything and things are very hands on.
Isaac once told me to appreciate the things that are very concrete and this semester I feel as though
having concrete things to look at is very important. I am liking this reading course more than I used to.
\end{document}