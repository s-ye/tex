\documentclass[12pt]{article}
\usepackage[english]{babel}
\usepackage[utf8x]{inputenc}
\usepackage[T1]{fontenc}
\usepackage{listings}
\usepackage{tikz}
\makeatletter
\def\input@path{{../../style/}}
\makeatother

\usepackage{../../style/quiver}
\makeatletter
\def\input@path{{../../style/}}
\makeatother

\usepackage{../../style/scribe}

\begin{document}
Songyu Ye

These are notes from my meeting with Allen on March 8.

\section{Introduction}

The third guy is the affine completion of the first guy. We 
relate it to the story in the flag variety.

\begin{center}
    \includegraphics[width=\textwidth]{/Users/songye03/Desktop/math_tex/img/IMG_1845.JPG}
\end{center}
Allen says that he believes the map from the Sym is a surjection. It is not obvious.
Moreover we expect taht as in the 
\begin{center}
    \includegraphics[width=\textwidth]{/Users/songye03/Desktop/math_tex/img/IMG_1841.JPG}
\end{center}

\begin{center}
    \includegraphics[width=\textwidth]{/Users/songye03/Desktop/math_tex/img/IMG_1843.JPG}
\end{center}
The first guy plays the role of our torus bundle $G/N \to G/B$. In the story of the flag
variety, there were two things we could do to the total space of such bundle. We could stick it in 
$G//N$ which is an affine variety or we could stick it in the total space of the line bundles with the zeros left in.

There is an affinization map from the total space of the line bundle to $G//N$ and the map is a surjection,
and then $G//N$ includes into $\bigoplus_{\omega}H^0(L_{\omega})$ over the fundamental weights.

\hfill

In particular the map in the box plays the role of the composition of $G/N \hookrightarrow \bigoplus V_\omega$.
The map on functions, in the case of the flag variety, is a surjection with relations in degree 2.

\hfill

We left off talking about the following result from Brion's paper. 

\begin{align*}
    H^0(\bar{G},\cL_{\lambda}) = \bigoplus_{\mu} \cL_{\mu}
\end{align*} where $\mu$ is dominant of the form $\lambda + \sum \Z_{\leq 0}\gamma$
 the sum is over $\gamma$ the spherical roots. See the references below

\section{References}
\begin{itemize}
    \item Sur la cohomologie à support des fibrés en droites sur les variétés symétriques complètes
    \item GROUPE DE PICARD ET NOMBRES CARACTERISTIQUES DES VARIETES SPHERIQUES
\end{itemize}
\end{document}