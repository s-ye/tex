\documentclass[12pt]{article}
\usepackage[english]{babel}
\usepackage[utf8x]{inputenc}
\usepackage[T1]{fontenc}
\usepackage{listings}
\usepackage{tikz}
\usepackage{/Users/songye03/Desktop/math_tex/style/quiver}
\usepackage{/Users/songye03/Desktop/math_tex/style/scribe}

\begin{document}
Songyu Ye

\today

\hfill

This is a note for the talk with Mike Stillman on March 7, 2024.

We are talking about Cartier divisors, line bundles, and Chern classes.

Next time we are talking about the adjunction formula and the normal bundle sequence, as well
as some examples for product of projective spaces.
\section{Introduction}
There seem to be many words floating around for the same ideas. This note tries to 
reconcile the different terminologies. In particular, we will discuss the following identifications:
\begin{itemize}
    \item Cartier divisor, line bundles, and Weil divisors
    \item Chern classes, Picard group, Chow ring, and divisor class group
\end{itemize}

\section{Chern classes}
This exposition will closely follow Chapter 5 of Eisenbud and Harris's book \textit{Intersection Theory}.
\subsection{The first Chern class of a line bundle}
\begin{definition}
    Let $\cL$ be a line bundle on a variety $X$. Let $\tau$ be a rational section of $\cL$ and 
    consider $\div(\tau)$, the divisor of zeros and poles of $\tau$.
    The first Chern class $c_1(\cL)\in A^1(X)$ is defined to be the class of $\div(\tau)$ in the Chow ring $A^*(X)$.
\end{definition}

More precisely trivialize $\cL$ over an open cover $\{U_i\}$. Then locally we can identify the rational section
$\tau$ with a rational function $f_i \in K(U_i)$. Write $f_i = g_i/h_i$ where 
$g_i,h_i \in \cO_X(U_i)$. Locally we can define the divisor \begin{align*}
    \div(f_i) = \div(g_i) - \div(h_i)
\end{align*}
where we know what the divisor of a regular function $g\in \cO_X(U)$ is via \begin{align*}
    \div(g) = \sum_{Y\subset U} \ord_Y(g) [Y]
\end{align*} sum over all irreducible closed subvarieties $Y\subset U$ of codimension 1.

\hfill

\red{Why is it enough to look at just one open subset $U$?}
Although this is happening locally, this procedure defines a global divisor class $[\div(\tau)]\in A^1(X)$. This is because
$K(X) = K(U_i)$ and the transition functions between two affine open charts corresponds to multiplication by a unit, 
which does not contribute any zeros or poles.

\begin{remark}
    Note that even though we are able to locally 
    identify a rational section with a rational function, we do not know how to talk about 
    the value of a rational section at a point. This is because two different trivializations correspond to 
    multiplying by a unit, and therefore the value of a rational function at a point is not well-defined. However by the same
    token it does make sense to always talk about the zeros and poles of a rational section.
\end{remark}

Moreover, we need to check that another choice of rational section $\tau'$ gives the same divisor class. This is because
$\tau/\tau'$ is in fact a rational function and therefore $\div{\tau}$ and $\div{\tau'}$ are rationally equivalent.

\hfill 

\red{Why is $\tau/\tau'$ a rational function?} It's because it makes sense to talk about the value of $\tau/\tau'$ at a point.
In particular, if we have two trivializations, then transitioning corresponds to multiplying the numerator and denominator
by the same unit and so the value of the fraction at a point is well-defined.

\hfill

Therefore we have stated the definition of the first Chern class of a line bundle.
\subsection{Intuition for Chern classes}
Chern classes are about undertanding the cohomology of the degeneracy (dependency) locus of some general
collection of sections of a vector bundle.

\begin{example}
    Let $\cL$ be a line bundle on a variety $X$. Then $c_1(\cL) = 0$ if and only if $\cL$ has a nowhere vanishing section.
    This is tautological. A choice of rational section $\tau$ has no zeros or poles, so it is defined 
    globally and nowhere vanishing, thus producing a trivialization of $\cL$.
\end{example}
We extend the above intuition as follows. Suppose that $\cE$ is a rank $r$ vector bundle on a variety $X$ which is globally generated. This means that 
there is a map $\cO_X^{\oplus r} \to \cE$ which is surjective on the stalks (\red{beware: this is not
the same as being surjective on global sections}). Then $\cE$ is trivial if and only if there exist
$r$ eerywhere independent global sections. Pick a general collection of $r$ global sections $\tau_0,\dots, \tau_{r-1}$
of $\cE$. The determinantal section $\tau_0\wedge \cdots \wedge \tau_{r-1}$ is a global section of $\bigwedge^r \cE$
and its divisor of zeros is the degeneracy locus of the collection of sections $\tau_0,\dots, \tau_{r-1}$ where
the sections are linearly dependent. 

\begin{definition}
    The first Chern class of a vector bundle $\cE$ is defined to be the first Chern class of the line bundle
    $\bigwedge^r \cE$.
\end{definition}

[The indexing is wrong here but use top and 1 to calibrate it] More generally if $\cE$ is globally generated, we can consider for any $i$ the scheme where
$r-i$ general sections of $\cE$ are linearly dependent, defined by the vanishing of 
the determinantal section $\tau_0\wedge \cdots \wedge \tau_{r-i-1} \in \bigwedge^{r-i} \cE$.

\hfill 

We want to know that this condition cuts out a subscheme of codimension $i$ in $X$. This is indeed true for 
a general collection of sections. For example, consider the degernacy locus of a single section of $\cE$, locally 
this is the data of functions $f_1,\dots,f_r$. Then the principal ideal theorem says that the codimension of each component
of $V(\tau)$ is at most $r$. Moreover, if $\cE$ is globally generated and $\tau$ is a general section,
then $f_{i+1}$ does not vanish identically where $f_1,\dots,f_i$ do, so the codimension is exactly $r$.

\hfill

In general we have the following theorem.
\begin{theorem}
    Let $\cE$ be a globally generated vector bundle of rank $r$ on a variety $X$. Then the degeneracy locus of a general
    collection of $r-i$ sections of $\cE$ is a subscheme of codimension $i$ in $X$.
\end{theorem}

\subsection{Chern classes in full generality}
The following discussion is about the most important properties of Chern classes.
\begin{theorem}
    There is a unique way of assigning to each vector bundle $\cE$ on a smooth quasiprojective variety $X$ 
    a class $c(\cE) = 1 + c_1(\cE) + c_2(\cE) + \cdots$ in the Chow ring $A^*(X)$ such that the following hold:
    \begin{enumerate}
        \item The class of a line bundle $c_1(\cL)$ is $1 + c_1(\cL)$.
        \item If $\tau_0,\dots,\tau_{r-1}$ are global sections of $\cE$ and the degernacy locus $D$ has codimension $i$
        then $c_i(\cE) = [D] \in A^i(X)$.
        \item \textbf{Whitney sum formula} A short exact sequence of vector bundles \begin{align*}
            0 \to \cE' \to \cE \to \cE'' \to 0
        \end{align*} induces the relation \begin{align*}
            c(\cE) = c(\cE')c(\cE'') \in A(X)
        \end{align*}
        \item \textbf{Functoriality} If $f:X\to Y$ is a morphism of smooth quasiprojective varieties
        and $\cE$ is a vector bundle on $X$, then $c(f^*\cE) = f^*c(\cE)$.
    \end{enumerate}
\end{theorem}

An immediate result of this characterization of Chern classes is that we can quickly write down 
Chern classes for vector bundles which are sums of line bundles, or more generally admit
filtrations whose quotients are line bundles. In this case \begin{align*}
    c(\cE) = c(\cL_1)\cdots c(\cL_r) = (1+c_1(\cL_1))\cdots (1+c_1(\cL_r))
\end{align*} so $c_i(\cE)$ is the i-th elementary symmetric polynomial in the $c_1(\cL_i)$.

\section{Projective bundles and the construction of Chern classes}
Projective bundles are what we use to construct Chern classes, and they also lead to the splitting principle.
\begin{definition}
    Let $\cE$ be a vector bundle on a scheme $X$. The projectivization of $\cE$ is the morphism \begin{align*}
        \pi_{\cE}:\Proj(\Sym^*(\cE)) \to X
    \end{align*} In particular the fiber over a point $x\in X$ is the set of all lines in the vector space $\cE_x$.
\end{definition}
The bundle $\pi:\P\cE\to X$ has a tautological line bundle $\cS_\cE \to \P\cE$ whose fiber
over a point $(x,l) \in \P\cE$ is the line $l\subset \cE_x$. The following theorem
is the key to constructing Chern classes.

\begin{theorem}
    Let $X$ smooth and $\cE, \P\cE$ as above. Let $\zeta$ be the first Chern class of the dual $\cS_\cE^*$
    of the tautological line bundle. Then \begin{itemize}
        \item The flat pullback map $\pi^*:A^*(X) \to A^*(\P\cE)$ is injective.
        \item $\zeta$ satisfies a unique monic polynomial 
        $f(\zeta)$ of degree $\rank(\cE)$ with coefficients in $\pi^*(A^*(X))$.
    \end{itemize}
\end{theorem}
Therefore one defines the Chern classes of $\cE$ to be the coefficients of the polynomial $f(\zeta)$. 
In particular the $i$-th Chern class $c_i(\cE)$ is the coefficient of $\zeta^{\rank(\cE)-i}$. 
\red{Stop here and ask why this theorem should be true}

\begin{theorem}
    [Splitting principle] Any identity among Chern classes of bundles that is 
    true for direct sums of line bundles is true in general.
\end{theorem}

This is because every vector bundle on a smooth variety admits a filtration whose successive quotients are line bundles.
The proof of this is by looking at the tautological subbundle and the quotient bundle.

\subsection{Corollaries of splitting}
\begin{corollary}
    Chern classes above the rank of a vector bundle are zero.
\end{corollary}
\begin{corollary}
    The Chern class of the dual of a vector bundle $c_i(\cE^*) = (-1)^i c_i(\cE)$.
\end{corollary}
\section{Examples}
\begin{example}
    [Line bundles on projective space]
Let $\cO_{\P^n}(1)$ be the tautological line bundle on $\P^n$. Then \begin{align*}
    c(\cO_{\P^n}(1)) = 1 + c_1(\cO_{\P^n}(1)) = 1 + \zeta
\end{align*} where $\zeta$ is the class of a hyperplane in $\P^n$. Similarly
\begin{align*}
    c(\cO_{\P^n}(m)) = 1 + m\zeta
\end{align*}
and the way that I think about this is that I'm thinking about the zero locus of a section of $\cO_{\P^n}(m)$, which 
is a hypersurface of degree $m$. We know that this is rationally equivalent to $m$ hyperplanes. \red{This doesn't
quite make sense for $m$ negative so we have to take duals.} 

\hfill

We can continue and consider the tautological line bundle and quotient bundle on $\P^n$. They fit into
the short exact sequence \begin{align*}
    0 \to \cS_{\P^n} = \cO_{\P^n}(-1) \to \cO_{\P^n} \otimes V \to \cQ_{\P^n} \to 0
\end{align*} from which we compute \begin{align*}
    c(\cQ) = \frac{c(\cO_{\P^n})}{c(\cO_{\P^n}(-1))} = \frac{1}{1-\zeta} = 1 + \zeta + \zeta^2 + \cdots + \zeta^n
\end{align*}
\end{example}

\begin{example}
    [Tangent bundle of projective space]
    We need to understand the Euler exact sequence \begin{align*}
        0 \to \cO_{\P^n} \to \cO_{\P^n}(1)^{\oplus n+1} \to \cT_{\P^n} \to 0
    \end{align*} from which we compute \begin{align*}
        c(\cT_{\P^n}) = (1 + \zeta)^{n+1}
    \end{align*}  
    The Euler sequence is about vector fields on $\P^n$. 
    If $V$ is an $n$-dimensional vector space $U = V\setminus \{0\}$, and $q:U\to \P V$ is the quotient map, then
    \begin{align*}
        T_vU = T_vV = V \\
        dq_v:TU_v\to T_{q(v)}\P V
    \end{align*} has kernel $\ideal{v}$ so we can identify $V/\ideal{v}$ with $T_{q(v)}\P V$.
    However this identification depends on a choice of representative for $v$ because \begin{align*}
        dq_{\lambda v} = dq_v/\lambda
    \end{align*} However if we scale by a linear functional, then we have a natural idnetification \begin{align*}
        \ideal{v}^* \otimes V/\ideal{v} \cong T_{q(v)}\P V
    \end{align*} which fits into a global identification \begin{align*}
        \cT_{\P V} \cong \cO_{\P V}(1) \otimes \cQ \cong \Hom(\cS,\cQ)
    \end{align*} 
    Put it another way, if $x_0,\dots,x_n$ are coordinates on $V$ 
    then the vector field $\partial /\partial x_i$ is not a vector feild on $\P V$ but $x_i \partial/\partial x_i$ is.
    Thus there is a map $\cO_{\P V}(1)\otimes V \to \cT_{\P V}$ whose kernel is generated by the Euler field \begin{align*}
        x_0\frac{\partial}{\partial x_0} + \cdots + x_n\frac{\partial}{\partial x_n}
    \end{align*} This is the kernel because of the identity for homogenous polynomials of degree $d$ \begin{align*}
        1/d\cdot \sum x_i \frac{\partial p}{\partial x_i} = p
    \end{align*} Thus we get the Euler sequence above.
\end{example}

\begin{example}
    [Normal bundle sequence and adjunction formula]
\end{example}
\end{document}