\documentclass[12pt]{article}
\usepackage[english]{babel}
\usepackage[utf8x]{inputenc}
\usepackage[T1]{fontenc}
\usepackage{listings}
\usepackage{hyperref}
\usepackage{tikz}
\usepackage{/Users/songye03/Desktop/math_tex/style/quiver}
\usepackage{/Users/songye03/Desktop/math_tex/style/scribe}

\begin{document}
Songyu Ye

\today

\hfill

This is a note from my meeting with Allen on March 25, 2024.
I included my notes from after the meeting with Allen, and notes from my meeting with Tara are also in this document.

\section{Introduction}
We were considering $\tilde X = \Spec R$ where \begin{align*}
    R = \bigoplus_\lambda H^0(\bar G, \cL_\lambda)
\end{align*} As it turns out this actually has a name, it is called Vinberg's Enveloping Monoid.
Let $G$ be our semisimple adjoint group and $\tilde G$ be the simply connected cover of $G$.
Then $\tilde X$ is the enveloping monoid of $\tilde G$.

\section{Enveloping monoid}
Let $M'$ be an algebraic normal irreducible monoid with unit group $G'$. $G'$ acts by left
and right translation. Suppose that $G$ is reductive and semisimple and $G$ is the commutator of $G'$. 
The quotient $A' = M'//G\times G$ exists and is a commutative monoid of unit group the torus $G'/G$.
The quotient map is called the abelianization of $M'$. $M'$ is said to be very flat if the abelianization is
a flat morphism with reduced irreducible fibers.

\hfill

Given a semisimple group $G$, the category of very flat monoids who has $G$ the commutator of the unit group admits
a terminal object $M^+$. This terminal object is the enveloping monoid of $G$ and
has an abelianization $M^+\to A^+$ of affine space of dimension equal to the rank of $G$. 
Every very flat monoid $M\in \text{VF}(G)$ is a fibered product of $A'\to A^+$ and $M^+$.

\hfill

Now suppose that $G$ is semisimple simply connected and let $G^+ = T\times G/Z$ where $Z$
is the diagonal embedding. We can extend $\rho_i:G\to \GL(V_i)$ to $\rho_i^+:G^+\to \GL(V_i)$ via
the formula \begin{align*}
    \rho_i^+(t, g) = \omega_i(w_0t^{-1})\rho_i(g)
\end{align*}
We can also extend the root $\alpha_i$ to $G^+$ via the formula \begin{align*}
    \alpha_i^+(t, g) = \alpha_i(t)
\end{align*} Then we have a map \begin{align*}
    (\alpha^+, \rho^+):G^+\to \Gm^r \times \prod \GL(V_i)
\end{align*}
whose closure in $\Ga^r\times \prod\End(V_i)$ is the enveloping monoid of $G$ and the closure in 
$\Ga^r\times \prod\End(V_i) \backslash 0$ is the semistable locus. The 
quotient of $M^+_{ss}$ by the free actino of $T$ is the wonderful compactification of $G_{ad}$.

\section{Examples}
\begin{example}
    $G = \SL_2$ and $G^+ = \SL_2\times \Gm/\{\pm 1\}$. The map \begin{align*}
        \omega_1(w\cdot \diag(t, t^{-1})) = t \\
        \alpha_1(\diag(t, t^{-1})) = t^2 \\
        (\alpha^+, \rho^+):\SL_2\times \Gm/\{\pm 1\}\to \Gm\times \GL_2 \\
        \diag(t, t^{-1}),g \mapsto (t^2, \diag(t, t)\cdot g)
    \end{align*}
    is a monoid in $\A^5$ cut out by $\det g = t$.
\end{example}


\begin{example}
    \red{This example is wrong, what is the torus of $\SL_3$...?}
    $G = \SL_3$ so that $G^+ = \SL_3\times \diag(t,\inv{t},1)/\mu_3$. We have the standard representation and 
    its dual. We have \begin{align*}
        w_0t^{-1} = \diag(1,t,\inv{t}) \\
        \omega_1(w_0t^{-1}) = 1\\
        \omega_2(w_0t^{-1}) = t\\
        \alpha_1(\diag(t,\inv{t},1)) = t^2\\
        \alpha_2(\diag(t,\inv{t},1)) = \inv{t}\\
        (t,g)\mapsto (M,tM^{-T},t^2,\inv{t})
    \end{align*} which stasfies the relations on $\A^9\times\A^9\times \A\times \A$ coordinates $A,B,c,d$ 
    \begin{align*}
        AB^T = A^TB = cdI \\
        \Lambda^2 A = dB \\
        \Lambda^2 B = cA
    \end{align*}
\end{example}


\section{After meeting Allen}
I explained the Vinberg monoid to Allen. We toyed around with the $\SL_3$ example and in particular the equations
for the Vinberg monoid. He typed some things into Macaulay2 and we found that the equations are indeed the 
defining equations for the Vinberg monoid.

\hfill

He insisted that the bottom equations would imply the top equations whenever $c,d$ are not both zero (this 
is because of Cramer's rule). He then showed me a really stupid way to prove Cramer's rule. The point is that we 
need the first set of equations in the case that $c,d$ are zero, this is something which often happens when one
 takes the closure of the image of a map.

\begin{center}
    \includegraphics[scale = .12]{/Users/songye03/Desktop/math_tex/img/IMG_1880.JPG}
\end{center}

The stuff in black is a recipe for degenerating to a toric variety. What toric variety is it going to be?
Allen sort of just read it off but basically it is the toric variety associated to the fibered product of two 
Gelfand-Tsetlin patterns. This has appeared before in another context, see the red marker.

\begin{center}
    \includegraphics[scale = .12]{/Users/songye03/Desktop/math_tex/img/IMG_1879.JPG}
\end{center}

Therefore, we are interested in the following cone and the following questions.

\section{Tara}
Tara insisted that we investigate the matter of orientation in the computation of intersection form. 
She believes that the answer has to be in Tu's book. That book also has homework problems.

\hfill

There is the following remark. If we have a $T^2$ acting on simply connected closed $4$ manifold, then these are 
completely classified [see the paper she sent us]. Moreover, simply connected tells us that the odd cohomology is zero and therefore, all of the
cohomology is captured in $H^2$ and in particular the intersection form. This motivates 
why Tara is interested in the intersection form.

\section{Getting gapped}


\section{References}
\begin{itemize}
    \item Very flat reductive monoids Rittatore
    \item On reductive algebraic semigroups Vinberg
    \item Semigroups and basic functions Ngo Bao Chau
\end{itemize}


\end{document}