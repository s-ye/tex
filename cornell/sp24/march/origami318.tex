\documentclass[12pt]{article}
\usepackage[english]{babel}
\usepackage[utf8x]{inputenc}
\usepackage[T1]{fontenc}
\usepackage{listings}
\usepackage{tikz}
\usepackage{/Users/songye03/Desktop/math_tex/style/quiver}
\usepackage{/Users/songye03/Desktop/math_tex/style/scribe}

\begin{document}
Songyu Ye

These are notes from my meeting with Tara on March 18.

\section{Example of $S^4$ and the conjecture}

\begin{example}
    Cohomology of $T^2$ on $S^4$. The moment polytope is two right triangles identified along the hypotenuse. 
    In particular if we let $a,b$ denote the classes of degree
    $2$ and $c,d$ denote the classes of degree $4$, then the relations are $ab = c + d$ and $cd = 0$.
    Note that when one sets $a = b = 0$, then this is precisely the ordinary cohomology of $S^4$.

    \hfill

    As a module over $H_T^*(pt)$, we can find generators and relations as follows. The moment graph
    tells us that we are considering pairs of polynomials $(f,g)\in \Z[x,y]$ such that $xy\vert f-g$.

    \hfill

    This can be presented as a module over the polynomial ring $\Z[x,y]$ with generators \begin{align*}
        x &= (x,x) \\
        y &= (y,y) \\
        c &= (xy,0)
    \end{align*} and the relation is precisely $c(xy-c) = 0$. Then setting $x = y = 0$ gives us the ordinary cohomology of $S^4$.
\end{example}

\begin{conjecture}
    The cohomology of a toric origami manifold can be presentated as follows. There is a 
    free generator for each face of the moment polytope (of all dimensions) and the relations are given by the
    components of the intersections of the faces.
\end{conjecture}

\section{MMP}
We talked about the minimal model program for toric manifolds. In particular there is the following result for 
four manifolds with a circle action. 

\begin{theorem}
    Suppose $M$ is a compact four manifold with a circle action. Then $M$ is equivariantly 
    diffemorphic to one of the following: \begin{itemize}
        \item $\C\P^2$
        \item $S^2\times S^2$
        \item $H_k$ where $k$ is even or odd
        \item $S^2\times \Sigma_g$ where $\Sigma_g$ is a Riemann surface of genus $g$
        \item $S^2 \tilde \times\Sigma_g$ some bundle
    \end{itemize}
    In the case of effective Hamiltonian actions, the only possibilities are $\C\P^2$, $S^2\times S^2$, Hirzebruch surfaces,
    and the $k$-fold blowups of any of those guys at fixed points. The proof of this classification is by Delzant polytopes and
    corner chopping. 
\end{theorem}
\red{Also topological blowup corresponds to connet sum with $\C\P^2$}
Tara mentioned some hope for a classification of toric origami manifolds in dimension four via similar lines. 
In particular we would have to consider how corner chopping interacts with templates.

\hfill

In particular observe how in the following example, we can see that the $S^4$ connect sum $\C\P^2$ is a toric origami manifold, 
which came from corner chopping at a single vertex on a single polytope. One can wonder what happens if you corner chop on the glued polytope.


\section{Computation for $S^4$ connect sum $\C\P^2$} 
We discussed a verification of the above conjecture for some other examples such as $S^4$ connect sum $\C\P^2$.
In particular we also tried to compute its intersection form. \red{Note that some care is required because as we pass over the fold, 
we change orientation}
\begin{example}
    Tara told us that the following four manifold is $S^4$ connect sum $\C\P^2$ (or potentially with reversed orientation). 
    The polytopal picture is two right triangles, one blown up at the right identified along the hypotenuse.

    \hfill

    The computation of the intersection form using the ABBV formula tells us that (up to orientation) the self intersection number of the 3 obvious spheres is $1,1,1$ 
    (\red{note that they are all cohomoogous because the betti numbers of this manifold are $1,0,1,0,1$})   
\end{example}

\section{Next step}
\begin{itemize}
    \item Learn more about the ABBV formula and see Tara's references
    \item Verify the conjecture for other examples of toric origami manifolds. In particular perhaps one way of showing the conjecture
    is by observing that imposing such relations and then setting degree 2 = 0 recovers the ordinary cohomology. This worked in our particular example
    for $S^4$ but the issue is that we don't know the ordinary cohmology of the other examples. 
\end{itemize}
\end{document}