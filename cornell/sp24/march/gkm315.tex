\documentclass[12pt]{article}
\usepackage[english]{babel}
\usepackage[utf8x]{inputenc}
\usepackage[T1]{fontenc}
\usepackage{listings}
\usepackage{tikz}
\usepackage{/Users/songye03/Desktop/math_tex/style/quiver}
\usepackage{/Users/songye03/Desktop/math_tex/style/scribe}

\begin{document}
Songyu Ye

These are notes from my meeting with Allen on March 15.


\section{Toric origami manifolds}

Today I asked Allen to weigh in on the project that I am working on with Tara.
In particular we talked about $T^2$ acting on $S^4$. 

\subsection{Equivariantly formal}

\begin{center}
    \includegraphics[scale = .38]{/Users/songye03/Desktop/math_tex/img/allen-conjecture.png}
\end{center}

Equivariantly formal is about when the cohomology of a fiber bundle is not the tensor product of the cohomology of the fiber and the base (i.e.
the Kunneth theorem is true).

\hfill 

The counterexample is the Hopf fibration. In particular we compared the Betti numbers of $S^2\times S^1$ and $S^3$ and
we found that $1,1,1,1$ is not the same as $1,0,0,1$. He explained that in the \red{Serre spectral sequence}, the middle cohomology groups kill each other.

\hfill 

\red{Leray Hirsch} says that when the map upstairs induces a surjection on cohomology then the Kunneth theorem is true. 
This is the case toric varieties for example, is because we are thinking about subvarieties
and we can find a representative which is T invariant by taking limits of the varieties.

\begin{example}
    We considered $S^1$ acting on $\C\P^1$ and the injection on equivariant cohomology to the two fixed points.
    Allen said he was maybe worried about the orientation changing since the folding
    hypersurface seperates the north and south poles. We decided that it was ultimately not worth worrying about.
\end{example}

\begin{example}
    Cohomology of $T^2$ on $S^4$. In particular if we let $a,b$ denote the classes of degree
    $2$ and $c,d$ denote the classes of degree $4$, then the relations are $ab = c + d$ and $cd = 0$.
    Note that when one sets $a = b = 0$, then this is precisely the ordinary cohomology of $S^4$.

    \hfill

    As a module over $H_T^*(pt)$, we can find generators and relations as follows. The moment graph
    tells us that we are considering pairs of polynomials $(f,g)\in \Z[x,y]$ such that $xy\vert f-g$.

    \hfill

    This can be presented as a module over the polynomial ring $\Z[x,y]$ with generators \begin{align*}
        x &= (x,x) \\
        y &= (y,y) \\
        c &= (xy,0)
    \end{align*} and the relation is precisely $c(xy-c) = 0$. Then setting $x = y = 0$ gives us the ordinary cohomology of $S^4$.
\end{example}

We want to make sense of Danilov's theorem for these origami manifolds. In particular Tara and Allen seem
to believe that the following statement is true:

\begin{conjecture}
    The cohomology of a toric origami manifold can be presentated as follows. There is a 
    free generator for each face of the moment polytope (of all dimensions) and the relations are given by the
    components of the intersections of the faces.
\end{conjecture}

We started talking about basic elements in posets.
The thing about our situation as opposed to the study of 
toric varieties is that the poset isn't simplicial, i.e. edges are not uniquely determined by their faces.

\section{Representation theory}
In the last five minutes, Allen insisted that we talk about representation theory.
Recall we are considering the following setup of $\bar{G}$.

\begin{center}
    \includegraphics[scale = .4]{/Users/songye03/Desktop/math_tex/img/reptheory.png}
\end{center}

Recall the above decomposition of $H^0(\bar{G},\cL_\lambda)$ into representations of $G\times G$.

We had the ring on the board.
We need to write down the kernel of the surjection from the Sym to the elements in degree 1, i.e fundamental weights.
This in turn will us the equations in affine space which cut out the image of the embedding of $\bar{G}$.

\hfill

$2\mu = 2222220000$
$\mu + \lambda = 22211111$
and the operation we are thinking about is subtract from $i$, add to $i+1$. \red{Composing these allows us 
to have the most general operation of subtracting from $i$ and adding to $j$ for $i < j$.}

\end{document}

