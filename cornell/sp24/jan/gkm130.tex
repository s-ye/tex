\documentclass[12pt]{article}
\usepackage[english]{babel}
\usepackage[utf8x]{inputenc}
\usepackage[T1]{fontenc}
\usepackage{listings}
\usepackage{tikz}
\usepackage{/Users/songye03/Desktop/math_tex/style/quiver}
\usepackage{/Users/songye03/Desktop/math_tex/style/scribe}



\begin{document}
Songyu Ye

\today

\hfill

\section{Brion 1998}
We are interested in studying the $B\times B^-$ orbits on the wonderful compactification $\overline G$
of a semisimple algebraic group $G$.

\hfill

Brion characterizes them in the following way (see references). The following result is for
any regular $G\times G$-equivariant completion $X$ of $G$. For a parabolic subgroup $P$,
let $W_P$ be the Weyl group of $P$, and let 
$W^P = \set{w\in W\mid \ell(vw)\geq \ell(v) \text{ for all } v\in W}$ system of representatives 
of the cosets of $W/W_P$.

\hfill 

Now let $Y$ be a $B\times B^-$ orbit on $\overline G$. Consider the orbit $G\times G\cdot Y$. We can associate 
this to a subset of the simple roots $I\subset \Sigma$. Recall that each $G\times G$ orbit $Z$ has 
a unique basepoint $z$ so that $B\times B^-\cdot z$ is open in $Z$ and $Z$ is the limit of a $1$-parameter 
subgroup of $T$. Then one can write every $B\times B^-$ orbit as \begin{align*}
    \overline{B\times B^-(\omega,\tau)z}
\end{align*} where $\omega\in W$ and $\tau \in W^P$ are uniquely determined.

\hfill

In the case that $X = \bar G$ Brion has the following theorem (Section 3.3). 

\begin{theorem}
    $A_{T\times T}^*(X)$ has a basis given by the classes \begin{align*}
        X(\omega,\tau) = [\overline{B\times B^-(\omega,\tau)z_\Phi}]
    \end{align*} where $\Phi = \set{\alpha\in\Sigma \st \tau(\alpha)\in R^+}$ and $z_\Phi$ is the basepoint of the $G\times G$ orbit corresponding
    to $\Phi$. The restriction to $G\times B^-\times G/B$ is equal to \begin{align*}
        (D_\omega \otimes D_\tau) \prod_{\alpha\in\Sigma\backslash\Phi}c^{T\times T}(\alpha,-\alpha)\sum_{w\in W_P(\Phi)}[\overline{B^-wB}/B\times \overline{Bw_{0,P}B}/B]
    \end{align*}
\end{theorem}
where $c^{T\times T}:X^*(T\times T)\to A_{T\times T}^*(G/B\times G/B)$ is the characteristic map.

\section{Schubert calculus and divided difference operators in the flag variety}
There is a left and right action of $W$ on classes \begin{align*}
    (w\cdot p)\vert_v := w\cdot (p\vert_{wv})
    (p\cdot w)\vert_v := p\vert_{vw}
\end{align*} The restriction of a class is a polynomial and $w$ acts by simply 
permuting the variables. The left action is a ring morphism but the right action 
is an algebra automorphism. One can check the GKM conditions to see that they are again classes.

\hfill

One defines left and right divded difference operators for $\alpha$ a simple root: \begin{align*}
    \partial_\alpha p &:= \frac{p - s_\alpha\cdot p}{\alpha} \\
    \partial^\alpha p &:= \frac{p - p\cdot s_\alpha}{c_{-\alpha}}
\end{align*} where $c_{-\alpha}$ is a class (check GKM conditions) defined by \begin{align*}
    c_{-\alpha}\vert_v = w\cdot (-\alpha)
\end{align*} Geometrically $c_{-\alpha}$ is the equivariant first Chern class of the Borel Weil
line bundle associated to $-\alpha$. One can show that \begin{align*}
    \partial_\alpha S_w &= S_{s_\alpha w} \text{ if } \ell(s_\alpha w) = \ell(w) + 1 \\
    \partial^\alpha S_w &= S_{ws_\alpha} \text{ if } \ell(ws_\alpha) = \ell(w) + 1
\end{align*}
In particular these operators $\partial_\alpha, \partial^\alpha$ act on $H_T^*(G/B)$ and preserve the Schubert basis.
One thinks of them as acting on polynomials when you pull them back into $H_T^*((G/B)^T) = H_T^*(S_n)$. 

\hfill

The right divided difference operator is a $H_T^*$ module homomorphism, obtained by composing pushforward
and pullback of the map $G/B\to G/P_\alpha$.

\section{Wonderful compactification}
The maps of interest (in particular all the fixed points lie in the closed orbit) are \begin{align*}
    X^{T\times T}\hookrightarrow G/B^-\times G/B \hookrightarrow X
\end{align*} giving us \begin{align*}
    H_{T\times T}^*(X) \hookrightarrow H_{T\times T}^*(G/B^-\times G/B) \hookrightarrow H_{T\times T}^*(X^{T\times T})
\end{align*}

\hfill

Brion tells us that the first module is has a basis given by the classes of the $B\times B^-$ orbits,
and he tells us how to expand them in terms of the canonical basis in the second module.
The second module has a basis given by the classes \begin{align*}
    \Omega(w,\tau) = [\overline{BwB^-}/B^- \times \overline{B^-\tau B}/B]
\end{align*}
The second module is where we can make sense of the divided difference operators.
The third module is where we want to compute things, so the divided difference operators
should manifest themselves as operations on polynomials. They should be helpful tools for the following question perhaps.

\hfill

One thing we can ask for a theorem for the wonderful compactification like the following one for the flag variety.
The analagous statement would be a combinatorial characterization of
the localization of a $B\times B^-$ orbit closure to a particular $(\pi,\rho) \in S_n\times S_n$)

\begin{theorem}
    Let $I$ be a reduced expression for $w\in W$. Then for each $v\in W$ \begin{align*}
        s_v\vert_w = \sum_{J\subset I}\prod_I(\hat{\alpha_i}^{[i\in J]}r_i)\cdot 1
    \end{align*}
\end{theorem}
In particular this is stronger than asking for the structure constants.

\begin{thebibliography}{99}
    \bibitem{Brion}
    Brion, M. (1998). The behaviour at infinity of the Bruhat decomposition. Commentarii Mathematici Helvetici, 73(1), 137-174.

    \bibitem{knutson2003}
    Knutson, A. (2003) A Schubert calculus recurrence from the noncomplex W-action on G/B. Journal of Algebraic Combinatorics, 17(2), 173-178.
\end{thebibliography}
\end{document}