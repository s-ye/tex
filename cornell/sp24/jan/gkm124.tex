\documentclass[12pt]{article}
\usepackage[english]{babel}
\usepackage[utf8x]{inputenc}
\usepackage[T1]{fontenc}
\usepackage{listings}
\usepackage{tikz}
\usepackage{/Users/songye03/Desktop/math_tex/style/quiver}
\usepackage{/Users/songye03/Desktop/math_tex/style/scribe}



\begin{document}
Songyu Ye

\today

\section{Grassmannian}
Let $E$ be a vector space of dimension $m$ and consider the Grassmannian of dimension $r$ and codimension $n$ in $E$.
\begin{align*}
	\Gr_r(E) = \Gr^n(E) = \set{V\subset E\st \dim V = r}
\end{align*} For each partition with at most $r$ rows and $n$ columns, and a fixed complete flag $F_\bullet$ in $E$, one can define a Schubert variety
\begin{align*}
	\Omega_\lambda = \set{V\in \Gr_r(E)\st \dim V\cap F_{n+i-\lambda_i} \geq i \text{ for } i = 1, \cdots, r}
\end{align*}jj
One also considers the Schubert cell \begin{align*}
	\Omega_\lambda^\circ = \set{V\in \Gr_r(E)\st \dim V\cap F_k = i \text{ for } n+i-\lambda_i \leq k \leq n+i-\lambda_{i+1} \text{ for } i = 0,1, \cdots, r}
\end{align*}

\begin{example}
	For the partition $\lambda = (2,1,1)$ corresponding to $3$-planes in $\C^5$ we have \begin{align*}
		\Omega_{2,1,1}       & = \set{V\in \Gr_3(\C^5)\st \dim V\cap F_1 \geq 1, \dim V\cap F_3 \geq 2, \dim V\cap F_4 \geq 3}                           \\
		\Omega_{2,1,1}^\circ & = \set{V \in \Gr_3(\C^5)\st \dim V\cap F_1 = \dim V\cap F_2 = 1, \dim V\cap F_3 = 2, \dim V\cap F_4 = \dim V\cap F_5 = 3}
	\end{align*}
	In particular one sees that the "jumps" happen as late as possible in the filtration. If one considers the standard flag,
	then the each $V$ in the Schubert cell is the row span of a unique $3\times 5$ matrix in reduced echelon form \begin{align*}
		\begin{pmatrix}
			1 & 0 & 0 & 0 & 0 \\
			0 & * & 1 & 0 & 0 \\
			0 & * & 0 & 1 & 0
		\end{pmatrix}
	\end{align*} When we consider the closure of $\Omega_{2,1,1}^\circ$ in $\Gr_3(\C^5)$, this amounts to plugging in $* = \infty$.
	For the $*$ in the second row, this corresponds to the flag \begin{align*}
		\begin{pmatrix}
			1 & 0 & 0 & 0 & 0 \\
			0 & 1 & 0 & 0 & 0 \\
			0 & 0 & * & 1 & 0
		\end{pmatrix}
	\end{align*} This corresponds to the dimension constraints $\dim V\cap F_1 = 1, \dim V\cap F_2 = 2, \dim V\cap F_3 = 2, \dim V\cap F_4 = 3$,
	equivalently the Schubert cell of the partition $\lambda = (2,2,1)$
	For the $*$ in the third row, this corresponds to the flag \begin{align*}
		\begin{pmatrix}
			1 & 0 & 0 & 0 & 0 \\
			0 & 1 & 0 & 0 & 0 \\
			0 & 0 & 1 & 0 & 0
		\end{pmatrix}
	\end{align*} which corresponds to the dimension constraints $\dim V\cap F_1 = 1, \dim V\cap F_2 = 2, \dim V\cap F_3 = 3, \dim V\cap F_4 = 3$,
	equivalently the Schubert cell of the partition $(2,2,2)$.
	The union of these three cells is the closure of the Schubert cell of the partition $(2,1,1)$.
\end{example}
This example illustrates the following facts which are desirable for a cell decomposition:
\begin{enumerate}
	\item $\Gr_r(E)$ is a disjoint union of Schubert cells.
	\item The closure of $\Omega_\lambda^\circ$ is the Schubert variety $\Omega_\lambda$.
	      $\Omega_\lambda$ is the disjoint union of $\Omega_\mu^\circ$ for $\mu\geq \lambda$.
	\item The difference $\Omega_\lambda - \Omega_\lambda^\circ$ is a union of Schubert varieties $\Omega_\mu$
	      where $\mu$ is obtained from $\lambda$ by adjoining $1$ box.
	\item Consider the filtration \begin{align*}
		      \emptyset = X_{rn} \subset \cdots \subset \cdots \subset X_0 = \Gr_r(E)
	      \end{align*} given by $X_i = \bigcup_{\abs{\lambda}\geq i} \Omega_\lambda$. Then the successive differences are disjoint unions of affine spaces. \begin{align*}
		      X_i - X_{i-1} = \bigsqcup_{\abs{\lambda} = i} \Omega_\lambda^\circ
	      \end{align*} and therefore the classes of the Schubert varieties form an additive basis for the integral cohomology of $\Gr_r(E)$.
\end{enumerate}

A priori these classes are just some random basis. One builds a map from the symmetric algebra $\Lambda \to H^*(\Gr_r(E))$ the Schur polynomial (a priori just understood combinatorially)
$s_\lambda$ to the class of $\Omega_\lambda$ if $\lambda$ has at most $r$ rows and $n$ columns, and $0$ otherwise. One shows that this map is a map of rings by examining
the special Schubert classes, those corresponding to $\lambda = (k)$ for $1\leq k\leq r$. Observe that the corresponding Schur polynomials $s_{(k)}$ are the complete symmetric polynomials
with $(k)$ corresponding to the one of degree $k$. One knows that the complete symmetric polynomials generate the symmetric algebra.

Then one shows that Pieri's formula (which holds for Schur polynomials due to combinatorics) holds for the Schubert classes. \begin{align*}
	[\Omega_{(k)}] \cdot [\Omega_\lambda] = \sum_{\mu} [\Omega_\mu]
\end{align*} the sum is over all $\mu$ obtained by adding $k$ boxes to $\lambda$, no two in the same column. This establishes that
the map $\Lambda \to H^*(\Gr_r(E))$ is is a map of rings.

\section{Equivariant cohomology of the Grassmannian}

\section{Flag Variety}
Recall the Bruhat decomposition for $\GL(n,\C)$:
\begin{align*}
	\GL(n,\C) = \bigsqcup_{w \in W} BwB
\end{align*}
This decomposition gives us a decomposition of the flag variety:
\begin{align*}
	G/B = \bigsqcup_{w \in W} BwB/B
\end{align*} and each double coset $BwB/B$ is isomorphic to $\C^{\dim w}$, where $\dim w$ is the number of positive roots in $w$.
In particular we have a cell decomposition since each $BwB/B$ is isomorphic to an affine space.
The $BwB/B$ are called \red{opposite} Schubert cells, and the closures of the Schubert cells are called Schubert varieties.

\hfill

The key observation is that each Schubert variety is the closure of an open subvariety isomorphic to affine space, and
the union of the Schubert cells in the closure is the Schubert variety. \begin{align*}
	\overline{BwB/B} = \bigsqcup_{v \leq w} BvB/B
\end{align*}

This means that the classes of the Schubert varieties form an additive basis for homology. This is a consequence of the following fact:

\begin{theorem}
	Suppose that $0 = X_0 \subset X_1 \subset \cdots \subset X_n = X$ is a filtration of an algebraic variety $X$
	by closed algebraic subsets so that successive differences are disjoint unions of varieties $U_{i,j}$ each isomorphic to affine space. Then the classes
	of the closures of the $U_{i,j}$ form an additive basis for the integral Borel Moore homology of $X$.
\end{theorem}

Recall further that for $X$ an oriented $n$-manifold, one can compute the Borel Moore homology by embedding $X$ inside itself and thus \begin{align*}
	\overline{H}_i(X) = H^{n-i}(X, X - X) = H^{n-i}(X)
\end{align*} the Borel Moore homology is isomorphic to the singular cohomology.

\hfill

There is a far reaching generalization of this phenomenon, obtained in the work of Bialynicki-Birula. He considered smooth projective
variety $X$ with $\C^*$ action. In this case, the limit $\lim_{t \to 0} t \cdot x$ exists for all $x \in X$, and in particular the limit
is a fixed point for the $\C^*$-action. Consider the case when $X^{\C^*} = \{x_1, \cdots, x_n\}$ is finite.

\hfill

Let $p\in X^{\C^*}$ and $X_p = \set{x\in X\st \lim_{t\to 0} t\cdot x = p}$. Since $p$ is $\C^*$ invariant, $\C^*$ acts on the tangent space $T_pX$.
Decompose into weight spaces $T_pX = \oplus_{n\in\Z} T_pX[n]$. Then $T_pX[0] = 0$ since $X^{\C^*}$ is discrete and therefore
there is a decomposition $T_pX =  T_pX^+ \oplus T_pX^-$. Bialynicki-Birula showed the following.

\begin{theorem}
	The decomposition \begin{align*}
		X = \bigsqcup_{p\in X^{\C^*}} X_p
	\end{align*} is a decomposition into locally closed subvarieties, and each $X_p$ is $\C^*$-isomorphic to $T_pX^+$. $X_p$ is
	open in its closure and isomorphic to affine space.
\end{theorem}

Furthermore Bialynicki-Birula showed that the variety $X$ is obtained by a sequence of attachments of the $X_p$'s, so the classes of the
closures form an integral basis for homology.

\hfill

Put $X_i = \bigcup_{\dim X_p\leq i} \overline{X_p}$. This way we get a filtration on $X$ by closed subvarieties with the property that
successive differences are disjoint unions of affine space \begin{align*}
	X_i - X_{i-1} = \bigsqcup_{\dim X_p = i} X_p
\end{align*}

\section{Things to wonder about}
\begin{enumerate}
	\item Writing down wonderful compactifications of semisimple adjoint groups.
	\item Finding the $1$-dimensional $T\times T$-orbits in the wonderful compactification
	\item The computation for the loop group
	\item A canonical basis for the equivariant cohomology
	\item Other torus actions on familiar projective varieties.
	\item
\end{enumerate}

\section{Appendix A: The class of a subvariety}
One uses Borel Moore homology to write down the class of a subvariety. Recall that Borel Moore homology is defined for
a closed embedding of $X$ into an oriented differentiable manifold $X\hookrightarrow Y$. In this situation one puts \begin{align*}
	\overline{H}_i(X) = H^{n-i}(Y, Y - X)
\end{align*} It is a consequence of the Thom isomorphism that this is independent of the embedding.
For $X$ oriented $n$-manifold, one can compute the Borel Moore homology by embedding $X$ inside itself and thus \begin{align*}
	\overline{H}_i(X) = H^{n-i}(X, X - X) = H^{n-i}(X)
\end{align*} the Borel Moore homology is isomorphic to the singular cohomology.

If further $X$ is compact, then by Poincare duality, the Borel Moore homology is isomorphic to the singular homology.

\hfill

There is a long exact sequence in Borel Moore homology associated $U\subset X$ open. Let $Y$ the complement of $U$ in $X$.
Then we have \begin{align*}
	\cdots \to \overline{H}_i(Y) \to \overline{H}_i(X) \to \overline{H}_i(U) \to \overline{H}_{i-1}(Y) \to \cdots
\end{align*} coming from the long exact sequence in cohomology of the triple $M - X \subset M - Y \subset M$.

\hfill

One of the consequences of this long exact sequence is the following.

\begin{theorem}
	Let $V$ be an algebraic subset of a smooth algebraic variety of dimension $k$. Then $\overline{H}_i(V) = 0$ for $i > 2k$ and
	$\overline{H}_{2k}(V)$ is a free abelian group on the irreducible components of $V$.
\end{theorem}

Now suppose that $V$ is an irreducible closed subvariety of dimension $k$ of a smooth projective (or compact) variety $X$.
Then $\overline{H}_{2k}(V)\cong \Z$ has a canonical generator and the closed embedding of $V$ into $X$ induces a map \begin{align*}
	\overline{H}_{2k}(V) \to \overline{H}_{2k}(X) = H^{2n-2k}(X)
\end{align*} where $n = \dim X$. The image of the generator is called the \red{fundamental class} of $V$ in $X$ and is denoted $[V]$.
\end{document}