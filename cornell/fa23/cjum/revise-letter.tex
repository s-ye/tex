\documentclass[12pt]{article}
\usepackage[english]{babel}
\usepackage[utf8x]{inputenc}
\usepackage[T1]{fontenc}
\usepackage{listings}
\usepackage{tikz}
\usepackage{/Users/songye03/Desktop/math_tex/style/quiver}
\usepackage{/Users/songye03/Desktop/math_tex/style/scribe}



\begin{document}
Songyu Ye

\today

\section{Introduction}
We thank the referees for the detailed reading of the manuscript and 
the numerous helpful comments. We did not disagree
with any of the comments and have made a number of changes to the manuscript in response to 
the feedback.

\section{Major changes}
\begin{itemize}
    \item \textbf{An introduction to the paper}: We present a roadmap for the exposition as well as highlight
    the key ideas of the paper, which is to compare and contrast group objects in the smooth and algebraic 
    settings.
    \item \textbf{Details about reducibility of representations have been added to the paper}: brief remarks
    about Weyl's unitary trick and references to the literature have been added.
    \item \textbf{The issue of freely moving between the smooth and algebraic category} [UG general 5]:
    This is the largest change to the paper and we thank the referees for pointing out
    the need for more details. We have adjusted the paper as follows. When we discuss the aspects
    of the proof of the Theorems of the Highest Weight, we make it clear that we are working in the smooth setting. In particular
    we make use of the exponential map. Immediately after, we turn our attention to the algebraic setting and in particular, 
    redescribe the Lie algebra of $G$ in terms of the Zariski tangent space. We then proceed to discuss the theorem in such context.
    \item \textbf{The incorrect statement} "Studying the action of $\gl(2,\C)$ is
    equivalent to studying the action of $\GL(2,\C$) on $V$ because $\GL(2, \C)$ is
    connected" has been corrected to reflect the simply-connectedness of $\GL(2, \C)$.
    \item \textbf{References have been bibtexed} and chapter references have been added to the paper.
\end{itemize}

\section{Minor changes}
\begin{itemize}
    \item Spelling and style corrections have been made throughout the manuscript,
    wherever the referees have pointed out errors [UG Specific comments].
    \item More details about conventions and notations have been added to the paper [UG specific comments].
    \item The examples about the standard and symmetric power representations have
    been upgraded to highlight the highest weights [UG general 4].
    \item Motivation for Hopf algebras has been added 
    \item Uniqueness of comultiplication map addressed [G major 1]
    \item Definition of positive system in the case of $\gl(2)$ has been made more clear.

\end{itemize}
\end{document}