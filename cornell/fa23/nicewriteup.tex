\documentclass[12pt]{article}
\usepackage[english]{babel}
\usepackage[utf8x]{inputenc}
\usepackage[T1]{fontenc}
\usepackage{/Users/songye03/Desktop/math_tex/style/scribe}
\usepackage{/Users/songye03/Desktop/math_tex/style/quiver}
\usepackage{listings}
\usepackage{tikz}


\begin{document}
Songyu Ye

\today
\section{Introduction}
In this note, we discuss representations of complex tori and $\GL(2,\C)$ as algebraic groups. 

\section{Representations of Tori}
\subsection{Real tori}
Before people thought about $\Gm(\C)$ as an algebraic group, people thought about $S^1$ as a Lie group.
In this setting, one can ask about the finite dimensional complex representations of $S^1$.

\hfill 

One can also show that all finite dimensional representations of $S^1$ are decomposable, i.e.
can be written as a direct sum of irreducible representations. Thus it is enough to just consider the irreducible representations of $S^1$.

\hfill 

By Schur's Lemma they are all $1$-dimensional and therefore are indexed by characters $\chi: S^1\to \C^*$.
Since $S^1$ is compact, its image in $\C^*$ must also be compact, and moreover
it is connected and contains the identity. Therefore, the image of $\chi$ must lie in $S^1$.


\begin{proposition}
All characters of $S^1$ are isomorphic to $\chi_n: S^1\to S^1$ given by $z\mapsto z^n$ for $n\in \Z$.
\end{proposition}

\begin{proof}
    One can prove by making use of the universal covering map $\exp: \R\to S^1$.
    Given a character $\chi: S^1\to S^1$, we can lift it to a map $\tilde \chi: S^1\to \R$.
    Since $\chi$ is a group map, it carries $1$ to $1$, and the fiber over $1$ under $\exp$ is $\Z$.
\end{proof}

In general, a real torus $T$ is a real Lie group whihch is isomorphic to the product of $n$ circles, 
in which case we say that $T$ has rank $n$. Since characters for $S^1\times\cdots\times S^1$ are the same as products of characters for $S^1$, 
all characters of $T$ are indexed by $\Z^n$. Explicitly if $T$ has rank $n$, then a character $\chi: T\to S^1$ is given by a tuple of integers $(n_1,\ldots, n_k)$, and the character is given by
\begin{align*}
    (z_1,\ldots, z_k)\mapsto z_1^{n_1}\cdots z_k^{n_k}
\end{align*}
From our discussion above, we have the following classification statement for representations of real tori as Lie groups.

\begin{theorem}
    Let $T$ be a real torus of rank $n$. Then every finite dimensional representation $V$ of $T$ is isomorphic 
    to a direct sum of $1$-dimensional irreducible representations with some multiplicities.
    \begin{align*}
        V \cong \bigoplus_{\chi\in \Z^n} W_\chi^{\oplus n_\chi}
    \end{align*} where $W_\chi$ denotes the unique $1$-dimensional irreducible representation for which $T$ acts by $\chi$.
\end{theorem}

In particular, we can decompose $V$ into eigenspaces for the action of $T$. \begin{align*}
    V \cong \bigoplus_{\chi\in\Z^n} V_\chi
\end{align*} where $V_\chi = \set{v\in V \st t\cdot v = \chi(t)v \text{ for all $v\in V$}}$. This is
referred to as the \textbf{weight space decomposition} of $V$ and we refer to the $\chi$ which appear in 
the decomposition as the \textbf{weights} of $V$. 
We say that $v\in V_\chi$ is a \textbf{weight vector} of weight $\chi$.


\subsection{Complex tori}
We want an analagous story in algebraic geometry. To get ahold of such a story, 
we need the following framework.
\begin{definition}
    An \textbf{algebraic group} $G$ over $\C$ is an algebraic variety over $\C$ with a group structure
    so that the multipliation map $G\times G\to G$ and inversion map $G\to G$ are morphisms
    of algebraic varieties.
\end{definition}

\begin{definition}
    A \textbf{morphism of algebraic groups} $G\to H$ is a morphism of algebraic varieties that is also a group homomorphism.
\end{definition}

\begin{definition}
    Let $G$ be an algebraic group. A \textbf{rational representation} of $G$ is a morphism of algebraic groups
    $G\to \GL(V)$ for some vector space $V$ (for us $V$ will always be finite dimensional over $\C$). 
\end{definition}

We will consider complex tori $T = \Gm(\C)\times\cdots\times\Gm(\C)$.
This is an algebraic group because $T$ is the zero locus of 
the polynomial equations \begin{align*}
    T = \Spec \C[x_1^{\pm 1},\ldots, x_n^{\pm 1}] := \Spec\big(\C[x_1,\dots,x_n,y_1,\dots,y_n]/(x_1y_1-1,\dots,x_ny_n-1)\big)
\end{align*} This is a group in the familiar way, 
and it is clear that the group law is indeed a morphism of algebraic varieties. The rest of this section will discuss 
the finite dimensional rational representations of $T$ as an algebraic group.

\begin{example}
    $\GL(n,\C)$ is a familiar group which can be endowed with the structure of an algebraic group.
    $\GL(n,\C)$ is the zero locus of the polynomial equations
    \begin{align*}
        \GL(n,\C) = \Spec \big(\C[x_{ij},\inv{\det}]\big) := \Spec\big(\C[x_{ij},t]/(\det(x_{ij})t-1)\big)
    \end{align*} This variety becomes a group in the familiar way,
    and it is clear that the group law is indeed a morphism of algebraic varieties.
\end{example}




\begin{theorem}\label{thm:ratrep}
    Let $T$ be a complex torus of rank $n$.
    Then every finite dimensional rational representation of $T$ 
    decomposes into the direct sum of weight spaces for the action of $T$.
    \begin{align*}
        V \cong \bigoplus_{\chi\in \Z^n} V_\chi
    \end{align*} where $V_\chi = \set{v\in V \st t\cdot v = \chi(t)v \text{ for all $v\in V$}}$
    and $\chi:T\to\GL(1,\C)$ is identified with tuples of integers $(n_1,\ldots, n_k)$ as before.
\end{theorem}
We will give an proof of this theorem after we introduce the language of Hopf algebras.
\subsection{Hopf algebras}
\begin{definition}\label{def:hopf}
    Let $A$ be a $\C$-algebra. Then we say $A$ is a \textbf{Hopf algebra} if there are maps \begin{align*}
        \text{comultiplication\quad}\Delta: A &\to A\otimes A \\
        \text{counit (augmentation)\quad}\epsilon: A &\to \C \\
        \text{coinverse (antipode)\quad}S: A &\to A
    \end{align*}
    so that the following diagrams commute:
    \begin{center}
        \begin{tikzcd}
            A \arrow[r, "\Delta"] \arrow[d, "\Delta"] & A\otimes A \arrow[d, "\Delta\otimes \id"] \\
            A\otimes A \arrow[r, "\id\otimes \Delta"] & A\otimes A\otimes A
        \end{tikzcd} \\

        \begin{tikzcd}
            A \arrow[r, "\Delta"] \arrow[d, "\id"] & A\otimes A \arrow[d, "\epsilon\otimes \id"] \\
            A \arrow[r, "\cong"] & \C\otimes A
        \end{tikzcd} \\
        
        \begin{tikzcd}
            A \arrow[r, "\Delta"] \arrow[d, "\epsilon"] & A\otimes A \arrow[d, "S \otimes \id"] \\
            \C \arrow[r, ""] & A
        \end{tikzcd}

    \end{center}
\end{definition}
This definition seemingly comes from nowhere, but the following observation will make them more natural
(the names also will make more sense!).

\hfill

Let $G$ be an algebraic group and $\mc O(G)$ its ring of regular functions. Then the multipliation,
inversion, and identity maps \begin{align*}
    \mu: G\times G &\to G \\
    \iota: G &\to G \\
    e: \Spec \C &\to G
\end{align*}
induce maps on the coordinate rings
\begin{align*}
    \Delta: \mc O(G) &\to \mc O(G)\otimes \mc O(G) \\
    \epsilon: \mc O(G) &\to \C \\
    S: \mc O(G) &\to \mc O(G)
\end{align*} where we made the identification $\mc O(G\times G) \cong \mc O(G)\otimes \mc O(G)$.
Because the group axioms hold, these maps satisfy the commutative diagrams above, and therefore
$\mc O(G)$ has the structure of a Hopf algebra.

\begin{remark}
    These maps can be worked out very explicitly. In particular the points of $G$ are in correspondence with the elements of $\Hom_{\kAlg}(\cO(G),\C)$.
    The correspondence is very explicitly given by $g\mapsto \ev_g$ where $\ev_g: \cO(G)\to \C$ is the evaluation map. 
    The key idea is that for an arbitrary algebraic group $G$ and two points $x,y$ of $G$,
     $f,g: \cO(G)\to \C$ the corresponding morphisms of $\C$-algebras, then the composition $(f\otimes g)\circ \Delta$ is again a map $\cO(G)\to \cO(G)\otimes \cO(G)\to \C$.
    The condition that we ask from comultiplication is precisely that the composition $(f\otimes g)\circ \Delta$
    corresponds to the product $xy\in G$.
\end{remark}

\begin{example}
Recall that
    \begin{align*}
        \cO(\Ga(\C)) &= \C[x]
    \end{align*} where $\Ga(\C)$ is the additive group of $\C$. 
    Let $f,g\in \Hom_{\kAlg}(\cO(G),\C)$ with $f(x) = a$ and $g(x) = b$. 
    We want to find a map $\Delta: \cO(\Ga(\C))\to \cO(\Ga(\C))\otimes \cO(\Ga(\C))$ so that
    \begin{align*}
        ((f\otimes g)\circ \Delta)(X) = (a+b)
    \end{align*}
    We write down the map $\Delta$ explicitly as
    \begin{align*}
        \Delta(X) = X\otimes 1 + 1\otimes X
    \end{align*} and notice that it does the job. We see that $\Delta$ then must be the 
    comultiplication map for $\Ga(\C)$ because such map is unique, given the prescribed group law on $\Ga(\C)$.
\end{example}

\begin{example}
    By the same token, we can work out the Hopf algebra structure for $\Gm(\C)$ to be
    \begin{align*}
        \Delta(x) &= x\otimes x \\
        \epsilon(x) &= 1 \\
        S(x) &= \inv{x}
    \end{align*}
\end{example}

\begin{example}
    Consider the example of $\GL(2,\C)$ as an algebraic group. The Hopf algebra structure is given by
    \begin{align*}
        \Delta(x_{ij}) &= \sum_{k=1}^2 x_{ik}\otimes x_{kj} \\
        \epsilon(x_{ij}) &= \delta_{ij} \\
        S(x_{ij}) &= M_{ij}
    \end{align*} where \begin{align*}
        M = \begin{bmatrix}
            x_{11} & x_{12} \\
            x_{21} & x_{22}
        \end{bmatrix}^{-1} = \frac{1}{\det(M)}\begin{bmatrix}
            x_{22} & -x_{12} \\
            -x_{21} & x_{11}
        \end{bmatrix}
    \end{align*}
\end{example}

\hfill

Now we want to translate the representation theory of algebraic groups $G$ into the language of comodules over Hopf algebras.
\begin{theorem}
    Let $G$ be an algebraic group. Then rational representations $V$ of $G$ 
    correspond to linear maps $\rho: V\to V\otimes \mc O(G)$ so that the following diagrams commute:
    \begin{center}
        \begin{tikzcd}
            V \arrow[r, "\rho"] \arrow[d, "\rho"] & V\otimes \mc O(G) \arrow[d, "\rho\otimes \id"] \\
            V\otimes \mc O(G) \arrow[r, "\id\otimes \Delta"] & V\otimes \mc O(G)\otimes \mc O(G)
        \end{tikzcd} \\

        \begin{tikzcd}
            V \arrow[r, "\rho"] \arrow[d, "\id"] & V\otimes \mc O(G) \arrow[d, "\id\otimes \epsilon"] \\
            V \arrow[r, "\cong"] & V\otimes \C
        \end{tikzcd} \\
    \end{center}
\end{theorem}
\begin{proof}
    [Sketch of proof] The first diagram says that the action of $G$ on $V$ is associative 
    and the second diagram says that $e\in G$ acts by the identity transformation on $V$. 
    These are precisely the conditions that say that $V$ is a representation of $G$.
\end{proof}
We refer the reader to \cite{waterhouse} for a more detailed discussion of this theorem.

\begin{definition}
    $\rho$ is called a \textbf{comodule structure} on $V$.
\end{definition}

\begin{example}
    Consider the action of $\Gm(\C)$ on $\C^2$ given by
    \begin{align*}
        t\cdot (a,b) = (ta, t^{-1}b)
    \end{align*} This is a rational representation of $\Gm(\C)$ which we can write as 
    \begin{align*}
        \tau:\Gm(\C) &\to \GL(2,\C) \\
        t &\mapsto \begin{bmatrix}
            t & 0 \\
            0 & \inv{t}
        \end{bmatrix}
    \end{align*} This induces a comodule structure on $\C^2$ given by the map $\rho: \C^2\to \C^2\otimes \mc O(\Gm(\C))$ given by
    \begin{align*}
        \rho(a) &= a\otimes x \\
        \rho(b) &= b\otimes \inv{x}
    \end{align*} where $x$ is the coordinate function on $\Gm(\C)$.
\end{example}

\subsection{Weight space decomposition}
We are now ready to give a proof of Theorem \ref{thm:ratrep} using the language of Hopf algebras.
\begin{proof}
    [Proof of \ref{thm:ratrep}] Let $V$ be a finite dimensional rational representation of $T$ 
    and let $\rho: V\to V\otimes \mc O(T)$ be the corresponding comodule structure. We write as a vector space 
    decomposition \begin{align*}
        V \otimes \mc O(T) \cong \bigoplus_{m\in\Z^n} V\otimes \C\cdot x^m
    \end{align*} Expanding $\rho(v)$ in terms of this basis, we find that \begin{align*}
        \rho(v) &= \sum_{m\in\Z^n} v_m\otimes x^m \texty{finitely many nonzero $m_k$} \\
        \implies (\id \otimes \Delta)(\rho(v)) &= \sum_{m\in\Z^n} v_m\otimes x^m\otimes x^m \\
        \implies (\rho\otimes \id)(\rho(v)) &= \sum_{m\in\Z^n} \rho(v_m)\otimes x^m \\
        \implies \rho(v_m) &= v_m\otimes x^m \texty{for those nonzero $v_m$}
    \end{align*} The second step comes from our computation that $\Delta(x_i) = x_i\otimes x_i$ for $x_i\in \cO(T) \cong \C[x_i^{\pm 1}]_{1\leq i \leq n}$
    and the fact that $\Delta$ is an coalgebra homomorphism. The claim that $\Delta$ is a 
    morphism of coalgebras is not immediate, but it ultimately reduces to the statement that, if $B$ is
    a $k$-algebra, then the multiplciation map $B\otimes B\to B$ is a morphism of $k$-algebras if and only if $B$ is commutative.
    We are working with (co)commutative (co)algebras, so this is not an issue. The fourth step comes from equating the second and third left hand sides.

    \hfill 

    Finally, we apply the second diagram in \ref{def:hopf} to get \begin{align*}
        (\id\otimes \epsilon)(\rho(v)) = v = \sum_{m\in\Z^n} v_m\epsilon(x^m) = \sum_{m\in\Z^n} v_m
    \end{align*} Thus we see that the comodule $V$ decomposes as a direct sum of subcomodules \begin{align*}
        V &= \bigoplus_{m\in\Z^n} V_m
    \end{align*} where $V_m := \set{v\in V \st \rho(v) = v\otimes x^m}$. This is precisely saying that $T$ acts on $V_m$ by the character $\chi_m: T\to \Gm(\C)$ given by $t\mapsto t^m$.
\end{proof}

\section{Representations of $\GL(2,\C)$}

\subsection{Reducibility}
We saw in Section 2 that every rational representation of $T$ decomposes into a direct sum of irreducible representations,
and that the irreducible representations are indexed by $\Z^n$. As it turns out, rational representations of $\GL(2,\C)$
also decompose into a direct sum of irreducible representations. The standard way of showing this is via 
Weyl's Unitary Trick. We refer the reader to \cite{milne} for a proof of this fact.


\subsection{Highest weight vectors}
To obtain the classification statement, we need to introduce highest weight vectors.
Let $T\subset \GL(2,\C)$ be the subgroup of diagonal matrices and $B \subset \GL(2,\C)$ be the subgroup of upper triangular matrices.
These ad hoc definitions will work for us, but in general $T$ is a \textbf{maximal torus} and $B$ is a \textbf{Borel subgroup} of $\GL(2,\C)$.
\begin{definition}
    Let $V$ be a finite dimensional rational representation of $\GL(2,\C)$. A \textbf{highest weight vector} $v\in V$ is 
    a weight vector so that $B\cdot v = \C^*\cdot v$. 
\end{definition}



\begin{example}
$\GL(2,\C)$ has a standard representation on $\C^2$ given by the matrix multiplication map. This action
is transitive on the nonzero vectors, so $\C^2$ is irreducible. If one considers the torus action $T\subset \GL(2,\C)$
then we see that $\C^2$ decomposes into a direct sum of weight spaces \begin{align*}
    \C^2 \cong \C\cdot e_1 \oplus \C\cdot e_2
\end{align*} where $e_1$ and $e_2$ are standard basis with weights $(1,0)$ and $(0,1)$ respectively.
\end{example}

\begin{example}
Since $\GL(2,\C)$ acts on $\C^2$, it also acts on $(\C^2)^{\otimes n}$ for $n\in \Z_{\geq 0}$ via \begin{align*}
    g\cdot (v_1\otimes \cdots \otimes v_n) = (gv_1\otimes \cdots \otimes gv_n)
\end{align*} This is known as the \textbf{tensor product representation} of $\GL(2,\C)$. We can further 
quotient by the submodule generated by vectors of the form \begin{align*}
    v_1\otimes \cdots \otimes v_i \otimes v_{i+1} \otimes \cdots \otimes v_n - v_1\otimes \cdots \otimes v_{i+1} \otimes v_i \otimes \cdots \otimes v_n
\end{align*} for $1\leq i \leq n-1$. This is known as the \textbf{symmetric power representation} of $\GL(2,\C)$, denoted by $\Sym^n\C^2$.
Choosing a basis $e_1,e_2$ for $\C^2$ gives a basis for $\Sym^n\C^2$ given by \begin{align*}
    \set{e_1^ke_2^{n-k} \st 0\leq k \leq n}
\end{align*} and the action of $\GL(2,\C)$ on $\Sym^n\C^2$ is given by \begin{align*}
g\cdot e_1^ke_2^{n-k} = (ge_1)^k(ge_2)^{n-k}
\end{align*} Setting $g\in T$ we see that $e_1^ke_2^{n-k}$ is a weight vector with weight $(k,n-k)$. One 
can quickly check that $\Sym^n\C^2$ is irreducible for all $n\in \Z_{\geq 0}$.
\end{example}

\begin{example}
    We have a familiar $1$-dimensional representation of $\GL(2,\C)$ given by the determinant map. The determinant of a diagonal matrix is the product of its diagonal entries,
    and so this representaion has weight $(1,1)$. We will denote 
    the $k$th power of the determinant map by $\det^k$ for $k\in\Z$. This is a $1$-dimensinal representation with weight $(k,k)$.
\end{example}

We are now ready to state the classification theorem for finite dimensional rational irreducible representations of $\GL(2,\C)$.
\begin{theorem}
Every finite dimensional rational irreducible representation of $\GL(2,\C)$ is isomorphic to \begin{align*}
    \Sym^n\C^2 \otimes \det{}^k
\end{align*} for some $n\in \Z_{\geq 0}$ and $k\in \Z$.
\end{theorem}
We will prove this theorem by considering the weights that appear in the weight space decomposition of $V\vert_{T}$, where $T\subset \GL(2,\C)$ is the subgroup of diagonal matrices.

\hfill 

In particular we appeal to the following facts from representation theory.
\begin{theorem}\label{thm:hw}
    \begin{enumerate}
        \item A finite dimensional rational representation $V$ of $\GL(2,\C)$ is irreducible if and only if it has a unique highest weight vector.
        In this case it makes sense to talk about the \text{highest weight} of $V$, defined as the weight corresponding to the highest weight vector.
        \item Two finite dimensional rational irreducible representations of $\GL(2,\C)$ are isomorphic if and only if they have the same highest weight.
        \item Let $V$ be a finite dimensional irreducible rational representation of $\GL(2,\C)$ with highest weight vector $v$. Then the highest weight of $V$ is contained in the set \begin{align*}
            \set{(a,b)\in \Z^2 \st a\geq b }
        \end{align*}
        \item Every such weight above is a highest weight for some irreducible representation of $\GL(2,\C)$.
    \end{enumerate}
\end{theorem}

We will give a short discussion of the proof of these theorems in the case of $\GL(n,\C)$, collectively referred to as the theorems of the highest weight.
This theorem holds in great generality for other algebraic groups such as $\SL(n,\C)$ and $\SO(n)$ and in general for representations of complex semisimple Lie algebras,
but in order to make sense of such a theorem, one has to find the right way to define the notion of a highest weight vector.
In the case of $\GL(2,\C)$, I boldly asserted that a highest weight vector is a weight vector $v$
 so that $B\cdot v = \C^*\cdot v$. This is not the most general definition, but it is the
right definition for $\GL(2,\C)$.

\hfill 

The proof in full generality is quite technical and we refer the reader to \cite{milne} for a more detailed discussion.

\hfill

The theorems of the highest weight immediately implies the classification theorem for finite dimensional rational irreducible representations of $\GL(2,\C)$.
In particular let $V$ be a finite dimensional rational irreducible representation of $\GL(2,\C)$ with highest weight $(a,b)$. Then
by looking at the highest weights (observe that if $v$ is a weight vector for $V$ with weight $\mu$ and 
$w$ is a weight vector for $W$ with weight $\nu$, then $v\otimes w$ is a weight vector for $V\otimes W$ with weight $\mu + \nu$)
, we see that $V \cong \Sym^{a-b}\C^2 \otimes \det{}^{b}$. 


\subsection{Theorem of the highest weight}
In this section we will discuss \ref{thm:hw} in the case of $\GL(2,\C)$.

\hfill

One of the main ingredients in the proof of \ref{thm:hw} is considering the induced action of $\gl(2,\C)$ on $V$,
where $\gl(2,\C)$ is the Lie algebra of $\GL(2,\C)$. Recall that \begin{align*}
    \gl(2,\C) = \Mat(2,\C)
\end{align*} is a vector space equipped with a bracket operation given by the commutator. $\gl(2,\C)$
can be identified with the tangent space of $\GL(2,\C)$ at the identity matrix. We can then consider the action of $\gl(2,\C)$ on $V$ given by \begin{align*}
    X\cdot v = \frac{d}{dt}\bigg\vert_{t=0} \exp(tX)\cdot v
\end{align*} where $\exp: \gl(2,\C)\to \GL(2,\C)$ is the exponential map. In particular this map is the differential of the action of $\GL(2,\C)$ on $V$.

\hfill

Studying the action of $\gl(2,\C)$ on $V$ is equivalent to studying the action of $\GL(2,\C)$ on $V$ because $\GL(2,\C)$ is connected.
This is a general principal which reflects the fact that any map of Lie groups $G\to H$ is determined by its differential at the identity.
Using the exponential map, we can get ahold of the following lemma.
\begin{lemma}
    A subspace $W$ of a representation of $\GL(2,\C)$ is a subrepresentation if and only if $W$ stable under the action of $\gl(2,\C)$.
\end{lemma}
This discussion justifies our passing from the study of $\GL(2,\C)$ to the study of $\gl(2,\C)$.

\hfill

Just as we obtained a decomposition of $V$ as a $\GL(2,\C)$ into eigenspaces for the action of $T$,
there is an analagous decomposition for the action of $\gl(2,\C)$. 
The object which replaces our maximal torus $T\subset\GL(2,\C)$ is the \textbf{Cartan subalgebra} $\mf h\subset \gl(2,\C)$.
For us, $\mf h$ will be the subspace of diagonal matrices in $\gl(2,\C)$. In general, $\mf h$ is a maximal abelian subalgebra of $\gl(2,\C)$.

We can obtain a decomposition of $V$ into eigenspaces for the action of $\mf h$: \begin{align*}
    V &= \bigoplus_{\chi\in \mf h^*} V_\chi
\end{align*} where $V_\chi = \{v\in V \st X\cdot v = \chi(X)v \text{ for all $X\in \mf h$}\}$. Moreover
$\gl(2,\C)$ acts on itself via the bracket (adjoint representation) and we can 
decompose this action \begin{align*}
    \gl(2,\C) &\cong \mf h \oplus \mf g_{\alpha} \oplus \mf g_{-\alpha} \\
    &\cong \mf h \oplus \C e_1 \oplus \C e_2
\end{align*} where $\alpha(\begin{bmatrix}
    d_1 & 0 \\
    0 & d_2
\end{bmatrix}) = d_1 - d_2$ and \begin{align*}
    e_1 &= \begin{bmatrix}
        0 & 1 \\
        0 & 0
    \end{bmatrix}\\
    e_2 &= \begin{bmatrix}
        0 & 0 \\
        1 & 0
    \end{bmatrix}
\end{align*} One can check that \begin{align*}
    [\mf h, e_1] &= \alpha(\mf h)e_1 \\ [\mf h, e_2] &= -\alpha(\mf h)e_2
\end{align*} The weights which appear in the adjoint representation of $\gl(2,\C)$ are called the \textbf{roots}.
We will say $\alpha$ is a positive root and $-\alpha$ is a negative root, $\C e_1$ and $\C e_2$ the corresponding positve and negative root spaces.
Then a weight vector $v\in V$ is a highest weight vector if and only if $e_1\cdot v = 0$.

\hfill

We care about roots of the adjoint representation for the following reason. Let $V$ be a finite dimensional rational representation of $\gl(2,\C)$.
Decompose $V$ into eigenspaces for the action of $\mf h$ as before: \begin{align*}
    V = \bigoplus_{\chi \in \mf h^*} V_\chi
\end{align*} Now we know how $\mf h$ acts on $V$, we need to investigate the action of $e_1$ and $e_2$. As it turns out,
$e_1$ and $e_2$ are operators which translate between the weight spaces. 
Specifically, let $v$ be an weight vector for the action of $\mf h$ with weight $\chi$. 
Then $e_1\cdot v$ is a weight vector with weight $\chi + \alpha$. Indeed for $X\in \mf h$ we have (recall the action of the Lie algebra respects brackets) 
\begin{align*}
    X\cdot e_1v &= e_1 \cdot Xv + [X,e_1]\cdot v \\
    &= \chi(X)e_1v + \alpha(X)e_1v
\end{align*} 

A priori, we know nothing about the weights of $V$. Now we know that all of the weights of $V$ are translates of each other by the roots of 
$\gl(2,\C)$. Now let $\mu$ be any weight which appears in the decomposition of $V$. Then we can consider the transtates \begin{align*}
    \mu + \Z\alpha
\end{align*} and since $V$ is finite dimensional, only finitely many of the weight spaces of $V$ corresponding to these weights are nonzero. 
Recall we picked a postive system, so now it makes to talk about the highest weight (it is the weight $\chi$ so that all of the weights $\chi + \N\alpha$ 
correspond to empty weight spaces).

\hfill

If $V$ is an irreducible representation then a highest weight vector must span its root space. This is because
if $v$ is a highest weight vector, then one can show that the subspace generated by $v$, $e_2\cdot v$, $e_2^2\cdot v$, $\ldots$ is a subrepresentation.
It follows that an irreducible representation can have only one highest weight vector (up to scale).

\hfill

There are many facts which I omitted and many details that I have brushed over in this discussion. We encourage
the reader to look in \cite{fulton-harris} or \cite{fulton} for a more detailed discussion of the proof of the theorems of the highest weight, 
and more generally the classification of complex semisimple Lie algebras.


\begin{thebibliography}{9}

\bibitem{waterhouse}
Waterhouse, W. C. (1996). \textit{Introduction to Affine Group Schemes}. Springer.

\bibitem{milne}
Milne, J. S. (2017). \textit{Algebraic Groups: The Theory of Group Schemes of Finite Type over a Field}. Cambridge University Press.


\bibitem{fulton-harris}
Harris, J., \& Fulton, W. (1991). \textit{Representation theory: A first course}. Springer.

\bibitem{fulton}
Fulton, W. (1997). \textit{Young tableaux: with applications to representation theory and geometry}. Cambridge university press.

\end{thebibliography}
\end{document}