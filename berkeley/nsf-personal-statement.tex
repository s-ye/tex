\documentclass[11pt]{article}
\usepackage[margin=1in]{geometry}
\usepackage{mathptmx,amssymb,amsmath,enumitem,hyperref}
\usepackage[compact]{titlesec}
\linespread{0.96}
\setlength{\parskip}{0.4em}
\setlength{\parindent}{0pt}
\titlespacing*{\section}{0pt}{0.6em}{0.3em}

\begin{document}

\begin{center}
{\Large \textbf{Personal, Relevant Background, and Future Goals Statement}}\\[0.5em]
\end{center}
I first fell in love with mathematics in high school, long before I could articulate why. What drew me in was the sense of discovery---the way a simple pattern could hide something infinite behind it. I remember spending hours on geometry problems, drawn to the way a new idea could completely transform how I saw the same problem. Those small moments of clarity sparked a deeper curiosity: I wanted to understand where beauty in mathematics comes from.

As an undergraduate at Cornell University, I discovered that behind the beauty of mathematics lies an enormous amount of careful work. My courses were intense---each week packed with problem sets and projects---but I found myself going beyond them, digging through textbooks and trying to decode research papers. I majored in both mathematics and computer science, drawn to the way each discipline sharpened the other: programming gave me tools to make the abstract tangible, and mathematics gave structure to the way I thought about computation.

\section*{Intellectual Merit}

My first major research experience was the University of Maryland REU in Combinatorics and Algorithms, where my group developed visualization software for hyperbolic metrics. Translating abstract geometry into visual and computational form showed me how powerful explicit examples can be. Our work culminated in a presentation at the Canadian Conference on Computational Geometry in Montréal~\cite{cccg}, which was my first time presenting research to an international audience. \textbf{Standing beside my collaborators and sharing what we had built was deeply meaningful because I realized that mathematics is not only about individual insight, but also about communicating ideas that connect abstraction with the tangible.} This blend of abstraction and concreteness has since become a defining theme in my work.

Returning to Cornell, I wanted to understand the algebraic structures underlying geometry. In my sophomore year, I took Professor Allen Knutson's graduate algebraic geometry course, which fundamentally changed how I thought about mathematics. I was struck by how symmetry groups could describe ideas that initially seemed ungraspable. In algebraic geometry, algebraic invariants are often defined in highly abstract terms, yet for spaces with a large group of symmetries acting, they often have beautifully concrete descriptions. The same invariants that arise from advanced tools, such as Betti or sheaf cohomology, can sometimes be computed using simple combinatorial data coming from the underlying symmetry, like fans, polytopes, or Bruhat order. Discovering that deep algebraic information could be encoded so explicitly convinced me that symmetry is not only a feature of these spaces, but a key to making their abstract structure tangible.

That year, I began a research project with Professor Knutson, using computation to explore the combinatorial structure of Bruhat order in the stratification of Richardson varieties. Programming allowed me to make these abstract relationships visible and test small examples that revealed the underlying geometry. That summer, I also had the opportunity to attend several mathematics conferences, including \emph{Singularities} in Ann Arbor and \emph{Motivic Homotopy Theory} at the Park City Mathematics Institute in Utah. Experiencing such a wide spectrum of mathematical ideas, ranging from geometry to topology to number theory, was both humbling and energizing. I was fascinated by how different areas of mathematics, though seemingly distant, often shared a common language built on symmetry and geometry.

With the questions from that summer still fresh in my mind, I spent the fall strengthening the technical background needed for my research, and in the spring took a semester away from coursework to work with Professors Allen Knutson and Tara Holm on problems in toric topology and equivariant cohomology---areas that translate deep geometric ideas into explicit combinatorial models. Each week, we met to discuss my latest attempts to untangle the combinatorial structure behind moment graphs and GKM theory. The process was slow and full of detours. Many ideas failed, and I often had to start over from scratch. But those meetings became a rhythm of exploration: I would spend the week testing new approaches, sometimes coding small examples or reworking proofs by hand, and then return to our discussions ready to try again.

Through this cycle of failure and refinement, I learned that real mathematical progress is rarely linear. It requires patience, resilience, and the willingness to rebuild understanding from the ground up. Over time, I began to see how even the most abstract ideas could be made tangible when approached through explicit models and examples. This experience reshaped how I think about mathematics: \textbf{it is not a solitary pursuit of right answers, but a continuous dialogue between intuition, computation, and abstraction.}

That summer, I attended the Toric Topology conference in Poland, where I had the chance to present my ideas informally and learn from experts whose work connected geometry, combinatorics, and symmetry in ways that inspired me. It was an exceptional experience and deepened my commitment to pursuing mathematics as a lifelong research career.

In my senior year, I completed my undergraduate journey with a thesis under Professor Holm on equivariant cohomology and toric geometry~\cite{thesis}. I was honored to receive the Kieval Prize, recognizing the top undergraduates in mathematics at Cornell. \textbf{By then, I understood not only why I loved mathematics, but also how I wanted to contribute to it: by building explicit, computational perspectives on abstract structures.}

I am now a Ph.D.\ student in mathematics at the University of California, Berkeley. In my first year, I passed my preliminary exam and began exploring the representation theory of loop groups. This area, which sits at the intersection of algebraic geometry, topology, and mathematical physics, perfectly captures my fascination with the interplay between abstraction and explicit construction. I am currently in a one-on-one reading course with Professor Richard Borcherds, studying infinite-dimensional Lie algebras and exploring potential directions for new research.

\section*{Broader Impacts}

Beyond my research, I am deeply committed to mentorship and mathematical outreach. Throughout my journey, I have sought opportunities to emulate the professors and mentors who shaped my own development, learning that teaching is itself a form of discovery. As an undergraduate, I volunteered with the Big Red Math Competition, helping organize and run problem-solving sessions for high school students. I remember watching students encounter mathematics in a setting that valued creativity and persistence more than correctness, which reminded me of the curiosity that first drew me to the subject.

I also served as a teaching assistant for four semesters in Cornell’s computer science department, assisting in courses with enrollments of up to 400 students. The scale of those classes made it especially meaningful to work closely with students during office hours, helping them bridge the gap between abstract algorithms and concrete implementations. I found it rewarding to watch students overcome intellectual hurdles and, through these experiences, I developed a deeper appreciation for how teaching can reinforce research: explaining complex ideas with precision and empathy often sharpens one’s own thinking.

At Berkeley, I have continued to pursue outreach and mentorship through the Directed Reading Program, where I work one-on-one with undergraduates on self-chosen topics. Guiding students through their first taste of research has been one of the most rewarding parts of graduate school. \textbf{These experiences have convinced me that mentorship is not a side project but an integral part of being a mathematician.}

In the long term, I hope to extend this work by mentoring students through REUs and building inclusive research environments where undergraduates can engage directly with mathematical questions. I want to show that the same creativity and persistence that drive mathematical discovery also make it accessible to everyone. The NSF Graduate Research Fellowship will enable me to strengthen these efforts while continuing to grow as a researcher, educator, and mentor---someone who can help future students find the same excitement in mathematics that first drew me in.

\section*{Future Goals}

Looking forward, I aim to pursue a research career in representation theory and algebraic geometry. The process of earning a Ph.D.\ is, to me, the essential path toward mastering a field. It is a rare opportunity to develop the depth of understanding and independence required to contribute new and lasting ideas to mathematics. My research interests center on using explicit and computable models to make abstract theories more tangible, blending conceptual depth with computational experimentation. I hope to use my multifaceted technical background to bring clarity to ideas that are often considered opaque.

In the long term, my goal is to obtain a tenure-track position at a research university where I can balance research and teaching. I want to continue developing as both a scholar and an educator, creating a research program that unites rigorous theory with accessibility and mentorship. I also plan to make STEM education for underrepresented minorities a priority, whether through outreach initiatives within academia or broader community partnerships. As I have done since my undergraduate years, I will seek opportunities to collaborate with organizations that expand access to advanced mathematics and help students connect with the creative side of the discipline.

The NSF Graduate Research Fellowship will provide the freedom and support necessary to establish this foundation. It will allow me to take creative risks, pursue innovative research directions, and grow into a mathematician who contributes not only to theory, but also to the community that sustains it.

\begin{thebibliography}{9}
\bibitem{cccg} M.~Bumpus, C.~Dai, A.~Gezalyan, S.~Munoz, S.~Santhoshkumar, S.~Ye, and D.~Mount. Software and Analysis for Dynamic Voronoi Diagrams in the Hilbert Metric. In \textit{Proceedings of the Canadian Conference on Computational Geometry (CCCG)}, Montréal, 2023.

\bibitem{thesis} S.~Ye. Moment Maps and Equivariant Cohomology in Toric Geometry. Senior Thesis, Cornell University, 2024.
\end{thebibliography}

\end{document}