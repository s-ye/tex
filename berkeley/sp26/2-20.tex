
\documentclass[12pt]{article}
\usepackage[english]{babel}
\usepackage[utf8x]{inputenc}
\usepackage{tikz}
\usepackage{tikz-cd}
\usepackage{mathtools}
\usepackage[T1]{fontenc}
\usepackage{listings}
\usepackage{bookmark}


\makeatletter
\def\input@path{{../../style/}}
\makeatother

\usepackage{../../style/quiver}
\makeatletter
\def\input@path{{../../style/}}
\makeatother

\usepackage{../../style/scribe}
\usepackage{fancyhdr}

\usepackage{parskip}
\setlength{\parskip}{1em}
\setlength{\parindent}{0pt}


\newcommand{\fg}{\mathfrak g}

\DeclareMathOperator{\Frac}{Frac}
\begin{document}

\lhead{Songyu Ye}
\rhead{\today}
\cfoot{\thepage}

\title{Title}
\author{Songyu Ye}
\date{\today}
\maketitle

\begin{abstract}
    Abstract
\end{abstract}

\tableofcontents

\section{The setup}
Recall that we have a line bundle $\mathcal L:=\mathcal L_{\det}$ on $\cX:=\cX_{G,g,I}$ and an invariant norm $\|\cdot\|$ on $X_*(T)_\mathbb R$.

For a point $x\in \cX(k)$ and a nontrivial $f:\Theta\to \cX$ with $f(1)\simeq x$, define
$$\mu(x,f)\;:=\;\frac{\mathrm{wt}_{\mathcal L}(f)}{\|\lambda_f\|},$$
where $\mathrm{wt}_{\mathcal L}(f)$ is the $\mathbb G_m$-weight on the fiber of $\mathcal L$ at the special point $f(0)$, and $\lambda_f$ is the associated cocharacter data of $f$ coming from the action of $\mathbb G_m$ on the object $f(0)$.

We wanted to prove the following theorem.
\begin{theorem}[Need to prove]
    For any $x\in \cX_{G,g,I}(k)$, there exists a (unique up to something) maximally destabilizing $\Theta$-filtration $f_{\mathrm{HN}}:\Theta\to \cX_{G,g,I}$ of $x$ with respect to $\mathcal L_{\det}$ and any invariant norm on $X_*(T)_\R$.
\end{theorem}

Constantin remarked that the uniqueness part of the theorem is not important for our application, and that we can get away with existence of a maximally destabilizing $\Theta$-filtration for each $x\in \cX(k)$, provided that we can organize the points $x$ into strata in such a way which mirrors the setup of the Kirwan Ness stratification.

\section{Kirwan Ness stratification}
This section follows \cite{kirwan}.
Suppose we have linear action of a reductive group $G$
on a projective variety $X$, singular or nonsingular, defined over an algebraically closed field. Let $T \subset G$ be a maximal torus, $V$ the representation giving the linearization.
Pick an invariant norm on the cocharacter lattice of $G$. Then Kirwan and Ness construct a stratification of $X$ into locally closed subvarieties \begin{align*}
    X = \bigsqcup_{\beta \in \mathcal B} S_\beta,
\end{align*} \textbf{The index set $\mathcal B$ is in correspondence with connected components $Z \subset X^{\beta}$ of the fixed locus of a dominant cocharacter $\beta$ of $T$ such that the semistable locus $Z^\circ$ of the divided action of the Levi subgroup $L_\beta$ on $Z$ is nonempty.}

If the fixed loci is connected, then the index set $\mathcal B$ can be identified with the set of
$G$ conjugacy classes of rays in $X_*(G) \otimes \R$ together with the condition that the associated “center” $Z_\beta^{ss}$ is nonempty. Note that every cocharacter of $G$ is $G$-conjugate to a dominant cocharacter of $T$. It is harmless to choose a rational dominant cocharacter $\beta$ in each conjugacy class as they determine the same parabolic subgroup $P_\beta$ and the same stratum $S_\beta$. Note that one cannot simply group the components together under a single conjugacy class of rays, as each connected component $Z_i$ could possibly a different weight of the linearized line bundle.


Then we have a weight decomposition \begin{align*}
    V = \bigoplus_{\chi} V_\chi,
\end{align*}
and so for any point $x\in X$, we can write its homogeneous coordinates as $x = [v]$ for some $v\in V$ with $v = \sum_\chi v_\chi$. For a point $x$, let $W_x$ be the set of weights appearing in its support, meaning the set of $\chi$ such that $v_\chi \neq 0$. Then Kirwan identifies $\beta$ as the closest point to $0$ in $\mathrm{Conv}(W_x)$.

Alternatively, we can identify $\beta$ as the $G$-conjugacy class of ray which minimzes the following function on $X_*(G) \otimes \R$:
\begin{align*}
    \lambda \mapsto \frac{\mu(x,\lambda)}{\|\lambda\|}.
\end{align*}
where $\mu(x,\lambda)$ is defined as the minimum $\lambda$-weight of the nonzero coordinates of $x$.
\begin{align*}
\mu(x,\lambda)= \min_{\chi \in W_x}\langle \chi, \lambda \rangle
\end{align*}

For each $\beta\in \mathcal B$, pick a dominant rational representative of the corresponding conjugacy class of rays which we also denote by $\beta$. The unstable strata are indexed by those $Z$ with dominant $\beta$ for which
the semistable locus $Z^\circ \subset Z$ of the divided $L$-action on $\mathcal L$
is not empty. 

In particular we look at the fixed locus 
$X^\beta = \{ x \in X \mid \beta(t)\cdot x = x \ \forall t \}$ and take a connected component $Z \subset X^\beta$. Now define the attracting set:
\[Y=\{ x \in X \mid \lim_{t\to\infty} \lambda(t)\cdot x \in Z \}\] which gives us a natural map $\varphi: Y \to Z$ defined by the limit of the $T$-flow. 

\begin{proposition}
    $\varphi : Y \to Z$ is a locally trivial fibration in affine spaces. At a point $z\in Z$, the tangent space decomposes into weight spaces under $\beta$ as:
\begin{align*}
    T_zX = T_zZ \oplus (T_zX)_{>0}
\end{align*}
The positive-weight directions integrate to affine fibers.
\end{proposition}

Now we describe what $Z^\circ$ is. Let $L = Z_G(\beta)$
be the Levi subgroup. $\beta$ acts on the fiber of the linearized line bundle $\mathcal L$ over $Z$ by the character $\beta$.

To remove the destabilizing contribution, one twists the linearization by subtracting this character. After twisting, the central $\mathbb G_m$ coming from $\beta$ acts trivially, and so we are left with a genuine GIT problem for the Levi $L$ acting on $Z$. Let $Z^\circ$ be the semistable locus for this GIT problem, and put $Y^\circ = \varphi^{-1}(Z^\circ)$.

To recover the stratum $S_\beta$, we take the $G$-orbit of $Y^\circ$, i.e. $S = G \cdot Y^\circ$ and then one can show that we have the following isomorphism of $G$-varieties:
\[S \cong G \times^{P_{\beta}} Y^\circ\]
where we are dividing by the relation \begin{align*}
    (g, y) \sim (gp^{-1}, py) \quad \forall p \in P_\beta.
\end{align*}
So geometrically, we see $S \sim (G/P) \times (\text{affine}) \times Z^\circ$.

\begin{proposition}
To recap, in this setup, we have the following properties:
\begin{enumerate}
\item[(i)]
$Y$ is a fiber bundle over $Z$, with affine spaces as fibers, under the
morphism $\varphi$ defined by the limiting value of the $T$-flow.

\item[(ii)]
$Y$ is stabilized by the parabolic subgroup $P \subset G$ whose nilpotent
Lie algebra radical $\mathfrak u$ is spanned by the negative $T$-eigenspaces
in $\mathfrak g$.

\item[(iii)]
The $G$-orbit $S$ of $Y^\circ$ is isomorphic to $G \times^P Y^\circ$.
Under $\varphi$, it fibers in affine spaces over $G \times^P Z^\circ$,
if we let $P$ act on $Z^\circ$ via its reductive quotient $L$.

\item[(iv)]
The various $S$, together with $X^\circ = X^{ss}$, smoothly stratify $X$.

\item[(v)]
$Z^\circ$ has a projective, good quotient under $L$; $X^{ss}$ has a good projective quotient under $G$.
\end{enumerate}
\end{proposition}







Then there is a smooth locally closed subvariety $Y_\beta \subset X$ acted on by a parabolic subgroup $P_\beta$ of $G$ such that
\begin{equation}\label{eq:KN-stratum}
    S_\beta \cong G \times_{P_\beta} Y_\beta^{ss}.
\end{equation}

There is also a nonsingular closed subvariety $Z_\beta \subset X$
and a locally trivial fibration
\begin{equation}\label{eq:KN-fibration}
    P_\beta : Y_\beta^{ss} \longrightarrow Z_\beta^{ss},
\end{equation}
whose fibres are all affine spaces.
Here $Z_\beta^{ss}$ is the set of semistable points of $Z_\beta$
under the action of the Levi subgroup $L_\beta$ of $P_\beta$.

\section{Refernences}
Kirwan Cohomology of quotients in symplectic and algebraic geometry, 1984.
\end{document}