\documentclass[12pt]{article}
\usepackage[english]{babel}
\usepackage[utf8x]{inputenc}
\usepackage[T1]{fontenc}
\usepackage{listings}
\usepackage{bookmark}
\usepackage{tikz-cd}

\makeatletter
\def\input@path{{../../style/}}
\makeatother

\usepackage{../../style/quiver}
\makeatletter
\def\input@path{{../../style/}}
\makeatother

\usepackage{../../style/scribe}
\usepackage{fancyhdr}

\usepackage{parskip} % Automatically respects blank lines
\setlength{\parskip}{1em} % Adds more space between paragraphs
\setlength{\parindent}{0pt} % Removes paragraph indentation
\DeclareMathOperator{\Eul}{Eul}
\DeclareMathOperator{\Ind}{Ind}
\DeclareMathOperator{\pt}{pt}
\DeclareMathOperator{\sm}{sm}
\DeclareMathOperator{\Sing}{Sing}

\begin{document}


\lhead{Songyu Ye}
\rhead{\today}
\cfoot{\thepage}

\title{Teleman Woodward}

\author{Songyu Ye}
\date{\today}
\maketitle


\begin{abstract}
Abstract
\end{abstract}

\tableofcontents
\section{Overview of the finiteness theorem from Teleman Woodward}
Let $G$ be a connected reductive group and $\mf M = \Bun_G(C)$ the moduli stack of $G$-bundles on a smooth projective curve $C$. The goal of Teleman-Woodward is to compute the index of certain $K$-theory classes on $\mf M$, generalizing the Verlinde formula for line bundles.

However, our goal is to merely establish the finiteness of the index in the case of nodal curves. Abstracting from the paper, the finiteness theorem has the following structure:
\begin{enumerate}
    \item Stratify the full stack $\mathfrak{M}$ by Harder–Narasimhan type $\mathfrak{M} = \bigsqcup_\xi \mathfrak{M}_\xi$. \red{Include some explanation of HN type here.}
    \item Filter $R\Gamma(\mathfrak{M},E)$ by local cohomology along the closed unions $\bigcup_{\xi' \ge \xi} \mathfrak{M}_{\xi'}$.
    \item Show that admissibility forces the local terms for $\xi$ sufficiently large to vanish.
    \item Deduce that only finitely many strata contribute, hence the index is finite.
\end{enumerate}

\subsection*{Admissible classes (Teleman--Woodward)}

Let $\Sigma$ be a smooth projective curve, $G$ a connected reductive group, and
$\mf M=\Bun_G(\Sigma)$ the moduli stack of $G$--bundles.  Let $\mathcal E$ denote
the universal $G$--bundle on $\Sigma\times\mf M$, and for a finite--dimensional
representation $V$ of $G$ write $\mathcal E^*V$ for the associated vector bundle.

Teleman--Woodward single out the following natural $K$--theory classes on $\mf M$:

\begin{enumerate}[(i)]
\item $E_x^*V\in K^0(\mf M)$, the restriction of $\mathcal E^*V$ to $\{x\}\times\mf M$,
for a point $x\in\Sigma$;

\item $E_C^*V := \mathcal E^*V/[C]\in K^{-1}(\mf M)$, the slant product of $\mathcal E^*V$
with a $1$--cycle $C$ on $\Sigma$;

\item $E_\Sigma^*V := R\pi_*(\mathcal E^*V\otimes\sqrt K)\in K^0(\mf M)$, the Dirac index
bundle along $\Sigma$, where $\pi:\Sigma\times\mf M\to\mf M$ and $\sqrt K$ is a square
root of the canonical bundle of $\Sigma$;

\item $D_\Sigma V := \det^{-1}E_\Sigma^*V$, the inverse determinant of cohomology.
\end{enumerate}

The classes in (i)--(iii) are called the \emph{Atiyah--Bott generators}.  The classes
in (iv) are determinant line bundles on $\mf M$.

The first Chern class of any determinant line bundle
$\mathcal L$ defines an invariant quadratic form
\[
h_\mathcal L\in H^4(BG;\R)\cong \Sym^2(\mf g^*)^G \cong
\{\text{invariant symmetric bilinear forms on }\mf g\},
\]
called the \emph{level} of $\mathcal L$.  Let $c$ denote the distinguished quadratic
form corresponding to the canonical bundle
$\mathcal K=\det E_\Sigma^*\mf g$.

\begin{definition}[Teleman--Woodward]
A line bundle $\mathcal L$ on $\mf M$ is called \emph{admissible} if the shifted
quadratic form
\[
h_\mathcal L + c
\]
is positive definite on $\mf g$. An \emph{admissible class} in $K^*(\mf M)$ is any finite product of an admissible line
bundle with Atiyah--Bott generators.
\end{definition}

\subsection*{Shatz stratification}

Recall that any $G$--bundle over $\Sigma$ admits a canonical reduction of structure
group to a standard parabolic subgroup $P\subset G$, for which the associated
bundle with Levi structure group is semistable. 

\begin{remark}
    For a principal $G$-bundle $P$ on a smooth curve $\Sigma$, there is a Harder–Narasimhan (HN) theory generalizing the usual HN filtration of vector bundles. The outcome is a canonical reduction of $P$ to a parabolic subgroup $P\subset G$. “Canonical” means: determined functorially by $P$ (up to unique isomorphism), not a choice. The defining property is that if you pass from $P$ (the parabolic) to its Levi quotient $L=P/R_u(P)$, the induced $L$-bundle is semistable.
“Standard parabolic” means: a parabolic containing a fixed Borel $B$ (chosen once), so parabolics are indexed by subsets of simple roots.

Intuition: this parabolic reduction packages “the most destabilizing” subobject(s) of the bundle.
\end{remark}


Topologically, this reduction
is classified by a coweight of $P/[P,P]$; we identify this with a (possibly
fractional) dominant coweight $\xi$ of $\mf g$, called the \emph{instability type}
of the original bundle.  Then $P$ is the standard parabolic subgroup defined by
$\xi$; we denote it by $P_\xi$ and its Levi subgroup by $G_\xi$.  If $\mf M_\xi$
denotes the stack of $G$--bundles of type $\xi$, we have an algebraic
stratification 
\[
\mf M = \bigsqcup_\xi \mf M_\xi .
\]
Sending a $P_\xi$--bundle to its associated Levi bundle defines a morphism
from $\mf M_\xi$ to the stack $\mf M^{ss}_{G_\xi,\xi}$ of semistable principal
$G_\xi$--bundles of type $\xi$; the fibres are quotient stacks of affine spaces
by unipotent groups (equivalently the corresponding Lie algebra is nilpotent).  The virtual normal bundle for the morphism
$\mf M^{ss}_{G_\xi,\xi} \to \mf M$ is the complex
\[
\nu_\xi = R\pi_*\mathcal E^*(\mf g/\mf g_\xi)[1].
\]

Its $K$--theory Euler class should be the alternating sum of exterior powers
\[
\lambda_{-1}(\nu_\xi^\vee) := \sum_p (-1)^p \wedge^p(\nu_\xi^\vee),
\]
but for now this infinite sum is only a formal expression, whose meaning is to
be spelled out.

\subsection*{Local cohomology}

Finite open unions of Shatz strata
\[
\mf M_{\le \xi} = \bigcup_{\mu\le \xi} \mf M_\mu
\]
can be presented as quotient stacks of smooth quasi--projective varieties by
reductive groups.  The cohomology with supports over $\mf M_\xi$ of a vector
bundle $\mathcal E$ is
\[
H^\bullet_{\mf M_\xi}(\mf M_{\le \xi},\mathcal E_{\le \xi})
= H^{\bullet + d_\xi}(\mf M_\xi, \mathcal R_\xi \mathcal E),
\tag{1.9}
\]
where $d_\xi$ is the codimension of $\mf M_\xi$ and $\mathcal R_\xi \mathcal E$
denotes the sheaf of "$\mathcal E$--valued residues along $\mf M_\xi$." In particular \[\mathcal R_\xi \mathcal E \;:=\; i_\xi^! (\mathcal E_{\le \xi})[-d_\xi]\]

where \(i_\xi:\mf M_\xi\hookrightarrow \mf M_{\le \xi}\) is the inclusion and $i^!$ is the extraordinary pullback (local duality functor).

This is a stacky derived version of the fact that for a smooth closed subvariety $Z \subset X$, local cohomology along $Z$ equals cohomology on $Z$ twisted by the normal bundle and shifted by codimension.

\red{Basically I think we need to find the right stratification and Dan HL has a machine that produces such stratifications, known as $\theta$-stratifications. }

\subsection*{Role of the Shatz stratification in Teleman--Woodward}

The proof of the finiteness theorem in \cite{TelemanWoodward} is organized around
the Harder--Narasimhan (Shatz) stratification of the moduli stack
\[
\mf M=\Bun_G(\Sigma)=\bigsqcup_{\xi}\mf M_\xi,
\]
indexed by dominant rational coweights $\xi$.  This stratification plays the role
of a Morse stratification for the Yang--Mills functional, and replaces compactness
in the non--finite--type stack $\mf M$.

\medskip
\noindent\textbf{(1) Filtration by supports.}
The partial order on instability types defines an increasing filtration by open
substacks
\[
\mf M_{\le \xi} := \bigcup_{\mu\le\xi}\mf M_\mu.
\]
For any sheaf or complex $\mathcal E$ on \(\mf M\), this filtration produces a filtration on derived global sections
$R\Gamma(\mf M,\mathcal E)$
by the subcomplexes
$R\Gamma_{\mf M_{\le \xi_k}}(\mf M,\mathcal E)$.

\begin{remark}[General mechanism of local cohomology filtration]
    Suppose you have a space/stack $X$ and an increasing sequence of open substacks
$\emptyset = U_0 \subset U_1 \subset U_2 \subset \cdots \subset X$
with closed complements
$Z_k := X \setminus U_k$. In our situation \(X = \mf M\), \(U_k = \mf M_{\le \xi_k}\), \(Z_k = \bigcup_{\mu > \xi_k} \mf M_\mu\). For any sheaf or complex $\mathcal E$ on $X$, there is a canonical exact triangle\begin{align*}
R\Gamma_{Z_k}(X,\mathcal E)
\;\to\;
R\Gamma(X,\mathcal E)
\;\to\;
R\Gamma(U_k,\mathcal E|_{U_k})
\;\to\;
\end{align*}
This triangle is the definition of local cohomology with supports in $Z_k$.
\end{remark}


The decreasing family of closed substacks $Z_k=X\setminus U_k$ induces a
decreasing filtration $F^k:=R\Gamma_{Z_k}(X,\mathcal E)$ of $R\Gamma(X,\mathcal E)$.
Its successive graded pieces are
\[
\gr^k F \simeq R\Gamma_{Z_k\setminus Z_{k+1}}(U_{k+1},\mathcal E|_{U_{k+1}}).
\]
In the Shatz situation (refining the indexing so that $U_k=U_{k-1}\sqcup \mf M_{\xi_k}$),
this becomes
\[
\gr^{k-1}F \simeq R\Gamma_{\mf M_{\xi_k}}(\mf M_{\le\xi_k},\mathcal E_{\le\xi_k}).
\]
Equivalently there is a spectral sequence with
\[
E_1^{\xi,*} = R\Gamma_{\mf M_\xi}(\mf M_{\le\xi},\mathcal E_{\le\xi})
\quad\Longrightarrow\quad
R\Gamma(\mf M,\mathcal E).
\]

\medskip
\noindent\textbf{(2) Reduction to semistable Levi moduli.}
Each stratum $\mf M_\xi$ carries a canonical morphism
\[
\mf M_\xi \longrightarrow \mf M^{ss}_{G_\xi,\xi}
\]
to the moduli stack of semistable principal $G_\xi$--bundles of fixed topological
type, whose fibres are quotient stacks of affine spaces by unipotent groups.
This identifies $\mf M_\xi$ as a bundle of unstable directions over a semistable
core.

\medskip
\noindent\textbf{(3) Virtual normal complex.}
The stratification provides a uniform description of the virtual normal complex
of $\mf M^{ss}_{G_\xi,\xi}$ in $\mf M$:
\[
\nu_\xi = R\pi_*\mathcal E^*(\mf g/\mf g_\xi)[1].
\]
Consequently, local cohomology along $\mf M_\xi$ may be expressed formally as
\[
R\Gamma_{\mf M_\xi}(\mf M_{\le\xi},\mathcal E)
\simeq
R\Gamma\bigl(\mf M_\xi,
\mathcal E\otimes \lambda_{-1}(\nu_\xi^\vee)^{-1}\bigr),
\]
where $\lambda_{-1}(\nu_\xi^\vee)$ is the $K$--theoretic Euler class of the normal
complex.

\medskip
\noindent\textbf{(4) Weight decomposition and admissibility.}
The representation $\mf g/\mf g_\xi$ decomposes into positive $\xi$--weight
spaces.  This induces a natural grading on $\nu_\xi$ and hence on
$\lambda_{-1}(\nu_\xi^\vee)^{-1}$.  Admissibility of the twisting line bundle
forces all sufficiently unstable types $\xi$ to contribute only strictly negative
weights, so that the formal inverse Euler class becomes summable and the local
contributions vanish for $\xi$ sufficiently large.

\medskip
\noindent\textbf{(5) Finiteness.}
Since only finitely many instability types can contribute nontrivially, the
local--cohomology filtration terminates after finitely many steps.  This yields
the finiteness of the index.

\medskip
In summary, the Shatz stratification supplies a canonical filtration, a reduction
to semistable Levi moduli, and a uniform normal complex whose weight decomposition
is controlled by admissibility.  All finiteness statements in
\cite{TelemanWoodward} are ultimately consequences of this structure.


\section{Toy model: $G=\GL_2$ on a smooth curve}

Let $\Sigma$ be a smooth projective curve of genus $g\ge 2$ over $\C$, and let
\[
\mf M=\Bun_{\GL_2}(\Sigma)
\]
be the moduli stack of rank $2$ vector bundles on $\Sigma$.
We explain explicitly the Shatz stratification, the Levi description of strata,
the virtual normal complex, and the weight bookkeeping behind the
Teleman--Woodward finiteness mechanism in this case.

\subsection{Harder--Narasimhan type and Shatz strata}

Every $E\in \mf M$ admits a unique Harder--Narasimhan filtration
\[
0\subset L \subset E,
\qquad M:=E/L,
\]
where $L$ is a line subbundle of maximal slope.  Write
\[
\deg(L)=d_1,\qquad \deg(M)=d_2,\qquad m:=d_1-d_2\ge 0.
\]
Then $E$ is semistable iff $m=0$ (equivalently $d_1=d_2$).

The Shatz (HN) stratum of type $(d_1,d_2)$ is the locally closed substack
\[
\mf M_{d_1,d_2}\subset \mf M
\]
parametrizing bundles whose HN filtration has graded pieces $(L,M)$ of degrees
$(d_1,d_2)$ (so $m>0$ on unstable strata).  One has the stratification
\[
\mf M=\bigsqcup_{d_1\ge d_2}\mf M_{d_1,d_2}.
\]

\subsection{Parabolic and Levi}

Fix the standard Borel $B\subset \GL_2$ of upper triangular matrices.
For $m>0$, the destabilizing reduction is to the standard parabolic
\[
P=\left\{\begin{pmatrix} * & * \\ 0 & * \end{pmatrix}\right\},
\]
with Levi subgroup
\[
G_\xi \cong \GL_1\times \GL_1
\quad\text{(diagonal matrices).}
\]
Equivalently, the associated dominant coweight (instability type) may be taken as
\[
\xi=\left(\frac m2,-\frac m2\right)\in \mf t_\Q,
\]
so that the positive root $\alpha$ satisfies $\alpha(\xi)=m$.

\subsection{Semistable Levi core and structure of the stratum}

A principal $G_\xi$--bundle is the same as a pair of line bundles $(L,M)$, hence
the moduli stack of semistable $G_\xi$--bundles of type $(d_1,d_2)$ is
\[
\mf M^{ss}_{G_\xi,\xi}\;\cong\;\Pic^{d_1}(\Sigma)\times \Pic^{d_2}(\Sigma).
\]
There is a canonical morphism
\[
q:\mf M_{d_1,d_2}\longrightarrow \Pic^{d_1}(\Sigma)\times \Pic^{d_2}(\Sigma),
\qquad E\mapsto (L,E/L).
\]
Fixing $(L,M)$, the fibre of $q$ over $(L,M)$ classifies extensions
\[
0\to L\to E\to M\to 0,
\]
hence is governed by
\[
\Ext^1(M,L)\cong H^1(\Sigma,L\otimes M^{-1}).
\]
Automorphisms of a given extension (fixing $(L,M)$) come from
\[
\Hom(M,L)\cong H^0(\Sigma,L\otimes M^{-1}),
\]
which is a unipotent group (additively) acting on the affine space
$H^1(\Sigma,L\otimes M^{-1})$ by the usual change-of-splitting.
Thus the fibre is a quotient stack
$\Bigl[H^1(\Sigma,L\otimes M^{-1})\Big/\;H^0(\Sigma,L\otimes M^{-1})\Bigr]$
making $\mf M_{d_1,d_2}$ a (stacky) affine fibration over the semistable Levi core.

\subsection{Virtual normal complex}

The tangent complex of $\Bun_G$ at a $G$--bundle $E$ is
\[
T_{\mf M,E}\simeq R\Gamma(\Sigma,\End(E))[1].
\]
Over the Levi core $(L,M)$ the adjoint representation decomposes as
\[
\End(L\oplus M)
\;=\;
\underbrace{\End(L)\oplus\End(M)}_{\mf g_\xi}
\;\oplus\;
\underbrace{\Hom(M,L)\oplus \Hom(L,M)}_{\mf g/\mf g_\xi}.
\]
Along the stratum $\mf M_{d_1,d_2}$, the relevant (unstable) normal directions
are governed by the positive--weight root space $\Hom(M,L)$, and the virtual normal
complex for the inclusion of the Levi moduli into $\mf M$ is
\[
\nu_\xi \;\simeq\; R\Gamma(\Sigma, L\otimes M^{-1})[1].
\]
Equivalently, $\nu_\xi$ has cohomology
\[
H^{-1}(\nu_\xi)\cong H^0(\Sigma,L\otimes M^{-1}),
\qquad
H^0(\nu_\xi)\cong H^1(\Sigma,L\otimes M^{-1}).
\]

\subsection{$K$--theoretic Euler class and its formal inverse}

Formally, the $K$--theory Euler class of the dual normal complex is
\[
\lambda_{-1}(\nu_\xi^\vee)
=\sum_{p\ge 0}(-1)^p\wedge^p(\nu_\xi^\vee).
\]
Because $\nu_\xi$ is a shifted cohomology complex, its inverse Euler class
expands into exterior powers of the $H^0$--piece and symmetric powers of the
$H^1$--piece.  Schematically one may think of
\[
\lambda_{-1}(\nu_\xi^\vee)^{-1}
\sim
\frac{\Sym^\bullet\bigl(H^1(\Sigma,L\otimes M^{-1})^\vee\bigr)}
{\Lambda^\bullet\bigl(H^0(\Sigma,L\otimes M^{-1})^\vee\bigr)}
\]
an infinite sum in ordinary $K$--theory which is made meaningful in
Teleman--Woodward by working in a suitable completion determined by $\xi$--weights.

\subsection{Weight bookkeeping: linear vs.\ quadratic growth}

The one--parameter subgroup $\xi$ acts on $\Hom(M,L)$ with weight $\alpha(\xi)=m$.
Consequently, $\xi$ acts on $H^i(\Sigma,L\otimes M^{-1})$ with weight $m$, and hence
on the graded summand
\[
\Sym^p\bigl(H^1(\Sigma,L\otimes M^{-1})^\vee\bigr)
\]
with weight $p\,m$.  This is the \emph{linear} growth in the instability parameter $m$.

On the other hand, a determinant line bundle $\mathcal L$ on $\mf M$ has a
level $h_{\mathcal L}\in \Sym^2(\mf g^*)^G$, and Teleman--Woodward introduce the
canonical correction $c$ coming from $\mathcal K=\det E_\Sigma^*\mf g$.
For an \emph{admissible} $\mathcal L$, the form $h_{\mathcal L}+c$ is positive
definite, so
\[
(h_{\mathcal L}+c)(\xi,\xi)\to +\infty \quad\text{as } \|\xi\|\to\infty.
\]
In the $\GL_2$ normalization $\xi=(m/2,-m/2)$ and the standard invariant form
$(X,Y)=\tr(XY)$ on diagonal matrices gives
\[
(\xi,\xi)=\frac{m^2}{2},
\]
so $(h_{\mathcal L}+c)(\xi,\xi)$ grows like a positive constant times $m^2$.
In Teleman--Woodward's local cohomology calculation, twisting by $\mathcal L$
shifts the $\xi$--weight spectrum by a \emph{negative} amount with leading term
\[
-(h_{\mathcal L}+c)(\xi,\xi)\sim -\kappa m^2 \qquad (\kappa>0).
\]

Thus, on the $\xi$--stratum, the inverse Euler class contributes graded pieces with
weights growing at most \emph{linearly} in $m$ (e.g.\ $p\,m$), while an admissible
twist shifts weights by a \emph{quadratic} negative term $\sim-\kappa m^2$.
This is the mechanism behind the eventual vanishing of sufficiently unstable
strata in the Teleman--Woodward finiteness theorem.

\subsection*{Why $\xi$--invariants control finiteness of the index}

Let $\Sigma$ be a smooth projective curve and $\mf M=\Bun_G(\Sigma)$.
For a class $\mathcal E\in K^*(\mf M)$ one defines its index by the Euler
characteristic
\[
\Ind(\mf M,\mathcal E)\;:=\;\chi(\mf M,\mathcal E)
\;=\;\sum_i(-1)^i\dim H^i(\mf M,\mathcal E),
\]
whenever the right-hand side is finite.

Since $\mf M$ is not of finite type, finiteness is proved by filtering
$\mf M$ by finite-type open substacks using the Shatz stratification
\[
\mf M=\bigsqcup_{\xi}\mf M_\xi,
\qquad
\mf M_{\le \xi}:=\bigcup_{\mu\le\xi}\mf M_\mu.
\]
The open substacks $\mf M_{\le\xi}$ form an increasing filtration of $\mf M$,
and for any complex $\mathcal E$ on $\mf M$ this induces a filtration of
$R\Gamma(\mf M,\mathcal E)$ by local cohomology with supports in the complements.
Equivalently, there is a spectral sequence whose $E_1$--page is built from the
local cohomology complexes
\[
E_1^{\xi,*}\;=\;R\Gamma_{\mf M_\xi}(\mf M_{\le\xi},\mathcal E_{\le\xi})
\quad\Longrightarrow\quad
R\Gamma(\mf M,\mathcal E).
\]
In particular, finiteness of $\Ind(\mf M,\mathcal E)$ follows once one knows:
\begin{enumerate}[(a)]
\item for each $\xi$, the contribution of $R\Gamma_{\mf M_\xi}(\mf M_{\le\xi},\mathcal E_{\le\xi})$
to Euler characteristic is finite-dimensional; and
\item all but finitely many $\xi$ contribute trivially.
\end{enumerate}

Teleman--Woodward identify each local term by a purity/local-duality statement:
\[
R\Gamma_{\mf M_\xi}(\mf M_{\le\xi},\mathcal E_{\le\xi})
\simeq
R\Gamma\bigl(\mf M_\xi,\mathcal R_\xi\mathcal E\bigr)[d_\xi],
\qquad
\mathcal R_\xi\mathcal E:= i_\xi^!(\mathcal E_{\le\xi})[-d_\xi],
\]
where $d_\xi=\codim(\mf M_\xi,\mf M_{\le\xi})$ and $i_\xi:\mf M_\xi\hookrightarrow\mf M_{\le\xi}$
is the inclusion.


Moreover, the residue object $\mathcal R_\xi\mathcal E$ may be expressed
formally in terms of the virtual normal complex
\[
\nu_\xi=R\pi_*\mathcal E^*(\mf g/\mf g_\xi)[1]
\]
as
\[
\mathcal R_\xi\mathcal E\;\sim\;
\mathcal E|_{\mf M_\xi}\otimes \Eul(\nu_\xi)^{-1}.
\]

\begin{remark}
Let $X$ be an algebraic stack, let $i:Z\hookrightarrow X$ be a closed immersion,
and let $\mathcal E$ be a (bounded) complex of coherent sheaves on $X$.
Recall that the \emph{local cohomology} of $\mathcal E$ with supports in $Z$ is
defined by the exact triangle
\[
R\Gamma_Z(X,\mathcal E)\;\longrightarrow\;R\Gamma(X,\mathcal E)
\;\longrightarrow\;R\Gamma(X\setminus Z,\mathcal E|_{X\setminus Z})\;\longrightarrow,
\]
or, equivalently, by the functor of sections with supports
$R\Gamma_Z(X,-)=R\Gamma(X,R\Gamma_Z(-))$.
Local duality packages these groups as ordinary cohomology on $Z$:
one defines the \emph{residue object} along $Z$ by
\[
\mathcal R_Z(\mathcal E)\;:=\; i^!(\mathcal E)[-\codim(Z)],
\]
so that (under standard hypotheses, e.g.\ $i$ a local complete intersection)
\[
R\Gamma_Z(X,\mathcal E)\;\simeq\;R\Gamma\bigl(Z,\mathcal R_Z(\mathcal E)\bigr)[\codim(Z)].
\]
In Teleman--Woodward's setting one takes $X=\mf M_{\le\xi}$ and $Z=\mf M_\xi$, so
\[
\mathcal R_\xi\mathcal E := i_\xi^!(\mathcal E_{\le\xi})[-d_\xi],
\qquad d_\xi=\codim(\mf M_\xi,\mf M_{\le\xi}).
\]

\smallskip
\noindent\textbf{Why an Euler factor appears.}
If $i:Z\hookrightarrow X$ is a \emph{regular embedding} between smooth schemes
with normal bundle $N_{Z/X}$, then $i^!$ is controlled by the normal directions.
At the level of $K$--theory one has the standard identity
\begin{equation}
\label{eq:regular-embed-euler}
\bigl[i^!(\mathcal E)\bigr]
=
\bigl[i^*(\mathcal E)\bigr]\cdot \lambda_{-1}(N_{Z/X}^\vee)^{-1}
\qquad\text{in }K^0(Z),
\end{equation}
where
\[
\lambda_{-1}(W):=\sum_{p\ge 0}(-1)^p[\wedge^p W]
\]
is the $K$--theoretic Euler class.
Heuristically, local cohomology measures ``principal parts along $Z$'',
and principal parts are obtained by expanding in the normal directions; the
inverse Euler class $\lambda_{-1}(N_{Z/X}^\vee)^{-1}$ is the $K$--theoretic avatar
of this expansion.

\smallskip
\noindent\textbf{Virtual normal complex.}
In the Shatz situation, $\mf M_\xi\hookrightarrow \mf M_{\le\xi}$ is not presented
globally as a single regular embedding into a smooth ambient space.  Instead,
Teleman--Woodward use the fact that $\mf M_\xi$ maps to a finite--type semistable
Levi stack $\mf M^{ss}_{G_\xi,\xi}$ with fibres quotient stacks of affine spaces
by unipotent groups, and there is a canonical \emph{virtual normal complex}
(perfect complex playing the role of $N_{Z/X}$)
\[
\nu_\xi \;=\; R\pi_*\mathcal E^*(\mf g/\mf g_\xi)[1]
\qquad\text{on }\mf M^{ss}_{G_\xi,\xi},
\]
whose pullback to $\mf M_\xi$ controls the unstable directions transverse to the
Levi moduli.  Consequently, the $K$--theory Euler class is defined by
\[
\Eul(\nu_\xi^\vee)\;:=\;\lambda_{-1}(\nu_\xi^\vee),
\]
and the same formal identity as \eqref{eq:regular-embed-euler} holds with
$N_{Z/X}$ replaced by $\nu_\xi$:
\begin{equation}
\label{eq:virtual-euler}
\bigl[\mathcal R_\xi\mathcal E\bigr]
=
\bigl[\mathcal E|_{\mf M_\xi}\bigr]\cdot \lambda_{-1}(\nu_\xi^\vee)^{-1}.
\end{equation}
\end{remark}



\subsubsection{The polarized inverse Euler class and the formula for $\Eul(\nu_\xi)^{-1}_+$}

We explain the origin of Teleman--Woodward's formula
\[
\Eul(\nu_\xi)^{-1}_+
:=
\Sym\Bigl(
R\pi_*\mathcal E^*(\mf p_\xi/\mf g_\xi)[1]^\vee
\ \oplus\
R\pi_*\mathcal E^*(\mf g/\mf p_\xi)[1]
\Bigr)\ \otimes\
\det\Bigl(R\pi_*\mathcal E^*(\mf g/\mf p_\xi)[1]\Bigr)[d_\xi],
\tag{$\ast$}
\]
It is a formal inverse to the Euler class with a weight decomposition satisfing the key properties that only weights $\le 0$ occur and each weight space is finite.

\medskip
\noindent\textbf{(1) $\xi$--weights and the parabolic splitting.}
Fix a maximal torus $T\subset G$ and a Borel $B\supset T$.
For a dominant rational coweight $\xi\in X_*(T)\otimes\Q$, let
$P_\xi$ be the associated \emph{standard} parabolic subgroup (so $B\subset P_\xi$),
and let $G_\xi$ be its Levi subgroup.  At the Lie algebra level one has a canonical
$\xi$--weight decomposition
\[
\mf g = \mf g_\xi \oplus \mf n_\xi \oplus \mf n_\xi^-,
\qquad
\mf n_\xi=\bigoplus_{\langle\alpha,\xi\rangle>0}\mf g_\alpha,
\quad
\mf n_\xi^-=\bigoplus_{\langle\alpha,\xi\rangle<0}\mf g_\alpha.
\]
Equivalently,
\[
\mf p_\xi=\mf g_\xi\oplus \mf n_\xi,
\qquad
\mf g/\mf g_\xi \cong (\mf p_\xi/\mf g_\xi)\oplus (\mf g/\mf p_\xi),
\]
where $\mf p_\xi/\mf g_\xi\cong \mf n_\xi$ carries strictly \emph{positive} $\xi$--weights
and $\mf g/\mf p_\xi\cong \mf n_\xi^-$ carries strictly \emph{negative} $\xi$--weights.

\medskip
\noindent\textbf{(2) The virtual normal complex and its $\xi$--grading.}
Let $\mf M_\xi$ be the Shatz stratum of instability type $\xi$ and let
$\mf M^{ss}_{G_\xi,\xi}$ be the semistable Levi moduli.

Sending a $P_\xi$--bundle to its associated Levi bundle defines a morphism
$q_\xi:\mf M_\xi\to \mf M^{ss}_{G_\xi,\xi}$.
\red{The fibres are quotient stacks of affine spaces by unipotent groups. Whenever we define our stratification, we need to make sure this is true.}
The deformation theory transverse to the Levi directions is governed by the
perfect complex on $\mf M^{ss}_{G_\xi,\xi}$
\[
\nu_\xi:= R\pi_*\mathcal E^*(\mf g/\mf g_\xi)[1],
\]
\red{How did Teleman-Woodward identify this? Is there some general theory?}
whose pullback along $q_\xi$ controls the virtual normal directions along the stratum and 
where \[\pi:\Sigma\times \mf M^{ss}_{G_\xi,\xi}\to \mf M^{ss}_{G_\xi,\xi}\] and
$\mathcal E$ is the universal bundle.
Using the splitting above,
\[
\nu_\xi \simeq \nu_\xi^+ \oplus \nu_\xi^-,
\qquad
\nu_\xi^+ := R\pi_*\mathcal E^*(\mf p_\xi/\mf g_\xi)[1],
\quad
\nu_\xi^- := R\pi_*\mathcal E^*(\mf g/\mf p_\xi)[1].
\]
Thus $\nu_\xi$ carries a canonical $\Z$--grading by $\xi$--weights: $\nu_\xi^+$ has
strictly positive weights and $\nu_\xi^-$ has strictly negative weights.

\medskip
\noindent\textbf{(3) Why an ``inverse Euler class'' is not a genuine $K$--class.}
For a vector bundle $W$, the $K$--theoretic Euler class is
\[
\lambda_{-1}(W^\vee)=\sum_{p\ge 0}(-1)^p[\wedge^p W^\vee].
\]
Even for a line bundle $L$, the inverse of $1-L^\vee$ is \emph{not} a finite $K$--class:
\[
(1-L^\vee)^{-1} = \sum_{n\ge 0}(L^\vee)^n
\qquad\text{(a formal geometric series).}
\]
Teleman--Woodward therefore work in a \emph{completion} of equivariant $K$--theory
determined by the $\xi$--weights.  In such a completion one is allowed to expand
$1-L^\vee$ as a geometric series \emph{in whichever direction is convergent in the chosen
completion}.  This is the meaning of the phrase ``prefers $\xi$--negative eigenvalues.''

\medskip
\noindent\textbf{(4) The basic one--dimensional identity and the determinant correction.}
Let $\Gm$ act on a one--dimensional representation of weight $w\neq 0$, so the character is $t^w$.
Then
\[
(1-t^w)^{-1}
=
\sum_{n\ge 0} t^{nw}
\qquad\text{as a formal series in the direction of weights }w,w,2w,\dots.
\]
If we instead want an expansion which involves only \emph{nonpositive} weights (i.e.\ which
``prefers negative eigenvalues''), we rewrite
\[
(1-t^w)^{-1} = -t^{-w}\,(1-t^{-w})^{-1}
= -t^{-w}\sum_{n\ge 0} t^{-nw}.
\]
Compared to the naive geometric series, this introduces a prefactor $-t^{-w}$.
In higher rank, multiplying these prefactors over all negative--weight lines produces a
\emph{determinant factor}. 

\medskip
\noindent\textbf{(5) From weights to a polarized inverse for a split complex.}
Suppose a perfect complex $K$ carries a $\xi$--grading and splits as
\[
K \simeq K^+ \oplus K^-,
\]
where all $\xi$--weights in $K^+$ are $>0$ and all $\xi$--weights in $K^-$ are $<0$.
Then \[\lambda_{-1}(K^\vee)=\lambda_{-1}((K^+)^\vee)\cdot \lambda_{-1}((K^-)^\vee)\]
and one defines a \emph{polarized inverse} $\lambda_{-1}(K^\vee)^{-1}_+$ by inverting
each factor in the completion which expands in the direction of \emph{$\xi$--negative weights}.
The outcome is the standard schematic identity
\begin{equation}
\label{eq:polarized-euler}
\lambda_{-1}(K^\vee)^{-1}_+
\;=\;
\Sym\bigl((K^+)^\vee \oplus K^-\bigr)\ \otimes\ \det(K^-)\,[\mathrm{shift}],
\end{equation}
where:
\begin{itemize}
\item $\Sym(-)$ denotes the total symmetric algebra
$\Sym^\bullet(-)=\bigoplus_{n\ge 0}\Sym^n(-)$, interpreted in $K$--theory (or in the
corresponding completed $K$--group) as a formal power series;
\item $\det(K^-)$ is the determinant line of the perfect complex $K^-$, which precisely
packages the product of the one--dimensional prefactors in \textbf{(4)};
\item $[\mathrm{shift}]$ is the cohomological degree shift dictated by local duality/purity
(and in the Shatz situation becomes $[d_\xi]$).
\end{itemize}


\begin{remark}[Origin of the determinant factor in the polarized inverse]
\label{rem:det-factor}
Let $\Gm$ act with a $\Z$--grading, and let $W$ be a finite--rank $\Gm$--equivariant
vector bundle with \emph{strictly negative} weights.
Write the $K$--theoretic Euler class as
\[
\lambda_{-1}(W^\vee)=\prod_{i}(1-L_i^\vee),
\]
after splitting $W=\bigoplus_i L_i$ into $\Gm$--eigenlines (locally on the base).
Formally,
\[
(1-L_i^\vee)^{-1}=\sum_{n\ge 0}(L_i^\vee)^n
\]
is the geometric expansion in nonnegative powers of $L_i^\vee$.
However, if $L_i$ has \emph{negative} $\xi$--weight, then $L_i^\vee$ has \emph{positive}
weight, so this expansion lives in the completion which prefers \emph{positive} weights.
To invert in the opposite completion (the one preferring negative weights), we rewrite
\[
(1-L_i^\vee)^{-1}
= -\,L_i\cdot (1-L_i)^{-1}
= -\,L_i\sum_{n\ge 0} L_i^{n},
\]
which is now a series in nonnegative powers of $L_i$ (hence in nonpositive weights).
The price paid for using this expansion is the prefactor $(-L_i)$.

Multiplying over $i$ gives
\[
\lambda_{-1}(W^\vee)^{-1}\Big|_{\text{prefer negative}}
=
\Bigl(\prod_i (-L_i)\Bigr)\cdot \prod_i (1-L_i)^{-1}
=
(-1)^{\rank W}\,\det(W)\cdot \Sym(W),
\]
where $\Sym(W):=\bigoplus_{n\ge 0}\Sym^n(W)$.
Up to the harmless sign $(-1)^{\rank W}$ (often suppressed in $K$--theory conventions),
this explains the appearance of the factor $\det(W)$ in the polarized inverse.

For a perfect complex $K^-$ of strictly negative weights, the same argument applied
to any local splitting into graded line bundles (together with the multiplicativity
of $\lambda_{-1}$ in $K$--theory) produces the factor $\det(K^-)$ in the polarized inverse
$\lambda_{-1}(K^\vee)^{-1}_+$.
\end{remark}

\noindent\textbf{(6) Specialization to $\nu_\xi$.}
Apply \eqref{eq:polarized-euler} to $K=\nu_\xi$ and the splitting
$\nu_\xi\simeq \nu_\xi^+\oplus \nu_\xi^-$ from \textbf{(2)}.
Then $(K^+)^\vee=(\nu_\xi^+)^\vee$ and $K^-=\nu_\xi^-$, and we obtain
\[
\Eul(\nu_\xi)^{-1}_+
:=
\lambda_{-1}(\nu_\xi^\vee)^{-1}_+
=
\Sym\bigl((\nu_\xi^+)^\vee \oplus \nu_\xi^-\bigr)\ \otimes\ \det(\nu_\xi^-)\,[d_\xi].
\]
Unwinding the definitions of $\nu_\xi^\pm$ gives exactly the formula $(\ast)$ above.

\subsubsection*{Weight bookkeeping and the finiteness mechanism}

Fix an instability type $\xi$ and consider the local contribution supported on the Shatz stratum
$\mf M_\xi$.  Teleman--Woodward control this contribution by analysing the
$\xi$--weight decomposition of the \emph{polarized inverse Euler factor}
$\Eul(\nu_\xi)^{-1}_+$ and its tensor product with an admissible class
$\mathcal E$.

\medskip
\noindent\textbf{(A) The determinant weight and the quadratic form $c(\xi,\xi)$.}
Recall the polarized inverse Euler factor (cf.\ \cite[\S1.10--1.11]{TelemanWoodward})
\begin{equation}
\label{eq:polarized-eul}
\Eul(\nu_\xi)^{-1}_+
:=
\Sym\Bigl(
R\pi_*\mathcal E^*(\mf p_\xi/\mf g_\xi)[1]^\vee
\ \oplus\
R\pi_*\mathcal E^*(\mf g/\mf p_\xi)[1]
\Bigr)\ \otimes\
\det\Bigl(R\pi_*\mathcal E^*(\mf g/\mf p_\xi)[1]\Bigr)[d_\xi].
\tag{$\ast$}
\end{equation}
The second tensor factor is a determinant line bundle
\[
\det\Bigl(R\pi_*\mathcal E^*(\mf g/\mf p_\xi)[1]\Bigr).
\]
Because $\mf g/\mf p_\xi$ is a direct sum of root spaces $\mf g_\alpha$ with
$\langle\alpha,\xi\rangle<0$, the one--parameter subgroup determined by $\xi$
acts on $\mf g/\mf p_\xi$ with strictly \emph{negative} weights.  Consequently,
the induced $\Gm$--action on the determinant line above has a well-defined
$\xi$--weight which may be computed as a signed sum of these negative integers,
counted with the appropriate cohomological multiplicities coming from
$R\pi_*$.

Teleman--Woodward package this total determinant weight by a distinguished
invariant quadratic form
\[
c\in \Sym^2(\mf g^*)^G,
\]
namely the quadratic form attached (via Grothendieck--Riemann--Roch) to the
canonical bundle
\[
\mathcal K := \det(E_\Sigma^*\mf g)
\qquad\text{on }\mf M.
\]
With this notation, the determinant factor in \eqref{eq:polarized-eul} has
$\xi$--weight
\begin{equation}
\label{eq:weight-is-c}
\wt_\xi\Bigl(\det(R\pi_*\mathcal E^*(\mf g/\mf p_\xi)[1])\Bigr)=c(\xi,\xi).
\end{equation}
In Teleman--Woodward's conventions, $c(\xi,\xi)$ is \emph{negative} when $\xi$
is viewed in the compact real form $i\mf t_k$; equivalently, $c$ is negative
definite on $i\mf t_k$.


\begin{proof}[Justification of \eqref{eq:weight-is-c}]
Fix $\xi$ and let $\lambda_\xi:\Gm\to G$ be the canonical one--parameter subgroup.
Consider the determinant line
\[
\mathcal D_\xi = \det \Bigl(R\pi_*\mathcal E^*(\mf g/\mf p_\xi)[1]\Bigr)
\]
which appears in the polarized inverse Euler class $\Eul(\nu_\xi)^{-1}_+$.

As a $T$--module (and hence as a $\lambda_\xi$--module),
\[
\mf g/\mf p_\xi\;\cong\;\bigoplus_{\langle\alpha,\xi\rangle<0}\mf g_\alpha,
\]
a direct sum of root spaces on which $\lambda_\xi$ acts with weights
$\langle\alpha,\xi\rangle<0$.
The corresponding trace form is
\[
\Tr_{\mf g/\mf p_\xi}(\eta,\eta)
\;=\;\sum_{\langle\alpha,\xi\rangle<0}\langle\alpha,\eta\rangle^2
\qquad (\eta\in \mf t),
\]
because $\eta$ acts on $\mf g_\alpha$ by the scalar $\langle\alpha,\eta\rangle$.

Now compare with the adjoint trace form:
\[
\Tr_{\mf g}(\eta,\eta)\;=\;\sum_{\alpha\in \Phi}\langle\alpha,\eta\rangle^2,
\]
since $\eta$ acts trivially on $\mf t$ and by $\langle\alpha,\eta\rangle$ on $\mf g_\alpha$.
The root system is symmetric $\alpha\leftrightarrow -\alpha$, so the sum over
$\{\alpha:\langle\alpha,\xi\rangle<0\}$ is exactly half the sum over all roots:
\[
\Tr_{\mf g/\mf p_\xi}(\eta,\eta)\;=\;\frac12\,\Tr_{\mf g}(\eta,\eta).
\]
Therefore the level of $\mathcal D_\xi$ is
\[
\mathrm{lev}(\mathcal D_\xi)
\;=\;\Tr_{\mf g/\mf p_\xi}
\;=\;\frac12\,\Tr_{\mf g}.
\]

There is the classical indentification of the level of the Pfaffian square root $\mathcal K^{-1/2}$
\[
c\;:=\;-\frac12\,\Tr_{\mf g},
\]
where $\mathcal K=\det(E_\Sigma^*\mf g)$ is the canonical bundle on $\mf M$.
Combining with the computation above gives
\[
\mathrm{lev}(\mathcal D_\xi) = -c.
\]
However the shift $[1]$ in the determinant line $\mathcal D_\xi$: \begin{align*}
    \det \Bigl(R\pi_*\mathcal E^*(\mf g/\mf p_\xi)[1]\Bigr) = \det \Bigl(R\pi_*\mathcal E^*(\mf g/\mf p_\xi)\Bigr)^{-1}
\end{align*} leaves us with $c$ as desired.
\end{proof}

\medskip
\noindent\textbf{(B) Tensoring by an admissible class $\mathcal E$.}
Lemma \cite[\S1.11]{TelemanWoodward} concerns the $\xi$--invariant part of
\[
\mathcal E\otimes \Eul(\nu_\xi)^{-1}_+,
\qquad
\bigl(\mathcal E\otimes \Eul(\nu_\xi)^{-1}_+\bigr)^{\xi\text{-inv}},
\]
i.e.\ the weight--$0$ subobject for the $\Gm$--action defined by $\xi$.

Write $\mathcal E$ as a product
\[
\mathcal E \;=\; \mathcal L\ \otimes\ (\text{Atiyah--Bott generators}),
\]
where $\mathcal L$ is a determinant line bundle and the remaining factors are
built from the Atiyah--Bott generators $E_x^*V$, $E_C^*V$, $E_\Sigma^*V$.

\medskip
\noindent\textbf{(B1) Quadratic shift from $\mathcal L$.}
By Grothendieck--Riemann--Roch, the first Chern class of $\mathcal L$ determines
an invariant quadratic form
\[
h=h_{\mathcal L}\in \Sym^2(\mf g^*)^G,
\]
called the \emph{level} of $\mathcal L$.  Teleman--Woodward's GRR calculation
shows that, on the $\xi$--stratum, the $\xi$--weight contributed by $\mathcal L$
has leading behaviour controlled by this level:
\begin{equation}
\label{eq:L-weight}
\wt_\xi(\mathcal L)\sim h(\xi,\xi),
\qquad\text{quadratic in }\xi.
\end{equation}
Combining \eqref{eq:L-weight} with the determinant contribution
\eqref{eq:weight-is-c} coming from $\Eul(\nu_\xi)^{-1}_+$, the \emph{net} quadratic
behaviour is governed by
\begin{equation}
\label{eq:h-plus-c}
(h+c)(\xi,\xi).
\end{equation}

Recall that $\mathcal L$ is \emph{admissible} precisely when $h+c$ is positive
definite on $\mf g$, equivalently when
\[
(h+c)(\xi,\xi)\to +\infty\qquad\text{as }\|\xi\|\to\infty.
\]

\medskip
\noindent\textbf{(B2) Linear perturbation from Atiyah--Bott factors.}
The remaining factors in $\mathcal E$ are Atiyah--Bott generators attached to
representations $V$ of $G$.  Their $\xi$--weights are governed by ordinary
representation theory: if $\lambda$ is a weight of $V$, then $\xi$ acts with
weight $\langle\lambda,\xi\rangle$.  In particular, these contributions are at
most \emph{linear} in $\xi$:
\begin{equation}
\label{eq:AB-linear}
\wt_\xi(\text{Atiyah--Bott factors}) = O(\|\xi\|).
\end{equation}

\medskip
\noindent\textbf{(C) Finite-dimensionality of the $\xi$--invariant part.}
The polarized inverse Euler factor $\Eul(\nu_\xi)^{-1}_+$ has a $\xi$--weight
decomposition with two crucial properties:
\begin{enumerate}[(i)]
\item only weights $\le 0$ occur; and
\item each weight space has finite multiplicity.
\end{enumerate}
These follow from the fact that in \eqref{eq:polarized-eul} the symmetric algebra
is generated by strictly negative $\xi$--weight summands.

Fix $\xi$.  The weight--$0$ piece of
$\mathcal E\otimes \Eul(\nu_\xi)^{-1}_+$
is obtained by summing those weight spaces of $\Eul(\nu_\xi)^{-1}_+$ whose weights
cancel the (finite) set of weights appearing in $\mathcal E$.  Since each weight
space of $\Eul(\nu_\xi)^{-1}_+$ has finite multiplicity, it follows that
\begin{equation}
\label{eq:finite-invariants}
\bigl(\mathcal E\otimes \Eul(\nu_\xi)^{-1}_+\bigr)^{\xi\text{-inv}}
\ \text{is finite-dimensional.}
\end{equation}
This is the first conclusion of \cite[\S1.11]{TelemanWoodward}.

\medskip
\noindent\textbf{(D) Eventual vanishing for $\|\xi\|\gg 0$.}
Now let $\xi$ vary in the dominant cone.  The symmetric algebra part of
$\Eul(\nu_\xi)^{-1}_+$ produces weights by taking symmetric powers of
negative--weight generators.  The possible weights contributed in this way move
away from $0$ in steps controlled by the individual $\xi$--weights of the
generators; these steps scale \emph{linearly} in $\xi$ (because root weights
$\langle\alpha,\xi\rangle$ are linear in $\xi$).

On the other hand, twisting by an admissible line bundle $\mathcal L$ produces
the quadratic shift \eqref{eq:h-plus-c}.  Combining with the linear perturbation
\eqref{eq:AB-linear} from Atiyah--Bott generators, the net effect is that the set
of $\xi$--weights appearing in
$\mathcal E\otimes \Eul(\nu_\xi)^{-1}_+$
is translated by a term which grows like $(h+c)(\xi,\xi)$, up to linear error.
Since $(h+c)(\xi,\xi)$ grows quadratically while all available ``corrections''
coming from symmetric powers grow at most linearly, it follows that for $\|\xi\|$
sufficiently large the total $\xi$--weight $0$ cannot occur.  Equivalently, there
exists $B>0$ such that
\begin{equation}
\label{eq:eventual-vanishing}
\|\xi\|>B \quad\Longrightarrow\quad
\bigl(\mathcal E\otimes \Eul(\nu_\xi)^{-1}_+\bigr)^{\xi\text{-inv}}=0.
\end{equation}
This is the second conclusion of \cite[\S1.11]{TelemanWoodward}.

\medskip
\noindent\textbf{(E) Consequence for finiteness of the index.}
The local cohomology filtration of $R\Gamma(\mf M,\mathcal E)$ by Shatz supports
has graded pieces controlled by the strata $\mf M_\xi$.
Identifying the residue contribution along $\mf M_\xi$ with the $\xi$--invariant
part of $\mathcal E\otimes \Eul(\nu_\xi)^{-1}_+$, the finiteness statement
\eqref{eq:finite-invariants} gives finite-dimensionality of each stratum
contribution, while the vanishing \eqref{eq:eventual-vanishing} shows that only
finitely many $\xi$ contribute.  Therefore the local-to-global spectral sequence
has only finitely many nonzero columns, and the index $\Ind(\mf M,\mathcal E)$ is
finite.


\section{General idea}

Let $S = \C[[s]], S^* = \C((s))$ and $B$ be an $S$-scheme.  Let $C_S \to S$ be a projective flat family of curves with generic fiber $\C_{S^*}$ smooth and special fiber $C_0$ nodal with unique node $p$. Let $C_B = C_S \times_S B$.

Solis \cite{solis} defines the $S$-stack $\mathcal{X}_G(C_S)$ whose points evaluated at a test scheme $B/S$ are given by elements $(C'_B,P_B)$ where $C'_B$ is a twisted modification of $C_B$ and $P_B$ is an admissible $G$-bundle on $C'_B$. This stack is over a fixed curve $C_S$ and Solis shows that it is algebraic, locally of finite type, and complete over $S$. It contains $M_G(C_S)$ and $M_G(C_{S^*})$ as dense open substacks, and the complement of $M_G(C_{S^*})$ is a divisor with normal crossings.

In this section, we discuss how to generalize Solis' construction to families of curves by working over the universal curve over the moduli stack of stable curves $\overline{\mathfrak{M}}_{g,I}$. Let $\pi:\overline{\mathcal{C}_{g,I}} \to \overline{\mathfrak{M}}_{g,I}$ be the universal curve over the moduli stack of stable curves of genus $g$ with $I$ marked points.

Let $\pi:C\to B$ be a prestable family of nodal curves. Let
\[
\Sigma := \mathrm{Sing}(C/B)\subset C
\]
be the relative singular locus. It is finite étale over $B$ after restricting to the locus where the number of nodes is constant; globally it is at least finite unramified in good situations.
\begin{definition}
    

A \textbf{modification of $C/B$} is a proper morphism $m:C'\to C$ over $B$ such that:
\begin{enumerate}
\item $C'\to B$ is flat prestable curve, and $m$ is finitely presented and projective.

\item $m$ is an isomorphism away from the nodes:
\[
m:\; C'\setminus m^{-1}(\Sigma)\ \xrightarrow{\sim}\ C\setminus \Sigma.
\]

\item For every geometric point $b\to B$ and every node $p\in \Sigma_b\subset C_b$, the fiber of $m_b:C'_b\to C_b$ over $p$ is either a point (no modification at that node) or a chain of $\mathbb P^1$'s meeting the two branches in the standard way, and $m_b$ contracts that chain to $p$ and is an isomorphism elsewhere.
\end{enumerate}
\end{definition}

A \textbf{length $\le n$ condition} can be stated as:
\begin{itemize}
\item for every $b$ and every node $p\in \Sigma_b$, the chain over $p$ has at most $n$ components.
\end{itemize}

\begin{definition}[Twisted nodal curves over a base]\label{def:twisted-curve}
Let $B$ be a scheme over $\C$.
A \textbf{twisted nodal curve over $B$} is a proper Deligne--Mumford stack
\[
\pi:\mathcal C \longrightarrow B
\]
such that:

\begin{enumerate}
\item
The geometric fibers of $\pi$ are connected, one--dimensional, and the
coarse moduli space $\overline{\mathcal C}$ is a nodal curve.

\item
Let $\mathcal U \subset \mathcal C$ be the complement of the relative
singular locus $\mathrm{Sing}(\mathcal C/B)$.  
Then the restriction
\[
\mathcal U \hookrightarrow \mathcal C
\]
is an open immersion.

\item
For any geometric point $p:\Spec k\to \mathcal C$ mapping to a node of the
fiber over $b\in B$, there exists an integer $k\ge1$ and an element
$t\in \mathfrak m_{B,b}$ such that, étale-locally on $B$ at $b$ and strictly
henselian locally on $\mathcal C$ at $p$, there is an isomorphism
\[
\Spec \mathcal O_{\mathcal C,p}^{sh}
\;\cong\;
\Bigl[\,\Spec\bigl(\mathcal O_{B,b}^{sh}[u,v]/(uv-t)\bigr)\ \big/\ \mu_k\,\Bigr],
\]
where $\zeta\in\mu_k$ acts by
\[
(u,v)\longmapsto(\zeta u,\zeta^{-1}v).
\]
\end{enumerate}
\end{definition}

\begin{definition}
    A \textbf{twisted modification of $C/B$} is a twisted nodal curve $\mathcal{C} \to B$ whose coarse moduli space $\overline{\mathcal{C}}$ is a modification of $C/B$.
\end{definition}

Let $r = \operatorname{rk}(G)$. The ordered simple roots $\{\alpha_0,\alpha_1,\dots,\alpha_r\}$ determine
ordered vertices $\{\eta_0,\dots,\eta_r\}$ determined by the conditions
\[
\langle \eta_i,\alpha_j\rangle = 0 \text{ for } i\neq j
\quad\text{and}\quad
\langle \eta_0,\alpha_0\rangle =1.
\]
If we write $\theta = \sum_{i=1}^r n_i\alpha_i$ and set $n_0=1$ then one can
check these conditions can be expressed as
\begin{equation}
\label{eq:alpha-eta}
\langle \alpha_i,\eta_j\rangle = \frac{1}{n_i}\delta_{i,j}.
\end{equation}

Following \cite{solis}, if $C'_B$ is a twisted modification of length $\le r$, then a 
$G$--bundle on $C'_B$ is called \textbf{admissible} if the co--characters determining the 
equivariant structure at all nodes are linearly independent over $\mathbb{Q}$ and are given by 
a subset of $\{\eta_0,\dots,\eta_r\}$.

\begin{definition}
    We define a stack $\mathcal{X}_{G,g,I}$ over $\overline{\mathfrak{M}}_{g,I}$ whose points over a test scheme $B \to \overline{\mathfrak{M}}_{g,I}$ are given by pairs $(C'_B,P_B)$ where $C'_B$ is a twisted modification of the pullback $C_B$ of the universal curve $\overline{\mathfrak{C}_{g,I}}$ to $B$, and $P_B$ is an admissible $G$-bundle on $C'_B$. 
\end{definition}

\begin{proposition}\label{prop:X-algebraic}
The projection
\[
F:\mathcal X_{G,g,I}\to \overline{\mathcal M}_{g,I}
\]
is algebraic and locally of finite type.
\end{proposition}

\subsection{The $\PGL_2$ toy model: gluing at a node and the wonderful compactification}

Let $\widetilde C$ be a smooth connected curve and fix two distinct points
$p,q\in \widetilde C$.  Let $C$ be the nodal curve obtained by identifying
$p\sim q$, and write $\nu:\widetilde C\to C$ for the normalization; the node
$x\in C$ satisfies $\nu^{-1}(x)=\{p,q\}$.

\subsubsection*{1. A point of $G$ produces a $G$--bundle on the nodal curve}

Let $G=\PGL_2(\C)$ (or any algebraic group).  Fix a principal $G$--bundle
$E$ on $\widetilde C$ together with framings (trivializations of the fibres as
$G$--torsors)
\[
f_p:E|_p\xrightarrow{\sim} G,
\qquad
f_q:E|_q\xrightarrow{\sim} G.
\]
Given $g\in G$, define a $G$--equivariant isomorphism of $G$--torsors
\[
\phi_g:E|_p\longrightarrow E|_q
\]
by the composite
\[
E|_p \xrightarrow{\,f_p\,} G \xrightarrow{\,\cdot g\,} G \xrightarrow{\,f_q^{-1}\,} E|_q,
\]
where $x\mapsto xg$ denotes right multiplication (any consistent left/right
convention works).

Using $\phi_g$, one descends $E$ from $\widetilde C$ to a principal $G$--bundle
on the nodal curve $C$: informally, one glues the two fibres $E|_p$ and $E|_q$
over the branches of the node using the identification $\phi_g$.  Denote the
resulting descended bundle by $E(\phi_g)$.

Equivalently, a principal $G$--bundle on $C$ is the same as a principal $G$--bundle
on $\widetilde C$ together with an identification of the fibres over $p$ and $q$;
the isomorphism $\phi_g$ is exactly such an identification.  Thus, once the data
$(E,f_p,f_q)$ is fixed, the element $g\in G$ determines a principal $G$--bundle
on $C$.

\subsubsection*{2. Rephrasing the gluing as a $\Delta(G)$--reduction}

Consider the $G\times G$--torsor $E|_p\times E|_q$ over $\Spec\C$.
An isomorphism $\phi:E|_p\to E|_q$ is equivalent to the choice of a point in the
contracted product
\[
(E|_p\times E|_q)/\Delta(G),
\]
since $\Delta(G)$ acts by simultaneous change of trivializations, and the graph
of $\phi$ is a $\Delta(G)$--orbit.

Equivalently, giving $\phi$ is the same as giving a reduction of structure group
\[
E|_p\times E|_q
\quad\text{from }G\times G\text{ to }\Delta(G)\subset G\times G.
\]
This is the sense in which one identifies $G\simeq (G\times G)/\Delta(G)$ and
interprets the gluing map $\phi_g$ as a $\Delta(G)$--reduction at the pair
$(p,q)$.

\subsubsection*{3. One--parameter families and the need for compactification}

A morphism
\[
\gamma:\Spec\C[t^{\pm 1}]\longrightarrow G
\]
gives a family of gluing maps $\phi_{\gamma(t)}$, hence a family of principal
$G$--bundles on the fixed nodal curve $C$ parametrized by $t\in\C^\times$.
However, $\gamma$ may fail to extend to $t=0$ as a morphism
$\Spec\C[t]\to G$.  A basic example is
\[
t\longmapsto \diag(t,t^{-1})\in \SL_2(\C)\twoheadrightarrow \PGL_2(\C),
\]
which ``goes to infinity'' in $G$.  In that case the family of glued bundles on
$C$ has no \emph{a priori} limit inside the original moduli problem.  One remedy
is to enlarge the parameter space $G$ to a compactification $\overline G$, so
that $\gamma$ extends and the boundary value can be given a modular
interpretation.

\subsubsection*{4. For $G=\PGL_2$, the wonderful compactification is $\P^3$}

An element of $G=\PGL_2(\C)$ may be represented by a $2\times 2$ matrix up to
overall scaling,
\[
g=\begin{pmatrix}a&b\\c&d\end{pmatrix},
\qquad (a,b,c,d)\neq 0,
\]
giving an open embedding
\[
\PGL_2 \hookrightarrow \P^3,\qquad [a:b:c:d],
\]
whose image is the open subset $\{ad-bc\neq 0\}$.  The boundary is the quadric
\[
\{ad-bc=0\}\subset \P^3,
\]
which is the locus of rank $\le 1$ matrices.  Concretely,
\[
\{ad-bc=0\}
=
\{\text{rank}\le 1\}
=
\bigl\{[u v^T]: u\in \C^2\setminus\{0\},\, v\in(\C^2)^\vee\setminus\{0\}\bigr\}/\C^\times
\cong \P^1\times \P^1,
\]
sending a rank--one matrix $uv^T$ to the pair $([u],[v])$.

\subsubsection*{5. Modular meaning of boundary points: Borel reductions at $p$ and $q$}

Let $B\subset \PGL_2$ be a Borel subgroup (e.g.\ the image of upper triangular
matrices).  Then
\[
G/B \cong \P^1
\]
is the flag variety.  In this case, the boundary of the wonderful
compactification admits an identification
\[
\partial\overline G:=\overline G\setminus G \ \cong\ (G/B)\times (G/B)
\cong \P^1\times \P^1.
\]

Given a point $(s_p,s_q)\in (G/B)\times (G/B)$, the framings $f_p,f_q$ identify
the fibres $E|_p\simeq G$ and $E|_q\simeq G$, and hence:
\begin{itemize}
\item the point $s_p\in G/B$ is the same as a $B$--reduction of the framed fibre $E|_p$;
\item the point $s_q\in G/B$ is the same as a $B$--reduction of the framed fibre $E|_q$.
\end{itemize}
Thus allowing the gluing parameter to land in the boundary replaces an honest
identification $E|_p\simeq E|_q$ (equivalently a $\Delta(G)$--reduction of
$E|_p\times E|_q$) by weaker boundary data: a $B\times B$--reduction of the
$G\times G$--torsor $E|_p\times E|_q$.

In other words, the ``completed'' moduli problem includes:
\begin{itemize}
\item a principal $G$--bundle $E$ on $\widetilde C$, and
\item either a $\Delta(G)$--reduction at $(p,q)$ (giving a genuine $G$--bundle on $C$),
\item or a boundary datum consisting of $B$--reductions at $p$ and $q$
(i.e.\ a $B\times B$--reduction).
\end{itemize}
This is the simplest instance of completing a moduli problem by allowing
controlled degenerations at the node.

\subsubsection*{6. Relation to the affine/loop--group story}

In the affine story one replaces the finite--dimensional parameter space $G$ by
loop--group data (often together with loop rotation $\C^\times$).  The boundary
of an affine compactification encodes parahoric reductions (equivalently,
coweight data) at the node; this is where affine Weyl combinatorics enters.
For $G=\PGL_2$ in the finite--dimensional toy model, the boundary is merely
$(G/B)^2$, whereas in the affine case the boundary stratifies by affine Weyl
data and leads to admissibility constraints.

\subsubsection*{7. A concrete limit computation}

Let $\gamma(t)=[\diag(t,t^{-1})]\in \PGL_2$.  In homogeneous coordinates
$[a:b:c:d]$ on $\P^3$, this is
\[
[t:0:0:t^{-1}] = [t^2:0:0:1].
\]
As $t\to 0$ this tends to $[0:0:0:1]$, a rank--one matrix, hence a boundary point.
Under $\partial\overline G\cong \P^1\times \P^1$, this point corresponds to a
pair of flags, i.e.\ to a choice of Borels at $p$ and $q$, which is exactly the
$B\times B$--reduction boundary datum described above.

Let $S=\C[[t]]$ and take a family $C\to S$ smoothing a node, locally $xy=t$. Near the node you have two formal branches and the “middle” is a punctured disc in the generic fiber.
To give a $G$-bundle on a curve, it is enough to give it on the complement of a point and on the formal disc, plus an identification on the punctured disc.

That identification is a transition function in $G(\!(z)\!)$, i.e. an element of $LG$. In particular, if we allow limits of such transition functions, we are forced into allowing other “integral models” on the disc, i.e. parahorics. 

The classification of parahorics is affine-Weyl/alcove combinatorics. In partiuclar, parahorics are classified by facets of the fundamental alcove, and maximal parahorics correspond to vertices of the alcove, which are labeled by the $\eta_i$.

\begin{remark}[Why the $\eta_i$ should be viewed as \emph{parahoric types}]
\label{rem:eta-parahoric}
The slogan is that in Solis' compactification one is not parametrizing ordinary
$G$--bundles alone, but rather torsors under a \emph{sheaf of groups}
$\mathcal G$ which agrees with $G$ away from the nodes and is replaced by a
\emph{parahoric} subgroup of the loop group near each node.  The labels
$\eta_i$ encode precisely which parahoric is allowed.

\smallskip
\noindent\textbf{(1) $G$--bundles as torsors for a sheaf of groups.}
Let $C$ be a smooth curve.  Write
\[
\mathcal G^{\mathrm{std}}(U):=\Hom_{\mathrm{Sch}}(U,G)
\]
for the usual sheaf of groups on $C$.
A principal $G$--bundle $E$ determines a $\mathcal G^{\mathrm{std}}$--torsor by
\[
\mathcal F_E(U):=\Gamma(U,E|_U),
\]
and conversely $\mathcal G^{\mathrm{std}}$--torsors are the same thing as
principal $G$--bundles (this is the standard equivalence between $G$--bundles and
$G$--torsors).

\smallskip
\noindent\textbf{(2) Modifying the local structure group: parabolics and parahorics.}
Fix a point $p\in C$ with a formal parameter $z$.
Let $D=\Spec\C[[z]]$ and $D^\times=\Spec\C((z))$.
A principal $G$--bundle may be described by giving it on $C\setminus\{p\}$ and on
$D$, together with a gluing isomorphism on the overlap $D^\times$; after
choosing trivialisations this gluing is an element of the loop group
\[
G((z))=LG.
\]

Now let $P\subset G$ be a parabolic subgroup.  Set
\[
L^+_P G:=\{\gamma\in G[[z]] \mid \gamma(0)\in P\}\subset G[[z]].
\]
One may package the condition ``a $G$--bundle with reduction to $P$ at $p$''
by replacing the local gauge group $G[[z]]$ on the disc by $L^+_P G$.
Equivalently, define a new sheaf of groups $\mathcal G^P$ on a neighborhood of
$p$ by
\[
\mathcal G^P(D)=L^+_P G,
\qquad
\mathcal G^P(D^\times)=G((z)),
\]
and gluing this with $\mathcal G^{\mathrm{std}}|_{C\setminus\{p\}}$ along the
overlap $D^\times$.
Then $\mathcal G^P$--torsors are exactly quasi--parabolic $G$--bundles
(i.e.\ $G$--bundles on $C$ with a $P$--reduction at $p$).

More generally one may replace $L^+_P G$ by any \emph{parahoric} subgroup
$K\subset G((z))$ (in the sense of Bruhat--Tits), obtaining a sheaf of groups
$\mathcal G^K$ whose torsors are ``$G$--bundles with parahoric structure at $p$''.
For \emph{exotic} parahorics these torsors need not be identifiable with ordinary
$G$--bundles plus extra structure; they are genuinely new objects.

\smallskip
\noindent\textbf{(3) Where the $\eta_i$ enter.}
Parahoric subgroups of $G((z))$ are classified by facets of the affine alcove,
and the \emph{vertices} of the fundamental alcove correspond to \emph{maximal}
parahorics.  The affine simple roots $\{\alpha_0,\alpha_1,\dots,\alpha_r\}$ have
dual ``vertex'' coweights $\{\eta_0,\dots,\eta_r\}$, and one may view $\eta_i$ as
labels for these maximal parahorics:
\[
\eta_i \quad\longleftrightarrow\quad \text{a maximal parahoric } \mathcal P_{\eta_i}\subset G((z)).
\]
Thus, specifying ``$\eta_i$ at a node'' should be interpreted as specifying that
the local structure group near that node is the parahoric $\mathcal P_{\eta_i}$,
i.e.\ that we are working with torsors for the corresponding modified sheaf of
groups.

\smallskip
\noindent\textbf{(4) Conceptual rephrasing of admissibility.}
In this language, Solis' admissibility condition is a finiteness restriction on
the allowed local group schemes at the nodes: on a twisted modification one only
allows those parahoric types indexed by the finite set
$\{\eta_0,\dots,\eta_r\}$ (and imposes a further independence condition when
several nodes occur).  This is why it is natural to think of the $\eta_i$ as
\emph{parahoric types}.
\end{remark}

\begin{definition}[$\mathcal P$--parahoric $G$--bundles at a point]
\label{def:parahoric-torsor}
Fix a smooth curve $C$ over $\C$ and a point $p\in C$ with a choice of formal
parameter $z$ at $p$, so that the completed local ring is $\widehat{\mathcal O}_{C,p}\cong\C[[z]]$
and the punctured disc is $D^\times=\Spec\C((z))$.
Let
\[
\mathcal P\ \subset\ G((z))
\]
be a \emph{parahoric} subgroup (for instance a \emph{maximal} parahoric, i.e.\ one
corresponding to a vertex of the fundamental alcove).

Define a sheaf of groups $\mathcal G^{\mathcal P}$ on $C$ by gluing the standard
sheaf $\mathcal G^{\mathrm{std}}$ away from $p$ with the local sheaf determined by
$\mathcal P$ at $p$ as follows:
\begin{itemize}
\item if $U\subset C$ is an open subset with $p\notin U$, set
\(\mathcal G^{\mathcal P}(U):=\mathcal G^{\mathrm{std}}(U)=\Hom_{\mathrm{Sch}}(U,G)\);
\item for the formal disc $D=\Spec\C[[z]]$ and punctured disc $D^\times=\Spec\C((z))$, set
\[
\mathcal G^{\mathcal P}(D):=\mathcal P,\qquad \mathcal G^{\mathcal P}(D^\times):=G((z)),
\]
with restriction map given by the inclusion $\mathcal P\hookrightarrow G((z))$;
\item on the overlap $(C\setminus\{p\})\cap D \simeq D^\times$, we identify both
restrictions with $G((z))$ and glue.
\end{itemize}

A \emph{$\mathcal P$--parahoric $G$--bundle on $C$ (with parahoric structure of type $\mathcal P$ at $p$)} is a
$\mathcal G^{\mathcal P}$--torsor on $C$.
Equivalently, it is the data of:
\begin{enumerate}[(i)]
\item a principal $G$--bundle $E$ on $C\setminus\{p\}$;
\item a $\mathcal P$--torsor $E_D$ on the formal disc $D$ (i.e.\ a principal homogeneous space under the group
$\mathcal P=\mathcal G^{\mathcal P}(D)$);
\item an identification of the induced $G((z))$--torsors over $D^\times$.
\end{enumerate}

When $\mathcal P=L^+_P G=\{\gamma\in G[[z]]\mid \gamma(0)\in P\}$ for a parabolic $P\subset G$, this recovers the usual notion
of a quasi--parabolic $G$--bundle with a $P$--reduction at $p$. 
In particular $L^+_{P_J}G$ corresponds to the facet containing the hyperspecial vertex $\eta_0$ together with the vertices $\eta_j$ for $j\in J$.


When $\mathcal P$ is a maximal parahoric corresponding to a vertex $\eta_i$ of the fundamental alcove, we say the parahoric structure at $p$ is \emph{of type $\eta_i$}.
\end{definition}

\begin{proposition}[Normalization description of parahoric bundles]
Let $C$ be a nodal curve with normalization
$\nu:\widetilde C\to C$ and nodes $x_i$ with preimages
$\nu^{-1}(x_i)=\{p_i,q_i\}$.
Fix parahoric subgroups
$\mathcal P_i\subset G((z_i))$ at each node.

A $\mathcal G^{\mathcal P}$--torsor on $C$ is equivalent to:

\begin{enumerate}
\item a principal $G$--bundle $E$ on $\widetilde C$;

\item for each $i$, $\mathcal P_i$--torsors
$E_{p_i}$ and $E_{q_i}$ on the formal discs at $p_i,q_i$
whose restrictions to the punctured discs agree with
$E|_{\widetilde C\setminus\{p_i,q_i\}}$;

\item for each $i$, a gluing class
\[
[\phi_i]\in
\mathcal P_i \backslash G((z_i)) / \mathcal P_i ,
\]
giving an identification of the punctured restrictions
$E_{p_i}|_{D^\times}\cong E_{q_i}|_{D^\times}$.
\end{enumerate}

If the node is stacky with stabilizer $\mu_k$,
one must additionally specify homomorphisms
$\rho_i:\mu_k\to \mathcal P_i$
and require the gluing to be $\mu_k$--equivariant.
\end{proposition}
\subsection*{How the compactification includes ordinary bundles}

How does this compactify $G$-bundles on a nodal curve? Ordinary bundles sit inside as the trivial parahoric structure.

\medskip
\noindent\textbf{(1) Ordinary bundles on nodal curves.}
Choose trivializations of $E$ on formal discs at $p,q$. Then a $G$-bundle on $C$ is the same as
\begin{itemize}
\item a principal $G$-bundle $E$ on $\widetilde C$, and
\item an identification of fibers $E_p\simeq E_q$.
\end{itemize}
After choosing framings, that identification is an element $g\in G$.
So the gluing parameter space for bundles on the fixed nodal curve is $G$.

Equivalently (using discs), the gluing is a loop $\gamma(z)\in G((z))$ which actually lies in $G[[z]]$ and has the same value at $z=0$ on both branches, so it descends across the node. Concretely, the descent condition forces you into the ``diagonal'' subgroup
\[
\Delta(G[[z]])\subset G[[z]]\times G[[z]].
\]

\medskip
\noindent\textbf{(2) Parahoric bundles enlarge the allowed local data.}
Now pick a parahoric $\mathcal P\subset G((z))$. A $\mathcal P$-parahoric torsor on $C$ is (after choosing local trivializations) given by a double coset
\[
[\gamma]\in \mathcal P\backslash G((z))/\mathcal P,
\]
i.e.\ you allow gluing by any loop $\gamma(z)$, but you declare two loops equivalent if they differ by ``integral'' gauge transformations on each side lying in $\mathcal P$.

This enlarges the moduli because $\mathcal P\backslash G((z))/\mathcal P$ contains more than the constant loops $G$.

\medskip
\noindent\textbf{(3) Where ordinary bundles sit inside: the hyperspecial case.}
There is a distinguished maximal parahoric, the \emph{hyperspecial}
\[
\mathcal P_0:=G[[z]]\subset G((z)).
\]
Then
\[
\mathcal P_0\backslash G((z))/\mathcal P_0
\]
contains the constant loops $G$ as an open subset (the big cell) in the Bruhat decomposition
\begin{align*}
G((z)) &= \bigsqcup_{\lambda\in X_*(T)_+} \mathcal P_0 z^\lambda \mathcal P_0 
\end{align*} where $X_*(T)_+$ are the dominant coweights. The constant loops correspond to $\lambda=0$.

In words:
\begin{itemize}
\item trivial parahoric structure means: at the node you are not allowing any degeneration of the local group scheme; you keep the standard integral model $G[[z]]$;
\item ordinary $G$-bundles on $C$ correspond to those parahoric torsors whose gluing class can be represented by a constant loop (equivalently, whose local modification at the node is trivial).
\end{itemize}



\subsection*{Fixed nodal curve}

Let $C_{0,[k]}$ be a twisted nodal curve with a single twisted node $p$.
Let $C_0$ be its coarse moduli space and, by abuse of notation, also write
$p\in C_0$ for the coarse node. The stabilizer of $p\in C_{0,[k]}$ is $\mu_k$,
and in particular
\[
C_{0,[k]}\times_{C_0} D_0 \;\cong\; \Bigl[\,D_0^{1/k}/\mu_k\,\Bigr],
\]

For a parahoric $\mathcal P$, let $\mathcal L\mathcal U$ be its Levi
decomposition and set
\[
\mathcal P^{\Delta} \;:=\; \Delta(\mathcal L)\ltimes (\mathcal U\times \mathcal U).
\]
One can construct a sheaf of groups $\mathcal G^{\Delta}$
over $C_0$ such that
\[
\mathcal G^{\Delta}(\widehat{\mathcal O}_{C_0,p})=\mathcal P^{\Delta},
\qquad
\mathcal G^{\Delta}\big|_{C_0-p}=\mathcal G^{\mathrm{std}}.
\]
Let $\mathcal M_{\mathcal G^{\Delta}}(C_0)$ denote the moduli stack of
$\mathcal G^{\Delta}$--torsors on $C_0$ and let $T_{\mathcal G^{\Delta}}(C_0)$
denote the moduli space of pairs $(\mathcal F,\tau)$ where
$\mathcal F\in \mathcal M_{\mathcal G^{\Delta}}(C_0)$ and $\tau$ is a
trivialization of $\mathcal F$ over $C_0-p$. Define $T_{\mathcal G^{\Delta}}(D_0)$
similarly.

Let $\eta\in \Hom(\C^\times,T)\otimes_{\Z}\Q$ and consider the moduli stack
$\mathcal M_{G,\eta}(C_{0,[k]})$ of $G$--bundles on $C_{0,[k]}$ with equivariant
structure at $p$ determined by $\eta$. Let $T_{G,\eta}(C_{0,[k]})$ denote the
moduli space of pairs $(P,\tau)$ with $P\in \mathcal M_{G,\eta}(C_{0,[k]})$ and
$\tau$ a trivialization of $P$ on $C_{0,[k]}-p$. Define
$T_{G,\eta}\bigl([D_0^{1/k}/\mu_k]\bigr)$ similarly.

\begin{proposition}[{\cite[Prop.\ 3.4.5]{solis}}]
\label{prop:solis-345}
Suppose $k\eta\in \Hom(\C^\times,T)$ and set $\mathcal P=\mathcal P(\eta)$.
Let
\[
D_0=\Spec\C[[x,y]]/(xy),
\qquad
\Bigl[\,D_0^{1/k}/\mu_k\,\Bigr]
\ \ \text{as above.}
\]
Choose $k$th roots $u,v$ of $x,y$ so that $D_0^{1/k}=\C[[u,v]]/(uv)$.
Let
\[
i_{0,[k]}:\Bigl[\,D_0^{1/k}/\mu_k\,\Bigr]\longrightarrow C_{0,[k]},
\qquad
i_0:D_0\longrightarrow C_0
\]
be the natural maps. Let
\[
G^{\Delta}_{u,v}
:=\Bigl\{(g_1,g_2)\in L_u^+G\times L_v^+G \ \Big|\ g_1(0)=g_2(0)\Bigr\}.
\]
Then we have isomorphisms fitting into the diagram
\[
\begin{tikzcd}[column sep=large,row sep=large]
T_{\mathcal G^{\Delta}}(D_0)
&
T_{\mathcal G^{\Delta}}(C_0)
\arrow[l,"i_0^{*}"']
\arrow[r,"\Xi_{C_0}"]
\arrow[d,"\Psi^{\mathcal P^\Delta}_{C}"']
&
T_{G,\eta}(C_{0,[k]})
\arrow[r,"i_{0,[k]}^{*}"]
\arrow[d,"\Psi^{\eta}_{C}"]
&
T_{G,\eta}\Bigl(\bigl[D_0^{1/k}/\mu_k\bigr]\Bigr)
\\
{} % <--- placeholder row 1
&
LG\times LG\,/\,\mathcal P^{\Delta,\eta}
\arrow[r,"\eta^{-1}(\ \cdot\ )\eta"']
&
(L_uG\times L_vG)^{\mu_k}\big/\bigl(G^\Delta_{u,v}\bigr)^{\mu_k}
&
{}  % <--- placeholder column 4
&
\end{tikzcd}
\]

where $\Xi_{C_0}$ is defined to be
\[
\Xi_{C_0}
:=\bigl(\Psi^{\eta}_{C}\bigr)^{-1}\circ \eta(\ \cdot\ )\eta^{-1}\circ
\Psi^{\mathcal P^\Delta}_{C},
\]
$\Psi^{\mathcal P^\Delta}_{C}$ is the map in \emph{(3.8)}, $\Psi^{\eta}_{C}$ is
the map in \emph{(3.10)}, and the bottom horizontal map is the product map
\[
g(z)\mathcal P
\longmapsto
\eta(w)\,g(w^{k})\,\eta^{-1}(w)\cdot (L_w^+G)^{\mu_k}.
\]
The isomorphism $\Xi_{C_0}$ descends to an isomorphism of stacks
\[
\Xi:\ \mathcal M_{\mathcal G^{\mathcal P}}(C_0)\ \xrightarrow{\ \sim\ }\ 
\mathcal M_{G,\eta}(C_{0,[k]}).
\]
\end{proposition}

\subsection*{Connection with $L^{\times}_{\mathrm{poly}}G$}

In Chapter~2 a stacky orbit closure $\partial X^{aff,poly}$ was constructed,
analogous to the boundary of the wonderful compactification of a semisimple
adjoint group.  In particular, the boundary components are smooth and intersect
transversely.  Let $r=\rank(G)$.  There are $2^{r+1}-1$ boundary orbits
labeled by the nonempty subsets of $\{0,\dots,r\}$.

\begin{proposition}[{\cite[Prop.\ 3.4.6]{solis}}]
\label{prop:solis-346}
Let $L_I$, $\mathcal P_I^{\pm}$, $\mathcal U_I^{\pm}$ be as in \emph{(3.2)} of
Section~3.2 and let $Z_0(L_I)$ be the connected component of the center
$Z(L_I)$.  Define
\[
\mathcal P_I^{\Delta,\pm}
:= \Delta(L_I)\ltimes (\mathcal U_I^+\times \mathcal U_I^-).
\]
Then the orbit $\mathcal O_I$ in the boundary of $X^{aff,poly}$ is
\[
\mathcal O_I
=
\frac{L_{\mathrm{poly}}G\times L_{\mathrm{poly}}G}
     {\,Z_0(L_I)\times Z_0(L_I)\cdot \mathcal P_I^{\Delta,\pm}}.
\]

In particular, the orbit $\mathcal O_I$ fibers over
\[
LG/\mathcal P_I \times LG/\mathcal P_I^{-}
\]
with fiber the adjoint Levi
\[
L_{I,\mathrm{ad}} = L_I/Z_0(L_I).
\]
Furthermore,
when $I=\{i\}$ is a singleton the group $Z_0(L_I)$ is trivial, while for
$|I|>1$ one has $Z_0(L_I)=Z(L_I)$.
\end{proposition}

\begin{remark}
The isomorphisms of Proposition~\ref{prop:solis-345} allow one to identify
the singleton orbit $\mathcal O_{\{i\}}$ with the moduli spaces
\[
T_{G,\eta_i}\!\left(\Bigl[D_0^{1/k}/\mu_k\Bigr]\right)
\qquad\text{and}\qquad
T_{G,\eta_i}(C_{0,[k]}),
\]
where $\eta_i$ is the $i$th vertex of the affine alcove.
The natural expectation is that the moduli problem
$T_{G,\eta_i}\!\left(\bigl[D_0^{1/k}/\mu_k\bigr]\right)$
can further degenerate to moduli problems parametrized by the higher
codimensional orbits in
$\C^\times\times L_{\mathrm{poly}}G$, and similarly for
$T_{G,\eta_i}(C_{0,[k]})$.
This is established in the next subsection.
\end{remark}

\subsection*{$G$--bundles on twisted chains}

In the previous section we saw that associated to the singleton sets
$\{i\}\subset \{0,\dots,r+1\}$ there is a moduli space parametrizing $G$--bundles
on a twisted nodal curve, and further the moduli space can be identified with an
orbit of the wonderful embedding of the loop group.  In this section we
introduce a more general moduli problem which we show is isomorphic to the orbit
$O_I$ in the wonderful embedding for any $I\subset \{0,\dots,r+1\}$.

\medskip

Let $R_n$ denote the rational chain of projective lines with $n$ components;
There is an action of $\C^\times$ on $R_n$ which scales each component.  Let $p_0,\dots,p_n$ denote the fixed points of this action.

Recall that $u,v$ are $k$th roots of $x,y$ which are our coordinates near a node.
Let $p',p''$ denote the closed points of $\Spec \C[[u]]$ and $\Spec \C[[v]]$.
Finally, let $D^{1/k}_n$ be the curve obtained from
\[
\Spec \C[[u]]\ \sqcup\ R_n\ \sqcup\ \Spec \C[[v]]
\]
by identifying $p'$ with $p_0$ and $p''$ with $p_n$.

The group $\mu_k$ acts on $D^{1/k}_n$ through its usual action on $u,v$ and
through the inclusion $\mu_k\subset \C^\times$ on the chain $R_n$.
For an $n$--tuple $(\beta_0,\dots,\beta_n)\in \Hom(\C^\times,T)^n$, we can speak
about the equivariant $G$--bundles on $D^{1/k}_n$ with equivariant structure at
$p_i$ determined by $\beta_i$.  We refer to this equivalently as $G$--bundles on
\[
\bigl[D^{1/k}_n/\mu_k\bigr]
\]
of type $(\beta_0,\dots,\beta_n)$.

Further, we can also glue $\bigl[D^{1/k}_n/\mu_k\bigr]$ to $C_0-p_0$ to obtain a
curve $C_{n,[k]}$.  Let $C_n$ denote the coarse moduli space of $C_{n,[k]}$.
We call $C_n$ a \emph{modification} of $C_0$ and $C_{n,[k]}$ a \emph{twisted
modification} of $C_0$.

\medskip

Recall the specific cocharacters $\eta_0,\dots,\eta_r$ defined in \emph{(3.1)} in
\S3.2.  For $I=\{i_1,\dots,i_n\}\subset \{0,\dots,r\}$, let
$T_{G,I}\!\bigl([D^{1/k}_n/\mu_k]\bigr)$ denote the moduli space of pairs $(P,\tau)$
where $P$ is a $G$--bundle on $\bigl[D^{1/k}_n/\mu_k\bigr]$ of type
$(\eta_{i_1},\dots,\eta_{i_n})$ and $\tau$ is a trivialization on
\[
\bigl[\Spec \C((u))\times \C((v))/\mu_k\bigr].
\]
Let $H=\Aut(P)$; then restriction to $\Spec \C[[u]]$ and $\Spec \C[[v]]$ realizes
\[
H\ \subset\ (L_u G)^{\mu_k}\times (L_u G)^{\mu_k}.
\]

\begin{theorem}[{\cite[Thm.\ 3.4.7]{solis}}]
Let $I\subset \{0,\dots,r\}$ and $T_{G,I}\!\bigl([D^{1/k}_n/\mu_k]\bigr)$ be as
above.  Then there is an isomorphism
\[
T_{G,I}(C_{0,[k]})
\xrightarrow{\ \Psi^{\eta_I}\ }
(L_u G)^{\mu_k}\times (L_u G)^{\mu_k}/H
\xrightarrow{\ \eta_I^{-1}(\ )\eta_I\ }
\frac{L_{\mathrm{poly}}G\times L_{\mathrm{poly}}G}
{Z(L_I)\times Z(L_I)\cdot P_I^{\Delta,\pm}}.
\]
Here $\Psi^{\eta_I}$ is as in \emph{(3.10)} and $\eta_I^{-1}(\ )\eta_I$ is
described in Proposition~3.4.5.  Let
\[
i:\ [D^{1/k}_n/\mu_k]\ \longrightarrow\ C_{0,[k]}
\]
be the natural map.  Then
\[
i^*:\ T_{G,I}(C_{0,[k]})\ \longrightarrow\ [D^{1/k}_n/\mu_k]
\]
is an isomorphism.  In particular,
$T_{G,I}(C_{0,[k]})$ and $T_{G,I}\!\bigl([D^{1/k}_n/\mu_k]\bigr)$ are isomorphic to
an orbit in the wonderful embedding of $L_{\mathrm{poly}}^{\times}G$.
\end{theorem}


\subsection{Refined stratification of $\mathcal X_{G,g,I}$}

Let $C/B$ be a prestable curve with dual graph $\Gamma$, and let
$(C'_B,P_B)$ be an object of $\mathcal X_{G,g,I}$; that is, $C'_B$ is a twisted
modification of $C_B$ and $P_B$ is an admissible $G$--bundle on $C'_B$.

For each vertex $v\in V(\Gamma)$ let $\xi_v$ denote the Harder--Narasimhan
type of the restriction of $P_B$ to the normalization component indexed by $v$.
For each node $e\in E(\Gamma)$ we have two additional pieces of boundary data:

\begin{enumerate}
\item a \emph{parahoric type}
      \[
        I_e\subset \{0,\dots,r\},
      \]
      specifying the parahoric subgroup
      $\mathcal P_{I_e}\subset G((z))$ which governs the local structure of
      the bundle at $e$;

\item a \emph{relative position label}
      \[
        \mathbf w_e \in
        W_{I_e}\backslash \widetilde W / W_{I_e},
      \]
      equivalently an orbit
      $O_e\subset \mathcal P_{I_e}\backslash G((z))/\mathcal P_{I_e}$
      describing the gluing of the two branches at $e$.
\end{enumerate}

Let $\tau_e$ denote the length of the modification chain over the node $e$.
Collect the data into
\[
\alpha=(\Gamma,\tau,\mathbf I,\mathbf w,\boldsymbol{\xi}),
\qquad
\mathbf I=(I_e)_{e\in E(\Gamma)},\;
\mathbf w=(\mathbf w_e)_{e\in E(\Gamma)},\;
\boldsymbol{\xi}=(\xi_v)_{v\in V(\Gamma)} .
\]

\begin{definition}
The \emph{refined stratum} of type $\alpha$ is the locally closed substack
\[
\mathcal X_\alpha \subset \mathcal X_{G,g,I}
\]
consisting of objects $(C'_B,P_B)$ such that:
\begin{enumerate}[(i)]
\item the coarse curve $C_B$ has dual graph $\Gamma$ and the modification
      lengths at the nodes are $\tau_e$;

\item for every vertex $v$, the restriction of $P_B$ to the corresponding
      normalization component has Harder--Narasimhan type $\xi_v$;

\item at each node $e$, the parahoric structure of $P_B$ is of type $I_e$,
      and the gluing of the two branches lies in the orbit $O_e$
      corresponding to $\mathbf w_e$.
\end{enumerate}
\end{definition}

The collection $\{\mathcal X_\alpha\}_\alpha$ forms a stratification of
$\mathcal X_{G,g,I}$, and its closure relations are governed by:
\begin{itemize}
\item the usual specialization of dual graphs and modification lengths;
\item the dominance order on the HN types $\xi_v$;
\item the Bruhat order on the double cosets
      $W_{I_e}\backslash \widetilde W / W_{I_e}$.
\end{itemize}

\begin{remark}[Why Shatz data alone is insufficient]
For a smooth curve the Shatz stratification indexed by HN types $\xi$
is adequate, because a $G$--bundle has no additional local structure.
On a nodal curve, however, two bundles with identical Shatz types on the
normalization can differ essentially at the node.


Consider two objects: (1) a genuine $G$-bundle on the nodal curve with gluing element $g\in G \subset G((z))$, and (2) a limit object where the gluing is $z^\lambda \in G((z))$ with $\lambda>0$.

Both have the same normalization bundles, the same HN type $\xi_v=0$, and the same parahoric type $I_e=\{0\}$. However, (1) lies in the open stratum corresponding to actual $G$-bundles, while (2) lies in a boundary stratum of the wonderful compactification.


\smallskip
\noindent\textbf{(1) Parahoric choice.}
The admissible object is not an honest $G$--bundle but a torsor under a
sheaf of groups that equals $G$ away from the nodes and a parahoric
$\mathcal P_{I_e}$ near each node.  
Different choices of $I_e$ give non--isomorphic deformation theories and
different normal complexes, so they must label distinct strata.

\smallskip
\noindent\textbf{(2) Relative position.}
Even after fixing $I_e$, the gluing of the two branches is classified by
orbits in
$\mathcal P_{I_e}\backslash G((z))/\mathcal P_{I_e}$,
indexed by double cosets
$W_{I_e}\backslash\widetilde W/W_{I_e}$.
These correspond to different boundary directions in the wonderful
compactification and produce different weights in the local Euler factors.

\smallskip
Hence Shatz types $\boldsymbol{\xi}$ describe only the instability on the
normalization components; they do not control the parahoric structure nor
the affine--Weyl relative position at the nodes.
A filtration of $R\Gamma(\mathcal X_{G,g,I},\mathcal E)$ analogous to the
Teleman--Woodward argument therefore requires the refined indexing
$(\Gamma,\tau,\mathbf I,\mathbf w,\boldsymbol{\xi})$.
\end{remark}

\red{Requires proof and I am not even sure it is true}
\begin{lemma}
\label{lem:finite-type-open}
For any fixed combinatorial type (graph $\Gamma$ + expansion length bound + allowed parahoric types $I_e$ in a finite set) and any bound on $(\boldsymbol{\xi}_v, \mathbf{w}_e)$, the open substack $\mathcal{X}_{\le \alpha}$ is algebraic and of finite type over an étale chart of $\overline{\mathfrak{M}}_{g,I}$.
\end{lemma}
\subsection{Virtual normal complex and local cohomology for a refined stratum}

Fix a refined type
\[
\alpha=(\Gamma,\tau,\mathbf I,\mathbf w,\boldsymbol\xi)
\]
as in the refined stratification of $\mathcal X_{G,g,I}$, and let
\[
i_\alpha:\mathcal X_\alpha \hookrightarrow \mathcal X_{\le \alpha}
\]
be the inclusion into a finite--type open substack $\mathcal X_{\le \alpha}$
obtained by bounding the HN types on vertices, the modification lengths, and
restricting the node data to the prescribed finite sets
$(I_e,\mathbf w_e)$.

Let
\[
\pi:\mathcal C_{\le \alpha}\longrightarrow \mathcal X_{\le \alpha}
\]
be the universal twisted modification and let $\mathcal P$ be the universal
(parahoric) $G$--torsor on $\mathcal C_{\le \alpha}$.
Write
\[
\ad(\mathcal P):=\mathcal P\times^G \mf g
\]
for the adjoint vector bundle on $\mathcal C_{\le \alpha}$.

\subsubsection*{1. Tangent complexes: global principle}

A basic deformation--theoretic fact (for principal bundles on a curve, and
likewise for torsors under a smooth affine group scheme on a twisted curve) is:
\begin{equation}
\label{eq:tangent-bundle-stack}
T_{\Bun_G,\mathcal P}\ \simeq\ R\pi_*\ad(\mathcal P)[1].
\end{equation}
Intuitively: first--order deformations of $\mathcal P$ are controlled by
$H^1(\ad(\mathcal P))$ and infinitesimal automorphisms by $H^0(\ad(\mathcal P))$,
hence the shift $[1]$.

In our setting, $\mathcal X_{\le \alpha}$ is not just $\Bun_G$ on a fixed curve:
it also includes the expansion/twisting data.  However, after fixing
$\Gamma$ and bounding $\tau$, the expansion part is finite type and its
tangent directions are independent of the bundle instability directions.
Thus, for the purpose of the \emph{normal directions to the refined stratum},
one isolates the contribution coming from the $G$--torsor deformation theory,
which is governed by \eqref{eq:tangent-bundle-stack}.
(One can either include the expansion tangent complex everywhere and cancel it
in cones below, or simply work relative to the expansion stack.)

\subsubsection*{2. The ``Levi/parahoric core'' controlling the stratum}

By definition of the refined stratum $\mathcal X_\alpha$:

\begin{itemize}
\item for each vertex $v\in V(\Gamma)$, the restriction of $\mathcal P$
to the corresponding normalization component has instability type $\xi_v$,
hence admits a canonical reduction to a parabolic $P_{\xi_v}\subset G$
with Levi $G_{\xi_v}$;

\item for each node $e\in E(\Gamma)$, the local structure group is the
parahoric $\mathcal P_{I_e}\subset G((z))$, and the gluing lies in the orbit
indexed by $\mathbf w_e\in W_{I_e}\backslash\widetilde W/W_{I_e}$.
\end{itemize}

Package these constraints into a subsheaf of Lie algebras
\[
\ad_\alpha(\mathcal P)\ \subset\ \ad(\mathcal P)
\]
on $\mathcal C_{\le\alpha}$ defined as follows:

\begin{enumerate}[(i)]
\item on the smooth locus away from the nodes, $\ad_\alpha(\mathcal P)=\ad(\mathcal P)$;

\item on the normalization component corresponding to $v$, $\ad_\alpha(\mathcal P)$
is the Lie algebra bundle associated to the \emph{Levi} reduction, i.e.
\[
\ad_\alpha(\mathcal P)|_{C_v^{\sm}}
\;=\;
\mathcal P_{G_{\xi_v}}\times^{G_{\xi_v}} \mf g_{\xi_v}
\ \subset\
\mathcal P|_{C_v^{\sm}}\times^G \mf g;
\]

\item at a node $e$, $\ad_\alpha(\mathcal P)$ is the Lie algebra of the
\emph{parahoric Levi} dictated by $I_e$ and the chosen orbit $\mathbf w_e$.
Concretely, after choosing a formal parameter $z$ and trivializing on the punctured
disc, the allowed gauge transformations are $\mathcal P_{I_e}$, and the orbit
$\mathbf w_e$ fixes a relative position; infinitesimally, this replaces
$\mf g((z))$ by the Lie algebra $\mf p_{I_e}\subset \mf g((z))$ and, on the stratum,
by the Levi subalgebra $\mf l_{I_e}\subset \mf p_{I_e}$.
\end{enumerate}

Define the \emph{unstable quotient sheaf}
\[
\mathcal Q_\alpha\ :=\ \ad(\mathcal P)\big/\ad_\alpha(\mathcal P).
\]

This $\mathcal Q_\alpha$ is the geometric object which simultaneously encodes:
\begin{itemize}
\item the usual unstable normal directions on components (root spaces outside $\mf g_{\xi_v}$), and
\item the boundary/gluing directions at nodes (tangent directions to the Schubert
strata in $\mathcal P_{I_e}\backslash G((z))/\mathcal P_{I_e}$ determined by $\mathbf w_e$).
\end{itemize}

\subsubsection*{3. Definition of the virtual normal complex $\nu_\alpha$}

\begin{definition}[Virtual normal complex for the refined stratum]
\label{def:virtual-normal-alpha}
The \emph{virtual normal complex} to $\mathcal X_\alpha$ inside $\mathcal X_{\le\alpha}$
(relative to the fixed expansion data) is the perfect complex on $\mathcal X_\alpha$
\[
\nu_\alpha \ :=\ R\pi_*(\mathcal Q_\alpha)[1]\Big|_{\mathcal X_\alpha}.
\]
\end{definition}

\subsubsection*{4. Justification: why this is the correct normal complex}

The key point is that the refined stratum is cut out by imposing
\emph{linear conditions on infinitesimal gauge data}, encoded by the inclusion
$\ad_\alpha(\mathcal P)\subset \ad(\mathcal P)$.

Indeed, the standard deformation theory gives the tangent complex to the ambient
moduli (again, relative to expansions) as $R\pi_*\ad(\mathcal P)[1]$.  On the stratum,
allowed first--order deformations are precisely those preserving:
\begin{itemize}
\item the HN--Levi reductions on each component, and
\item the parahoric type and relative position orbit at each node,
\end{itemize}
which infinitesimally means deformations governed by $R\pi_*\ad_\alpha(\mathcal P)[1]$.

Since $\ad_\alpha(\mathcal P)\to\ad(\mathcal P)\to\mathcal Q_\alpha$ is exact,
pushing forward and shifting yields a distinguished triangle
\[
R\pi_*\ad_\alpha(\mathcal P)[1]
\ \longrightarrow\
R\pi_*\ad(\mathcal P)[1]
\ \longrightarrow\
R\pi_*\mathcal Q_\alpha[1]
\ \longrightarrow,
\]
which identifies $R\pi_*\mathcal Q_\alpha[1]$ as the cone of
$T_{\mathcal X_\alpha}\to T_{\mathcal X_{\le\alpha}}$.
This is exactly what ``normal complex'' means in derived deformation theory.

\subsubsection*{5. Local cohomology term attached to $\mathcal X_\alpha$}

Let $\mathcal E$ be a coherent sheaf or perfect complex on $\mathcal X_{\le\alpha}$.
Define the local cohomology of $\mathcal E$ with supports in $\mathcal X_\alpha$ by
\[
R\Gamma_{\mathcal X_\alpha}(\mathcal X_{\le\alpha},\mathcal E)
\ :=\
R\Gamma\bigl(\mathcal X_{\le\alpha},\,R\Gamma_{\mathcal X_\alpha}(\mathcal E)\bigr),
\]
characterized by the exact triangle
\[
R\Gamma_{\mathcal X_\alpha}(\mathcal X_{\le\alpha},\mathcal E)
\ \to\
R\Gamma(\mathcal X_{\le\alpha},\mathcal E)
\ \to\
R\Gamma(\mathcal X_{\le\alpha}\setminus \mathcal X_\alpha,\mathcal E)
\ \to.
\]

\red{Needs proof and I am not even sure it is true}
\begin{lemma}\label{lem:local-purity}
Each inclusion $i_\alpha:\mathcal X_\alpha\hookrightarrow \mathcal X_{\le\alpha}$ is (derived) lci/perfect so that
\[
R\Gamma_{\mathcal X_\alpha}(\mathcal X_{\le\alpha},\mathcal E)
\;\simeq\;
R\Gamma\bigl(\mathcal X_\alpha,\, i_\alpha^!\mathcal E\bigr)[d_\alpha]
\]
for a codimension shift $d_\alpha$.
Then there is a purity/local-duality identification:
\[
R\Gamma_{\mathcal X_\alpha}(\mathcal X_{\le\alpha},\mathcal E)
\ \simeq\
R\Gamma\bigl(\mathcal X_\alpha,\ \mathcal R_\alpha(\mathcal E)\bigr)[d_\alpha],
\]
where $d_\alpha$ is the (virtual) codimension and
\[
\mathcal R_\alpha(\mathcal E)\ :=\ i_\alpha^!(\mathcal E)[-d_\alpha]
\]
is the residue object along $\mathcal X_\alpha$.
\end{lemma}

\subsubsection*{6. Relation to the inverse Euler class (formal, but canonical once polarized)}

Under the same lci/perfectness hypotheses, the extraordinary pullback $i_\alpha^!$
is controlled by the normal complex.  In $K$--theory this yields the formal identity
\[
[\mathcal R_\alpha(\mathcal E)]
\ \sim\
[\mathcal E|_{\mathcal X_\alpha}]\cdot \Eul(\nu_\alpha^\vee)^{-1}.
\]
Here $\sim$ means equality in the appropriate \emph{completed} (equivariant)
$K$--group determined by a chosen polarization/weight convention, exactly as in
Teleman--Woodward: one must choose a direction of expansion so that the inverse
Euler factor is summable and weight spaces are finite.

In the refined situation, the relevant weight data comes from:
\begin{itemize}
\item the one--parameter subgroups determined by the vertex instability types
$\xi_v$ (componentwise Shatz data), and
\item the parahoric/relative position label $(I_e,\mathbf w_e)$ at each node,
which fixes the affine--Weyl combinatorics of the boundary directions and thus
the weight decomposition of the node contribution to $\mathcal Q_\alpha$.
\end{itemize}

Consequently, the local cohomology term attached to the refined stratum is:
\[
R\Gamma_{\mathcal X_\alpha}(\mathcal X_{\le\alpha},\mathcal E)
\ \simeq\
R\Gamma\Bigl(\mathcal X_\alpha,\ 
\mathcal E|_{\mathcal X_\alpha}\otimes \Eul(\nu_\alpha^\vee)^{-1}
\Bigr)[d_\alpha],
\]
interpreted in the same completed sense as Teleman--Woodward (and with a polarized
inverse Euler class if one wants uniform ``weights $\le 0$'' properties).


\subsubsection*{The analogue of the virtual normal complex $\nu_\alpha$}

For each refined stratum $\mathcal X_\alpha$, you want:
\begin{itemize}
\item a ``semistable Levi core'' stack $\mathcal Y_\alpha$ (finite type), and
\item a morphism $q_\alpha:\mathcal X_\alpha\to \mathcal Y_\alpha$ whose fibers are ``unstable directions'' (affine/unipotent), so that the normal theory is controlled by a perfect complex pulled back from $\mathcal Y_\alpha$.
\end{itemize}

\medskip\noindent\textbf{3.1 What $\mathcal Y_\alpha$ should be}

Let $\nu:\widetilde C\to C$ be the normalization of the underlying curve (over the relevant base). Over $\mathcal X_\alpha$, you have:
\begin{itemize}
\item components $\widetilde C_v$ indexed by vertices $v$,
\item marked points corresponding to original markings plus the preimages of nodes (two per edge),
\item at each preimage of a node: a specified parahoric type $I_e$ and a relative-position orbit $\mathbf w_e$.
\end{itemize}

Define $\mathcal Y_\alpha$ as the product over vertices $v$ of moduli of semistable $G_{\xi_v}$-bundles on $\widetilde C_v$ with the prescribed parahoric structures at the special points (the ones lying over nodes), together with whatever matching constraints encode $\mathbf w_e$ (often best phrased as belonging to a fixed Schubert cell in an affine flag space).

\medskip\noindent\textbf{3.2 What $\nu_\alpha$ should be}

Let $\pi:\mathcal C_\alpha\to \mathcal X_\alpha$ be the universal (twisted) curve and $\mathcal P$ the universal parahoric $G$-torsor (or $\mathcal G$-torsor for the Bruhat--Tits group scheme $\mathcal G$ that equals $G$ away from nodes).

Define a subsheaf of Lie algebras $\ad_\alpha(\mathcal P)\subset \ad(\mathcal P)$ consisting of infinitesimal automorphisms that preserve:
\begin{itemize}
\item the vertexwise canonical $P_{\xi_v}$-reductions on each normalization component, and
\item the edge constraints (parahoric type $I_e$ and relative-position orbit $\mathbf w_e$) at each node.
\end{itemize}

Then set the quotient
\[
\mathcal Q_\alpha := \ad(\mathcal P)/\ad_\alpha(\mathcal P),
\]
and define the virtual normal complex
\[
\nu_\alpha := R\pi_*(\mathcal Q_\alpha)[1]\quad\in \Perf(\mathcal X_\alpha).
\]

\medskip\noindent\textbf{What you must check here (deformation theory input)}

You need:
\begin{enumerate}
\item \textbf{Tangent complex for parahoric torsors:}
$T_{\mathcal X,\,(C',P)}\simeq R\Gamma(C',\ad(P))[1]$ still holds in the parahoric/twisted nodal setting.

\item \textbf{Compatibility of ``refined stratum constraints'' with deformation theory:}
The tangent complex of $\mathcal X_\alpha$ is the subcomplex cut out by $\ad_\alpha(\mathcal P)$, so the quotient controls the normal directions.

\item \textbf{Perfectness:}
$\ad(\mathcal P)$, $\ad_\alpha(\mathcal P)$, and $\mathcal Q_\alpha$ are perfect on the curve (ideally vector bundles), so $\nu_\alpha$ is perfect of amplitude $[-1,0]$ after shifting.
\end{enumerate}

This replaces TW’s formula \(R\pi_*\mathcal E^*(\mf g/\mf g_\xi)[1]\).

\subsection{An explicit refined stratification in the one--node, two--component case}

\subsubsection*{Setup}
Let $C=C_1\cup_x C_2$ be a connected nodal curve over $\C$ with a single node $x$,
where $C_1,C_2$ are smooth and meet transversely at $x$.
Let $\nu:\widetilde C=C_1\sqcup C_2\to C$ be the normalization and write
$\nu^{-1}(x)=\{p\in C_1,\ q\in C_2\}$.
Fix a connected reductive group $G$.

Fix a formal parameter $z$ at $p$ and $q$ (after choosing étale neighborhoods if desired),
so that the formal discs are $D_p=\Spec\C[[z]]$, $D_q=\Spec\C[[z]]$ and the punctured discs
are $D_p^\times=\Spec\C((z))$, $D_q^\times=\Spec\C((z))$.
Write $LG:=G((z))$ and $L^+G:=G[[z]]$.

\subsubsection*{Gluing description of $\Bun_G(C)$}
Let $\Bun_G(C_i,p)$ denote the stack of $G$--bundles on $C_i$ equipped with a framing
(trivialization) along $D_p$ (respectively $D_q$).
Then a $G$--bundle on $C$ is equivalent to a triple
\[
(E_1,\tau_p)\in \Bun_G(C_1,p),\qquad (E_2,\tau_q)\in \Bun_G(C_2,q),\qquad
\gamma\in LG,
\]
modulo the change of framings by $L^+G\times L^+G$, acting by
\[
(a,b)\cdot (E_1,\tau_p,E_2,\tau_q,\gamma)
=
(E_1,a\cdot\tau_p,\ E_2,b\cdot\tau_q,\ a\,\gamma\,b^{-1}).
\]
Equivalently,
\begin{equation}
\label{eq:BunC-gluing}
\Bun_G(C)\ \simeq\
\Bigl[\bigl(\Bun_G(C_1,p)\times \Bun_G(C_2,q)\times LG\bigr)\big/\bigl(L^+G\times L^+G\bigr)\Bigr].
\end{equation}
(One can formulate \eqref{eq:BunC-gluing} invariantly without choosing $z$ by working with
torsors over punctured formal neighborhoods; the displayed model is the usual ``loop group''
presentation after choosing parameters.)

\subsubsection*{Parahoric variant and the refined labels}
Fix a parahoric subgroup $\mathcal P\subset LG$.
For definiteness, you may keep in mind either the hyperspecial parahoric
$\mathcal P=L^+G$ or a maximal parahoric corresponding to a vertex $\eta_i$ of the
fundamental alcove in the affine--Weyl combinatorics.

Replacing the local gauge group $L^+G$ by $\mathcal P$ gives the parahoric gluing stack
\begin{equation}
\label{eq:X-parahoric-gluing}
\Bun_{\mathcal P}(C)
\ :=\
\Bigl[\bigl(\Bun_G(C_1,p)\times \Bun_G(C_2,q)\times LG\bigr)\big/\bigl(\mathcal P\times \mathcal P\bigr)\Bigr],
\end{equation}
with $(a,b)\in\mathcal P\times\mathcal P$ acting as above.

The double coset space $\mathcal P\backslash LG/\mathcal P$ carries a stratification by
$\mathcal P\times\mathcal P$--orbits.  Fix an indexing set $\Omega_{\mathcal P}$ for these
orbits; for example one may take
\[
\Omega_{\mathcal P}\;=\;W_{\mathcal P}\backslash W_{\mathrm{aff}}/W_{\mathcal P},
\]
and denote the orbit corresponding to $w\in\Omega_{\mathcal P}$ by $\mathcal O_w\subset LG$.

\begin{definition}[Orbit--type strata]
\label{def:orbit-strata}
For $w\in\Omega_{\mathcal P}$ define the locally closed substack
\[
\Bun_{\mathcal P}(C)_w
\ :=\
\Bigl[\bigl(\Bun_G(C_1,p)\times \Bun_G(C_2,q)\times \mathcal O_w\bigr)\big/\bigl(\mathcal P\times \mathcal P\bigr)\Bigr]
\ \subset\ \Bun_{\mathcal P}(C).
\]
\end{definition}

\begin{remark}[Why this is genuinely new on nodal curves]
Even if one fixes a ``Shatz type on each component'', the gluing parameter $\gamma\in LG$
may move in distinct $\mathcal P$--double cosets, and the closure relations of these orbits
(Bruhat order in $\Omega_{\mathcal P}$) contribute additional boundary directions at the node.
Thus Shatz data alone does not control the boundary geometry once parahoric/loop--group
gluing is allowed.
\end{remark}

\subsubsection*{Refinement by Shatz types on the components}
Let $\Bun_G(C_i)_{\xi_i}\subset \Bun_G(C_i)$ be the Shatz stratum of instability type $\xi_i$,
and let $\Bun_G(C_i,p)_{\xi_i}$ be its pullback to framed bundles.
(Equivalently: the HN type is a condition on the underlying bundle and does not depend on the framing.)

\begin{definition}[Refined strata]
\label{def:refined-strata}
A \emph{refined label} is a triple
\[
\alpha=(\xi_1,\xi_2,w),
\qquad \xi_i\ \text{dominant rational coweights},\quad w\in\Omega_{\mathcal P}.
\]
Define the refined stratum
\begin{equation}
\label{eq:X-alpha}
\mathcal X_\alpha
\ :=\
\Bigl[\bigl(\Bun_G(C_1,p)_{\xi_1}\times \Bun_G(C_2,q)_{\xi_2}\times \mathcal O_w\bigr)\big/\bigl(\mathcal P\times \mathcal P\bigr)\Bigr]
\ \subset\ \Bun_{\mathcal P}(C).
\end{equation}
\end{definition}

\begin{remark}[Closure order]
Let $\le$ denote the partial order on Shatz types (dominance order) and let $\le_{\mathrm{Br}}$
denote the Bruhat order on $\Omega_{\mathcal P}$.
In all standard situations (e.g.\ hyperspecial $\mathcal P=L^+G$ or more generally any parahoric),
one expects the closure relations to satisfy
\[
\overline{\mathcal X_{\xi_1,\xi_2,w}}
\ \supset\
\mathcal X_{\xi_1',\xi_2',w'}
\qquad\Longleftrightarrow\qquad
\xi_1'\le \xi_1,\ \xi_2'\le \xi_2,\ \text{and } w'\le_{\mathrm{Br}} w.
\]
In particular, for a fixed bound $\Xi_1,\Xi_2$ and a fixed $w_0$, the union
\[
\mathcal X_{\le(\Xi_1,\Xi_2,w_0)}
:=
\bigcup_{\xi_1\le\Xi_1,\ \xi_2\le\Xi_2,\ w\le_{\mathrm{Br}} w_0}\ \mathcal X_{\xi_1,\xi_2,w}
\]
is a \emph{finite union} of strata (the Bruhat interval is finite, and Shatz truncations are finite).
This is the basic finite--type truncation used for index finiteness arguments.
\end{remark}

\subsubsection*{A workable virtual normal complex for $\mathcal X_\alpha\subset \mathcal X_{\le\alpha}$}

Write $\pi_i:C_i\times \Bun_G(C_i)\to \Bun_G(C_i)$ for the projections,
and let $\mathcal E_i$ be the universal $G$--bundle on $C_i\times \Bun_G(C_i)$.
Let $P_{\xi_i}\subset G$ be the canonical parabolic of type $\xi_i$ and $G_{\xi_i}$ its Levi.

On the smooth--curve side, the Teleman--Woodward normal complex for the Shatz stratum
is (after pulling back to the semistable Levi core)
\[
\nu_{\xi_i}^{(i)}
\ :=\
R(\pi_i)_*\bigl(\mathcal E_i^*(\mf g/\mf g_{\xi_i})\bigr)[1].
\]

On the node side, the orbit closure $\overline{\mathcal O_w}$ carries a normal complex
for the locally closed embedding $\mathcal O_w\hookrightarrow \overline{\mathcal O_w}$:
\[
\nu_w^{\mathrm{node}}
\ :=\
\mathbb L_{\mathcal O_w/\overline{\mathcal O_w}}[-1],
\]
a perfect complex concentrated in degrees $[-1,0]$ whenever $\overline{\mathcal O_w}$
is normal with rational singularities (as happens for Schubert varieties in the standard cases).
Pull this back to $\mathcal X_\alpha$ via the projection to the $LG$--factor.

\begin{definition}[Candidate normal complex on the refined stratum]
\label{def:nu-alpha}
Define the (virtual) normal complex on $\mathcal X_\alpha$ by
\begin{equation}
\label{eq:nu-alpha}
\nu_\alpha
\ :=\
\nu_{\xi_1}^{(1)}\ \oplus\ \nu_{\xi_2}^{(2)}\ \oplus\ \nu_w^{\mathrm{node}}.
\end{equation}
\end{definition}

\begin{remark}[Justification for \eqref{eq:nu-alpha}]
Heuristically, the refined stratum $\mathcal X_\alpha$ is cut out by three independent
conditions:
(i) the Shatz condition on $C_1$, (ii) the Shatz condition on $C_2$, and
(iii) the orbit--type condition at the node (the $\mathcal P$--double coset of the gluing).
Infinitesimally these contribute transverse deformation directions:
the first two are measured by the smooth--curve deformation theory of $G$--bundles
(TW's $\nu_{\xi_i}^{(i)}$), and the third is measured by the deformation theory of the
orbit embedding in the local loop--group parameter space.
The direct sum in \eqref{eq:nu-alpha} is the cleanest way to package this additivity;
it becomes canonical after choosing a presentation \eqref{eq:X-parahoric-gluing} and
identifying tangent complexes via the standard quotient formula
$T_{[X/H]}\simeq [T_X\to \mf h]$.
\end{remark}

\subsubsection*{Local cohomology terms for the refined filtration}
Fix an ordering of refined labels so that $\mathcal X_{\le\alpha}$ is open and
$\mathcal X_\alpha$ is locally closed with complement a union of ``more unstable'' labels.
Let $i_\alpha:\mathcal X_\alpha\hookrightarrow \mathcal X_{\le\alpha}$ be the inclusion.

For a sheaf/complex $\mathcal E$ on $\mathcal X_{\le\alpha}$ define
\[
R\Gamma_{\mathcal X_\alpha}(\mathcal X_{\le\alpha},\mathcal E)
:=R\Gamma\bigl(\mathcal X_{\le\alpha},\,R\Gamma_{\mathcal X_\alpha}(\mathcal E)\bigr),
\qquad
\mathcal R_\alpha(\mathcal E):=i_\alpha^!(\mathcal E)[-\codim(\mathcal X_\alpha)].
\]

Under standard purity hypotheses (e.g.\ $\mathcal X_\alpha$ regularly embedded in a smooth
ambient presentation of $\mathcal X_{\le\alpha}$, or more generally lci in the derived sense),
one expects an identification parallel to the smooth--curve story:
\begin{equation}
\label{eq:localcoh-alpha}
R\Gamma_{\mathcal X_\alpha}(\mathcal X_{\le\alpha},\mathcal E)
\ \simeq\
R\Gamma\Bigl(\mathcal X_\alpha,\ \mathcal E|_{\mathcal X_\alpha}\otimes \Eul(\nu_\alpha)^{-1}\Bigr)[d_\alpha],
\end{equation}
where $d_\alpha=\codim(\mathcal X_\alpha,\mathcal X_{\le\alpha})$ and $\Eul(\nu_\alpha)$
denotes the $K$--theoretic Euler class of the perfect complex $\nu_\alpha$.

\begin{remark}[What one must check to use \eqref{eq:localcoh-alpha}]
To make \eqref{eq:localcoh-alpha} a theorem (rather than a guiding formula), one needs:
\begin{enumerate}[(1)]
\item a filtration by finite--type opens $\mathcal X_{\le\alpha}$ and a locally closed stratification
$\mathcal X_{\le\alpha}=\bigsqcup_{\beta\le\alpha}\mathcal X_\beta$ with good closure relations;
\item a purity/local--duality statement identifying $i_\alpha^!$ in terms of a normal complex;
\item perfectness of $\nu_\alpha$ (and compatibility with base change over the global base);
\item a $\Gm$--grading on $\nu_\alpha$ (from Shatz $1$--PS data on the components and the standard
weight grading on the Schubert normal directions) so that a \emph{polarized inverse Euler class}
can be defined in a completed equivariant $K$--theory;
\item an admissibility condition ensuring eventual vanishing of the $\Gm$--weight--$0$ part of
$\mathcal E\otimes \Eul(\nu_\alpha)^{-1}_+$ for $\alpha$ sufficiently unstable.
\end{enumerate}
The one--node model above is useful because each item can be tested explicitly:
(2)--(4) reduce to the known smooth--curve Shatz theory on $C_i$ and to standard properties
of Schubert strata in $\mathcal P\backslash LG/\mathcal P$.
\end{remark}
\section{References}
\begin{enumerate}
    \bibitem{TelemanWoodward} Teleman, C., \& Woodward, C. (2012). The index formula on the moduli of G-bundles. Annals of Mathematics, 176(2), 601-77.
\end{enumerate}
\end{document}