\documentclass[12pt]{article}
\usepackage[english]{babel}
\usepackage[utf8x]{inputenc}
\usepackage[T1]{fontenc}
\usepackage{listings}
\usepackage{bookmark}
\usepackage{tikz}

\makeatletter
\def\input@path{{../../style/}}
\makeatother

\usepackage{../../style/quiver}
\makeatletter
\def\input@path{{../../style/}}
\makeatother

\usepackage{../../style/scribe}
\usepackage{fancyhdr}

\usepackage{parskip} % Automatically respects blank lines
\setlength{\parskip}{1em} % Adds more space between paragraphs
\setlength{\parindent}{0pt} % Removes paragraph indentation

\begin{document}


\lhead{Songyu Ye}
\rhead{\today}
\cfoot{\thepage}

\title{Deformation, formal moduli, and algebraic spaces}

\author{Songyu Ye}
\date{\today}
\maketitle


\begin{abstract}
    In this note, we introduce basic ideas central to deformation theory and moduli problems, leading naturally to the notion of algebraic spaces. We discuss the subtleties of the étale topology and how it differs from the Zariski topology, particularly in the context of algebraic spaces. We also explore Artin's criteria for contractions of curves on surfaces and his algebraic approximation theorem, which are fundamental results in the theory of algebraic spaces and their applications to moduli problems.
\end{abstract}

\tableofcontents


\section{The $T^i$ functors}
The $T^i$ functors are a family of functors that measure the infinitesimal deformation theory of a morphism of rings. They are defined using the cotangent complex, which is a fundamental object in deformation theory and homological algebra. We give the definition of the $T^i$ functors ($i=0,1,2$) following the construction given by Lichtenbaum and Schlessinger.
\begin{definition}
    Let $A\to B$ be a homomorphism of rings and let $M$ be a $B$--module. We will
    construct the groups $T^i(B/A,M)$ for $i=0,1,2$. The rings are assumed to be
    commutative with identity, but we do not impose any finiteness conditions yet.

    \medskip

    First choose a polynomial ring
    \[
        R = A[x]
    \]
    in a set of variables $x=\{x_i\}$ (possibly infinite) such that $B$ can be
    written as a quotient of $R$ as an $A$--algebra. Let $I$ be the ideal defining
    $B$, so that we have an exact sequence
    \[
        0 \longrightarrow I \longrightarrow R \longrightarrow B \longrightarrow 0.
    \]

    Second choose a free $R$--module $F$ and a surjection
    \[
        j:F \twoheadrightarrow I \longrightarrow 0,
    \]
    and let $Q=\ker(j)$, so that we have an exact sequence
    \[
        0 \longrightarrow Q \longrightarrow F \xrightarrow{\,j\,} I \longrightarrow 0.
    \]

    Having chosen $R$ and $F$ as above, the construction proceeds with no further
    choices. Let $F_0$ be the submodule of $F$ generated by all ``Koszul relations''
    of the form $j(a)b-j(b)a$ for $a,b\in F$. Note that $j(F_0)=0$, so $F_0\subset Q$.

    \medskip

    We define a complex of $B$--modules, called the \textbf{cotangent complex},
    \[
        L_\bullet:\qquad
        L_2 \xrightarrow{\,d_2\,} L_1 \xrightarrow{\,d_1\,} L_0,
    \]
    as follows.
    \begin{itemize}
        \item Take $L_2 = Q/F_0$.  Why is $L_2$ a $B$--module? A priori it is an
              $R$--module. But if $x\in I$ and $a\in Q$, we can write $x=j(x')$ for some
              $x'\in F$ and then
              \[
                  xa=j(x')a \equiv j(a)x' \pmod{F_0}.
              \]
              But $j(a)=0$ since $a\in Q$, hence $xa=0$ in $Q/F_0$. Therefore the action of
              $I$ is trivial and $L_2$ is naturally a $B=R/I$--module.

        \item Take $L_1 = F\otimes_R B = F/IF$, and let $d_2:L_2\to L_1$ be the map
              induced from the inclusion $Q\hookrightarrow F$.

        \item Take $L_0 = \Omega_{R/A}\otimes_R B$, where $\Omega_{R/A}$ is the module of
              relative differentials. To define $d_1$, first map $L_1$ to $I/I^2$, then apply
              the universal derivation $d:R\to \Omega_{R/A}$, which induces a $B$--module map
              \[
                  I/I^2 \longrightarrow L_0,
              \]
              and define $d_1$ as the composite $L_1\to I/I^2\to L_0$.
    \end{itemize}
    Clearly $d_1d_2=0$, so this is a complex of $B$--modules. Note also that $L_1$
    and $L_0$ are free $B$--modules: $L_1$ is free because it is defined from the
    free $R$--module $F$; $L_0$ is free because $R$ is a polynomial ring over $A$ and
    so $\Omega_{R/A}$ is a free $R$--module.
\end{definition}

\begin{definition}
    For any $B$--module $M$ we define
    \[
        T^i(B/A,M) = h^i\!\left(\Hom_B(L_\bullet,M)\right),
    \]
    the cohomology modules of the complex of homomorphisms from $L_\bullet$ to $M$.
\end{definition}

To show that these modules are well--defined (up to isomorphism), we must verify
that they are independent of the choices made in the construction. The following two lemmas establish this independence, see Hartshorne's book on deformation theory for details.

\begin{lemma}
    The modules $T^i(B/A,M)$ constructed above are independent of the choice of
    $F$ (keeping $R$ fixed).
\end{lemma}

\begin{lemma}
    The modules $T^i(B/A,M)$ are independent of the choice of $R$.
\end{lemma}

We are now ready to state the main properties of the $T^i$ functors and give examples of their computation. Hartshorne proves these properties by choose an appropriate resolution and then applying the definition of $T^i$ to that resolution. The examples are more interesting to study.
\begin{theorem}
    Let $A\to B$ be a homomorphism of rings and let $M$ be a $B$--module. The functors $T^i(B/A,M)$ have the following properties:
    \begin{enumerate}
        \item $T^i$ is a covariant addivive functor from the category of $B$--modules to itself.
        \item There is a nine term exact sequence relating the $T^i$ functors for a short exact sequence of $B$--modules. In particular, if $0\to M'\to M\to M''\to 0$ is a short exact sequence of $B$--modules, then there is a long exact sequence
              \[\cdots \to T^i(B/A,M') \to T^i(B/A,M) \to T^i(B/A,M'') \to T^{i+1}(B/A,M') \to \cdots\]
        \item Given two ring homomorphisms $A\to B\to C$ and $M$ a $C$--module, there is an exact sequence
              \[\cdots \to T^i(C/B,M) \to T^i(C/A,M) \to T^i(B/A,M) \to T^{i+1}(C/B,M) \to \cdots\]
        \item If $B$ is a polynomial ring over $A$, then $T^i(B/A,M) = 0$
              for $i = 1,2$ and for all $M$.
        \item For any map of rings $A\to B$ and any $B$--module $M$, there is a natural isomorphism
              \[T^0(B/A,M) = \Hom_{B}(\Omega_{B/A},M) =  \Der_A(B,M),\]
              where $\Der_A(B,M)$ is the module of $A$--derivations from $B$.
        \item If $A$ is a local ring and $B = A/I$ where $I$ is cut out by a regular sequence, then $T^2(B/A,M) = 0$ for all $M$. This in fact characterizes regular sequences.
        \item If $I$ is the kernel of a surjective homomorphism $A\to B$, then $T^0(B/A,M) = 0$ because there are no nontrivial derivations from $B$ to $M$ that factor through $A$ ($A$-linearity forces the derivation to vanish on $A$ and hence on $B$). Moreover $T^1(B/A,M) = \Hom_B(I/I^2,M)$ is the normal module of $B$ over $A$.
        \item
              Suppose $A=k[x_1,\dots,x_n]$ and $B=A/I$. Then for any $B$--module $M$ there is
              an exact sequence
              \[
                  0 \longrightarrow T^0(B/k,M)
                  \longrightarrow \Hom_B(\Omega_{A/k}\otimes_A B,\,M)
                  \longrightarrow \Hom_B(I/I^2,\,M)
                  \longrightarrow T^1(B/k,M)
                  \longrightarrow 0
              \]
              and an isomorphism
              \[
                  T^2(B/A,M)\xrightarrow{\ \sim\ } T^2(B/k,M).
              \]
    \end{enumerate}
\end{theorem}

\begin{example}
    Let $B=k[x,y]/(xy)$. We will compute that $T^1(B/k,M)=M\otimes k$ and $T^2(B/k,M)=0$
    for any $B$--module $M$. Let $R=k[x,y]$ and $I=(xy)$, so that $B=R/I$. For $B=k[x,y]/(xy)$, we have $I=(xy)$ is principal, so $I/I^2\cong B\cdot \overline{xy}$ is free of rank $1$ over $B$. Thus
    \[
        \Hom_B(I/I^2,M)\cong M.
    \]
    Also $\Omega_{R/k}\cong R\,dx\oplus R\,dy$, so
    \[
        \Omega_{R/k}\otimes_R B\cong B\,dx\oplus B\,dy,
        \quad
        \Hom_B(\Omega_{R/k}\otimes B,M)\cong M\oplus M,
    \]
    by sending a $B$-linear map $\varphi$ to $(\varphi(dx),\varphi(dy))$.

    The map $d:I/I^2\to \Omega_{R/k}\otimes B$ sends the generator $\overline{xy}$ to
    \[
        d(xy)=y\,dx+x\,dy.
    \]
    Therefore $d^\vee: M\oplus M\to M$ is
    \[
        (m_x,m_y)\longmapsto y\,m_x+x\,m_y.
    \]
    Its image is $xM+yM$. Hence
    \[
        T^1(B/k,M)\cong \coker(d^\vee)\cong M/(xM+yM)=M/(x,y)M.
    \]
    But $B/(x,y)\cong k$, so
    \[
        M/(x,y)M\cong M\otimes_B B/(x,y)\cong M\otimes_B k.
    \] The calculation of $T^2(B/k,M)$ follows directly from the fact that $L_2 = Q/F_0$ where $Q$ is the kernel of choice of free resolution $Q \hookrightarrow F \twoheadrightarrow I$ and $F_0$ is the submodule generated by Koszul relations. In this case, since $I$ is principal, we can take $F$ to be a free module of rank $1$ generated by an element $e$ mapping to $xy$. Then $Q=0$ and hence $L_2=0$, so $T^2(B/k,M)=h^2(\Hom_B(L_\bullet,M))=0$.

    More generally, the same argument shows if $B = k[x,y]/(f)$ for some polynomial $f$, then \[T^1(B/k,M) \cong M/(f_x,f_y)M\] where $f_x$ and $f_y$ are the partial derivatives of $f$ with respect to $x$ and $y$ and $T^2(B/k,M) = 0$.
\end{example}

\begin{example}
    Let
    \[
        B=k[x,y]/(x^2,xy,y^2),\qquad \mathfrak m=(x,y)\subset B.
    \]
    Then $\mathfrak m^2=0$ and, as a $k$--vector space,
    \[
        B\cong k\oplus kx\oplus ky.
    \]
    We compute $T^i(B/k,B)$ using the Lichtenbaum--Schlessinger construction.

    Set $R=k[x,y]$ and $I=(x^2,xy,y^2)\subset R$, so $B=R/I$. Take the free
    $R$--module $F=R^{\oplus 3}$ with basis $e_1,e_2,e_3$ and define a surjection
    \[
        j:F\twoheadrightarrow I,
        \qquad
        j(e_1)=x^2,\; j(e_2)=xy,\; j(e_3)=y^2.
    \]
    The kernel $Q=\ker(j)$ is generated by the two evident syzygies
    \[
        s_1:=y e_1-x e_2,\qquad s_2:=y e_2-x e_3.
    \]
    The Koszul submodule $F_0\subset Q$ is generated by
    \[
        j(e_1)e_2-j(e_2)e_1= x^2 e_2-xy e_1 = -x s_1,\qquad
        j(e_2)e_3-j(e_3)e_2= xy e_3-y^2 e_2 = -y s_2,
    \]
    \[
        j(e_1)e_3-j(e_3)e_1= x^2 e_3-y^2 e_1 = -(y s_1+x s_2).
    \]
    Hence the cotangent complex $L_\bullet$ has the form
    \[
        L_2\xrightarrow{d_2} L_1\xrightarrow{d_1} L_0,
        \qquad
        L_2=Q/F_0,\; L_1=F\otimes_R B\cong B^3,\; L_0=\Omega_{R/k}\otimes_R B\cong B^2.
    \]

    \smallskip
    \noindent\textbf{The differentials.}
    Write $\overline{e}_i$ for the image of $e_i$ in $L_1\cong B^3$.
    Then $d_1$ is induced by the universal derivation $d:R\to\Omega_{R/k}$, hence
    \[
        d_1(\overline e_1)=d(x^2)=2x\,dx,\qquad
        d_1(\overline e_2)=d(xy)=y\,dx+x\,dy,\qquad
        d_1(\overline e_3)=d(y^2)=2y\,dy.
    \]
    Moreover $d_2$ is induced by the inclusion $Q\hookrightarrow F$, so modulo $I$
    we have
    \[
        d_2(\overline{s}_1)=y\,\overline e_1-x\,\overline e_2,\qquad
        d_2(\overline{s}_2)=y\,\overline e_2-x\,\overline e_3.
    \]

    \smallskip
    \noindent\textbf{Computing $T^0$.}
    A $B$--linear map $\varphi:L_0\to B$ is determined by $a:=\varphi(dx)$ and
    $b:=\varphi(dy)$, so $\Hom_B(L_0,B)\cong B^2$. Under this identification,
    $d_1^\vee:\Hom_B(L_0,B)\to\Hom_B(L_1,B)\cong B^3$ is
    \[
        (a,b)\longmapsto (2x a,\; y a+x b,\; 2y b).
    \]
    Writing $a=a_0+a_1x+a_2y$ and $b=b_0+b_1x+b_2y$ in the basis $\{1,x,y\}$ and
    using $\mathfrak m^2=0$, we get
    \[
        2x a = 2a_0 x,\qquad y a+x b = a_0 y+b_0 x,\qquad 2y b=2b_0 y.
    \]
    Thus $d_1^\vee(a,b)=0$ iff $a_0=b_0=0$, i.e. iff $a,b\in\mathfrak m$. Hence
    \[
        T^0(B/k,B)=\ker(d_1^\vee)\cong \mathfrak m\oplus \mathfrak m\cong k^4.
    \]

    \smallskip
    \noindent\textbf{Computing $T^1$.}
    Identify $\Hom_B(L_1,B)\cong B^3$ by sending a $B$--linear map $\psi$ to the
    triple $(u_1,u_2,u_3)$ where $u_i=\psi(\overline e_i)$. Then the map
    $d_2^\vee:B^3\to\Hom_B(L_2,B)$ is determined by
    \[
        (d_2^\vee\psi)(\overline s_1)= y u_1-x u_2,\qquad (d_2^\vee\psi)(\overline s_2)= y u_2-x u_3.
    \]
    Since $\mathfrak m^2=0$, the condition $d_2^\vee\psi=0$ forces the constant
    terms of $u_1,u_2,u_3$ to vanish, and imposes no further constraints. Hence
    \[
        \ker(d_2^\vee)\cong \mathfrak m^3,
        \qquad \dim_k\ker(d_2^\vee)=6.
    \]
    On the other hand, $\mathrm{im}(d_1^\vee)$ depends only on the constants
    $(a_0,b_0)\in k^2$ and is spanned by the two vectors
    \[
        (2x,\;y,\;0),\qquad (0,\;x,\;2y)\in B^3.
    \]
    Thus $\dim_k\mathrm{im}(d_1^\vee)=2$, and therefore
    \[
        T^1(B/k,B)=\ker(d_2^\vee)/\mathrm{im}(d_1^\vee)\cong k^{6-2}=k^4.
    \]

    \smallskip
    \noindent\textbf{Computing $T^2$.}
    We have $T^2(B/k,B)=\coker(d_2^\vee)$. Note that $\Hom_B(L_2,B)$ is a
    $k$--vector space of dimension $4$ (indeed a $B$--linear map $L_2\to B$ is
    determined by the images of $\overline s_1,\overline s_2$ in $\mathfrak m$, so
    $\Hom_B(L_2,B)\cong \mathfrak m\oplus\mathfrak m$). Moreover the image of
    $d_2^\vee$ depends only on the constant terms $(c_1,c_2,c_3)\in k^3$ of
    $(u_1,u_2,u_3)$:
    \[
        (u_1,u_2,u_3)\mapsto (c_1 y-c_2 x,\; c_2 y-c_3 x)\in \mathfrak m\oplus\mathfrak m.
    \]
    Hence $\dim_k\mathrm{im}(d_2^\vee)=3$, so
    \[
        \dim_k T^2(B/k,B)=4-3=1,
        \qquad\text{i.e. }\quad T^2(B/k,B)\cong k.
    \]

    In summary,
    \[
        T^0(B/k,B)\cong k^4,\qquad T^1(B/k,B)\cong k^4,\qquad T^2(B/k,B)\cong k.
    \]
\end{example}

\begin{proposition}
    The construction of the $T^i$ functors is compatible with localization, and thus defines sheaves $T^i(X/Y,\mathcal F)$ for any morphism of schemes $f:X\to Y$ and any sheaf $\mathcal F$ of $\mathcal O_X$-modules, such that for any open affine $V\subseteq Y$ and any open affine $U\subseteq f^{-1}(V)$, where $\mathcal F|_U=\widetilde{M}$, the sections of $T^i(X/Y,\mathcal F)$ over $U$ give $T^i(U/V,M)$.
\end{proposition}

\begin{theorem}
    Let $X = \Spec B$ be an affine scheme over an algebraically
    closed field $k$. Then $X$ is nonsingular if and only if $T^1(B/k,M) = 0$ for all
    $B$-modules $M$. Furthermore, if $X$ is nonsingular, then also $T^2(B/k,M) = 0$
    for all $M$.
\end{theorem}

\begin{proof}
    Write $B$ as a quotient of a polynomial ring $A=k[x_1,\dots,x_n]$ over $k$.
    Then $\Spec A$ is nonsingular and so $X$ is nonsingular if and only if the conormal sequence
    \[
        0\longrightarrow I/I^2
        \longrightarrow \Omega_{A/k}\otimes_A B
        \longrightarrow \Omega_{B/k}
        \longrightarrow 0
    \]
    is exact and $\Omega_{B/k}$ is locally free, i.e.\ a projective $B$--module.
    Since $\Omega_{A/k}$ is a free $A$--module, the sequence splits, so we see that
    $X$ is nonsingular if and only if this sequence is split exact. By the exact sequence
    $T^1(B/k,M)=0$ for all $M$ if and only if the map
    \[
        \Hom_B(\Omega_{A/k}\otimes_A B,M)\longrightarrow \Hom_B(I/I^2,M)
    \]
    is surjective for all $M$, and this is equivalent to the splitting of the
    sequence above (just consider the case $M=I/I^2$). Thus $X$ is nonsingular if
    and only if $T^1(B/k,M)=0$ for all $M$.

    For the vanishing of $T^2(B/k,M)$, suppose $X$ is nonsingular. We have that
    \[
        T^2(B/k,M)\cong T^2(B/A,M).
    \]
    Localizing at any point $x\in X$, the ideal $I_x$ is generated by
    $n-r=\dim A-\dim B$ elements in the regular local ring $A_x$. Hence these
    generators form a regular sequence, and thus we see that
    \[
        T^2(B_x/A_x,M)=0
    \]
    for all $B_x$--modules $M$. Thus $T^2(B/A,M)=0$ by localization (recall that the $T^i$ functors are compatible with localization), and therefore $T^2(B/k,M)=0$ for all $M$.
\end{proof}

Localization yields the following corollary:

\begin{corollary}
    Let $B$ be a local $k$-algebra with residue field $k$ algebraically
    closed. Then $B$ is a regular local ring if and only if $T^1(B/k,M) = 0$ for all
    $B$-modules $M$, and in this case $T^2(B/k,M) = 0$ for all $M$.
\end{corollary}

We also have the relative version of the above result:

From this theorem we can deduce a relative version. We say that a morphism of
finite type $f:X\to Y$ of noetherian schemes is \textbf{smooth} if $f$ is flat,
and for every point $y\in Y$, the geometric fiber
\[
    X_y\otimes_{k(y)}\overline{k(y)}
\]
is nonsingular over $\overline{k(y)}$, where $\overline{k(y)}$ is the algebraic
closure of $k(y)$ (cf.~\cite[III,~10.2]{EGA}).

\begin{theorem}[4.11]
    A morphism of finite type $f:X\to Y$ of noetherian schemes is smooth if and only
    if it is flat and
    \[
        T^1(X/Y,\mathcal F)=0
    \]
    for all coherent sheaves $\mathcal F$ on $X$. Furthermore, if $f$ is smooth,
    then also
    \[
        T^2(X/Y,\mathcal F)=0
    \]
    for all $\mathcal F$.
\end{theorem}


We observe that the functor $T^{1}(B/A,-)$ is an additive functor from finitely
generated $B$--modules to finitely generated $B$--modules, and is
\textbf{semi--exact} in the sense that to each short exact sequence of modules it
gives a sequence of three modules that is exact in the middle.  It follows from
the lemma of Dévissage below (4.12) that $T^{1}(B/A,M)=0$ for all finitely
generated $B$--modules, and hence for all $B$--modules, since the $T^{i}$
functors commute with direct limits.  The same argument shows also that
$T^{2}(B/A,M)=0$ for all $M$.

\begin{lemma}[4.12, Dévissage]
    Let $B$ be a noetherian ring, and let $F$ be a semi--exact additive functor
    from finitely generated $B$--modules to finitely generated $B$--modules.
    Assume that $F(B/\mathfrak m)=0$ for every maximal ideal $\mathfrak m$ of $B$.
    Then $F(M)=0$ for all finitely generated $B$--modules $M$.
\end{lemma}

\begin{proof}
    Any finitely generated $B$--module $M$ has a composition series whose quotients
    are $B/\mathfrak p_i$ for various prime ideals $\mathfrak p_i$.  By
    semi--exactness, it is sufficient to show that $F$ vanishes on each of these.
    Thus we may assume $M=B/\mathfrak p$.

    We proceed by induction on the dimension of the support of $M$.  If
    $\dim \operatorname{Supp} M=0$, then $M$ is just $B/\mathfrak m$ for some
    maximal ideal $\mathfrak m$, and $F(M)=0$ by hypothesis.

    For the general case, let $M=B/\mathfrak p$ have dimension $r>0$.  For any
    maximal ideal $\mathfrak m\supset \mathfrak p$, choose an element
    $t\in \mathfrak m\setminus \mathfrak p$.  Then $t$ is a non--zero--divisor on
    $M$ and we can write an exact sequence
    \[
        0\longrightarrow M
        \xrightarrow{\;t\;}
        M\longrightarrow M'\longrightarrow 0,
    \]
    where $M'$ is a module with support of dimension $<r$.  Hence by the induction
    hypothesis, $F(M')=0$, and we get a surjection
    \[
        F(M)\xrightarrow{\;t\;} F(M)\longrightarrow 0.
    \]
    It follows from Nakayama's lemma that $F(M)$ localized at $\mathfrak m$ is zero.
    This holds for any $\mathfrak m\supset \mathfrak p$, i.e.\ for any point of
    $\Spec B/\mathfrak p$, and so $F(M)=0$.
\end{proof}

If $A$ is a regular local ring and $B=A/I$ is a quotient, we say that $B$ is a
\textbf{local complete intersection in $A$} if the ideal $I$ can be generated by
$\dim A-\dim B$ elements.

\begin{theorem}[4.13]
    Let $A$ be a regular local $k$--algebra, with residue field $k$ algebraically
    closed, and let $B=A/I$ be a quotient of $A$.  Then $B$ is a local complete
    intersection in $A$ if and only if $T^{2}(B/k,M)=0$ for all $B$--modules $M$.
\end{theorem}

\begin{exercise}[4.1]
    Let $A$ be a local $k$-algebra with residue field $k$. Let $f:A\to A$ be a
    $k$-algebra homomorphism inducing an isomorphism $A/\mathfrak m^2\to A/\mathfrak
        m^2$, where $\mathfrak m$ is the maximal ideal. Show that $f$ is an
    isomorphism.
\end{exercise}

\begin{proof}
    Since $f$ is a local homomorphism (it induces the identity on the residue field
    $A/\mathfrak m\cong k$), we have $f(\mathfrak m)\subseteq \mathfrak m$. The
    assumption that $A/\mathfrak m^2\to A/\mathfrak m^2$ is an isomorphism implies
    that the induced map on cotangent spaces
    \[
        \mathfrak m/\mathfrak m^2 \longrightarrow \mathfrak m/\mathfrak m^2
    \]
    is an isomorphism. By Nakayama's lemma, this forces $f(\mathfrak m)=\mathfrak
        m$; in particular, $f$ is surjective on $\mathfrak m$.

    We claim that $f$ is surjective on $A$. Let $a\in A$. Modulo $\mathfrak m$ the
    map is the identity on $k$, so we can lift the residue class of $a$ and reduce
    to the case $a\in\mathfrak m$. For $a\in\mathfrak m$, surjectivity of
    $f:\mathfrak m\to\mathfrak m$ gives $b\in\mathfrak m$ with $f(b)=a$. Hence $f$ is
    surjective.

    To prove injectivity, let $x\in\ker(f)$. Then $x\in\mathfrak m$ since $f$ is the
    identity on $A/\mathfrak m$. Moreover, the induced map on $A/\mathfrak m^2$ is
    injective, so $x\in\mathfrak m^2$. Applying the same argument to the induced
    map on $\mathfrak m^n/\mathfrak m^{n+1}$ (obtained by functoriality from
    $f(\mathfrak m)=\mathfrak m$), we get inductively that $x\in\mathfrak m^n$ for
    all $n$. If $A$ is artinian, this forces $x=0$ since $\mathfrak m^N=0$ for $N\gg
        0$. This forces
    $x=0$ because $\bigcap_{n\ge 1}\mathfrak m^n=0$. Thus $f$ is injective, hence an
    automorphism.
\end{proof}

\begin{exercise}[4.2]
    Let $A$ be a local artinian $k$-algebra. Let $X_1$ and $X_2$ be schemes of
    finite type, flat over $A$, and let $f:X_1\to X_2$ be an $A$-morphism which
    induces an isomorphism on closed fibers
    \[
        f\otimes_A k:\ X_1\times_A k \longrightarrow X_2\times_A k.
    \]
    Show that $f$ is an isomorphism.
\end{exercise}

\begin{proof}
    Let $\mathfrak m$ be the maximal ideal of $A$. Since $A$ is artinian, there is
    $N$ with $\mathfrak m^{N}=0$. Set $A_n:=A/\mathfrak m^{n+1}$ and write
    $X_{i,n}:=X_i\times_A A_n$ for the $n$-th infinitesimal thickening of the closed
    fiber.

    We prove by induction on $n$ that the base change
    \[
        f_n:\ X_{1,n}\longrightarrow X_{2,n}
    \]
    is an isomorphism. For $n=0$ this is exactly the hypothesis on closed fibers.

    Assume $f_n$ is an isomorphism. Consider the square-zero extension
    $A_{n+1}\twoheadrightarrow A_n$ with kernel $\mathfrak m^{n+1}/\mathfrak
        m^{n+2}$. Because $X_1$ and $X_2$ are flat over $A$, the closed immersions
    $X_{i,n}\hookrightarrow X_{i,n+1}$ are defined by square-zero ideal sheaves, and
    the structure sheaves fit into short exact sequences
    \[
        0\to \mathfrak m^{n+1}/\mathfrak m^{n+2}\otimes_k \mathcal O_{X_{i,0}}
        \to \mathcal O_{X_{i,n+1}}\to \mathcal O_{X_{i,n}}\to 0.
    \]
    The morphism $f_{n+1}$ induces a morphism of these extensions. Since $f_n$ is an
    isomorphism, the only issue is whether the map on the middle terms is an
    isomorphism. But locally on $X_{2,n+1}$ we can choose affine opens and reduce to
    a ring-theoretic statement: if $B_1$ and $B_2$ are flat $A_{n+1}$-algebras of
    finite type and a map $B_2\to B_1$ becomes an isomorphism after tensoring with
    $A_n$, then it is an isomorphism. This follows from Nakayama's lemma applied to
    the kernel and cokernel (which are finitely generated and killed by
    $\mathfrak m^{n+1}$): if $K$ and $C$ denote kernel and cokernel, then
    $K\otimes_{A_{n+1}}A_n=C\otimes_{A_{n+1}}A_n=0$ implies $K=C=0$.

    Hence $f_{n+1}$ is an isomorphism. Induction gives that $f_{N-1}:X_1\to X_2$ is
    an isomorphism.
\end{proof}
The $T^i$ functors are useful because they help us understand deformation theory of arbitrary schemes. We will see that the deformations of an affine scheme
$X = \Spec B$ over an algebraically closed field $k$ are governed by the functor $T^1(B/k,-)$, and the obstructions to deforming $X$ are governed by $T^2(B/k,-)$. In particular, the first order deformations of a nonsingular scheme are controlled by the cohomology group $H^1(X,T_X)$, where $T_X$ is the tangent sheaf of $X$.

\section{Infinitesimal characterisation of smoothness}

Recall the following characterizations of smoothness, one of them being clearly intrinsic.
\begin{definition}
    A scheme $X$ over an algebraically closed field $k$ is \textbf{smooth} if it satisfies any of the following equivalent conditions:
    \begin{enumerate}
        \item $X$ is embedded in some affine space $\mathbb A^n_k$ and the ideal of $X$ is generated by polynomials whose Jacobian matrix has maximal rank at every point of $X$.
        \item If $X$ is finite type over $k$, then $X$ is smooth if and only if the local ring $\mathcal O_{X,x}$ is regular for every $x\in X$, i.e. the Krull dimension of $\mathcal O_{X,x}$ equals the dimension of the Zariski tangent space at $x$.
        \item $X$ is smooth if and only if the cotangent sheaf $\Omega_{X/k}$ is locally free of rank equal to the dimension of $X$.
    \end{enumerate}
\end{definition}
Smooth schemes have the following infinitesimal lifting property. Let $f:Y \to X$ be a morphism of schemes. We say that $f$ has the \textbf{infinitesimal lifting property} or \textbf{formally smooth} if for every infinitesimal thickening $Y \to Y'$ meaning that $Y'$ is a scheme and $Y$ is a closed subscheme of $Y'$ defined by a nilpotent ideal, there exists a morphism $Y' \to X$ making the obvious diagram commute.

It turns out that smoothness can be characterized by the infinitesimal lifting property, at least for morphisms of affine schemes. \begin{theorem}
    Let $X$ be a smooth affine scheme and $Y$ be an affine scheme over an algebraically closed field $k$. Then any morphism $f:Y \to X$ has the infinitesimal lifting property. Conversely, if $X$ is an affine scheme such that any finite morphism $f$ from $\Spec A$ where $A$ is an Artinian local $k$-algebra to $X$ has the infinitesimal lifting property, then $X$ is smooth.
\end{theorem}

This corollary follows immediately from the fact that $T^1(B/k,B)=0$ when $B$ is smooth over $k$, and the fact that we can give an interpretation of $T^1(B/k,B)$ in terms of infinitesimal deformations of $B$ over the dual numbers, as we will see in the next section.

\begin{corollary}
    Let $X$ be a nonsingular affine scheme over $k$. Let $A$ be a local Artin ring over $k$, and let $X'$ be a scheme, flat over $\Spec A$, such that $X' \times_A k$ (where by abuse of notation we mean $X' \times_{\Spec A} \Spec k$) is isomorphic to $X$. Then $X'$ is isomorphic to the trivial deformation $X \times_k A$ of $X$ over $A$.
\end{corollary}


\begin{proof}
    Let $D=k[\varepsilon]/(\varepsilon^{2})$ be the ring of dual numbers.
    An \textbf{infinitesimal deformation} of the $k$-algebra $B$ over $D$ is a flat $D$-algebra $B'$ equipped with an isomorphism
    \[
        B'/\varepsilon B' \;\cong\; B .
    \]

    As a $B$-module any such $B'$ is necessarily isomorphic to $B\oplus \varepsilon B$,
    so the deformation is determined by the choice of a multiplication on $B\oplus \varepsilon B$ lifting the original multiplication on $B$.
    Writing elements as $a+\varepsilon b$, the most general such product has the form
    \[
        (a+\varepsilon b)(c+\varepsilon d)
        = ac + \varepsilon(ad+bc+\varphi(a,c)),
    \]
    where $\varphi:B\times B\to B$ is a $k$-bilinear map.

    Associativity of this product imposes the condition
    \[
        a\,\varphi(b,c)-\varphi(ab,c)
        +\varphi(a,bc)-\varphi(a,b)\,c = 0 ,
    \]
    which is precisely the cocycle condition describing an element of
    \[
        \operatorname{Ext}^{1}_{B}(\Omega_{B/k},B).
    \]

    If we change the chosen identification $B'\simeq B\oplus\varepsilon B$ by
    \[
        a+\varepsilon b \longmapsto a+\varepsilon(b+\alpha(a))
    \]
    for a derivation $\alpha\in\operatorname{Der}_{k}(B,B)$,
    then the function $\varphi$ is replaced by
    \[
        \varphi'=\varphi+\delta(\alpha),
    \]
    so two such bilinear maps determine isomorphic deformations exactly when they differ by a coboundary.

    Consequently the set of isomorphism classes of infinitesimal deformations of $B$ over $D$
    is naturally identified with
    \[
        T^{1}(B/k,B)
        = \operatorname{Ext}^{1}_{B}(\Omega_{B/k},B).
    \]
\end{proof}
\begin{example}


    We sketch the computation of $T^1(B/k,B)=\operatorname{Ext}^1_B(\Omega_{B/k},B)$ for
    \[
        B=k[x,y]/(xy).
    \]

    From the presentation $B=k[x,y]/(f)$ with $f=xy$, we have the standard exact sequence
    \[
        B \xrightarrow{\,df\,} B\,dx\oplus B\,dy \longrightarrow \Omega_{B/k}\longrightarrow 0,
    \]
    where $df = y\,dx + x\,dy$.
    Hence
    \[
        \Omega_{B/k}\cong (B\,dx\oplus B\,dy)/\langle y\,dx+x\,dy\rangle .
    \]

    The above presentation gives a two--term resolution
    \[
        0\longrightarrow B
        \xrightarrow{\,(y,x)\,}
        B^{\oplus 2}
        \longrightarrow \Omega_{B/k}
        \longrightarrow 0.
    \]


    Applying $\operatorname{Hom}_B(-,B)$ gives
    \[
        0\longrightarrow \operatorname{Hom}_B(\Omega_{B/k},B)
        \longrightarrow B^{\oplus 2}
        \xrightarrow{\,(y,x)\,}
        B
        \longrightarrow \operatorname{Ext}^1_B(\Omega_{B/k},B)
        \longrightarrow 0.
    \]

    Thus
    \[
        T^1(B/k,B)\cong B/(x,y).
    \]

    Since $B/(x,y)\cong k$, we obtain
    \[
        T^1(B/k,B)\cong k.
    \]

    A generator of this $k$ corresponds to the deformation of $B$ over the dual numbers
    $k[\varepsilon]/(\varepsilon^2)$ given by
    \[
        k[x,y,\varepsilon]/(xy-\varepsilon),
    \]
    which is the first--order smoothing of the node.

\end{example}


\section{Deformations of rings}
We will see that the global avatar to the $T^1$ functor which classifies extensions of $B$ over the dual numbers is the cohomology group $H^1(X,T_X)$, which classifies deformations of a nonsingular variety $X$ over the dual numbers.

\begin{theorem}
    Let $A$ be a ring, $B$ an $A$-algebra, and $M$ a $B$-module. Then
    equivalence classes of extensions of $B$ by $M$ as $A$-algebras are in natural one-
    to-one correspondence with elements of the group $T^1(B/A,M)$. The trivial
    extension corresponds to the zero element.
\end{theorem}

\begin{corollary}[5.2]
    Let $k$ be a field and let $B$ be a $k$-algebra.
    Then the set of deformations of $B$ over the dual numbers is in natural one-to-one correspondence with the group
    \[
        T^{1}(B/k,B).
    \]
\end{corollary}

\begin{proof}
    This follows from the theorem and the discussion at the beginning of this section,
    which showed that such deformations are in one-to-one correspondence with the $k$-algebra extensions of $B$ by $B$.
\end{proof}

\begin{definition}
    Recall that a deformation of a scheme $X$ over a base scheme $S$ is a flat morphism $\mathcal X \to S$ together with an isomorphism $\mathcal X \times_S \Spec k \cong X$, where $k$ is the residue field of a point in $S$.
\end{definition}

\begin{theorem}[5.3]
    Let $X$ be a nonsingular variety over $k$. Then the deformations of $X$ over the dual numbers
    are in natural one-to-one correspondence with the elements of the group
    \[
        H^1(X,T_X),
    \]
    where $T_X=\mathcal{H}om_X(\Omega_{X/k},\mathcal{O}_X)$ is the tangent sheaf of $X$.
\end{theorem}

\begin{proof}
    Let $X'$ be a deformation of $X$, and let $\mathcal{U}=(U_i)$ be an open affine covering of $X$.
    Over each $U_i$ the induced deformation $U_i'$ is trivial so we can choose an isomorphism
    \[
        \varphi_i : U_i \times_k D \xrightarrow{\;\sim\;} U_i'.
    \]

    Then on $U_{ij}=U_i\cap U_j$ we get an automorphism
    \[
        \psi_{ij}=\varphi_j^{-1}\varphi_i
    \]
    of $U_{ij}\times_k D$, which corresponds to an element
    \[
        \theta_{ij}\in H^0(U_{ij},T_X)
    \]
    since automorphisms of $U\times D$ over $D$ that restrict to the identity on $U$
    correspond exactly to derivations of $\mathcal{O}_U$. By construction, on $U_{ijk}$ we have
    \[
        \theta_{ij}+\theta_{jk}+\theta_{ki}=0,
    \]
    since composition of automorphisms corresponds to addition of derivations.
    Therefore $(\theta_{ij})$ is a Čech $1$-cocycle for the covering $\mathcal{U}$ and the sheaf $T_X$.

    If we replace the original chosen isomorphisms
    \[
        \varphi_i:U_i\times_k D \xrightarrow{\sim} U_i'
    \]
    by some others $\varphi_i'$, then $\varphi_i'^{-1}\varphi_i$ will be an automorphism of
    $U_i\times_k D$ coming from a section
    \[
        \alpha_i\in H^0(U_i,T_X),
    \]
    and the new $\theta_{ij}'$ satisfies
    \[
        \theta_{ij}'=\theta_{ij}+\alpha_i-\alpha_j.
    \]
    So the new $1$-cocycle differs from $\theta_{ij}$ by a coboundary, and we obtain a well-defined element
    \[
        \theta\in \check H^1(\mathcal{U},T_X).
    \]

    Since $\mathcal{U}$ is an open affine covering and $T_X$ is a coherent sheaf, this is equal to the usual cohomology group $H^1(X,T_X)$.
    Clearly $\theta$ is independent of the covering chosen.

    Reversing this process, an element $\theta\in H^1(X,T_X)$ is represented on $\mathcal{U}$ by a $1$-cocycle $\theta_{ij}$, and these $\theta_{ij}$ define automorphisms of the trivial deformations $U_{ij}\times_k D$ that can be glued together to make a global deformation $X'$ of $X$.
    So we see that the deformations of $X$ over $D$ are given by $H^1(X,T_X)$.
\end{proof}

\begin{example}[5.3.1]
    If $X=\mathbf{P}^n$ for $n\ge 1$, then $H^{1}(T_X)=0$, so every deformation of $X$ over the dual numbers is trivial.
    Thus $X$ is an example of a \textbf{rigid scheme}, by which we mean a scheme all of whose deformations over the dual numbers are trivial. Smooth affine schemes are also rigid as we have seen, but there are many singular rigid schemes as well.

    One needs to exercise some caution around this notion of rigid scheme, lest intuition lead one into error.
    One might think, for example, that a rigid scheme has no nontrivial deformations.
    We will show indeed that every deformation over an Artin ring is trivial.
    However, there can be nontrivial \textbf{global} deformations of a rigid affine scheme.
\end{example}

\begin{example}
    Let $k$ be an algebraically closed field, fix $\lambda\in k$, $\lambda\neq 0,1$, and consider the family of affine elliptic curves over $k[t]$ defined by
    \[
        y^{2}=x(x-1)(x-(\lambda+t)).
    \]

    \begin{enumerate}[(a)]
        \item
              This family is not trivial over any neighborhood of $t=0$ because over a field, the $j$-invariant is already determined by any open affine piece of an elliptic curve, and the $j$-invariant varies in this family---cf.\ \cite[IV, \S4]{ref}.

        \item
              The family is still not trivial over the complete local ring $k[[t]]$ at the origin, because one can look at the $j$-invariant over the field of fractions of this ring.

        \item
              The projective completion of this family in $\mathbf{P}^{2}$ is not trivial even over the Artin ring
              \[
                  C=k[t]/t^{n}
                  \qquad (n\ge2),
              \]
              because the computation of the $j$-invariant can be made to work over the ring $C$.

        \item
              However, the \textbf{affine} family over the Artin ring $C$ is trivial for any $n$ because of general theory. Find an explicit isomorphism of this family over $C$ with the trivial family
              \[
                  y^{2}=x(x-1)(x-\lambda).
              \]

              Find $a,b,c,d\in C$ such that the transformation
              \[
                  x'=\frac{ax+b}{cx+d}
              \]
              sends $0,1,\lambda$ to $0,1,\lambda+t$.
              Substitute for $x'$ in the equation
              \[
                  y'^{2}=x'(x'-1)(x'-(\lambda+t))
              \]
              and show that the result can be written as
              \[
                  \frac{y'^{2}(cx+d)^{3}}{a(a-c)(a-c(\lambda+t))}
                  = x(x-1)(x-\lambda).
              \]
              Now, using the fact that $t$ is nilpotent in $C$, show that one can find $f(x,t)$ and $g(x,t)$ in $C[x]$ such that the substitutions
              \[
                  x' = x+t\,f(x,t), \qquad
                  y' = y\,(1+t\,g(x,t)),
              \]
              bring the equation into the form
              \[
                  y^{2}=x(x-1)(x-\lambda).
              \]
              Show that the transformation $(x,y)\mapsto(x',y')$ is an automorphism of the ring $C[x,y]$, so the two families are isomorphic over $C$.
    \end{enumerate}
    This provides an example of a \textbf{nontrivial global deformation of a rigid affine scheme}:
    the special fiber at $t=0$ is the affine line $\mathbf{A}^{1}$, which is rigid,
    yet it occurs as the limit of a nontrivial family of affine elliptic curves.
\end{example}


\section{Functors of Artin rings}
Let $k$ be a fixed algebraically closed ground field, and let $\mathcal C$
be the category of local artinian $k$--algebras with residue field $k$. These rings model infinitesimal thickenings of a $k$-point. We consider a covariant functor
\[
    F : \mathcal C \longrightarrow (\mathrm{Sets}).
\]

One example of such a functor is obtained as follows. Let $R$ be a complete
local $k$--algebra, and for each $A\in\mathcal C$, let $h_R(A)$ be the set of
$k$--algebra homomorphisms $\Hom(R,A)$. For any morphism $A\to B$ in
$\mathcal C$ we get a map of sets
\[
    h_R(A)\longrightarrow h_R(B),
\]
so $h_R$ is a covariant functor from $\mathcal C \to \mathrm{Set}$.

\begin{definition}
    A covariant functor $F:\mathcal C\to\mathrm{Set}$ that is isomorphic to a
    functor of the form $h_R$ for some complete local $k$--algebra $R$ is called
    \textbf{pro--representable}.
\end{definition}

To explain the nature of an isomorphism between $h_R$ and $F$, let us consider
more generally any homomorphism of functors
\[
    \varphi : h_R \longrightarrow F
\]
for a complete local $k$--algebra $R$ with maximal ideal $\mathfrak m$.
In particular, for each $n$ this will give a map
\[
    \varphi_n : \Hom(R,R/\mathfrak m^n)\longrightarrow F(R/\mathfrak m^n),
\]
and the image of the quotient map $R\to R/\mathfrak m^n$ gives an element
\[
    \xi_n\in F(R/\mathfrak m^n).
\]
These elements $\xi_n$ are compatible, in the sense that the natural map
$R/\mathfrak m^{n+1}\to R/\mathfrak m^n$ induces a map of sets
\[
    F(R/\mathfrak m^{n+1})\longrightarrow F(R/\mathfrak m^n)
\]
that sends $\xi_{n+1}$ to $\xi_n$. Thus the collection $\{\xi_n\}$ defines
an element
\[
    \xi\in \varprojlim F(R/\mathfrak m^n).
\]
We will call such a collection $\xi=\{\xi_n\}$ a \textbf{formal family} of $F$
over the ring $R$.

Here it is useful to introduce the category $\widehat{\mathcal C}$ of complete
local $k$--algebras with residue field $k$. The category $\widehat{\mathcal{C}}$ contains
the category $\mathcal C$. Every Artinian local ring is automatically complete for its maximal ideal topology. This is because if $A$ is Artinian local with maximal ideal $\mathfrak{m}$, then $\mathfrak{m}^N = 0$ for some $N$, the $\mathfrak{m}$-adic completion is
\[
    \widehat{A} = \varprojlim A/\mathfrak{m}^n,
\] but for $n \ge N$, we have $A/\mathfrak{m}^n = A$,
so the limit stabilizes and $\widehat{A} \cong A$.

We can extend any functor $F$ on
$\mathcal C$ to a functor $\widehat F$ from $\widehat{\mathcal C}$ to sets by
defining
\[
    \widehat F(R)=\varprojlim F(R/\mathfrak m^n)
\]
for any $R\in\widehat{\mathcal C}$. In this notation, $\widehat F(R)$ is the
set of formal families of $F$ over $R$.

Conversely, a formal family $\xi=\{\xi_n\}$ of $\widehat F(R)$ defines a
homomorphism of functors
\[
    \varphi : h_R \longrightarrow F
\]
as follows. For any $A\in\mathcal C$ and any homomorphism
$f:R\to A$, since $A$ is artinian, it factors through $R/\mathfrak m^n$
for some $n$, say $f=g\circ \pi_n$ where
$\pi_n:R\to R/\mathfrak m^n$ and $g:R/\mathfrak m^n\to A$.
Then let $\varphi(f)$ be the image of $\xi_n$ under the map
$F(g):F(R/\mathfrak m^n)\to F(A)$. It is easy to check that these
constructions are well--defined and inverse to each other, so we have the
following.

\begin{proposition}[15.1]
    If $F$ is a functor $\mathcal C \to \mathrm{Set}$ and $R$ is a
    complete local $k$--algebra with residue field $k$, then there is a natural
    bijection between the set
    \[
        \widehat F(R)
        = \left\{\, \{\xi_n\}\ \big|\ \xi_n\in F(R/\mathfrak m^n) \right\}
    \]
    of formal families and the set of homomorphisms of functors
    \[
        h_R \longrightarrow F .
    \]
\end{proposition}

\begin{definition}
    Thus we say that $F$ is \textbf{pro--representable} if there is an isomorphism
    \[
        \xi : h_R \longrightarrow F
    \]
    for some complete local $k$--algebra $R$, and we can think of $\xi$ as an element of $\widehat F(R)$.
    We say that the pair $(R,\xi)$ pro--represents the functor $F$.
    One can verify easily that if $F$ is pro--representable, the pair $(R,\xi)$
    is unique up to unique isomorphism.
\end{definition}

In many cases of interest, the functors we consider will not be
pro--representable, so we define the weaker notions of having a
versal family and a miniversal family.

\begin{definition}
    Let $F:\mathcal C\to\mathrm{Set}$ be a functor. A pair $(R,\xi)$ with
    $R\in\widehat{\mathcal C}$ and $\xi\in \widehat F(R)$ is a
    \textbf{versal family} for $F$ if the associated map
    \[
        h_R \longrightarrow F
    \]
    is \textbf{strongly surjective}. Here we say that a morphism of functors
    $G\to F$ is strongly surjective if for every $A\in\mathcal C$ the map
    $G(A)\to F(A)$ is surjective, and furthermore, for every surjection
    $B\to A$ in $\mathcal C$, the map $G(B)\longrightarrow G(A)\times_{F(A)} F(B)$ is also surjective.

    In our case this means that given a map $R\to A$ inducing an element
    $\eta\in F(A)$, and given $\theta\in F(B)$ mapping to $\eta$, one can lift
    the map $R\to A$ to a map $R\to B$ inducing $\theta$.
    Some authors call this property of a morphism of functors
    $G\to F$ \textbf{smooth}, by analogy with the infinitesimal lifting property
    of smooth morphisms of schemes.
\end{definition}

If in addition the map $h_R(D)\to F(D)$ is bijective, where
$D=k[t]/t^2$ is the ring of dual numbers, we say that $(R,\xi)$ is a
\textbf{miniversal family}, or that the functor has a
\textbf{pro--representable hull} $(R,\xi)$.
We say that $(R,\xi)$ is a \textbf{universal family} if it pro--represents
the functor $F$. The following proposition explains the significance of
this terminology.

\begin{proposition}
    Let $(R,\xi)$ be a formal family of the functor $F$.
    \begin{enumerate}
        \item[(a)] If $(R,\xi)$ is a versal family, then for any other formal family
              $(S,\eta)$, there is a ring homomorphism $f:R\to S$ such that the induced map
              $\widehat F(R)\to \widehat F(S)$ sends $\xi$ to $\eta$.

        \item[(b)] If $(R,\xi)$ is miniversal, then for any $(S,\eta)$ the map
              $f:R\to S$ of {\rm (a)} induces a unique homomorphism
              \[
                  R/\mathfrak m_R^2 \longrightarrow S/\mathfrak m_S^2.
              \]

        \item[(c)] If $(R,\xi)$ is a universal family, then for any $(S,\eta)$ as in
                  {\rm (a)}, the corresponding map $f:R\to S$ is unique.
    \end{enumerate}
\end{proposition}

\begin{remark} We can understand the above abstract via an analogy with differential geometry.
    \begin{center}


        \begin{tabular}{@{}ll@{}}
            \textbf{Differential geometry} & \textbf{Deformation theory}           \\
            manifold $M$                   & moduli problem $F$                    \\
            point $p$                      & object $X_0$                          \\
            local chart                    & versal deformation                    \\
            smooth map                     & formally smooth map of functors       \\
            tangent space $T_pM$           & $t_F=F(k[\varepsilon]/\varepsilon^2)$ \\
            differential                   & map on dual numbers                   \\
            reparametrization of chart     & automorphisms                         \\
            manifold with stabilizers      & stacky moduli                         \\
            local coordinates              & complete local ring $R$               \\
        \end{tabular}
    \end{center}
    To give a versal family is to give a local chart for the moduli problem with surjective differential. To say that a versal family is miniversal is to say is that the chart is an immersion at the point $p$.
\end{remark}

\begin{example}
    Suppose that $\mathcal F$ is a globally defined contravariant functor from
    $\mathrm{Sch}/k \to \mathrm{Set}$. For example, think of the functor
    $\mathrm{Hilb}$, which to each scheme $S/k$ associates the set of closed
    subschemes of $\mathbb P^n_S$, flat over $S$. Given a particular element
    $X_0\in \mathcal F(k)$, we can define a local functor
    \[
        F:\mathcal C\longrightarrow \mathrm{Set}
    \]
    by taking, for each $A\in\mathcal C$, the subset
    \[
        F(A)=\mathcal F(\Spec A)
    \]
    consisting of those elements $X\in \mathcal F(\Spec A)$ that reduce to
    $X_0\in\mathcal F(k)$. We can and should think of $F$ as the functor of infinitesimal deformations of $X_0$.

    If the global functor $\mathcal F$ is representable, say by a scheme $M$, and if $x_0\in M(k)$ is the point corresponding to $X_0$, then the local functor $F$ will be pro--representable. The representing object is the completed local ring at the chosen point. Let $R = \widehat{\mathcal O}_{M,x_0}$ the completed local ring of $M$ at $x_0$. For any Artin local $A$ with residue field $k$,
    \[
        \Hom_{\text{loc}}(R,A)
        \cong
        \left\{
        \text{morphisms }\Spec A\to M
        \text{ sending closed point to }x_0
        \right\}.
    \]

\end{example}

\begin{example}
    The converse is false because the local functor may be pro--representable
    when the global functor is not representable. Take for example deformations of
    $\mathbb P^1$. It is easy to see that this functor is not representable. The moduli of genus $0$ curves is not a fine moduli space and its failure is witnessed in multiple ways. In particular, $\mathbb P^1$ has huge automorphism group
    $\operatorname{Aut}(\mathbb P^1)=\mathrm{PGL}_2$, and any family of $\mathbb P^1$ is Zariski locally trivial but not canonically trivial. In fact Hirzebruch surfaces are examples of families of $\mathbb P^1$s that are not globally trivial.

    However, all local deformations over Artin rings are trivial since we have $H^1(\mathbb P^1,T_{\mathbb P^1}) = 0$. Thus, for any $A\in\mathcal C$, the set $F(A)$ consists of a single element, so the local functor is pro--represented by the ring $k$.
\end{example}

\begin{example}[Local deformations of plane curve singularities]\label{ex:miniversal-plane-singularity}
    Let $k$ be algebraically closed and let $f(x,y)=0$ define an \textbf{isolated} plane
    curve singularity at the origin. In \cite[Thm.~14.1]{hartshorne-deformation-theory}
    Hartshorne constructs an explicit deformation $X/T$ and proves it is \textbf{miniversal}
    in the following sense.

    Consider a polynomial or power series $f(x,y)\in k[x,y]$ or $k[[x,y]]$ defining a plane curve singularity at the origin. The Tjurina ideal
    \[
        J=(f,f_x,f_y),
    \]
    where $f_x,f_y$ are the partial derivatives, will be primary for the maximal ideal $\mathfrak m=(x,y)$.

    To guess the versal deformation space of this singularity, we take a hint from
    the calculation of the $T^1$--functor, which parametrizes deformations over the
    dual numbers. Let $R=k[x,y]$ and $B=R/(f)$. Recall that
    \[
        T^1(B/k,B)\cong B/(f_x,f_y)\cong R/J.
    \]
    Take polynomials $g_1,\dots,g_r\in R$ whose images in $R/J$ form a $k$--vector
    space basis. Then we take $r$ new variables $t_1,\dots,t_r$ and define a
    deformation $X$ over
    \[
        T=\Spec k[t_1,\dots,t_r]
    \]
    by
    \[
        F(x,y,t)=f(x,y)+\sum_{i=1}^r t_i g_i(x,y)=0.
    \]
    The deformation $X/T$
    constructed above is miniversal in the following sense:
    \begin{enumerate}
        \item[(a)] For any other deformation $X'/S$, with $S=\Spec C$ for $C$ a complete
              local ring, there is a morphism $\varphi:S\to T$ such that $X'$ and
              $X\times_T S$ become isomorphic after completing along the closed fiber over
              $0$.
        \item[(b)] Although $\varphi$ may not be unique, the induced map on Zariski
              tangent spaces of $S$ and $T$ is uniquely determined.
    \end{enumerate}


    \begin{proof}[Idea of the construction]
        Write $S=\Spec C$ with $C$ complete local. Since any infinitesimal deformation
        of a complete intersection is a complete intersection, the fiber $X'_n$ over any
        artinian quotient $C_n$ of $C$ is a complete intersection in $\A^2_{C_n}$ defined
        by a single equation. Passing to the inverse limit over $n$ produces a single
        function
        \[
            G(x,y,s)\in C[x,y]^\wedge
        \]
        (the $\mathfrak m_C$--adic completion) cutting out the completion of $X'$ along
        the closed fiber. Writing $C$ as a quotient of a formal power series ring
        $k[[s_1,\dots,s_m]]$, we may lift $G$ to $k[[s_1,\dots,s_m]][x,y]^\wedge$; hence
        it suffices to treat the case $C=k[[s_1,\dots,s_m]]$.

        To produce the required isomorphism after completion, one looks for power series
        \(T_i\in C\) $(i=1,\dots,r)$ with $T_i(0)=0$ and for
        \(X,Y,U\in k[x,y][[s_1,\dots,s_m]]\) restricting to $x,y,1$ when $s_i=0$, such
        that
        \begin{equation}\label{eq:hartshorne-versal-equation}
            U\,F(X,Y,T)=G(x,y,s),
        \end{equation}
        where $T=(T_1,\dots,T_r)$. One constructs $T,X,Y,U$ inductively in the
        $s$--adic filtration.

        Assume by induction that we have partial power series
        $T^{(\nu)},X^{(\nu)},Y^{(\nu)},U^{(\nu)}$ such that
        \eqref{eq:hartshorne-versal-equation} holds modulo $s^{\nu+1}$ (here $s$ denotes
        $(s_1,\dots,s_m)$ and $s^{\nu+1}$ the ideal $(s_1,\dots,s_m)^{\nu+1}$). Define
        \[
            H^{(\nu)}
            =U^{(\nu)}F\bigl(X^{(\nu)},Y^{(\nu)},T^{(\nu)}\bigr)-G(x,y,s).
        \]
        By construction $H^{(\nu)}\in (s^{\nu+1})$, hence $H^{(\nu)}\bmod (s^{\nu+2})$ is
        homogeneous of degree $\nu+1$ in $s$. One can therefore write
        \[
            H^{(\nu)}\equiv f(x,y)\,\Delta U + f_x\,\Delta X + f_y\,\Delta Y + \sum_{i=1}^r g_i\,\Delta T_i
            \pmod{s^{\nu+2}},
        \]
        where $\Delta T_i$ are polynomials in $s$, and $\Delta U,\Delta X,\Delta Y$ are
        polynomials in $x,y,s$, all homogeneous of degree $\nu+1$ in $s$, and where the
        $g_i$ form a basis of the Tjurina algebra $R/J$ (notation as in Hartshorne).
        Now set
        \[
            T_i^{(\nu+1)}=T_i^{(\nu)}-\Delta T_i,\qquad
            X^{(\nu+1)}=X^{(\nu)}-\Delta X,\qquad
            Y^{(\nu+1)}=Y^{(\nu)}-\Delta Y,\qquad
            U^{(\nu+1)}=U^{(\nu)}-\Delta U.
        \]
        A first--order Taylor expansion (Lemma~14.2 below) shows that these corrections
        improve the congruence to hold modulo $s^{\nu+2}$. Passing to the limit gives
        power series $T,X,Y,U$ satisfying \eqref{eq:hartshorne-versal-equation} and hence
        the desired isomorphism after completion.

        \begin{lemma}
            Let $F(x_1,\dots,x_n)$ be a polynomial or power series and let $h_1,\dots,h_n$ be
            new variables. Then
            \[
                F(x_1+h_1,\dots,x_n+h_n)\equiv F(x_1,\dots,x_n)+\sum_{i=1}^n h_i\,\frac{\partial F}{\partial x_i}(x_1,\dots,x_n)
                \pmod{(h)^2}.
            \]
        \end{lemma}

        In particular, the linear parts of the functions $T_1,\dots,T_r$ are uniquely
        determined by the construction, which gives the uniqueness statement on Zariski
        tangent spaces in (b). This provides a \textbf{miniversal deformation space} for
        isolated plane curve singularities. On the other hand, the associated local
        deformation functor need not be pro--representable in general.
    \end{proof}
\end{example}

\begin{example}[15.2.4]
    For an example of a functor with no versal family, we note that if
    $(R,\xi)$ is a versal family for the functor $F$, then the map
    \[
        \Hom(R,D)\longrightarrow F(D)
    \]
    is surjective, so $F(D)$ is a quotient of a finite--dimensional vector space.
    If $F$ is the functor of deformations of a $k$--algebra $B$, then $F(D)$ is
    given by $T^1_{B/k}$. If $T^1_{B/k}$ is not finite--dimensional,
    $F$ cannot have a versal deformation space. For example, let
    $B=k[x,y,z]/(xy)$. Then $T^1_{B/k}=k[z]$. The trouble is that $B$ does not
    have isolated singularities.
\end{example}

\section{Schlessinger's Criterion}

Let $k$ be an algebraically closed field and let $\mathcal C$ denote the
category of local Artinian $k$--algebras with residue field $k$.
Let
\[
    F:\mathcal C \longrightarrow \mathrm{Sets}
\]
be a covariant functor with $F(k)$ consisting of a single element.

For a small extension
\[
    0 \to I \to A'' \xrightarrow{p} A \to 0
\]
(i.e.\ $I\cdot \mathfrak m_{A''}=0$),
and any morphism $A' \to A$, there is a natural map
\[
    F(A'\times_A A'') \longrightarrow
    F(A') \times_{F(A)} F(A'')
\] given by the universal property of the fiber product $A'\times_A A''$.

\begin{theorem}[Schlessinger's Criterion]
    The functor $F$ has a miniversal family if and only if the following conditions hold:

    \begin{enumerate}
        \item[(H0)] $F(k)$ consists of a single element.

        \item[(H1)] For every small extension $A'' \to A$, the natural map
              \[
                  F(A'\times_A A'') \to F(A') \times_{F(A)} F(A'')
              \]
              is surjective.

        \item[(H2)] In the special case $A'' = D := k[\varepsilon]/(\varepsilon^2)$
              and $A = k$, the map in $(H1)$ is bijective.

        \item[(H3)] The tangent space
              \[
                  t_F := F(D)
              \]
              is a finite-dimensional $k$--vector space. This condition is clearly necessary because by miniversality, $t_R \to t_F$ is bijective, so we can just carry over the
              vector space structure on $t_R$ to $t_F$, where $t_R$ is the tangent space of the representing object $R$, which identifies with $\Hom_k(R,D) = \mathfrak m_R/\mathfrak m_R^2$.
    \end{enumerate}

    Moreover, $F$ is pro-representable (i.e.\ represented by a complete local
    $k$--algebra) if and only if, in addition,

    \begin{enumerate}
        \item[(H4)] For every small extension $I \to A'' \twoheadrightarrow A$ and every
              $\eta \in F(A)$, the natural action of $t_F \otimes_k I$ on the set of liftings
              $p^{-1}(\eta) \subset F(A'')$ is simply transitive.
    \end{enumerate}
\end{theorem}


\begin{remark}[Meaning of the conditions]
    \leavevmode
    \begin{enumerate}
        \item[(H1)] expresses an infinitesimal gluing property:
              deformations over $A'$ and $A''$ that agree over $A$ can be glued.
              It is a weak formal smoothness condition.

        \item[(H2)] ensures that first-order deformations behave linearly.
              It guarantees that the tangent space $t_F = F(D)$ is well-defined
              and functorial.

        \item[(H3)] ensures that the space of first-order deformations is
              finite-dimensional, so a formal parameter space can exist.

        \item[(H4)] eliminates infinitesimal automorphisms.
              If it fails, liftings over small extensions need not form torsors
              under $t_F$, and $F$ may admit a hull but not be pro-representable.
    \end{enumerate}
\end{remark}

\begin{example}[Isolated plane curve singularity]
    We illustrate Schlessinger's criterion with the example of an isolated plane curve singularity. Note that we already checked directly that this example has a miniversal deformation space in Example~\ref{ex:miniversal-plane-singularity}.

    Let $B = k[x,y]/(f)$ define a plane curve singularity at the origin, and assume it is \textbf{isolated}, i.e.\ the Jacobian ideal
    \[
        J=(f,f_x,f_y)
    \]
    is $\mathfrak m$--primary for $\mathfrak m=(x,y)$. Let $F$ be the functor of infinitesimal deformations of $B$, i.e. to a local Artinian $k$--algebra $A$ with residue field $k$, it assigns the set of isomorphism classes of flat deformations of $B$ over $A$. Then first--order deformations are governed by
    \[
        t_F \cong T^1(B/k,B)
        \cong B/(f_x,f_y)
        \cong k[x,y]/(f,f_x,f_y),
    \]
    Since $J$ is $\mathfrak m$--primary, this ring is finite--dimensional over $k$. Thus condition $(H3)$ holds.

    Because $B$ is a hypersurface, it is a local complete intersection.
    For complete intersections one has
    \[
        T^2(B/k,M)=0
        \quad\text{for all $B$--modules $M$,}
    \]
    so there are no obstructions to lifting infinitesimal deformations.
    This unobstructedness is precisely what ensures the surjectivity condition $(H1)$ for small extensions.


    Condition $(H2)$ concerns first--order compatibility.
    For hypersurfaces, $T^1$ already controls all first--order
    deformations, and the linearity of $T^1$ ensures the required bijectivity
    when $A''=k[\varepsilon]/(\varepsilon^2)$ and $A=k$.

    \smallskip
    \noindent\textbf{Miniversality.}
    Since $(H1)$--$(H3)$ hold, Schlessinger's theorem guarantees the existence of
    a miniversal deformation space.
    Concretely, choosing a $k$--basis $g_1,\dots,g_r$ of the Tjurina algebra,
    one obtains the explicit miniversal family
    \[
        f(x,y)+\sum_{i=1}^r t_i g_i(x,y)=0
    \]
    over $\Spec k[t_1,\dots,t_r]$ (completed at the origin).

    \smallskip
    \noindent\textbf{When is it pro--representable?}
    Condition $(H4)$ amounts to the absence of infinitesimal automorphisms,
    i.e.\ to $T^0(B/k,B)=\Der_k(B,B)=0$.
    For an isolated hypersurface singularity, derivations correspond to
    vector fields preserving the defining equation.

    For a nonexample, consider the node defined by $f(x,y)=xy$. The derivation
    \[
        \delta = x\frac{\partial}{\partial x}
        - y\frac{\partial}{\partial y}.
    \]
    $\delta$ preserves the defining equation since $\delta(xy) = xy - xy = 0 \in (xy)$,
    and descends to a nonzero element of $\Der_k(B,B)$. In fact, for any $f(x,y)$, the derivation $\delta = f_x \frac{\partial}{\partial y} - f_y \frac{\partial}{\partial x}$ preserves the ideal $(f)$, so $T^0(B/k,B)$ is nonzero for any plane curve singularity.

    Thus
    \[
        T^0(B/k,B)\neq 0.
    \]
    Geometrically this corresponds to the $\mathbf G_m$--action
    \[
        (x,y)\mapsto (\lambda x,\lambda^{-1}y),
    \]
    which rescales the two branches of the node. In particular, the global $\mathbf{G}_m$--action on the node differentiates by writing $\lambda = 1+\varepsilon$ with $\varepsilon^2=0$ so infinitesimally we get
    $x \mapsto (1+\varepsilon)x = x + \varepsilon x$,
    $y \mapsto (1-\varepsilon)y = y - \varepsilon y$. However not every infinitesimal automorphism integrates to a global $\mathbf{G}_m$--action. A $\mathbf{G}_m$-action on an affine scheme $\Spec B$ is equivalent to giving a $\mathbb{Z}$-grading
    \[
        B = \bigoplus_{n\in\mathbb{Z}} B_n
    \]
    such that
    \[
        \lambda \cdot b = \lambda^n b
        \quad \text{for } b\in B_n.
    \]
    So integrating a derivation to a $\mathbf{G}_m$-action means that there exists a grading of $B$, and the derivation acts by
    \[
        \delta(b) = n b
        \quad \text{for } b\in B_n.
    \]
    So $\delta$ must be diagonalizable with integer eigenvalues. For a nonexample, consider the derivation $\delta = x^2 \partial_x$ which preserves the ideal $(xy)$ but has $\delta(x) = x^2$ does not act semisimply on $B$.

\end{example}




\begin{remark}[Obstructions are measured by $T^2$]
    Let
    \[
        0 \to I \to A'' \to A \to 0
    \]
    be a small extension in $\mathcal C$. Suppose we are given a deformation
    $B_A$ of $B$ over $A$. General deformation theory (via the cotangent complex or the Lichtenbaum--Schlessinger construction of the $T^i$) produces a canonical obstruction class
    \[
        \mathrm{ob}(B_A,A''/A) \in T^2(B/k,B\otimes_k I)
    \]
    whose vanishing is necessary and sufficient for the existence of a lift of
    $B_A$ to a deformation over $A''$.

    Thus $T^2$ plays the role of an \textbf{obstruction space} for lifting deformations
    across small extensions. If
    \[
        T^2(B/k,M)=0 \quad \text{for all $B$--modules $M$,}
    \]
    then every obstruction class vanishes. Consequently every deformation over $A$
    lifts to $A''$, and more generally compatible deformations over $A'$ and $A''$
    can be glued over $A'\times_A A''$. This lifting and gluing property is
    precisely the surjectivity condition $(H1)$ in Schlessinger's criterion.
\end{remark}

\begin{example}[Non-isolated singularity]
    Let $B = k[x,y,z]/(xy)$.
    Then
    \[
        T^1(B/k,B) \cong k[z],
    \]
    which is infinite-dimensional.

    Condition $(H3)$ fails, so no miniversal deformation space exists.
\end{example}

\begin{example}[Smooth affine variety]
    If $X = \Spec B$ is smooth over $k$, then
    \[
        T^1(B/k,B) = 0.
    \]
    Hence $t_F = 0$ and the only deformation is the trivial one. The functor is pro-representable by $k$.
\end{example}

\section{Hilb and Pic are pro-representable}
An important application of Schlessinger's criterion is to show that the local deformation functors associated to the Hilbert scheme and the Picard scheme are pro-representable. It turns out that the global functors themselves are representable by the Hilbert scheme and the Picard scheme, so the local functors are pro-representable by the completed local rings of these schemes at the relevant points. However, the representability of the global functors is a deeper result whereas the pro-representability of the local functors can be checked directly using Schlessinger's criterion.

Let $X$ be a projective scheme over $k$. There is the global Hilbert functor $\mathrm{Hilb}_X$ which to each $S/k$ associates the set of closed subschemes of $X_S = X \times_k S$ that are flat over $S$. Fixing $x\in \mathrm{Hilb}_X(k)$ corresponding to a closed subscheme $X_0 \subseteq X$, we can define the local Hilbert functor $F$ of deformations of $X_0$ by simply restricting the global functor to local Artinian k-algebras $A \in \mathcal C$ and keeping only families whose special fiber is $X_0$.

Let $X_0$ be a given closed subscheme of $\mathbf P_k^n$. For each local artinian
$k$-algebra $A$ we let $F(A)$ be the set of deformations of $X_0$ over $A$, that is,
the set of closed subschemes $X \subset \mathbf P_A^n$, flat over $A$, such that
$X \times_A k \cong X_0$. (Here by abuse of notation,
$X \times_A k$ means $X \times_{\Spec A} \Spec k$, the fibered product in
the category of schemes, not the fibered product of sets!) Then $F$ is a functor
from the category $\mathcal C$ of local artinian $k$-algebras to (Sets), which we
call the local Hilbert functor of deformations of $X_0$.

\begin{theorem}
    For a given closed subscheme $X_0 \subseteq \mathbf P_k^n$, the local Hilbert
    functor $F$ is pro-representable.
\end{theorem}

\begin{proof}
    We verify the conditions of Schlessinger's criterion.

    Condition $(H_0)$ says that $F(k)$ should have just one element, which it
    does, namely $X_0$ itself.

    Condition $(H_1)$ says that for every small extension $A'' \to A$, and any
    map $A' \to A$, the map
    \[
        F(A' \times_A A'') \longrightarrow F(A') \times_{F(A)} F(A'')
    \]
    should be surjective. So suppose we are given closed subschemes
    $X' \subset \mathbf P_{A'}^n$ and $X'' \subset \mathbf P_{A''}^n$, flat over
    $A'$ and $A''$, respectively, and both restricting to $X \subset \mathbf P_A^n$.
    We let $X^*$ be the scheme structure on the topological space $X_0$ defined by
    the fibered product of sheaves of rings
    \[
        \mathcal O_{X'} \times_{\mathcal O_X} \mathcal O_{X''}
    \]
    Letting $A^* = A' \times_A A''$, we have surjective maps of sheaves
    \[
        \mathcal O_{\mathbf P_{A^*}^n} \longrightarrow \mathcal O_{X'}
        \qquad\text{and}\qquad
        \mathcal O_{\mathbf P_{A^*}^n} \longrightarrow \mathcal O_{X''},
    \]
    giving the same composed map to $\mathcal O_X$, hence a surjective map to
    $\mathcal O_{X^*}$. Therefore $X^*$ is a closed subscheme of $\mathbf P_{A^*}^n$.
    It is flat over $A^*$ and restricts to $\mathcal O_{X'}$ and $\mathcal O_{X''}$
    over $A'$ and $A''$. Thus $X^*$ is an element of $F(A^*)$ mapping
    to $X'$ and $X''$, and $(H_1)$ is satisfied.

    \begin{remark}[Core mechanism behind the verification of $(H_1)$]
        Let
        \[
            A^* := A' \times_A A'',
        \]
        where $A''\to A$ is a small extension, and suppose we are given flat closed
        subschemes
        \[
            X' \subset \mathbf P^n_{A'},
            \qquad
            X'' \subset \mathbf P^n_{A''}
        \]
        whose reductions over $A$ coincide with a subscheme
        \[
            X \subset \mathbf P^n_A.
        \]

        \smallskip
        \noindent
        \textbf{(1) Same underlying space.}
        Since all deformations are infinitesimal thickenings of the same closed fiber
        $X_0$, the underlying topological spaces agree. Hence one does not glue spaces,
        but rather structure sheaves.

        Define a sheaf of rings on $|X_0|$ by
        \[
            \mathcal O_{X^*}
            :=
            \mathcal O_{X'} \times_{\mathcal O_X} \mathcal O_{X''}.
        \]
        Sections are pairs $(s',s'')$ whose reductions to $X$ agree. This is the
        universal object encoding compatible deformations.

        \smallskip
        \noindent
        \textbf{(2) Realization as a closed subscheme.}
        Because projective space commutes with base change,
        \[
            \mathbf P^n_{A^*}
            \cong
            \mathbf P^n_{A'} \times_{\mathbf P^n_A} \mathbf P^n_{A''}.
        \]
        Hence its structure sheaf maps compatibly to both
        $\mathcal O_{X'}$ and $\mathcal O_{X''}$.
        By the universal property of fibered products of rings, we obtain a morphism
        \[
            \mathcal O_{\mathbf P^n_{A^*}}
            \longrightarrow
            \mathcal O_{X^*}
            =
            \mathcal O_{X'} \times_{\mathcal O_X} \mathcal O_{X''}.
        \]
        Since the maps to $\mathcal O_{X'}$ and $\mathcal O_{X''}$ are surjective,
        this map is surjective as well. Therefore $X^*$ is cut out by an ideal sheaf
        and defines a closed subscheme
        \[
            X^* \subset \mathbf P^n_{A^*}.
        \]

        \smallskip
        \noindent
        \textbf{(3) Flatness.}
        Locally on affine charts, write
        \[
            B' \text{ flat over } A', \qquad
            B'' \text{ flat over } A'',
        \]
        with common reduction $B$ over $A$, and set
        \[
            B^* := B' \times_B B''.
        \]
        Because $A''\to A$ is a small extension (square-zero kernel), flatness is
        preserved under this pullback construction, and one has
        \[
            B^* \otimes_{A^*} A' \cong B',
            \qquad
            B^* \otimes_{A^*} A'' \cong B''.
        \]
        Thus $X^*$ is flat over $A^*$ and restricts to the given deformations.

        \smallskip
        \noindent
        Hence $X^*$ defines an element of $F(A^*)$ mapping to
        $(X',X'')$, proving the surjectivity required by condition $(H_1)$.
        The key idea is that compatibility of infinitesimal deformations is encoded
        by a fiber product of structure sheaves on a fixed underlying space.
    \end{remark}

    \begin{remark}[Flatness in the fiber--product construction]
        The verification of condition $(H_1)$ uses the following commutative algebra fact.

        Let
        \[
            A^* = A' \times_A A'',
        \]
        where $A'' \twoheadrightarrow A$ is a small extension with kernel $J$ satisfying
        $J^2=0$. Assume we are given flat algebras
        \[
            B' \text{ over } A',
            \qquad
            B'' \text{ over } A'',
        \]
        together with an identification of their reductions
        \[
            B' \otimes_{A'} A \;\cong\; B'' \otimes_{A''} A =: B .
        \]
        Define
        \[
            B^* := B' \times_B B''.
        \]

        Then:

        \begin{enumerate}
            \item[(1)] $B^*$ is flat over $A^*$;
            \item[(2)] base change recovers the original algebras:
                  \[
                      B^* \otimes_{A^*} A' \cong B',
                      \qquad
                      B^* \otimes_{A^*} A'' \cong B''.
                  \]
        \end{enumerate}

        \smallskip
        \noindent
        \textbf{Why the square--zero hypothesis matters.}
        The projection
        \[
            A^* \to A'
        \]
        has kernel naturally identified with $J$, hence is itself a square--zero
        extension:
        \[
            J^2=0.
        \]
        For square--zero extensions, flatness is controlled by reduction modulo the
        kernel: roughly,
        \[
            M \text{ flat over } R
            \quad\Longleftrightarrow\quad
            M/IM \text{ flat over } R/I
        \]
        when $I^2=0$ (equivalently, no hidden higher-order torsion can appear).
        Thus compatibility of $B'$ and $B''$ over the common reduction $B$ suffices to
        guarantee flatness of $B^*$ over $A^*$.

        The square--zero condition is essential: for general fiber products
        (flatness need not be preserved), which is precisely why Schlessinger’s
        criterion is formulated using small extensions.
    \end{remark}

    Note that in general $(H_2)$ is a consequence of $(H_4)$.

    For $(H_3)$ we note that
    \[
        t_F = F(k[t]/(t^2))
    \]
    is the set of deformations of $X_0$ over $k[t]/(t^2)$, which identifies with the space of global sections of the normal sheaf $H^0(X_0,\mathcal N_{X_0/\mathbf P^n})$. This is because for an embedded deformation problem, first-order deformations are identified with maps $I/I^2 \to \mathcal O_{X_0}$, hence with global sections of the normal sheaf $\mathcal N_{X_0/\mathbf P^n}$. Since $X_0$ is projective, this is a finite-dimensional vector space.

    For $(H_4)$, let $\eta \in F(A)$ be given by a deformation
    $X \subset \mathbf P_A^n$ of $X_0$. Then $p^{-1}(\eta)$ consists of subschemes
    $X' \subset \mathbf P_{A'}^n$, flat over $A'$, with
    \[
        X' \times_{A'} A \cong X.
    \]
    If such exist, they form a torsor under the action of $t_F$ by the following result. Thus all the conditions are satisfied and $F$ is pro-representable.
\end{proof}

\begin{theorem}[Extensions of closed subschemes]

    Let $Y_0$ be a closed subscheme of a scheme $X_0$ over a field $k$.
    Let $X$ be a deformation of $X_0$ over an Artin ring $C$, and let
    $Y \subset X$ be a closed subscheme such that
    \[
        Y \times_C k \cong Y_0.
    \]

    Let
    \[
        0 \to J \to C' \to C \to 0
    \]
    be a small extension, and let $X'$ be an extension of $X$ over $C'$,
    i.e.\ $X'$ is flat over $C'$ and there is a closed immersion
    \[
        X \hookrightarrow X'
    \]
    inducing an isomorphism
    \[
        X \cong X' \times_{C'} C.
    \]

    We seek to classify extensions of $Y$ over $C'$ as closed subschemes of
    $X'$, namely closed subschemes
    \[
        Y' \subset X', \qquad Y' \text{ flat over } C',
    \]
    such that
    \[
        Y' \times_{C'} C \cong Y.
    \]

    Let $\mathcal N_0 = \mathcal N_{Y_0/X_0}$ denote the normal sheaf.

    Then:

    \begin{enumerate}
        \item[(a)]
              The set of extensions of $Y$ over $C'$ in $X'$ is a
              \textbf{pseudotorsor} under the group
              \[
                  H^0(Y_0,\mathcal N_0 \otimes_k J).
              \]
              meaning either the set is empty or it is a genuine torsor.
        \item[(b)]
              If extensions exist locally on $X$, then there is an obstruction class
              \[
                  \alpha \in H^1(Y_0,\mathcal N_0 \otimes_k J)
              \]
              whose vanishing is necessary and sufficient for the existence of a
              global extension $Y'$.
              If one extension exists, then the set of all such extensions is a
              torsor under
              \[
                  H^0(Y_0,\mathcal N_0 \otimes_k J).
              \]
    \end{enumerate}
\end{theorem}

\begin{proof}
    We outline the mechanism. The torsor structure comes from comparing two liftings. If
    $Y'_1,Y'_2\subset X'$ are liftings reducing to the same $Y$, then their ideals
    agree modulo $J$, and their difference determines a homomorphism
    \[
        I_Y/I_Y^2 \longrightarrow \mathcal O_Y\otimes J.
    \]
    Since
    \[
        \mathcal N_{Y/X}=\mathcal Hom(I_Y/I_Y^2,\mathcal O_Y),
    \]
    this difference is a global section of
    $\mathcal N_{Y_0/X_0}\otimes_k J$. Conversely, adding such a section to one
    lifting produces another lifting. Thus liftings differ by normal vector fields,
    which explains the simply transitive action.


    Let
    \[
        0 \to J \to C' \to C \to 0
    \]
    be a small extension with $J^2=0$, and suppose we are given a deformation
    $Y\subset X$ over $C$ together with an extension $X'$ of $X$ over $C'$.
    Assume that local liftings of $Y$ inside $X'$ exist.

    Choose an open cover $\{U_i\}$ of the closed fiber $Y_0$ such that for each
    $i$ there is a lifting
    \[
        Y_i' \subset X'|_{U_i}
    \]
    of $Y|_{U_i}$.

    On an overlap $U_{ij}=U_i\cap U_j$, the liftings
    \[
        Y_i'|_{U_{ij}},\qquad Y_j'|_{U_{ij}}
    \]
    are both extensions of the same subscheme $Y|_{U_{ij}}$.
    By the local torsor description, their difference is a unique section
    \[
        s_{ij}\in H^0(U_{ij},\mathcal N_0\otimes_k J)
    \]
    such that
    \[
        Y_j'|_{U_{ij}}
        =
        \bigl(Y_i'|_{U_{ij}}\bigr)+s_{ij}.
    \]

    On triple overlaps $U_{ijk}=U_i\cap U_j\cap U_k$, associativity of adding
    differences implies
    \[
        s_{ij}+s_{jk}+s_{ki}=0.
    \]
    Hence the collection $(s_{ij})$ is a \v{C}ech $1$--cocycle with values in
    the sheaf $\mathcal N_0\otimes_k J$. Its cohomology class
    \[
        \alpha := [(s_{ij})]
        \in H^1(Y_0,\mathcal N_0\otimes_k J)
    \]
    is called the obstruction class.

    We claim that $\alpha=0$ if and only if the local liftings glue to a global
    lifting $Y'\subset X'$.

    Suppose $\alpha=0$. Then the cocycle is a coboundary: there exist sections
    \[
        t_i\in H^0(U_i,\mathcal N_0\otimes_k J)
    \]
    such that
    \[
        s_{ij}=t_j-t_i.
    \]
    Define adjusted liftings
    \[
        Y_i'' := Y_i' - t_i.
    \]
    On overlaps we compute
    \[
        Y_j''
        = Y_j' - t_j
        = (Y_i' + s_{ij}) - t_j
        = Y_i' - t_i
        = Y_i'',
    \]
    so the $Y_i''$ agree and glue to a global lifting
    \[
        Y'\subset X'.
    \]


    Conversely, if a global lifting $Y'\subset X'$ exists, take
    \[
        Y_i' := Y'|_{U_i}.
    \]
    Then the differences on overlaps vanish:
    \[
        s_{ij}=0,
    \]
    hence $\alpha=0$.

    Therefore the class
    \[
        \alpha \in H^1(Y_0,\mathcal N_0\otimes_k J)
    \]
    vanishes exactly when a global extension exists.
\end{proof}

\section{Deformations of abstract schemes}
In the previous section, we saw that the local Hilbert functor of deformations of a closed subscheme $X_0 \subseteq \mathbf P^n$ is pro-representable. The local Hilbert functor deforms embedded subschemes, while the deformation functor of a scheme deforms the scheme itself up to abstract isomorphism.
Passing from embedded to abstract deformations introduces automorphisms, identifications, and descent issues.

For another example, we will consider starting with a global moduli problem, such as the moduli of curves
of genus $g$, the global functor considers flat families $X/S$ for a scheme $S$,
whose geometric fibers are projective nonsingular curves of genus $g$, up to
isomorphism of families.

The formal local version of this functor around a
given curve $X_0/k$ would assign to each Artin ring $A$ with residue field $k$
the set $F_1(A)$ of isomorphism classes of flat families $X/A$ such that
\[
    X \otimes_A k \cong X_0.
\]
We call this the \textbf{crude local functor}.

The functor $F_1$ is not well behaved. \textbf{We need to rigidify the functor by keeping track of the identification of the special fiber with $X_0$.} Recall that a deformation of $X_0$
over $A$ is a pair $(X,i)$, where $X$ is a scheme flat over $A$, and
\[
    i : X_0 \hookrightarrow X
\]
is a closed immersion such that the induced map
\[
    i \otimes k : X_0 \longrightarrow X \otimes_A k
\]
is an isomorphism. We consider the functor $F(A)$ that to each $A$ assigns the
set of deformations $(X,i)$ of $X_0$ over $A$, up to equivalence, where an
equivalence of $(X_1,i_1)$ and $(X_2,i_2)$ means an isomorphism
\[
    \varphi : X_1 \xrightarrow{\sim} X_2
\]
compatible with the maps $i_1,i_2$ from $X_0$.

The effect of using the functor $F$ instead of $F_1$ is to leave possible
automorphisms of $X_0$ out of the picture and thus simplify the discussion.
We will consider the relation between these two functors later.

\begin{theorem}
    Let $X_0$ be a scheme over $k$. Then the functor $F$
    (of deformations of $X_0$ over local Artin rings, defined above)
    has a miniversal family under either of the following hypotheses:

    \begin{enumerate}
        \item[(a)] $X_0$ is affine with isolated singularities;
        \item[(b)] $X_0$ is projective.
    \end{enumerate}
\end{theorem}
Note that the hypothesis are only necessarily to verify that $t_F$ is finite--dimensional. Also note that we have seen the nodal curve $xy=0$ has a miniversal deformation space but it is not pro-representable, so we cannot expect to get a pro-representable functor without further assumptions.

\begin{theorem}[18.2]
    Let $X_0/k$ be given and assume the hypotheses of (18.1) satisfied.
    Then the functor $F$ of deformations of $X_0$ is pro-representable
    if and only if for each small extension $A' \to A$, and for each
    deformation $X'$ over $A'$ restricting to a deformation $X$ over $A$,
    the natural map
    \[
        \Aut(X'/X_0) \longrightarrow \Aut(X/X_0)
    \]
    of automorphisms of $X'$ (and $X$) restricting to the identity
    automorphism of $X_0$ is surjective.
\end{theorem}

\begin{proof}
    Suppose that $\Aut(X'/X_0) \to \Aut(X/X_0)$ is surjective for every
    $X'$ lying over $X$. We need to verify condition $(H_4)$ of (16.2).

    Let $X_1', X_2'$ be elements of $F(A')$ inducing the same element
    $X$ of $F(A)$. If
    \[
        X \hookrightarrow X_1'
        \qquad \text{and} \qquad
        X \hookrightarrow X_2'
    \]
    are maps inducing the isomorphisms
    \[
        X_1' \otimes_{A'} A \cong X
        \qquad \text{and} \qquad
        X_2' \otimes_{A'} A \cong X,
    \]
    and if $X_1'$ and $X_2'$ are isomorphic as deformations of $X_0$,
    then I claim that the inclusions
    \[
        X \hookrightarrow X_1'
        \qquad \text{and} \qquad
        X \hookrightarrow X_2'
    \]
    are isomorphic as extensions of $X$ over $A'$.

    Indeed, let $u' : X_1' \to X_2'$ be an isomorphism over $X_0$.
    Then $u = u' \otimes_{A'} A$ is an automorphism of $X$ over $X_0$.
    By hypothesis this lifts to an automorphism $\sigma$ of $X_1'$.
    Then
    \[
        v = u \circ \sigma^{-1} : X_1' \to X_2'
    \]
    is an isomorphism inducing the identity on $X$, so
    $X_1'$ and $X_2'$ are equivalent as extensions of $X$ over $A'$. $\Def(X/A, A')$ is a principal homogeneous
    space under the action of $t_F$, so condition $(H_4)$ of Schlessinger's
    criterion is satisfied, and $F$ is pro-representable.

    Conversely, suppose that $F$ is pro-representable. Let
    $X' \in F(A')$ restrict to $X \in F(A)$, and choose a map
    \[
        u : X \hookrightarrow X'
    \]
    inducing the isomorphism
    \[
        X \cong X' \otimes_{A'} A.
    \]
    Let $\sigma \in \Aut(X/X_0)$. Then
    \[
        u' = u \circ \sigma : X \hookrightarrow X'
    \]
    gives another element of $\Def(X/A, A')$, and so $u$ and $u'$ differ
    by an element of $t_F$. But $u$ and $u'$ define
    the same element $X' \in F(A')$ lying over $X$, so by condition $(H_4)$
    this element of $t_F$ must be zero. Hence $u$ and $u'$ are equal as
    elements of $\Def(X/A, A')$; in other words there exists an isomorphism
    \[
        \tau : X' \to X'
    \]
    over $X_0$ such that
    \[
        u' = \tau \circ u.
    \]
    Restricting to $X$ we obtain
    \[
        \sigma = \tau|_X.
    \]
    Thus $\tau \in \Aut(X'/X)$ lifts $\sigma$, and the map is surjective.
\end{proof}

\begin{remark}[Interpretation of condition $(H_4)$]\label{rem:H4-automorphisms}
    Let
    \[
        0 \to J \to A' \to A \to 0
    \]
    be a small extension, and let $\eta \in F(A)$ correspond to a deformation
    $X/A$. The set of liftings
    \[
        p^{-1}(\eta) \subset F(A')
    \]
    consists of deformations $X'/A'$ restricting to $X$.

    Schlessinger's condition $(H_4)$ requires that this set be a principal
    homogeneous space (torsor) under the action of
    \[
        t_F \otimes_k J.
    \]
    Equivalently, any two liftings differ by a unique tangent vector and the
    action has no stabilizers.

    Geometrically, the subtlety is that the deformation functor $F$ remembers
    deformations only up to isomorphism. Two extensions
    \[
        X \hookrightarrow X_1', \qquad X \hookrightarrow X_2'
    \]
    may define the same element of $F(A')$ if there exists an isomorphism
    $X_1' \cong X_2'$ over $X_0$ that does not induce the identity on $X$.
    Such identifications arise from automorphisms of $X$ over $X_0$.

    Thus infinitesimal automorphisms act as possible stabilizers of the torsor of
    extensions. Condition $(H_4)$ is precisely the statement that no such
    stabilizers occur. Equivalently, every automorphism of $X/X_0$ lifts across
    small extensions:
    \[
        \Aut(X'/X_0) \longrightarrow \Aut(X/X_0)
        \quad \text{is surjective}.
    \]
\end{remark}

\begin{remark}
    Note that there are two infinitesimal spaces to consider here. There are deformation directions
    $t_F \cong T^1$ which move the object in moduli. There are infinitesimal automorphisms $\Aut(X/X_0)_{\mathrm{inf}} \sim T^0$ which change the presentation of the same deformation without moving its moduli point. Schlessinger (H1)-(H3) give the existence of a formal parameter space whereas (H4) is about the uniqueness of coordinates on that parameter space.
\end{remark}

We now give a verifiable criterion for the existence of a miniversal deformation space for projective schemes. \begin{corollary}
    Let $X_0$ be a projective scheme over $k$. If $H^0(X_0,\mathcal T_{X_0})=0$, then the functor $F$ of deformations of $X_0$ is pro-representable.
\end{corollary}

\section{Algebraization of formal moduli}

Once we have a formal family of deformations of some object—either versal,
miniversal, or universal—we may ask whether it extends to an \textbf{actual}
family, defined over a scheme of finite type over $k$. If such a family exists,
does it retain the corresponding versality property? In effect, can we pass from
a formal family to a global moduli space?


Suppose $F$ is the deformation functor of a scheme $X_0/k$, and assume we have a versal family, i.e. a complete local $k$-algebra
$(R,\mathfrak m)$, and a formal point
$\xi \in \widehat F(R)=\varprojlim F(R/\mathfrak m^{n+1})$.

Unpacking the inverse limit means exactly that for each $n$, you have a deformation
\[
    X_n \to \Spec R_n,\qquad R_n=R/\mathfrak m^{n+1},
\]
compatible under restriction.
i.e. a compatible system of schemes
\[
    X_0 \subset X_1 \subset X_2 \subset \cdots ,
\]
where for each $n$, the scheme $X_n$ is a deformation of $X_0$ over
\[
    R_n := R/\mathfrak m^{n+1},
\]
and this tower encodes a formal family that dominates every infinitesimal deformation problem. 

\subsection{Step 1: Passing to formal schemes}
The first step is to form the corresponding formal scheme as the limit of the $X_n$.


\begin{remark}[Formal schemes]

    Let $R$ be a complete local ring with maximal ideal $\mathfrak m$.
    In deformation theory one is interested only in the infinitesimal
    neighborhood of the closed point of $\Spec R$.  This is encoded by
    the \textbf{formal spectrum} of $R$, denoted
    \[
        \Spf R.
    \]

    The underlying topological space of $\Spf R$ is
    \[
        |\Spf R| = \Spec(R/\mathfrak m),
    \]
    so if $R$ is local it consists of a single point.  The structure sheaf
    is defined as the inverse limit
    \[
        \mathcal O_{\Spf R}
        =
        \varprojlim_{n} \mathcal O_{\Spec R/\mathfrak m^{n+1}},
    \]
    so that sections over the unique open set are simply elements of the
    completed ring $R$.

    More generally, if $R$ is complete with respect to an ideal $I$,
    then $\Spf R$ is defined similarly with respect to the system
    $R/I^{n+1}$.

    The formal spectrum may be viewed as the infinitesimal
    neighborhood of the closed point in $\Spec R$.  In particular,
    maps of formal schemes
    \[
        \Spf A \longrightarrow \Spf R
    \]
    correspond to continuous homomorphisms of complete local rings
    \[
        R \longrightarrow A.
    \]

    In deformation theory, a versal family over a complete local ring
    $R$ naturally defines a formal scheme over $\Spf R$,
    since the family is given by compatible deformations over the
    Artinian quotients $R/\mathfrak m^{n+1}$.

    Then we say a locally ringed space $\mathfrak X$ is a \textbf{formal scheme} if every point has an open neighborhood $U$ such that
    \[
        (U,\mathcal O_{\mathfrak X}|_U) \cong \Spf A
    \]
    for some adic ring $A$.
\end{remark}

\begin{proposition}
    Let $R,\mathfrak m$ be a complete local ring with residue field $k$, and
    suppose we are given a formal family of deformations of $X_0$ over $R$, that is,
    for each $n$, schemes $X_n$ flat and of finite type over
    \[
        R_n = R/\mathfrak m^{n+1}
    \]
    and maps
    \[
        X_n \to X_{n+1}
    \]
    inducing isomorphisms
    \[
        X_n \xrightarrow{\sim} X_{n+1}\otimes_{R_{n+1}} R_n.
    \]
    Then there is a noetherian formal scheme $\mathcal X$, flat over
    \[
        \Spf R,
    \]
    the formal spectrum of $R$, such that for each $n$,
    \[
        X_n \cong \mathcal X \times_R R_n .
    \]
\end{proposition}

\begin{proof}
    We define $\mathcal X$ to be the locally ringed space formed by taking the
    topological space $X_0$, together with the sheaf of rings
    \[
        \mathcal O_{\mathcal X} = \varprojlim \mathcal O_{X_n}.
    \]
    To show that $\mathcal X$ is a noetherian formal scheme, it is enough show that
    $\mathcal X$ has an open cover $U_i$ such that on each $U_i$, the induced
    ringed space is obtained as the formal completion of a scheme $U_i$ along a
    closed subset $Z_i$. This is because if \(X=\Spec B\) is Noetherian and $I\subset B$ is an ideal, then $\widehat X_Z := \Spf(\widehat B_I)$ is a Noetherian affine formal scheme.


    Let $U$ be an open affine subset of $X_0$, with
    \[
        U = \Spec B_0.
    \]
    Then for each $n$ the restriction of $X_n$ to $U$ will be $\Spec B_n$ for a
    suitable ring $B_n$. This is because the underlying topological space of all $X_n$ is the same $X_0$, so if $U\subset X_0$ is affine, then each infinitesimal thickening restricts to something affine.

    Furthermore, by the compatibility of the $X_n$, the rings $B_n$ form a surjective inverse
    system with
    \[
        \varprojlim B_n = B_\infty,
    \]
    and
    \[
        H^0(U,\mathcal O_{\mathcal X}) = B_\infty.
    \]
    We want to show that $B_\infty$ is Noetherian and complete.
    Take a polynomial ring
    \[
        A_0 = k[x_1,\dots,x_r]
    \]
    together with a surjective map
    \[
        A_0 \twoheadrightarrow B_0.
    \]
    For each $n$, let
    \[
        A_n = R_n[x_1,\dots,x_r].
    \]
    Lifting the images of the $x_i$ by the surjectivity of the maps $B_{n+1} \to B_n$, we get compatible maps
    \[
        A_n \twoheadrightarrow B_n,
    \]
    with kernel $I_n$. Because of the flatness of $B_n$ over $R_n$, we find that
    the inverse system $\{I_n\}$ is also surjective, and hence the induced map on
    inverse limits
    \[
        \varprojlim A_n \longrightarrow B_\infty
    \]
    is surjective.

    Now
    \[
        \varprojlim A_n = R\{x_1,\dots,x_r\},
    \]
    the ring of convergent power series in $x_1,\dots,x_r$ over $R$. A convergent power series ring over a complete local ring is Noetherian. Being a quotient of a noetherian ring, $B_\infty$ is a noetherian ring also, complete with
    respect to the $\mathfrak m B_\infty$-adic topology, and each
    \[
        B_n = B_\infty/\mathfrak m^{n+1}B_\infty.
    \]
    Thus the ringed space $(U,\mathcal O_{\mathcal X}|_U)$ is just the formal
    completion of $\Spec B_\infty$ along the closed subset $U$. Such open sets $U$ cover $X_0$, so by definition
    $(\mathcal X,\mathcal O_{\mathcal X})$ is a noetherian formal scheme.
\end{proof}

\begin{remark}
    If in addition we are given a compatible system of coherent/locally free/invertible sheaves $\cF_n$ on $X_n$, then we can form a coherent/locally free/invertible sheaf $\mathcal F$ on $\mathcal X$ by taking the inverse limit of the $\cF_n$.
\end{remark}

\subsection{Step 2: Algebraization of formal schemes}
The next step, once we have a formal scheme $X$ over $\Spf R$, is to ask
whether there exists a scheme $X$, flat and of finite type over the complete local ring $R$, whose formal
completion along the closed fiber is $X$. In this case, following Artin, we say
that the formal scheme $X$ is \textbf{effective/algebraizable}. This is not always possible and in that case we can go
no further. However, there is a good case in which it is possible, namely when
$X$ is projective, thanks to an existence theorem of Grothendieck.

\begin{theorem}[Grothendieck's existence theorem]
    Let $\mathcal{X}$ be a formal scheme, proper over
    $\Spf R$, where $(R,\mathfrak{m})$ is a complete local ring, and suppose there exists an invertible sheaf $\mathcal{L}$ on $\mathcal{X}$ such that $\mathcal{L}_0 = \mathcal{L} \otimes_R k$ is ample on $\mathcal{X}_0 = \mathcal{X} \otimes_R k$. Then
    there exists a scheme $X$ over $R$, together with an ample line bundle $\mathcal{L}$, such that $\hat{X} = \mathcal{X}$ and $\hat{\mathcal{L}} = \mathcal{L}$, taking completions along the closed fiber over $R$.
    In particular, $X$ is effective.
\end{theorem}

The existence of an ample line bundle allows us to recover the scheme $X$ by taking the relative Proj of the graded algebra. In particular, higher cohomology vanishes and we can produce embeddings uniformly in $n$. The properness guarantees that cohomology is dimensional and inverse limits are coherent.

\begin{remark}
    It is worth mentioning some of the things that can fail. The key step in algebraization is
    \[H^0(\mathcal X,\mathcal L^{\otimes m})
        \;\cong\;
        \varprojlim_n H^0(X_n,\mathcal L_n^{\otimes m})\]
    and the fact that these modules are finite and controlled uniformly in $n$.

    \begin{enumerate}
        \item Without properness, $H^0$ may not be finite--dimensional, and the resulting graded algebra may not be finite type.
        \item Without properness, the cohomology of the inverse limit may not be the inverse limit of the cohomology.
        \item Without an ample line bundle, we cannot even recover the scheme $X$ as a relative Proj.
    \end{enumerate}
\end{remark}

\begin{example}[A noneffective formal deformation]\label{ex:noneffective-formal-deformation}
This example shows that in Grothendieck's algebraization theorem it is not enough
to assume that $X_0$ is projective. \textbf{One must also assume that $X_0$ admits an
ample invertible sheaf which lifts to the formal scheme $\mathcal X$.}

Let $X_0$ be a nonsingular quartic surface in $\mathbf P^3$ over a field of
characteristic $0$. This will be a K3 surface since we have the canonical bundle \begin{align*}
    \omega_{X_0} &\cong \mathcal O_{X_0}(4-4) \cong \mathcal O_{X_0}.
\end{align*} and the first cohomology of the structure sheaf sits in an exact sequence 
\[
\longrightarrow H^1(\mathcal O_{\mathbf P^3})
\longrightarrow H^1(\mathcal O_{X_0})
\longrightarrow H^2(\mathcal O_{\mathbf P^3}(-4))
\] coming from the exact sequence 
\[0 \longrightarrow \mathcal O_{\mathbf P^3}(-4)
\longrightarrow \mathcal O_{\mathbf P^3}
\longrightarrow \mathcal O_{X_0}
\longrightarrow 0
\]
The outer terms vanish by the cohomology of projective space, so $H^1(\mathcal O_{X_0})=0$.
\begin{proposition}
Let $X$ be a nonsingular quartic surface in $\mathbf P^3$ over a field
$k$ of characteristic $0$. Then for any nontrivial line bundle
$\mathcal L$ on $X$, there exists an abstract deformation $X'$ of $X$
over the dual numbers to which $\mathcal L$ does not lift.
\end{proposition}

\begin{proof}
Deformations of the pair $(X,\mathcal L)$ are
classified by $H^1(X,\mathcal P_{\mathcal L})$, where
$\mathcal P_{\mathcal L}$ is the Atiyah bundle of first order differential operators on $\mathcal L$, whose sections on an open set $U$ are $k$-linear maps
\[D:\mathcal L|_U \to \mathcal L|_U
\] so that there exists a (necessarily unique) vector field $\xi\in H^0(U,T_X)$ such that
\[D(fs) = fD(s) + \xi(f)s
\]for all $f\in \mathcal O_X(U)$ and $s\in \mathcal L(U)$. It fits into an exact sequence
\[
0 \longrightarrow \mathcal O_X
\longrightarrow \mathcal P_{\mathcal L}
\longrightarrow T_X
\longrightarrow 0 .
\]
Taking cohomology gives an exact sequence
\[
\cdots \to H^1(\mathcal O_X)
\to H^1(\mathcal P_{\mathcal L})
\to H^1(T_X)
\xrightarrow{\delta}
H^2(\mathcal O_X)
\to \cdots .
\]

For a deformation class
\[
\tau \in H^1(T_X),
\]
the obstruction to extending $\mathcal L$ over the corresponding
deformation of $X$ is $\delta(\tau)\in H^2(\mathcal O_X)$.
Moreover, $\delta(\tau)$ is the cup product of $\tau$ with the class
\[
c(\mathcal L)\in H^1(\Omega_X^1)
\]
via the pairing
\[
T_X\otimes \Omega_X^1 \longrightarrow \mathcal O_X .
\]

Thus it suffices to show that there exists
$\tau\in H^1(T_X)$ with $\delta(\tau)\neq 0$.
Since \[H^2(\mathcal O_X) \cong H^0(\mathcal \omega_X)^\vee \cong H^0(\mathcal O_X)^\vee \cong k\] is one-dimensional, this is equivalent to showing that the map $\delta$ is nonzero.

Because the canonical class of $X$ is trivial, Serre duality identifies
\[
H^1(T_X)^\vee \cong H^1(\Omega_X^1),
\qquad
H^2(\mathcal O_X)^\vee \cong H^0(\mathcal O_X).
\]
Under this identification, $\delta$ is dual to the map
\[
H^0(\mathcal O_X)\longrightarrow H^1(\Omega_X^1)
\]
sending $1$ to $c(\mathcal L)$.
Hence it is enough to show that $c(\mathcal L)\neq 0$. Note that $c(\mathcal L)$ is the first Chern class of $\mathcal L$, coming from the map of sheaves $d\log: \mathcal O_X^\times \to \Omega_X^1$ and the induced map on $H^1$, given by $H^1(\mathcal O_X^\times) \cong \Pic(X) \to H^1(\Omega_X^1)$.
\end{proof}

\begin{lemma}
Let $X$ be a nonsingular surface in $\mathbf P^3$ over a field
$k$ of characteristic $0$. Let $\mathcal L$ be a nontrivial line bundle.
Then
\[
c(\mathcal L)\in H^1(X,\Omega_X^1)
\]
is nonzero.
\end{lemma}

\begin{proof}
The cohomology class $c(\mathcal L)$ is compatible with intersection
theory: for divisor classes $D,E$ on $X$, 
\[
c(\mathcal O_X(D))\smile c(\mathcal O_X(E))
\in H^2(\omega_X)\cong k
\]
is equal to
\[
(D\cdot E)\,1.
\]

Let $H$ be an ample divisor on $X$. If $D$ is a divisor with
$(H\cdot D)\neq 0$, then
\[
c(\mathcal O_X(D))\neq 0,
\]
since the pairing with $c(\mathcal O_X(H))$ is nonzero.

If instead $(H\cdot D)=0$ but $D\neq 0$, then by the Hodge index theorem
one has
\[
D^2<0 .
\]
Therefore
\[
c(\mathcal O_X(D))\smile c(\mathcal O_X(D))
= (D^2)\,1 \neq 0,
\]
so again $c(\mathcal O_X(D))\neq 0$. Hence every nontrivial line bundle has nonzero class in
$H^1(\Omega_X^1)$.
\end{proof}

\begin{remark}
    We recall the statement of the \textbf{Hodge index theorem}. Let $X$ be a smooth projective surface and let $H$ be an ample divisor. Intersection pairing gives a symmetric bilinear form on the Néron-Severi space\[N^1(X)_\mathbf{R} := (\operatorname{Div}(X)/\equiv_{\text{num}})\otimes_\mathbf{Z}\mathbf{R}\]

Then the intersection form has signature $(1,\rho-1)$, where $\rho=\dim N^1(X)_\mathbf{R}$. Equivalently, the class of an ample divisor $H$ spans the unique positive direction:
$H^2>0$, and on the orthogonal complement
\[H^\perp=\{\alpha \mid (\alpha\cdot H)=0\},\]  
the intersection form is negative definite. 

We can also sketch a proof of the theorem. Being on a Kahler manifold, we can choose a Kahler class representative of $H$. On a Kahler surface we get an orthogonal decpomposition \begin{align*}
    H^2(X,\R) = \R\omega \oplus P^2(X,\R)
    \end{align*} where $P^2(X,\R) = \set{\alpha \in H^2(X,\R) \mid \alpha \wedge \omega = 0}$ is the primitive cohomology. Here we are invoking the Hard Lefschetz theorem. The Hodge-Riemann bilinear relations say, for primitive classes of type $(p,q)$ with $p+q=k$, we have
    \[(-1)^{\frac{k(k-1)}{2}} i^{p-q}\, Q(\alpha,\overline{\alpha}) > 0
    \quad\text{for } 0\neq \alpha \in P^k\cap H^{p,q}\]In our particular situation $k=2$ and $p=q=1$, so this says that $Q(\alpha,\overline{\alpha})<0$ for $\alpha \in P^2(X,\R)$, and the Hodge index theorem follows.
\end{remark}

Continuing the example, for K3 surfaces we have $H^2(T_{X_0})=0$, Schlessinger's criteria imply that the deformation functor of $X_0$ has a
miniversal formal family, hence there exists a formal scheme
\[
\mathcal X \longrightarrow \Spf R
\]
whose special fibre is $X_0$. However, no ample line bundle on $X_0$ lifts to $\mathcal X$, so Grothendieck's algebraization theorem does not apply.

\emph{Claim.} In this case the formal family $\mathcal X$ is not
effective: there does not exist a scheme $X$ of finite type, flat over $\Spec R$,
whose formal completion along the closed fibre is isomorphic to $\mathcal X$.

Suppose for contradiction that such a scheme $X\to \Spec R$ exists with
$\widehat X \cong \mathcal X$. Then $X$ is a family of smooth surfaces over $R$.
Since $H^2(T_{X_0})=0$ there are no obstructions to deforming
$X_0$, so $R$ is a power series ring; in particular, $X$ is regular.

Choose an affine open subset $U\subset X$ meeting the closed fibre $X_0$, and
choose a hyperplane section $Y\subset U$ that also meets $X_0$. Let $\overline Y$
be the scheme-theoretic closure of $Y$ in $X$. Then $\overline Y$ has codimension
$1$ in $X$, and since $X$ is regular, $\overline Y$ is a Cartier divisor. Its
intersection with the closed fibre is a nonzero effective divisor on $X_0$, hence
the invertible sheaf $\mathcal O_X(\overline Y)$ restricts to a nontrivial line
bundle on $X_0$. But by construction $\mathcal O_X(\overline Y)$ defines a line
bundle on the formal completion $\widehat X\cong \mathcal X$ lifting that
restriction to $X_0$. This is a contradiction and therefore $\mathcal X$ is not effective.
\end{example}

\subsection{Step 3: Descending to a finite type base}
The third and most difficult step is to descend from a family over a complete local ring \(R\) to a family defined over a scheme of finite type over \(k\).
More specifically, we ask for a scheme \(S\) of finite type over \(k\), a scheme
\(X\) flat and of finite type over \(S\), and a point \(s_0 \in S\), such that
the fiber of \(X\) over \(s_0\) is \(X_0\), and the formal completion of \(X\)
along \(X_0\) is isomorphic to the formal scheme \(\mathfrak X\) above.
This problem was addressed in a series of deep papers by Michael Artin. 

Schlessinger gives formal coordinates on moduli. Artin says these formal coordinates are actually restrictions of algebraic coordinates. Artin's theorem explains why local deformation theory produces algebraic spaces, stacks, and local charts. Instead of constructing global moduli, one constructs formal local moduli, algebraize, and glue étale-locally.

\begin{theorem}
Let \(X_0\) be a projective scheme over \(k\), and assume that \(X_0\)
admits an effective formal versal deformation \(\widehat{X}\) over the
complete local ring \(R\). Then \(\widehat{X}\) is algebraizable in the
following sense: there exists a scheme \(S\) of finite type over \(k\),
a point \(s_0 \in S\), and a flat finite--type family \(X\) over \(S\),
with fiber \(X_0\) over \(s_0\), such that
\[
R \cong \widehat{\mathcal O}_{S,s_0}
\qquad\text{and}\qquad
\widehat{X} \cong X \times_S \Spec R .
\]

Furthermore, the triple \((X,S,s_0)\) is unique locally around \(s_0\)
in the \'{e}tale topology, meaning that if \((X',S',s_0')\) is another
such triple, then there exist a scheme \(S''\), a point \(s_0''\), and
\'{e}tale morphisms
\[
S'' \longrightarrow S,
\qquad
S'' \longrightarrow S'
\]
sending \(s_0''\) to \(s_0\) and \(s_0'\), respectively, such that
\[
X \times_S S'' \;\cong\; X' \times_{S'} S'' .
\]
\end{theorem}
	

\begin{remark}
Note that the resulting family \(X/S\) is unique only up to \'{e}tale
coverings, so one does not obtain an actual moduli space by this method.
However, in the absence of automorphisms, the notion of algebraic space
works well; in the general case one needs the notion of stack.
In both cases, the definitions are designed so that algebraic families
defined only up to local \'{e}tale isomorphism still determine the
correct geometric object.

Thus the language of schemes naturally extends to algebraic spaces and
stacks, allowing one to work with moduli problems in this broader context.
The question of whether a moduli space is actually a scheme can often be
deferred or even ignored. The drawback is that working in these larger
categories introduces substantial technical overhead.
\end{remark}

It is well known that etale morphisms preserve infinitesimal neighborhoods. Artin's theorem gives a stunning converse to this fact: if two schemes have isomorphic infinitesimal neighborhoods, then they are locally isomorphic in the etale topology.
\begin{theorem}
Let $S$ be a scheme of finite type over a field $k$, let
$X_1$ and $X_2$ be schemes of finite type over $S$, let
$x_i \in X_i$ be points, and suppose that there is an
isomorphism of completed local rings
\[
    \widehat{\mathcal O}_{X_1,x_1}
    \;\cong\;
    \widehat{\mathcal O}_{X_2,x_2}
\]
over $S$. Then $X_1$ and $X_2$ are locally isomorphic in the
\'etale topology, that is, there exists another scheme
$X'$ of finite type over $S$ together with a point
$x' \in X'$, and \'etale morphisms
\[
    X' \longrightarrow X_1,
    \qquad
    X' \longrightarrow X_2,
\]
sending $x'$ to $x_1$ and $x_2$, respectively, and inducing
isomorphisms on the residue fields at $x',x_1,x_2$.
\end{theorem}

\section{Application to formal contractions}
This section follows an internal talk which I gave in Martin Olsson's seminar in the spring of 2026. We outline some of the content in \cite{artin-i} and \cite{artin-ii}. 
All algebraic spaces are defined over a field $\Spec k$. Artin works over a base algebraic space $S$ over $k$, but we will not need this level of generality.


\subsection{Motivating example}
Let $E \subset \mathbb{P}^2$ be a smooth plane cubic.
Choose a point $p_1$ on $E$ and blow up $\mathbb{P}^2$ at $p_1$. Take the unique point $p_2$ on the strict transform of $E$ living over $p_1$ and blow up at $p_2$. Continue this process, taking the unique $p_i$ on the strict transform of $E$ living over $p_{i-1}$ and blowing up at $p_i$ for $i=3,\dots,n$,
and let
\[
\pi : X' = \operatorname{Bl}_{\{p_1,\dots,p_n\}}(\mathbb{P}^2)
\longrightarrow
\mathbb{P}^2
\]
be the blowup at these points.
Let $\widetilde{E} \subset X'$ be the strict transform of $E$.
Then
\[
\widetilde{E}^2
=
E^2 - \sum_{i=1}^n (\operatorname{mult}_{p_i} E)^2
=
9 - n,
\]
since each $p_i$ is a smooth point of $E$ and
$\operatorname{mult}_{p_i} E = 1$.

Choosing $n = 10$, we obtain
\[
\widetilde{E}^2 = -1.
\]

Thus $\widetilde{E}$ is a smooth irreducible curve with
negative self-intersection.
The intersection matrix of
\[
Y' := \widetilde{E}
\]
is simply $(-1)$, which is negative definite.
By Artin's contraction criterion, there exists a proper morphism
$f : X' \longrightarrow X$ to an algebraic space $X$ which contracts $\widetilde{E}$ to a point $p := f(\widetilde{E})$ and is an isomorphism away from $\widetilde{E}$.

\noindent\textbf{$X$ is not a scheme.} Suppose, for contradiction, that $X$ is a scheme. Then $p$ admits an affine open neighbourhood
\[
U = \operatorname{Spec} R.
\]
Choose $g \in R$ with $g(p) \neq 0$.
The vanishing locus
\[
D := V(g) \subset U \subset X
\]
is a curve not containing $p$.
Since $f$ is an isomorphism away from $\widetilde{E}$,
the inverse image
\[
f^{-1}(D) \subset X'
\]
is a curve that does not intersect $\widetilde{E}$.
Let
\[
D_0 := \pi\bigl(f^{-1}(D)\bigr) \subset \mathbb{P}^2.
\]
Then $D_0$ is a plane curve which intersects the original cubic
$E \subset \mathbb{P}^2$ only at the point $p_1$.
Let
\[
D_0|_{E} = mp_1
\]
as a divisor on $E$. It is a general fact that if a plane curve $D_0$ intersects a smooth plane cubic $E$ in a divisor $D = \sum m_i P_i$, then $\sum m_i P_i = 0$ in the group law on $E$. This implies that
\[
mp_1 = 0
\]
in the group law on $E$. But we can choose $p_1$ to be a non-torsion point of $E$, so this is a contradiction.

\subsection{Artin's contraction criterion}

\begin{definition}
A \textbf{modification} consists of a proper morphism of algebraic spaces
    \[
        f : X' \to X,
    \]
    together with a closed subset $Y \subset X$, such that the restriction
\[
    f : X' \setminus f^{-1}(Y) \longrightarrow X \setminus Y
\]
is an isomorphism.
\end{definition}

The key example to keep in mind: \begin{example}
    Let $X$ be a scheme, $\cI \subset \cO_X$ a coherent ideal sheaf, and $Y = V(\cI)$ the closed subscheme defined by $\cI$. The blowup of $X$ along $\cI$ \begin{align*}
    \operatorname{Bl}_\cI(X) = \operatorname{Proj} \bigoplus_{n \geq 0} \cI^n
    \end{align*} is a modification of $X$ along $Y$.
\end{example}

\begin{proposition}\label{prop:main-example}
Let $X'$ be an algebraic space and let $Y' \subset X'$ be a closed
subspace such that the ideal sheaf $I' = \mathcal I(Y')$ is locally
principal. Let
\[
f_0 : Y' \to Y
\]
be a proper morphism.

Suppose the following two conditions hold, the first being an obstruction-vanishing condition and the second being a lifitng condition for functions.

\begin{enumerate}
\item For every coherent sheaf $F$ on $Y'$,
\[
R^1 f_{0*}\left(F \otimes (I'/I'^2)^{\otimes n}\right)=0
\quad\text{for } n \gg 0.
\]

\item For every $n$, the canonical map
\[
f_{0*}(\mathcal O_{X'}/I'^n)
\times_{f_{0*}(\mathcal O_{Y'})}
\mathcal O_Y
\longrightarrow
\mathcal O_Y
\]
is surjective where $\cO_Y\to f_{0*}(\mathcal O_{Y'})$ is the map coming from the contraction $f_0 : Y' \to Y$ and the map $f_{0*}(\mathcal O_{X'}/I'^n) \to f_{0*}(\mathcal O_{Y'})$ is the natural map coming from the inclusion $Y' \subset X'$.
\end{enumerate}

Then there exists a modification
\[
f : X' \to X
\]
and a closed subspace $Y \subset X$ whose set-theoretic restriction
to $Y$ is $f_0$.
\end{proposition}

\begin{corollary}
Suppose $Y'$ is proper over $\Spec k$. Then $Y'$ can be contracted to a point in $X'$ if condition \textup{(i)} of \ref{prop:main-example} holds. In particular, this is so in the following cases:

\begin{enumerate}
\item[(a)] $X'$ and $Y'$ are non-singular of dimensions $d$ and $d-1$
respectively, and the conormal bundle of $Y'$ in $X'$ is ample.

\item[(b)] $X'$ is regular of dimension $2$, and
$Y' = C_1 \cup \cdots \cup C_r$ is a complete connected curve on $X'$
with negative definite intersection matrix
$\big\| (C_i \cdot C_j) \big\|$,
the $C_i$ being the irreducible components of $Y'$.
\end{enumerate}
\end{corollary}


\begin{proof}
Note condition~(ii) of \ref{prop:main-example} holds automatically when $Y = \Spec k$. The map in question is \begin{align*}
f_{0*}(\mathcal O_{X'}/I'^n)
\times_{H^0(Y',\mathcal O_{Y'})} k \to k
\end{align*} and since we have the constant functions $k\to \cO_{X'}$, the projection will be surjective. We are simply asking do constant functions on the base lift to the $n$-th infinitesimal neighborhood?

We now explain why condition~(i) holds in cases~(a) and~(b).

\textbf{Case (a).}
Follows immediately from the properness of $Y'$ which allows us to apply Serre vanishing to the ample line bundle $I'/I'^2$ on $Y'$.

\textbf{Case (b).}
Since $X'$ is regular of dimension $2$, the divisor $Y'$ is Cartier and
\[
I'/I'^2 \cong \mathcal O_{Y'}(-Y').
\]
We want to show that this line bundle is ample on $Y'$. Since $Y'$ is a complete connected curve, it suffices to show that the degree of $I'/I'^2$ on every irreducible component of $Y'$ is positive. Computing, we find that \begin{align*}
\deg(I'/I'^2|_{C_i})
&=
\deg(\mathcal O_{Y'}(-Y')|_{C_i})
=
\deg(\mathcal O_{C_i}(-Y'))
=
-Y' \cdot C_i
\end{align*}
However, it is not going to be true that this degree is positive for every $i$, there can be some $i$ such that $Y' \cdot C_i \geq 0$. The workaround here is that we are only thinking about set theoretic contractions, so we can replace $Y'$ by a different effective divisor with the same support as $Y'$. 

It is a fact from linear algebra that if $M$ is symmetric negative definite and $M_{ii} < 0$ and $M_{ij} \geq 0$ for $i \neq j$, then all of the entries of $M^{-1}$ are negative.
Let $M$ be the intersection matrix of $Y'$ and consider the vector $a = (a_i)$ defined by $(-M)^{-1} \vec{1}$ and clear denominators to make $a$ integral. Then $Z = \sum a_i C_i$ is a effective divisor with the same support as $Y'$ with the property that $Z \cdot C_i = -1$ for every $i$. Replacing $Y'$ by $Y' + NZ$ for $N \gg 0$ gives a divisor with positive degree on every component, hence ample.
\end{proof}

\subsection{Jacobian ideal}
Let $f:\widehat{A} \to \widehat{B}$ be a finite type map of Noetherian adic rings. This means that $I\widehat{B}$ is an ideal of definition of $\widehat{B}$, and $\widehat{B}/I\widehat{B}$ is a finitely generated $\widehat{A}/I\widehat{A}$-algebra. 

This means for some ideal of definition $I \subset \widehat{A}$, we can write
\[
\widehat{B}
\cong
\widehat{A}\langle x_1,\dots,x_n\rangle/(f_1,\dots,f_m),
\]
where $\widehat{A}\langle x_1,\dots,x_n\rangle$
denotes the $I$-adic completion of the polynomial ring
$\widehat{A}[x_1,\dots,x_n]$, i.e. the ring of restricted power series whose coefficients tend to $0$ in the $I$-adic topology.
\begin{definition}
Let
\[
J = \left(\frac{\partial f_i}{\partial x_j}\right)
\]
be the $m\times n$ Jacobian matrix, where differentiation is
performed termwise on restricted power series. The \emph{Jacobian ideal}
\[
\mathcal{J}(\widehat{B}/\widehat{A})
\subset \widehat{B}
\]
is the ideal generated by the residues in $\widehat{B}$
of the $n\times n$ minors of $J$. If $m<n$, this ideal is defined to be $0$.
\end{definition}

\begin{proposition}
The Jacobian ideal is independent of the choice of presentation and commutes with etale localization, i.e. if $\widehat{B}'$ is an étale $\widehat{B}$-algebra, then $\cJ(B') = \cJ(B)\widehat{B}'$.
\end{proposition}

\begin{remark}
The Jacobian ideal detects where the module of differentials
$\Omega_{B/A}$ fails to vanish. Since for morphisms of finite presentation, formal étaleness is equivalent to $\Omega_{B/A}=0$, the vanishing of the Jacobian ideal characterizes the locus where the morphism is formally étale. The relevant statement is that for a morphism of rings $f:A \to B$ of finite presentation, $f$ is etale if and only if $\Omega_{B/A}=0$ and $f$ is flat. 
\end{remark}

\subsection{Cramer/Fitting ideal}
Let $A$ be any ring and let $M$ be a finitely presented $A$-module.
Choose a presentation
\[
L_1 \xlongrightarrow{f} L_0 \longrightarrow M \longrightarrow 0,
\]
with $L_i$ free $A$-modules of finite type.
Let $N$ be the rank of $L_0$.

For an integer $0 \leq d \leq N$, consider the homomorphism of $A$-modules
\[
\bigwedge^{N-d} L_1 \;\otimes\; \bigwedge^{d} L_0
\longrightarrow
\bigwedge^{N} L_0 \;\cong\; A,
\]
sending
\[
v_1 \wedge \cdots \wedge v_{N-d}
\;\otimes\;
u_1 \wedge \cdots \wedge u_d
\longmapsto
f(v_1) \wedge \cdots \wedge f(v_{N-d})  
\wedge
u_1 \wedge \cdots \wedge u_d.
\]
The image of this map is an ideal $\operatorname{Cr}_d(M) \subset A$.
\begin{itemize}
    \item The ideal $\operatorname{Cr}_d(M)$ is independent of the choice of presentation, and is called the \emph{Cramer ideal} of $M$.
    \item $\operatorname{Cr}_d(M)$ is the unit ideal if and only if
$M$ is locally generated on $\operatorname{Spec} A$ by $d$ elements.
\end{itemize}
Recall the presentation of $\widehat{B}$ as a quotient of a restricted power series ring over $\widehat{A}$:
\[\widehat{B}
\cong
\widehat{A}\langle x_1,\dots,x_n\rangle/(f_1,\dots,f_m) \cong \widehat{A}\langle x_1,\dots,x_n\rangle/\mathfrak{a}
\] We define $\mathcal{C}(\overline{B})$ to be the residue in $\overline{B}$ of the Cramer ideal
$\operatorname{Cr}_n(\mf a)$:
\begin{equation*}
\mathcal{C}(\overline{B})
=
(\operatorname{Cr}_n(\mf a) + \mathfrak{a})
/ \mathfrak{a}.
\end{equation*}

\begin{proposition}
    The Cramer ideal $\mathcal{C}(\overline{B})$ is independent of the choice of presentation of $\widehat{B}$ and commutes with étale localization.
\end{proposition}

\begin{remark}
    The Cramer ideal detects where the kernel of the presentation fails to be generated by the expected number of equations. When solving a linear system $Ax=b$, Cramer's rule says that the vanishing of minors witnesses the failure of the system to have maximal rank.
\end{remark}


\subsection{Formal modifications and algebraization}
Let $X$ be a formal algebraic space with defining ideal $\mathcal{I}$, and assume the closed algebraic subspace
$Y = V(\mathcal{I})$ is of finite type over $k$. This roughly means that $X$ is assembled from formal affine schemes $\operatorname{Spf}(A)$ where $(A,I)$ is a Noetherian adic ring, quotiented by an etale equivalence relation.

In order to state the precise definition of what a formal modification is, we need to introduce the Jacobian ideal and the Cramer (Fitting) ideal of a morphism of formal algebraic spaces. 

We are now prepared to give the definition of a formal modification. Let
\[
\widehat{f} : \widehat{X}' \to \widehat{X}
\]
be a proper morphism of formal algebraic spaces of finite type.

\begin{definition}
We say $\widehat{f}$ is a \textbf{formal modification along $Y$} if the following conditions hold:
\begin{itemize}
    \item \textbf{Formal étaleness outside $Y$.}
    Let $\mathcal{J}(\widehat{f})$ denote the Jacobian ideal
    and $\mathcal{C}(\widehat{f})$ the Cramer (Fitting) ideal of $\widehat{f}$.
    Then both ideals contain an ideal of definition of $\widehat{X}'$. This roughly says that $\widehat{f}$ is formally étale and locally a complete intersection outside $Y'$.
    \item \textbf{Monomorphism outside $Y$.}
    Let
    \[
    \Delta : \widehat{X}' \longrightarrow
    \widehat{X}' \times_{\widehat{X}} \widehat{X}'
    \]
    be the diagonal morphism.
    Let $\mathcal{I}'$ be the ideal defining the diagonal,
    and let $\mathcal{I}''$ be a defining ideal of
    $\widehat{X}' \times_{\widehat{X}} \widehat{X}'$. Then there exists $N \gg 0$ such that
    \[
    (\mathcal{I}'')^N \mathcal{I}' = 0.
    \]
    \item \textbf{Formal surjectivity outside $Y$.}
    Let $R$ be a complete discrete valuation ring whose residue field is of finite type over $S$.
    Let $Z = \operatorname{Spf}(R)$. Then every adic morphism
    \[
    Z \to \widehat{X}
    \]
    whose closed point maps outside $Y$
    lifts to a morphism
    \[
    Z \to \widehat{X}'
    \]
\end{itemize}
\end{definition}
The most forthcoming examples of formal modifications is the following: if $f : X' \to X$ is a modification, then we can complete along $Y$ and $Y' = f^{-1}(Y)$ to get a proper morphism of formal algebraic spaces $\widehat{f} : \widehat{X}' \to \widehat{X}$ which is a formal modification along $Y$. A major result of Artin is that the converse also holds: every formal modification can be algebraized to a modification.

\begin{theorem}
    Given a formal modification $\widehat{f} : \widehat{X}' \to \widehat{X}$ along $Y$, there exists a modification $g: X' \to X$ and an isomorphism of formal modifications $\phi:\widehat{f} \cong \widehat{g}$. The pair $(g,\phi)$ is unique up to unique isomorphism.
\end{theorem}

In order to apply the above result, Artin gives a mechanism for constructing formal modifications. The following theorem says that to construct a formal modification, it is enough to construct a contraction on the closed fiber and check two extension conditions that we saw earlier.
\begin{theorem}\label{thm:artin-formal-modif}
Let $\mathfrak X'$ be a formal algebraic space, let
$Y' = V(I') \subset \mathfrak X'$ be the closed subspace defined by
a defining ideal $I'$, and let
$f : Y' \to Y$ be a proper morphism of algebraic spaces.
Assume
\begin{enumerate}
\item For every coherent sheaf $F$ on $\mathfrak X'$, we have
\[
R^1 f_*(I'^n F / I'^{n+1} F) = 0
\quad\text{for } n \gg 0.
\]
\item For every $n$, the natural map of sheaves on $Y$
\[
f_*(\mathcal O_{\mathfrak X'}/I'^n)
\times_{f_*(\mathcal O_{Y'})}
\mathcal O_Y
\longrightarrow
\mathcal O_Y
\]
is surjective.
\end{enumerate}

Then there exists a formal algebraic space $\mathfrak X$
and a defining ideal $I \subset \mathcal O_{\mathfrak X}$
such that $V(I)=Y$ and a formal modification
\[
\widehat f : \mathfrak X' \to \mathfrak X
\]
whose restriction to $Y'$ is the given morphism $f$.
\end{theorem}

\subsection{Proof sketch}
In this section, I will give a sketch of the proof of Theorem \ref{thm:artin-formal-modif}. We are showing that given a formal modification $\widehat{f} : \widehat{X}' \to \widehat{X}$ along $Y$, we can algebraize it to a modification $g : X' \to X$.

The strategy is to re-phrase the construction as a modular problem, i.e. representability of a suitable functor on noetherian affine schemes \(Z=\Spec A\).  

We assign to $Z$ the set of triples consisting of
\begin{enumerate}
    \item a closed subset $C\subset Z$,
    \item a map on the complement $g_V:V:=Z\setminus C\to U$
	\item an adic map on the formal completion $\Spf(\widehat A)\to \mathfrak X$ subject to a compatibility (commutative diagram) condition, where $\widehat A$ is the $I$-adic completion of $A$, $I$ being the defining ideal of $C$ in $Z$.
\end{enumerate}

Artin then shows that this functor is representable by an algebraic space $X$ of finite type over $k$, using criteria which he developed in \cite{artin-i}.

\begin{theorem}[Artin representability criterion]
Let $F$ be a contravariant functor from $k$-schemes to sets. Then $F$ is represented by a locally separated algebraic space (respectively, a separated algebraic space) locally of finite type over $k$ if and only if the following conditions hold:

\begin{enumerate}
\item[\textup{[0]}] (\textbf{Sheaf condition})  
$F$ is a sheaf for the étale topology.

\item[\textup{[1]}] (\textbf{Finite presentation})  
$F$ is locally of finite presentation.

\item[\textup{[2]}] (\textbf{Effective pro-representability})  
$F$ is effectively pro-representable.

\item[\textup{[3]}] (\textbf{Representable diagonal})  
For every $k$-scheme $X$ of finite type and $\xi,\eta \in F(X)$, the condition $\xi = \eta$ is represented by a (closed) subscheme of $X$.

\item[\textup{[4]}] (\textbf{Openness of formal étaleness})  
Let $X$ be an $k$-scheme of finite type and let $\xi : X \to F$ be a morphism. If $\xi$ is formally étale at a point $x \in X$, then it is formally étale in a neighborhood of $x$.
\end{enumerate}
\end{theorem}

\begin{remark}
    Recall that functor $F$ is pro-representable at a point if ts restriction to Artinian local rings looks like
\[
\Hom(\Spf R, -)
\]
for some complete local ring $R$. It is effectively pro-representable if formal solutions glue to a genuine formal object, i.e. if we have a compatible family of elements $\xi_n \in F(\Spec R/\mathfrak{m}^n)$, then there exists $\xi \in F(\Spf R)$ such that $\xi|_{\Spec R/\mathfrak{m}^n} = \xi_n$ for every $n$.

Recall that a morphism of functors
$\xi : X \to F$
is formally étale at a point if:

For every square-zero extension $A \to A/I$ every lifting problem
$\Spec A/I \to X$ over $\Spec A \to F$ has a unique solution.
\end{remark}

Establishing the effective pro representability condition is where we must invoke the two conditions in the statement of the theorem. So suppose we are given compatible contraction data $\xi_n \in F(R/\mathfrak m^{n+1})$
for all $n$.

We must construct a unique $\xi \in F(R)$
lifting all the $\xi_n$. To extend from level $n$ to $n+1$, we analyze the short exact sequence
\[0 \to I'^n/I'^{n+1}
\to \mathcal O/I'^{n+1}
\to \mathcal O/I'^n
\to 0\]
The hypotheses of the theorem allow us to construct $\mathcal A = \varprojlim_n f_*(\mathcal O_{\mathfrak X'}/I'^n)$ whose relative formal spectrum $\mathfrak X := \operatorname{Spf}_Y(\mathcal A)$ represents the restriction of the functor $F$ to the category of complete local rings.

At this point we have constructed a formal algebraic space $\mathfrak X $representing the contraction functor on complete local rings. However, this is only a formal object. The remaining step is algebraization: to show that $\mathfrak X$ is the formal completion of an honest algebraic space $X$ along $Y$. This is highly nontrivial and uses Artin's deep algebraization theorem.

\section{Algebraic spaces}
Our discussion of deformation theory and moduli has led us to the conclusion that the best we can hope for is to construct a family of deformations over a base scheme $S$ that is unique only up to étale localization on $S$. This motivates the following definition of algebraic spaces, which are designed to be the correct generalization of schemes in which we can still do geometry, but where we allow objects that are only locally schemes in the étale topology.
\begin{definition}
    An \textbf{algebraic space} over a base scheme $S$ is a functor $X \colon (\text{Sch}/S)^{op} \to \text{Sets}$ satisfying
    \begin{enumerate}
        \item $X$ is a sheaf in the fppf topology,
        \item the diagonal morphism $\Delta \colon X \to X \times_S X$ is representable by schemes,
        \item there exists a surjective étale morphism $U \to X$ from a scheme $U$.
    \end{enumerate}
\end{definition}


By definiton, if $x\in X$ is a point of an algebraic space $X$, then $x$ admits an étale neighborhood $U \to X$ where $U$ is a scheme.


\begin{example}[An \'{e}tale map need not induce an isomorphism on ordinary local rings]
    Let
    \[
        X = \Spec \mathbb{C}[t,t^{-1}] \qquad
        Y = \Spec \mathbb{C}[u,u^{-1}],
    \]
    and let $f\colon Y \to X$ be the finite morphism given by $t = u^2$.
    This morphism is \'{e}tale: indeed, locally it is given by adjoining a root of
    $g(T)=T^2-t$, and on $Y$ we have $g'(u)=2u\in \mathbb{C}[u,u^{-1}]^{\times}$.

    Consider the generic point $\eta\in X$, so $\mathcal{O}_{X,\eta}=\kappa(\eta)=\mathbb{C}(t)$.
    Let $\eta'\in Y$ be the generic point lying over $\eta$, so
    $\mathcal{O}_{Y,\eta'}=\kappa(\eta')=\mathbb{C}(u)=\mathbb{C}(t^{1/2})$.
    The induced map on local rings
    \[
        \mathcal{O}_{X,\eta}=\mathbb{C}(t) \longrightarrow \mathcal{O}_{Y,\eta'}=\mathbb{C}(t^{1/2})
    \]
    coming from $t\mapsto u^2$ is not an isomorphism.
\end{example}

In this example, we are making use of the following general criterion for \'{e}taleness:
\begin{definition}[Standard \'{e}tale]\label{def:standard-etale}
    Let $R$ be a ring and let $g,f\in R[x]$. Assume that $f$ is monic and that the
    formal derivative $f'$ maps to a unit in the localization
    \[
        R[x]_g/(f).
    \]
    In this case the ring map
    \[
        R \longrightarrow R[x]_g/(f)
    \]
    is said to be \textbf{standard \'{e}tale}.
\end{definition}

\begin{lemma}\label{lem:standard-etale-basic}
    Let $R \to R[x]_g/(f)$ be standard \'{e}tale.
    \begin{enumerate}
        \item[(1)] The ring map $R \to R[x]_g/(f)$ is \'{e}tale.
        \item[(2)] For any ring map $R\to R'$, the base change
              \[
                  R' \longrightarrow R'[x]_{g'}/(f')
                  \qquad\text{with } R'[x]_{g'}/(f') \cong R'\otimes_R \bigl(R[x]_g/(f)\bigr)
              \]
              is standard \'{e}tale (where $f',g'$ denote the images of $f,g$ in $R'[x]$).
        \item[(3)] Any principal localization of $R[x]_g/(f)$ is standard \'{e}tale over $R$.
        \item[(4)] A composition of standard \'{e}tale maps need not be standard \'{e}tale in general.
    \end{enumerate}
\end{lemma}

The example shows that the ordinary local ring $\mathcal{O}_{X,x}$ is not preserved
under passing to an \'{e}tale neighborhood. The correct object that is invariant under passing to an \'{e}tale neighborhood is not theordinary local ring $\mathcal{O}_{X,x}$ but the \textbf{\'{e}tale stalk} of the structure sheaf, which is canonically identified with a strict henselization of the ordinary local ring.



\begin{definition}
    Given a site $(\mathcal{C}, J)$, a point of the associated topos $\mathbf{Sh}(\mathcal{C}, J)$ is (equivalently) a functor
    $$p^{-1} : \mathbf{Sh}(\mathcal{C}, J) \to \mathbf{Sets}$$
    that is left exact and has a right adjoint (a ``geometric morphism'' $\mathbf{Sets} \to \mathbf{Sh}(\mathcal{C}, J)$). The stalk of a sheaf $\mathcal{F}$ at the point $p$ is $p^{-1}(\mathcal{F})$.
\end{definition}

Points are not objects of $\mathcal{C}$ a priori; they are topos-theoretic. One can show a geometric point of $X$ is a morphism
\[
    \bar{x}\colon \Spec(\Omega) \to X
\]
with $\Omega$ separably closed. It lies over an underlying (ordinary) point $x \in |X|$, and choosing $\bar{x}$ is equivalent to choosing an embedding
\[
    \kappa(x) \hookrightarrow \Omega
\]
into a separably closed field.

The subtlety is:
\begin{itemize}
    \item The étale stalk $(\mathcal{O}_X)_{\bar{x}}$ is a stalk in the étale topology, and stalks are taken at points of the étale topos, i.e.\ geometric points, not just underlying Zariski points.
    \item The strict henselization $(\mathcal{O}_{X,x})^{\mathrm{sh}}$ is not determined by $x$ alone: it depends on a choice of a separable closure of the residue field $\kappa(x)$ inside some separably closed field. That choice is precisely the extra datum carried by $\bar{x}$.
\end{itemize}

So if you specify only $x$, then $(\mathcal{O}_{X,x})^{\mathrm{sh}}$ is well-defined only up to noncanonical isomorphism; once you specify $\bar{x}$, it becomes canonical. Let $X$ be an algebraic space.
\begin{itemize}
    \item An underlying point $x\in|X|$ does not determine a canonical local ring. Any attempt to define $\mathcal O_{X,x}$ requires choosing an \'{e}tale chart and a lift of $x$, and different choices need not agree.
    \item A geometric point $\bar x\to X$, however, does determine a canonical local ring:
          \[
              \mathcal O_{X,\bar x}:=(\mathcal O_X)_{\bar x},
          \]
          which is invariant under \'{e}tale pullback.
\end{itemize}

\begin{definition}[\'{E}tale local ring]
    Let $X$ be an algebraic space and let $\bar x\colon \Spec(\Omega)\to X$ be a geometric point lying over $x\in |X|$.
    The \textbf{\'{e}tale local ring} of $X$ at $\bar x$ is the stalk
    \[
        \mathcal{O}_{X,\bar x} := (\mathcal{O}_X)_{\bar x} = \varinjlim_{(U,\bar u)\to (X,\bar x)} \Gamma(U,\mathcal{O}_U)
    \]
    of the structure sheaf on the \'{e}tale site $X_{\'{e}t}$.
\end{definition}
If $X$ is a scheme (or after choosing an \'{e}tale chart $U\to X$ with a lift of $\bar x$), the following lemma above identifies this canonically with the strict henselization
\[
    \mathcal{O}_{X,\bar x} \;\cong\; \bigl(\mathcal{O}_{X,x}\bigr)^{sh}.
\]

\begin{lemma}[Stacks Project, Lemma 59.33.1]
    Let $S$ be a scheme and let $\bar s$ be a geometric point of $S$ lying over $s\in S$.
    Let $\kappa=\kappa(s)$ and let $\kappa^{\mathrm{sep}}\subset \kappa(\bar s)$ denote the separable algebraic closure of $\kappa$ inside $\kappa(\bar s)$.
    Then there is a canonical identification
    \[
        \bigl(\mathcal{O}_{S,s}\bigr)^{sh} \;\cong\; (\mathcal{O}_S)_{\bar s},
    \]
    where the left-hand side is the strict henselization of the local ring at $s$, and the right-hand side is the stalk of the structure sheaf on the \'{e}tale site $S_{\'{e}t}$ at the geometric point $\bar s$.
\end{lemma}

Even though the etale local ring is defined as a colimit over all etale neighborhoods, it can be computed from any single etale neighborhood. This is a consequence of the following proposition about weakly etale maps, and the fact that etale maps are weakly etale.

\begin{definition}
    A map $f \colon A \to B$ of rings is called weakly étale if it is flat and the diagonal map $B \otimes_A B \to B$ is also flat.
\end{definition}

\begin{proposition}
    Let $A \to B$ be a weakly étale map of local rings. Then the induced map on strict henselizations $A^{sh} \to B^{sh}$ is an isomorphism, provided you choose the strict henselizations compatibly (pick a separable closure of $\kappa(x)$, and take the separable closure of $\kappa(y)$ inside it) where $x$ and $y$ are the closed points of $\Spec A$ and $\Spec B$ respectively.
\end{proposition}

\begin{definition}[Ideal sheaves on an algebraic space]
    Let $X$ be an algebraic space.
    An \textbf{ideal sheaf} $I \subset \mathcal O_X$ is a subsheaf of rings of the
    structure sheaf on the \'{e}tale site $X_{\'{e}t}$ such that, \'{e}tale-locally,
    it is an ideal in the usual sense.

    Concretely, choosing an \'{e}tale surjection $U \to X$ with $U$ a scheme,
    an ideal sheaf $I \subset \mathcal O_X$ is equivalent to an ideal
    $I_U \subset \mathcal O_U$ such that the pullbacks of $I_U$ to
    $U \times_X U$ via the two projections agree.
\end{definition}

\begin{remark}[Ideal sheaves versus local rings on algebraic spaces]
    There is an important asymmetry between ideal sheaves and local rings on an
    algebraic space.

    \smallskip
    \noindent
    \textbf{(1) Ideal sheaves descend \'{e}tale-locally.}
    Ideal sheaves are sheaf-theoretic objects on the \'{e}tale site.
    Because \'{e}tale morphisms are flat and the \'{e}tale topology admits effective
    descent for quasi-coherent sheaves, an ideal of $\mathcal O_X$ may be defined
    on any \'{e}tale chart $U \to X$ and uniquely descended to $X$.
    In particular, closed subspaces and formal completions of algebraic spaces are
    defined \'{e}tale-locally on schemes.

    \smallskip
    \noindent
    \textbf{(2) Ordinary local rings do \textbf{not} descend \'{e}tale-locally.}
    The ordinary local ring $\mathcal O_{X,x}$ at a point $x \in |X|$ is defined as a
    filtered colimit over Zariski neighborhoods of $x$.
    Since \'{e}tale morphisms need not identify Zariski neighborhoods, passing to an
    \'{e}tale chart generally changes the ordinary local ring.
    Thus ordinary local rings are not invariant under \'{e}tale pullback.

    \smallskip
    \noindent
    \textbf{(3) The correct local object is the \'{e}tale stalk.}
    The local invariant preserved by \'{e}tale morphisms is instead the
    \'{e}tale stalk $(\mathcal O_X)_{\bar x}$ at a geometric point
    $\bar x \colon \Spec(\Omega)\to X$, which is canonically identified with the
    strict henselization $(\mathcal O_{X,x})^{sh}$.
    Unlike the Zariski local ring, this object is \'{e}tale-local by construction.

    \smallskip
    \noindent
    In summary, ideal sheaves are global objects defined by \'{e}tale descent, while
    ordinary local rings depend on the Zariski topology and are not compatible with
    \^{e}tale localization. This distinction underlies the use of \'{e}tale topology
    and formal geometry in the theory of algebraic spaces.
\end{remark}

\subsection{The \'{e}tale topology over $\mathbf C$}

Even when $X$ is a scheme of finite type over $\mathbf C$, the \'{e}tale topology is
strictly finer than the Zariski topology and leads to a genuinely different notion
of locality.
\begin{example}
    Over $\mathbf{C}$, finite étale morphisms are exactly finite unbranched coverings in the analytic topology. Consider the map
    \[
        \mathbb{G}_m \to \mathbb{G}_m,\quad z\mapsto z^n
    \]
    is finite étale but not Zariski locally trivial.

    \[
        X=\Spec \mathbf C[t,t^{-1}],\qquad Y=\Spec \mathbf C[u,u^{-1}],\qquad t=u^n.
    \]
    Then
    \[
        \mathbf{C}[u,u^{-1}] \cong \mathbf{C}[t,t^{-1}][T]/(T^n-t).
    \]
    So $Y \to X$ is finite (monic polynomial).

    For étale: use the standard criterion for $A[T]/(f)$ with $f$ monic.
    Here $f(T) = T^n - t$, so $f'(T) = nT^{n-1}$. In $B = \mathbf{C}[u,u^{-1}]$,
    $f'(u) = n u^{n-1}$,
    which is a unit since $u$ is invertible on $\mathbb{G}_m$. Hence the map is étale everywhere.

    However, this map is not Zariski locally trivial: there is no Zariski open cover of $X$ over which the map splits as a disjoint union of copies of the base. Equivalently, on rings this says: for each i, the $A_i$-algebra
    $A_i[T]/(T^n-t)$
    \quad\text{with } $A_i=\Gamma(U_i,\mathcal{O}_X)$
    splits as a product $A_i\times\cdots\times A_i$. That happens iff $T^n-t$ factors into linear factors over $A_i$, i.e.\ iff $t$ admits an $n$-th root in $A_i$:
    $t = s_i^n$ \quad\text{in } $A_i$.

    But $X=\Spec \mathbf{C}[t,t^{-1}]$ is irreducible, so at least one $U_i$ contains the generic point. For that $U_i$, the inclusion of rings
    $\mathbf{C}(t)\hookrightarrow \operatorname{Frac}(A_i)$
    holds, and $t=s_i^n$ in $A_i$ would imply $t$ is an $n$-th power in the function field $\mathbf{C}(t)$. That is false: in $\mathbf{C}(t)$, the element $t$ has valuation $1$ at $t=0$, so it cannot be an $n$-th power (an $n$-th power would have valuation divisible by $n$). Contradiction.


\end{example}

Let $x\in X$ be a closed point and let $\bar x$ be a geometric point lying over $x$.
Since $\kappa(x)=\mathbf C$ is already separably closed, the \'{e}tale local ring is
given by
\[
    \mathcal{O}^{\acute et}_{X,\bar x}
    \;=\;
    (\mathcal{O}_{X,x})^{sh}
    \;=\;
    (\mathcal{O}_{X,x})^{h},
\]
the henselization of the Zariski local ring.
In general this ring is strictly larger than $\mathcal{O}_{X,x}$, although the two
have the same completion:
\[
    \widehat{\mathcal{O}_{X,x}}
    \;\cong\;
    \widehat{\mathcal{O}^{\acute et}_{X,\bar x}}.
\]

Thus completion only captures the formal neighborhood of $x$, while the \'{e}tale
local ring retains the minimal henselian enlargement required to solve \'{e}tale
equations (via Hensel's lemma).
In this sense, \'{e}tale locality is strictly stronger than Zariski locality even
over $\mathbf C$.


\section{Algebraic approximation}
This section exposits some of Artin's results on algebraic approximation, which he details in \cite{artin-alg}. Let $A$ be a Noetherian ring and $\mf m$ be an ideal. Suppose we are given a moduli functor
\[
    F : (A\text{-algebras}) \longrightarrow (\text{sets}),
\]
so that for every $A$-algebra $B$, the set $F(B)$ is the set of isomorphism classes of structures over $B$.

Then an element $\xi \in F(A)$ or $\bar{\xi} \in F(\widehat{A})$
induces by functoriality an element of $F(A/\mathfrak m^c)$ for each $c$,
and we will say that $\xi$ and $\bar{\xi}$ are
\textbf{congruent modulo $\mathfrak m^c$} if they induce the same element there.

Artin introduces the following question.
\begin{remark}\label{question-1.4}
    Let $c$ be an integer, and let
    $\bar{\xi} \in F(\widehat{A})$.
    Does there exist an element $\xi \in F(A)$ such that
    \[
        \xi \equiv \bar{\xi} \pmod{\mathfrak m^c} \, ?
    \]
    In order to answer this question, Artin restricts to the case where $F$ is a functor of finite presentation, which means that $F$ commutes with filtered colimits of $A$-algebras. This is a natural condition in moduli problems, and it is satisfied by many moduli functors of interest.
\end{remark}

\begin{proposition}[Functorial characterisation of finite presentation]
    Let $A$ be a ring and $B$ an $A$--algebra.  The following are equivalent:
    \begin{enumerate}
        \item $B$ is finitely presented as an $A$--algebra, i.e.\ $B$ admits a presentation
              \[
                  B \cong A[x_1,\dots,x_n]/(f_1,\dots,f_m)
              \]
              for some finite $n,m$ and some $f_i\in A[x_1,\dots,x_n]$.
        \item for every filtered inductive system of $A$--algebras $\{B_i\}$,
              the canonical map
              \[
                  \varinjlim_i \Hom_A(B,B_i)
                  \;\longrightarrow\;
                  \Hom_A\!\left(B,\varinjlim_i B_i\right)
              \]
              is bijective.
    \end{enumerate}
\end{proposition}

\begin{proof}
    \textbf{(1) $\Rightarrow$ (2).}
    Assume that $B$ is finitely presented over $A$.
    Choose a presentation
    \[
        B \cong A[x_1,\dots,x_n]/(f_1,\dots,f_m).
    \]
    A morphism $B\to \varinjlim_i B_i$ is determined by the images of the
    generators $x_j$ subject to the finitely many relations $f_k=0$.
    In a filtered colimit, finitely many elements and finitely many
    equalities already occur in some stage $B_{i_0}$.
    Hence every map $B\to \varinjlim_i B_i$ factors through some $B_{i_0}$,
    and two such factorizations become equal in a later stage.
    This proves bijectivity.

    \textbf{(2) $\Rightarrow$ (1).}
    Write $B=\varinjlim_i B_i$ as a filtered colimit of finitely presented
    $A$--algebras (possible for any algebra).
    Apply the hypothesis to the identity map
    \[
        \id_B \in \Hom_A(B,B)
        = \Hom_A\!\left(B,\varinjlim_i B_i\right).
    \]
    By assumption, this element comes from some
    $\Hom_A(B,B_{i_0})$; hence the identity factors as
    \[
        B \longrightarrow B_{i_0} \longrightarrow B .
    \]
    Thus $B$ is a direct summand of the finitely presented algebra $B_{i_0}$.
    Since finite presentation is stable under retracts,
    $B$ is finitely presented over $A$.
\end{proof}
\begin{corollary}
    For a morphism of schemes $f:X\to S$, the following are equivalent:
    \begin{enumerate}
        \item $f$ is locally of finite presentation in the EGA sense, i.e.\ for every affine open $\Spec A\subset S$ and every affine open $\Spec B\subset f^{-1}(\Spec A)$, the ring $B$ is a finitely presented $A$--algebra.
        \item for every affine $T=\Spec A$ over $S$, the functor
              $T'\mapsto \Hom_S(T',X)$ on $A$--algebras
              commutes with filtered colimits.
    \end{enumerate}
\end{corollary}

\begin{corollary}[1.6]
    Let $F$ be a functor \textbf{locally of finite presentation} as in \textbf{(1.3)},
    let $B$ be an $A$-algebra, and let $\xi \in F(B)$.
    Then there exist:

    \begin{enumerate}
        \item[(i)] A finite system of polynomial equations
              \[
                  f(Y)=0,
              \]
              where $Y=(Y_1,\dots,Y_N)$ and
              $f=(f_1,\dots,f_m)\in A[Y]$;

        \item[(ii)] A functorial rule associating to every solution of this system of
              equations in an $A$-algebra $C$ an element of $F(C)$;

        \item[(iii)] A solution of the system \textbf{(i)} in $B$, so that the rule
              \textbf{(ii)} applied to this solution yields $\xi$.
    \end{enumerate}
\end{corollary}

\begin{remark}\label{question-1.7}
    Let $Y=(Y_1,\dots,Y_N)$ be variables, let
    $f=(f_1,\dots,f_m)\in A[Y]$ be polynomials, and suppose given elements
    $\bar y=(\bar y_1,\dots,\bar y_N)\in \widehat{A}$ which solve the system of
    polynomial equations
    \[
        f(Y)=0. \tag{$*$}
    \]
    Let $c$ be an integer.
    Does there exist a solution $y=(y_1,\dots,y_N)\in A$ of $(*)$ such that
    \[
        y_i \equiv \bar y_i \pmod{\mathfrak m^c} \, ?
    \]
\end{remark}

\begin{corollary}
    Suppose for the given pair $(A,\mathfrak m)$ that the answer to
Remark~\ref{question-1.7} is affirmative for each system $(*)$ of polynomial equations.
    Then Remark~\ref{question-1.4} has an affirmative answer for every functor
    \textbf{(1.3)} locally of finite presentation.
\end{corollary}

We now ask for conditions under which Remark~\ref{question-1.7} has an affirmative answer.
If the system of equations $(*)$ is linear, then it is sufficient that $A$ be a
local ring and $\mathfrak m \neq A$ (or, more generally, that $\mathfrak m$ be
in the Jacobson radical of $A$), as follows immediately from the faithful
flatness of $\widehat A$ over $A$ in that case.
Consequently, Remark~\ref{question-1.4} also has an affirmative answer for functors $F$
which are "sufficiently linear." 
\begin{remark}
    This linearity condition is also exemplified by the fact that finite modules $M,M'$ over a local ring $A$ are isomorphic if and only if their completions $\widehat{M},\widehat{M'}$ are isomorphic as modules over $\widehat{A}$.
\end{remark}

But most structures are not described by linear equations, and so it is natural
to study these questions locally for the \'{e}tale topology.
This amounts to assuming that $A$ is a \textbf{henselian} local ring.

\begin{definition}
    A local ring $(A,\mathfrak m)$ is \textbf{henselian} if for every system of polynomial equations
    \[
        f(Y)=0,
    \]where $Y=(Y_1,\dots,Y_N)$ and $f=(f_1,\dots,f_m)\in A[Y]$, and for every solution $\bar y=(\bar y_1,\dots,\bar y_N)\in k=A/\mathfrak m$ of the system of equations modulo $\mathfrak m$, so that the Jacobian matrix $\left(\frac{\partial f_i}{\partial Y_j}\right)$ has full rank at $\bar y$,
    there exists a solution $y=(y_1,\dots,y_N)\in A$ of the system of equations such that
    \[
        y_i \equiv \bar y_i \pmod{\mathfrak m}.
    \]
\end{definition}
We are now in a position to state Artin's main algebraic approximation result.

\begin{theorem}\label{thm:artin-alg-approx}
    Let $R$ be a field and let $A$ be
    the henselization of an $R$-algebra of finite type
    at a prime ideal.
    Let $\mathfrak m$ be a proper ideal of $A$.
    Given an arbitrary system of polynomial equations
    \[
        f(Y)=0, \qquad Y=(Y_1,\dots,Y_N),
    \]
    with coefficients in $A$, a solution
    $\bar y=(\bar y_1,\dots,\bar y_N)$ in the $\mathfrak m$-adic completion
    $\widehat A$ of $A$, and an integer $c$,
    there exists a solution $y=(y_1,\dots,y_N)\in A$ with
    \[
        y_i \equiv \bar y_i \pmod{\mathfrak m^c}.
    \]
\end{theorem}

\medskip

Thus we obtain an affirmative response to
Remark~\ref{question-1.4}:

\begin{theorem}
    With the assumptions of Theorem~\ref{thm:artin-alg-approx}, let $F$ be a functor which is locally of finite presentation.
    Given any $\bar\xi \in F(\widehat A)$, there is a
    $\xi \in F(A)$ such that
    \[
        \xi \equiv \bar\xi \pmod{\mathfrak m^c}.
    \]
\end{theorem}

\section{References}
\begin{enumerate}
    \bibitem{artin-i}
    M. Artin,
    ``Algebraization of formal moduli I,''
    in \textit{Global Analysis (Papers in Honor of K. Kodaira)},
    Univ. Tokyo Press, Tokyo, 1969, pp.~21–71.

    \bibitem{artin-ii}
    M. Artin,
    ``Algebraization of formal moduli II: Existence of modifications,''
    \textit{Ann. of Math. (2)} \textbf{91} (1970), 88–135.

    \bibitem{artin-alg}
    M. Artin, ``Algebraic approximation of structures over complete local rings,''
    \textit{Inst. Hautes Études Sci. Publ. Math.} \textbf{36} (1969), 23–58.
    \bibitem{hartshorne}
    R. Hartshorne,
    \textit{Deformation Theory},
    Graduate Texts in Mathematics, Vol.~257,
    Springer, New York, 2010.
    \bibitem{knutson}
    D. Knutson,
    \textit{Algebraic Spaces},
    Lecture Notes in Mathematics, Vol.~203,
    Springer, Berlin, 1971.
\end{enumerate}
\end{document}