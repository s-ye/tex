\documentclass[12pt]{article}
\usepackage[english]{babel}
\usepackage[utf8x]{inputenc}
\usepackage[T1]{fontenc}
\usepackage{listings}
\usepackage{bookmark}
\usepackage{tikz}

\makeatletter
\def\input@path{{../../style/}}
\makeatother

\usepackage{../../style/quiver}
\makeatletter
\def\input@path{{../../style/}}
\makeatother

\usepackage{../../style/scribe}
\usepackage{fancyhdr}

\usepackage{parskip} % Automatically respects blank lines
\setlength{\parskip}{1em} % Adds more space between paragraphs
\setlength{\parindent}{0pt} % Removes paragraph indentation

\begin{document}


\lhead{Songyu Ye}
\rhead{\today}
\cfoot{\thepage}

\title{Title}

\author{Songyu Ye}
\date{\today}
\maketitle


\begin{abstract}
    Abstract
\end{abstract}

\tableofcontents

\section{}
\begin{definition}
    An \emph{algebraic space} over a base scheme $S$ is a functor $X \colon (\text{Sch}/S)^{op} \to \text{Sets}$ satisfying
    \begin{enumerate}
        \item $X$ is a sheaf in the fppf topology,
        \item the diagonal morphism $\Delta \colon X \to X \times_S X$ is representable by schemes,
        \item there exists a surjective étale morphism $U \to X$ from a scheme $U$.
    \end{enumerate}
\end{definition}


By definiton, if $x\in X$ is a point of an algebraic space $X$, then $x$ admits an étale neighborhood $U \to X$ where $U$ is a scheme.


\begin{example}[An \'{e}tale map need not induce an isomorphism on ordinary local rings]
    Let
    \[
        X = \Spec \mathbb{C}[t,t^{-1}] \qquad
        Y = \Spec \mathbb{C}[u,u^{-1}],
    \]
    and let $f\colon Y \to X$ be the finite morphism given by $t = u^2$.
    This morphism is \'{e}tale: indeed, locally it is given by adjoining a root of
    $g(T)=T^2-t$, and on $Y$ we have $g'(u)=2u\in \mathbb{C}[u,u^{-1}]^{\times}$.

    Consider the generic point $\eta\in X$, so $\mathcal{O}_{X,\eta}=\kappa(\eta)=\mathbb{C}(t)$.
    Let $\eta'\in Y$ be the generic point lying over $\eta$, so
    $\mathcal{O}_{Y,\eta'}=\kappa(\eta')=\mathbb{C}(u)=\mathbb{C}(t^{1/2})$.
    The induced map on local rings
    \[
        \mathcal{O}_{X,\eta}=\mathbb{C}(t) \longrightarrow \mathcal{O}_{Y,\eta'}=\mathbb{C}(t^{1/2})
    \]
    coming from $t\mapsto u^2$ is not an isomorphism.
\end{example}

In this example, we are making use of the following general criterion for \'{e}taleness:
\begin{definition}[Standard \'{e}tale]\label{def:standard-etale}
    Let $R$ be a ring and let $g,f\in R[x]$. Assume that $f$ is monic and that the
    formal derivative $f'$ maps to a unit in the localization
    \[
        R[x]_g/(f).
    \]
    In this case the ring map
    \[
        R \longrightarrow R[x]_g/(f)
    \]
    is said to be \emph{standard \'{e}tale}.
\end{definition}

\begin{lemma}\label{lem:standard-etale-basic}
    Let $R \to R[x]_g/(f)$ be standard \'{e}tale.
    \begin{enumerate}
        \item[(1)] The ring map $R \to R[x]_g/(f)$ is \'{e}tale.
        \item[(2)] For any ring map $R\to R'$, the base change
              \[
                  R' \longrightarrow R'[x]_{g'}/(f')
                  \qquad\text{with } R'[x]_{g'}/(f') \cong R'\otimes_R \bigl(R[x]_g/(f)\bigr)
              \]
              is standard \'{e}tale (where $f',g'$ denote the images of $f,g$ in $R'[x]$).
        \item[(3)] Any principal localization of $R[x]_g/(f)$ is standard \'{e}tale over $R$.
        \item[(4)] A composition of standard \'{e}tale maps need not be standard \'{e}tale in general.
    \end{enumerate}
\end{lemma}

The example shows that the ordinary local ring $\mathcal{O}_{X,x}$ is not preserved
under passing to an \'{e}tale neighborhood. The correct object that is invariant under passing to an \'{e}tale neighborhood is not theordinary local ring $\mathcal{O}_{X,x}$ but the \emph{\'{e}tale stalk} of the structure sheaf, which is canonically identified with a strict henselization of the ordinary local ring.



\begin{definition}
    Given a site $(\mathcal{C}, J)$, a point of the associated topos $\mathbf{Sh}(\mathcal{C}, J)$ is (equivalently) a functor
    $$p^{-1} : \mathbf{Sh}(\mathcal{C}, J) \to \mathbf{Sets}$$
    that is left exact and has a right adjoint (a ``geometric morphism'' $\mathbf{Sets} \to \mathbf{Sh}(\mathcal{C}, J)$). The stalk of a sheaf $\mathcal{F}$ at the point $p$ is $p^{-1}(\mathcal{F})$.
\end{definition}

Points are not objects of $\mathcal{C}$ a priori; they are topos-theoretic. One can show a geometric point of $X$ is a morphism
\[
    \bar{x}\colon \Spec(\Omega) \to X
\]
with $\Omega$ separably closed. It lies over an underlying (ordinary) point $x \in |X|$, and choosing $\bar{x}$ is equivalent to choosing an embedding
\[
    \kappa(x) \hookrightarrow \Omega
\]
into a separably closed field.

The subtlety is:
\begin{itemize}
    \item The étale stalk $(\mathcal{O}_X)_{\bar{x}}$ is a stalk in the étale topology, and stalks are taken at points of the étale topos, i.e.\ geometric points, not just underlying Zariski points.
    \item The strict henselization $(\mathcal{O}_{X,x})^{\mathrm{sh}}$ is not determined by $x$ alone: it depends on a choice of a separable closure of the residue field $\kappa(x)$ inside some separably closed field. That choice is precisely the extra datum carried by $\bar{x}$.
\end{itemize}

So if you specify only $x$, then $(\mathcal{O}_{X,x})^{\mathrm{sh}}$ is well-defined only up to noncanonical isomorphism; once you specify $\bar{x}$, it becomes canonical. Let $X$ be an algebraic space.
\begin{itemize}
    \item An underlying point $x\in|X|$ does not determine a canonical local ring. Any attempt to define $\mathcal O_{X,x}$ requires choosing an \'{e}tale chart and a lift of $x$, and different choices need not agree.
    \item A geometric point $\bar x\to X$, however, does determine a canonical local ring:
    \[
        \mathcal O_{X,\bar x}:=(\mathcal O_X)_{\bar x},
    \]
    which is invariant under \'{e}tale pullback.
\end{itemize}

\begin{definition}[\'{E}tale local ring]
    Let $X$ be an algebraic space and let $\bar x\colon \Spec(\Omega)\to X$ be a geometric point lying over $x\in |X|$.
    The \emph{\'{e}tale local ring} of $X$ at $\bar x$ is the stalk
    \[
        \mathcal{O}_{X,\bar x} := (\mathcal{O}_X)_{\bar x} = \varinjlim_{(U,\bar u)\to (X,\bar x)} \Gamma(U,\mathcal{O}_U)
    \]
    of the structure sheaf on the \'{e}tale site $X_{\'{e}t}$.
\end{definition}
If $X$ is a scheme (or after choosing an \'{e}tale chart $U\to X$ with a lift of $\bar x$), the following lemma above identifies this canonically with the strict henselization
\[
    \mathcal{O}_{X,\bar x} \;\cong\; \bigl(\mathcal{O}_{X,x}\bigr)^{sh}.
\]

\begin{lemma}[Stacks Project, Lemma 59.33.1]
    Let $S$ be a scheme and let $\bar s$ be a geometric point of $S$ lying over $s\in S$.
    Let $\kappa=\kappa(s)$ and let $\kappa^{\mathrm{sep}}\subset \kappa(\bar s)$ denote the separable algebraic closure of $\kappa$ inside $\kappa(\bar s)$.
    Then there is a canonical identification
    \[
        \bigl(\mathcal{O}_{S,s}\bigr)^{sh} \;\cong\; (\mathcal{O}_S)_{\bar s},
    \]
    where the left-hand side is the strict henselization of the local ring at $s$, and the right-hand side is the stalk of the structure sheaf on the \'{e}tale site $S_{\'{e}t}$ at the geometric point $\bar s$.
\end{lemma}

Even though the etale local ring is defined as a colimit over all etale neighborhoods, it can be computed from any single etale neighborhood. This is a consequence of the following proposition about weakly etale maps, and the fact that etale maps are weakly etale.

\begin{definition}
    A map $f \colon A \to B$ of rings is called weakly étale if it is flat and the diagonal map $B \otimes_A B \to B$ is also flat.
\end{definition}

\begin{proposition}
    Let $A \to B$ be a weakly étale map of local rings. Then the induced map on strict henselizations $A^{sh} \to B^{sh}$ is an isomorphism, provided you choose the strict henselizations compatibly (pick a separable closure of $\kappa(x)$, and take the separable closure of $\kappa(y)$ inside it) where $x$ and $y$ are the closed points of $\Spec A$ and $\Spec B$ respectively.
\end{proposition}

\begin{definition}[Ideal sheaves on an algebraic space]
Let $X$ be an algebraic space.
An \emph{ideal sheaf} $I \subset \mathcal O_X$ is a subsheaf of rings of the
structure sheaf on the \'{e}tale site $X_{\'{e}t}$ such that, \'{e}tale-locally,
it is an ideal in the usual sense.

Concretely, choosing an \'{e}tale surjection $U \to X$ with $U$ a scheme,
an ideal sheaf $I \subset \mathcal O_X$ is equivalent to an ideal
$I_U \subset \mathcal O_U$ such that the pullbacks of $I_U$ to
$U \times_X U$ via the two projections agree.
\end{definition}

\begin{remark}[Ideal sheaves versus local rings on algebraic spaces]
There is an important asymmetry between ideal sheaves and local rings on an
algebraic space.

\smallskip
\noindent
\textbf{(1) Ideal sheaves descend \'{e}tale-locally.}
Ideal sheaves are sheaf-theoretic objects on the \'{e}tale site.
Because \'{e}tale morphisms are flat and the \'{e}tale topology admits effective
descent for quasi-coherent sheaves, an ideal of $\mathcal O_X$ may be defined
on any \'{e}tale chart $U \to X$ and uniquely descended to $X$.
In particular, closed subspaces and formal completions of algebraic spaces are
defined \'{e}tale-locally on schemes.

\smallskip
\noindent
\textbf{(2) Ordinary local rings do \emph{not} descend \'{e}tale-locally.}
The ordinary local ring $\mathcal O_{X,x}$ at a point $x \in |X|$ is defined as a
filtered colimit over Zariski neighborhoods of $x$.
Since \'{e}tale morphisms need not identify Zariski neighborhoods, passing to an
\'{e}tale chart generally changes the ordinary local ring.
Thus ordinary local rings are not invariant under \'{e}tale pullback.

\smallskip
\noindent
\textbf{(3) The correct local object is the \'{e}tale stalk.}
The local invariant preserved by \'{e}tale morphisms is instead the
\'{e}tale stalk $(\mathcal O_X)_{\bar x}$ at a geometric point
$\bar x \colon \Spec(\Omega)\to X$, which is canonically identified with the
strict henselization $(\mathcal O_{X,x})^{sh}$.
Unlike the Zariski local ring, this object is \'{e}tale-local by construction.

\smallskip
\noindent
In summary, ideal sheaves are global objects defined by \'{e}tale descent, while
ordinary local rings depend on the Zariski topology and are not compatible with
\^{e}tale localization. This distinction underlies the use of \'{e}tale topology
and formal geometry in the theory of algebraic spaces.
\end{remark}

\subsection*{The \'{e}tale topology over $\mathbf C$}

Even when $X$ is a scheme of finite type over $\mathbf C$, the \'{e}tale topology is
strictly finer than the Zariski topology and leads to a genuinely different notion
of locality.
\begin{example}
    Over $\mathbf{C}$, finite étale morphisms are exactly finite unbranched coverings in the analytic topology. Consider the map
    \[
        \mathbb{G}_m \to \mathbb{G}_m,\quad z\mapsto z^n
    \]
    is finite étale but not Zariski locally trivial.

    \[
        X=\Spec \mathbf C[t,t^{-1}],\qquad Y=\Spec \mathbf C[u,u^{-1}],\qquad t=u^n.
    \]
    Then
    \[
        \mathbf{C}[u,u^{-1}] \cong \mathbf{C}[t,t^{-1}][T]/(T^n-t).
    \]
    So $Y \to X$ is finite (monic polynomial).

    For étale: use the standard criterion for $A[T]/(f)$ with $f$ monic.
    Here $f(T) = T^n - t$, so $f'(T) = nT^{n-1}$. In $B = \mathbf{C}[u,u^{-1}]$,
    $f'(u) = n u^{n-1}$,
    which is a unit since $u$ is invertible on $\mathbb{G}_m$. Hence the map is étale everywhere.

    However, this map is not Zariski locally trivial: there is no Zariski open cover of $X$ over which the map splits as a disjoint union of copies of the base. Equivalently, on rings this says: for each i, the $A_i$-algebra
    $A_i[T]/(T^n-t)$
    \quad\text{with } $A_i=\Gamma(U_i,\mathcal{O}_X)$
    splits as a product $A_i\times\cdots\times A_i$. That happens iff $T^n-t$ factors into linear factors over $A_i$, i.e.\ iff $t$ admits an $n$-th root in $A_i$:
    $t = s_i^n$ \quad\text{in } $A_i$.

    But $X=\Spec \mathbf{C}[t,t^{-1}]$ is irreducible, so at least one $U_i$ contains the generic point. For that $U_i$, the inclusion of rings
    $\mathbf{C}(t)\hookrightarrow \operatorname{Frac}(A_i)$
    holds, and $t=s_i^n$ in $A_i$ would imply $t$ is an $n$-th power in the function field $\mathbf{C}(t)$. That is false: in $\mathbf{C}(t)$, the element $t$ has valuation $1$ at $t=0$, so it cannot be an $n$-th power (an $n$-th power would have valuation divisible by $n$). Contradiction.


\end{example}

Let $x\in X$ be a closed point and let $\bar x$ be a geometric point lying over $x$.
Since $\kappa(x)=\mathbf C$ is already separably closed, the \'{e}tale local ring is
given by
\[
    \mathcal{O}^{\acute et}_{X,\bar x}
    \;=\;
    (\mathcal{O}_{X,x})^{sh}
    \;=\;
    (\mathcal{O}_{X,x})^{h},
\]
the henselization of the Zariski local ring.
In general this ring is strictly larger than $\mathcal{O}_{X,x}$, although the two
have the same completion:
\[
    \widehat{\mathcal{O}_{X,x}}
    \;\cong\;
    \widehat{\mathcal{O}^{\acute et}_{X,\bar x}}.
\]

Thus completion only captures the formal neighborhood of $x$, while the \'{e}tale
local ring retains the minimal henselian enlargement required to solve \'{e}tale
equations (via Hensel's lemma).
In this sense, \'{e}tale locality is strictly stronger than Zariski locality even
over $\mathbf C$.

\section{Elliptic curves and contractions}

A smooth elliptic curve in $\mathbb{P}^2$ is a smooth plane cubic $E \subset \mathbb{P}^2$. As a divisor $E \sim 3H$, so
\[
    E^2 = (3H)^2 = 9,
\]
so it is not $(-1)$.

You can produce an elliptic curve of self-intersection $-1$ on a blowup of $\mathbb{P}^2$: blow up $n$ distinct points $p_1, \dots, p_n$ on a smooth cubic $E$, and let $\widetilde{E}$ be the strict transform. Then
\[
    \widetilde{E}^2 = E^2 - \sum_{i=1}^n (\operatorname{mult}_{p_i} E)^2 = 9 - n,
\]
so choosing $n = 10$ gives $\widetilde{E}^2 = -1$.


For a smooth projective surface $X$, the classical Castelnuovo contraction theorem says that a curve that can be contracted by a morphism $X \to Y$ to a smooth surface is necessarily a smooth rational curve with self-intersection $-1$.

More generally, if a proper birational morphism from a smooth surface contracts a single irreducible curve to a point and the target is a scheme, then the exceptional curve has to be $\mathbb{P}^1$ (in particular, genus $0$). An elliptic curve can't be the exceptional locus of such a contraction to a scheme.

Artin developed criteria ensuring that certain negative-definite curves on surfaces admit contractions in the category of algebraic spaces (even when no contraction exists as a scheme).

\begin{definition}Let $X$ be a surface (algebraic space or scheme). A proper curve $C \subset X$ is \emph{contractible in the sense of Artin} if there exists a proper surjective morphism
    \[f \colon X \to Y
    \]to an algebraic space $Y$ such that
    \begin{enumerate}
        \item $f$ is an isomorphism on $X \setminus C$,
        \item $f(C) = \{p\}$ is a single point,
        \item $Y$ is separated and of finite type over the base field,
        \item (usually imposed) $f_* \mathcal{O}_X = \mathcal{O}_Y$ (so you are not introducing extra connected components in the fibers; this is the ``Stein'' condition).
    \end{enumerate}
\end{definition}


If $C$ is an elliptic curve with $C^2 < 0$ and it is "contractible" in Artin's sense, then:
\begin{itemize}
    \item A contraction $f \colon X \to Y$ exists in the category of algebraic spaces.
    \item The target $Y$ is an algebraic space (of finite type, separated) which is a scheme away from the point $p$.
\end{itemize}

Let $S$ be a finite type scheme over a field $k$.
\begin{theorem}[Existence of contractions, Artin]\label{thm:existence-contractions}
    Let $Y' \subset X'$ be a closed subset of an algebraic space $X'$ of finite type
    over a base scheme $S$, and let $\mathfrak X'$ be the formal completion of $X'$
    along $Y'$.
    Given a formal modification
    \[
        f_1 \colon \mathfrak X' \longrightarrow \mathfrak X_1,
    \]
    there exists a modification
    \[
        f \colon X' \longrightarrow X,
        \qquad Y \subset X,
    \]
    together with an isomorphism $\varphi$ between the completion of $f$ along $Y$
    and the given formal modification $f_1$.
    The pair $(f,\varphi)$ is determined up to unique isomorphism.
\end{theorem}

\begin{theorem}[Artin, Formal Moduli II, Thm.\ 6.2]\label{thm:artin-formal-modification}
    Let $\mathfrak X'$ be a formal algebraic space, let
    $Y' = V(I')$
    be the closed formal subspace defined by a defining ideal $I' \subset \mathcal O_{\mathfrak X'}$,
    and let
    $f \colon Y' \longrightarrow Y$ be a proper morphism.

    Suppose the following conditions hold:

    \begin{enumerate}[(i)]
        \item For every coherent sheaf $\mathcal F$ on $\mathfrak X'$, we have
              \[
                  R^1 f_*\bigl(I'^n \mathcal F / I'^{n+1}\mathcal F\bigr) = 0
                  \qquad \text{for } n \gg 0.
              \]

        \item For every $n \ge 0$, the natural morphism of sheaves on $Y_0 := V(I')_{\mathrm{red}}$
              \[
                  f_*\bigl(\mathcal O_{\mathfrak X'} / I'^n\bigr)
                  \times_{f_* \mathcal O_{Y'}} \mathcal O_Y
                  \;\longrightarrow\;
                  \mathcal O_Y
              \]
              is surjective.
    \end{enumerate}

    Then there exists a formal modification
    $
        \widehat f \colon \mathfrak X' \longrightarrow \mathfrak X
    $
    and a defining ideal $I \subset \mathcal O_{\mathfrak X}$ such that
    $
        V(I) = Y
    $
    and such that $\widehat f$ induces the given morphism
    $f \colon Y' \to Y$.
\end{theorem}

\begin{theorem}[Knutson, Thm.\ III.6.2]\label{thm:knutson-contraction}
    Let $X'$ be a nonsingular algebraic space of finite type over a field $k$, and let
    $Y' \subset X'$ be a closed nonsingular subspace.
    Assume that $\dim X' = d$ and $\dim Y' = d-1$, and that $Y'$ is proper over $k$.
    Let
    \[
        \mathcal L := N^\vee_{Y'/X'}
    \]
    denote the conormal bundle of $Y'$ in $X'$, and assume that $\mathcal L$ is ample on $Y'$.

    Then there exists a proper birational morphism
    \[
        f \colon X' \longrightarrow X
    \]
    to an algebraic space $X$ such that:
    \begin{enumerate}
        \item $f$ is an isomorphism over $X \setminus \{p\}$,
        \item $f(Y') = \{p\}$ is a single point,
        \item $f^{-1}(p) = Y'$.
    \end{enumerate}
    In other words, $Y'$ can be contracted to a point in the category of algebraic spaces.
\end{theorem}

\begin{proof}
    The proof proceeds by reducing the problem to Artin's formal algebraization
    theorem (Theorem~\ref{thm:artin-formal-modification}).

    Let $\mathfrak X' := \widehat{X'}_{Y'}$ denote the formal completion of $X'$ along
    $Y'$, with defining ideal $I' \subset \mathcal O_{\mathfrak X'}$.
    Since $Y'$ is a smooth divisor in the smooth space $X'$, the graded pieces of the
    $I'$-adic filtration are given by
    \[
        I'^n / I'^{n+1} \;\cong\; \operatorname{Sym}^n(N^\vee_{Y'/X'}).
    \]

    Because $Y'$ is proper and the conormal bundle $N^\vee_{Y'/X'}$ is ample, Serre
    vanishing implies that for any coherent sheaf $\mathcal F$ on $X'$,
    \[
        H^1\!\left(Y', \operatorname{Sym}^n(N^\vee_{Y'/X'}) \otimes \mathcal F|_{Y'}\right) = 0
        \quad \text{for } n \gg 0.
    \]
    This verifies condition~(i) of Theorem~\ref{thm:artin-formal-modification}.

    As for condition~(ii), we have $Y=\{p\}$ a point, so $\mathcal O_Y=k$.
    Writing $A_n:=\Gamma\bigl(Y',\mathcal O_{\mathfrak X'}/I'^n\bigr)$ and $A_0:=\Gamma\bigl(Y',\mathcal O_{Y'}\bigr)$, Artin's map in~(ii) becomes the ring map
    \[
        A_n\times_{A_0} k \longrightarrow k,
    \]
    where $k\to A_0$ is the structure map. This map is the projection to the second factor, hence is automatically surjective. Thus, in this special case, Artin's condition~(ii) is automatic.
\end{proof}




\end{document}