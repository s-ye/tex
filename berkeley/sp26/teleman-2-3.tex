
\documentclass[12pt]{article}
\usepackage[english]{babel}
\usepackage[utf8x]{inputenc}
\usepackage[T1]{fontenc}
\usepackage{listings}
\usepackage{bookmark}
\usepackage{tikz}

\makeatletter
\def\input@path{{../../style/}}
\makeatother

\usepackage{../../style/quiver}
\makeatletter
\def\input@path{{../../style/}}
\makeatother

\usepackage{../../style/scribe}
\usepackage{fancyhdr}

\usepackage{parskip} % Automatically respects blank lines
\setlength{\parskip}{1em} % Adds more space between paragraphs
\setlength{\parindent}{0pt} % Removes paragraph indentation
\DeclareMathOperator{\Eul}{Eul}
\DeclareMathOperator{\Ind}{Ind}
\DeclareMathOperator{\pt}{pt}
\DeclareMathOperator{\sm}{sm}
\DeclareMathOperator{\Sing}{Sing}

\begin{document}


\lhead{Songyu Ye}
\rhead{\today}
\cfoot{\thepage}

\title{Title}

\author{Songyu Ye}
\date{\today}
\maketitle


\begin{abstract}
Abstract
\end{abstract}

\tableofcontents

\section{Teleman Woodward}
Let $C$ be a smooth projective curve and $\mathfrak{M}=\mathrm{Bun}_G(C)$. Fix an ``admissible class''
\[
\mathcal{E}=\mathcal{L}\otimes(\text{Atiyah--Bott generators})
\]
where $\mathcal{L}$ is a determinant line bundle of level $h=h_{\mathcal{L}}$, and $c$ is the canonical level coming from $\mathcal{K}=\det(E_C^*\mathfrak{g})$. Admissible means $h+c$ is positive definite.

\textbf{Goal:} show the index $\mathrm{Ind}(\mathfrak{M},\mathcal{E})=\chi(\mathfrak{M},\mathcal{E})$ is well-defined (finite) even though $\mathfrak{M}$ is not finite type.


\subsection{Shatz stratification and finite-type exhaustions}

There is the Harder--Narasimhan (Shatz) stratification
\[
\mathfrak{M}=\bigsqcup_{\xi}\mathfrak{M}_\xi
\]
indexed by dominant rational coweights $\xi$. The partial order on $\xi$ gives open substacks
\[
\mathfrak{M}_{\le \xi}:=\bigcup_{\mu\le \xi}\mathfrak{M}_\mu,
\]
and a key geometric input is: each $\mathfrak{M}_{\le \xi}$ is of finite type (and can be presented as a quotient of a quasi-projective variety by a reductive group \red{Why is this important? Do I need a stratification with this property?}). This gives an exhaustion of $\mathfrak{M}$ by finite-type opens.



\subsection{Filter $R\Gamma(\mathfrak{M},\mathcal{E})$ by local cohomology along the strata}

From the increasing opens $\mathfrak{M}_{\le \xi}$, you get a filtration of $R\Gamma(\mathfrak{M},\mathcal{E})$ whose graded pieces are local cohomology complexes supported on the successive strata:
\[
\gr_\xi \simeq R\Gamma_{\mathfrak{M}_\xi}(\mathfrak{M}_{\le \xi},\mathcal{E}_{\le \xi}).
\]

So finiteness of the global index reduces to:
\begin{enumerate}
\item each local term has finite Euler characteristic, and
\item all but finitely many $\xi$ have zero contribution.
\end{enumerate}



\subsection{Purity/local duality converts local cohomology into cohomology on the stratum}

Finite open unions of Shatz strata
\[
\mf M_{\le \xi} = \bigcup_{\mu\le \xi} \mf M_\mu
\]
can be presented as quotient stacks of smooth quasi--projective varieties by
reductive groups.  
Recall that local cohomology $R\Gamma_Z(X, -)$ is defined using the closed subset $Z \subset X$:
\[
R\Gamma_Z(X, \mathcal{E})
:= R\!\operatorname{Hom}_X(R\underline{\Gamma}_Z\mathcal{O}_X, \mathcal{E}).
\]

\begin{lemma}
    Let $i:Z\hookrightarrow X$ be a closed immersion of schemes. By construction there is an adjunction between $Ri_*$ and $i^!$ and so unwinding we find that local cohomology along $Z$ may be computed as
\[
R\Gamma_Z(X,\mathcal{E}) \simeq R\Gamma\bigl(Z,\, i^!(\mathcal{E})\bigr),
\]
\end{lemma}

\begin{lemma}[https://stacks.math.columbia.edu/tag/0AU3]
    Let $i:Z\hookrightarrow X$ be a regular embedding of codimension $d$ between smooth schemes, with normal bundle $N=N_{Z/X}$. One has the formula
\[
i^!(\mathcal{E}) \simeq Li^*(\mathcal{E})\otimes \omega_{Z/X}[d], \qquad \omega_{Z/X}\cong \det(N)^{-1}.
\]
\end{lemma}
Applying these two lemmas together to the map of stacks $i_\xi:\mf M_\xi\hookrightarrow \mf M_{\le \xi}$ (which is a regular embedding of codimension $d_\xi$) gives the following formula for local cohomology along the stratum $\mf M_\xi$.
\[
H^\bullet_{\mf M_\xi}(\mf M_{\le \xi},\mathcal E_{\le \xi})
= H^{\bullet}(\mf M_\xi, \mathcal R_\xi \mathcal E),
\]
where $d_\xi$ is the codimension of $\mf M_\xi$ and $\mathcal R_\xi \mathcal E$
denotes the sheaf of "$\mathcal E$--valued residues along $\mf M_\xi$." In particular \[\mathcal R_\xi \mathcal E \;:=\; i_\xi^! (\mathcal E_{\le \xi})\] where \(i_\xi:\mf M_\xi\hookrightarrow \mf M_{\le \xi}\) is the inclusion and $i^!$ is the extraordinary pullback. \red{What is the mechanism that lets us do this for stacks?}


Moreover, the residue object $\mathcal R_\xi\mathcal E$ may be expressed formally as
\[
\mathcal R_\xi\mathcal E = 
\mathcal E|_{\mf M_\xi}\otimes \Eul(\nu_\xi)^{-1}[d_\xi]
\]
where $\nu_\xi$ is the virtual normal complex 
\[
\nu_\xi=R\pi_*\mathcal E^*(\mf g/\mf g_\xi)[1]
\]


\subsection{Identify the virtual normal complex $\nu_\xi$}
\red{The virtual normal complex is identified via deformation theory.}
Each stratum $\mathfrak{M}_\xi$ maps to a finite-type ``semistable Levi core''
\[
q_\xi:\mathfrak{M}_\xi \longrightarrow \mathfrak{M}^{\mathrm{ss}}_{G_\xi,\xi},
\]
and the transverse deformation theory is governed by the perfect complex
\[
\nu_\xi = R\pi_*\mathcal{E}^*(\mathfrak{g}/\mathfrak{g}_\xi)[1]
\quad \text{on } \mathfrak{M}^{\mathrm{ss}}_{G_\xi,\xi}.
\]
We might just have to think really hard about deformation theory to work this out, once I identify a line bundle on the moduli stack that gives me a notion of stability.


\section{Solis compactification}

\begin{definition}[$\mathcal P$--parahoric $G$--bundles at a point]
\label{def:parahoric-torsor}
Fix a smooth curve $C$ over $\C$ and a point $p\in C$ with a choice of formal
parameter $z$ at $p$, so that the completed local ring is $\widehat{\mathcal O}_{C,p}\cong\C[[z]]$
and the punctured disc is $D^\times=\Spec\C((z))$.
Let
\[
\mathcal P\ \subset\ G((z))
\]
be a \emph{parahoric} subgroup (for instance a \emph{maximal} parahoric, i.e.\ one
corresponding to a vertex of the fundamental alcove).

Define a sheaf of groups $\mathcal G^{\mathcal P}$ on $C$ by gluing the standard
sheaf $\mathcal G^{\mathrm{std}} = \Hom_{\mathrm{Sch}}(-,G)$ away from $p$ with the local sheaf determined by
$\mathcal P$ at $p$ as follows:
\begin{itemize}
\item if $U\subset C$ is an open subset with $p\notin U$, set
\(\mathcal G^{\mathcal P}(U):=\mathcal G^{\mathrm{std}}(U)=\Hom_{\mathrm{Sch}}(U,G)\);
\item for the formal disc $D=\Spec\C[[z]]$ and punctured disc $D^\times=\Spec\C((z))$, set
\[
\mathcal G^{\mathcal P}(D):=\mathcal P,\qquad \mathcal G^{\mathcal P}(D^\times):=G((z)),
\]
with restriction map given by the inclusion $\mathcal P\hookrightarrow G((z))$;
\item on the overlap $(C\setminus\{p\})\cap D^\times\simeq D^\times$, we identify both
restrictions with $G((z))$ and glue.
\end{itemize}

A \emph{$\mathcal P$--parahoric $G$--bundle on $C$ (with parahoric structure of type $\mathcal P$ at $p$)} is a
$\mathcal G^{\mathcal P}$--torsor on $C$.
Equivalently, it is the data of:
\begin{enumerate}[(i)]
\item a principal $G$--bundle $E$ on $C\setminus\{p\}$;
\item a $\mathcal P$--torsor $E_D$ on the formal disc $D$ (i.e.\ a principal homogeneous space under the group
$\mathcal P=\mathcal G^{\mathcal P}(D)$);
\item an identification of the induced $G((z))$--torsors over $D^\times$.
\end{enumerate}

When $\mathcal P=L^+_P G=\{\gamma\in G[[z]]\mid \gamma(0)\in P\}$ for a parabolic $P\subset G$, this recovers the usual notion
of a quasi--parabolic $G$--bundle with a $P$--reduction at $p$.
When $\mathcal P$ is a maximal parahoric corresponding to a vertex $\eta_i$ of the fundamental alcove, we say the parahoric structure at $p$ is \emph{of type $\eta_i$}.
\end{definition}

\subsection{Refined stratification of $\mathcal X_{G,g,I}$}
Let $C/B$ be a prestable curve with dual graph $\Gamma$, and let
$(C'_B,P_B)$ be an object of $\mathcal X_{G,g,I}$; that is, $C'_B$ is a twisted
modification of $C_B$ and $P_B$ is an admissible $G$--bundle on $C'_B$.

For each vertex $v\in V(\Gamma)$ let $\xi_v$ denote the Harder--Narasimhan
type of the restriction of $P_B$ to the normalization component indexed by $v$.
For each node $e\in E(\Gamma)$ we have two additional pieces of boundary data:

\begin{enumerate}
\item a \emph{parahoric type}
      \[
        I_e\subset \{0,\dots,r\},
      \]
      specifying the parahoric subgroup
      $\mathcal P_{I_e}\subset G((z))$ which governs the local structure of
      the bundle at $e$;

\item a \emph{relative position label}
      \[
        \mathbf w_e \in
        W_{I_e}\backslash \widetilde W / W_{I_e},
      \]
      equivalently an orbit
      $O_e\subset \mathcal P_{I_e}\backslash G((z))/\mathcal P_{I_e}$
      describing the gluing of the two branches at $e$. We include this data to mod out by changes of trivialization on each side.
\end{enumerate}

Let $\tau_e$ denote the length of the modification chain over the node $e$.
Collect the data into
\[
\alpha=(\Gamma,\tau,\mathbf I,\mathbf w,\boldsymbol{\xi}),
\qquad
\mathbf I=(I_e)_{e\in E(\Gamma)},\;
\mathbf w=(\mathbf w_e)_{e\in E(\Gamma)},\;
\boldsymbol{\xi}=(\xi_v)_{v\in V(\Gamma)} .
\]

\begin{definition}
The \emph{refined stratum} of type $\alpha$ is the locally closed substack
\[
\mathcal X_\alpha \subset \mathcal X_{G,g,I}
\]
consisting of objects $(C'_B,P_B)$ such that:
\begin{enumerate}[(i)]
\item the coarse curve $C_B$ has dual graph $\Gamma$ and the modification
      lengths at the nodes are $\tau_e$;

\item for every vertex $v$, the restriction of $P_B$ to the corresponding
      normalization component has Harder--Narasimhan type $\xi_v$;

\item at each node $e$, the parahoric structure of $P_B$ is of type $I_e$,
      and the gluing of the two branches lies in the orbit $O_e$
      corresponding to $\mathbf w_e$.
\end{enumerate}
\end{definition}

The collection $\{\mathcal X_\alpha\}_\alpha$ forms a stratification of
$\mathcal X_{G,g,I}$, and its closure relations are governed by:
\begin{itemize}
\item the usual specialization of dual graphs and modification lengths;
\item the dominance order on the HN types $\xi_v$;
\item the Bruhat order on the double cosets
      $W_{I_e}\backslash \widetilde W / W_{I_e}$.
\end{itemize}

\begin{remark}
For a smooth curve the Shatz stratification indexed by HN types $\xi$
is adequate, because a $G$--bundle has no additional local structure.
On a nodal curve, however, two bundles with identical Shatz types on the
normalization can differ essentially at the node.

Consider two objects: (1) a genuine $G$-bundle on the nodal curve with gluing element $g\in G \subset G((z))$, and (2) a limit object where the gluing is $z^\lambda \in G((z))$ with $\lambda>0$.

Both have the same normalization bundles, the same HN type $\xi_v=0$, and the same parahoric type $I_e=\{0\}$. However, (1) lies in the open stratum corresponding to actual $G$-bundles, while (2) lies in a boundary stratum of the wonderful compactification.
\end{remark}

\begin{lemma}
\label{lem:finite-type-open}
For any fixed combinatorial type (graph $\Gamma$ + expansion length bound + allowed parahoric types $I_e$ in a finite set) and any bound on $(\boldsymbol{\xi}_v, \mathbf{w}_e)$, the open substack $\mathcal{X}_{\le \alpha}$ is algebraic and of finite type over an étale chart of $\overline{\mathfrak{M}}_{g,I}$.
\end{lemma}

\begin{enumerate}
    \item \red{Do I need to find a presentation of this stratum as a quotient of a quasi-projective variety by a reductive group?} The answer is yes this is important because once you show that only finitely many strata contribute, this is how each contribution has finitely many weight spaces.

\item \red{Do you expect the formula \[
i^!(\mathcal{E}) \simeq Li^*(\mathcal{E})\otimes \omega_{Z/X}[d], \qquad \omega_{Z/X}\cong \det(N)^{-1}.
\] to hold for me in this situation? What do I need to know about the embedding $i:\mathcal{X}_\alpha \hookrightarrow \mathcal{X}_{\le \alpha}$ to be able to apply this?} Shouldn't be a problem because deformation theory is controlled by $H^2$ of the stack which vanishes because we are on a curve.

\item \red{How do I identify the virtual normal complex $\nu_\alpha$?} Atiyah Bott.
\end{enumerate}

Apparently there should be a completely canonical way of writing down a stratification, once you choose a line bundle on the moduli stack $\cX_{G,g,I}$. This line bundle should give you a notion of stability, and the strata should be indexed by the instability types. \red{How do I write down this line bundle? How is it done in the case of $G$-bundles on a smooth curve?}

\end{document}