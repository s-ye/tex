\documentclass[12pt]{article}
\usepackage[english]{babel}
\usepackage[utf8x]{inputenc}
\usepackage[T1]{fontenc}
\usepackage{listings}
\usepackage{bookmark}
\usepackage{tikz}
\usepackage{/Users/songye03/Desktop/math_tex/style/quiver}
\usepackage{/Users/songye03/Desktop/math_tex/style/scribe}
\usepackage{fancyhdr}

\usepackage{parskip} % Automatically respects blank lines
\setlength{\parskip}{1em} % Adds more space between paragraphs
\setlength{\parindent}{0pt} % Removes paragraph indentation

\begin{document}


\lhead{Songyu Ye}
\rhead{\today}
\cfoot{\thepage}

\title{Loop groups}

\author{Songyu Ye}
\date{\today}
\maketitle


\begin{abstract}
    These are reading notes for the book "Loop Groups" by Pressley and Segal.
\end{abstract}

\tableofcontents

\section{Introduction}
\begin{definition}[Infinite dimensional Lie groups]
    An \textbf{infinite dimensional Lie group} is a group $\Gamma$ which is at the
    same time an infinite dimensional smooth manifold, and is such that
    the composition law $\Gamma \times \Gamma \to \Gamma$ and the
    operation of inversion $\Gamma \to \Gamma$ are given by smooth maps.
    The tangent space to $\Gamma$ at the identity element is its Lie
    algebra, the bracket being defined by identifying tangent vectors at
    the identity element with left-invariant vector fields on $\Gamma$.
    If for each element $\xi$ of the Lie algebra there is a unique
    one-parameter subgroup
    \[
        \gamma_\xi : \mathbb{R} \to \Gamma
    \]
    such that $\gamma_\xi'(0) = \xi$, then the exponential map is defined.
    This is the case in all known examples.
\end{definition}

\begin{example}
    The simplest example of an infinite dimensional Lie group is the group $\mathrm{Map}_{\mathrm{cts}}(X; G)$
    of all continuous maps from a compact space $X$ to a finite dimensional
    Lie group $G$. (The group law, of course, is pointwise composition in $G$.)
    The natural topology on $\mathrm{Map}_{\mathrm{cts}}(X; G)$ is the topology
    of uniform convergence. We see that it is a smooth manifold as follows.

    If $U$ is an open neighbourhood of the identity element in $G$ which is
    homeomorphic by the exponential map to an open set $\tilde U$ of the Lie
    algebra $\mathfrak{g}$ of $G$, then
    \[
        \mathcal{U} = \mathrm{Map}_{\mathrm{cts}}(X; U)
    \]
    is an open neighbourhood of the identity in $\mathrm{Map}_{\mathrm{cts}}(X; G)$
    which is homeomorphic to the open set
    \[
        \tilde{\mathcal{U}} = \mathrm{Map}_{\mathrm{cts}}(X; \tilde U)
    \]
    of the Banach space $\mathrm{Map}_{\mathrm{cts}}(X; \mathfrak{g})$. If $f$
    is any element of $\mathrm{Map}_{\mathrm{cts}}(X; G)$, then
    \[
        \mathcal{U}_f = \mathcal{U}\cdot f
    \]
    is a neighbourhood of $f$ which is also homeomorphic to $\tilde{\mathcal{U}}$.
    The sets $\mathcal{U}_f$ provide an atlas which makes
    $\mathrm{Map}_{\mathrm{cts}}(X; G)$ into a smooth manifold, and in fact into a Lie group: there is no difficulty at all in checking that the transition functions are smooth, or that multiplication and inversion are smooth maps.
\end{example}

\begin{definition}[Loop groups]
    Suppose now that $X$ is a finite dimensional compact smooth manifold,
    and let $\mathrm{Map}(X; G)$ denote the group of \textbf{smooth} maps $X \to G$. The case we are primarily interested in is when $X$ is the circle $S^1$;  then $\mathrm{Map}(X; G)$ is the \textbf{loop group} of $G$, which is denoted by $LG$. We shall think of the circle as consisting interchangeably of real numbers $\theta$ modulo $2\pi$ or of complex numbers $z = e^{i\theta}$ of  modulus one.
\end{definition}


Fix once and for all $G$ a compact connected Lie group. A fundamental property of the loop group $LG$ is the existence of interesting central extensions
\[
    \mathbb{T} \;\to\; \widetilde{LG} \;\to\; LG
\]
of $LG$ by the circle $\mathbb{T}$. (In other words, $\widetilde{LG}$ is a group
containing $\mathbb{T}$ in its centre and such that the quotient group
$\widetilde{LG}/\mathbb{T}$ is $LG$.)

The $\widetilde{LG}$ are
analogous to the finite-sheeted covering groups of a finite dimensional
Lie group, in that any projective unitary representation of $LG$ comes
from a genuine representation of some $\widetilde{LG}$. We recall that a
projective unitary representation of a group $L$ on a Hilbert space $H$
is the assignment to each $\lambda \in L$ of a unitary operator
$U_\lambda : H \to H$ so that
\[
    U_\lambda U_{\lambda'} = c(\lambda, \lambda') U_{\lambda\lambda'}
\]
holds for all $\lambda, \lambda' \in L$, where $c(\lambda,\lambda')$ is a
complex number of modulus $1$. $c : L \times L \to \mathbb{T}$ is called the \emph{projective multiplier} or \emph{cocycle} of the representation.

As topological spaces the $\widetilde{LG}$ are fibre bundles over $LG$ with the circle as fibre. Except for the product extension $LG \times \mathbb{T}$ they are non-trivial fibre bundles: that is to say
$\widetilde{LG}$ is not homeomorphic to the cartesian product
$LG \times \mathbb{T}$, and there is no continuous cross-section
$LG \to \widetilde{LG}$. In fact the group extension $\widetilde{LG}$ is
completely determined by its topological type as a fibre bundle, and
every circle bundle on $LG$ can be made into a group extension. It is
interesting that the behaviour of $\mathrm{Map}(X;G)$ when
$\dim(X) > 1$ is completely different. There are often non-trivial circle
bundles on $\mathrm{Map}(X;G)$, but if $X$ is simply connected only the flat ones can be made into groups.

When $G$ is a simple and simply connected group, there is a universal central extension among the $\widetilde{LG}$, i.e.\ one of
which all the others are quotient groups. This is analogous to the
universal covering group of a finite dimensional group. Any central
extension $E$ of $LG$ by any abelian group $A$ arises from the universal
extension $\widetilde{LG}$ by a homomorphism
$\mathbb{T} \to A$.

$\theta : \mathbb{T} \to A$, in the sense that
\[
    E = \widetilde{LG} \times_{\mathbb{T}} A.
\]
(The last notation denotes the quotient group of $\widetilde{LG} \times A$ by the subgroup consisting of all elements
\[
    \{(z, -\theta(z)) : z \in \mathbb{T}\}.
\])
When $G$ is simply connected but not simple there is still a universal central extension, but, as we shall see, it is an extension of $LG$ by the homology group $H_3(G;\mathbb{T})$, a torus whose dimension is the number of simple factors in $G$.

It is worth noticing that the central extensions of $LG$ are closely related to its natural affine action on the space of \emph{connections} in the trivial principal $G$-bundle on the circle. (See (4.3.3).)

\subsection{The Lie algebra extensions}

On the level of Lie algebras the extensions can be defined and classified
very simply: \red{they correspond precisely to invariant symmetric bilinear
    forms on $\mathfrak{g}$.} As a vector space
\[
    \widetilde{L\mathfrak{g}} = L\mathfrak{g} \oplus \mathbb{R},
\]
and the bracket is given by
\begin{equation} \label{eq:4.2.1}
    [(\xi,\lambda), (\eta,\mu)] = ([\xi,\eta], \, \omega(\xi,\eta))
\end{equation}
for $\xi,\eta \in L\mathfrak{g}$ and $\lambda,\mu \in \mathbb{R}$, where
$\omega : L\mathfrak{g} \times L\mathfrak{g} \to \mathbb{R}$ is the bilinear map
\begin{equation} \label{eq:4.2.2}
    \omega(\xi,\eta) = \frac{1}{2\pi} \int_0^{2\pi} \langle \xi(\theta), \eta'(\theta)\rangle \, d\theta
\end{equation}
and $\langle \ , \ \rangle$ is a symmetric invariant form on the Lie algebra
$\mathfrak{g}$. Recall that if $\mathfrak{g}$ is semisimple then every invariant bilinear form on $\mathfrak{g}$ is symmetric.

\begin{remark}
    Notice that the bracket \eqref{eq:4.2.1} does not depend on the value of $\lambda$ or $\mu$. In other words, the central $\mathbb{R}$ commutes with everything in $\widetilde{L\mathfrak{g}}$.
\end{remark}

For the formula \eqref{eq:4.2.1} to define a Lie algebra, $\omega$ must be skew—
which is clear by integrating by parts in \eqref{eq:4.2.2}—and must satisfy the
'cocycle condition'
\begin{equation} \label{eq:4.2.3}
    \omega([\xi,\eta],\zeta) + \omega([\eta,\zeta],\xi) + \omega([\zeta,\xi],\eta) = 0.
\end{equation}
This condition follows from the Jacobi identity in the Lie algebra
$L\mathfrak{g}$ and the fact that the inner product on $\mathfrak{g}$ is invariant:
\[
    \langle [\xi,\eta],\zeta \rangle = \langle \xi,[\eta,\zeta]\rangle.
\]

There are essentially no other cocycles on $L\mathfrak{g}$ than the $\omega$ given
by \eqref{eq:4.2.2}. To make this precise, notice that $\omega$ is invariant under
conjugation by constant loops, i.e.\ $\omega(\xi,\eta) = \omega(g\xi, g\eta)$
for $g \in G$, where $g\xi, g\eta$ are the adjoint action of $g$ on $\xi,\eta$.

\begin{remark}
    We elaborate a little on the invariance of $\omega$ under the adjoint action of $G$. Recall that for a Lie algebra $\mathfrak{a}$ with trivial coefficients, a $2$-cocycle is a bilinear form $\omega : \mathfrak{a}\times\mathfrak{a}\to \mathbb{R}$ that is skew-symmetric and satisfies the cocycle condition
    \[
        \delta\omega(\xi,\eta,\zeta) =
        \omega([\xi,\eta],\zeta) + \omega([\eta,\zeta],\xi) + \omega([\zeta,\xi],\eta) = 0.
    \]
    On the loop algebra $L\mathfrak{g}$, the group $G$ (constant loops) acts by conjugation:
    \[
        (g\cdot \xi)(\theta) = \mathrm{Ad}_g \xi(\theta).
    \]
    If we push forward a cocycle $\omega$ by $g$, we get a new cocycle
    \[
        (g\cdot \omega)(\xi,\eta) = \omega(g^{-1}\cdot \xi, \, g^{-1}\cdot \eta).
    \]

    To see that this transformation preserves the cohomology class, we can pass to the infinitesimal adjoint action. In particular, for $\zeta$ in the Lie algebra $\mathfrak{g}$, it is enough to show that
    \[
        [\omega] = [\omega + (\zeta\cdot\omega)] \quad\text{in } H^2.
    \]
    where the infinitesimal action is given by
    \[
        (\zeta\cdot \omega)(\xi,\eta)
        \;:=\; \left.\frac{d}{dt}\right|_{t=0}\big( \exp(t\zeta)\cdot \omega \big)(\xi,\eta).
    \]
    Use $\operatorname{Ad}_{\exp(-t\zeta)} = \exp(-t\,\operatorname{ad}\zeta)
        = \mathrm{id}-t\,\operatorname{ad}\zeta + o(t)$. Then
    \begin{align*}
        (\exp(t\zeta)\cdot \omega)(\xi,\eta)
         & = \omega\Big((\mathrm{id}-t\,\operatorname{ad}\zeta)\xi,\ (\mathrm{id}-t\,\operatorname{ad}_\zeta)\eta\Big) + o(t) \\
         & = \omega(\xi,\eta)\;-\; t\,\omega([\zeta,\xi],\eta)\;-\; t\,\omega(\xi,[\zeta,\eta])\;+\;o(t).
    \end{align*}
    Differentiating at $t=0$ gives
    \[
        (\zeta \cdot \omega)(\xi,\eta)
        = - \omega([\zeta,\xi],\eta) - \omega(\xi,[\zeta,\eta])
    \]

    Define the $1$-cochain $\phi_\zeta$ by
    \[
        \phi_\zeta(\xi) := \omega(\zeta, \xi).
    \]
    With trivial coefficients, the Chevalley–Eilenberg differential on a $1$-cochain is
    \[
        (\delta\phi_\zeta)(\xi, \eta) = -\,\phi_\zeta([\xi, \eta]) = -\,\omega(\zeta, [\xi, \eta]).
    \]
    Now compare $(\zeta\cdot\omega)$ with $\delta\phi_\zeta$:
    \begin{align*}
        (\zeta\cdot\omega)(\xi, \eta) - (\delta\phi_\zeta)(\xi, \eta)
         & = -\omega([\zeta, \xi], \eta) - \omega(\xi, [\zeta, \eta]) + \omega(\zeta, [\xi, \eta]).
    \end{align*}
    Use the $2$-cocycle identity (cyclic sum zero):
    \[
        \omega([\zeta, \xi], \eta) + \omega([\xi, \eta], \zeta) + \omega([\eta, \zeta], \xi) = 0.
    \]
    Rewrite the last two terms:
    \[
        \omega(\zeta, [\xi, \eta]) = -\,\omega([\xi, \eta], \zeta), \qquad
        \omega(\xi, [\zeta, \eta]) = -\,\omega([\zeta, \eta], \xi).
    \]
    Plugging these into the difference gives exactly the negative of the cyclic sum above, hence zero:
    \[
        -\omega([\zeta, \xi], \eta) - \omega(\xi, [\zeta, \eta]) + \omega(\zeta, [\xi, \eta]) = 0.
    \]
    Therefore,
    \[
        (\zeta\cdot\omega) = \delta\phi_\zeta
    \]
    is a coboundary, and $[\omega] = [\omega + (\zeta\cdot\omega)]$ in $H^2(L\mathfrak{g};\mathbb{R})$.
\end{remark}

So the extension defined by $\alpha$ is also given by the invariant cocycle
\[
    \int_G g\cdot \alpha \, dg
\]
obtained by averaging $\alpha$ over the compact group $G$ because they are in the same cohomology class in $H^2(L\mathfrak{g};\mathbb{R})$. Therefore, every cocycle of $L\mathfrak{g}$ is equivalent in cohomology to a $G$-invariant cocycle. The cocycle identity \eqref{eq:4.2.3} expresses precisely that the cohomology class of the cocycle does not change under an infinitesimal conjugation.


\begin{proposition}[Invariant cocycles]\label{prop:invariant_cocycles}
    If $\mathfrak{g}$ is semisimple then the only continuous $G$-invariant
    cocycles on the Lie algebra $L\mathfrak{g}$ are those given by
    \eqref{eq:4.2.2}.
\end{proposition}

\begin{proof}
    A cocycle $\alpha : L\mathfrak{g} \times L\mathfrak{g} \to \mathbb{R}$
    extends to a complex bilinear map
    $\alpha : L\mathfrak{g}_{\mathbb{C}} \times L\mathfrak{g}_{\mathbb{C}} \to \mathbb{C}$.
    An element $\xi \in L\mathfrak{g}_{\mathbb{C}}$ can be expanded in a Fourier series
    $\sum \xi_k z^k$, with $\xi_k \in \mathfrak{g}_{\mathbb{C}}$. By continuity $\alpha$ is
    completely determined by its values on elements of the form $\xi_p z^p$.
    Let us write
    \[
        \alpha_{p,q}(\xi,\eta) = \alpha(\xi z^p, \eta z^q), \quad \xi,\eta \in \mathfrak{g}_{\mathbb{C}}.
    \]
    Then $\alpha_{p,q}$ is a $G$-invariant bilinear map
    $\mathfrak{g}_{\mathbb{C}} \times \mathfrak{g}_{\mathbb{C}} \to \mathbb{C}$, which is necessarily symmetric, and
    $\alpha_{p,q} = -\alpha_{q,p}$. The cocycle identity \eqref{eq:4.2.3} translates into the statement
    \begin{equation} \label{eq:4.2.5}
        \alpha_{p+q,r} + \alpha_{q+r,p} + \alpha_{r+p,q} = 0
    \end{equation}
    for all $p,q,r$. Putting $q=r=0$ we find $\alpha_{p,0}=0$ for all $p$.

    $r = -p-q$ we find
    \[
        \alpha_{p+q,-p-q} = \alpha_{p,-p} + \alpha_{q,-q},
    \]
    whence
    \[
        \alpha_{p,-p} = p \, \alpha_{1,-1}.
    \]
    Putting $r = n-p-q$ in (4.2.5) we find
    \[
        \alpha_{n-p-q,p+q} = \alpha_{n-p,p} + \alpha_{n-q,q},
    \]
    whence
    \[
        \alpha_{n-k,k} = k \alpha_{n-1,1}.
    \]
    This implies that $\alpha_{p,q} = 0$ if $p+q \neq 0$, for
    \[
        n\alpha_{n-1,1} = \alpha_{0,n} = 0.
    \]
    Returning to $\xi = \sum \xi_p z^p$ and $\eta = \sum \eta_q z^q$, we have
    \[
        \alpha(\xi,\eta) = \sum p \, \alpha_{1,-1}(\xi_p, \eta_{-p})
        = \frac{i}{2\pi} \int_0^{2\pi} \alpha_{1,-1}(\xi(\theta), \eta'(\theta))\, d\theta,
    \]
    which is of the form (4.2.2).
\end{proof}

Proposition \ref{prop:invariant_cocycles} determines the universal central extension of $L\mathfrak{g}$.
We can reformulate it in the following way. For any finite dimensional
Lie algebra $\mathfrak{g}$ there is a universal invariant symmetric bilinear form
\begin{equation} \label{eq:4.2.6}
    \langle \ , \ \rangle_K : \mathfrak{g} \times \mathfrak{g} \to K
\end{equation}
from which every $\mathbb{R}$-valued form arises by a unique linear map $K \to \mathbb{R}$.

The cocycle $\omega_K$ given by
\begin{equation} \label{eq:4.2.7}
    \omega_K(\xi,\eta) = \frac{1}{2\pi} \int_0^{2\pi}
    \langle \xi(\theta), \eta'(\theta)\rangle_K \, d\theta
\end{equation}
defines an extension of $L\mathfrak{g}$ by $K$, which by Proposition (4.2.4) is the
universal central extension of $L\mathfrak{g}$ when $\mathfrak{g}$ is semisimple.
For semisimple groups $K$ can be identified with $H^3(\mathfrak{g};\mathbb{R})$,
because a bilinear form $\langle \ , \ \rangle$ on $\mathfrak{g}$ gives rise to an
invariant skew $3$-form
\[
    (\xi,\eta,\zeta) \mapsto \langle \xi,[\eta,\zeta]\rangle,
\]
and all elements of $H^3(\mathfrak{g};\mathbb{R})$ are so obtained. When $\mathfrak{g}$ is simple then $K = \mathbb{R}$.

\begin{remark}
    If $\mathfrak{g}$ is semisimple, then every invariant symmetric bilinear form on $\mathfrak{g}$ is a multiple of the Killing form. So in that case, the cocycle
    \[
        \omega(\xi,\eta) = \frac{1}{2\pi}\int_0^{2\pi} \langle \xi(\theta), \eta'(\theta)\rangle \, d\theta
    \]
    is unique up to scalar.

    But in general (if $\mathfrak{g}$ is not simple), the space of invariant symmetric bilinear forms on $\mathfrak{g}$ may have higher dimension. So instead of fixing one $\langle\ ,\ \rangle$, introduce the universal bilinear form:
    \[
        \langle\ ,\ \rangle_K : \mathfrak{g} \times \mathfrak{g} \;\to\; K,
    \]
    where $K$ is a vector space that “records all possible invariant bilinear forms at once.”

    Concretely: $K = (\text{space of invariant bilinear forms on }\mathfrak{g})^*$.
    Then for any actual $\mathbb{R}$-valued invariant form $\beta$, there is a unique linear functional $f:K\to \mathbb{R}$ such that
    \[
        \beta(x,y) = f(\langle x,y\rangle_K).
    \]


    Using this universal bilinear form, we define a universal cocycle:
    \[
        \omega_K(\xi,\eta) = \frac{1}{2\pi}\int_0^{2\pi} \langle \xi(\theta), \eta'(\theta)\rangle_K \, d\theta.
    \]
    This cocycle takes values in $K$, not just in $\mathbb{R}$. For any linear functional $f:K\to \mathbb{R}$, composing gives you back an $\mathbb{R}$-valued cocycle. So $\omega_K$ parametrizes all possible central extensions of $L\mathfrak{g}$ by $\mathbb{R}$.
\end{remark}

\subsection{Extensions of $\mathrm{Map}(X;\mathfrak{g})$}

Before leaving the subject of Lie algebra extensions, it is worth pointing
out that very little extra work is needed to determine all central extensions of
$\mathrm{Map}(X;\mathfrak{g})$ for any smooth manifold $X$. We shall indicate briefly
a proof of the following result, which is a very simple case of a general theory of
Loday and Quillen relating the cohomology of Lie algebras to Connes's cohomology. We shall content ourselves with the case of a simple Lie
algebra $\mathfrak{g}$. There is then an essentially unique inner product
$\langle \ , \ \rangle$.

\begin{proposition}[4.2.8]
    If $\mathfrak{g}$ is simple then the kernel of universal central extension
    of $\mathrm{Map}(X;\mathfrak{g})$ is the space
    $K = \Omega^1(X)/d\Omega^0(X)$ of $1$-forms on $X$ modulo exact $1$-forms.
    The extension is defined by the cocycle
    \begin{equation} \label{eq:4.2.9}
        (\xi,\eta) \mapsto \langle \xi, d\eta \rangle.
    \end{equation}
    Equivalently, the extensions of $\mathrm{Map}(X;\mathfrak{g})$ by $\mathbb{R}$
    correspond to the one-dimensional closed currents $C$ on $X$, the cocycle being
    given by integrating \eqref{eq:4.2.9} over $C$.
\end{proposition}

\medskip

Before proving this let us remark that from one point of view it is a
disappointing result, as it tells us that there are no 'interesting' extensions
of $\mathrm{Map}(X;\mathfrak{g})$ when $\dim(X) > 1$. More precisely,
if $f : S^1 \to X$ is any smooth loop in $X$ one can always obtain an extension
of $\mathrm{Map}(X;\mathfrak{g})$ by pulling back the universal extension of
$L\mathfrak{g}$ by $f$. Proposition (4.2.8) asserts that any extension is a weighted
linear combination of extensions of this form. The first 'interesting' cohomology
class of $\mathrm{Map}(X;\mathfrak{g})$, for a compact $(n-1)$-dimensional manifold $X$,
is in dimension $n$, and is defined by the cocycle
\[
    (\xi_1,\ldots,\xi_n) \mapsto P(\xi_1, d\xi_2,\ldots,d\xi_n),
\]

\noindent
\begin{proof}
    Let us write $\mathrm{Map}(X;\mathfrak{g})$ as
    $A \otimes \mathfrak{g}$, where $A$ is the ring of smooth functions on $X$.
    Any $G$-invariant real-valued bilinear form on $A \otimes \mathfrak{g}$ must be of the form
    \[
        (f \otimes \xi, \, g \otimes \eta) \mapsto \alpha(f \otimes g) \langle \xi,\eta \rangle,
    \]
    where $\alpha : A \otimes A \to \mathbb{R}$ is linear. Such an $\alpha$ can be identified
    with a distribution with compact support on $X \times X$. The cocycle condition translates
    into the statement that $\alpha$ vanishes on functions of the form
    \begin{equation} \label{eq:4.2.10}
        fg \otimes h + gh \otimes f + hf \otimes g,
    \end{equation}
    where $f,g,h$ are smooth functions on $X$. This means that $\alpha(f \otimes g) = 0$
    when $f$ and $g$ have disjoint support, for then $fg=0$ and one can find $h$ so that
    $fh=f$ and $gh=0$. Thus the distribution $\alpha$ has support along the diagonal.
    Proposition (4.2.8) is the assertion that $\alpha(f \otimes g)$ depends only on the
    $1$-form $fdg$. This in turn reduces to two facts:
    \begin{enumerate}[(i)]
        \item $\alpha(f \otimes 1)=0$ for all $f$; and
        \item $\alpha\vert_{I^2} = 0$, where $I$ is the ideal of functions in $A \otimes A$
              which vanish on the diagonal.
    \end{enumerate}
    Put $h=1$:
    \[
        \alpha(fg \otimes 1) + \alpha(g \otimes f) + \alpha(f \otimes g) = 0.
    \]
    By (skew), $\alpha(g \otimes f) = -\alpha(f \otimes g)$, so those two cancel and we get
    \[
        \alpha(fg \otimes 1) = 0 \quad \forall f,g.
    \]
    Since finite sums of products $fg$ span $A$, it follows that
    \[
        \alpha(f \otimes 1) = 0 \quad \forall f \in A.
    \]

    Let $I \subset A \otimes A$ be the ideal of functions vanishing on the diagonal $\Delta = \{(x,x)\}$. It is generated (as an ideal) by the differences $a \otimes 1 - 1 \otimes a$ ($a \in A$). Thus $I^2$ is generated by products
    \[
        (a \otimes 1 - 1 \otimes a)\,(b \otimes 1 - 1 \otimes b).
    \]

    It therefore suffices to check that $\alpha$ vanishes on each such generator. Expand:
    \[
        (a \otimes 1 - 1 \otimes a)(b \otimes 1 - 1 \otimes b)
        = ab \otimes 1 \;-\; a \otimes b \;-\; b \otimes a \;+\; 1 \otimes ab.
    \]
    Apply $\alpha$ and use (i) and (skew):
    \[
        \alpha(ab \otimes 1) = 0, \quad \alpha(1 \otimes ab) = 0, \quad
        \alpha(a \otimes b) + \alpha(b \otimes a) = 0.
    \]
    Hence
    \[
        \alpha\big((a \otimes 1 - 1 \otimes a)(b \otimes 1 - 1 \otimes b)\big) = 0.
    \]
    By linearity,
    \[
        \alpha|_{I^2} = 0
    \]

    Define the canonical linear map
    \[
        \theta: A \otimes A \longrightarrow \Omega^1(X),\qquad \theta(f \otimes g) = f\,dg.
    \]
    A quick check on the generators above shows $\theta(I^2) = 0$:
    \[
        \theta\big(ab \otimes 1 - a \otimes b - b \otimes a + 1 \otimes ab\big)
        = ab\,d1 - a\,db - b\,da + d(ab) = 0.
    \]
    So $\theta$ descends to a well-defined map $\bar\theta: I/I^2 \to \Omega^1(X)$, which is the standard isomorphism $I/I^2 \cong \Omega^1(X)$ (Kähler differentials).

    Since $\alpha$ kills $I^2$, there is a unique linear functional
    \[
        \Lambda: \Omega^1(X) \longrightarrow \mathbb{R}
        \quad\text{such that}\quad
        \alpha(f \otimes g) = \Lambda(f\,dg)
    \]

    Using $(†)$, compute skew-symmetry:
    \[
        0 = \alpha(f \otimes g) + \alpha(g \otimes f)
        = \Lambda(f\,dg) + \Lambda(g\,df)
        = \Lambda\big(f\,dg + g\,df\big)
        = \Lambda\big(d(fg)\big).
    \]
    Because $f,g$ were arbitrary, the linear span of $\{d(fg)\}$ is all of $d\Omega^0(X)$. Hence
    \[
        \Lambda|_{d\Omega^0(X)} = 0.
    \]
    $\Lambda$ factors through the quotient $\Omega^1(X)/d\Omega^0(X)$, so the only data that survives is the class $[\Lambda] \in (\Omega^1/d\Omega^0)^*$.
\end{proof}

\begin{remark}
    In the case $X = S^1$, we have $\Omega^1(S^1)/d\Omega^0(S^1) \cong H^1_{\mathrm{dR}}(S^1) \cong \mathbb{R}$ (generated by the period functional $[\alpha] \mapsto \int_{S^1}\alpha$).
    Taking $\Lambda(\omega) = \frac{1}{2\pi}\int_{S^1}\omega$ gives
    \[
        \alpha(f \otimes g) = \frac{1}{2\pi}\int_{S^1} f\,dg,
        \qquad
        c(\xi,\eta) = \frac{1}{2\pi}\int_{S^1}\!\langle \xi,\eta'\rangle\,d\theta,
    \]
    the standard Kac-Moody cocycle.
\end{remark}

\subsection{Extensions of $\mathrm{Vect}(S^1)$}

Another calculation that fits in very naturally at this point is that for the
Lie algebra $\mathrm{Vect}(S^1)$ of smooth vector fields on the circle, i.e.\ the Lie algebra of the group $\mathrm{Diff}(S^1)$. A complex-linear $2$-cocycle
\[
    \alpha : \mathrm{Vect}_{\mathbb{C}}(S^1) \times \mathrm{Vect}_{\mathbb{C}}(S^1) \to \mathbb{C},
\]
where $\mathrm{Vect}_{\mathbb{C}}(S^1) = \mathrm{Vect}(S^1)\otimes \mathbb{C}$, is determined by the numbers
\[
    \alpha_{p,q} = \alpha(L_p,L_q), \qquad L_n = e^{in\theta}\tfrac{d}{d\theta}.
\]
Recall the Witt algebra basis
\[
    L_n = i e^{in\theta} \tfrac{d}{d\theta}, \quad n \in \mathbb{Z},
\]
with brackets
\[
    [L_n, L_m] = i(m-n) L_{n+m}.
\] The bracket identity follows from the definition of the commutator of derivations:
\[
    [X,Y] = X(Y(h)) - Y(X(h)) \quad \text{for } h \in C^\infty(M)
\] and the general formula for brackets of vector fields in one variable:
\[[f(\theta)\tfrac{d}{d\theta}, g(\theta)\tfrac{d}{d\theta}] = \big(f(\theta)g'(\theta) - g(\theta)f'(\theta)\big)\tfrac{d}{d\theta}.\]

Now check the three vector fields:
\[
    L_{-1}, \quad L_0, \quad L_1.
\]
The brackets close as
\[
    [L_1, L_{-1}] = 2i L_0, \qquad [L_0, L_{\pm 1}] = \mp i L_{\pm 1}.
\]
which up to rescaling gives a copy of $\mathfrak{sl}_2(\mathbb{R})$.

The cocycle identity for $(L_0,L_p,L_q)$ shows that the cohomology class of
$\alpha$ is not changed by rotation, and so we can (by averaging) assume that
$\alpha$ is itself invariant. Then $\alpha_{p,q}=0$ unless $p+q=0$. If we write
$\alpha_{p,-p}=\alpha_p$, and notice that $\alpha_{-p}=-\alpha_p$, then the cocycle
identity gives
\[
    (p+2q)\alpha_p - (2p+q)\alpha_q = (p-q)\alpha_{p+q}.
\]
This determines all the $\alpha_p$ in terms of $\alpha_1$ and $\alpha_2$. The general
solution is $\alpha_p = \lambda p^3 + \mu p$. But $\alpha_p = p$ is a coboundary, so the
value of $\mu$ is unimportant. We have proved

\begin{proposition}[Virasoro cocycle]
    The most general central extension of $\mathrm{Vect}(S^1)$ by $\mathbb{R}$
    is described by the cocycle $\alpha$, where
    \[
        \alpha\!\left(e^{in\theta}\tfrac{d}{d\theta},\, e^{im\theta}\tfrac{d}{d\theta}\right)
        = \begin{cases}
            i\lambda n(n^2-1), & \text{if } n+m=0,      \\
            0,                 & \text{if } n+m \neq 0,
        \end{cases}
    \]
    for some $\lambda \in \mathbb{R}$.
\end{proposition}

The representing cocycle given here is characterized by the fact that it is invariant under rotation and vanishes on the subalgebra $\mathfrak{sl}_2(\mathbb{R})$ of
$\mathrm{Vect}(S^1)$.

\subsection{Adjoint and coadjoint actions of loop groups}
\begin{proposition}
    The adjoint action of $L\mf{g}$ on its central extension $\widetilde{L\mathfrak{g}}$ comes from an action of $LG$ given by \begin{align*}
        (\gamma)\cdot(\xi,\lambda) = (\operatorname{Ad}_\gamma \xi, \lambda - \langle \gamma^{-1}\gamma', \xi \rangle)
    \end{align*}
\end{proposition}

\begin{proof}
    We will differentiate the group action along the one-parameter subgroup $\gamma(t) = \exp(t\eta)$, where $\eta \in L\mathfrak{g}$. Differentiate at $t=0$:
    \begin{itemize}
        \item First coordinate:
              \[
                  \frac{d}{dt}\Big|_{t=0}\operatorname{Ad}_{\gamma(t)}\xi
                  = [\eta,\xi].
              \]
        \item Second coordinate: use $\gamma(t)^{-1}\gamma'(t)=t\,\eta'+O(t^2)$ (standard Maurer–Cartan expansion along the loop variable), hence
              \[
                  \frac{d}{dt}\Big|_{t=0}\!\left(-\big\langle\gamma(t)^{-1}\gamma'(t),\xi\big\rangle\right)
                  = -\frac{1}{2\pi}\int_0^{2\pi}\!\langle \eta'(\theta),\xi(\theta)\rangle_{\mathfrak g}\,d\theta
                  = \frac{1}{2\pi}\int_0^{2\pi}\!\langle \eta(\theta),\xi'(\theta)\rangle_{\mathfrak g}\,d\theta,
              \]
              where the last equality is by integration by parts (boundary term vanishes by periodicity).
    \end{itemize}

    Define the Kac-Moody $2$-cocycle
    \[
        \omega(\eta,\xi)\;:=\;\frac{1}{2\pi}\int_0^{2\pi}\!\langle \eta(\theta),\xi'(\theta)\rangle_{\mathfrak g}\,d\theta.
    \]
    Thus the derivative of the action is
    \[
        \frac{d}{dt}\Big|_{t=0}\big(\gamma(t)\cdot(\xi,\lambda)\big)
        = \big([\eta,\xi],\,\omega(\eta,\xi)\big).
    \]
    This is exactly the standard adjoint action of $L\mathfrak g$ on the central extension $\widetilde{L\mathfrak g}=L\mathfrak g\oplus\mathbb R K$:
    \[
        \operatorname{ad}_{(\eta,0)}(\xi,\lambda)=\big([\eta,\xi],\,\omega(\eta,\xi)\big),\qquad [K,\cdot]=0.
    \] as desired.
\end{proof}


\begin{proposition}[Loop group coadjoint action]
    The coadjoint action of $LG$ on $\widetilde{L\mathfrak g}^{\,*}\cong (L\mathfrak g)^{*}\oplus\mathbb R$ is given by
    \[
        \gamma\cdot(\phi,\lambda)=\big(\Ad_\gamma \phi+\lambda\,\gamma'\gamma^{-1},\ \lambda\big).
    \]
\end{proposition}
\begin{proof}
    Identify $\widetilde{L\mathfrak g}^{\,*}\cong (L\mathfrak g)^{*}\oplus\mathbb R$ with pairing (note that $(\phi,\lambda) \in (\tilde L \mathfrak g \oplus \mathbb R)^*$ and $(\xi,a) \in \tilde L \mathfrak g \oplus \mathbb R$)
    \[
        \langle (\phi,\lambda),(\xi,a)\rangle\;=\;\phi(\xi)+\lambda a.
    \]
    By definition of coadjoint action,
    \[
        \big\langle \gamma\cdot(\phi,\lambda),\,(\xi,a)\big\rangle
        =\big\langle (\phi,\lambda),\,\gamma^{-1}\cdot(\xi,a)\big\rangle.
    \]
    Insert the adjoint formula with $\gamma^{-1}$. Using $(\gamma^{-1})^{-1}(\gamma^{-1})'=\gamma(\gamma^{-1})'=-\,\gamma'\gamma^{-1}$,
    \[
        \gamma^{-1}\cdot(\xi,a)
        =\Big(\Ad_{\gamma^{-1}}\xi,\ a-\big\langle (\gamma^{-1})^{-1}(\gamma^{-1})',\,\xi\big\rangle\Big)
        =\Big(\Ad_{\gamma^{-1}}\xi,\ a+\langle \gamma'\gamma^{-1},\,\xi\rangle\Big).
    \]
    Hence
    \[
        \begin{aligned}
            \big\langle \gamma\cdot(\phi,\lambda),\,(\xi,a)\big\rangle
             & =\phi\big(\Ad_{\gamma^{-1}}\xi\big)+\lambda\Big(a+\langle \gamma'\gamma^{-1},\,\xi\rangle\Big)         \\
             & =(\phi\circ\Ad_{\gamma^{-1}})(\xi)\;+\;\lambda a\;+\;\lambda\,\langle \gamma'\gamma^{-1},\,\xi\rangle.
        \end{aligned}
    \]
    Since this holds for all $(\xi,a)$, we read off
    \[
        \gamma\cdot(\phi,\lambda)
        =\big(\phi\circ\Ad_{\gamma^{-1}}+\lambda\,\langle \gamma'\gamma^{-1},\,\cdot\,\rangle,\ \lambda\big).
    \]

    Using the invariant inner product to identify $(L\mathfrak g)^*\cong L\mathfrak g$, write $\phi(\,\cdot\,)=\langle \phi,\,\cdot\,\rangle$. Then
    \[
        \phi\circ\Ad_{\gamma^{-1}}=\langle \Ad_\gamma \phi,\,\cdot\,\rangle,\qquad
        \langle \gamma'\gamma^{-1},\,\cdot\,\rangle\ \leftrightarrow\ \gamma'\gamma^{-1},
    \]
    so the coadjoint action becomes
    \[
        \gamma\cdot(\phi,\lambda)=\big(\Ad_\gamma \phi+\lambda\,\gamma'\gamma^{-1},\ \lambda\big).
    \]
    as desired. \end{proof}

   \begin{remark}[Reminder about the infinitesimal adjoint and coadjoint actions]
Let $G$ be compact, connected, and simply connected, with Lie algebra $\mathfrak g$
and an $\Ad$–invariant inner product $\langle\,,\,\rangle_{\mathfrak g}$.
For loops we use the $L^2$–pairing
\[
\langle \xi,\eta\rangle_{L\mathfrak g}
:= \frac{1}{2\pi}\int_0^{2\pi}\langle \xi(\theta),\eta(\theta)\rangle_{\mathfrak g}\,d\theta,
\qquad \xi,\eta\in L\mathfrak g .
\]
Define the $2$–cocycle
\[
\omega(\eta,\xi)
:= \frac{1}{2\pi}\int_0^{2\pi}\!\!\langle \eta(\theta),\xi'(\theta)\rangle_{\mathfrak g}\,d\theta.
\]
The (Lie–algebra) central extension is
\[
\widetilde{L\mathfrak g} \;=\; L\mathfrak g \oplus \mathbb R K,
\qquad
[(\eta,aK),(\xi,bK)]
= \big([\eta,\xi],\;\omega(\eta,\xi)\,K\big).
\]
Here $K$ is central, $[K,\cdot]=0$.
We identify $\widetilde{L\mathfrak g}^{\,*}\cong (L\mathfrak g)^*\oplus\mathbb R$ and,
via the $L^2$–pairing, $(L\mathfrak g)^*\cong L\mathfrak g$.
We write the dual pairing as
\[
\big\langle (\phi,\lambda),(\xi,aK)\big\rangle
= \langle \phi,\xi\rangle_{L\mathfrak g} + \lambda a .
\]

\paragraph{Adjoint (Lie–algebra) action.}
By definition, $\ad_{(\eta,aK)}(\xi,bK)=[(\eta,aK),(\xi,bK)]$; since $K$ is central,
\[
\ad_{(\eta,0)}(\xi,bK) = \big([\eta,\xi],\,\omega(\eta,\xi)\,K\big), \qquad
\ad_{(0,aK)}(\xi,bK)=0 .
\]
Equivalently, the adjoint representation of $\widetilde{L\mathfrak g}$ on itself is
\[
\ad_{(\eta,aK)}
\begin{pmatrix}\xi\\ bK\end{pmatrix}
=
\begin{pmatrix}
[\eta,\xi]\\ \omega(\eta,\xi)\,K
\end{pmatrix}.
\]

\paragraph{Coadjoint (Lie–algebra) action.}
Recall the coadjoint action is defined by
\[
\big\langle \ad^*_{X}(\Phi),\,Y\big\rangle
= \big\langle \Phi,\,[Y,X]\big\rangle
\qquad (X,Y\in \widetilde{L\mathfrak g},\;\Phi\in \widetilde{L\mathfrak g}^{\,*}),
\]
which matches the sign convention that integrates to the group formula in Pressley--Segal.
Take $X=(\eta,0)$, $Y=(\xi,aK)$, $\Phi=(\phi,\lambda)$:
\begin{align*}
\big\langle \ad^*_{(\eta,0)}(\phi,\lambda),\,(\xi,aK)\big\rangle
&= \big\langle (\phi,\lambda),\,[(\xi,aK),(\eta,0)]\big\rangle\\
&= \big\langle (\phi,\lambda),\,\big([\xi,\eta],\,\omega(\xi,\eta)K\big)\big\rangle\\
&= \langle \phi,\,[\xi,\eta]\rangle_{L\mathfrak g} + \lambda\,\omega(\xi,\eta).
\end{align*}
Use $\Ad$–invariance of $\langle\,,\,\rangle_{\mathfrak g}$:
$\langle \phi,[\xi,\eta]\rangle=\langle [\phi,\eta],\xi\rangle$, and integrate by parts
(using periodicity) for the cocycle term
\[
\omega(\xi,\eta) = \frac{1}{2\pi}\int_0^{2\pi}\!\!\langle \xi,\eta'\rangle\,d\theta.
\]
Thus
\[
\big\langle \ad^*_{(\eta,0)}(\phi,\lambda),\,(\xi,aK)\big\rangle
= \left\langle [\eta,\phi]+\lambda\,\eta',\,\xi\right\rangle_{L\mathfrak g},
\]
which identifies
\[
\ad^*_{(\eta,0)}(\phi,\lambda)\;=\;\big([\eta,\phi]+\lambda\,\eta',\;0\big).
\]
Since $K$ is central, $\ad^*_{(0,aK)}=0$. Therefore, for general $(\eta,aK)$:
\[
\ad^*_{(\eta,aK)}(\phi,\lambda) \;=\; \big([\eta,\phi]+\lambda\,\eta',\;0\big).
\]

\paragraph{Consistency with the group formula.}
Let $\gamma(t)=\exp(t\eta)\in LG$. The group–level coadjoint action is
\[
\gamma\cdot(\phi,\lambda)
= \big(\Ad_\gamma\phi+\lambda\,\gamma'\gamma^{-1},\,\lambda\big).
\]
Differentiating at $t=0$ gives
\[
\frac{d}{dt}\Big|_{0}\big(\Ad_{\gamma(t)}\phi\big)=[\eta,\phi],
\qquad
\frac{d}{dt}\Big|_{0}\big(\gamma'(t)\gamma(t)^{-1}\big)=\eta',
\]
hence
\[
\frac{d}{dt}\Big|_{0}\big(\gamma(t)\cdot(\phi,\lambda)\big)
= \big([\eta,\phi]+\lambda\,\eta',\,0\big)=\ad^*_{(\eta,0)}(\phi,\lambda),
\]
as derived above.
\end{remark}



Let us assume that the inner product on $\mathfrak{g}$ is positive-definite.
Then $L\mathfrak{g}$ is identified with a dense subspace of $(L\mathfrak{g})^*$
which we shall call the ‘smooth part’ of the dual. We can describe the orbits of
the action of $LG$ on this in the following way.

For each smooth element $(\phi,\lambda) \in (\widetilde{L\mathfrak{g}})^*$
with $\lambda \neq 0$ we can find a unique smooth path $f : \mathbb{R} \to G$
by solving the differential equation
\begin{equation} \label{eq:4.3.4}
    f'f^{-1} = \lambda^{-1}\phi
\end{equation}
with the initial condition $f(0)=1$.

\begin{definition}[Parallel transport ODE]
    Define $f:\mathbb R\to G$ by the first-order ODE
    \begin{equation}\label{eq:ODE}
        f'(\theta)\,f(\theta)^{-1} \;=\; \lambda^{-1}\,\phi(\theta),
        \qquad f(0)=\mathbf{1}.
    \end{equation}
\end{definition}

\begin{lemma}[Existence and uniqueness]
    For any smooth $\phi$ and $\lambda\neq 0$, the initial value problem
    \eqref{eq:ODE} has a unique smooth solution on all of $\mathbb R$. Moreover,
    \begin{equation}\label{eq:POexp}
        f(\theta) \;=\; \mathcal P\exp\!\left(\lambda^{-1}
        \int_0^\theta \phi(s)\,ds\right),
    \end{equation}
    where $\mathcal P\exp$ denotes the path-ordered exponential.
\end{lemma}

Because $\phi$ is periodic in $\theta$ we have
\[
    f(\theta+2\pi) = f(\theta)\cdot M_\phi,
\]
where $M_\phi = f(2\pi)$. If $(\phi,\lambda)$ is transformed by $\gamma \in LG$
then $f$ is changed to $\tilde{f}$, where
\begin{equation} \label{eq:4.3.5}
    \tilde{f}(\theta) = \gamma(\theta) f(\theta) \gamma(0)^{-1}.
\end{equation}
Thus $M_\phi$ is changed to $\gamma(0)M_\phi \gamma(0)^{-1}$. In fact
\eqref{eq:4.3.4} defines a bijection between $L\mathfrak{g} \times \{\lambda\}$
and the space of maps $f$ such that $f(0)=1$ and
$f(\theta+2\pi)=f(\theta)\cdot M$ for some $M \in G$.

\begin{definition}[Monodromy / holonomy]
    Because $\phi$ is $2\pi$-periodic, there exists a unique $M_\phi\in G$
    (the \emph{monodromy}) such that
    \begin{equation}\label{eq:monodromy-eq}
        f(\theta+2\pi) \;=\; f(\theta)\,M_\phi \qquad (\theta\in\mathbb R),
    \end{equation}
    equivalently $M_\phi=f(2\pi)$. In terms of \eqref{eq:POexp},
    \begin{equation}\label{eq:monodromy-POexp}
        M_\phi \;=\; \mathcal P\exp\!\left(\lambda^{-1}
        \int_0^{2\pi}\phi(\theta)\,d\theta\right).
    \end{equation}
\end{definition}

\begin{proof}[Proof of \eqref{eq:monodromy-eq}]
    Let $g(\theta):=f(\theta+2\pi)$. Then
    \begin{align*}
        g' g^{-1} & = f'(\theta+2\pi) f(\theta+2\pi)^{-1} \\
                  & = \lambda^{-1}\phi(\theta+2\pi)       \\
                  & = \lambda^{-1}\phi(\theta),
    \end{align*}
    and $g(0)=f(2\pi)$. By uniqueness for \eqref{eq:ODE}, $g(\theta)=f(\theta)\,f(2\pi)$, giving
    $f(\theta+2\pi)=f(\theta)M_\phi$ with $M_\phi=f(2\pi)$.
\end{proof}

\begin{proposition}[Transformation under the $LG$–coadjoint action]
    Let $\gamma\in LG$. The $LG$–coadjoint action on
    $(\widetilde{L\mathfrak g})^*$ is
    \[
        \gamma\cdot(\phi,\lambda) \;=\; (\operatorname{Ad}_\gamma\phi + \lambda\,\gamma'\gamma^{-1},\,\lambda).
    \]
    If $f$ solves \eqref{eq:ODE} for $(\phi,\lambda)$, then
    \begin{equation}\label{eq:ftilde}
        \tilde f(\theta) \;:=\; \gamma(\theta)\,f(\theta)\,\gamma(0)^{-1}
    \end{equation}
    solves \eqref{eq:ODE} for $(\operatorname{Ad}_\gamma\phi + \lambda\,\gamma'\gamma^{-1},\,\lambda)$
    and satisfies $\tilde f(0)=\mathbf{1}$.
    Consequently the monodromy transforms by conjugation:
    \begin{equation}\label{eq:monodromy-conj}
        M_{\gamma\cdot\phi} \;=\; \gamma(0)\,M_\phi\,\gamma(0)^{-1}.
    \end{equation}
\end{proposition}

\begin{proof}
    Let $\gamma\in LG$, and suppose $f:\mathbb R\to G$ solves
    \[
        f'(\theta)\,f(\theta)^{-1} \;=\; \lambda^{-1}\,\phi(\theta),\qquad f(0)=\mathbf 1.
    \]
    Define
    \[
        \tilde f(\theta) \;:=\; \gamma(\theta)\,f(\theta)\,\gamma(0)^{-1}.
    \]
    We claim that
    \[
        \tilde f'(\theta)\,\tilde f(\theta)^{-1}
        \;=\; \lambda^{-1}\Big(\Ad_{\gamma(\theta)}\phi(\theta)\;+\;\lambda\,\gamma'(\theta)\gamma(\theta)^{-1}\Big),
    \]
    so $\tilde f$ solves the ODE corresponding to $(\Ad_\gamma\phi+\lambda\,\gamma'\gamma^{-1},\lambda)$ and satisfies $\tilde f(0)=\mathbf 1$.

    \medskip

    \noindent\textit{Proof.}
    First compute the derivative:
    \[
        \tilde f'(\theta)
        = \gamma'(\theta)\,f(\theta)\,\gamma(0)^{-1}
        + \gamma(\theta)\,f'(\theta)\,\gamma(0)^{-1}.
    \]
    Next note that
    \[
        \tilde f(\theta)^{-1}
        = \gamma(0)\,f(\theta)^{-1}\,\gamma(\theta)^{-1}.
    \]
    Hence
    \begin{align*}
        \tilde f'(\theta)\,\tilde f(\theta)^{-1}
         & = \Big(\gamma' f \gamma(0)^{-1} + \gamma f' \gamma(0)^{-1}\Big)
        \Big(\gamma(0) f^{-1} \gamma^{-1}\Big)                                 \\
         & = \gamma' f f^{-1} \gamma^{-1} \;+\; \gamma f' f^{-1} \gamma^{-1}   \\
         & = \gamma' \gamma^{-1} \;+\; \gamma \big(f' f^{-1}\big) \gamma^{-1}.
    \end{align*}
    Insert the original ODE $f' f^{-1} = \lambda^{-1}\phi$:
    \[
        \tilde f'\,\tilde f^{-1}
        = \gamma'\gamma^{-1} +  \lambda^{-1}\,\gamma \phi \gamma^{-1}
        = \lambda^{-1}\Big(\Ad_\gamma \phi \;+\; \lambda\,\gamma'\gamma^{-1}\Big).
    \]
    Finally, $\tilde f(0)=\gamma(0)\,f(0)\,\gamma(0)^{-1}=\mathbf 1$, as required. Also $\tilde f(0)=\gamma(0)\mathbf{1}\gamma(0)^{-1}=\mathbf{1}$. Evaluating at $\theta=2\pi$ and using $\gamma(2\pi)=\gamma(0)$ (loop),
    \[
        M_{\gamma\cdot\phi}=\tilde f(2\pi)=\gamma(0)\,f(2\pi)\,\gamma(0)^{-1}
        =\gamma(0)\,M_\phi\,\gamma(0)^{-1}.
    \] as desired.
\end{proof}

\begin{remark}
    Equation \eqref{eq:POexp} identifies $f$ as the parallel transport for the
    connection one-form $A=\lambda^{-1}\phi(\theta)\,d\theta$ on the trivial $G$-bundle
    over $S^1$, and $M_\phi$ as its holonomy around the circle.
\end{remark}

The following proposition follows from the previous discussion.
\begin{proposition}[Coadjoint orbits and conjugacy classes]
    \begin{enumerate}[(i)]
        \item If $G$ is simply connected and $\lambda \neq 0$ then the orbits of $LG$
              on the smooth part of $(L\mathfrak{g})^* \times \{\lambda\} \subset (\widetilde{L\mathfrak{g}})^*$
              correspond precisely to the conjugacy classes of $G$ under the map
              $(\phi,\lambda) \mapsto M_\phi$.
        \item The stabilizer of $(\phi,\lambda)$ in $LG$ is isomorphic to the centralizer
              $Z_\phi$ of $M_\phi$ in $G$ by the map $\gamma \mapsto \gamma(0)$; and
              $\gamma$ stabilizes $(\phi,\lambda)$ if and only if
              \[
                  \gamma(\theta) = f(\theta)\gamma(0)f(\theta)^{-1}.
              \]
    \end{enumerate}
\end{proposition}
\begin{proof}
    The relation $M_{\gamma\cdot(\phi,\lambda)} = \gamma(0)\,M_\phi\,\gamma(0)^{-1}$ shows that $(\phi,\lambda)$ and $(\phi',\lambda)$ are in the same $LG$-orbit implies $M_\phi$ and $M_{\phi'}$ are conjugate in $G$. Conversely, if $M_{\phi'} = g M_\phi g^{-1}$ for some $g \in G$, there exists a loop $\gamma \in LG$ with $\gamma(0) = g$. Then by the previous proposition, $\gamma \cdot (\phi,\lambda)$ has monodromy $M_{\phi'}$.

    The map $(\phi,\lambda) \mapsto M_\phi$ is surjective onto conjugacy classes. Let $C\subset G$ be a conjugacy class. Choose $g\in C$. Pick $X\in\mathfrak g$ with $\exp(2\pi X)=g$ (for compact connected $G$ this is always possible since every element lies in a maximal torus and $\exp:\mathfrak t\to T$ is surjective). Take $\phi(\theta)\equiv -\lambda X$ (constant). Then the solution is $f(\theta)=\exp(\theta X)$, hence $M_\phi=g\in C$.

    we can find $\phi''$ such that $M_{\phi''} = M_{\phi'}$. Thus $(\phi',\lambda)$ and $(\phi'',\lambda)$ have the same monodromy and hence are in the same $LG$-orbit. This establishes the bijection between $LG$-orbits and conjugacy classes in $G$.

    Now we show injectivity of fixed monodromy. Suppose $(\phi,\lambda)$ and $(\phi',\lambda)$ have the same monodromy:
    $M_\phi=M_{\phi'}=M$.
    Let $f,f'$ be their ODE solutions with $f(0)=f'(0)=\mathbf 1$. Define
    $\gamma(\theta):=f'(\theta)\,f(\theta)^{-1}$.
    Then $\gamma(0)=\mathbf 1$ and, using $f(\theta+2\pi)=f(\theta)M$, $f'(\theta+2\pi)=f'(\theta)M$,
    $\gamma(\theta+2\pi)=f'(\theta)M\,(f(\theta)M)^{-1}=\gamma(\theta)$,
    so $\gamma\in LG$. A direct calculation gives (with the "+" convention)
    \[
        \gamma\cdot\phi=\Ad_\gamma\phi+\lambda\,\gamma'\gamma^{-1}=\phi'.
    \]
    Hence points with the same monodromy lie in the same $LG$-orbit.

    As for the second claim, suppose $\gamma \in LG$ stabilizes $(\phi,\lambda)$. Then by definition of the action, $(\phi,\lambda) = \gamma\cdot(\phi,\lambda)$. This means the transformed ODE solution $\tilde f(\theta) = \gamma(\theta) f(\theta)\gamma(0)^{-1}$ equals the original $f(\theta)$ (since both solve the same ODE with same initial condition). So we must have $f(\theta) = \gamma(\theta) f(\theta)\gamma(0)^{-1}$, or equivalently, $\gamma(\theta) = f(\theta)\gamma(0)f(\theta)^{-1}$. In particular, at $\theta=2\pi$, $\gamma(2\pi) = f(2\pi)\gamma(0)f(2\pi)^{-1}$. But since $\gamma$ is a loop, $\gamma(2\pi) = \gamma(0)$. This forces $\gamma(0) \in Z_G(M_\phi)$, i.e. $\gamma(0)$ lies in the centralizer of $M_\phi$.

    So the stabilizer subgroup of $LG$ maps isomorphically to the centralizer $Z_\phi$ under the map $\gamma \mapsto \gamma(0)$.
\end{proof}

\begin{remark}
    In general, the coadjoint action only integrates to an action of the component containing $\gamma$. To guarantee there's no component obstruction (and to integrate the infinitesimal formulas globally), we want $LG$ to be connected. For connected $G$, $\pi_0(LG)\;\cong\;\pi_1(G)$. Thus if $G$ is simply connected, then $LG$ is connected, and the coadjoint action integrates on all of $LG$ with no ambiguity.
\end{remark}


According to Kirillov's idea, the irreducible unitary representations of a
group $\Gamma$ correspond to the coadjoint orbits $\Omega$ with the property

\begin{quote}
    (C) if the stabilizer of $\Phi \in \Omega$ is the subgroup $H$ of $\Gamma$
    then $\Phi$ is the derivative of a character of the identity component of $H$.
\end{quote}

The group--level central extension is
$1 \;\to\; \mathbb T \;\to\; \widetilde{LG} \;\to\; LG \;\to\; 1$,
where $\mathbb T = U(1)$ is the circle. The Lie algebra of this circle is just $\mathbb R K$ with basis element $K$. In the dual $(\widetilde{L\mathfrak g})^*$, the functional $(\phi,\lambda)$ evaluates to
$\langle (\phi,\lambda), K \rangle = \lambda$.


Kirillov's condition (C) says:
If $H$ is the stabilizer of a coadjoint point $\Phi$, then the restriction $\Phi|_{\mathfrak h}$ must equal the differential of a unitary character of $H^0$. For every $(\phi,\lambda)$, the central subgroup $\mathbb T \subset \widetilde{LG}$ is contained in its stabilizer (since it's central, it fixes everything). So $H^0$ contains $\mathbb T$, and we must check condition (C) on that subgroup.

So we need: the restriction of $\Phi$ to the Lie algebra of $\mathbb T$ (spanned by $K$) must be the differential of some unitary character $\chi:\mathbb T\to U(1)$. The circle group $\mathbb T = \{ e^{i\theta} : \theta\in\mathbb R \}$ has all unitary characters given by $\chi_n(e^{i\theta}) = e^{in\theta}$ for $n\in\mathbb Z$. Differentiate at the identity $(\theta=0)$: $\chi_n'(0) = in$. By definition of $(\phi,\lambda)$,
$\langle (\phi,\lambda),K\rangle = \lambda$. Condition (C) requires this number $\lambda$ to equal the derivative of some unitary character of $\mathbb T$. Therefore we have $\lambda \in \mathbb Z$.

By the above argument, if $(L\mathfrak{g})^* \times \{\lambda\}$ is allowable then $\lambda$ must be an integer. Then an orbit in the smooth part of the dual corresponds to the
conjugacy class of an element $g \in G$, which we can assume to belong to a given
maximal torus $T$. If we choose
\[
    \xi \in \mathfrak{t} \subset \mathfrak{g} \subset L\mathfrak{g} \subset (L\mathfrak{g})^*
\]
so that $\exp(\lambda^{-1}\xi) = g$, then $(\xi,\lambda)$ belongs to the orbit. This is because one can check that the solution of \eqref{eq:ODE} is $f(\theta) = \exp(\lambda^{-1}\theta \xi)$, which has monodromy $M_\phi = \exp(2\pi \lambda^{-1}\xi) = g$.



If $g$ is sufficiently generic then its centralizer in $G$ is $T$. Recall that we say an element $g \in G$ (or equivalently $X \in \mathfrak{g}$) is regular if its centralizer has minimal possible dimension. The minimal possible centralizer in a compact Lie group is precisely a maximal torus $T$. Concretely, if $X \in \mathfrak{t}$, then $Z_G(X)$ consists of the torus $T$ plus all root subgroups $\mathfrak{g}_\alpha$ for which $\alpha(X)=0$. If $\alpha(X)=0$ for some root, then the centralizer strictly contains $T$. So for $X$ regular (i.e. $\alpha(X)\neq 0$ for all roots), $Z_G(X)=T$. Therefore, if $g=\exp(X)$ with $X$ regular in $\mathfrak{t}$, then $Z_G(g) = T$.

And the condition (C) amounts to the requirement that $\xi \in \mathfrak{t} \subset \mathfrak{t}^*$ belongs to the lattice $\hat{T}$.

Recall that the stabilizer $H$ of $(\phi,\lambda)$ in $LG$ is isomorphic to $Z_G(g)=T$ by $\gamma\mapsto \gamma(0)$; more concretely, after conjugating by the associated $f$, one may (and we will) work at the representative $(\xi,\lambda)$ with $\xi\in\mathfrak{t}$ constant. Then $H^0 \cong T$ and $\mathfrak{h} \cong \mathfrak{t}$. Condition (C) says that the restriction of $(\xi,\lambda)$ to $\mathfrak{h}\cong\mathfrak{t}$ must be the differential of a character of $T$. The restriction is simply $Y\in\mathfrak{t} \mapsto \langle \xi,Y\rangle\in\mathbb{R}$ (using the fixed invariant inner product to identify $\mathfrak{t}\cong\mathfrak{t}$). This linear form exponentiates to a character of $T$ if and only if it takes integral values on the period lattice $\widehat{T}:=\ker(\exp: \mathfrak{t} \to T)$. That is, $\langle \xi,\eta\rangle \in 2\pi\mathbb{Z}$ for all $\eta\in \widehat{T}$. With our normalization (Pressley-Segal identify $\mathfrak{t}\simeq \mathfrak{t}^*$ using the basic inner product and absorb the $2\pi$ in the definition of $\widehat{T}$), this is precisely the statement $\xi \in \widehat{T}$.

On the other hand, $(\xi,\lambda)$ and $(\tilde{\xi},\lambda)$ belong to the same orbit if  $\tilde{\xi} = w \cdot \xi + \lambda \eta$ for some $\eta \in \hat{T}$ and some $w$ in the Weyl group $W$ of $G$.

\begin{proposition}[Coadjoint orbits satisfying (C)]
    If $\lambda$ is a non-zero integer then the coadjoint orbits in the smooth part of
    $(L\mathfrak{g})^* \times \{\lambda\}$ which satisfy the condition (C) correspond
    to the orbits of the affine Weyl group
    $W_{\mathrm{aff}} = W \ltimes \hat{T}$ on the lattice $\hat{T}$, where
    $(w,\eta) \in W_{\mathrm{aff}}$ acts on $\hat{T}$ by
    \[
        \xi \mapsto w \cdot \xi + \lambda \eta.
    \]
\end{proposition}

\begin{proof}
    Every orbit contains a representative $(\xi,\lambda)$ with $\xi\in\mathfrak t$,
    since any conjugacy class of $G$ meets $T$ and the monodromy of $(\xi,\lambda)$
    is $M_\xi = \exp(2\pi \xi/\lambda)\in T$.
    Suppose $(\xi,\lambda)$ and $(\tilde\xi,\lambda)$ are in the same orbit.
    Then there exists $\gamma\in LG$ such that
    \[
        \tilde\xi = \Ad_\gamma\xi + \lambda\,\gamma'\gamma^{-1}.
    \]

    Both $M_\xi$ and $M_{\tilde\xi}$ lie in $T$. Since
    $M_{\tilde\xi}=\gamma(0)M_\xi\gamma(0)^{-1}$, the endpoint $\gamma(0)$ normalizes $T$.
    This is because if $(\xi,\lambda)$ and $(\tilde\xi,\lambda)$ are in the same orbit, then their monodromies are conjugate by $\gamma(0)$, but since they both lie in $T$, they are in fact equal and therefore \begin{align*}
        M_{\tilde\xi} = M_\xi & \implies \gamma(0)M_\xi\gamma(0)^{-1} = M_\xi
    \end{align*}
    So $\gamma(0)\in N_G(T)$. Modulo $T$ this determines an element $w\in W$, and the
    constant loop $\gamma(\theta)\equiv n$ with $n\in N_G(T)$ representing $w$ acts by
    \[
        \gamma\cdot(\xi,\lambda) = (w\cdot\xi,\lambda).
    \]

    There are a second class of loops $\gamma(\theta)$ in $T$ act trivially on $\xi$ but contribute through the cocycle term:
    \[
        \Ad_\gamma \xi = \xi, \qquad \gamma'\gamma^{-1}\in \widehat T.
    \]
    Concretely, if $\gamma(\theta)=\exp(\theta\eta)$ with $\eta\in\widehat T$,
    then $\gamma'\gamma^{-1}=\eta$ and
    \[
        \gamma\cdot(\xi,\lambda) = (\xi+\lambda\eta,\lambda).
    \]
    because of the general formula for the coadjoint action. \[
        \gamma\cdot(\phi,\lambda) \;=\; (\operatorname{Ad}_\gamma\phi + \lambda\,\gamma'\gamma^{-1},\,\lambda).
    \]
This shows that the orbit contains all points of the form
    $(w\cdot\xi+\lambda\eta,\lambda)$ with $w\in W$ and $\eta\in\widehat T$.

The converse will be treated in the following lemma. This establishes a bijection between orbits and $W_{\mathrm{aff}}$-orbits
    in $\hat{T}$.
\end{proof}



\begin{lemma}[Exhaustion by Weyl and lattice moves]
Let $G$ be compact, connected and simply connected, $T\subset G$ a maximal
torus with Lie algebra $\mathfrak t$, Weyl group $W=N_G(T)/T$, and
$\widehat T=\ker(\exp:\mathfrak t\to T)$. Fix $\lambda\in\mathbb Z\setminus\{0\}$.
If $(\xi,\lambda)$ and $(\tilde\xi,\lambda)$ with $\xi,\tilde\xi\in\mathfrak t$
lie in the same $LG$–orbit, then there exist $w\in W$ and $\eta\in\widehat T$
such that
\[
\tilde\xi \;=\; w\!\cdot\xi \;+\; \lambda\,\eta.
\]
\end{lemma}

\begin{proof}
Assume $(\tilde\xi,\lambda)=\gamma\cdot(\xi,\lambda)$ for some $\gamma\in LG$.

Let $M_\xi:=\exp(2\pi\xi/\lambda)\in T$ and $M_{\tilde\xi}:=\exp(2\pi\tilde\xi/\lambda)\in T$
be their monodromies. The general monodromy formula gives
\[
M_{\tilde\xi} \;=\; \gamma(0)\,M_\xi\,\gamma(0)^{-1}.
\]
Since $M_\xi,M_{\tilde\xi}\in T$, by the standard conjugacy theorem
(“$G$–conjugacy on $T$ is $W$–conjugacy”), there exists $n\in N_G(T)$ with
\[
M_{\tilde\xi} \;=\; n\,M_\xi\,n^{-1}.
\]
Let $w\in W$ be the class of $n$. Replace $\gamma$ by
\[
\gamma_1 \;:=\; n^{-1}\gamma \in LG,
\qquad
\text{and set}\qquad
\hat\xi \;:=\; \Ad_{n^{-1}}\tilde\xi \;=\; w^{-1}\!\cdot\tilde\xi \in \mathfrak t.
\]
Then $\gamma_1\cdot(\xi,\lambda)=(\hat\xi,\lambda)$ and
\[
M_{\hat\xi}
= \gamma_1(0)\,M_\xi\,\gamma_1(0)^{-1}
= n^{-1}\gamma(0)\,M_\xi\,\gamma(0)^{-1}n
= n^{-1}M_{\tilde\xi}n
= M_\xi.
\]
Thus $\xi,\hat\xi\in\mathfrak t$ have \emph{equal} monodromy:
$\exp(2\pi\hat\xi/\lambda)=\exp(2\pi\xi/\lambda)$.

Consider the solutions
\[
f(\theta)=\exp\!\Big(\tfrac{\theta}{\lambda}\xi\Big),\qquad
\hat f(\theta)=\exp\!\Big(\tfrac{\theta}{\lambda}\hat\xi\Big)\qquad(\in T),
\]
and define the $T$–valued loop
\[
\delta(\theta)\;:=\;\hat f(\theta)\,f(\theta)^{-1}\in T.
\]
Since $T$ is abelian, we have
\[
\delta(\theta)=\exp\!\Big(\tfrac{\theta}{\lambda}(\hat\xi-\xi)\Big),
\qquad
\delta'(\theta)\,\delta(\theta)^{-1}=\tfrac{1}{\lambda}(\hat\xi-\xi)\in\mathfrak t.
\]
Moreover $\delta(2\pi)=1$ because $M_{\hat\xi}=M_\xi$. Hence
\[
\eta\;:=\;\tfrac{1}{\lambda}(\hat\xi-\xi)\;\in\;\widehat T,
\quad\text{and}\quad
\hat\xi=\xi+\lambda\eta.
\]
By the coadjoint action formula,
\[
\delta\cdot(\xi,\lambda)=(\xi+\lambda\eta,\lambda)=(\hat\xi,\lambda).
\]
Finally, undoing the $n^{-1}$–conjugation gives
\[
(\tilde\xi,\lambda)
=(n\cdot\delta)\cdot(\xi,\lambda)
=\Big(w\!\cdot(\xi+\lambda\eta),\,\lambda\Big)
=\Big(w\!\cdot\xi+\lambda\,w\!\cdot\eta,\,\lambda\Big).
\]
Since $\widehat T$ is $W$–stable, $w\!\cdot\eta\in\widehat T$; renaming
$\eta\leftarrow w\!\cdot\eta$ yields the claimed form
$\tilde\xi=w\!\cdot\xi+\lambda\eta$.
\end{proof}

\begin{remark}[Rotation action]
    We can rotate the loop parameter: $(R_\alpha\phi)(\theta) := \phi(\theta+\alpha)$, where $\alpha\in\mathbb T=S^1$. This gives an action of the rotation group $\mathbb T$ on $L\mathfrak g$, hence also on $(\widetilde{L\mathfrak g})^*$.

An orbit $\mathcal O$ is in the smooth part if it is stable under circle rotations. In other words, if rotating the loop parameter can be undone by some $LG$-coadjoint action.

At every point $(\phi,\lambda)\in\mathcal O$, the vector field generating rotations is tangent to the orbit. Equivalently: the infinitesimal variation $\delta_{\mathrm{rot}}\phi = \phi'$ must lie in the tangent space of the orbit. The tangent space at $(\phi,\lambda)$ to the coadjoint orbit is spanned by infinitesimal coadjoint actions: \[T_{(\phi,\lambda)}(\mathcal O) = \{([\eta,\phi] + \lambda \eta',\,0)\;:\; \eta \in L\mathfrak g\}\]
So for stability we require $\phi' \in \{[\eta,\phi] + \lambda \eta' : \eta\in L\mathfrak g\}$. Thus there must exist some $\eta \in L\mathfrak g$ such that $\phi'(\theta) = [\eta(\theta),\phi(\theta)] + \lambda \eta'(\theta)$.

If $\eta \in L\mathfrak g$ is smooth, then both $[\eta,\phi]$ and $\eta'$ are smooth in $\theta$. Hence $\phi'$ is smooth, which forces $\phi$ to be smooth. Representation-theoretically, this matches the fact that positive energy representations (the ones stable under rotations) correspond to smooth coadjoint orbits.
\end{remark}
\end{document}