\documentclass[12pt]{article}
\usepackage[english]{babel}
\usepackage[utf8x]{inputenc}
\usepackage[T1]{fontenc}
\usepackage{listings}
\usepackage{bookmark}
\usepackage{tikz}
\usepackage{/Users/songye03/Desktop/math_tex/style/quiver}
\usepackage{/Users/songye03/Desktop/math_tex/style/scribe}
\usepackage{fancyhdr}

\usepackage{parskip} % Automatically respects blank lines
\setlength{\parskip}{1em} % Adds more space between paragraphs
\setlength{\parindent}{0pt} % Removes paragraph indentation

\begin{document}


\lhead{Songyu Ye}
\rhead{\today}
\cfoot{\thepage}

\title{Finite Gauge Theory and Mednykh's Formula}

\author{Songyu Ye}
\date{\today}
\maketitle


\begin{abstract}
    We construct untwisted Dijkgraaf-Witten theory as an example of a 2-dimensional topological quantum field theory (TQFT) with gauge group a finite group $G$. We then use this TQFT to give a new proof of Mednykh's formula, which relates the number of homomorphisms from the fundamental group of a surface $\Sigma$ to $G$ and the dimensions of the irreducible representations of $G$.
\end{abstract}

\tableofcontents
\section{Construction of classical Dijkgraaf--Witten theory}
Dijkgraaf-Witten theory with gauge group a finite group $G$ is a toy model of Chern-theory, in which $G$ is replaced by a Lie group. It has a single field, which is a principal $G$-bundle on some manifold $M$. Since $G$ is finite, a principal $G$-bundle on $M$ is precisely a $G$-cover, which is in turn precisely a groupoid homomorphism
\[
    \Pi_1(M) \;\longrightarrow\; BG
    \tag{3}
\]
from the fundamental groupoid of $M$ to the groupoid $BG$ with one object whose automorphism group is $G$. The category of principal $G$-bundles on $M$ is precisely the functor category
\[
    \Pi_1(M) \;\Rightarrow\; BG,
\]
and classical Dijkgraaf--Witten theory assigns this functor category to $M$. This category, which is really a groupoid, should be thought of as the ``moduli stack'' $A_G(M)$ of principal $G$-bundles on $M$, since we keep track of automorphisms of bundles.

\begin{example}[Connected $M$]
    If $M$ connected, there is an equivalent groupoid $\cD$ with objects given by group homomorphisms $\pi_1(M) \to G$. The morphisms between group homomorphisms are given by conjugation in $G$.

    Pick a basepoint. Choose $x_0\in M$. In the fundamental groupoid $\Pi_1(M)$, the endomorphisms of $x_0$ are precisely $\pi_1(M,x_0)$. A functor $F\colon\Pi_1(M)\to BG$ sends objects (points of $M$) to the unique object of $BG$, and sends each morphism (homotopy class of path) to an element of $G=\mathrm{Aut}_{BG}(*)$. In particular, the restriction $\rho_F \;:=\; F\big|_{\mathrm{End}(x_0)}\colon \pi_1(M,x_0)\longrightarrow G$
    is a group homomorphism, because $F$ preserves composition and inverses. So every object $F$ determines a representation $\rho_F\colon\pi_1(M,x_0)\to G$. A morphism $\eta\colon F\Rightarrow F'$ is a natural transformation, which is determined by the element $\eta_{x_0}\in G$. The naturality square for $\eta$ and any loop $\gamma\in \pi_1(M,x_0)$ given by \[
        \begin{tikzcd}
            F(x) \arrow[r,"F(\gamma)"] \arrow[d,"\eta_x"'] & F(x) \arrow[d,"\eta_x"] \\
            F'(x) \arrow[r,"F'(\gamma)"'] & F'(x)
        \end{tikzcd}
    \]

    says that
    \[\eta_{x_0} \, \rho_F(\gamma) = \rho_{F'}(\gamma)\, \eta_{x_0}\] so $\rho_F$ and $\rho_{F'}$ differ by conjugation in $G$. Note that this conjugation can be thought of as changing the basepoint $x_0 \in M$ to another point $x\in M$, because the conjugation is using the functoriality of $F$ on a path from $x_0$ to $x$. There is another conjugation what one divides out by which is changing the anchor point in the fiber of the principal $G$-bundle over $x_0$.

    Conversely, for each $x\in M$, choose once and for all a path $c_x:x_0\to x$ with $c_{x_0}=\mathrm{id}$. (Any choice will do; different choices produce naturally isomorphic functors.) Given a homotopy class $[\gamma]:x\to y$, form the loop at $x_0$ and define $\ell(\gamma)\ :=\ [\,c_y^{-1}\cdot \gamma \cdot c_x\,]\ \in \pi_1(M,x_0)$.
    Set
    $F_\rho([\gamma])\ :=\ \rho\big(\ell(\gamma)\big)\ \in G$.
    This respects composition:
    \[F_\rho([\gamma_2\circ\gamma_1])
        = \rho\!\big([c_z^{-1}\gamma_2\gamma_1 c_x]\big)
        = \rho\!\big([c_z^{-1}\gamma_2 c_y]\big)\ \rho\!\big([c_y^{-1}\gamma_1 c_x]\big)
        = F_\rho([\gamma_2])\,F_\rho([\gamma_1])\]

    Suppose $g:\rho\to\rho'$ in $\mathcal{D}$, so $\rho'(\gamma)=g \rho(\gamma) g^{-1}$.
    Define a natural transformation $\eta:\Psi(\rho)\Rightarrow \Psi(\rho')$ by setting
    $\eta_x := g \in G$ \quad \text{for all } $x\in M$.
    Naturality is automatic because
    \[\eta_y\,\Psi(\rho)(\gamma) = g\,\rho(\ell(\gamma)) = \rho'(\ell(\gamma))\,g = \Psi(\rho')(\gamma)\,\eta_x\] So $F_\rho$ is indeed a functor $\Pi_1(M)\to BG$, and by construction $F_\rho\big|_{\pi_1(M,x_0)}=\rho$.

    This gives an equivalence of groupoids $\cD\to \Pi_1(M)\Rightarrow BG$.
\end{example}


\begin{example}
    Let $M = S^1$. A principal $G$-bundle on $M$ may then be identified with a homomorphism $\mathbb{Z} \to G$, hence with an element of $G$. In this case $\Pi_1(S^1) \Rightarrow BG$ is the groupoid $G\sslash G$ of elements of $G$ up to conjugation. That is, it is the groupoid whose objects are the elements of $G$ and where the morphisms $g \to h$ are given by elements $a \in G$ such that $a g a^{-1} = h$.

    Where does the conjugation action come from geometrically? Take the canonical generator $\gamma$ of $\pi_1(S^1)$, i.e. the loop that goes once around the circle.

    Lift $\gamma$ starting at $u$ in the fiber over $x_0$. Because of the path lifting property, the lifted path $\tilde\gamma$ is unique and ends at some $u\cdot g$ for a unique $g\in G$. Define this $g$ as the holonomy element associated to $\gamma$ and the choice of $u$.

    So with a fixed choice of $u$, you get a well-defined element $g$. But what if you had chosen a different anchor $u' = u\cdot h$ in the fiber at $x_0$? Lift $\gamma$ starting at $u'$. You'll end at $u' \cdot g' = (u\cdot h)\cdot g' = u\cdot (h g')$. On the other hand, uniqueness of lifts forces this endpoint to agree with the previous one: $u\cdot g$. Therefore $h g' = g h \quad\implies\quad g' = h^{-1} g h$.

    So the holonomy element depends on the choice of anchor up to conjugation.
\end{example}

Since we only consider compact manifolds $M$, the fundamental group of each connected component of $M$ is finitely generated, hence $\Pi_1(M) \to BG$ is an essentially finite groupoid (a groupoid equivalent to a groupoid with finitely many morphisms).


The assignment $M \mapsto A_G(M)$ defined by \begin{align*}
    M \mapsto A_G(M) & := \Pi_1(M) \Rightarrow BG
\end{align*}
is contravariant functor
\begin{equation} \label{eq:AG-functor}
    A_G : \mathbf{Man} \to \mathbf{Gpd}
\end{equation}
from the category of manifolds to the category of groupoids. It is a composition of two functors, namely the covariant functor
\[
    \Pi_1(-) : \mathbf{Man} \to \mathbf{Gpd}
\]
and the contravariant functor
\[
    (-) \Rightarrow BG : \mathbf{Gpd} \to \mathbf{Gpd}.
\]
Since we want to build a TQFT, we would like to extend $A_G$ to cobordisms. This is done as follows. If $M$ is an $n$-dimensional compact oriented manifold with boundary $\overline{X} \sqcup Y$ (so a cobordism $X \to Y$), then $M,X,Y$ together define a cospan
\begin{equation} \label{eq:cospan-manifolds}
    \begin{tikzcd}
        & M & \\
        X \arrow[ur] & & Y \arrow[ul]
    \end{tikzcd}
\end{equation}
of manifolds. Applying $\Pi_1(-)$ gives a cospan
\begin{equation} \label{eq:cospan-groupoids}
    \begin{tikzcd}
        \Pi_1(X) \arrow[dr] & & \Pi_1(Y) \arrow[dl] \\
        & \Pi_1(M) &
    \end{tikzcd}
\end{equation}
of groupoids, and applying $(-)\Rightarrow BG$ gives a span
\begin{equation} \label{eq:span-groupoids}
    \begin{tikzcd}
        & A_G(M) \arrow[dl] \arrow[dr] & \\
        A_G(X) & & A_G(Y)
    \end{tikzcd}
\end{equation}
of groupoids.

There is a category $\mathbf{Span}(\mathbf{FinGpd})$ whose objects are essentially finite groupoids and whose morphisms are (isomorphism classes of) spans of essentially finite groupoids, where composition of spans is given by taking pullbacks as follows:

\begin{equation}\label{eq:pullback-span}
    \begin{tikzcd}[row sep=2.2em, column sep=2.4em]
        & & Y_1 \times_{X_2} Y_2 \arrow[ld, "p_1"'] \arrow[rd, "p_2"] & & \\
        & Y_1 \arrow[ld, "f_1"'] \arrow[rd, "g_1"] & & Y_2 \arrow[ld, "f_2"'] \arrow[rd, "g_2"] & \\
        X_1 & & X_2 & & X_3
    \end{tikzcd}
\end{equation}

The assignment $M \mapsto A_G(M)$ then extends, by the groupoid Seifert-van Kampen theorem, to a (symmetric monoidal) functor
\begin{equation} \label{eq:AG-functor-extended}
    A_G : n\mathbf{Cob} \to \mathbf{Span}(\mathbf{FinGpd})
\end{equation}
which we might call \textbf{classical (untwisted) Dijkgraaf-Witten theory}.

The point is that one has to check that the assignment is functorial and symmetric monoidal. The justification is that if $M_1:X\to Y$ and $M_2:Y\to Z$ glue along $Y$ to $M_2\circ M_1$, then for inclusions $X\hookrightarrow M_1\hookleftarrow Y$ and $Y\hookrightarrow M_2\hookleftarrow Z$ the groupoid Seifert--van Kampen theorem gives a pushout:
$\Pi_1(M_2\circ M_1)\ \simeq\ \Pi_1(M_1)\ \amalg_{\Pi_1(Y)}\ \Pi_1(M_2)$.
Applying $\mathrm{Fun}(-,BG)$ (which is contravariant) turns that pushout into a pullback:
$A_G(M_2\circ M_1)
    \;\simeq\;
    A_G(M_1)\ \times_{A_G(Y)}\ A_G(M_2)$.
But composition in $\mathbf{Span}(\mathbf{FinGpd})$ is pullback of the middle objects. Hence the span for the glued cobordism equals the pullback of spans—so the assignment is functorial.

To check that the functor is symmetric monoidal, note that $\Pi_1$ sends disjoint unions to coproducts
\[\Pi_1(X\sqcup X')\cong \Pi_1(X)\amalg \Pi_1(X')\]
Functors into $BG$ turn coproducts into products:
\[\mathrm{Fun}(\Pi_1(X)\amalg \Pi_1(X'),BG)
    \;\cong\;
    \mathrm{Fun}(\Pi_1(X),BG)\times \mathrm{Fun}(\Pi_1(X'),BG)\]
Hence $A_G(X\sqcup X')\ \cong\ A_G(X)\times A_G(X')$ and similarly for cobordisms (spans tensor by taking products). The unit $\varnothing$ maps to the terminal groupoid $1=\mathrm{Fun}(\varnothing,BG)$. The symmetry (swap of components) is preserved, so the functor is symmetric monoidal.

\section{Linearization and quantization}
Quantum (untwisted) Dijkgraaf–Witten theory $Z_G$ is obtained from the classical theory $A_G$ by applying a linearization functor
\[
    \mathbb{C}^{(-)} : \mathbf{Span}(\mathbf{FinGpd}) \to \mathbf{FinVect}.
\]
This functor is heuristically given by Feynman integrals over “moduli stacks” of principal $G$-bundles and rigorously defined using a push-pull construction as follows.

\begin{remark}[Topological field theory and path integrals]
    In a quantum field theory, the partition function on a manifold $M$ is heuristically written as
    \[Z(M) \;=\; \int_{\text{fields on }M} e^{iS(\text{field})}\, \mathcal{D}(\text{field})\]

    The fields are the configurations of the theory (here, principal $G$-bundles on $M$ with connection, if you were in Chern-Simons).
    $S$ is the action functional and $\mathcal{D}(\text{field})$ is the (heuristic) "measure" on the space of fields.

    In the topological case (finite gauge group, no action term), this integral reduces to "summing over all gauge fields," i.e. over all principal $G$-bundles.
\end{remark}

\begin{remark}[Moduli stacks instead of sets]
    Naively try $Z(M) \stackrel{?}{=} \#\{\text{principal $G$-bundles on $M$}\}$.
    But that's not quite right. To get a correct measure you shouldn't just count isomorphism classes, but rather weight each object by $1/|\mathrm{Aut}(P)|$.

    This leads to the notion of the groupoid cardinality
    \[\#\!\!\left[\tfrac{\{\text{$G$-bundles on $M$}\}}{\text{iso}}\right]
        \;=\; \sum_{[P]} \frac{1}{|\mathrm{Aut}(P)|}\]
    In general if $G$ acts on a set $X$, the groupoid cardinality of the action groupoid $X\sslash G$ is
    \[\#(X\sslash G) \;=\; \sum_{[x]\in X/G} \frac{1}{|\mathrm{Stab}_G(x)|} \;=\; \frac{|X|}{|G|}.\]
    This is the correct measure for counting or integrating over moduli stacks. \red{I really don't understand this remark. I need to see an example computation where this weighting gives the right answer.}
\end{remark}

If $X$ is an essentially finite groupoid, let $\mathbb{C}^X$ denote the space of complex-valued functions on the objects of $X$ such that if there exists a morphism $p : x \to y$ in $X$, then $f(x)=f(y)$. Equivalently, $\mathbb{C}^X$ denotes the space of complex-valued functions on the isomorphism classes $\pi_0(X)$ of objects of $X$. A functor $F : X \to Y$ induces two linear maps between these spaces of functions. One of them is the pullback
\[
    F^* : \mathbb{C}^Y \to \mathbb{C}^X
\]
given by
\[
    (F^*(f))(x) = f(F(x)).
\]

The other is the pushforward
\[
    F_* : \mathbb{C}^X \to \mathbb{C}^Y
\]
given by
\[
    (F_*(f))(y) = \sum_{x \in \pi_0(X):\,F(x)\cong y} \frac{|\mathrm{Aut}(y)|}{|\mathrm{Aut}(x)|}\, f(x).
\]

The pushforward is adjoint to the pullback in the following sense. $\mathbb{C}^X$ admits a distinguished linear functional
\[
    \int_X : \mathbb{C}^X \ni f \longmapsto \sum_{x \in \pi_0(X)} \frac{f(x)}{|\mathrm{Aut}(x)|} \;\in \mathbb{C},
\]
which should be thought of as integration over $X$. The vector space $\mathbb{C}^X$ is also a commutative algebra under pointwise multiplication, and the integral of the identity recovers the groupoid cardinality of $X$. These two structures combine to give an inner product
\[
    \langle f,g \rangle_X \;=\; \int_X f(x)g(x)\, dx
    = \sum_{x \in \pi_0(X)} \frac{f(x)g(x)}{|\mathrm{Aut}(x)|}
\]
on $\mathbb{C}^X$, and the pushforward is adjoint to the pullback with respect to this inner product. It should be thought of as integration along fibers.

\begin{remark}
    [Unwinding the adjointness]
    We need to check that if $f\in \mathbb{C}^X$ and $g\in \mathbb{C}^Y$, and $F : X \to Y$ is a functor between essentially finite groupoids, then
     \begin{align*}
        \langle F^* g, f\rangle_X & = \langle g, F_* f\rangle_Y
    \end{align*}
Expanding the LHS gives:
\[
\langle F^* g, f\rangle_X
= \sum_{[x]\in \pi_0(X)} \frac{(F^* g)(x)\, f(x)}{|\Aut(x)|}
= \sum_{[x]} \frac{g(F(x))\, f(x)}{|\Aut(x)|}.
\]
Expanding the RHS gives:
\begin{align*}
    \langle g, F_* f\rangle_Y &= \sum_{[y]\in \pi_0(Y)} \frac{g(y)\, (F_* f)(y)}{|\Aut(y)|} \\
    &= \sum_{[y]} \frac{g(y)}{|\Aut(y)|} \left( \sum_{[x]:F(x)\cong y} \frac{|\Aut(y)|}{|\Aut(x)|}\, f(x) \right) \\
    &= \sum_{[y]} \sum_{[x]:F(x)\cong y} \frac{g(y)\, f(x)}{|\Aut(x)|}
\end{align*}
which is the same as LHS after reindexing the sum over $[x]$. 
\end{remark}
Pushforward and pullback allow us to linearize a span of groupoids
\[
\begin{tikzcd}
& Z \arrow[dl, "p"'] \arrow[dr, "q"] & \\
X & & Y
\end{tikzcd}
\]
by associating to it the composition
\[
q_* \circ p^* : \mathbb{C}^X \to \mathbb{C}^Y
\]
which explicitly takes the form
\[
\big((q_* \circ p^*)(f)\big)(y) \;=\; 
\sum_{z \in \pi_0(Z)\,:\,q(z)\cong y} \frac{|\Aut(y)|}{|\Aut(z)|}\, f(p(z)).
\]

This defines a functor $\mathbb{C}^{(-)} : \mathbf{Span}(\mathbf{FinGpd}) \to \mathbf{FinVect}$ which, when composed with $A_G$, gives \textbf{untwisted Dijkgraaf-Witten theory}
\[
Z_G = \mathbb{C}^{A_G} : n\mathbf{Cob} \to \mathbf{FinVect}.
\]

\begin{example}
Suppose $M$ is a compact oriented $n$-manifold without boundary. Then 
\[
Z_G(M) : \mathbb{C} \to \mathbb{C}
\]
is multiplication by the groupoid cardinality
\[
\sum_{x \in \pi_0(A_G(M))} \frac{1}{|\Aut(x)|}.
\]
When $M$ is connected this sum is equal to
\[
\frac{|\Hom(\pi_1(M),G)|}{|G|}.
\]
\end{example}
\section{2D TQFT}
We now restrict our attention to the case $n=2$.

A 2D TQFT is a symmetric monoidal functor
$Z : 2\mathbf{Cob} \;\to\; \mathbf{Vect}$.
    •	Objects of $2\mathbf{Cob}$: finite disjoint unions of circles.
    •	Morphisms: oriented compact surfaces with boundary, viewed as cobordisms between collections of circles.

Under such a functor:
\begin{itemize}
    \item $Z(S^1) = A$, a vector space.
    \item pair of pants (2 in, 1 out): multiplication $m:A\otimes A\to A$,
    \item pair of pants (1 in, 2 out): comultiplication $\Delta:A\to A\otimes A$,
    \item cap (no in, 1 out): unit $\eta:\mathbb{C}\to A$,
    \item cup (1 in, no out): counit $\varepsilon:A\to\mathbb{C}$.
\end{itemize}
This structure is exactly a commutative Frobenius algebra. This linear functional $\lambda$ is the trace map of the Frobenius algebra. The nondegeneracy condition on the Frobenius pairing $(a,b)\mapsto \lambda(ab)$ comes precisely from gluing two cups with a pair of pants (geometrically giving a sphere).

In this case 2-dimensional TQFTs have a known classification as follows. If $Z$ is such a TQFT, then $Z(S^1)$ is a commutative Frobenius algebra (an algebra $A$ equipped with a linear functional $\lambda$ such that the bilinear form $\lambda(ab)$ is nondegenerate), and conversely given any commutative Frobenius algebra $A$ there is a unique (up to equivalence) TQFT such that $Z(S^1)\cong A$. The structure maps of $A$ and the axioms they satisfy come from a description of $2\mathbf{Cob}$ in terms of generators and relations; for example, the multiplication in $A$ comes from the pair of pants and the linear functional comes from the cup.

The commutative Frobenius algebra associated to untwisted Dijkgraaf-Witten theory can be explicitly described. We saw above that $A_G(S^1)$ is the groupoid of elements of $G$ up to conjugation. It follows that $Z_G(S^1)$ can be identified with the space of class functions on $G$. The linear functional on $Z_G(S^1)$ is induced by the cap, which, after applying the classical functor $A_G(-)$, gives the span of groupoids
\[
\begin{tikzcd}
& BG \arrow[dl] \arrow[dr] & \\
G\sslash G & & 1
\end{tikzcd}
\]
where $G\sslash G$, as above, denotes the groupoid of elements of $G$ up to conjugation; $BG$, as above, denotes the groupoid with one object and automorphism group $G$; and $1$ denotes the trivial groupoid. The map $BG \to G\sslash G$ is the inclusion of the identity into $G\sslash G$. Linearizing, we obtain the linear functional
\[
Z_G(S^1) \ni f \longmapsto \frac{f(1)}{|G|} \in Z_G(\varnothing) \cong \mathbb{C},
\]
where $1$ denotes the identity element of $G$.

The multiplication on $Z_G(S^1)$ is induced by the pair of pants, which, after applying the classical functor $A_G(-)$, gives the span of groupoids
\[
\begin{tikzcd}
& G\times G \sslash G \arrow[dl] \arrow[dr] & \\
G\sslash G \times G\sslash G & & G\sslash G
\end{tikzcd}
\]
\end{document}