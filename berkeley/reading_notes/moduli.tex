\documentclass[12pt]{article}
\usepackage[english]{babel}
\usepackage[utf8x]{inputenc}
\usepackage[T1]{fontenc}
\usepackage{listings}
\usepackage{bookmark}
\usepackage{tikz}

\makeatletter
\def\input@path{{../../style/}}
\makeatother

\usepackage{../../style/quiver}
\makeatletter
\def\input@path{{../../style/}}
\makeatother

\usepackage{../../style/scribe}

\usepackage{fancyhdr}

\usepackage{parskip} % Automatically respects blank lines
\setlength{\parskip}{1em} % Adds more space between paragraphs
\setlength{\parindent}{0pt} % Removes paragraph indentation

\DeclareMathOperator{\Spm}{Spm}
\begin{document}


\lhead{Songyu Ye}
\rhead{\today}
\cfoot{\thepage}

\title{Title}

\author{Songyu Ye}
\date{\today}
\maketitle


\begin{abstract}
    Abstract
\end{abstract}

\tableofcontents


\section{Primer on GIT}

\section{Moduli of semistable bundles}
In this section, we follow the treatment of \cite{mukai}.
Let $C$ be a smooth projective curve. Fixing any line
bundle $L$ on $C$, the set of isomorphism classes of stable vector bundles of rank $2$ with
determinant line bundle isomorphic to $L$ carries the structure of an algebraic variety.
\[SU_C(2,L) = \{E \text{ stable of rank } 2, \det E \cong L\} / \sim
\]
We assume that $L$ has sufficiently high degree to guarantee
that $E$ is generated by global sections, and we consider the skew-symmetric
bilinear map
\[ H^0(E) \times H^0(E) \to H^0(\wedge^2 E) \cong H^0(L)
\] given by wedge product of sections.
This form has rank 2, and we denote by $\mathrm{Alt}^2(H^0(L))$ the affine variety
which parametrises such skew-symmetric forms of rank $< 2$ in dimension
$N = \dim H^0(E)$. We will use this wedge
product to reduce our moduli problem to the quotient problem for the action
of $\GL(N)$ on $\mathrm{Alt}^2(H^0(L))$. One encounters various difficulties that do not
appear in the line bundle case of the last chapter, but it turns out that the notion
of stability is the correct way to resolve these problems, and one proves the
following.

\begin{theorem}[Moduli of rank 2 vector bundles with fixed determinant]\label{thm:moduli-of-rank-2-vb-with-fixed-det}
    Suppose that the line bundle $L$ has degree $\ge 4g - 1$.
    \begin{enumerate}[(i)]
        \item There exists a Proj quotient
              \[
                  \Alt^{ss}_{N,2}(H^0(L)) \sslash GL(N)
              \]
              which is a projective variety of dimension $3g - 3$.

        \item The open set
              \[
                  \Alt^{s}_{N,2}(H^0(L)) / GL(N)
              \]
              has an underlying set $SU_C(2,L)$. Moreover, it is nonsingular and at each point $E \in SU_C(2,L)$ its tangent space is isomorphic to $H^1(\mathfrak{sl}\,E)$.

        \item If $\deg L$ is odd, then
              \[
                  \Alt^{ss}_{N,2}(H^0(L)) \sslash GL(N)
                  \;=\;
                  \Alt^{s}_{N,2}(H^0(L)) / GL(N)
                  \;=\;
                  SU_C(2,L)
              \]
              is a smooth projective variety.
    \end{enumerate}
\end{theorem}

\subsection{Slope stability and Pfaffians}
We begin by recalling the notion of slope stability for vector bundles on curves. We will then introduce the Pfaffian of a skew-symmetric matrix, which will be the key semiinvariant we will use to study the Gieseker points associated to rank 2 vector bundles.

\begin{definition}
    Let $E$ be a vector bundle. A coherent subsheaf $F \subset E$ is a subbundle if $F$ is a vector bundle (i.e. locally free) and the cokernel $E/F$ is also locally free.
\end{definition}

\begin{remark}
    In particular, note that a subsheaf $F \subset E$ of a vector bundle can be a vector bundle in its own right, but not a subbundle if the cokernel $E/F$ has torsion.

    However, one can always saturate a subsheaf $F$ to get a subbundle $\overline{F}$ defined as the kernel of the composition
    \[E \to E/F \to (E/F)/\text{torsion}.
    \]
    The saturation $\overline{F}$ is the largest subbundle of $E$ containing $F$, and $\deg \overline{F} \ge \deg F$. One can check that any saturated subsheaf of a locally free sheaf is again locally free.
\end{remark}

\begin{definition}[Slope stability]
    A vector bundle $E$ on $C$ is stable (resp. semistable) if for every proper subbundle $F \subset E$, we have \[\frac{\deg F}{\mathrm{rank} F} < \frac{\deg E}{\mathrm{rank} E}\] (resp. $\le$). The ratio $\mu(E) = \frac{\deg E}{\mathrm{rank} E}$ is called the slope of $E$, and this condition is often called slope stability (semistability).
\end{definition}

\begin{definition}
    A vector bundle $E$ is \textbf{simple} if $\mathrm{End}(E) = k \cdot \mathrm{Id}_E$.
    A vector bundle $E$ is decomposable if it is isomorphic to the
    direct sum $E_1 \oplus E_2$ of two nonzero vector bundles; otherwise, $E$ is \textbf{indecomposable}.
\end{definition}

If $f\in\End(E)$ is an idempotent, $f^2=f$, then $f$ is the projection onto its image and $1-f$ is the projection onto its kernel. Hence the natural direct--sum decomposition of $E$ is
\[
    E=\operatorname{im}f\oplus\operatorname{ker}f,
\]
because $\operatorname{im}f\cap\operatorname{ker}f=0$ and $\operatorname{im}f+\operatorname{ker}f=E$.

Conversely, if $E=E_1\oplus E_2$ is a nontrivial decomposition, the projection onto the first summand along the second is an idempotent $f\in\End(E)$, $f^2=f$, whose image is $E_1$ and whose kernel is $E_2$. Thus decomposability of $E$ is equivalent to the existence of a nontrivial idempotent $f\neq0,\,1$ in $\End(E)$.
\begin{proposition}
    Every vector bundle $E$ can be uniquely decomposed as a direct sum of indecomposable vector bundles, up to isomorphism and permutation of the summands.
\end{proposition}

\begin{proof}
    Omitted.
\end{proof}

Given an endomorphism \(f : E \to E\), consider the determinant
\(\det f : \det E \to \det E\). This is just multiplication by a scalar because \(\det E\) is a line bundle, and this scalar is nonzero if and only if \(f\) is an isomorphism. Now, for an arbitrary scalar
\(\lambda\) consider \(\det(f - \lambda \mathrm{id})\). This is a
polynomial of degree \(r(E)\) in \(\lambda\) and is the characteristic
polynomial of the endomorphism \(f\). In particular, if \(\alpha\) is an eigenvalue, then \(f - \alpha\,\mathrm{id}\) fails to be an
isomorphism because its determinant is zero.
\begin{lemma}
    If \(E\) is indecomposable, then \(f \in \End(E)\) has only one eigenvalue.
\end{lemma}

\begin{proof}
    Suppose \(f\) has distinct eigenvalues \(\alpha\) and \(\beta\). Then its
    characteristic polynomial can be expressed as a product of two
    polynomials without common factors:
    \[
        \det(f - \lambda \cdot \mathrm{id})
        = p(\lambda)\, q(\lambda),
        \qquad p(\alpha) = 0, \quad q(\beta) = 0.
    \]
    There exist polynomials \(a(\lambda)\), \(b(\lambda)\) satisfying
    \[
        p(\lambda)\, a(\lambda) + q(\lambda)\, b(\lambda) = 1,
    \]
    Let $h(f) = p(f) a(f)$. Then \begin{align*}
        h(1-h) = \bigl(p(f)a(f)\bigr)\bigl(q(f)b(f)\bigr) = p(f)\,q(f)\,a(f)b(f) = 0
    \end{align*} by the Cayley-Hamilton theorem. This implies that \(E\) is
    the direct sum \(\ker(h) \oplus \ker(1-h)\), and since \(h\) and \(1-h\)
    are both nonzero, we conclude that \(E\) is decomposable.
\end{proof}

\begin{proposition}
    Consider a short exact sequence of vector bundles on a curve
    \[0 \longrightarrow F \longrightarrow E \longrightarrow G \longrightarrow 0\] Then \[\mu(F) > \mu(E) \quad \iff \quad \mu(E) > \mu(G)\] with equalities holding in the semistable case. In particular, (semi)stability of $E$ is witnessed by slopes of quotients as well as subbundles.
\end{proposition}

\begin{proof}
    One computes that \[\mu(E) = \frac{r_F}{r_F+r_G}\,\mu(F)
        + \frac{r_G}{r_F+r_G}\,\mu(G)\]
    In particular $\mu(E)$ is a convex combination of $\mu(F)$ and $\mu(G)$, it must lie between them. So the only possibilities are $\mu(F) < \mu(E) < \mu(G)$, or $\mu(G) < \mu(E) < \mu(F)$, or they are all equal.
\end{proof}


\begin{proposition}
    Let $E, E'$ be semistable vector bundles of the same rank and degree, and
    suppose that one of them is stable. Then every nonzero homomorphism
    between $E$ and $E'$ is an isomorphism.
\end{proposition}

\begin{proof}
    Let $r$ and $d$ be the common rank and degree of the two bundles, and let
    $f : E \to E'$ be a homomorphism with image $F \subset E'$.

    $F$ is not necessarily a subbundle of $E'$, but we can consider its saturation, which is a subbundle $\overline{F} \subset E'$. Note that $\deg \overline{F} \ge \deg F$ by the short exact sequence
    \[0 \longrightarrow F \longrightarrow \overline{F} \longrightarrow \text{torsion} \longrightarrow 0\] and note that degree is additive on short exact sequences.

    Since $\overline{F}$ is a subbundle of the semistable bundle $E'$, we have
    \[\mu(F) \leq \mu(\overline{F}) \le \mu(E') = \frac{d}{r}.\]

    Since $F$ is a quotient of $E$, we have
    \[\mu(F) = \frac{\deg F}{\mathrm{rank} F} \ge \frac{d}{r}\]

    So therefore $\mu(\overline{F}) = \mu(F)=d/r$. If $\operatorname{rank} F < r$, then $F$ is either a proper quotient bundle of $E$ with the same slope (which would contradict the stability of $E$) or would induce a proper subbundle $F \subset E'$ with the same slope (which would contradict the stability of $E'$). So this contradicts the stability of $E$ or $E'$, and hence $\operatorname{rank} F = r$. In particular, this means that the induced map
    \[
        f_{\mathrm{gen}} : E_{\mathrm{gen}} \longrightarrow E'_{\mathrm{gen}}
    \]
    is an isomorphism of vector spaces over $k(C)$, and so
    $\det(f_{\mathrm{gen}})$ is also an isomorphism. This implies that
    $\det f : \det E \to \det E'$ is injective, and since
    $\deg E = \deg E'$, it follows that $\det f$ is an isomorphism. Hence $f$ is an isomorphism.
\end{proof}

\begin{corollary}\label{cor:stable-implies-simple}
    Every stable vector bundle is simple.
\end{corollary}

\begin{proof}
    An endomorphism $f \in \End E$ induces, at each point $p \in C$, an
    endomorphism of the fiber $E/E(-p) \cong k^{\oplus r}$. Let $\alpha \in k$
    be an eigenvalue of this map, and consider $f - \alpha \cdot \mathrm{id}
        \in \End E$. This is not an isomorphism, so by the previous proposition it must be zero.
\end{proof}

We are going to study the semistability of Gieseker points associated to rank 2 vector bundles. For this, the central notion, with which we will build our semiinvariants, is that of the Pfaffian of a skew-symmetric matrix.

Let $\Alt_N(k)$ denote the space of $N\times N$ skew-symmetric matrices with entries in $k$. Thinking of a point $A \in \Alt_N(k)$ as a skew-symmetric bilinear form on $k^N$, there is an action of $\GL_N(k)$ on $\Alt_N(k)$ by change of basis: \[(g,A) \mapsto g A g^t.\] The orbits of this action are classified by the rank of the skew-symmetric form, which is always even. Because of this, the properties of the action
depend in an essential way on whether N is even or odd.


\begin{definition}[Pfaffian for even $N$]
    Let $N$ be an even integer. The Pfaffian of a skew-symmetric matrix $A \in \Alt_N(k)$ is defined as \[\mathrm{Pf}(A) = \frac{1}{2^{N/2}(N/2)!} \sum_{\sigma \in S_N} \mathrm{sgn}(\sigma) \prod_{i=1}^{N/2} a_{\sigma(2i-1),\sigma(2i)}\]
\end{definition}

\begin{proposition}[Properties of the Pfaffian]
    \leavevmode
    \begin{enumerate}
        \item $\mathrm{Pf}(A)^2 = \det(A)$ for all $A \in \Alt_N(k)$. In particular, $\Pf(A) \neq 0$ if and only if $A$ is nondegenerate.
        \item $\Pf(g A g^t) = \det(g) \Pf(A)$ for all $g \in \GL_N(k), A \in \Alt_N(k)$. In particular, the Pfaffian is a semiinvariant of weight $1$ (with respect to the character $\det: \GL_N(k) \to k^*$).
        \item For any $B \in \Alt_{N/2}(k)$ and $C\in \Mat_{N/2}(k)$, we have \[\Pf\begin{pmatrix}
                      0    & B \\
                      -B^T & C
                  \end{pmatrix} = (-1)^{N/2+1}\Pf(B)\]
    \end{enumerate}
\end{proposition}

For odd $N$, the determinant of any skew-symmetric matrix is zero, so we cannot define the Pfaffian as above. However, one can be clever and define a radical vector associated to any skew-symmetric matrix, which will play the role of the Pfaffian in this case. In particular we can define semiinvariants using the radical vector.
\begin{definition}[Radical for odd $N$]
    Let $N$ be an odd integer. For $A \in \Alt_N(k)$, the radical vector $\mathrm{rad}(A) \in k^N$ is \begin{align*}
        \mathrm{rad}(A)_i & = (-1)^{i+1} \Pf(A_{[i]}) \quad \text{for } i=1,\ldots,N
    \end{align*} where $A_{[i]}$ is the $(N-1)\times (N-1)$ skew-symmetric matrix obtained by deleting the $i$-th row and column from $A$.
\end{definition}

\begin{proposition}[Properties of the radical]\label{prop:radical}
    Consider $A \in \Alt_N(k)$.
    \begin{enumerate}[(i)]
        \item $\operatorname{rank} A \le N-1$, and
              $\operatorname{rank} A < N-1$ if and only if $\operatorname{rad} A = 0$.
        \item $A \cdot \operatorname{rad} A = 0$.
        \item If $X$ is an $N\times N$ matrix and $X^*$ is its matrix of cofactors, then
              \[
                  \operatorname{rad}(X A X^t) = X^{*,t}\, \operatorname{rad} A.
              \]
    \end{enumerate}
\end{proposition}

\subsection{Gieseker points}
Fix a line bundle $L$ and consider rank $2$ vector bundles $E$ with $\det E \cong L$. Fix the number $N = \chi(E) = h^0(E) - h^1(E) = \deg L + 2 -2g$. A set $S \subset H^0(E)$ of $N$ linearly independent global sections is called a marking of the vector bundle $E$, and the pair $(E, S)$ is called a marked vector bundle.

We will need the key properties that \begin{enumerate}
    \item $H^1(E) = 0$
    \item $E$ is generated by global sections, i.e. the evaluation map $H^0(E) \otimes \mathcal O_C \to E$ is surjective as a sheaf map.
\end{enumerate}
In the moduli story of line bundles, these conditions were guaranteed by taking the degree sufficiently large. However, this is not sufficient for arbitrary rank $2$ vector bundles.

However they are satisfied by semistable vector bundles of sufficiently large degree. In this case we have $N = h^0(E)$, and a marking
$S$ is a basis of $H^0(E)$. Moreover, generation by global sections means that the homomorphism
\[
    (s_1, \ldots, s_N) : \mathcal{O}_C^{\oplus N} \longrightarrow E,
    \qquad (f_1, \ldots, f_N) \longmapsto \sum_{i=1}^N f_i s_i,
    \tag{10.13}
\]
is surjective. At the same time, there is a homomorphism
\[
    (s_1 \wedge, \ldots, s_N \wedge) : E \longrightarrow (\det E)^{\oplus N},
    \qquad t \longmapsto (s_1 \wedge t, \ldots, s_N \wedge t),
    \tag{10.14}
\]
which, if $E$ is generated by global sections, is injective.
To explain this, recall that the stalk at the generic point
$E_{\mathrm{gen}}$ is a $2$-dimensional vector space over the function field $k(C)$,
so there is a skew-symmetric bilinear form
\[
    \wedge : E_{\mathrm{gen}} \times E_{\mathrm{gen}} \longrightarrow
    \det E_{\mathrm{gen}} \;\cong\; k(C).
    \tag{10.15}
\]

Thus $s \wedge s = 0$ and $s \wedge s' + s' \wedge s = 0$ for
$s, s' \in E_{\mathrm{gen}}$. Moreover, if $s, s'$ are global sections of $E$,
then $s \wedge s'$ is a global section of $\det E$, and so restriction
of~(10.15) defines a skew-symmetric $k$-bilinear map
\[
    H^0(E) \times H^0(E) \;\longrightarrow\; H^0(\det E),
    \qquad (s,s') \longmapsto s \wedge s'.
\]
The bilinear form~(10.15) induces an isomorphism
\[
    E_{\mathrm{gen}} \;\xrightarrow{\ \sim\ }\;
    \Hom(E_{\mathrm{gen}}, \det E_{\mathrm{gen}}).
\]
In particular, each global section $s \in H^0(E)$ determines a homomorphism
\[
    s \wedge : E \longrightarrow \det E,
    \qquad t \longmapsto s \wedge t.
\]

\begin{definition}[Gieseker point]
    Given a vector space $V$, we denote by $\Alt_N(V)$
    the set of skew-symmetric $N \times N$ matrices whose entries belong to $V$.
    Given a marked vector bundle $(E,S)$ with $\det E = L$,
    the skew-symmetric matrix
    \[
        T_{E,S}
        =
        \begin{pmatrix}
            s_1 \\ \vdots \\ s_N
        \end{pmatrix}
        \wedge (s_1, \ldots, s_N)
        =
        \begin{bmatrix}
            s_1 \wedge s_2     & s_1 \wedge s_3 & \cdots & s_1 \wedge s_N \\
            s_2 \wedge s_3     & \cdots         &        & s_2 \wedge s_N \\
            \vdots             &                & \ddots & \vdots         \\
            s_{N-1} \wedge s_N &                &        &
        \end{bmatrix}
        \in \Alt_N(H^0(L))
    \]
    will be called the \textbf{Gieseker matrix}, or \textbf{Gieseker point},
    of $E$ corresponding to the marking $S$.
\end{definition}

\begin{proposition}
    Given $S = \{s_1, \ldots, s_N\} \subset H^0(E)$, the composition of
    \textup{(10.13)} and \textup{(10.14)}
    \[
        \mathcal{O}_C^{\oplus N}
        \xrightarrow{(s_1, \ldots, s_N)}
        E
        \xrightarrow{(s_1 \wedge, \ldots, s_N \wedge)}
        L^{\oplus N}
    \]
    is given by the matrix $T_{E,S} \in \Alt_N(H^0(L))$.
\end{proposition}


Note that any matrix $T \in \Alt_N(H^0(L))$ determines a vector bundle map
\[
    \langle T \rangle : \mathcal{O}_C^{\oplus N} \longrightarrow L^{\oplus N},
\]
and this is skew-symmetric in the sense that the dual map
\[
    \langle T \rangle^t : (L^{-1})^{\oplus N} \longrightarrow \mathcal{O}_C^{\oplus N},
\]
after tensoring with $L$, is equal to $-\langle T \rangle$.


\begin{proposition}[Reconstruction from Gieseker point]\label{reconstruction-from-gieseker-point}
    Suppose that $H^1(E)=0$ and that $E$ is generated by global sections.
    Then, for any marking $S$, the bundle $E$ is isomorphic to the image of the homomorphism
    \[
        \langle T_{E,S} \rangle : \mathcal{O}_C^{\oplus N} \longrightarrow L^{\oplus N}
    \]
    defined by its Gieseker point.
\end{proposition}

\begin{proof}
    Consider the sequence
    \[
        \mathcal{O}_C^{\oplus N} \xrightarrow{\ \mathrm{ev}_S\ } E \xrightarrow{\ i_S\ } L^{\oplus N}
    \]
    where $\alpha = \mathrm{ev}_S$ is surjective (since $E$ is generated by global sections) and $\beta = i_S$ is injective (as explained above). Since $i_S$ is injective, we have $\ker(i_S \circ \mathrm{ev}_S) = \ker(\mathrm{ev}_S)$. Therefore,
    \[
        \operatorname{im}\langle T_{E,S}\rangle
        = \operatorname{im}(i_S \circ \mathrm{ev}_S)
        \cong \mathcal{O}_C^{\oplus N} / \ker(\mathrm{ev}_S)
        \cong \operatorname{im}(\mathrm{ev}_S) = E.
    \]
\end{proof}
We now consider the action
\[
    GL(N) \curvearrowright \Alt_N(H^0(L)),
    \qquad
    T \longmapsto XTX^t,
    \quad T \in \Alt_N(H^0(L)),\; X \in GL(N),
\]
where we view $\Alt_N(H^0(L))$ as an affine space $\A^n$,
with $n = h^0(L)\, N(N-1)/2$.

If we assume $H^1(E)=0$, so that the marking $S$ is a basis of $H^0(E)$, then the $GL(N)$-orbit of its Gieseker points depends only on the isomorphism class of $E$  and not on the choice of $S$.  Conversely, the vector bundle $E$ can be recovered from any Gieseker point by Proposition~\ref{reconstruction-from-gieseker-point}, so we have the following.

\begin{corollary}
    The mapping
    \[
        \left\{
        \begin{array}{c}
            \text{isomorphism classes of vector bundles $E$ with $H^1(E)=0$} \\[3pt]
            \text{and generated by global sections}
        \end{array}
        \right\}
        \longrightarrow
        \left\{
        \begin{array}{c}
            GL(N)\text{-orbits} \\[3pt]
            \text{in }\Alt_N(H^0(L))
        \end{array}
        \right\}
    \]
    sending $E$ to the orbit of its Gieseker points $T_{E,S}$ is injective.
\end{corollary}

\subsection{Semistability of Gieseker points}

We now need to consider the question of (semi)stability of a point
$T \in \Alt_N(H^0(L))$ under the action of $GL(N)$, with respect to the determinant
character $g \mapsto \det g$.
We will show that if $E$ is a rank $2$ vector bundle with $H^1(E)=0$
and $\deg E \ge 4g - 2$, then the Gieseker points $T_{E,S}$ are semistable
if and only if $E$ is slope-semistable as a vector bundle.
Conversely, we will see that if $\deg L \ge 4g - 2$,
then every semistable $T \in \Alt_N(H^0(L))$ is a Gieseker point of a semistable vector bundle.

\begin{definition}[Gieseker semistability]
    A \textbf{semiinvariant} of weight $w$ is a polynomial function
    \[F = F(T) \in k[\Alt_N(H^0(L))]\] with the property
    \[
        F(g \cdot T) = (\det g)^w F(T),
        \qquad \text{for all } g \in GL(N),
    \]
    and the unstable set in $\Alt_N(H^0(L))$ is the common zero-set
    of all semiinvariants of positive weight. In particular, if there exists a semiinvariant $F$ of positive weight with $F(T) \ne 0$, then $T$ is semistable.
\end{definition}


Recall that a point $T$ is unstable if and only if the closure of its
$SL(N)$-orbit contains the origin.
A “Gieseker point” $\Psi(\xi, S, T)$ of a line bundle $\xi$ is always stable. However, for vector bundles this is no longer the case. For rank greater than $1$ the following phenomenon appears.

\begin{proposition}
    Let $S$ be a marking and $M \subset E$ a line subbundle of the vector bundle $E$,
    and consider the vector subspaces
    $\langle S \rangle \subset H^0(E)$ (of dimension $N$)
    and $H^0(M) \subset H^0(E)$.
    \begin{enumerate}[(i)]
        \item If there exists $M \subset E$ such that
              \[
                  \dim\bigl(H^0(M) \cap \langle S \rangle\bigr) > \frac{N}{2},
              \]
              then the Gieseker point $T_{E,S} \in \Alt_N(H^0(L))$ is unstable.

        \item If there exists $M \subset E$ such that
              \[
                  \dim\bigl(H^0(M) \cap \langle S \rangle\bigr) \ge \frac{N}{2},
              \]
              then $T_{E,S} \in \Alt_N(H^0(L))$ fails to be stable.
    \end{enumerate}
\end{proposition}

\begin{proof}
    The strategy goes to pick a basis aligned with the subspace $H^0(M)\subset H^0(E)$, then use the Hilbert-Mumford criterion.

    Let
    \[
        a = \dim\big(H^0(M) \cap \langle S \rangle\big), \qquad b = N - a.
    \]
    Reorder $S = (s_1, \dots, s_N)$ so that $s_1, \dots, s_a \in H^0(M)$. Since $M$ is a line bundle, we have $s_i \wedge s_j = 0$ for $1 \leq i, j \leq a$. Therefore, the Gieseker matrix $T_{E,S}$ has the block form
    \[
        T_{E,S} =
        \begin{pmatrix}
            0      & B \\[2pt]
            -B^{t} & C
        \end{pmatrix}
    \]
    where the blocks have sizes $a \times a$, $a \times b$, and $b \times b$ respectively. Take
    \[
        g(t)=\diag(t^{-b} I_a,\; t^{a} I_b)\in SL(N) \quad(\det g(t)=t^{-ab}\cdot t^{ab}=1).
    \]
    Since the action is $T \mapsto g T g^{t}$, we have
    \[
        g(t) T_{E,S} g(t)^{t} =
        \begin{pmatrix}
            0                & t^{a-b} B \\[2pt]
            -\,t^{a-b} B^{t} & t^{2a} C
        \end{pmatrix}.
    \]


    If $a > b$, the exponents $a-b > 0$ and $2a > 0$, so as $t \to 0$ the right-hand side tends to the zero matrix. Thus, $0$ lies in the closure of the $\mathrm{SL}(N)$-orbit of $T_{E,S}$. By the standard GIT criterion (the orbit closure contains the origin), $T_{E,S}$ is unstable.


    If $a=b$, then the right hand side tends to
    \[
        T_0 = \begin{pmatrix} 0 & B \\ -B^{t} & 0 \end{pmatrix}.
    \]
    The $1$-parameter subgroup
    \[
        \lambda \mapsto \operatorname{diag}(\lambda I_a,\, \lambda^{-1} I_b) \subset \mathrm{SL}(N)
    \]
    stabilizes $T_0$, because
    \[
        \operatorname{diag}(\lambda I_a,\, \lambda^{-1} I_b)\, T_0\, \operatorname{diag}(\lambda I_a,\, \lambda^{-1} I_b)^{t} = T_0,
    \]
    Recall that a point $x\in X$ is stable iff $x$ is semistable, the orbit $G\cdot x$ is closed in $X^{ss}$, and $\operatorname{Stab}_G(x)$ is finite. Hence, $T_{E,S}$ is not stable because either its orbit is not closed (if $T_0$ is not in the orbit of $T_{E,S}$), or if $T_0$ is in the orbit of $T_{E,S}$, in which case $T_0 = T_{E,S}$ has a positive-dimensional stabilizer (the $1$-parameter subgroup above).
\end{proof}


This phenomenon motivates the following definition.
\begin{definition}
    Let $E$ be a rank $2$ vector bundle.
    If
    \[
        h^0(M) \le \tfrac{1}{2} h^0(E) \quad (\text{resp. } <)
    \]
    for every line subbundle $M \subset E$, then we say that $E$ is
    \textbf{$H^0$-semistable} (resp.\ \textbf{$H^0$-stable}).
\end{definition}

The following corollary is an immediate consequence of the previous proposition.
\begin{corollary}[Gieseker semistability implies $H^0$-semistability]\label{cor:gieseker-stab-implies-h0-stab}
    Suppose that $H^1(E)=0$. Then $N = h^0(E)$ and $S$ is a basis of $H^0(E)$. Let $T = T_{E,S}$ be any Gieseker point of $E$.
    Then:
    \begin{enumerate}[(i)]
        \item If $T$ is $GL(N)$-semistable, then $E$ is $H^0$-semistable;
        \item If $T$ is $GL(N)$-stable, then $E$ is $H^0$-stable.
    \end{enumerate}
\end{corollary}



\begin{proposition}[Equivalence of $H^0$-semistability and slope semistability]\label{prop:h0-semistability-equivalence-slope-semistability}
    Suppose that $H^1(E)=0$ and $\deg E \ge 4g - 2$.
    Then $E$ is $H^0$-semistable if and only if it is slope-semistable.
\end{proposition}

\begin{proof}
    First observe that by Riemann--Roch any line bundle $M$ satisfies
    \[
        h^0(M) - h^1(M) - \frac{h^0(E) - h^1(E)}{2}
        = \deg M - \frac{\deg E}{2}.
    \]
    Since $H^1(E)=0$, this implies
    \[
        \frac{h^0(E)}{2} - h^0(M)
        \le \Big(\frac{h^0(E)}{2} - h^0(M)\Big) + h^1(M)
        = \frac{\deg E}{2} - \deg M.
        \tag{stability-inequality}
    \]
    Letting $M$ run through the line subbundles of $E$,
    this shows at once that $H^0$-semistability of $E$
    implies slope-semistability ($H^0$-semistability means that the left-hand side of~(stability-inequality) is nonnegative for all $M$).

    For the converse, suppose that there exists a line subbundle
    $M \subset E$ for which the left-hand side of~(stability-inequality) is negative. Note that, by hypothesis,
    \[
        h^0(E) = \deg E + 2 - 2g \;\ge\; 2g,
    \]
    and therefore $h^0(M) > \tfrac{1}{2}h^0(E) \ge g$.

    This implies that $H^1(M) = 0$. Indeed, suppose $h^1(M) > 0$. By Serre duality, this means there exists a nonzero section $s \in H^0(K \otimes M^{-1})$, where $K$ is the canonical bundle of $C$. Multiplication by $s$ gives an injective sheaf map $M \hookrightarrow K$, and thus an injective map on global sections $H^0(M) \hookrightarrow H^0(K)$. Therefore, $h^0(M) \leq h^0(K) = g$. Now take the contrapositive to conclude that $h^0(M) > g$ implies $h^1(M) = 0$.

    Thus we get the equality
    \[\frac{h^0(E)}{2} - h^0(M)= \Big(\tfrac{h^0(E)}{2}-h^0(M)\Big)+h^1(M)
        =\frac{\deg E}{2}-\deg M\]
    and the left hand side is negative and so we get the failure of slope-semistability.
\end{proof}

\begin{proposition}[$H^0$-semistability implies Gieseker semistability]\label{h0-semistability-implies-gieseker-semistability}
    Suppose that $H^1(E)=0$. Then, if the vector bundle $E$ is
    $H^0$-semistable, its Gieseker points $T_{E,S} \in \Alt_N(H^0(L))$
    are semistable for the action of $GL(N)$.
\end{proposition}

A quotient line bundle $Q = E/M$ of an $H^0$-semistable vector bundle $E$ satisfies
\[
    h^0(Q) \ge h^0(E) - h^0(M) \ge \tfrac{1}{2}h^0(E).
\]

\begin{lemma}[Global generation]\label{h0-semistable-generated}
    If $E$ is $H^0$-semistable and $h^0(E)\ge 2$,
    then $E$ is generated by global sections at a general point
    $p \in C$. In particular,
    \[
        h^0(E(-p)) = h^0(E) - 2
    \]
    at the general point.
\end{lemma}

\begin{proof}
    Consider the evaluation homomorphism
    \[
        H^0(E) \otimes \cO_C \longrightarrow E.
    \]
    The image sheaf has $h^0(E)$ linearly independent sections;
    if it had rank~$1$, then its saturation would be a line bundle
    violating $H^0$-semistability. So the image has rank~$2$.


    The image sheaf cannot have rank $1$. If $\operatorname{rk} I = 1$, then $I$ is a torsion-free sheaf of rank $1$, so its saturation $M := I^{\mathrm{sat}} \subset E$ is a line subbundle. Since $I \subset M$, the map $H^0(I) \hookrightarrow H^0(M)$ is injective, so
    \[
        h^0(M) \geq h^0(I) = h^0(E).
    \]
    But $H^0$-semistability requires $h^0(M) \leq \tfrac{1}{2} h^0(E)$ for every line subbundle $M \subset E$, so this is a contradiction. Therefore, $\operatorname{rk} I \neq 1$. It follows that $\operatorname{rk} I = 2$, so $I = E$ at the generic point; that is, $\mathrm{ev}$ is generically surjective. For a general point $p \in C$, the evaluation map
    \[
        \mathrm{ev}_p: H^0(E) \to E|_p
    \]
    is surjective. From the exact sequence
    \[
        0 \to E(-p) \to E \to E|_p \to 0
    \]
    we obtain
    \[
        0 \to H^0(E(-p)) \to H^0(E) \xrightarrow{\mathrm{ev}_p} E|_p \to 0,
    \]
    so $\dim E|_p = 2$ implies $h^0(E(-p)) = h^0(E) - 2$.
\end{proof}

\begin{remark}
    [Saturation]
    The image of the evaluation map
    \[
        \mathrm{ev}:H^0(E)\otimes\O_C\longrightarrow E
    \]
    is a coherent subsheaf $I\subset E$. It need not be a subbundle because a subbundle means locally free subsheaf with torsion-free (equivalently locally free on a curve) quotient. The quotient $E/I$ can have torsion at the base locus of the sections.

    On a smooth curve, $I$ is torsion-free (subsheaf of a vector bundle), hence locally free. But $E/I$ may have zero-dimensional torsion, so $I$ is not saturated, hence not a subbundle.

    On $\P^1$ consider the vector bundle $E = \mathcal{O} \oplus \mathcal{O}(1)$, consider a single section $s \in H^0(\mathcal{O}(1))$ with $\operatorname{div}(s) = p$. The evaluation by the two sections $(1,0)$ and $(0,s)$ gives a map
    \[
        \mathcal{O}_C^2 \longrightarrow \mathcal{O} \oplus \mathcal{O}(1), \quad (f,g) \mapsto (f,\, g s).
    \]
    Hence the image is
    \[
        I = \mathcal{O} \oplus (s \cdot \mathcal{O}_C) \subset \mathcal{O} \oplus \mathcal{O}(1).
    \]
    Locally at $p$, trivialize $\mathcal{O}(1)$ so that $s = t\,e$ with $t \in m_p$ a uniformizer. Then
    \[
        s \cdot \mathcal{O}_{C,p} = t\, \mathcal{O}_{C,p} \cdot e = m_p\, e,
    \]
    so globally $s \cdot \mathcal{O}_C = \mathcal{I}_p \otimes \mathcal{O}(1) = \mathcal{O}(1)(-p)$. Thus,
    \[
        I = \mathcal{O} \oplus \mathcal{O}(1)(-p),
    \]
    and the quotient is
    \[
        E/I \cong \mathcal{O}(1)/\mathcal{O}(1)(-p) \cong \mathcal{O}_p \cong k(p),
    \]
    a torsion skyscraper sheaf. The inclusion $I \hookrightarrow E$ is therefore not a subbundle inclusion (the quotient is not locally free), even though $I$ itself is locally free.
\end{remark}

\begin{lemma}[Semistability after twist]\label{h0-semistability-after-twist}
    If $E$ is $H^0$-semistable and $h^0(E) \ge 4$ then there exists a point $p \in C$ such that the vector bundle $E(-p)$ is $H^0$-semistable.
\end{lemma}

\begin{proof}
    Let $h^0(E)=n$.
    At a general point $p\in C$ we have
    $h^0(E(-p)) = n-2$ by Lemma \ref{h0-semistable-generated}.
    We suppose that at every point the bundle $E(-p)$ is $H^0$ unstable and therefore contains some line subbundle,
    which we denote by $M^p(-p)\subset E(-p)$, with
    \[
        h^0(M^p(-p)) > \frac{n}{2} - 1.
    \]

    \textbf{Claim.} The line subbundle $M^p\subset E$ is independent of the choice of the general point $p\in C$.

    Granted the claim, we have a line subbundle
    $M(=M^p)\subset E$ which satisfies
    \[
        h^0(M(-p)) > \frac{n}{2} - 1
    \]
    at a general point of the curve.
    But this implies $h^0(M) > n/2$ (using the generality of $p$ since every line bundle $M$ has finitely many points where every global section vanishes). This contradicts the $H^0$-semistability of $E$, and we are done.

    To prove the claim, let $q\in C$ be another, distinct, point.
    We first consider the case $n\ge5$.
    Then
    \[
        h^0(E(-p)) = n-2 > \frac{n}{2},
    \]
    and this implies that $E(-p)$ is generically generated by global sections
    (otherwise we would get a line subbundle of $E(-p)\subset E$
    violating the $H^0$-semistability of $E$).
    Hence $h^0(E(-p-q)) = n-4$.
    On the other hand,
    \[
        h^0(M^p(-p-q)) + h^0(M^q(-p-q))
        = h^0(M^p(-p)) - 1 + h^0(M^q(-q))
        > n - 4.
    \]
    This implies that
    \[
        0 \ne H^0(M^p(-p-q)) \cap H^0(M^q(-p-q))
        \subset H^0(E(-p-q)),
    \]
    and hence the line subbundles
    $M^p(-p-q)$ and $M^q(-p-q)\subset E(-p-q)$ coincide.
    Hence $M^p = M^q$.

    Now consider the case $n=4$.
    We have $h^0(E(-p))=2$ and $h^0(M^p(-p))\ge2$,
    so $H^0(E(-p))=H^0(M^p(-p))$.
    In particular,
    \[
        H^0(E(-p-q)) = H^0(M^p(-p-q)) \cong k.
    \]
    Similarly,
    \[
        H^0(E(-p-q)) = H^0(M^q(-p-q)) \cong k.
    \]
    So again the two line subbundles
    $M^p(-p-q)$ and $M^q(-p-q)\subset E(-p-q)$
    have a common global section, and they therefore coincide.
\end{proof}

\begin{proof}
    [Proof of Proposition \ref{h0-semistability-implies-gieseker-semistability}]

    To show semistability of a Gieseker point $T_{E,S}$ we have to exhibit
    a semiinvariant of positive weight which is nonzero at $T_{E,S}$.
    We consider separately the cases when $N$ is even or odd
    (note that $N \equiv \deg L \pmod 2$).

    \smallskip
    When $N$ is even we can construct semiinvariants as follows.
    For any linear form $f : H^0(L) \to k$, we can evaluate $f$ on the entries
    of a matrix $T \in \Alt_N(H^0(L))$ to obtain a skew--symmetric matrix
    $f(T) \in \Alt_N(k)$. The function
    \[
        \Alt_N(H^0(L)) \;\longrightarrow\; k, \qquad
        T \longmapsto \operatorname{Pfaff}\big(f(T)\big)
    \]
    is a semiinvariant of weight~$1$.

    \smallskip
    By repeated use of Lemmas \ref{h0-semistability-after-twist} and \ref{h0-semistable-generated} we can find points
    $p_1,\ldots,p_{N/2}\in C$ such that
    \begin{equation}
        H^0\big(E(-p_1-\cdots-p_{N/2})\big)=0.
        \tag{10.17}
    \end{equation}
    If we let $\mathrm{ev}_i=\mathrm{ev}_{p_i}:H^0(L)\to k$ be evaluation at
    the $i$-th point, then the above equation says that the linear map of
    $N$--dimensional vector spaces
    \[
        g := (\mathrm{ev}_1,\ldots,\mathrm{ev}_{N/2}) :
        H^0(E) \;\longrightarrow\;
        \bigoplus_{i=1}^{N/2} E/E(-p_i)
    \]
    is an isomorphism. Now consider the skew--symmetric pairing
    \[
        H^0(E)\times H^0(E)
        \;\xrightarrow{\;\wedge\;}\;
        H^0(L)
        \;\xrightarrow{\;f\;}\;
        k,
    \]
    where $f := \mathrm{ev}_1+\cdots+\mathrm{ev}_{N/2}:H^0(L)\to k$.
    This pairing has matrix $f(T_{E,S})$ and transforms, via the isomorphism
    $g$, to a skew--pairing $k^N\times k^N\to k$ with matrix
    \[
        \begin{pmatrix}
            0        & I_{N/2} \\[2pt]
            -I_{N/2} & 0
        \end{pmatrix}.
    \]
    In other words, there is a commutative diagram
    \begin{equation}
        \begin{tikzcd}[column sep=large]
            H^0(E)\times H^0(E)
            \arrow[d,"g\times g"']
            \arrow[r,"\wedge"]
            &
            H^0(L)
            \arrow[d,"f"]
            \\
            k^N\times k^N
            \arrow[r]
            &
            k
        \end{tikzcd}
        \tag{10.18}
    \end{equation}General theory of Pfaffians now shows that
    $\operatorname{Pfaff}\big(f(T_{E,S})\big)$ is equal to
    $\det g \neq 0$. Hence the Gieseker point $T_{E,S}\in\Alt_N(H^0(L))$ is semistable.

    We turn now to the case when $N$ is odd. In this case the strategy for
    producing semiinvariants is to use triples of linear forms
    \[
        f,\, f',\, h : H^0(L) \longrightarrow k.
    \]
    From these and from $T \in \Alt_N(H^0(L))$ we get vectors
    $\operatorname{rad} f(T),\, \operatorname{rad} f'(T) \in k^N$
    and a skew--symmetric matrix $h(T) \in \Alt_N(k)$ by applying the linear functionals entrywise.
    We then form the scalar product
    \[
        \Alt_N(H^0(L)) \longrightarrow k, \qquad
        T \longmapsto (\operatorname{rad} f(T))^t\,h(T)\,\operatorname{rad} f'(T).
    \]
    This is a semiinvariant of weight~$2$.

    Pick $N$ distinct points
    $p_1,\ldots,p_N \in C$ and let $f_i : H^0(L) \to k$
    be evaluation at the point $p_i$. Now take $n = (N-1)/2$ and let $f = f_1 + \cdots + f_n$ and
    $f' = f_{n+1} + \cdots + f_{2n}$ and $h = f_N$.
    \begin{lemma}
        If $E$ is $H^0$--semistable, then there exist points
        $p_1,\ldots,p_N \in C$ such that, for any marking
        $S \subset H^0(E)$,
        \[
            (\operatorname{rad} f(T_{E,S}))^t
            \,h(T_{E,S})\,
            \operatorname{rad} f'(T_{E,S}) \neq 0,
        \]
        where $f,f',h$ are defined as above.
    \end{lemma}

    \begin{proof}
        Let $n := (N-1)/2$.
        The function $f : H^0(L) \to k$ is the sum of the evaluation maps
        at the points $p_1,\ldots,p_n \in C$, and moreover
        \[
            \operatorname{rad} f(T_{S,E}) \neq 0
            \quad\Longleftrightarrow\quad
            h^0\big(E(-p_1-\cdots-p_n)\big) = 1.
        \]

        To see this let $V := H^0(E)$ and choose a basis $S$. For points $p_1,\dots,p_n$ set
        \[
            g=(\mathrm{ev}_{p_1},\ldots,\mathrm{ev}_{p_n}): V \longrightarrow
            W:=\bigoplus_{k=1}^n E|_{p_k},
        \]
        where $\mathrm{ev}_{p_k}$ is evaluation at $p_k$ and $E|_{p_k}$ denotes the fiber of $E$ at $p_k$. On each fiber $E|_{p_k}$ the wedge pairing $E|_{p_k}\wedge E|_{p_k}\to L|_{p_k}$
        gives a nondegenerate skew form $\omega_k$ after identifying $L|_{p_k}\cong k$.
        Write $\omega:=\omega_1\oplus\cdots\oplus\omega_n$ for the induced skew form on $W$. Define a skew form on $V$ by
        \[
            \beta_f(u,v):=f(u\wedge v)=\sum_{k=1}^n \omega_k\big(u(p_k),v(p_k)\big)
            =\omega\big(g(u),g(v)\big).
        \]
        Relative to the basis $S$, the matrix of $\beta_f$ is exactly $f(T_{S,E})$.

        Since $\omega$ is nondegenerate on $W$, we have $\ker\beta_f=\ker g$ and
        \[
            \ker g = H^0\big(E(-p_1-\cdots-p_n)\big),
        \]
        the space of sections vanishing at all the $p_k$. Hence
        \[
            \dim\ker\beta_f = h^0\big(E(-p_1-\cdots-p_n)\big).
        \]
        By Proposition \ref{prop:radical} for an odd-size skew matrix $A$ the radical vector $\operatorname{rad}A$ is nonzero precisely when $\ker A$ is one-dimensional.
        Therefore
        \[
            \operatorname{rad} f(T_{S,E})\neq 0
            \quad\Longleftrightarrow\quad
            \dim\ker\beta_f=1
            \quad\Longleftrightarrow\quad
            h^0\big(E(-p_1-\cdots-p_n)\big)=1.
        \]


        Moreover if these equivalent
        conditions hold, then the vector $\operatorname{rad} f(T_{S,E})$ spans the
        $1$--dimensional space
        \[
            H^0\big(E(-p_1-\cdots-p_n)\big)
            = \ker\big((\mathrm{ev}_1,\ldots,\mathrm{ev}_n):
            H^0(E) \to k^{2n}\big),
        \]
        relative to the basis $S \subset H^0(E)$.

        Now by repeated use of Lemma \ref{h0-semistability-after-twist}
        we can find points
        $p_1,\ldots,p_{n-1} \in C$
        such that $E(-p_1-\cdots-p_{n-1})$ is $H^0$--semistable and
        $h^0(E(-p_1-\cdots-p_{n-1})) = 3$.


        For a general point $q\in C$ the evaluation map
        $\mathrm{ev}_q: H^0(E') \to E'|_q$ is surjective. Since $\dim H^0(E')=3$ and
        $\dim E'|_q=2$, the kernel is one–dimensional, namely
        \[
            \ker(\mathrm{ev}_q)=H^0\bigl(E'(-q)\bigr).
        \]

        Choose a general $p_n$. Then $H^0\bigl(E'(-p_n)\bigr)$ is a $1$-dimensional space;
        take a generator $s\neq 0$, so $s(p_n)=0$. The zero locus of $s$ is finite,
        hence we may choose a general point $p_{n+1}$ with $s(p_{n+1})\neq 0$.
        For such $p_{n+1}$ the space $H^0\bigl(E'(-p_{n+1})\bigr)$ is again one-dimensional;
        choose a generator $t\neq 0$, so $t(p_{n+1})=0$. Because $s(p_{n+1})\neq 0$,
        the sections $s$ and $t$ are not proportional.

        Thus we obtain sections $s,t\in H^0(E')$ with $s(p_n)=0$, $t(p_{n+1})=0$,
        and $s,t$ linearly independent. Moreover, by $H^0$-semistability, two
        independent global sections cannot land in a rank-$1$ subsheaf (otherwise
        its saturation would be a line subbundle with $h^0\ge 2 > \tfrac12 h^0(E')$),
        so $s$ and $t$ generate a rank-$2$ subsheaf; in particular, at a general point
        the fibre is spanned by values of global sections.

        These sections are necessarily linearly independent and, by
        $H^0$--semistability, generate a subsheaf of rank~$2$.
        Thus if $p_n$ is general, the fibre at this point will be generated by global sections.

        Finally we check that with respect to the $N$ points
        \[
            p_1,\ldots,p_{n-1},p_n,p_{n+1},p_1,\ldots,p_{n-1},p_n,
        \]
        the scalar product of the lemma is nonzero. By our choice of $h = \mathrm{ev}_{p_N} = \mathrm{ev}_{p_n}$, we have
        \[
            (\operatorname{rad} f)^t\,h(T_{E,S})\,\operatorname{rad} f’
            = \beta_h(s,t) = (s\wedge t)(p_n)\in k.
        \]
        But $s(p_n)$ and $t(p_n)$ are linearly independent in $E|_{p_n}$,
        so $(s\wedge t)(p_n)\neq 0$. This concludes the proof of the lemma.

    \end{proof}
    We have shown that for both even and odd $N$ there exist semiinvariants of positive weight
    which are nonzero at the Gieseker point $T_{E,S}$. Hence $T_{E,S}$ is semistable. This concludes the proof of Proposition \ref{h0-semistability-implies-gieseker-semistability}.
\end{proof}

\subsection{Construction of the moduli space}
We are now prepared to prove Theorem \ref{thm:moduli-of-rank-2-vb-with-fixed-det} For this we need to study the
$GL(N)$--orbits in the affine space $\Alt_N(H^0(L))$ coming from vector
bundles via Corollary~10.62.

By identifying $L \cong \mathcal{O}_C(D)$ for some divisor $D \in \Div C$ we
can view elements $T \in \Alt_N(H^0(L))$ as skew--symmetric matrices with
entries in the function field $k(C)$; we then observe that the Gieseker
points $T_{E,S}$, as matrices over $k(C)$, have rank $2$. This is because $T$ is given by the composition $\cO_C^{\oplus N} \to E \to L^{\oplus N}$, and passing to the generic point $\eta$ of $C$, we see that the matrix $T: \mathcal{O}_{C,\eta}^{\oplus N} \cong k(C)^N \to L_\eta \cong k(C)$ has rank $2$ over $k(C)$ since $E_\eta$ is a $2$-dimensional vector space over $k(C)$. Geometrically, this is saying that the wedge product $E\otimes E\to \det E$ is a nondegenerate alternating form on a 2-dimensional space.

\begin{definition}
    The set of matrices $T \in \Alt_N(H^0(L))$ of rank $\le 2$ over $k(C)$
    is a closed subvariety which we denote by $\Alt_{N,2}(H^0(L)) \subset
        \Alt_N(H^0(L))$
\end{definition}

Let $x_{ij}^{(\alpha)}$, for $1 \le i,j \le N$ and $1 \le \alpha \le h^0(L)$,
be coordinates in the affine space $\Alt_N(H^0(L))$. Then
\[
    \Alt_{N,2}(H^0(L)) \subset \Alt_N(H^0(L))
\]
is defined by
\[
    \binom{N}{4} \, h^0(L^2)
\]
equations determined by the vanishing of global sections
\[
    \operatorname{Pfaff}
    \begin{bmatrix}
        x_{ij} & x_{ik} & x_{il} \\
        x_{jk} & x_{jl}          \\
               & x_{kl}
    \end{bmatrix}
    \;=\;
    x_{ij} \circ x_{kl}
    \;-\;
    x_{ik} \circ x_{jl}
    \;+\;
    x_{il} \circ x_{jk}
    \;\in\; H^0(L^2),
\]
for $1 \le i < j < k < l \le N$, and where
\[
    x_{ij} := (x_{ij}^{(\alpha)})_{1 \le \alpha \le h^0(L)} \in H^0(L),
\]
and $\circ : H^0(L) \times H^0(L) \to H^0(L^2)$ is the natural
multiplication map.

If $T \ne 0$, then the rank condition is equivalent to saying that the
image $E$ of the sheaf homomorphism
\[
    \langle T \rangle : \mathcal{O}_C^{\oplus N} \longrightarrow L^{\oplus N}
\]
is a rank $2$ vector bundle.

\begin{proposition}[Smoothness at Gieseker points]
    \label{prop:smoothness-at-gieseker-points}
    Let $E$ be a rank $2$ vector bundle with $\det E = L$ and $H^1(E)=0$.
    Then:
    \begin{enumerate}[(i)]
        \item $\Alt_{N,2}(H^0(L))$ is smooth at each Gieseker point $T_{E,S}$.
        \item If $E$ is simple, then the quotient of the tangent space to
              $\Alt_{N,2}(H^0(L))$ at a Gieseker point $T_{E,S}$ by the Lie space
              $\mathfrak{gl}(N)$ is isomorphic to $H^1(\mathfrak{sl}\,E)$:
              \[
                  T_{T_{E,S}}\Alt_{N,2}(H^0(L)) / \mathfrak{gl}(N)
                  \;\cong\;
                  H^1(\mathfrak{sl}\,E).
              \]
    \end{enumerate}
\end{proposition}

Given vector spaces $U, V$, the space $\Hom(U,V)$ of linear maps
$f : U \to V$ can be viewed as an affine space. For each natural number
$r$, there is then a subset $\Hom_r(U,V) \subset \Hom(U,V)$ consisting
of linear maps of rank $\le r$, defined as a closed subvariety by the
vanishing of all the $(r+1)\times (r+1)$ minors.

\begin{lemma}[Tangent space to rank varieties]
    Suppose that $f \in \Hom(U,V)$ has rank exactly equal to $r$.
    Then the tangent space to $\Hom_r(U,V)$ at $f$ is equal to
    \[
        S_f := \{\, h \mid h(\ker f) \subset \im f \,\} \subset \Hom(U,V).
    \]
\end{lemma}

\begin{proof}
    Choose bases of $U$ and $V$ so that the matrix representing
    $f : U \to V$ is in canonical form
    \[
        \mathrm{diag}(1,\ldots,1,0,\ldots,0).
    \]
    If $h : U \to V$ is another linear map, then $f + \varepsilon h$,
    where $\varepsilon^2 = 0$, $\varepsilon \ne 0$, is represented by a
    matrix
    \[
        \begin{pmatrix}
            I_r & 0 \\[4pt]
            0   & 0
        \end{pmatrix}
        \;+\;
        \varepsilon
        \begin{pmatrix}
            A & B \\[4pt]
            C & D
        \end{pmatrix}.
    \]
    Since $\varepsilon^2 = 0$, the only possible nonzero $(r+1)\times(r+1)$
    minors in this matrix are the entries of $D$ (concatenated with the
    block $I_r$). Hence the condition that all $(r+1)\times(r+1)$ minors
    vanish is equivalent to $D = 0$. But this is the case if and only if
    $h(\ker f) \subset \im f$.
\end{proof}




In the tangent vector space $S_f$ there are two vector subspaces to consider.
One consists of $h$ satisfying $h(\ker f)=0$, which is equivalent to factoring
through an element of $\Hom(\im f, V)$. The other consists of $h$ satisfying
$h(U)\subset \im f$, or, in other words, $h$ comes from an element of
$\Hom(U,\im f)$. The intersection consists of endomorphisms of $\im f$, and
in this way we obtain an exact sequence of vector spaces:
\begin{equation}
    0 \;\longrightarrow\; \End(\im f)
    \;\longrightarrow\; \Hom(\im f, V) \oplus \Hom(U,\im f)
    \;\longrightarrow\; S_f \;\longrightarrow\; 0.
    \tag{10.19}
\end{equation}

Now suppose that $V = U^\vee$, and consider the subset
$\Hom^- (U,U^\vee)$ of skew--symmetric linear maps: those
$f : U \to U^\vee$ that is, equal to minus their transpose (dual) map.
Suppose that $f \in \Hom^-(U,U^\vee)$ has rank $\le r$. This means that all
its $(r+2)\times (r+2)$ Pfaffian minors vanish, and these Pfaffians define a
closed subvariety $\Hom^-_r(U,U^\vee) \subset \Hom^-(U,U^\vee)$. The same argument as above gives the following lemma:

\begin{lemma}[Tangent space to skew--symmetric rank varieties]\label{lem:tangent-skew}
    Suppose that $f : U \to U^\vee$ is skew--symmetric and has rank equal to $r$.
    Then the tangent space to $\Hom^-_r(U,U^\vee)$ at $f$ is equal to
    \[
        S_f^- := \{\, h \mid h(\ker f)\subset \im f\,\} \subset \Hom^-(U,U^\vee).
    \]
\end{lemma}

The two subspaces $\{\,h \mid h(\ker f)=0\,\}$ and $\{\,h \mid h(U)\subset \im f\,\}$,
when the maps $f,h$ are skew--symmetric, are exchanged by taking the
transpose; moreover, the intersection
\[
    \{\,h \mid h(\ker f)=0\,\}\;\cap\;\{\,h \mid h(U)\subset \im f\,\}
    \;\cap\; \Hom^-(U,U^\vee)
\]
is exactly the space of endomorphisms of $\im f$ which preserve a skew--
symmetric form. We will denote this space by $\End^-(\im f)$. From (10.19) we
obtain an exact sequence:
\begin{equation}
    0 \;\longrightarrow\; \End^-(\im f)
    \;\longrightarrow\; \Hom(U,\im f)
    \;\longrightarrow\; S_f^- \;\longrightarrow\; 0.
    \tag{10.20}
\end{equation}
\begin{remark}
    When $r=2$, the space of endomorphisms preserving a skew-symmetric form is isomorphic to the special linear Lie algebra $\mathfrak{sl}_2(k)$.

    Suppose $g \in \GL(V)$ preserves a nondegenerate skew-symmetric form $\omega:V\times V\to k$. Then $g\in \Sp(V,\omega) = \{g\in \GL(V) \mid g^t J g = J\}$ where $J$ is the matrix of $\omega$ in some basis.

    Taking the derivative at the identity, we find that the Lie algebra \[\mathfrak{sp}(V,\omega) = \{X\in \End(V) \mid X^t J + J X = 0\}\] When $\dim V = 2$, we can take $J = \begin{pmatrix}0 & 1 \\ -1 & 0\end{pmatrix}$ and the condition becomes $X^t J + J X = 0$ which is equivalent to $\tr(X) = 0$. Thus $\mathfrak{sp}(2,k) \cong \mathfrak{sl}_2(k)$.
\end{remark}

We will need the following functorial characterization of smoothness.
\begin{definition}
    An Artin ring over $k$ is a finitely generated ring containing
    $k$ which satisfies the following equivalent conditions.
    \begin{enumerate}
        \item $R$ is finite-dimensional as a vector space over $k$.
        \item $R$ has only finitely many maximal ideals, and these are all nilpotent.
    \end{enumerate}
\end{definition}
It turns out that every Artin ring is a finite sum of local Artin rings,
\begin{lemma}
    For a variety $X$ the following properties are equivalent.
    \begin{enumerate}
        \item $X$ is nonsingular.
        \item For any surjective homomorphism of Artin local rings $f : A' \to A$ the map \[X(A') \to X(A)\] is surjective.
    \end{enumerate}
\end{lemma}

\begin{proof}
    \red{Come back to this later.}
\end{proof}

\begin{proof}
    [Proof of Proposition \ref{prop:smoothness-at-gieseker-points}]
    \leavevmode
    \begin{enumerate}
        \item[(ii)]
              Let $E$ be a simple rank~$2$ vector bundle with Gieseker point
              $T = T_{E,S} \in \Alt_{N,2}\bigl(H^0(L)\bigr)$.
              We apply Lemma \ref{lem:tangent-skew} to the map
              \[
                  \langle T \rangle : \O_C^{\oplus N} \longrightarrow L^{\oplus N}
              \]
              on stalks at the generic point, whose image is $E$.
              This determines a subbundle
              \[
                  S_T^{-}
                  := \{\, h \mid h(\ker\langle T\rangle ) \subset E \,\}
                  \;\subset\;
                  \Hom^{-}\bigl(\O_C^{\oplus N}, L^{\oplus N}\bigr)
                  \;\cong\;
                  L^{\oplus N(N-1)/2}.
              \]

              The tangent space to $\Alt_{N,2}(L_{\mathrm{gen}})$ at
              $\langle T\rangle_{\mathrm{gen}}$ is the space of rational sections
              of $S_T^{-}$, and that of $\Alt_{N,2}(H^0(L))$ is $H^0(S_T^{-})$.
              Corresponding to~(10.20), we have an exact sequence of vector bundles on~$C$:
              \[
                  0 \longrightarrow \sl(E)
                  \longrightarrow \Hom(\O_C^{\oplus N}, E)
                  \longrightarrow S_T^{-}
                  \longrightarrow 0.
              \]

              But $\Hom(\O_C^{\oplus N}, E) \cong E^{\oplus N}$ while $H^1(E)=0$
              by hypothesis, and so taking global sections gives an exact sequence
              \[
                  0 \longrightarrow H^0(\sl E)
                  \longrightarrow \Hom(\O_C^{\oplus N},E)
                  \longrightarrow H^0(S_T^{-})
                  \longrightarrow H^1(\sl E)
                  \longrightarrow 0.
              \]
              The term $\Hom(\O_C^{\oplus N},E)$ is the tangent space to the
              $GL(N)$--orbit of the Gieseker point $T_{E,S}$ and identifies with the Lie algebra $\mathfrak{gl}(N)$.


              This is because the $GL(N)$--action on the space of marked matrices is given by changing the ordered basis $S$; if $g\in GL(N)$ then $S'=g\cdot S$ and $T_{E,S'}=g\cdot T_{E,S}$.  Thus we get the orbit map
              \[
                  \alpha: GL(N)\longrightarrow \Alt_{N,2}(H^0(L)),\qquad
                  \alpha(g)=g\cdot T_{E,S},
              \]
              and the orbit $\mathcal O=GL(N)\cdot T_{E,S}$ equals $\operatorname{Im}(\alpha)$.  The tangent space to the orbit at $T_{E,S}$ is the image of the differential at the identity,
              \[
                  T_{T_{E,S}}\mathcal O=\operatorname{Im}\big(d\alpha_{\mathrm{id}}\big)
                  \subset T_{T_{E,S}}\Alt_{N,2}(H^0(L)).
              \]

              Infinitesimally, an element $A\in\mathfrak{gl}(N)=T_{\mathrm{id}}GL(N)$ acts by
              $g_\varepsilon=I+\varepsilon A$ ($\varepsilon^2=0$) sending the basis
              $S=(s_1,\dots,s_N)$ to
              \[
                  S_\varepsilon
                  =\bigl(s_1+\varepsilon\sum_j A_{j1}s_j,\dots,s_N+\varepsilon\sum_j A_{jN}s_j\bigr).
              \]
              Hence the first–order variation of the $i$th basis vector is
              \[
                  \delta s_i=\sum_j A_{ji}s_j,
              \]
              and the collection $(\delta s_1,\dots,\delta s_N)$ defines an element of $H^0(E)^{\oplus N}\cong\Hom(\mathcal O_C^{\oplus N},E)$.  In this way the differential
              \[
                  d\alpha_{\mathrm{id}}:\mathfrak{gl}(N)\longrightarrow T_{T_{E,S}}\Alt_{N,2}(H^0(L))
              \]
              factors through the natural map
              \[
                  \mathfrak{gl}(N)\longrightarrow \Hom(\mathcal O_C^{\oplus N},E)
                  \longrightarrow H^0(S_T^-)=T_{T_{E,S}}\Alt_{N,2}(H^0(L)),
              \]
              so that $T_{T_{E,S}}\mathcal O$ is exactly the image of $\Hom(\mathcal O_C^{\oplus N},E)$ inside $H^0(S_T^-)$ coming from the infinitesimal change of basis.

              Putting this together with the exact sequence (10.20) on global sections, we obtain the long exact sequence
              \[
                  0\longrightarrow H^0(\mathfrak{sl}E)\longrightarrow
                  \Hom(\mathcal O_C^{\oplus N},E)\longrightarrow
                  H^0(S_T^-)\longrightarrow H^1(\mathfrak{sl}E)\longrightarrow 0.
              \]
              Finally, if $E$ is simple then $H^0(\mathfrak{sl}E)=0$. Identifying $H^0(S_T^-)=T_{T_{E,S}}\Alt_{N,2}(H^0(L))$ and observing that the image of $\mathfrak{gl}(N)$ in $H^0(S_T^-)$ equals the image of $\Hom(\mathcal O_C^{\oplus N},E)$ (the infinitesimal orbit), we conclude
              \[
                  T_{T_{E,S}}\Alt_{N,2}(H^0(L)) / T_{T_{E,S}}\mathcal O \cong H^1(\mathfrak{sl}E).
              \]
              as desired.
        \item[(i)] We use the functorial characterization of smoothness as in the lemma above. Let $A' \to A$ be a surjective homomorphism of Artin local rings with maximal ideals $\mf n', \mf n$. To show that \(\Alt_{N,2}(H^0(L))\) is smooth at the Gieseker matrix $T$, it is enough to show any deformation of $T$ over an Artinian $A$ lifts along a small extension $A' \twoheadrightarrow A$. We may assume $\dim_k \ker f = 1$ so let $\varepsilon$ span $\ker f$.

              Let $T$ be an $A$-point of $\Alt_{N,2}(H^0(L))$ whose reduction modulo $\mf n$ is the Gieseker point $T_{E,S}$.  The matrix \(T\) can be expressed as
              \[
                  T \;=\;
                  \begin{bmatrix}
                      s_1 \wedge s_2 & s_1 \wedge s_3 & \cdots & s_1 \wedge s_N     \\
                      s_2 \wedge s_3 & \cdots         &        & \vdots             \\
                      \vdots         &                & \ddots &                    \\
                                     &                &        & s_{N-1} \wedge s_N
                  \end{bmatrix}
              \]
              for some rational sections \(s_i \in E_{\mathrm{gen}} \otimes_k A\).
              Since this is an \(A\)-valued point of \(\Alt_{N,2}(H^0(L))\), the entries
              \(a_{ij} := s_i \wedge s_j\) belong to \(H^0(L \otimes_k A)\).
              Since \(f\) is surjective, we can lift each \(s_i\) to an element
              \(s'_i \in E_{\mathrm{gen}} \otimes_k A'\) and each \(a_{ij}\) to an element
              \(a'_{ij} \in H^0(L \otimes_k A')\) since tensoring by a finite dimensional vector space is exact, preserving the skew--symmetry since if the lift for \(a_{ij}\) is chosen arbitrarily then we can set \(a'_{ji} = - a'_{ij}\).

              The matrix
              \[
                  \bigl( s'_i \wedge s'_j \;-\; a'_{ij} \bigr)_{1 \le i,j \le N} \tag{10.21}
              \] measures the failure of the lifts $a'_{ij}$ to equal the wedge products of the lifts $s'_i$.  It determines a rational section of
              \(\Hom^{-}(\mathcal O_C^{\oplus N}, L^{\oplus N}) \otimes_k A'\),
              and since this section vanishes when we apply \(f\), every entry actually lies in the subspace $L \otimes_k \ker f = L \otimes_k k\varepsilon$.  In other words, it is a rational
              section of
              \[
                  \Hom^{-}(\mathcal O_C^{\oplus N}, L^{\oplus N}) \otimes_k \ker f
                  \;=\;
                  \Hom^{-}(\mathcal O_C^{\oplus N}, L^{\oplus N})_\varepsilon.
              \]
              We want this section to be everywhere regular so that the lifts \(a'_{ij}\) equal the wedge products \(s'_i \wedge s'_j\). We arrange this as follows.

              Its principal part is \((s'_i \wedge s'_j)_{1 \le i,j \le N}\)
              and is contained in \(S_T^- \otimes_k A'\).
              It follows that at each point \(p \in C\) this matrix determines a principal part
              in the vector bundle \(S_T^- \otimes_k \ker f\).
              One checks that \(H^1(S_T^-)=0\) from the exact sequence
              \[
                  0 \longrightarrow \sl(E) \longrightarrow \Hom(\O_C^{\oplus N}, E)
                  \longrightarrow S_T^- \longrightarrow 0
              \] and the corresponding long exact sequence in cohomology, together with the fact that $\cH om (\O_C^{\oplus N}, E) \cong E^{\oplus N}$ has vanishing $H^1$ using that \(H^1(E)=0\) by hypothesis.

              Therefore, these principal parts come from a global rational section.
              In other words, there exist
              \[
                  s''_1,\dots,s''_N \in E_{\mathrm{gen}}
              \]
              such that \((10.21)\) is everywhere the principal part of
              \[
                  \bigl( (\bar s_i + s''_i\varepsilon) \wedge (\bar s_j + s''_j\varepsilon)
                  \;-\;
                  (\bar s_i \wedge \bar s_j) \bigr)_{1 \le i,j \le N},
              \]
              where \(\bar s_i\) is the reduction of \(s_i\) modulo \(\mathfrak n\).
              Hence, if we set
              \[
                  T' \;=\;
                  \bigl( (s'_i + s''_i\varepsilon)\wedge (s'_j + s''_j\varepsilon) \bigr)_{1 \le i,j \le N},
              \]
              then $T'$ has entries in $H^0(L \otimes_k A')$, in particular everywhere regular, and \(T'\) is an \(A'\)-valued point of \(\Alt_{N,2}(H^0(L))\) lifting \(T\).
    \end{enumerate}
\end{proof}

\begin{proposition}[Vanishing of \(H^1\)]\label{prop:vanishing-H1}
    If \(E\) is a semistable vector bundle with \(\mu(E) > 2g-2\), or
    if \(E\) is stable and \(\mu(E) \ge 2g-2\), then \(H^1(E)=0\).
\end{proposition}

\begin{proof}
    By Serre duality, it suffices to show that there is no nonzero homomorphism
    \[E \longrightarrow \Omega_C,
    \]
    where \(\Omega_C\) is the canonical line bundle on \(C\).
    If $\phi: E \to \Omega_C$ is nonzero, then $\operatorname{im}(\phi)$ is a nonzero subsheaf of $\Omega_C$.

    Since $\Omega_C$ is a line bundle, the image is a line subbundle, so
    \[
        \operatorname{im}(\phi) \cong \Omega_C(-D) \quad\text{for some effective divisor } D\ge 0.
    \]

    Thus $\phi$ factors as
    \[
        E \twoheadrightarrow Q \hookrightarrow \Omega_C,
    \]
    where $Q:=\operatorname{im}(\phi)$ is a quotient line bundle of $E$. Moreover
    \[
        \deg Q=\deg\big(\Omega_C(-D)\big)=2g-2-\deg D\le 2g-2,
    \]
    which contradicts the hypothesis $\mu(E)>2g-2$. Hence no nonzero map $\phi$ exists.
\end{proof}

\begin{proposition}[Generation by global sections]\label{prop:generation-by-global-sections}
    If $E$ is semistable and $\mu(E) > 2g-1$, or if $E$ is stable and
    $\mu(E) \ge 2g-1$, then $E$ is generated by global sections.
\end{proposition}

\begin{proof}
    By the previous proposition, $H^{1}(E(-p)) = 0$ for every point
    $p \in C$.  It follows that, for every positive divisor $D \ge 0$,
    the restricted principal part map
    \[
        H^{0}(E(D-p)) \;\longrightarrow\; E(D-p)/E(-p)
    \]
    is surjective.  In particular, taking $D=p$ shows that the evaluation map
    \[
        H^{0}(E) \;\longrightarrow\; E/E(-p)
    \]
    is surjective at every point $p \in C$.
\end{proof}

\subsection{Proof of the main theorem}
We now take our fixed line bundle \(L\) to have degree \(\ge 4g-1\), and we
consider the action of
$\GL(N)$ on $\Alt_{N,2}\bigl(H^0(L)\bigr)$.


Suppose that \(E \in SU_C(2,L)\).
Then by Proposition~\ref{prop:vanishing-H1} we have \(H^1(E)=0\), so the orbit
\(GL(N)\cdot T_{E,S}\) of a Gieseker point depends only on \(E\) and not on
the marking \(S\).
By Proposition \ref{prop:generation-by-global-sections}, moreover, \(E\) is generated by global sections and is
therefore recovered up to isomorphism from its Gieseker points
by Proposition \ref{reconstruction-from-gieseker-point}.

And by Propositions \ref{h0-semistability-equivalence-slope-semistability} and \ref{h0-semistability-implies-gieseker-semistability}, the Gieseker points of \(E\) are semistable for the action of \(GL(N)\).

Conversely, suppose that \(T\in\Alt_{N,2}(H^0(L))\) is a semistable point for
the \(GL(N)\)-action.
The columns of \(T\) are vectors in \(H^0(L)^{\oplus N}\), and as in
Proposition~9.63 in the line bundle case we can show the following.

\begin{lemma}[Linear independence of columns]\label{lem:col-independence}
    If \(T\in \Alt_{N,2}(H^0(L))\) is semistable, then the \(N\) columns of \(T\)
    are linearly independent vectors in \(H^0(L)^{\oplus N}\) over \(k\).
\end{lemma}

\begin{proof}
    Suppose not.
    Then by a suitable change of basis (that is, by moving within the
    \(GL(N)\)-orbit) we can assume that the first row and column of \(T\) are
    zero:
    \[
        T
        \,=\,
        \begin{pmatrix}
            0      & 0      & \cdots & 0      \\
            0      & *      & \cdots & *      \\
            \vdots & \vdots & \ddots & \vdots \\
            0      & *      & \cdots & *
        \end{pmatrix}.
    \]

    Consider the action of the \(1\)-parameter subgroup
    \[
        t \longmapsto g(t)
        :=
        \begin{pmatrix}
            t^{-N+1} & 0 & \cdots & 0 \\
            0        & t &        &   \\
            \vdots   &   & \ddots &   \\
            0        &   &        & t
        \end{pmatrix}
        \in \SL(N).
    \]
    Then
    \[
        g(t)Tg(t)^{t}
        =
        \begin{pmatrix}
            0      & 0       & \cdots & 0       \\
            0      & t^{2} * & \cdots & t^{2} * \\
            \vdots & \vdots  & \ddots & \vdots  \\
            0      & t^{2} * & \cdots & t^{2} *
        \end{pmatrix}.
    \]

    Letting \(t\to 0\) shows that the origin lies in the closure of the
    \(SL(N)\)-orbit of \(T\).
    Therefore \(T\) is unstable.
\end{proof}

\begin{proposition}[Properties of bundles from semistable Gieseker points]\label{prop:properties-from-semistable-gieseker}
    Suppose that $\deg L \ge 4g-2$ and that
    $T \in \Alt_{N,2}(H^0(L))$ is semistable for the action of $GL(N)$.
    Then $E := \operatorname{Im}\langle T\rangle \subset L^{\oplus N}$ satisfies:
    \begin{enumerate}[(i)]
        \item $H^1(E)=0$;
        \item $\det E \cong L$;
        \item $E$ is semistable.
    \end{enumerate}
\end{proposition}

\begin{proof}
    \leavevmode
    \begin{enumerate}
        \item Let $V \subset H^0(E)$ be the space of global sections coming from the
              surjection $\mathcal O_C^{\oplus N} \to E$.
              Lemma~10.80 implies that $\dim V = N$, and, in particular, that
              $h^0(E)\ge N$.
              By Serre duality, the vanishing of $H^1(E)$ implies that there is a
              nonzero homomorphism $f : E \to \Omega_C$, and this induces a linear map
              \[
                  V \longrightarrow H^0(\Omega_C).
              \]
              Since $\dim H^0(\Omega_C)= g$, the kernel of this map then has dimension at least
              \[
                  N - g \ge g,
              \]
              and so, letting $M := \ker(f) \subset E$, we have
              \[
                  \dim \bigl(H^0(M)\cap V\bigr)\;\ge\; \frac{N}{2}.
              \]
              Recall that we showed if for some line subbundle $M \subset E$,
              \[
                  \dim\bigl( H^0(M) \cap V\bigr) > \frac{N}{2},
              \]
              then the Gieseker matrix $T$ is unstable under the $GL(N)$-action. Here $S$ is the chosen basis giving the map $\mathcal O_C^{\oplus N} \to E$, and $V$ is the image of $H^0(\mathcal O_C^{\oplus N})$ in $H^0(E)$, and $M = \ker (f)$ is a line subbundle of $E$.

              But we assumed at the beginning that $T$ is semistable under $GL(N)$. Therefore it is impossible that $H^1(E) \neq 0$.

        \item
              Consider the bilinear pairing
              \[
                  \mathcal O_C^{\oplus N} \times \mathcal O_C^{\oplus N}
                  \;\longrightarrow\; L,
                  \qquad
                  (u,v) \longmapsto u^{t} T v.
              \]
              This is skew--symmetric and vanishes if $u$ or $v\in \ker\langle T\rangle$,
              and hence defines a sheaf homomorphism
              \[
                  \wedge^{2}E \;\longrightarrow\; L.
              \]
              Now recall there are only maps of line bundles from lower to higher degree. Also, note that any nonzero map of line bundles $A\to B$ of the same degree is an isomorphism. This is because
              \[ 0 \to A \to B \to A/B\] where $A/B$ is torsion sheaf whose degree is equal to the sum of the lengths at the support points, so if $\deg A = \deg B$ then $A/B=0$ and the map is an isomorphism.

              Thus to show that it is an isomorphism, it is enough to check that $\deg L \le \deg E$.
              \[
                  \deg L - 2g + 2
                  \;=\; N
                  \;\le\; h^0(E),
              \]
              while by part~(i) we have $H^1(E)=0$, so that
              \[
                  h^0(E) = \deg E - 2g + 2,
              \]
              and we are done.

        \item
              By construction $T$ is a Gieseker point of the vector bundle $E$, and so
              semistability follows from Propositions~\ref{prop:h0-semistability-equivalence-slope-semistability} and~\ref{h0-semistability-implies-gieseker-semistability}. \qedhere
    \end{enumerate}
\end{proof}

\begin{lemma}
    [Finite stabilizers of stable bundles]\label{lem:finite-stab} Suppose that $H^{1}(E)=0$, that $E$ is generated by global sections,
    and that $E$ is simple.
    Given a marking $S$ and a matrix $X\in GL(N)$,
    \[
        X\,T_{E,S}\,X^{t}=T_{E,S}
        \qquad\text{if and only if}\qquad
        X=\pm I_{N}.
    \]
\end{lemma}

\begin{proof}
    The hypothesis $X\,T_{E,S}\,X^{t}=T_{E,S}$ is equivalent to the commutativity
    of the diagram
    \[
        \begin{tikzcd}
            \O_{C}^{\oplus N} \arrow[r,"{\langle T_{E,S}\rangle}"] \arrow[d,"{X^{t}}"']
            & L^{\oplus N} \arrow[d,"{X}"] \\
            \O_{C}^{\oplus N} \arrow[r,"{\langle T_{E,S}\rangle}"]
            & L^{\oplus N}.
        \end{tikzcd}
    \]
    This diagram determines an endomorphism $\phi$ of $E$, and the assumption that
    $E$ is simple implies that $\phi = c\,\mathrm{id}_{E}$ for some $c\in k$.
    But then $X = X^{t} = c\cdot I_{N}$, and in particular $c^{2}=1$.
    Thus $X = \pm I_{N}$.
\end{proof}


\begin{proof}[Proof of Theorem \ref{thm:moduli-of-rank-2-bundles}]
    To construct the moduli space
    \[
        \Alt^{ss}_{N,2}(H^0(L)) // GL(N)
    \]
    as a projective GIT quotient we consider the graded semiinvariant ring with respect to the character \(\chi : GL(N) \to k^*\), choosing $\chi = \det$.
    \[
        R = \bigoplus_{m\ge 0} H^0(\Alt_{N,2}(H^0(L)), \mathcal{O}(m))^{GL(N), \chi^m}.
    \]
    However, this argument is a little subtle since the ring $R$ might not be an integral domain, because strictly semistable points create multiple components. So one needs a result guaranteeing that Proj still gives an algebraic variety (or a disjoint union of such).

    \begin{lemma}
        If $X^{ss}$ is smooth, then the Proj quotient exists as a disjoint union of varieties. We do not need $R$ to be integrally closed or irreducible.
    \end{lemma}

    \begin{proof}
        Follows from the general theory of GIT quotients.
    \end{proof}

    The smoothness follows from Proposition \ref{prop:properties-from-semistable-gieseker}, which guarantees the condition $H^1(E)=0$, together with Proposition \ref{prop:smoothness-at-gieseker-points}.

    Now consider the open set
    \[
        \Alt^{\mathrm{s}}_{N,2}(H^0(L))/GL(N)
    \]
    of stable orbits. First note that, for each stable Gieseker point $T$, the vector bundle
    $E=\im\langle T\rangle$ is stable.
    This follows from Corollary \ref{cor:gieseker-stab-implies-h0-stab} and the proof of Proposition \ref{prop:h0-semistability-equivalence-slope-semistability}.

    Conversely, if $E$ is stable as a vector bundle, then it is simple, and so by
    Lemma \ref{lem:finite-stab} its Gieseker points $T$ have a finite stabiliser and hence are stable
    for the $GL(N)$--action.
    We therefore arrive at a bijection:
    \[
        SU_C(2,L) \;\xrightarrow{\;\sim\;}\; \Alt^{\mathrm{s}}_{N,2}(H^0(L))/GL(N).
    \]

    By Lemma \ref{lem:finite-stab}, moreover under the action
    \[
        GL(N)/\{\pm I_N\} \;\curvearrowright\; \Alt^{\mathrm{s}}_{N,2}(H^0(L)),
    \]
    all orbits are free and closed. The stabilizer of any point is exactly $\{\pm I_N\}$, and so dividing by this subgroup makes the action free. Orbits are closed because stable points have closed orbits in GIT.

    \begin{proposition}
        If an affine variety $X$ is nonsingular at every point of a free closed orbit $G \cdot x$, then the affine quotient $X//G$ is nonsingular at the image point, with dimension = $\dim X - \dim G$.
    \end{proposition}

    \begin{proof}
        Follows from the general theory of GIT quotients.
    \end{proof}

    Applying the above proposition to the open set $\Alt^{\mathrm{s}}_{N,2}(H^0(L))$ with the free action of \[G'= GL(N)/\{\pm I_N\}\], we see that the quotient
    \[\Alt^{\mathrm{s}}_{N,2}(H^0(L))/G'\] is nonsingular since $\Alt^{\mathrm{s}}_{N,2}(H^0(L))$ is nonsingular by Proposition \ref{prop:smoothness-at-gieseker-points}. But note that \[\Alt^{\mathrm{s}}_{N,2}(H^0(L))/G' \cong \Alt^{\mathrm{s}}_{N,2}(H^0(L))/GL(N)\] since $\{\pm I_N\}$ acts trivially. Thus $\SU_C(2,L)$ is nonsingular.

    Moreover, when $E$ is stable,
    \[
        \dim H^1(\sl(E)) \;=\; 3g - 3
    \]
    This follows from Riemann-Roch for vector bundles and the fact that $H^0(\sl(E))=0$ for stable $E$. More generally, if $E$ is simple, then the only endomorphisms of $E$ are scalars, so $H^0(\sl(E))=0$ and $\dim H^0(\End(E))=1$, and in particular every stable bundle is simple.



    This proves parts \textnormal{(i)} and \textnormal{(ii)}. For part \textnormal{(iii)} we note that when $\deg L$ is odd, stability  and semistability of $E$ are equivalent.
\end{proof}

\section{The Picard functor}
\begin{definition}
    A functor \(\mathcal{F} : (\text{Schemes}/k)^{op} \to \text{Sets}\) is representable if there exists a scheme \(M\) over \(k\) and a natural isomorphism of functors
    \[\mathcal{F} \cong \Hom_{k}(-, M)\]
    The scheme $M$ is called a \textbf{fine moduli space} for the functor $\mathcal F$.
\end{definition}
For many moduli problems, a fine moduli space does not exist. Mumford introduced the more relaxed notion of a best approximation to a moduli functor, and a more refined version of this called the coarse moduli space.

\begin{definition}
    Given a functor \(\mathcal{F} : (\text{Schemes}/k)^{op} \to \text{Sets}\), a scheme \(M\) over \(k\) is called a \textbf{best approximation} to \(\mathcal{F}\) if there exists a natural transformation of functors
    \[\phi : \mathcal{F} \to \Hom_{k}(-, M)\]
    such that for any scheme \(N\) over \(k\) and any natural transformation \(\psi : \mathcal{F} \to \Hom_{k}(-, N)\), there exists a unique morphism \(f : M \to N\) such that \(\psi = f \circ \phi\).

    If in addition for any algebraically closed field extension \(K/k\), the map
    \[\phi(K) : \mathcal{F}(K) \to \Hom_{k}(\Spec K, M)\]
    is a bijection, then \(M\) is called a \textbf{coarse moduli space} for the functor \(\mathcal{F}\).
\end{definition}
It is clear that fine implies coarse, and best approximation implies unique.


\begin{remark}
    The second condition is meant to test bijectivity on the level of geometric points. If you only checked $k$-points for a non-algebraically-closed $k$, you could get false failures of surjectivity just because some polynomial has no $k$-rational root, not because the moduli problem is wrong. To show a scheme is not a coarse moduli space, it suffices to exhibit one algebraically closed extension $K$ where bijection fails. In practice you take $K=\bar k$.
\end{remark}

\begin{example}
    Take $G=\mathbf G_m$ acting on $X=\mathbb A^2=\Spec k[x,y]$ by
    $t\cdot (x,y)=(tx,t^{-1}y)$.
    Then
    \[
        k[x,y]^G = k[xy],
        \quad\text{so}\quad
        X//G \cong \Spec k[xy]\cong \mathbb A^1,
    \]
    with quotient map $(x,y)\mapsto xy$. The failure of this map to be a coarse moduli space
    is non-injectivity of
    \[X(K)/G(K)\;\to\;(X//G)(K), \qquad (x,y)\mapsto xy\]
    because many distinct orbits land on the same invariant value $xy=0$. Note that upon choosing a linearization of $\mathcal O_X$ (in particular the trivial character $\chi=1$), the semistable locus is $X^{ss}=D(xy)$ and since all orbits in $D(xy)$ are closed and free, $X^{ss} = X^s$. Thus the stable locus is $D(xy)$, and the map $X^s/G \to X^s//G$ is given by $(x,y)\mapsto xy\neq 0$, which is now a bijection on $K$-points for any algebraically closed $K/k$. Thus the stable quotient is a coarse moduli space for the stable locus.
\end{example}

\begin{example}
    Suppose that a reductive group \(G\) acts on an affine variety \(X = \Spec R\). Then the functor of points \[\mathcal{F} : (\text{Schemes}/k)^{op} \to \text{Sets}, \quad S \mapsto X(S)/G(S)\] is best approximated by the affine GIT quotient $X \to X//G$ induced by the inclusion of invariant rings \(R^{G} \hookrightarrow R\). Moreover, the stable quotient \(X^{s}/G\) is a coarse moduli space for the quotient functor $X^{s}/G$.
\end{example}

\begin{remark}
    The coarse quotient $X^s/G$ is a fine moduli space if and only if the
    $G$-action on $X^s$ is free. In general, finite stabilizers give a
    Deligne--Mumford stack $[X^s/G]$ whose coarse moduli space is $X^s/G$; the
    obstruction to the existence of a universal family is exactly the nontrivial
    stabilizer groups.
\end{remark}

\subsection{Cohomology modules and direct images}

Let $A$ be a finitely generated algebra over $k$. We shall consider the
extension $A \otimes_k k(C)$ of $A$, extending coefficients from $k$ to the
function field $k(C)$. Also, we denote by $A \otimes_k \mathcal O_C$ the
elementary sheaf on $C$ defined by the ring extensions
\[
    U \longmapsto A \otimes_k \mathcal O_C(U).
\]

The pair of topological space $C$ and the elementary sheaf
$A \otimes_k \mathcal O_C$ we denote by $C_A$.

\begin{definition}
    A \textbf{vector bundle on $C_A$} is an elementary sheaf $\mathcal E$ of
    $A \otimes_k \mathcal O_C$-modules satisfying the following conditions:
    \begin{enumerate}[(i)]
        \item The total set (denoted $\mathcal E_{\mathrm{gen}}$) is a locally free
              $A \otimes_k k(C)$-module.
        \item If $U \subset C$ is an affine open set, then $\mathcal E(U)$ is a locally
              free $A \otimes_k \mathcal O_C(U)$-module.
    \end{enumerate}
\end{definition}

In the case $A = k$, of course, $\mathcal E$ is nothing but a vector bundle on
the curve $C$. In what follows we will only consider $A$-modules of finite
rank. Since the rank of a locally free module is locally constant, this number
depends only on the connected component of $\Spec A$.

A vector bundle $\mathcal E$ on $C_A$ associates to each point
$t \in \Spm A$ (corresponding to a maximal ideal $\mathfrak m \subset A$)
a vector bundle $\mathcal E_t$ on $C$ (by pull-back via
$\Spm A/\mathfrak m \hookrightarrow \Spm A$), and we can observe that this
correspondence does not change if $\mathcal E$ is replaced with
$\mathcal E \otimes_A M$ for any invertible $A$-module $M$.

\begin{definition}\label{def:families-line-bundles}
    \leavevmode
    \begin{enumerate}[(i)]
        \item Two vector bundles $\mathcal E$, $\mathcal E'$ on $C_A$ are
              \textbf{equivalent} if
              \[
                  \mathcal E' \cong \mathcal E \otimes_A M
              \]
              for some invertible $A$-module $M$.
        \item By an \textbf{algebraic family of vector bundles on $C$ parametrised by
                  $\Spm A$} we mean an equivalence class (in the sense of \textup{(i)}) of
              vector bundles on $C_A$.
    \end{enumerate}
\end{definition}

The set of families of line bundles parametrised by $\Spm A$ becomes a group
under a tensor product, and this group is just
\[
    \Pic C_A / \Pic A.
\]
Furthermore, given a ring homomorphism $f : A \to A'$, the pullback of a
family via $\Spm A' \to \Spm A$ is well defined (if $\mathcal E$ and
$\mathcal E'$ are equivalent, then so are $\mathcal E \otimes_A A'$ and
$\mathcal E' \otimes_A A'$), and the pullback of families of line bundles is
a group homomorphism
\[
    \otimes f : \Pic C_A/\Pic A \longrightarrow \Pic C_{A'}/\Pic A',
    \qquad
    \mathcal L \longmapsto \mathcal L \otimes_A A'.
\]
If $g : A' \to A''$ is another ring homomorphism, then this operation
satisfies
\[
    (\otimes g)(\otimes f) = \otimes(gf).
\]

\begin{definition}
    The covariant functor
    \[
        \Pic_C :
        \left\{
        \begin{array}{c}
            \text{finitely generated} \\
            \text{rings over } k
        \end{array}
        \right\}
        \longrightarrow
        \{\text{groups}\}
    \]
    which assigns
    \[
        A \longmapsto \Pic C_A/\Pic A
    \]
    is called the \textbf{Picard functor} for the curve $C$.
\end{definition}

Given a family of line bundles $\mathcal L \in \Pic C_A/\Pic A$, the degree of
$\mathcal L|_{C\times t}$ is constant on connected components of $\Spm A$
(Corollary~11.20). We will denote by $\Pic_C^d \subset \Pic_C$ the subfunctor
which assigns families of line bundles of degree $d$.

\subsection{Construction of the Jacobian}

Fix a curve $C$ of genus $g$ and an integer $d\in \mathbb Z$. We are going to
construct in this section a $g$-dimensional nonsingular projective variety,
the Jacobian of $C$, whose underlying set is $\Pic^d C$, the set of
isomorphism classes of line bundles on $C$ of degree $d$.

\subsubsection*{(a) Some preliminaries}

We will assume throughout this section that $d\ge 2g$, and we fix a line
bundle $L\in \Pic^{2d}C$. We note that every line bundle $\xi\in \Pic^d C$
has the following properties:
\begin{enumerate}[(i)]
    \item $H^1(\xi)=0$.
    \item $\xi$ is generated by global sections.
    \item $\dim H^0(\xi)= d+1-g=:N>g$.
\end{enumerate}
We set $\widehat{\xi}:=L\otimes \xi^{-1}\in \Pic^d C$ and note that
$\widehat{\xi}$ also has all of the properties \textup{(i)}--\textup{(iii)}.
The key tool in the algebraic construction of the Jacobian is the
multiplication map
\[
    H^0(\xi)\times H^0(\widehat{\xi}) \longrightarrow H^0(L),
    \qquad (s,t)\longmapsto st.
\]

\begin{definition}
    Given a line bundle $\xi\in \Pic^d C$, a pair $(S,T)$ consisting of a basis
    $S=\{s_1,\ldots,s_N\}$ of $H^0(\xi)$ and a basis
    $T=\{t_1,\ldots,t_N\}$ of $H^0(\widehat{\xi})$ is called a
    \textbf{double marking} of $\xi$.
\end{definition}

Given a line bundle $\xi\in \Pic^d C$ and a double marking $(S,T)$, we
introduce the following $N\times N$ matrix with entries in $H^0(L)$:
\[
    \Psi(\xi,S,T):=
    \begin{pmatrix}
        s_1t_1 & \cdots & s_1t_N \\
        \vdots & \ddots & \vdots \\
        s_Nt_1 & \cdots & s_Nt_N
    \end{pmatrix}.
\]

If we fix a rational section of $L$, then $\Psi(\xi,S,T)$ can be viewed as a
matrix of rank $1$ over the function field $k(C)$.

\begin{definition}
    We denote by $\Mat_N\bigl(H^0(L)\bigr)$ the set of $N\times N$ matrices with
    entries in $H^0(L)$. The subset of matrices of rank $1$ over $k(C)$, or
    equivalently, those for which all $2\times 2$ minors vanish, is denoted by
    \[
        \Mat_{N,1}\bigl(H^0(L)\bigr).
    \]
\end{definition}

\begin{proposition}
    Given a matrix $\Psi \in \Mat_{N,1}\bigl(H^0(L)\bigr)$, the following two
    conditions are equivalent.
    \begin{enumerate}[(1)]
        \item The $N$ rows and the $N$ columns of $\Psi$ are linearly independent over $k$.
        \item $\Psi=\Psi(\xi,S,T)$ for some $\xi\in \Pic^d C$ and some double marking $(S,T)$.
    \end{enumerate}
    Moreover, the line bundle $\xi$ is the image $\xi\subset L^{\oplus N}$ of the
    sheaf homomorphism determined by $\Psi$,
    \[
        \langle \Psi\rangle:\ \mathcal O_C^{\oplus N}\longrightarrow L^{\oplus N}.
    \]
\end{proposition}

\begin{proof}
    $(2)\Rightarrow (1)$ is clear, so we will show $(1)\Rightarrow (2)$.

    Given a matrix
    \[
        \Psi=(\psi_{ij})\in \Mat_{N,1}(H^0(L)).
    \]
    This means:
    \begin{enumerate}
        \item Each entry $\psi_{ij}$ is a global section of $L$.
        \item After restricting to the generic point $\eta$ of $C$,
              $\psi_{ij,\eta}\in L_\eta\cong k(C)$,
              so $\Psi_\eta\in \Mat_N(k(C))$ and $\operatorname{rank}_{k(C)}(\Psi_\eta)=1$.
    \end{enumerate}

    Separately, hypothesis (1) of the proposition says the $N$ rows and $N$ columns are $k$-linearly independent as vectors in $H^0(L)^{\oplus N}$. That is a condition in the $k$-vector space of global sections, not over $k(C)$.

    \medskip\noindent
    \textbf{Step 1. Construct $\xi$ as an image sheaf, and show it is a line bundle}

    Let
    \[
        \langle\Psi\rangle:\ \mathcal O_C^{\oplus N}\longrightarrow L^{\oplus N}
    \]
    be the sheaf morphism whose matrix of global sections is $\Psi$. Define
    \[
        \xi:=\operatorname{im}\langle\Psi\rangle\subset L^{\oplus N}.
    \]

    Now look at the generic fiber. At $\eta$,
    \[
        \langle\Psi\rangle_\eta:\ k(C)^N\to k(C)^N
    \]
    has rank $1$, so $\xi_\eta$ is a $1$-dimensional $k(C)$-subspace of $k(C)^N$. Thus $\xi$ is a torsion-free sheaf of rank $1$. On a smooth curve, torsion-free rank $1$ sheaves are line bundles. So $\xi$ is a line bundle on $C$.

    \medskip\noindent
    \textbf{Step 2. Use row/column independence to get $h^0(\xi)\ge N$ and $h^0(\widehat{\xi})\ge N$}

    You have a surjection of sheaves
    \[
        \mathcal O_C^{\oplus N}\twoheadrightarrow \xi.
    \]
    Taking global sections gives a $k$-linear map
    \[
        k^N = H^0(\mathcal O_C^{\oplus N}) \longrightarrow H^0(\xi).
    \]
    Concretely, the standard basis vector $e_i$ maps to the $i$-th column of $\Psi$, viewed as an $N$-tuple of global sections of $L$, then projected into $\xi\subset L^{\oplus N}$.

    If the columns of $\Psi$ are $k$-linearly independent in $H^0(L)^{\oplus N}$, then these $N$ sections of $\xi$ are $k$-linearly independent. Hence
    \[
        h^0(\xi)\ge N.
    \]
    Similarly, considering the transpose picture (or dually the map $(L^{-1})^{\oplus N}\to \mathcal O_C^{\oplus N}$) and using row independence gives
    \[
        h^0(\widehat{\xi})\ge N
    \]
    for $\widehat{\xi}:=L\otimes \xi^{-1}$.

    \medskip\noindent
    \textbf{Step 3. Deduce $H^1(\xi)=0$ and the degree bound $\deg\xi\ge d$}

    From Step 2: $h^0(\xi)\ge N>g$. The standard fact which follows from Riemann--Roch and Serre duality is if a line bundle $M$ has $h^0(M)>g$, then $H^1(M)=0$.

    So $H^1(\xi)=0$. Then Riemann--Roch gives
    \[
        h^0(\xi)=\deg\xi+1-g.
    \]
    Thus
    \[
        \deg\xi = h^0(\xi)+g-1 \;\ge\; N+g-1 \;=\; d.
    \]

    Apply the same argument to $\widehat{\xi}:=L\otimes \xi^{-1}$. You get $h^0(\widehat{\xi})\ge N>g$, hence $H^1(\widehat{\xi})=0$, hence
    \[
        \deg(\widehat{\xi})\ge N+g-1=d.
    \]
    But $\deg(\widehat{\xi})=\deg L-\deg\xi=2d-\deg\xi$. So
    \[
        2d-\deg\xi \ge d \quad\Rightarrow\quad \deg\xi\le d.
    \]
    It follows that $\deg\xi=d$ and thus $\xi\in \Pic^d(C)$.

    \medskip\noindent
    \textbf{Step 5. Reconstruct the double marking $(S,T)$ and the equality $\Psi=\Psi(\xi,S,T)$}

    Now you know $\xi\in \Pic^d$ and $\widehat{\xi}\in \Pic^d$, so by the preliminary facts,
    \[
        h^0(\xi)=h^0(\widehat{\xi})=N,\quad \text{and both are generated by global sections.}
    \]

    At the generic point, rank $1$ implies a factorization
    \[
        \Psi_\eta = f\cdot g^t
    \]
    with $f,g\in k(C)^N$. The $f_i, g_j$ are only defined up to $f\mapsto uf$, $g\mapsto u^{-1}g$ with $u\in k(C)^\times$, and they can have poles.

    You then choose $\xi$ (equivalently an effective divisor $D$ with $\xi=\mathcal O_C(D)$) so that the poles of the $f_i$'s are absorbed into $\xi$:
    \[
        s_i := f_i\cdot 1_D \in H^0(\xi)\quad\text{(regular everywhere).}
    \]
    With that choice, the complementary twist is forced:
    \[
        \widehat{\xi} := L\otimes \xi^{-1},
    \]
    and then the $g_j$'s become regular sections of $\widehat{\xi}$ (after the same normalization), so that
    \[
        \Psi_{ij}= s_i t_j \in H^0(L).
    \]

    Finally, the $k$-linear independence of rows/columns forces these $N$ sections to be linearly independent in $H^0(\xi)$ and $H^0(\widehat{\xi})$, hence they are bases $S,T$. Then by construction
    \[
        \Psi_{ij}= s_i t_j,
    \]
    so $\Psi=\Psi(\xi,S,T)$.
\end{proof}

The space of matrices $\Mat_N\bigl(H^0(L)\bigr)$ is a vector space over $k$
isomorphic to the direct sum of $N^2$ copies of $H^0(L)$, and the general
linear group $GL(N)$ acts on this space by left and right multiplication.
In particular, this gives an action of the direct product $GL(N)\times GL(N)$,
under which the image of the group homomorphism
\[
    \mathbb G_m\longrightarrow GL(N)\times GL(N),\qquad t\longmapsto (tI_N,t^{-1}I_N)
\]
acts trivially. We therefore consider the cokernel
\[
    GL(N,N)\;:=\;GL(N)\times GL(N)\big/\mathbb G_m.
\]

Note that, since $GL(N)$ is linearly reductive, so is $GL(N,N)$.
As a representation of $GL(N,N)$ the space $\Mat_N\bigl(H^0(L)\bigr)$ is
isomorphic to a direct sum of $\dim H^0(L)$ copies of the space $\Mat_N(k)$ of
square matrices over $k$. This can be viewed as an affine space $\A^n$, where
\[
    n = N^2 \dim H^0(L),
\]
and $\Mat_{N,1}\bigl(H^0(L)\bigr)$ as a closed subvariety. In particular,
$\Mat_{N,1}\bigl(H^0(L)\bigr)$ is an affine variety (or, more precisely,
each irreducible component is an affine variety, and the discussion below
applies to each irreducible component) and is preserved by the action of
$GL(N,N)$. This action is of ray type.

\medskip
The set of matrices $\Psi$ satisfying the linear independence condition~(1) in
Proposition~9.56 forms an open set
\[
    \mathcal U(L)\subset \Mat_{N,1}\bigl(H^0(L)\bigr),
\]
which is therefore a parameter space for double-marked line bundles
$(\xi,S,T)$ of degree $d$. Moreover, the open set $\mathcal U(L)$ is preserved
by the action of $GL(N,N)$.

\begin{proposition}[Proposition~9.58]
    Matrices $\Psi,\Psi'\in \mathcal U(L)$ give isomorphic line bundles
    $\xi,\xi'$ if and only if they belong to the same $GL(N,N)$-orbit.
\end{proposition}

This identifies the set $\Pic^d C$ with the space of
$GL(N,N)$-orbits in $\mathcal U(L)\subset \Mat_{N,1}\bigl(H^0(L)\bigr)$. We are going to take the projective GIT quotient of
$\Mat_{N,1}\bigl(H^0(L)\bigr)$ by $GL(N,N)$ to construct a coarse moduli space for the Picard functor
$\Pic_C^d$. We need to spell out a linearization of the action. We fix the character $\chi : \GL(N,N) \to k^*$ induced by
\[(A,B)\mapsto \det(A)\det(B)\]
Moreover let $\SL(N,N)$ be the kernel of this character.
It turns out that with respect to this character, we have \begin{align*}
    \Mat^{s}_{N,1}\bigl(H^0(L)\bigr) = \Mat^{ss}_{N,1}\bigl(H^0(L)\bigr) & = \mathcal U(L)
\end{align*}
Moreover, one can identify the tangent space at each point of $\mathcal U(L)$ as the space of first order deformations of the corresponding line bundle, controlled by $H^1(\mathcal O_C)=k^g$. In particular, $\mathcal U(L)$ is nonsingular of dimension
$g$.

The Proj GIT quotient is built from a graded ring of semiinvariants. That graded ring can fail to be an integral domain (several components, strictly semistable behavior). However, when the semistable locus is smooth, the Proj quotient still exists as a disjoint union of varieties.

A smooth scheme is regular, hence reduced. It also implies that each local ring is a domain. From that you get: distinct irreducible components cannot meet. Hence a regular scheme is a disjoint union of its irreducible components. Therefore, if a connected component is regular, it is integral because it cannot contain two disjoint nonempty opens.


In GIT the quotient is built affine-locally from invariants, and for reductive $G$ invariants behave well with reducedness.
If $A$ is reduced and $G$ is linearly reductive (in characteristic $0$, reductive suffices), then $A^G$ is reduced because the Reynolds operator splits $A^G \hookrightarrow A$. So you do not accidentally create nilpotents in the quotient ring.

Therefore the projective GIT quotient is a projective variety. In this case, since all points are stable, the quotient is a good quotient in the sense that its points correspond one-to-one to the orbits of the group action.

\begin{theorem}
    The projective GIT quotient
    \[\Mat^{s}_{N,1}\bigl(H^0(L)\bigr)\big/GL(N,N) = \Proj k[\Mat_{N,1}\bigl(H^0(L)\bigr)]^{SL(N,N)}
    \]
    is a projective variety whose underlying set is in natural bijection with $\Pic^d C$.
\end{theorem}

\begin{remark}
    Note that in general for any reductive $G$ and character $\chi$,
    \[
        \bigoplus_{m\ge0} k[X]^{G,\chi^m}
        \;\cong\;
        k[X]^{\ker\chi}
    \]
    as a graded ring, and the projective quotient can be written using either description.
\end{remark}

\begin{remark}[The stable quotient as a geometric quotient]
    Let $G$ be a reductive group acting on a projective variety $X$ with a fixed
    linearization. Then there are open subsets
    \[
        X^{s} \subset X^{ss} \subset X
    \]
    of stable and semistable points. A fundamental theorem of GIT asserts that the
    quotient
    \[
        \pi : X^{s} \longrightarrow X^{s}/G
    \]
    exists and is a \textbf{geometric quotient}. Concretely, this means:

    \begin{enumerate}[(i)]
        \item (\textbf{Orbit fibers}) $\pi(x)=\pi(x')$ if and only if $x'$ lies in the
              $G$--orbit of $x$. In particular, the points of $X^{s}/G$ are in bijection with
              $G$--orbits in $X^{s}$.

        \item (\textbf{Topological quotient}) The map $\pi$ is surjective and open, and
              $X^{s}/G$ carries the quotient topology.

        \item (\textbf{Invariant functions}) The structure sheaf is the sheaf of
              invariants:
              \[
                  \mathcal O_{X^{s}/G} = (\pi_*\mathcal O_{X^{s}})^G.
              \]

        \item (\textbf{Closed orbits and finite stabilizers}) Every $G$--orbit in $X^{s}$
              is closed in $X^{s}$ and has finite stabilizer. Consequently, $X^{s}/G$ is a
              genuine orbit space in the sense of algebraic geometry.
    \end{enumerate}

    In particular, $X^{s}/G$ is always quasi--projective, and it is projective if
    and only if $X^{s}=X^{ss}$, in which case $X^{s}/G = X^{ss}//G$. By contrast,
    the projective GIT quotient $X^{ss}//G$ is only a categorical quotient in
    general: distinct orbits in $X^{ss}$ may be identified, and points correspond
    to closed orbits in orbit closures (S--equivalence classes). Thus the stable
    quotient $X^{s}/G$ is the part of the GIT quotient that behaves as a true moduli
    space of objects. In the example of the Picard functor above, since all semistable points are stable, the projective GIT quotient turns out to be a geometric quotient.
\end{remark}

\begin{remark}[Set-theoretic vs.\ functorial moduli]
    Saying that a variety $M$ has underlying set $\Pic^d(C)$ only means that its
    $k$-points classify isomorphism classes of degree $d$ line bundles on $C$.
    This is a purely set-theoretic statement: it records which objects exist, but
    not how they vary in algebraic families.

    By contrast, to say that $M$ represents the Picard functor $\Pic_C^d$ means
    that for every scheme $S$ there is a natural bijection
    \[
        \Hom(S,M) \;\cong\; \Pic_C^d(S),
    \]
    identifying morphisms $S \to M$ with families of degree $d$ line bundles on
    $C \times S$, up to the standard equivalence. Equivalently, $M$ carries a
    universal (Poincar\'e) line bundle $\mathcal P$ on $C \times M$ such that every
    family over $S$ is obtained uniquely, up to twisting by a line bundle from $S$,
    by pullback:
    \[
        \mathcal L \;\cong\; (\id_C \times f)^*\mathcal P \quad \text{for a unique } f:S\to M.
    \]

    Thus representability encodes not only the set of isomorphism classes, but the
    entire functorial behavior of line bundles in families and under base change.
    This is the extra structure that makes $M$ a genuine moduli space rather than
    just a parameter set.
\end{remark}


\begin{theorem}
    The quotient variety \[\Mat^{s}_{N,1}\bigl(H^0(L)\bigr)\big/GL(N,N)\] represents the Picard functor \(\Pic_C^d\). In particular, it is a fine moduli space for line bundles of degree \(d\) on \(C\).
\end{theorem}

In order to prove this theorem, we will first establish that the moduli problem admits a coarse moduli space structure. Then we will upgrade this to a fine moduli space structure by constructing a universal family, classically known as a Poincar\'e line bundle.


\begin{proposition}\label{prop:11.24}
    Let $L \in \Pic^{2d}C$, where $d \ge 2g$, and let $N=d+1-g$. Then the projective quotient
    \[
        \Mat^{s}_{N,1}\bigl(H^0(L)\bigr)\big/GL(N,N)
    \]
    is a coarse moduli space for the Picard functor $\Pic_C^d$.
\end{proposition}


\begin{corollary}\label{cor:11.25}
    The isomorphism class of the variety
    \[
        J_d \;:=\; \Mat^{s}_{N,1}\bigl(H^0(L)\bigr)\big/GL(N,N)
    \]
    depends only on $C$ and $d$, and not on the line bundle $L \in \Pic^{2d}C$.
\end{corollary}

\begin{proof}[Proof of Proposition~\ref{prop:11.24}]
    Since line bundles satisfy Zariski descent and morphisms into a fixed scheme form a Zariski sheaf, it suffices to construct the correspondence functorially for affine schemes \(S=\Spec A\). Compatibility on overlaps ensures the maps glue uniquely to arbitrary base schemes. Therefore, for each finitely generated $k$-algebra $A$, our aim is to find a natural
    bijection between line bundles $\Xi$ on $C_A$ such that $\deg \Xi_t = d$ at
    every $t \in \Spm A$, up to equivalence, and morphisms $\Spm A \to J_d$.

    Let $\widetilde{\Xi} := L_A \otimes \Xi$. Then both of
    $H^0(\Xi)$ and $H^0(\widetilde{\Xi})$ are locally free $A$-modules of rank $N$,
    and their fibres at a point $t\in\Spm A$ are the spaces
    $H^0(C,\Xi_t)$ and $H^0(C,\widetilde{\Xi}_t)$.

    \begin{remark}
        Here we are invoking the following fact. Suppose $\cE$ is a vector bundle on
        $C_A$ with $H^1(C_A,\cE)=0$. Then $H^0(C_A,\cE)$ is a locally free
        $A$-module, and for every ring homomorphism $f : A \to A'$, the base change
        homomorphism
        \[H^0(C_A,\cE) \otimes_A A' \longrightarrow H^0(C_{A'},\cE \otimes_A A')\]
        is an isomorphism. This is an incarnation of the more general theory of cohomology and base change. For the sake of completeness, we state the relevant result here without proof.

        \begin{theorem}[Grauert; cohomology and base change, curve case]\label{thm:grauert}
            Let $\pi : X \to S$ be a proper flat morphism, with $S$ reduced and locally Noetherian, and let $\cF$ be a coherent sheaf on $X$ flat over $S$. Suppose that for some $p\ge0$ the function
            \[
                s \longmapsto h^p(X_s,\cF_s)
            \]
            is locally constant on $S$. Then:
            \begin{enumerate}[(i)]
                \item $R^p\pi_*\cF$ is locally free on $S$;
                \item for every morphism $f:T\to S$, the natural base change map
                      \[
                          f^*(R^p\pi_*\cF)\;\longrightarrow\;R^p\pi'_*(\cF_T)
                      \]
                      is an isomorphism, where $\pi':X_T:=X\times_S T\to T$.
            \end{enumerate}
            In particular, when $\dim X_s=1$ and $H^1(X_s,\cF_s)=0$ for all $s$, then $\pi_*\cF$ is locally free and commutes with arbitrary base change.
        \end{theorem}

        In topology and complex geometry, a map can have all fibers homeomorphic (or diffeomorphic) and still not be a fiber bundle; extra conditions are needed to guarantee local triviality. Similarly, in algebraic geometry, given a proper morphism $\pi:X\to S$ and a coherent sheaf $\cF$, you can have all fibers $H^0(X_s,\cF_s)$ with the same dimension, but $\pi_*\cF$ may not be locally free, and base change can fail. Grauert's theorem upgrades a set of vector spaces depending on $s$ into a vector bundle on $S$, providing the crucial finiteness and continuity conditions that allow sheaf cohomology to behave well under base change.

        \begin{remark}[How Grauert is used in the Jacobian construction]
            We apply this with
            \[
                \pi : C_A = C \times \Spec A \longrightarrow \Spec A,
                \qquad \cF=\Xi \text{ or } \widetilde{\Xi}=L_A\otimes\Xi^{-1},
            \]
            where $\Xi$ is a family of line bundles of degree $d\ge 2g$.

            For every $t\in\Spec A$ we have, by Riemann--Roch,
            \[
                H^1(C,\Xi_t)=0, \qquad h^0(C,\Xi_t)=d+1-g=:N,
            \]
            and the same for $\widetilde{\Xi}$. Hence the hypotheses of Grauert are satisfied for $p=0$.

            Therefore
            \[
                \pi_*\Xi \;\cong\; H^0(C_A,\Xi),
                \qquad
                \pi_*\widetilde{\Xi} \;\cong\; H^0(C_A,\widetilde{\Xi})
            \]
            are locally free $A$-modules of rank $N$, and for every $A\to A'$,
            \[
                H^0(C_A,\Xi)\otimes_A A' \;\xrightarrow{\sim}\; H^0(C_{A'},\Xi\otimes_A A'),
            \]
            \[
                H^0(C_A,\widetilde{\Xi})\otimes_A A' \;\xrightarrow{\sim}\; H^0(C_{A'},\widetilde{\Xi}\otimes_A A').
            \]

            This is the key input that upgrades the fiberwise vector spaces
            $H^0(C,\Xi_t)$ into vector bundles on $\Spec A$, allowing one to choose
            local bases and construct the matrix-valued morphism
            $\Spec A \to \Mat_{N,1}(H^0(L))$ functorially.
        \end{remark}

        \begin{example}[What Grauert is saying in elementary terms]
            Let $\pi:\P^1_S \to S$ be the projection and take $\cF=\cO_{\P^1_S}(n)$.

            For every $s\in S$,
            \[
                H^1(\P^1,\cO(n))=0 \quad (n\ge0),
                \qquad h^0(\P^1,\cO(n))=n+1.
            \]

            Grauert’s theorem implies
            \[
                \pi_*\cO_{\P^1_S}(n) \;\cong\; \cO_S^{\oplus (n+1)},
            \]
            and for all $T\to S$,
            \[
                H^0(\P^1_S,\cO(n))\otimes_{\cO_S}\cO_T
                \;\cong\;
                H^0(\P^1_T,\cO(n)).
            \]

            The intuition is that the vector spaces $H^0(\P^1,\cO(n))$ fit together into a
            vector bundle over $S$ with constant fiber dimension, and global sections
            commute with base change.
        \end{example}


    \end{remark}

    There then exists a bilinear
    homomorphism of $A$-modules: \begin{equation}\label{eq:11.3}
        H^0(\Xi) \times H^0(\widetilde{\Xi})
        \;\longrightarrow\;
        H^0(L)\otimes_k A.
    \end{equation}

    \medskip\noindent
    \textbf{Step 1.}
    We first consider the case when both of $H^0(\Xi)$ and $H^0(\widetilde{\Xi})$
    are free $A$-modules. Let $S,\widetilde{S}$ be free bases. Via
    \eqref{eq:11.3}, these determine an $N\times N$ matrix with entries in
    $H^0(L)\otimes_k A$, and so we get a morphism to an (affine) space of
    matrices,
    \[
        \Spm A \longrightarrow \Mat_N\bigl(H^0(L)\bigr).
    \]
    This maps into the closed subvariety
    \[
        \Mat_{N,1}\bigl(H^0(L)\bigr)\subset \Mat_N\bigl(H^0(L)\bigr)
    \]
    defined by the vanishing of the $2\times 2$ minors because each fiber matrix factors as
    column vector of $H^0(\Xi_t)$ tensor row vector of $H^0(\widetilde{\Xi}_t)$.

    Moreover, the image lies in theopen set $\Mat^s_{N,1}\bigl(H^0(L)\bigr)$, since for all $t\in\Spm A$ the
    line bundles $\Xi_t$ and $\widetilde{\Xi}_t$ are generated by global
    sections. We will denote this map by
    \[
        \widetilde{\varphi}:\Spm A \longrightarrow \Mat^s_{N,1}\bigl(H^0(L)\bigr),
    \]
    and the composition of $\widetilde{\varphi}$ with the quotient map by
    \[
        \varphi:\Spm A \longrightarrow J_d
        =\Mat^s_{N,1}\bigl(H^0(L)\bigr)\big/GL(N,N).
    \]

    The map $\varphi$ depends only on the equivalence class of $\Xi$
    (in the sense of Definition \ref{def:families-line-bundles}) and not on the choice of $S,\widetilde S$.

    We now take an affine open cover
    \[
        \Spm A = U_1 \cup \cdots \cup U_n
    \]
    such that the $A$-modules $H^0(\Xi)$ and $H^0(\widetilde{\Xi})$ restrict to free
    modules on each $U_i$. For each $i$, by choosing free bases of
    $H^0(\Xi)|_{U_i}$ and $H^0(\widetilde{\Xi})|_{U_i}$ we obtain a map
    \[
        \widetilde{\varphi}_i : U_i \longrightarrow \Mat^s_{N,1}\bigl(H^0(L)\bigr)
    \]
    and on intersections $U_i \cap U_j$ the maps
    $\widetilde{\varphi}_i$ and $\widetilde{\varphi}_j$ differ only by the choice of
    free bases of $H^0(\Xi)|_{U_i \cap U_j}$ and
    $H^0(\widetilde{\Xi})|_{U_i \cap U_j}$. It follows that the corresponding maps
    \[
        \varphi_i : U_i \longrightarrow J_d
        \quad\text{and}\quad
        \varphi_j : U_j \longrightarrow J_d
    \]
    agree on the intersection $U_i \cap U_j$, and by gluing we therefore obtain a
    morphism
    \[
        \varphi : \Spm A \longrightarrow J_d.
    \]

    This is called the \textbf{classifying map} for the family of line bundles $\Xi$.
    Let $\Xi, \Xi'$ be two line bundles on $C_A$ which are locally equivalent as
    families of line bundles on $C$. By this we mean that there is an open cover of
    $\Spm A$ as above, such that on each open set the restrictions
    $\Xi|_{U_i}$ and $\Xi'|_{U_i}$ are equivalent.

    For these line bundles the classifying maps
    \[
        \varphi, \varphi' : \Spm A \longrightarrow J_d
    \]
    are the same, and in particular we see that $\varphi$ depends only on the
    equivalence class of $\Xi$ (Definition~11.22). This verifies the first
    requirement for $J_d$ to be a coarse moduli space: we have constructed a natural
    transformation of functors
    \[
        \Pic^d_C \longrightarrow J_d.
    \]
    Over an algebraically
    closed field $k$ this is bijective essentially because over an algebraically closed field, geometric points correspond to $k$-points. Over non-closed fields, points correspond to Galois orbits of geometric points, and bijectivity on $k$-points is false in general. (Recall that a geometric point of a $k$-scheme $X$ is a morphism
    \[
        \Spec \Omega \to X
    \]
    where $\Omega$ is an algebraically closed field. Equivalently: a point of $X(\bar k)$ together with an embedding $k\hookrightarrow\bar k$.)

    Finally, we have to show universality. Suppose that we have
    a natural transformation
    \[
        \psi : \Pic^d_C \longrightarrow Y
    \]
    for some variety $Y$. Over the product
    $C \times \Mat_{N,1}\bigl(H^0(L)\bigr)$ there is a tautological homomorphism of
    vector bundles
    \[
        \mathcal{O}^{\oplus N}_{C \times \Mat}
        \longrightarrow
        \bigl(L \otimes \mathcal{O}_{\Mat}\bigr)^{\oplus N}.
    \]

    Restricted to the open set
    $C \times \Mat^s_{N,1}\bigl(H^0(L)\bigr)$, this map has rank~$1$ at each point, and
    its image is a line bundle. We denote this line bundle on
    $C \times \Mat^s_{N,1}\bigl(H^0(L)\bigr)$ by $\mathcal{Q}$, called the
    \textbf{universal line bundle}. Composing the natural transformation
    \[
        \Mat^s_{N,1}\bigl(H^0(L)\bigr) \longrightarrow \Pic^d_C,
    \]
    defined in the obvious way by the pullback of $\mathcal{Q}$, with $\psi$ gives a
    natural transformation of functors
    \[
        \Mat^s_{N,1}\bigl(H^0(L)\bigr)
        \longrightarrow \Pic^d_C
        \overset{\psi}{\longrightarrow} Y.
    \]

    Since the line bundle $\mathcal{Q}$ is trivial on $GL(N,N)$--orbits, it follows
    that the corresponding morphism
    \[
        \Mat^s_{N,1}\bigl(H^0(L)\bigr) \longrightarrow Y
    \]
    descends to the quotient, and so we obtain a morphism
    \[
        J_d \longrightarrow Y
    \]
    with the required properties. Here we are invoking the fact that $J_d$ is a geometric quotient, which is true here because
    \[
        \Mat^s_{N,1}(H^0(L)) = \Mat^{ss}_{N,1}(H^0(L)) \qedhere
    \]
\end{proof}

Now we are ready to upgrade the coarse moduli space structure of $J_d$ to a fine
moduli space structure by constructing a universal family of line bundles
on $C \times J_d$. We recall some preliminaries which we will apply to the universal line bundle $\mathcal Q$.

Let $G$ be a linearly reductive algebraic group acting on an affine variety $X=\Spm R$, and
let $M$ be an $R$-module with a $G$-linearisation, i.e. $G$-action on $M$ as a $k$-vector space, compatible with the $G$-action on $R$ in the sense that
\[
    g\cdot(rm) = (g\cdot r)(g\cdot m)\]
Thus $M$ is also a representation of $G$ and has a subset of invariants $M^G\subset M$
which, by definition of a linearisation, is an $R^G$-module.

\begin{lemma}
    If $M$ is a finitely generated module over a Noetherian ring $R$, then $M^G$ is
    finitely generated as an $R^G$-module.
\end{lemma}

\begin{proof}
    The idea of the proof is the same as that of Hilbert's Theorem~4.51. Denote by
    $M'\subset M$ the submodule generated by $M^G$. Since $R$ (and $M$) are
    Noetherian, it follows that $M'$ is finitely generated. Letting
    $m_1,\dots,m_n\in M'$ be generators, the map
    \[
        R\oplus\cdots\oplus R \longrightarrow M',
        \qquad
        (a_1,\dots,a_n)\longmapsto a_1m_1+\cdots+a_nm_n
    \]
    is surjective, and hence by linear reductivity the induced map
    \[
        R^G\oplus\cdots\oplus R^G \longrightarrow (M')^G = M^G
    \]
    is surjective, so that $M^G$ is generated as an $R^G$-module by
    $m_1,\dots,m_n$.
\end{proof}

\begin{proposition}
    Suppose that all orbits of $G$ on $\Spm R$ are free closed orbits and that $M$ is
    a locally free $R$-module. Then $M^G$ is a locally free $R^G$-module and
    \[
        M \cong M^G \otimes_{R^G} R.
    \]
\end{proposition}

\begin{proof}
    Let $I\subset R$ be the ideal of an orbit, with corresponding maximal ideal
    $\mathfrak m = I\cap R^G \subset R^G$. Then $M/IM$ is a $k[G]$-module with a
    $G$-linearisation and so is a free $k[G]$-module by Lemma~9.49. By linear
    reductivity we can find $m_1,\dots,m_r\in M^G$ whose residue classes
    $\overline m_1,\dots,\overline m_r\in M/IM$ are a free basis, and hence the
    natural homomorphism of $R$-modules
    \[
        M^G\otimes_{R^G} R \longrightarrow M
    \]
    is an isomorphism along each orbit. Since, by Lemma~11.26, $M^G$ is finitely
    generated, it follows from Nakayama's Lemma that this homomorphism is
    surjective. But it is also injective because $M$ is locally free.
\end{proof}

\begin{remark}
    The proposition establishes the existence and uniqueness of descent data for locally free sheaves under the hypotheses stated. More precisely:

    You have an affine geometric quotient
    \[
        \pi:\ X=\Spec R \;\to\; Y=\Spec R^G
    \]
    with all orbits free and closed.

    A $G$-linearised vector bundle on $X$ is the same as a locally free $R$-module $M$ with a compatible $G$-action.

    To say that $M$ descends means: There exists a vector bundle $E_0$ on $Y$ such that $M \cong \pi^*E_0$. To say it descends uniquely means if $M \cong \pi^*E_0 \cong \pi^*E_1$, then $E_0 \cong E_1$.

    The proposition constructs $E_0 \leftrightarrow M^G$ and proves:
    \begin{itemize}
        \item $M^G$ is locally free over $R^G$,
        \item the natural map $M^G \otimes_{R^G} R \;\longrightarrow\; M$ is an isomorphism.
    \end{itemize}

    This is exactly $M \cong \pi^*(M^G)$, so the descended bundle exists.

    For uniqueness, suppose $M \cong N \otimes_{R^G} R$ for some locally free $R^G$-module $N$. Take invariants: $M^G \cong (N \otimes_{R^G} R)^G$. But because the orbits are free and closed and $G$ is linearly reductive, one proves $(N \otimes_{R^G} R)^G \cong N$. So $N \cong M^G$, meaning any descent must equal $M^G$. Therefore uniqueness is automatic once the invariant functor behaves well, which is exactly what linear reductivity plus freeness give you.
\end{remark}

Let $X$ be affine with a $G$-linearisation by a character $\chi$. Then
\[
    X//_\chi G = \Proj \bigoplus_{n\ge0} R^{G,\chi^n}.
\]

A basic fact from GIT: For any homogeneous invariant $f\in R^{G,\chi^n}$ with $n>0$,
$
    D_+(f) \subset X//_\chi G
$
is affine, and
$
    \Phi^{-1}(D_+(f)) = X_f^{ss}
$
is affine, with
$
    \Phi^{-1}(D_+(f)) \to D_+(f)
$
equal to the affine quotient map
$
    \Spec R_{(f)} \longrightarrow \Spec (R_{(f)})^G.
$

In particular, the projective quotient map in the direction of some character $\chi\in\Hom(G,\mathbb G_m)$,
\[
    \Phi=\Phi_\chi:X^{ss}\longrightarrow X//_\chi G,
\]
is locally an affine quotient map.

\begin{corollary}\label{cor:descent-for-git}
    Suppose that all semistable points are stable and that every orbit of the action
    $G$ on $X^{ss}=X^s$ is a free closed orbit. Then, given a vector bundle $E$ on
    $X$ with a $G$-linearisation, there exists a vector bundle $E_0$ on $X^s/G$ such
    that
    \[
        E \cong \Phi^*E_0.
    \]
\end{corollary}

\begin{proof}
    The proof reduces to the affine case:
Cover \(X//_{\chi} G\) by affines \(U_i = D_+(f_i)\).
On each \(U_i\),
    \(\Phi^{-1}(U_i) \to U_i\)
    is an affine quotient \(\Spec R_i \to \Spec R_i^G\).
Over each affine piece, equivariant locally free modules descend uniquely.
The descended bundles glue because everything was $G$-equivariant and the quotient is geometric.
\end{proof}
$\mathcal Q$ had the property that
\[
    (1 \times \tilde{\varphi})^*\mathcal Q \cong \Xi
\]
under
\[
    1 \times \tilde{\varphi} : C \times \Spm A \longrightarrow
    C \times \Mat^s_{N,1}\bigl(H^0(L)\bigr).
\]



Let $R$ be the coordinate ring of the affine variety
$\Mat_{N,1}\bigl(H^0(L)\bigr)$. There is a tautological homomorphism of
$R$-modules
\[
    \tau : R^{\oplus N} \longrightarrow R^{\oplus N} \otimes_k H^0(L)
\]
given in the obvious way by matrix multiplication. Given a linear map
$f : H^0(L) \to k$, we then get a homomorphism of $R$-modules as the
composition
\[
    \phi_f : R^{\oplus N}
    \xrightarrow{\ \tau\ }
    R^{\oplus N} \otimes_k H^0(L)
    \xrightarrow{\ 1 \otimes f\ }
    R^{\oplus N} \otimes_k k
    = R^{\oplus N}.
\]

When $f$ is the evaluation map at a point $p \in C$ this homomorphism
has rank $\le 1$ everywhere, and on the open set
$\Mat^s_{N,1}\bigl(H^0(L)\bigr)$ its image is precisely the line bundle
\[
    \mathcal Q_p := \mathcal Q|_{p \times \Mat}.
\]

The group $GL(N) \times GL(N)$ acts on $R$, and using this action we let
it act on the source and target $R^{\oplus N}$ of the homomorphism
$\phi_f$, respectively, by
\[
    \begin{pmatrix}
        f_1 \\ \vdots\\ f_N
    \end{pmatrix}
    \longmapsto
    A
    \begin{pmatrix}
        g \cdot f_1 \\ \vdots\\ g \cdot f_N
    \end{pmatrix},
    \qquad
    \begin{pmatrix}
        f_1 \\ \vdots\\ f_N
    \end{pmatrix}
    \longmapsto
    \begin{pmatrix}
        g \cdot f_1 \\ \vdots\\ g \cdot f_N
    \end{pmatrix}
    B^{t,-1}.
\]

The map $\phi_f$ is then a $GL(N) \times GL(N)$-homomorphism, and in
particular $GL(N) \times GL(N)$ acts on the $R$-module $\mathcal Q_p$.
Similarly, the universal line bundle $\mathcal Q$ carries a
$GL(N) \times GL(N)$-linearisation, under which $(t,t^{-1}) \in
    GL(N) \times GL(N)$ acts nontrivially. However, this element acts
trivially on the line bundle $\mathcal Q \otimes \mathcal Q_p^{-1}$ with
its induced linearisation, and so this line bundle possesses a
$GL(N,N)$-linearisation. Applying Corollary~\ref{cor:descent-for-git}, we conclude that
$\mathcal Q \otimes \mathcal Q_p^{-1}$ is the pullback of some line
bundle
\[
    \mathcal P_d \in \Pic(C \times J_d).
\]

This is called the \textbf{Poincar\'e line bundle}.



\begin{lemma}
    Let $\Xi$ be a line bundle on $C_A$ with classifying map
    $\varphi : \Spm A \to J_d$. Then $\Xi$ is equivalent to the pullback
    $(1 \times \varphi)^*\mathcal P_d$ via
    $1 \times \varphi : C_A \to C \times J_d$.
\end{lemma}

\begin{proof}
    Let $\mathcal L = (1 \times \varphi)^*\mathcal P_d$. By construction of
    the classifying map, $\Xi$ is already locally isomorphic to $\mathcal
        L$. In other words,
    \[
        \Xi|_{C \times U_i} \cong \mathcal L|_{C \times U_i}
    \]
    over some affine open cover $\Spm A = U_1 \cup \cdots \cup U_n$. Thus
    \[
        M := H^0(\Xi \otimes \mathcal L^{-1})
    \]
    is an invertible $A$-module, and the natural homomorphism
    \[
        \mathcal L \otimes_A M \longrightarrow \Xi
    \]
    is an isomorphism. Hence $\Xi$ and $\mathcal L$ are equivalent.
\end{proof}
This completes the proof that $J_d$ is a fine moduli space for the Picard functor
$\Pic_C^d$.


\begin{remark}
    The reason the Picard moduli problem admits a fine moduli space is that every
    line bundle has the same automorphism group $\mathbb G_m$, acting centrally.
    Thus the moduli stack of line bundles is a $\mathbb G_m$--gerbe over its coarse
    space. One can rigidify globally by this universal $\mathbb G_m$, killing all
    stabilizers at once. After rigidification the inertia is trivial, and the
    resulting moduli problem is representable by a scheme, namely the Jacobian.
    This uniformity of automorphism groups is special to rank one and fails in
    higher rank, where stabilizers vary and obstruct the existence of a fine moduli
    space.
\end{remark}

\section{The moduli functor for vector bundles}
\begin{definition}
    Given $L \in \Pic C$, we denote by
    \[
        \mathcal{VB}_C(r,L) : \{\text{finitely generated rings over } k\} \longrightarrow \{\text{sets}\}
    \]
    the functor which associates to objects $A$ the set of families $\mathcal{E}$ on $C_A$ for which
    $\det \mathcal{E}$ is equivalent to the constant family $L_A \in \Pic_C(A)$.
\end{definition}

A ``moduli space'' for vector bundles (of rank $r$) is an object which in some suitable sense approximates the functor $\mathcal{VB}_C(r)$ (or its ``connected components'' of vector bundles with fixed degree) or $\mathcal{VB}_C(r,L)$. However, as soon as $r \ge 2$, a \textbf{coarse moduli space}, or a best approximation in the sense of Definition 11.6, cannot exist for the following reason.

\begin{example}[Jumping phenomenon]
    There exist families of vector bundles of rank $\ge 2$, say $\mathcal{E}$ parametrised by $T = \Spec A$, with the following property. For some dense open set $U \subset T$ and all $u \in U$, $t \in T$,
    \[
        \mathcal{E}_t \cong \mathcal{E}_u \text{ if } t \in U,
        \qquad
        \mathcal{E}_t \not\cong \mathcal{E}_u \text{ if } t \in T \setminus U.
    \]

    For example, let $L \in \Pic C$ be a line bundle and fix an extension of $\mathcal{O}_C$ by $L$ with (nonzero) extension class $e \in H^1(L)$.

    An extension
    \[
        0 \longrightarrow L \longrightarrow E \longrightarrow \mathcal O_C \longrightarrow 0
    \]
    is classified by an element
    \[
        e \in \Ext^1(\mathcal O_C, L) \cong H^1(C,L).
    \]

    Choose an open cover $\{U,V\}$ of $C$ such that $L$ is trivial on both. Then \v{C}ech cohomology gives
    \[
        H^1(C,L) \cong \check H^1(\{U,V\},L)
        = \frac{L(U\cap V)}{L(U)+L(V)}.
    \]
    Thus picking a representative $b \in L(U\cap V)$ means that on the overlap $U\cap V$ we specify how to glue trivial extensions to obtain a possibly nontrivial one.

    On $U$ and $V$ define the trivial rank--$2$ bundles
    \[
        E|_U := \mathcal O_U \oplus L_U,
        \qquad
        E|_V := \mathcal O_V \oplus L_V.
    \]

    On the overlap, glue them by the transition matrix
    \[
        g_{UV} =
        \begin{pmatrix}
            1 & b \\
            0 & 1
        \end{pmatrix}.
    \]

    Concretely, this means that a local section $(f,\ell)$ over $U\cap V$ is identified with
    \[
        (f,\ell) \longmapsto (f + b\ell,\ \ell)
    \]
    when viewed in the other chart.

    Because $b$ is a section of $L$, the expression $b\ell$ makes sense and lies in
    $\mathcal O_{U\cap V}$. Since there are only two open sets, the cocycle condition is automatic. This gluing produces a rank--$2$ vector bundle $E_1$.

    The subbundle $L \subset E_1$ is the second summand, and the quotient is $\mathcal O_C$.
    If $b=0$, the transition matrix is the identity and one recovers $L \oplus \mathcal O_C$.
    If $b\neq 0$, the extension is non--split.

    Changing $b$ by an element coming from $L(U)$ or $L(V)$ exactly corresponds to changing
    trivializations, so the isomorphism class depends only on the cohomology class
    $[b]=e \in H^1(L)$.

    \medskip

    Now take $A = k[s]$. We want a vector bundle $\mathcal E$ on
    \[
        C_A = C \times \Spec k[s]
    \]
    whose fiber over $s=a$ is the extension with class $ae$.

    Over $U\times \Spec k[s]$ and $V\times \Spec k[s]$, take the trivial families
    \[
        (\mathcal O_U \oplus L_U)\otimes_k k[s],
        \qquad
        (\mathcal O_V \oplus L_V)\otimes_k k[s].
    \]

    On $(U\cap V)\times \Spec k[s]$, glue them by the transition matrix
    \[
        g_{UV}(s) =
        \begin{pmatrix}
            1 & bs \\
            0 & 1
        \end{pmatrix}.
    \]

    Here $bs$ means the section $b$ of $L(U\cap V)$ tensored with the function
    $s\in k[s]$, giving an element of
    \[
        L(U\cap V)\otimes_k k[s]
        = (L\otimes_k k[s])(U\cap V).
    \]
    This defines a vector bundle $\mathcal E$ on $C_A$.

    \medskip

    \noindent\textbf{Why the fibers are $E_a$.}
    Restricting to the fiber over $a\in k$ means quotienting by $(s-a)$. Then the transition matrix becomes
    \[
        \begin{pmatrix}
            1 & ba \\
            0 & 1
        \end{pmatrix}.
    \]

    Thus the glued bundle is exactly the extension with cocycle $ab$, i.e.\ extension class
    $ae \in H^1(L)$. Therefore:
    \begin{itemize}
        \item for $a\neq 0$, all fibers are isomorphic to $E_1$ (a scalar multiple of a nonzero class);
        \item for $a=0$, the matrix is the identity and the bundle splits as
              $L\oplus \mathcal O_C$.
    \end{itemize}

    This produces an algebraic family whose isomorphism type jumps at $0$.

    Now suppose that $X$ were a coarse moduli space for the moduli functor $\mathcal{VB}_C(r)$, $r \ge 2$, and $\{\mathcal{E}_t\}_{t \in T}$ some jumping family as above. By the coarse moduli property there is then a morphism $f : T \to X$ with the property that some dense open set $U \subset T$ maps to a single point, but whose image $f(T) \subset X$ contains more than one point, a contradiction.

    This phenomenon forces us, if we want to construct a moduli space as an algebraic variety, to restrict to some smaller class among the vector bundles that we are considering.
\end{example}

So we now introduce stability conditions on vector bundles, which will allow us to construct moduli spaces for stable vector bundles.

\begin{definition}
    Given a line bundle $L \in \Pic C$, we denote by
    \[
        SU_C(r,L) \subset \mathcal{VB}_C(r,L)
    \]
    the subfunctor which associates to a ring $A$ the set of families $\mathcal{E}$ on
    $C_A$ for which $\mathcal{E}_t$ is slope stable for all points $t \in \Spm A$.
\end{definition}

Note that stability depends only on the equivalence class of the vector bundle
$\mathcal{E}$ on $C_A$ representing the family. For the same reason,
tensoring with any line bundle $\xi \in \Pic C$, $\mathcal{E} \mapsto \mathcal{E}\otimes_{\mathcal{O}_C} \xi$
induces an isomorphism of functors
\begin{equation}
    SU_C(r,L) \;\xrightarrow{\ \sim\ }\; SU_C(r, L \otimes \xi^r).
\end{equation}

It follows that the isomorphism class of the functor $SU_C(r,L)$ depends only on
$\deg L \bmod r$, and not on $L$ itself. In the rest of this section we restrict to
$r=2$ and consider separately the cases when $\deg L$ is odd or even.

When $\deg L$ is odd we have already seen that the quotient
variety
\[
    \Alt^{s}_{N,2}\bigl(H^0(L)\bigr)/GL(N)
\]
has $SU_C(2,L)$ as its underlying set of points. When $\gcd(r,\deg L)=1$, semistability = stability, and every stable bundle is simple; this is why in the odd-degree rank-2 case the moduli space behaves particularly well. We assume that
$\deg L \ge 4g-1$ and we let $N = \deg L + 2 - 2g$.

We now verify that this quotient variety is indeed a coarse moduli space for the functor $SU_C(2,L)$. Later we will upgrade this to a fine moduli space by constructing a universal family of vector bundles on $C \times M_L$.
\begin{proposition}
    Suppose that $\deg L \ge 4g-1$ is odd. Then the quotient variety
    \[
        M_L := \Alt^{ss}_{N,2}\bigl(H^0(L)\bigr)//GL(N)
    \]
    is a coarse moduli space for the functor $SU_C(2,L)$.
\end{proposition}

\begin{proof}
    The proof is similar to that of the Jacobian case. We construct a natural transformation of functors \begin{align*}
        SU_C(2,L) & \longrightarrow M_L
    \end{align*} and check the universality property.

    Let $A$ be a finitely generated ring over the field $k$ and $\mathcal E$ be a rank $2$
    vector bundle on $C_A$. We suppose that $\mathcal E_t$ is stable for all
    $t \in \Spm A$ and that $\det \mathcal E \cong L_A \otimes M$ for some invertible
    $A$--module $M$ (that is, for some line bundle on $\Spm A$). We consider the
    $A$--module $H^0(\mathcal E)$ of global sections, which is a locally free $A$--module of rank $N$.

    We will consider the skew--symmetric $A$--bilinear map
    \[
        H^0(\mathcal E) \times H^0(\mathcal E)
        \;\longrightarrow\;
        H^0(\det \mathcal E) = H^0(L)\otimes_k M,
        \qquad
        (s,t) \longmapsto s \wedge t.
    \]

    \noindent\textbf{Step 1.}
    We first consider the case where $H^0(\mathcal E)$ and $M$ are both free $A$--modules.
    Then, by choosing free bases, the above map determines a skew--symmetric
    $N\times N$ matrix with entries in $H^0(L)\otimes_k A$. In other words, we get a
    morphism
    \[
        \tilde\varphi : \Spm A \longrightarrow \Alt_N\bigl(H^0(L)\bigr).
    \]

    The image of this map is contained in the zero--set of the $4\times 4$ Pfaffian
    minors
    \[
        \Alt_{N,2}\bigl(H^0(L)\bigr) \subset \Alt_N\bigl(H^0(L)\bigr),
    \]
    and the image of each $t \in \Spm A$ is exactly a Gieseker point of
    the vector bundle $\mathcal E_t = \mathcal E|_{C\times t}$. As we saw in the last
    chapter, this is stable for the action of $GL(N)$, and so the morphism
    $\tilde\varphi$ maps into the open set $\Alt^{s}_{N,2}\bigl(H^0(L)\bigr)$.
    We denote the composition of $\tilde\varphi$ with the quotient map
    \[
        \Alt^{s}_{N,2}\bigl(H^0(L)\bigr) \longrightarrow \mathcal M_L
    \]
    by
    \[
        \varphi : \Spm A \longrightarrow \mathcal M_L.
    \]
    This map depends only on $\mathcal E$ and not on the choice of basis for
    $H^0(\mathcal E)$.

    \medskip

    \noindent\textbf{Step 2.}
    Choose an affine open cover
    \[
        \Spm A = U_1 \cup \cdots \cup U_n
    \]
    such that the restriction of $H^0(\mathcal E)$ and $M$ to each $U_i$ are both free
    modules. Just as for the Picard functor, this gives, using Step~1, a morphism
    $\varphi : \Spm A \to \mathcal M_L$.

    \medskip

    \noindent\textbf{Step 3.}
    We have therefore constructed the natural transformation, and we will now
    check that it has the universal property. On the product
    $C \times \Alt_N\bigl(H^0(L)\bigr)$ there is a natural tautological homomorphism
    \[
        \mathcal O_{C\times\Alt}^{\oplus N}
        \;\longrightarrow\;
        (L \boxtimes \mathcal O_{\Alt})^{\oplus N},
    \]
    given by matrix multiplication, whose restriction to
    $C \times \Alt_{N,2}\bigl(H^0(L)\bigr)$ has rank $2$. The image is then a rank $2$
    vector bundle whose restriction to
    $C \times \Alt^{s}_{N,2}\bigl(H^0(L)\bigr)$ we denote by $\mathcal Q$.

    Now suppose that we have a natural transformation of functors
    $SU_C(2,L) \to X$ for some other variety $X$. Applying it to the vector bundle
    $\mathcal Q$ then determines a morphism
    \[
        \Alt^{s}_{N,2}\bigl(H^0(L)\bigr) \longrightarrow X;
    \]
    and since $\mathcal Q$ is preserved by the action of $GL(N)$, this descends to a
    morphism of the quotient
    \[
        \mathcal M_L \longrightarrow X.
    \]
\end{proof}

\begin{proof}[Proof of fine moduli property when $\deg L$ is odd]
    We have constructed a universal rank--$2$ vector bundle
    \[
        \mathcal Q \;\text{ on }\; C \times \Alt^{s}_{N,2}\bigl(H^0(L)\bigr),
    \]
    equipped with a natural $GL(N)$--linearisation.  We would like $\mathcal Q$ to
    \textbf{descend} to a vector bundle on the quotient
    \[
        C \times \mathcal M_L, \qquad \mathcal M_L := \Alt^{s}_{N,2}\bigl(H^0(L)\bigr)/GL(N).
    \]

    By the descent theorem (Corollary~\ref{cor:descent-for-git}), a
    $G$--linearised vector bundle descends along a geometric quotient provided:

    \begin{enumerate}[(i)]
        \item all orbits are free and closed;
        \item every stabiliser acts trivially on the fibres.
    \end{enumerate}

    In our situation, all stable orbits are closed, and the stabiliser of any point
    is exactly $\{\pm I_N\} \subset GL(N)$.  The problem is that although $-I_N$ acts
    \textbf{trivially on the base}
    \[
        T \longmapsto (-I_N)\,T\,(-I_N)^t = T,
    \]
    it acts as multiplication by $-1$ on the fibres of $\mathcal Q$.  Thus the
    stabiliser does not act trivially on $\mathcal Q$, and the descent theorem does
    not apply.  This is the obstruction to the existence of a universal bundle on
    the coarse moduli space.

    To remove this obstruction, we twist the linearisation.  Let $\mathcal D$ be
    the trivial line bundle on $\Alt^{s}_{N,2}(H^0(L))$ endowed with the
    $GL(N)$--linearisation given by the determinant character:
    \[
        g \cdot z := (\det g)\,z.
    \]
    Define
    \[
        \mathcal Q' := \mathcal Q \otimes \mathcal D.
    \]
    Then $-I_N$ acts on $\mathcal Q$ by $-1$, and on $\mathcal D$ by
    $\det(-I_N)=(-1)^N$.  Hence on $\mathcal Q'$ it acts by
    \[
        (-1)\cdot(-1)^N = (-1)^{N+1}.
    \]
    If $N$ is odd (equivalently, $\deg L$ is odd), this equals $+1$.  Therefore
    $-I_N$ acts trivially on the fibres of $\mathcal Q'$, and the linearisation
    factors through $GL(N)/\{\pm I_N\}$.

    Since the action of $GL(N)/\{\pm I_N\}$ on
    $\Alt^{s}_{N,2}(H^0(L))$ has free closed orbits, Corollary~\ref{cor:descent-for-git} implies that
    $\mathcal Q'$ descends to a vector bundle
    \[
        \mathcal U \;\text{ on }\; C \times \mathcal M_L,
        \qquad
        \mathcal Q' \cong (\mathrm{id}_C \times \pi)^*\mathcal U.
    \]
    This descended bundle $\mathcal U$ serves as the universal family and yields the
    inverse to the natural transformation defining the coarse moduli property.

    Conceptually, the issue is that the universal bundle exists naturally on the
    moduli \textbf{stack}, but does not descend to the coarse moduli space unless the
    stabiliser acts trivially on the fibres.  In rank $2$ and odd determinant, the
    determinant twist kills the residual $\pm1$--action, allowing descent.
\end{proof}

\begin{remark}[Why trivial stabiliser action is necessary for descent]
    Conceptually, descent to a coarse moduli space forgets stabilisers.  In stack
    language, the coarse moduli space is obtained by collapsing all automorphism
    groups to the identity.  Therefore, any geometric object on a quotient stack
    which descends to the coarse space must carry \textbf{trivial inertia action}:
    for every point $x$ and every $g$ in its stabiliser group $G_x$, the induced
    action of $g$ on the fibre must be the identity.

    Otherwise, the object retains representation--theoretic information of the
    stabiliser which has no place to live on the coarse space.  Equivalently, if
    $\mathcal E \cong \pi^*\mathcal F$ is pulled back from the coarse quotient, then
    every stabiliser necessarily acts trivially on the fibres of $\mathcal E$.
    Thus trivial stabiliser action is not merely sufficient but \textbf{necessary} for
    descent.

    This explains why universal bundles exist naturally on moduli stacks, but often
    fail to exist on coarse moduli spaces.
\end{remark}

\begin{example}[A point modulo $\mu_2$]
    Let $X=\Spec k$ and $G=\mu_2=\{\pm1\}$.  Then $X/G=\Spec k$ is a single point.
    Consider the trivial line bundle $\mathcal L=\mathcal O_X$ with the
    nontrivial linearisation given by
    \[
        -1 \cdot v = -v.
    \]
    Although $-1$ acts trivially on the base, it acts nontrivially on the fibre.

    If $\mathcal L$ descended to a line bundle $\mathcal M$ on $X/G$, then
    $\pi^*\mathcal M$ would carry the trivial $\mu_2$--action.  But $\mathcal L$
    carries the sign representation.  Hence no such $\mathcal M$ exists: $\mathcal
        L$ does not descend.  This is the simplest possible manifestation of the
    descent obstruction.
\end{example}

\begin{example}[A fixed point with nontrivial stabiliser action]
    Let $\mu_2$ act on $\A^1=\Spec k[x]$ by $x\mapsto -x$.  The quotient is again
    $\A^1$, with coordinate $t=x^2$.  Consider the trivial line bundle
    $\mathcal O_{\A^1}$ equipped with the linearisation
    \[
        (-1)\cdot s(x) := -s(-x).
    \]
    The point $x=0$ is fixed by all of $\mu_2$, and the stabiliser acts by $-1$ on the
    fibre over $0$.  If this bundle descended to the quotient, the fibre over
    $t=0$ would pull back to a line with trivial $\mu_2$--action, which is impossible.
    Thus the bundle does not descend.

    This example mirrors exactly what happens for universal bundles over coarse
    moduli spaces: automorphisms fix the moduli point but act nontrivially on the
    fibres, obstructing descent.
\end{example}

\subsection*{(c) Rank $2$ vector bundles of even degree}

We showed in the last chapter that when $\deg L$ is even the Gieseker points
$T_{E,s}$ of a semistable vector bundle $E \in SU_C(2,L)$ are semistable for the
action of $GL(N)$ on $\Alt_{N,2}(H^0(L))$ (Propositions~10.69 and~10.70).  In a
moment we will show that if $E$ is a stable vector bundle, then its Gieseker
points $T_{E,s}$ are $GL(N)$--stable.  This implies, in particular, that the
quotient variety
\[
    \Alt^{s}_{N,2}(H^0(L))/GL(N)
\]
has $SU_C(2,L)$ as its underlying set of points.  However, unlike the odd degree
case, there are now semistable vector bundles which are not stable, and as a
consequence the quotient variety
\[
    \Alt^{s}_{N,2}(H^0(L))/GL(N)
\]
is not complete.  It is contained as an open set in the projective variety
\[
    \overline{SU}_C(2,L) := \Alt^{ss}_{N,2}(H^0(L))//GL(N),
\]
and one can ask what the geometric points of this bigger variety correspond to
in terms of vector bundles. 

We will assume that $\deg L \ge 4g-2$.

\begin{proposition}
    The Gieseker points of a stable vector bundle are $GL(N)$--stable.
\end{proposition}

\begin{proof}
    We have already observed in the proof of Theorem~10.1 (Section~10.4(b)) that a
    Gieseker point $T_{E,s}$ of a stable bundle $E$ has a finite stabiliser, using
    Lemma~10.64 and the fact that $E$ is simple.  So we just have to show that the
    orbit of $T_{E,s}$ is closed.

    So suppose that $T \in \Alt_{N,2}(H^0(L))$ is in the closure $W$ of the orbit
    $GL(N)\cdot T_{E,s}$ of $E$.  We have seen in Section~10.4(b) that such $T$ is the
    Gieseker point of some semistable vector bundle $E'$.  Moreover, there exists a
    vector bundle $\mathcal E$ on $C \times W$ such that
    \[
        \mathcal E|_{C\times t} \cong E \quad \text{for $t$ in an open set of $W$}
        \qquad\text{and}\qquad
        \mathcal E|_{C\times t} \cong E' \quad \text{for $t$ in the boundary.}
    \]
    Namely, $\mathcal E$ is the restriction of the universal bundle $\mathcal Q$ which we constructed previously.

    Its fiber over $t$ has
    \[
        H^0(C, E^\vee \otimes \mathcal E_t) = \Hom(E, \mathcal E_t).
    \] If we apply semicontinuity to the bundle $E^\vee \otimes \mathcal E$, in particular dimensions cannot drop in specialization, so we see that
    \[
        \dim \Hom(E,E') \ge \dim \Hom(E,E) \ge 1,
    \]
    so that there exists a nonzero homomorphism $E \to E'$.  But this is then an
    isomorphism since we have seen previously that any nonzero homomorphism between semistable bundles of the same slope, with the source stable, is an isomorphism. Thus the orbit $GL(N)\cdot T_{E,s}$ is closed.
\end{proof}
If E is semistable but not stable of rank 2, then by definition there exists a proper subbundle
\[
    0 \;\longrightarrow\; L \;\longrightarrow\; E \;\longrightarrow\; M \;\longrightarrow\; 0
\]
with
\[
    \mu(L) = \mu(E) = \mu(M),
    \qquad\text{i.e.}\qquad
    \deg L = \deg M = \tfrac{1}{2}\deg E.
\]

Because $E$ has rank 2, any destabilising subbundle has rank 1, so this is the only possible shape.


\begin{definition}
    Given a bundle $E$ which is semistable but not stable, write $E$ as an extension and let
    \[
        \gr(E) = L \oplus M.
    \]
    If $E$ is stable, then $\gr(E) = E$.
\end{definition}

\begin{proposition}
    If $E,E'$ are rank $2$ vector bundles with the same determinant line bundle $L$
    and Gieseker points $T_{E,s}, T_{E',s'}$, then the following are equivalent:
    \begin{enumerate}[(i)]
        \item $\gr(E) \cong \gr(E')$
        \item $T_{E,s}$ and $T_{E',s'}$ are closure--equivalent under the action of
              $GL(N)$.
    \end{enumerate}
    Bundles $E,E'$ satisfying condition~(i) are said to be \textbf{$S$--equivalent}.
\end{proposition}

\begin{proof}
    The Gieseker orbit of a stable vector bundle is closed and
    of maximal dimension; so if either of $E,E'$ is stable, then by the polystable GIT package, condition (ii) is equivalent to $T_{E,s}$ and $T_{E',s'}$ being in
    the same $GL(N)$--orbit.  We therefore only need to consider the case where
    neither of $E,E'$ is stable; in other words, we assume that they are extensions
    of line bundles:
    \[
        0 \longrightarrow L \longrightarrow E \longrightarrow M \longrightarrow 0,
        \qquad \deg L = \deg M,
    \]
    \[
        0 \longrightarrow L' \longrightarrow E' \longrightarrow M' \longrightarrow 0,
        \qquad \deg L' = \deg M'.
    \]

    Since, by hypothesis, $\deg L = \deg M \ge 2g-1$, it follows that
    $h^0(L) = h^0(M) = \tfrac12 h^0(E)$.  So let $S=\{s_1,\dots,s_N\}$ be a basis of
    $H^0(E)$ in which $s_1,\dots,s_{N/2}$ are a basis of $H^0(L) \subset H^0(E)$.
    Then $T_{E,s}$ has the form
    \[
        \begin{pmatrix}
            0     & B \\
            - B^t & C
        \end{pmatrix}.
    \]

    Now let
    \[
        g(t) =
        \begin{pmatrix}
            \diag(t^{-1},\dots,t^{-1}) & 0                \\
            0                          & \diag(t,\dots,t)
        \end{pmatrix}
        \in SL(N).
    \]
    Then, as we have already seen in the proof of Proposition~10.66,
    \[
        \lim_{t\to 0} g(t)\,T_{E,s}\,g(t)^t
        =
        \begin{pmatrix}
            0     & B \\
            - B^t & 0
        \end{pmatrix}.
    \]

    But this is the Gieseker point of the decomposable vector bundle
    $L \oplus M = \gr(E)$, and we see that $T_{\gr(E),s}$ is contained in the closure
    of the orbit $GL(N)\cdot T_{E,s}$.  Hence we have shown that (i) implies (ii).

    For the converse, suppose that the two orbit closures have an intersection point:
    \[
        T \in \overline{GL(N)\cdot T_{E,s}} \cap \overline{GL(N)\cdot T_{E',s'}}.
    \]
    Then $T$ is a Gieseker point of some semistable vector bundle $F$, and as in the
    proof of Proposition~11.36, we can find a family of vector bundles $\mathcal E$
    which is equal to $E$ on an open set and jumps to $F$ on the boundary.  We then
    apply upper semicontinuity (Proposition~11.21) to the family
    $L^{-1} \otimes \mathcal E$, where $L \subset E$ is the same line subbundle as
    above.  This gives
    \[
        \dim \Hom(L,F) \ge \dim \Hom(L,E) \ge 1,
    \]
    so that $L$ is contained as a line subbundle in $F$.  By the same reasoning,
    $L'$ is also a line subbundle of $F$.  But since $F$ is semistable, this implies
    that either $L \cong L'$ or $F \cong L \oplus L'$.  Either way, we conclude that
    $\gr(E) \cong \gr(E')$.
\end{proof}

\subsection*{The polystable package in affine GIT}

Let $G$ be a linearly reductive algebraic group acting on an affine variety
$X = \Spec R$.

\medskip

\noindent
\textbf{Orbit closure equivalence.}
Define an equivalence relation on $X$ by
\[
    x \sim y
    \quad \Longleftrightarrow \quad
    \overline{G \cdot x} \cap \overline{G \cdot y} \neq \varnothing.
\]
Equivalence classes are called \textbf{closure--equivalence classes} or
\textbf{$S$--equivalence classes}.

\medskip

\noindent
\textbf{Affine quotient and invariants.}
Since $G$ is linearly reductive, the invariant ring $R^G$ is finitely generated,
and the affine GIT quotient
\[
    \pi : X \longrightarrow X//G := \Spec R^G
\]
exists.  The morphism $\pi$ is surjective, constant on orbit closures, and
satisfies the universal property of categorical quotients.

\medskip

\noindent
\textbf{Fibres of the quotient.}
For any $x,y \in X$,
\[
    \pi(x) = \pi(y)
    \quad \Longleftrightarrow \quad
    \overline{G \cdot x} \cap \overline{G \cdot y} \neq \varnothing.
\]
Thus the fibres of $\pi$ are exactly the closure--equivalence classes.

\smallskip

\textbf{Justification.}
Invariant functions are constant along orbits and hence on orbit closures, so
$\pi(x)=\pi(y)$ whenever the closures meet. Conversely, linear reductivity
implies that invariants separate disjoint closed orbits, so two points with the
same image must degenerate to a common closed orbit.

\medskip

\noindent
\textbf{Existence and uniqueness of polystable points.}
Each closure--equivalence class contains a unique closed orbit.  Moreover, this
closed orbit is contained in the closure of every orbit in the class.

\smallskip

Equivalently, for every $x \in X$ there exists a unique closed orbit
\[
    \mathcal O_{\min} \subset \overline{G \cdot x}.
\]
Points lying on closed orbits are called \textbf{polystable}.

\smallskip

\textbf{Justification.}
Every fibre of $\pi$ contains a closed orbit (Hilbert--Mumford).  If two closed
orbits lay in the same fibre, they could not be separated by invariant
functions, contradicting linear reductivity.

\medskip

\noindent
\textbf{Closed orbits and the quotient.}
The induced map
\[
    \{\text{closed $G$--orbits in $X$}\}
    \;\longleftrightarrow\;
    X//G
\]
is a bijection.

Thus the affine quotient parametrises closed orbits, and every orbit degenerates
uniquely to a closed one.

\medskip

\noindent
\textbf{Maximal dimension and stability.}
If an orbit is closed and has maximal dimension among all orbits, then it cannot
lie properly in the closure of any other orbit. Hence if such an orbit lies in a
closure--equivalence class, it must be the unique closed orbit in that class. More precisely:

\begin{lemma}
    Let $G$ act on a variety $X$.
    If $G \cdot x$ is closed and of maximal dimension, then:

    \[
        \overline{G \cdot y} \cap G \cdot x \neq \varnothing
        \;\Longrightarrow\;
        G \cdot y = G \cdot x.
    \]
\end{lemma}

\textbf{Justification.}
Orbit dimension is upper semicontinuous and strictly drops under nontrivial
specialisation. A closed orbit of maximal dimension therefore cannot be a
boundary orbit of another.

\medskip

\noindent
\textbf{Moduli interpretation.}
Closure--equivalence corresponds to having the same polystable degeneration.
The unique closed orbit in $\overline{G\cdot x}$ is the ``polystable
representative'' of $x$. In moduli problems, this is the object represented by
the point of the GIT quotient.

\subsection*{The structural heart of affine GIT}

Let $G$ be a linearly reductive algebraic group acting on an affine variety
$X=\Spec R$.

\begin{theorem}[Separation of closed orbits by invariants]\label{thm:separate-closed-orbits}
    If $Z_1,Z_2 \subset X$ are disjoint closed $G$--stable subsets, then there exists
    $f \in R^G$ such that
    \[
        f|_{Z_1} = 0,
        \qquad
        f|_{Z_2} = 1.
    \]
    In particular, if $G\cdot x$ and $G\cdot y$ are distinct closed orbits, then there
    exists $f \in R^G$ with $f(x) \neq f(y)$.
\end{theorem}

\begin{proof}[Idea of proof]
    Let $I_i = I(Z_i) \subset R$ be the radical $G$--stable ideals. Since $Z_1$ and
    $Z_2$ are disjoint, $I_1 + I_2 = R$. Choose $a \in I_1$, $b \in I_2$ with $a+b=1$.
    Applying the Reynolds operator
    \[
        \rho : R \longrightarrow R^G
    \]
    (which exists because $G$ is linearly reductive), we obtain
    \[
        \rho(a) + \rho(b) = 1,
        \qquad
        \rho(a) \in I_1 \cap R^G,
        \quad
        \rho(b) \in I_2 \cap R^G.
    \]
    Then $f := \rho(b)$ is invariant, vanishes on $Z_1$, and equals $1$ on $Z_2$.
\end{proof}

\begin{corollary}[Fibres of the affine quotient]\label{cor:fibres-are-closure-classes}
    Let $\pi : X \to X//G := \Spec R^G$ be the affine GIT quotient. Then for any
    $x,y \in X$,
    \[
        \pi(x) = \pi(y)
        \quad \Longleftrightarrow \quad
        \overline{G \cdot x} \cap \overline{G \cdot y} \neq \varnothing.
    \]
    Equivalently, the fibres of $\pi$ are exactly the closure--equivalence
    ($S$--equivalence) classes.
\end{corollary}

\begin{proof}[Idea of proof]
    Invariant functions are constant on orbits and hence on orbit closures, so the
    forward implication is formal.

    Conversely, suppose $\overline{G \cdot x}$ and $\overline{G \cdot y}$ are disjoint.
    By Theorem~\ref{thm:separate-closed-orbits}, there exists $f \in R^G$ separating
    them, so $f(x) \neq f(y)$ and hence $\pi(x) \neq \pi(y)$.
\end{proof}

\begin{corollary}[Polystable representatives]\label{cor:unique-closed-orbit}
    Every closure--equivalence class contains a unique closed orbit, and this closed
    orbit is contained in the closure of every orbit in the class.
\end{corollary}

\begin{proof}[Idea of proof]
    By Hilbert--Mumford, every orbit closure contains a closed orbit. If two distinct
    closed orbits lay in the same closure--equivalence class, they would map to the
    same point of $X//G$, contradicting Theorem~\ref{thm:separate-closed-orbits}.
\end{proof}

\medskip

\noindent
\textbf{Interpretation.}
The affine quotient $X//G$ parametrises closed orbits. Every orbit degenerates to
a unique closed one, called its \textbf{polystable representative}. Thus affine GIT
does not classify all orbits, but rather their canonical closed degenerations.

Recall that if $\xi$ is a line bundle, $\hat{\xi} = L \otimes \xi^{-1}$, and the multiplication map
\[
    H^0(\xi) \times H^0(\hat{\xi}) \longrightarrow H^0(L)
\]
is represented by a matrix $T$, then the vector bundle
\[
    E = \xi \oplus \hat{\xi}
\]
has as a Gieseker point the matrix
\[
    \begin{pmatrix}
        0       & T \\
        - T^{t} & 0
    \end{pmatrix}.
\]
This means that the map
\[
    \Pic^{d/2} C \;\longrightarrow\; \overline{SU}_C(2,L),
    \qquad
    \xi \longmapsto \xi \oplus \hat{\xi},
\]
is induced in the quotient by the map
\[
    \Mat_{N/2,1}\bigl(H^0(L)\bigr)
    \;\longrightarrow\;
    \Alt_{N,2}\bigl(H^0(L)\bigr),
    \qquad
    T \longmapsto
    \begin{pmatrix}
        0       & T \\
        - T^{t} & 0
    \end{pmatrix}.
\]
Thus $\overline{SU}_C(2,L)$ contains $SU_C(2,L)$ as an open set, with complement (the semistable boundary) equal to the image of $\Pic^{d/2} C$ (called the \textbf{Kummer variety}).


\begin{remark}
    When $\deg L$ is even, the obstruction to the existence of a universal bundle on the coarse moduli space $SU_C(2,L)$ remains. Ramanan showed that in this case there is no universal bundle.
    In particular, the moduli stack $\mathcal{SU}_C(2,L)$ is a $\mu_2$-gerbe over $SU_C(2,L)$.
    \begin{enumerate}
        \item Its class lies in $H^2(SU_C(2,L), \mu_2) \subset \mathrm{Br}(SU_C(2,L))$.
        \item A universal bundle exists iff this class vanishes.
        \item Ramanan proved it is nonzero when $\deg L$ is even.
    \end{enumerate}
\end{remark}

\subsection{Gerbes and the Brauer obstruction}

\begin{definition}
    A \textbf{$G$--gerbe over $X$} is a stack
$\pi:\mathcal G \to X$ such that:

\begin{enumerate}
    \item (Local nonemptiness) For every $x\in X$, there exists an \'etale
          neighbourhood $U\to X$ with $\mathcal G(U)\neq\varnothing$.

    \item (Local connectedness) For any $U\to X$ and any two objects
          $\xi,\eta\in\mathcal G(U)$, there exists an \'etale cover $\{U_i\to U\}$ such that
          $\xi|_{U_i}\cong \eta|_{U_i}$.

    \item (Banded by $G$) For every object $\xi\in\mathcal G(U)$, its automorphism sheaf
          $\underline{\Aut}(\xi)$ is canonically identified with $G|_U$. In particular, these identifications are part of the data of the gerbe.
\end{enumerate}

\end{definition}
Let $X$ be a scheme (or algebraic space), and let $G$ be a sheaf of groups on the
\'etale site of $X$ (e.g.\ $G=\mu_n$).  

Equivalently, étale--locally on $X$ the stack $\mathcal G$ is equivalent to the
trivial gerbe $BG\times X$, but globally it may be twisted.

\begin{example}
If $E$ is a stable rank $2$ vector bundle with fixed determinant, then
$\Aut(E)=\{\pm 1\}\cong\mu_2$.  Consequently, the moduli stack
$\mathcal{SU}_C(2,L)\to SU_C(2,L)$ is a $\mu_2$--gerbe.
\end{example}

\begin{definition}
The \textbf{cohomological Brauer group} of $X$ is
\[
    \Br(X) := H^2_{\et}(X,\mathbb G_m).
\]
\end{definition}
It classifies $\mathbb G_m$--gerbes on $X$.  For $n$ invertible on $X$, the Kummer
sequence
\[
    0 \to \mu_n \to \mathbb G_m \xrightarrow{(\cdot)^n} \mathbb G_m \to 0
\]
gives an injection
\[
    H^2_{\et}(X,\mu_n) \hookrightarrow \Br(X),
\]
so $\mu_n$--gerbes determine $n$--torsion Brauer classes.

\paragraph{Gerbes and descent.}
A $G$--gerbe $\mathcal G\to X$ is \textbf{neutral} (or \textbf{trivial}) if it admits a
global section $X\to\mathcal G$.  Equivalently, $\mathcal G\simeq BG\times X$.
For a $\mu_n$--gerbe, neutrality is equivalent to the vanishing of its class in
$H^2(X,\mu_n)$.

If $\mathcal G$ is the moduli stack of objects with automorphism group $G$, then
a universal object descends to $X$ if and only if the associated gerbe is
neutral.  Nontriviality of the gerbe is precisely the obstruction to the
existence of a universal family.

\paragraph{Vector bundle moduli.}
The universal bundle exists naturally on the moduli stack
$\mathcal{SU}_C(2,L)$.  Its descent to $C\times SU_C(2,L)$ is obstructed by the
$\mu_2$--gerbe $\mathcal{SU}_C(2,L)\to SU_C(2,L)$, which defines a Brauer class
\[
    \beta \in H^2(SU_C(2,L),\mu_2) \subset \Br(SU_C(2,L)).
\]
Ramanan proved that $\beta\neq 0$ when $\deg L$ is even, hence no universal bundle
exists on the coarse moduli space in that case.

\section{The Verlinde formula for rank 2}
Recall that for a complex vector bundle $E$ of rank $r$ on a compact complex manifold $X$, one has Chern classes $c_i(E) \in H^{2i}(X,\Z)$ for $i=1,\dots,r$ and a total Chern class
\[
    c(E) = 1 + c_1(E) + c_2(E) + \cdots + c_r(E).
\] which is multiplicative on short exact sequences \begin{align*}
    0 \;\longrightarrow\; E' \;\longrightarrow\; E \;\longrightarrow\; E'' \;\longrightarrow\; 0
    &\implies c(E) = c(E') \cup c(E'')
\end{align*}

Now we make a weird definition. Consider the product \begin{align*}
    \prod_{i=1}^\infty \frac{x_i}{1 - e^{-x_i}}
\end{align*} whose degree $m$ term is a symmetric homogeneous polynomial of degree $m$ in the variables $x_i$. By the theory of symmetric polynomials, this can be expressed as a polynomial in the elementary symmetric polynomials $\sigma_j$ in the $x_i$. Let us denote this polynomial by $\td_m(\sigma_1,\dots,\sigma_m)$. Then \begin{align*}
    \prod_{i=1}^\infty \frac{x_i}{1 - e^{-x_i}} &= \sum_{m=0}^\infty \td_m(\sigma_1,\dots,\sigma_m) \\
    &= 1 + \frac{1}{2}\sigma_1 + \frac{1}{12}(\sigma_1^2 + \sigma_2) + \frac{1}{24}\sigma_1 \sigma_2 + \cdots
\end{align*}

\begin{definition}
    [Todd class] The Todd class of a complex vector bundle $E$ is defined as
    \[
    \td(E) = \sum_{m=0}^\infty \td_m\bigl(c_1(E), c_2(E), \dots, c_m(E)\bigr)
    \]
    where $\td_m$ are the polynomials defined above in terms of the Chern classes of $E$. The Todd class of a complex manifold $X$ is defined as $\td(TX)$ where $TX$ is the holomorphic tangent bundle of $X$. The Todd characteristic number of $X$ is defined as \begin{align*}
        \td(X) = \int_X \td(TX)
    \end{align*}
\end{definition}

Like the Chern class, the Todd class is multiplicative on exact sequences.
\begin{align*}
    0 \;\longrightarrow\; E' \;\longrightarrow\; E \;\longrightarrow\; E'' \;\longrightarrow\; 0
    &\implies \td(E) = \td(E') \cup \td(E'')
\end{align*}
The Todd class can also be computed as follows. We can write
\[
\frac{x}{1 - e^{-x}} = e^{x/2} \cdot \frac{x/2}{\sinh(x/2)}.
\]

The last factor here is a power series in \(x^2\). Letting \(\pi_m(x_i) = \sigma(x_i^2)\) be the
elementary symmetric polynomial in the squares \(x_1^2, x_2^2, \dots\), the homogeneous term of degree \(2m\) in the expansion of the infinite product
\[
\prod_{i=1}^{\infty} \frac{x_i/2}{\sinh(x_i/2)}
\]
is a polynomial in \(\pi_1, \dots, \pi_m\), which we denote by
\(\widehat{A}_m(\pi_1, \dots, \pi_m)\).
Then we have
\[
\prod_{i=1}^{\infty} \frac{x_i}{1 - e^{-x_i}}
= e^{\sigma_1/2} \sum_{m=1}^{\infty} \widehat{A}_m(\pi_1, \pi_2, \dots, \pi_m).
\]

Given a vector bundle \(E\) of rank \(r\), set
\[
p_i(E) = (-1)^i c_{2i}(E \oplus E^\vee), \qquad i = 1, \dots, r,
\]
called the \textbf{Pontryagin classes} of \(E\).
In this language, the Todd class is given by
\[
\td(E) = e^{c_1(E)/2} \sum_{m=1}^{\infty} \widehat{A}_m\bigl(p_1(E), p_2(E), \dots, p_m(E)\bigr).
\]

\begin{theorem}
    [Hirzebruch--Riemann--Roch] Let $E$ be a holomorphic vector bundle on a compact complex manifold $X$. Then \begin{align*}
        \chi(X,E) = \int_X \ch(E) \cup \td(X)
    \end{align*} where $\ch(E)$ is the Chern character of $
    E$.
\end{theorem}

\subsection{Grothendieck–Riemann–Roch}
Given a proper morphism \( f : X \to Y \) between algebraic varieties and an
algebraic vector bundle \(\mathcal E\) on \(X\), one can define its direct image
sheaf \(f_*\mathcal E\) and its higher direct images \(R^q f_*\mathcal E\),
for \(q>0\), on \(Y\) (Definition~11.10). In general, these sheaves are not vector
bundles, but let us pretend that they are. (This assumption is justified by the
fact that the direct images are coherent sheaves. This means that they have
resolutions by locally free sheaves, so that the Chern character is defined
using the additivity property on exact sequences.) Then it is possible to
express the alternating sum of Chern characters
\[
\sum_q (-1)^q \ch\bigl(R^q f_*\mathcal E\bigr) \in H^*(Y,\mathbb Q)
\]
in terms of the Chern character \(\ch \mathcal E\) and the ``Todd character'' of
\(f\). This generalises Hirzebruch--Riemann--Roch (where \(Y\) is a point) and is
called the Grothendieck--Riemann--Roch Theorem. Here we shall consider only the
special case \(X = C \times Y\), where \(C\) is a curve and where
\(f : C \times Y \to Y\) is the projection to the second factor. 

First, take an affine open cover \(U_i = \Spm A_i\) of \(Y\) and consider the
restriction of \(\mathcal E\) to each open set \(C \times \Spm A_i\). This defines
a vector bundle on the curve \(C_{A_i}\) in the sense of Definition~11.9 and so
determines \(A_i\)-modules \(H^0(\mathcal E_i)\) and \(H^1(\mathcal E_i)\). If
these are locally free modules, then they define, by gluing over the open cover,
a pair of vector bundles \(f_*\mathcal E\) and \(R^1 f_*\mathcal E\), which are the
direct image bundles.

K\"unneth's Theorem says that the cohomology ring of the product \(C \times Y\)
is the tensor product of the cohomology rings of \(C\) and \(Y\). In other words,
there is an isomorphism of graded rings
\[
H^*(C \times Y) \cong H^*(C) \otimes H^*(Y).
\]

The cohomology of the curve \(C\) has three components \(H^0(C), H^1(C)\) and
\(H^2(C)\), and so the cohomology of \(C \times Y\) is the direct sum of three
pieces:
\[
H^0(C) \otimes H^*(Y), \qquad
H^1(C) \otimes H^*(Y), \qquad
H^2(C) \otimes H^*(Y).
\]

The Chern character of a vector bundle \(\mathcal E\) on the product \(C \times Y\)
can therefore be decomposed as
\[
\ch \mathcal E
= \ch^{(0)} \mathcal E + \ch^{(1/2)} \mathcal E + \ch^{(1)} \mathcal E,
\qquad
\ch^{(i)} \mathcal E \in H^{2i}(C) \otimes H^*(Y).
\]

Now \(\ch^{(0)} \mathcal E\) can be viewed as an element of \(H^*(Y)\). On the
other hand, the fundamental class of a point \(\eta \in H^2(C,\mathbb Z)\)
determines a natural isomorphism
\[
\int_C : H^2(C,\mathbb Q) \xrightarrow{\sim} \mathbb Q,
\]
and using this isomorphism we can view \(\ch^{(1)} \mathcal E\) as an element of
\(H^*(Y)\), which we will denote by \(\ch^{(1)}(\mathcal E)/\eta\).

\begin{theorem}[Grothendieck--Riemann--Roch for $f : C \times Y \to Y$]
\label{thm:GRR-curves}
Let $f : C \times Y \to Y$ be the projection, where $C$ is a curve of genus $g$.
Then
\[
\ch(f_*\mathcal E) - \ch(R^1 f_*\mathcal E)
=
\ch^{(1)}(\mathcal E)/\eta - (g-1)\ch^{(0)}(\mathcal E).
\]
When $Y$ is a point this is nothing but the Riemann--Roch formula.
\end{theorem}
From this formula one reads off the Chern character (of $f_*\mathcal E$ or
$R^1 f_*\mathcal E$). The following remark is useful for recovering the Chern
classes from the Chern character in general. First, for any vector bundle $F$
consider the derived Chern class
\[
c'(F) = \sum_{i \ge 1} i\, c_i(F).
\]

From the relations between elementary symmetric polynomials and sums of powers,
this derived class satisfies
\[
c'(F)
=
\left(\sum_{i \ge 1} (-1)^{i-1} i! \, \ch_i(F)\right) c(F)
\]
Since the logarithmic derivative $c'(F)/c(F)$ of the Chern class is additive on
exact sequences, it follows that the Chern class can be written
\[
c(F) = \exp \int \frac{c'(F)}{c(F)}
\]
where $\int$ denotes the formal inverse of the derivative. Hence
\[
c(F) = \exp \int \sum_{i \ge 1} (-1)^{i-1} i! \, \ch_i(F)
\]

\subsection{Riemann Roch with involution}

Suppose that the curve $C$ has an involution $\sigma : C \to C$ (that is, an
automorphism of order $2$), and suppose that $\sigma$ lifts to a vector bundle
$E$ (still with order $2$). In this case $\sigma$ acts also on the vector spaces
$H^0(E)$ and $H^1(E)$. We will denote the invariant and anti-invariant subspaces
(that is, the eigenspaces of $\pm 1$) by $H^i(E)^+$ and $H^i(E)^-$, and write
\[
\chi(E)^{\pm} = \dim H^0(E)^{\pm} - \dim H^1(E)^{\pm}.
\]

Moreover, if $p \in C$ is a fixed point of $\sigma$, then the involution acts in
the fibre $E_p$ and we denote by $E_p^+$ and $E_p^-$ the invariant and
anti-invariant subspaces. 

\begin{proposition}
Given an involution $\sigma$ acting on a vector bundle $E$ as above, we have
\[
\chi(E)^+ - \chi(E)^-
=
\frac{1}{2} \sum_{p \in \Fix(\sigma)}
\bigl( \dim E_p^+ - \dim E_p^- \bigr),
\]
where $\Fix(\sigma) \subset C$ denotes the set of fixed points.
\end{proposition}

\begin{proof}
Let $\sigma$ act on $H^i(C,E)$. Since $\sigma^2=1$, its eigenvalues are $\pm1$, hence
\[
\Tr(\sigma\mid H^i(C,E))=\dim H^i(C,E)^+ - \dim H^i(C,E)^-
\]
Therefore
\[
\chi(E)^+ - \chi(E)^-
=\sum_{i=0}^1 (-1)^i \Tr(\sigma\mid H^i(C,E))
=:L(\sigma,E),
\]
the Lefschetz number of $(\sigma,E)$.

By the holomorphic Lefschetz fixed point formula (Atiyah--Bott),
\[
L(\sigma,E)
=\sum_{p\in \Fix(\sigma)}
\frac{\Tr(\sigma\mid E_p)}{\det(1-d\sigma_p\mid T_pC)}.
\]
For $p\in\Fix(\sigma)$ the derivative $d\sigma_p$ acts on the one-dimensional
space $T_pC$ by $\pm1$. Since $\Fix(\sigma)$ is discrete, we must have
$d\sigma_p=-1$, so $\det(1-d\sigma_p)=1-(-1)=2$.

Moreover, the action of $\sigma$ on the fibre $E_p$ decomposes it as
$E_p=E_p^+\oplus E_p^-$ with eigenvalues $\pm1$, hence
$\Tr(\sigma\mid E_p)=\dim E_p^+ - \dim E_p^-$.
Substituting gives
\[
\chi(E)^+ - \chi(E)^-
= \frac12 \sum_{p\in\Fix(\sigma)} \bigl(\dim E_p^+ - \dim E_p^- \bigr),
\]
as claimed.
\end{proof}

\begin{example}
Let $F$ be any vector bundle on the curve $C$.
\begin{enumerate}[(i)]
\item If $E := F \oplus \sigma^*F$, then $\chi(E)^+ - \chi(E)^- = 0$.
\item If $E := F \otimes \sigma^*F$, then
\[
\chi(E)^+ - \chi(E)^-
=
\frac{1}{2} \sum_{p \in \Fix(\sigma)}
\left( \dim S^2 F_p - \dim \wedge^2 F_p \right)
=
\frac{1}{2}\,\rank(F)\,|\Fix(\sigma)|.
\]
\end{enumerate}
\end{example}

\begin{example}[Explanation of Example~12.25]
Let $F$ be a vector bundle on $C$, and recall the fixed–point formula
\[
\chi(E)^+ - \chi(E)^-
=
\frac12 \sum_{p \in \Fix(\sigma)} \bigl(\dim E_p^+ - \dim E_p^- \bigr).
\tag{$\ast$}
\]

\medskip

\noindent\textbf{(i) $E = F \oplus \sigma^*F$.}
If $p \in \Fix(\sigma)$, then canonically
\[
E_p = F_p \oplus F_p,
\]
and the lifted involution acts by exchanging the two summands:
\[
(v,w) \longmapsto (w,v).
\]
Hence
\[
E_p^+ = \{(v,v)\} \cong F_p,
\qquad
E_p^- = \{(v,-v)\} \cong F_p.
\]
Therefore $\dim E_p^+ = \dim E_p^- = \rank(F)$, so each fixed point contributes
zero to $(\ast)$. It follows that
\[
\chi(E)^+ - \chi(E)^- = 0.
\]

\medskip

\noindent\textbf{(ii) $E = F \otimes \sigma^*F$.}
If $p \in \Fix(\sigma)$, then
\[
E_p = F_p \otimes F_p,
\]
and the involution acts by permuting the tensor factors:
\[
v \otimes w \longmapsto w \otimes v.
\]
The $\pm 1$ eigenspace decomposition is the standard one:
\[
E_p^+ = S^2 F_p,
\qquad
E_p^- = \wedge^2 F_p.
\]
Hence
\[
\dim E_p^+ - \dim E_p^- = \dim S^2 F_p - \dim \wedge^2 F_p.
\]
If $r=\rank(F)$, then
\[
\dim S^2 F_p = \frac{r(r+1)}{2},
\qquad
\dim \wedge^2 F_p = \frac{r(r-1)}{2},
\]
so
\[
\dim S^2 F_p - \dim \wedge^2 F_p = r.
\]
Substituting into $(\ast)$ gives
\[
\chi(E)^+ - \chi(E)^-
=
\frac12 \sum_{p \in \Fix(\sigma)} r
=
\frac12\,\rank(F)\,|\Fix(\sigma)|.
\]
\end{example}

Let $f : F \to F'$ be a homomorphism of rank $2$ vector bundles, and suppose that
the two bundles have an isomorphic determinant line bundle. Then, by tensoring
the dual map $f^\vee : {F'}^\vee \to F^\vee$ with $\det F = \det F'$, we obtain a
bundle map
\[
f^{\mathrm{adj}} := f^\vee \otimes 1_{\det} : F' \to F,
\]
called the \textbf{adjoint} of $f$.

\begin{example}
Suppose that $F$ is a rank $2$ vector bundle on $C$ whose determinant is
$\sigma$-invariant, that is, $\det F \cong \sigma^*\det F$. Given any (local)
homomorphism $f : F \to \sigma^*F$, we can pull back the adjoint map
$f^{\mathrm{adj}} : \sigma^*F \to F$ to obtain a homomorphism
$\sigma^*f^{\mathrm{adj}} : F \to \sigma^*F$. Moreover, the mapping
\[
f \longmapsto \sigma^*f^{\mathrm{adj}},
\]
composed with itself, recovers $f$. It therefore defines a lift of $\sigma$ to
the vector bundle
\[
E := \Hom(F,\sigma^*F) = F^\vee \otimes \sigma^*F.
\]

At a fixed point $p \in C$, the invariant subspace $E_p^+$ is $1$--dimensional
(spanned by the identity endomorphism in $F_p$) while the anti-invariant
subspace $E_p^-$ is the $3$--dimensional space $\mathfrak{sl}(F_p)$ of tracefree
endomorphisms. It follows that
\[
\chi(E)^+ - \chi(E)^- = -|\Fix(\sigma)|.
\]
\end{example}
Let $Y$ be a complete nonsingular variety, and suppose that the involution $\sigma \times 1_Y$ of
$C \times Y$ lifts to a vector bundle $E$ on the product. Then the involution
acts in the direct images $\pi_*E$ and $R^1\pi_*E$, where
$\pi : C \times Y \to Y$ is the projection, and we set
\[
\chi_\pi(E)^{\pm}
\;:=\;
\ch(\pi_*E)^{\pm} - \ch(R^1\pi_*E)^{\pm}.
\]

For each fixed point $p \in C$ the vector bundle $E|_{p \times Y}$ decomposes
into invariant and anti-invariant subbundles $E^{\pm}|_{p \times Y}$, and we
have:

\begin{proposition}
\label{prop:global-involution-GRR}
\[
\chi_\pi(E)^+ - \chi_\pi(E)^-
\;=\;
\frac{1}{2} \sum_{p \in \Fix(\sigma)}
\Bigl( \ch(E^+|_{p \times Y}) - \ch(E^-|_{p \times Y}) \Bigr).
\]
\end{proposition}

Let $\cN^\pm_C$ denote the moduli space of rank $2$ stable vector bundles on $C$ with fixed
determinant of even or odd degree, respectively.  

\subsection*{12.3 \ The standard line bundle and the Mumford relations}

Let $C$ be a (complete nonsingular algebraic) curve of genus $g$, let $L$ be a
line bundle on $C$ of odd degree, and let $\mathcal U$ be a universal vector
bundle on the product $C \times SU_C(2,L)$. For every $E \in SU_C(2,L)$ the
restriction to $C \times [E]$ of the determinant line bundle $\det \mathcal U$
is isomorphic to $L$, and so $\det \mathcal U$ can be expressed as a tensor
product,
\[
\det \mathcal U \cong L \boxtimes \Phi,
\tag{12.20}
\]
where $\Phi$ is some line bundle pulled back from $SU_C(2,L)$. It is enough to
take $\Phi$ to be the direct image of $\det \mathcal U \otimes L^{-1}$. In particular, we have the following discussion.
    \begin{lemma}[Splitting of line bundles trivial on fibres]
Let $S$ be a variety and let $p_1 : C \times S \to C$, $p_2 : C \times S \to S$ be the
projections. Let $M \in \Pic(C \times S)$ be a line bundle such that for every
$s \in S$,
\[
M|_{C \times \{s\}} \cong \mathcal O_C.
\]
Then there exists a unique line bundle $\Phi \in \Pic(S)$ such that
\[
M \cong p_2^*\Phi.
\]
Moreover, one may take $\Phi := p_{2*}M$.
\end{lemma}


\begin{proof}
    Since for every $s \in S$ one has $M_s \cong \mathcal O_C$, we have
\[
h^0(C,M_s)=1, \qquad h^1(C,M_s)=g,
\]
independent of $s$. By Grauert's theorem \ref{thm:grauert} (cohomology and base change for curves), which applies since the dimensions above are constant, it follows that $p_{2*}M$ is locally free of rank $1$, hence a line bundle on $S$, and that for every $s \in S$ the natural base–change map
\[
(p_{2*}M)\otimes k(s) \;\xrightarrow{\sim}\; H^0(C,M_s)
\]
is an isomorphism.

There is a natural adjunction (evaluation) morphism
\[
\mathrm{ev} : p_2^*(p_{2*}M) \longrightarrow M.
\]
Restricting to the fibre over $s$, this becomes the evaluation map
\[
H^0(C,M_s) \otimes \mathcal O_C \longrightarrow M_s,
\]
which is an isomorphism since $M_s \cong \mathcal O_C$ and $h^0(C,M_s)=1$.
Therefore $\mathrm{ev}$ is an isomorphism on every fibre of $p_2$. Since both
sheaves are line bundles, $\mathrm{ev}$ is an isomorphism globally. Hence
\[
M \cong p_2^*(p_{2*}M).
\]

Finally, uniqueness follows because $p_2^* : \Pic(S) \to \Pic(C \times S)$ is
injective: if $p_2^*\Phi_1 \cong p_2^*\Phi_2$, restricting to $\{x\}\times S$
for any $x\in C$ gives $\Phi_1 \cong \Phi_2$.
\end{proof}


Let $\mathcal U$ be the universal bundle on $C \times SU_C(2,L)$. For each point
$[E] \in SU_C(2,L)$,
\[
(\det \mathcal U)|_{C \times \{[E]\}} \cong \det(E) \cong L
\]
by definition of $SU_C(2,L)$. Consider therefore the line bundle
\[
M := \det \mathcal U \otimes p_1^*L^{-1} \in \Pic(C \times SU_C(2,L)).
\]
For every $[E]$,
\[
M|_{C \times \{[E]\}} \cong \mathcal O_C.
\]
By the lemma, there exists $\Phi \in \Pic(SU_C(2,L))$ such that
\[
M \cong p_2^*\Phi.
\]
Undoing the definition of $M$, we obtain
\[
\det \mathcal U \cong p_1^*L \otimes p_2^*\Phi = L \boxtimes \Phi,
\]
as claimed.


In particular, the first Chern class of $\mathcal U$ can be
written:
\[
c_1(\mathcal U) = c_1(L) \otimes 1 + 1 \otimes \phi,
\qquad
\phi = c_1(\Phi) \in H^2(SU_C(2,L)).
\]

We are now going to define a natural line bundle on the moduli variety
$SU_C(2,L)$; this is the line bundle $\mathcal L$ which appears in the
left-hand side of~(12.2). First of all, we observe that by using the universal
bundle $\mathcal U$ we can construct, given a point $p \in C$, two vector
bundles on $SU_C(2,L)$, namely the direct image and the restriction:
\[
\pi_*\mathcal U, \qquad \mathcal U|_{p \times SU_C(2,L)},
\]
where $\pi : C \times SU_C(2,L) \to SU_C(2,L)$ is projection on the second
factor. In order for $\pi_*\mathcal U$ to be a vector bundle we assume that
$d := \deg L$ is sufficiently large, so that
\[
H^1(E) = 0 \quad \text{for all } E \in SU_C(2,L),
\tag{numerical condition}
\]
and again by Grauert's theorem \ref{thm:grauert} the direct image is locally free. 
Set $N := d + 2 - 2g$; this is the dimension of $H^0(E)$ for each $E \in SU_C(2,L)$ and hence the rank of $\pi_*\mathcal U$.

\begin{definition}
    \leavevmode
\label{def:standard-line-bundle}
\begin{enumerate}[(i)]
\item If $L \in \Pic C$ satisfies the condition~(numerical condition), then the
\textbf{standard line bundle} (not to be confused with the canonical line
bundle!) on the moduli variety $SU_C(2,L)$ is defined to be
\[
\mathcal L
:=
\bigl(\det(\mathcal U|_{p \times SU_C(2,L)})\bigr)^N
\otimes
\bigl(\det \pi_*\mathcal U\bigr)^{-2}.
\]
This is also called the \textbf{determinant line bundle}.

\item If $L \in \Pic C$ does not satisfy the condition~(numerical condition), then we choose a
line bundle $\xi$ on $C$ of sufficiently high degree that $L \otimes \xi^2$
satisfies~(numerical condition). The standard line bundle on $SU_C(2,L)$ is then defined to be
the pullback of $\mathcal L$ under the isomorphism
\[
SU_C(2,L) \xrightarrow{\ \otimes \xi\ } SU_C(2,L \otimes \xi^2).
\]
\end{enumerate}
\end{definition}

\begin{remark}
The definition of the standard line bundle is guided by the construction of the
\emph{determinant of cohomology}.  Given a family of curves
\[
\pi : C \times S \longrightarrow S
\]
and a vector bundle $\mathcal U$ on $C \times S$, one can canonically associate a
line bundle on $S$,
\[
\lambda(\mathcal U) := \det R\pi_*(\mathcal U)
:= \det(\pi_*\mathcal U) \otimes \det(R^1\pi_*\mathcal U)^{-1}.
\]
Fibrewise, this is
\[
\lambda(\mathcal U)|_s
= \det H^0(C,E_s) \otimes \det H^1(C,E_s)^{-1}.
\]

When $H^1(C,E_s)=0$ for all $s$, this simplifies to $\lambda(\mathcal U)=\det(\pi_*\mathcal U)$.
Thus $\det(\pi_*\mathcal U)$ should be thought of as the basic determinant line
bundle attached to the universal family.

However, the universal bundle $\mathcal U$ is not unique: if $M$ is a line bundle
on $SU_C(2,L)$, then
\[
\mathcal U' := \mathcal U \otimes p_2^*M
\]
is again universal.  A meaningful line bundle on the moduli space must therefore
be invariant under this operation.

Assume $H^1(E)=0$ for all $E$, and set $N=\rank(\pi_*\mathcal U)=h^0(C,E)$.

Then
\[
\pi_*(\mathcal U') = \pi_*(\mathcal U \otimes p_2^*M)
\cong \pi_*\mathcal U \otimes M
\]
by the projection formula.  Taking determinants,
\[
\det(\pi_*\mathcal U') \cong \det(\pi_*\mathcal U) \otimes M^N.
\tag{1}
\]

On the other hand, restricting to $p \times SU_C(2,L)$,
\[
\mathcal U'|_{p \times SU}
= \mathcal U|_{p \times SU} \otimes M,
\]
so
\[
\det(\mathcal U'|_{p \times SU})
\cong \det(\mathcal U|_{p \times SU}) \otimes M^2.
\tag{2}
\]

Combining (1) and (2), we find
\[
\bigl(\det(\mathcal U'|_{p \times SU})\bigr)^N
\otimes
\bigl(\det(\pi_*\mathcal U')\bigr)^{-2}
=
\bigl(\det(\mathcal U|_{p \times SU})\bigr)^N
\otimes
\bigl(\det(\pi_*\mathcal U)\bigr)^{-2}.
\]

Thus the line bundle
\[
\mathcal L
:=
\bigl(\det(\mathcal U|_{p \times SU_C(2,L)})\bigr)^N
\otimes
\bigl(\det \pi_*\mathcal U\bigr)^{-2}
\]
is \emph{independent of the choice of universal bundle}.  This invariance property
is the fundamental reason for this particular combination.
\end{remark}


\subsection{Proof of the Verlinde formula}
The Verlinde formula computes $\dim H^0(SU_C(2,L), \mathcal L^{\otimes k})$ the dimension of the space of generalized theta functions.
\begin{theorem}
    [Verlinde formula for odd degree]
Let $C$ be a curve of genus $g \ge 2$, and let $L$ be a line bundle of odd
degree on $C$. Then for every integer $k \ge 0$,
\begin{equation}
\dim H^0(\mathcal N_C^{-}, \mathcal L^{\otimes k})
=
(k+1)^{g-1}
\sum_{j=1}^{2k+1}
\frac{(-1)^{j-1}}{\bigl(\sin \frac{j\pi}{2k+2}\bigr)^{2g-2}}.
\tag{Verlinde formula}
\end{equation}
\end{theorem}
We will prove the Verlinde formula via the
intersection numbers among cohomology classes
\[
c_1(\mathcal L) = \alpha \in H^2(\mathcal N_C^-), \qquad
\beta \in H^4(\mathcal N_C^-), \qquad
\gamma \in H^2(\mathcal N_C^-),
\]
defined by means of the universal vector bundle. These classes are called the
Newstead classes, and the full intersection formula among them is
\begin{equation*}
(\alpha^m \beta^n \gamma^p)
=
(-1)^n 2^{2g-2-p}
\frac{g!\, m!}{(g-p)!}\,
b_{g-1-n-p},
\qquad
m + 2n + 3p = 3g - 3,
\end{equation*}
where $b_k \in \mathbb Q$ is the rational number defined by the Taylor expansion
\[
\frac{x}{\sin x} = \sum_k b_k x^{2k}
\quad\text{for } k \ge 0,
\qquad
b_k = 0 \text{ for } k < 0.
\]

In fact, we only need this for $p=0$,
\begin{equation}
(\alpha^m \beta^n)
=
(-1)^n m!\, 4^{\,g-1}\, b_{g-1-n},
\qquad
m + 2n = 3g - 3,
\tag{intersection formula}
\end{equation}

There are three main ingredients in the proof of the Verlinde formula:
\begin{enumerate}
    \item The standard line bundle. It is intristically defined on the moduli space, independent of the choice of universal bundle, generates the Picard group, is ample, and its curvature is the natural symplectic form.
    \item Grothendieck–Riemann–Roch on the universal family \begin{align*}
\chi(\mathcal L^k) = \int_{SU} \mathrm{ch}(\mathcal L^k)\,\mathrm{td}(SU).
    \end{align*}
    \item Structure of the cohomology ring (Newstead–Mumford relations). 
The cohomology ring of $SU_C(2,L)$ is generated by universal classes
$\alpha,\beta,\gamma$
with explicit relations.
\end{enumerate}


\subsubsection*{Origin of the Mumford relations (topological argument)}

A key point is that, as a differentiable manifold, $\mathcal N_C^-$ depends only
on the genus $g$ of $C$.  By the theorem of Narasimhan--Seshadri and Donaldson,
$\mathcal N_C^-$ is (up to a finite ambiguity coming from the determinant) the
moduli space of equivalence classes of representations of $\pi_1(C)$ in $SU(2)$.
In particular, its diffeomorphism type, and hence its real cohomology ring,
depends only on the topology of $C$.

Consequently, to prove cohomological relations on $\mathcal N_C^-$, we are free
to replace $C$ by any convenient curve of genus $g$.  We choose $C$ to be
hyperelliptic, with hyperelliptic involution $\sigma : C \to C$.

\medskip

Using the universal bundle on $C \times \mathcal N_C^-$, one constructs a natural
vector bundle $W$ on $\mathcal N_C^-$ as the first direct image of a suitable line
bundle on the product.  The involution $\sigma$ induces an involution on this
family, hence on $W$, and we obtain a decomposition into invariant and
anti-invariant parts
\[
W = W^+ \oplus W^-.
\]

By the Riemann--Roch theorem for curves with involution, the ranks of these
bundles can be computed purely from the fixed-point data of $\sigma$ on $C$.  One
finds
\[
\rank(W^+) = 3g+1, \qquad \rank(W^-) = g-1.
\]

In particular, since $W^-$ has rank $g-1$, its Chern classes satisfy the
vanishing
\[
c_i(W^-) = 0 \qquad \text{for all } i \ge g.
\tag{12.5}
\]

\medskip

On the other hand, by Grothendieck--Riemann--Roch with involution, each Chern
class $c_i(W^-)$ can be expressed explicitly as a universal polynomial in the
Newstead classes $\alpha,\beta,\gamma \in H^*(\mathcal N_C^-)$ arising from the
universal bundle.  Therefore the identities \emph{(12.5)} translate into
explicit polynomial relations among $\alpha,\beta,\gamma$.

These resulting relations are precisely the \emph{Mumford relations}.  Thus,
the Mumford relations ultimately arise from the existence of the hyperelliptic
involution and the fact that the anti-invariant part of a natural pushforward
bundle has unexpectedly small rank.
\section{References}
\begin{enumerate}
    \bibitem{mukai}
    Mukai, S.,
    \textit{An Introduction to Invariants and Moduli},
    Cambridge Studies in Advanced Mathematics, vol. 81,
    Cambridge University Press, Cambridge, 2003.
\end{enumerate}

\end{document}