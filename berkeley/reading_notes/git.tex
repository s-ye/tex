\documentclass[12pt]{article}
\usepackage[english]{babel}
\usepackage[utf8x]{inputenc}
\usepackage[T1]{fontenc}
\usepackage{listings}
\usepackage{bookmark}
\usepackage{tikz}

\makeatletter
\def\input@path{{../../style/}}
\makeatother

\usepackage{../../style/quiver}
\usepackage{../../style/scribe}
\usepackage{fancyhdr}

\usepackage{parskip} % Automatically respects blank lines
\usepackage{booktabs} % For \addlinespace command
\setlength{\parskip}{1em} % Adds more space between paragraphs
\setlength{\parindent}{0pt} % Removes paragraph indentation

\begin{document}


\lhead{Songyu Ye}
\rhead{\today}
\cfoot{\thepage}

\title{Geometric invariant theory}

\author{Songyu Ye}
\date{\today}
\maketitle


\begin{abstract}

These are reading notes for \emph{Geometric Invariant Theory} by Mumford, Fogarty and Kirwan.
\end{abstract}

\section{Example}
Consider the action of $G = \PGL(n+1)$ on $X = (\mathbb{P}^n)^{m+1}$ using the line bundle
\[
L = \mathcal{O}_{\mathbb{P}^n}(1)^{\boxtimes (m+1)} = \mathcal{O}(1, \ldots, 1) = \pi_1^* \mathcal{O}_{\mathbb{P}^n}(1) \otimes \cdots \otimes \pi_{m+1}^* \mathcal{O}_{\mathbb{P}^n}(1)
\] where $\pi_i : X \to \mathbb{P}^n$ is the projection to the $i$-th factor.
We need to lift the geometric action of $G$ on $X$ to a linear action on $L$. The natural group that acts linearly on $\mathcal{O}_{\mathbb{P}^n}(1)$ is $\GL(n+1)$. There is no canonical way to make an element of $\PGL(n+1)$ act linearly on the fibers of $\mathcal{O}(1)$, because it is only defined up to scalar. The exact sequence is
\[
1 \to \mathbb{G}_m \to \GL(n+1) \to \PGL(n+1) \to 1
\]
Why don't we stay with $\GL(n+1)$ instead of $\PGL(n+1)$? Because the center $\mathbb{G}_m$ acts trivially on $X$ and this introduces a useless symmetry which breaks stability.

We can restrict to the subgroup $\SL(n+1) \subset \GL(n+1)$, which kills most of the scalars except for the finite center $\mu_{n+1}$. We want the linearization to descend to $\PGL(n+1)$, so we need the center $\mu_{n+1} = \ker(\SL(n+1) \to \PGL(n+1)
)$ to act trivially on the fibers of $L$. 

On $\mathcal O_{\mathbb P^n}(1)$, a scalar $\zeta I \in \mu_{n+1} \subset SL(n+1)$ acts as multiplication by $\zeta$ on each fiber.

On 
\[
\mathcal L \;=\; \mathcal O_{\mathbb P^n}(1)^{\boxtimes m},
\]
it therefore acts as multiplication by $\zeta^m$.

Hence the $SL(n+1)$--linearization of $\mathcal L$ factors through $PGL(n+1)$ if and only if every $\zeta \in \mu_{n+1}$ acts trivially on $\mathcal L$, i.e.
\[
\zeta^m = 1 \quad \text{for all } \zeta \text{ with } \zeta^{n+1} = 1.
\]
This holds if and only if
\[
n+1 \mid m.
\] More generally, if we consider the line bundle
\[  L_i = \mathcal O_{\mathbb P^n}(a_i) \] on the $i$-th factor, then the same argument shows that the $SL(n+1)$--linearization of
\[ \mathcal L = \bigotimes_{i=1}^{m+1} \pi_i^* L_i = \mathcal O_{\mathbb P^n}(a_1, \ldots, a_{m+1}) \] descends to $PGL(n+1)$ if and only if
\[n+1 \mid \sum_{i=1}^{m+1} a_i.\]

In any case, by means of these linearizations we can define invariant sections of all the sheaves $\mathcal L_{\boldsymbol\alpha}$. To construct such invariant sections, let $X_0,\ldots,X_n$ be the canonical sections of $\mathcal O_{\mathbb P^n}(1)$ on $\mathbb P^n$. Let
\[
X_i^{(j)} = \pi_j^*(X_i)
\]
be the induced sections of $L_j$.


\begin{definition}
For all sequences $\boldsymbol\alpha = (\alpha_0,\ldots,\alpha_n)$ of integers such that $0 \le \alpha_i \le m$, let
\[
D_{\alpha_0,\ldots,\alpha_n}
=
\det\bigl( X_i^{(\alpha_j)} \bigr)_{0 \le i,j \le n}
\]
be the section of $L_{\alpha_0} \otimes \cdots \otimes L_{\alpha_n}$ obtained by addition and tensor product as in the determinant.
\end{definition}

 It is evident that $D_{\alpha_0,\ldots,\alpha_n}$ is an invariant section of
$L_{\alpha_0} \otimes \cdots \otimes L_{\alpha_n}$. The non-vanishing of suitable $D$'s defines the open sets we are looking for. Explicitly, a point of $(\mathbb{P}^n)^{m+1}$ is a tuple
\[
(p_0,\ldots,p_m), \quad p_j = [v_j], \quad v_j \in k^{n+1} \setminus \{0\}.
\]

Choose homogeneous lifts $v_j$. Put them as columns of a matrix
\[
M = [v_0 \mid v_1 \mid \cdots \mid v_m] \in \text{Mat}_{n+1,m+1}.
\]
Then 
\[
D_{\alpha_0,\ldots,\alpha_n} = \det(v_{\alpha_0}, \ldots, v_{\alpha_n}).
\]
The fact that $D_{\alpha_0,\ldots,\alpha_n}$ is well defined as a section of $L_{\alpha_0} \otimes \cdots \otimes L_{\alpha_n}$ follows from the following properties:
\begin{itemize}
\item $D_{\alpha_0,\ldots,\alpha_n} = 0$ iff the points $p_{\alpha_0}, \ldots, p_{\alpha_n}$ lie in a hyperplane.
\item Under $g \in \GL(n+1)$, all minors are multiplied by $\det(g)$.
\item Rescaling columns rescales the corresponding minors.
\end{itemize}

\begin{definition}
An $R$-partition of $\{0,1,\ldots,n\}$ is an ordered set of subsets $S_1,\ldots,S_\nu$ of $\{0,1,\ldots,n\}$ such that
\begin{enumerate}
\item[(i)] $S_i \cap (S_1 \cup \cdots \cup S_{i-1})$ consists of exactly one integer for $i=2,\ldots,
\nu$
\item[(ii)] $\bigcup_i S_i = \{0,1,\ldots,n\}$.
\end{enumerate}
\end{definition}

\begin{definition}
Given an $R$-partition $R=\{S_1,\ldots,S_\nu\}$, let $U_R \subset (\mathbb P^n)^{m+1}$ be the open subset defined by
\begin{enumerate}
\item[(i)] $D_{0,1,\ldots,n} \neq 0$,
\item[(ii)] for all $k$ between $1$ and $\nu$, and for all $i \in S_k$,
\[
D_{0,\ldots,i-1,i+1,\ldots,n,n+k} \neq 0
\]
\end{enumerate}
\end{definition}

Not only is $U_R$ affine, but the whole structure of the action of $PGL(n+1)$ on $U_R$ can be described explicitly. On each open set $U_R$, a configuration of points in $(\mathbb{P}^n)^{m+1}$ is uniquely the same thing as
\begin{enumerate}
\item a projective frame, and
\item a collection of free affine parameters
\end{enumerate}


\begin{proposition}
Let $R=\{S_1,\ldots,S_\nu\}$ be an $R$-partition of $\{0,1,\ldots,n\}$. Let $\PGL(n+1)$ act on $PGL(n+1)\times \mathbb A^{n\nu - n}$ by the product of left translation on itself and the trivial action on the affine space. Then there is a $\PGL(n+1)$-linear isomorphism:
\[
U_R \cong \PGL(n+1) \times \mathbb A^{n\nu - n}.
\]

Hence $U_R$ is a globally trivial principal fibre bundle with respect to the action of $\PGL(n+1)$, with base space $\mathbb A^{n\nu - n}$.
\end{proposition}
\end{document}