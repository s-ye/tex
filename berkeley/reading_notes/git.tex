\documentclass[12pt]{article}
\usepackage[english]{babel}
\usepackage[utf8x]{inputenc}
\usepackage[T1]{fontenc}
\usepackage{listings}
\usepackage{bookmark}
\usepackage{tikz}

\makeatletter
\def\input@path{{../../style/}}
\makeatother

\usepackage{../../style/quiver}
\usepackage{../../style/scribe}
\usepackage{fancyhdr}

\usepackage{parskip} % Automatically respects blank lines
\usepackage{booktabs} % For \addlinespace command
\setlength{\parskip}{1em} % Adds more space between paragraphs
\setlength{\parindent}{0pt} % Removes paragraph indentation

\begin{document}


\lhead{Songyu Ye}
\rhead{\today}
\cfoot{\thepage}

\title{Geometric invariant theory}

\author{Songyu Ye}
\date{\today}
\maketitle


\begin{abstract}

These are reading notes for \emph{Geometric Invariant Theory} by Mumford, Fogarty and Kirwan.
\end{abstract}

\section{Example}
Consider the action of $G = \PGL(n+1)$ on $X = (\mathbb{P}^n)^m$ using the line bundle
\[
L = \mathcal{O}_{\mathbb{P}^n}(1)^{\boxtimes m} = \mathcal{O}(1, \ldots, 1).
\]
We need to lift the geometric action of $G$ on $X$ to a linear action on $L$. The natural group that acts linearly on $\cO_{\mathbb{P}^n}(1)$ is $\GL(n+1)$, which is a central extension of $\PGL(n+1)$ by $\mathbb{G}_m$. Thus we have a linear action of $\GL(n+1)$ on $L$.
\end{document}