\documentclass[12pt]{article}
\usepackage[english]{babel}
\usepackage[utf8x]{inputenc}
\usepackage[T1]{fontenc}
\usepackage{listings}
\usepackage{bookmark}
\usepackage{tikz}
\usepackage[all]{xy}
\makeatletter
\def\input@path{{../../style/}}
\makeatother

\usepackage{../../style/quiver}
\makeatletter
\def\input@path{{../../style/}}
\makeatother

\usepackage{../../style/scribe}
\usepackage{fancyhdr}
\DeclareMathOperator{\Al}{Al}
\DeclareMathOperator{\Chmb}{Ch}

\usepackage{parskip} % Automatically respects blank lines
\setlength{\parskip}{1em} % Adds more space between paragraphs
\setlength{\parindent}{0pt} % Removes paragraph indentation

\begin{document}


\lhead{Songyu Ye}
\rhead{\today}
\cfoot{\thepage}

\title{Title}

\author{Songyu Ye}
\date{\today}
\maketitle


\begin{abstract}
Abstract
\end{abstract}

\tableofcontents

\section{Principal $G$-Bundles on Affine Curves}

It is a consequence of a theorem of Harder \cite[Satz 3.3]{Har67} that generically trivial principal $G$-bundles on a smooth affine curve $C$ over an arbitrary field $k$ are trivial if $G$ is a semisimple and simply connected algebraic group. When $k$ is algebraically closed and $G$ reductive, generic triviality, conjectured by Serre, was proved by Steinberg \cite{Ste65} and Borel-Springer \cite{BS68}.

It follows that principal bundles for simply connected semisimple groups over smooth affine curves over algebraically closed fields are trivial. This fact (and a generalization to families of bundles \cite{DS95}) plays an important role in the geometric realization of conformal blocks for smooth curves as global sections of line bundles on moduli-stacks of principal bundles on the curves (see the review \cite{Sor96} and the references therein).

\subsection{Derived Pushforward of Admissible Complexes}
Recall that a $\mathbb{C}^\times$-bundle on a nodal curve $\Sigma$ is defined by a $\mathbb{C}^\times$-bundle on the normalization of $\Sigma$ together with an identification of the two fibers at the preimages of each node. The stack $\text{Bun}_{\mathbb{C}^\times}(g, I)$ of $\mathbb{C}^\times$-bundles over the universal stable curve fails to be complete, because the space of identifications over a given node is isomorphic to $\mathbb{C}^\times$.

Following Gieseker \cite{Gie16} and Caporaso \cite{Cap09}, we add new strata which represent the limits where an identification goes to zero or infinity, by allowing projective lines carrying the line bundle $\mathcal{O}_{\mathbb{P}^1}(1)$ to appear at the nodes. 

The resulting stack denoted $\widetilde{\mathcal{M}}_{g,I}([pt/\mathbb{C}^\times])$ is complete but not separated, i.e., the limit of a family of bundles exists but may not be unique. 

This stack classifies maps from marked nodal curves to the quotient stack $[pt / \mathbb{C}^\times]$. These are the moduli stacks of principal $\mathbb{C}^\times$-bundles on such curves.

\red{My understanding is that the construction of this compactifications goes through Pablo's wonderful compactification of loop groups. But it seems this is not necessary in the $\C^*$ case. Then Teleman gives a modular interpretation of this compactification in terms of Gieseker bundles.}

\begin{theorem}
The derived pushforward $RF_* \alpha$ of an admissible complex $\alpha$ along the bundle-forgetting map $F : \widetilde{\mathcal{M}}_{g,I}([pt/G]) \to \overline{\mathcal{M}_{g,I}}$ is a bounded complex of coherent sheaves.
\end{theorem}

This theorem is a relative version over varying curves of the analogous finiteness result for $\text{Bun}_G(\Sigma)$ in \cite[34]{}.

\begin{enumerate}
    \item Section~1 reviews basic facts about nodal curves and principal $\mathbb{C}^\times$-bundles.
    The moduli stack $\widetilde{\mathcal{M}}_{g,I}([pt/\mathbb{C}^\times])$ of Gieseker bundles on stable curves is introduced with some key examples (small $g$ and $|I|$).

    \item Section~2 proves some basic facts about the geometry of our stack:
    it is an Artin stack, is stratified by topological type, and is complete (but not separated).

    \item Section~3 gives an (étale-local) presentation of $\widetilde{\mathcal{M}}_{g,I}([pt/\mathbb{C}^\times])$ as a quotient $A/G$ (where $G \simeq (\mathbb{C}^\times)^V$). We identify a stable subspace $A^\circ \subset A$ which leads to a smooth and proper quotient moduli space over $\mathcal{M}_{g,I}$.

    \item Section~4 refines the stratification by topological type by tracking the nodes smoothed under deformations. We use this to stratify $A/G$ by distinguished spaces $Z$, $W$ which are affine space bundles over their fixed-point loci under subgroups of $G$.

    \item Section~5 reviews the admissible $K$-theory classes and estimates the weights of the fixed-point fibers of subgroups of $G$.

    \item Section~6 uses a local cohomology vanishing argument to finish the proof of the main theorem.

    \item Section~7 constructs a moduli stack which we expect to carry Gromov-Witten invariants for $[X/\mathbb{C}^\times]$.
\end{enumerate}



We expect to recover the Gromov-Witten invariants of GIT quotients from our invariants by applying the Chern character to certain limits of our invariants. This was done for
smooth curves and G-bundles in Teleman-Woodward \cite{TW}.

\subsection{Construction of $\widetilde{\mathcal{M}}_{g,I}([pt/\mathbb{C}^\times])$}

One sees from this how $\mathbb{C}^\times$-bundles on $C$ can become singular in families: the space of gluing isomorphisms at a node $\sigma \in C$ is a copy of $\mathbb{C}^\times$; in a family, these isomorphisms can tend to the limit points $0$ and $\infty$. As a result, the stack $\text{Bun}_{\mathbb{C}^\times}(g, I)$ of $\mathbb{C}^\times$-bundles on stable marked curves of type $(g, I)$ fails the valuative criterion for completeness. This will be a problem for integration of cohomology or K-theory classes.



We always work over $\C$.  In everything that follows, $(C,\sigma_i)$ is a
family of prestable marked curves over a finitely generated complex base
scheme $B$.  More precisely, $\pi : C \to B$ is a flat proper morphism whose
fibers are connected complex projective curves of genus $g$ with at worst
nodal singularities, carrying a collection of smooth marked points
$\sigma_i : B \to C$ which are indexed by an ordered set $I$.

A point is \emph{special} if it is a node or a marked point.  Special points
are required to be pairwise disjoint.  We shall always assume that any
rational component of $C$ has at least \emph{two} special points.

We reserve the notation $(\Sigma,\sigma_i)$ for families of \emph{stable}
marked curves.  Recall that a marked curve is stable if each component of
genus $0$ carries at least $3$ special points and each component of genus
$1$ carries at least $1$ special point.  The \emph{stabilization morphism}
$\operatorname{st} : C \to C^{st}$ blows down every unstable rational curve
in $C$.  Stabilization can be implemented by a pluricanonical embedding and
thus works in families.

\begin{definition}[Modification of curves]\label{def:modification}
A morphism $m : C \to \Sigma$ of prestable curves is a \textbf{modification} if
\begin{enumerate}
\item $m$ is an isomorphism away from the preimage of the nodes of $\Sigma$, and
\item the preimage under $m$ of every node in $\Sigma$ is either a node or a
  $\mathbb P^1$ with two special points.
\end{enumerate}
A \textbf{modification of a family} $f : \Sigma \to B$ of marked prestable curves
is a morphism $m : C \to \Sigma$ such that, for each geometric $b\in B$,
the induced map
\[
m_b : C_b \longrightarrow \Sigma_{f(b)}
\]
is a modification.
\end{definition}

\begin{remark}\label{rem:modifications}
\begin{enumerate}
\item
Finding modifications with desirable properties---such as smoothness of
the total space $C$---may require us to change the base $B$; the reader
can be entrusted to write out the defining diagram.

\item
Modifications of marked curves do not introduce $\mathbb P^1$'s at marked
points, only at nodes.  The marked points in a family $\Sigma$ lift uniquely
to the modification, and will sometimes be denoted by the same symbol.
\end{enumerate}
\end{remark}

\begin{definition}[Gieseker bundle]\label{def:gieseker-bundle}
Let $(\Sigma,\sigma_i)$ be a stable marked curve.  A
\textbf{Gieseker $\C^\times$--bundle} on $(\Sigma,\sigma_i)$ is a pair
$(m,\mathcal P)$ consisting of
\begin{enumerate}
\item a modification $m : (C,\sigma_i) \to (\Sigma,\sigma_i)$, and
\item a principal $\C^\times$--bundle $p : \mathcal P \to C$,
\end{enumerate}
which satisfy the \textbf{Gieseker condition}:
\begin{enumerate}
\item[1.] the restriction of $\mathcal P$ to every unstable $\mathbb P^1$
  has degree $1$.
\end{enumerate}
\end{definition}
\red{We should learn what the Gieseker condition says. Is it a general formula or just an intuition?}

\begin{definition}
The stack $\widetilde{\mathcal M}_{g,I}([pt/\C^\times])$ of
\emph{Gieseker $\C^\times$–bundles on stable genus $g$, $I$–marked curves}
is a fibred category (over $\C$–schemes).  Its objects are tuples
$(B,C,\sigma_i,\mathcal P)$ consisting of
\begin{enumerate}
\item a test scheme $B$;
\item a flat projective family $\pi : C \to B$ of pre-stable, genus $g$
  curves with marked points $\sigma_i : B \to C$, $i\in I$; and
\item a principal $\C^\times$–bundle $p : \mathcal P \to C$ defining a family
  of Gieseker bundles on the stabilization $C \to C^{st}$.
\end{enumerate}
The morphisms in this category are commutative diagrams
\[
\xymatrix{
\mathcal P' \ar[r]^{\tilde f} \ar[d]_{p'} & \mathcal P \ar[d]^{p} \\
C' \ar[r]^{f} \ar[d]_{\pi'} & C \ar[d]^{\pi} \\
B' \ar[r] & B
}
\]
where $\tilde f$ is $\C^\times$-equivariant and $C' = B'\times_B C$ and the morphism of curves $f$ respects the marked points.
\end{definition}
There is a natural forgetful morphism
\[
F : \widetilde{\mathcal M}_{g,I}([pt/\C^\times]) \longrightarrow \overline{\mathcal M}_{g,I}
\]
which sends a Gieseker bundle $(C,\sigma_i,\mathcal P)$ to the stabilized curve $(C^{st},\sigma_i)$.

\begin{example}
    
\end{example}

\section{Pablo's modifications}

\subsection{Setup}

For any torus $T$ we have the lattice of characters $\hom(T,\C^\times)$ and
co-characters $\hom(\C^\times,T)$.  Further, for
$(\eta,\chi) \in \hom(\C^\times,T) \times \hom(T,\C^\times)$ we set
\[
\langle \eta,\chi\rangle := \chi\circ\eta \in \Z.
\]

For $T\subset G$ a maximal torus and for $\eta\in\hom(\C^\times,T)$ the set
\[
P(\eta) := \{ g\in G \mid \lim_{t\to0}\eta(t)g\eta(t)^{-1} \text{ exists}\}
\]
is a subgroup.  A \textbf{parabolic} subgroup is any subgroup
$P\subset G$ conjugate to some $P(\eta)$.

We can apply the same construction for
$\eta\in\hom(\C^\times,\C^\times\times T)$ to get a subgroup
$P(\eta)\subset L^\times G$.  A \textbf{parahoric} subgroup is any group
conjugate to one of the $P(\eta)$.  By abuse of notation, we use $P(\eta)$
to denote its image under the projection $L^\times G\to LG$.
Parahoric subgroups of $LG$ are any subgroups conjugate to one of the
$P(\eta)$.

Parabolic and parahoric subgroups come with natural factorizations
$P(\eta)=L(\eta)U(\eta)$ known as a Levi decomposition:
\[
L(\eta) = \{ g\in G \mid \lim_{t\to0}\eta(t)g\eta(t)^{-1}=g\},
\qquad
U(\eta) = \{ g\in G \mid \lim_{t\to0}\eta(t)g\eta(t)^{-1}=1\}.
\]
A simple example comes from $\eta_0:\C^\times\to\C^\times\times T$ defined
by $\eta_0(t)=(t,1)$.  Then $\eta_0(t)g(z)\eta_0(t)^{-1}=g(tz)$ and
\[
P(\eta_0)=G[[z]] = G(\C[[z]]) =: L^+G.
\]
The Levi factorization is $G\cdot N$ where $N$ is the kernel of the map
\[
G[[z]] \xrightarrow{\,z\mapsto 0\,} G.
\]

By $\mathfrak t_\Q$ we denote $\hom(\C^\times,T)\otimes_\Z\Q$.  The Weyl
chamber is defined as
\[
\Chmb := \{\eta\in\mathfrak t_\Q \mid \langle \alpha_i,\eta\rangle \ge 0\}.
\]
It is a simplicial cone whose faces are given by
$\{\langle \alpha,\eta\rangle =0 \mid \alpha\in I\}$ for subsets
$I\subset\{\alpha_1,\dots,\alpha_r\}$.

Similarly, we have the affine Weyl chamber
\[
\Chmb^{\mathrm{aff}} = \{\eta\in \Q\oplus\mathfrak t_\Q \mid
\langle \alpha_i,\eta\rangle >0\};
\]
now the faces are in bijection with subsets $\{\alpha_0,\dots,\alpha_r\}$.
It is convention to instead work with the affine Weyl alcove
\[
\Al := \Chmb^{\mathrm{aff}}\cap(1\oplus\mathfrak t_\Q)
     = \{\eta\in\mathfrak t_\Q \mid 0\le \langle \alpha_i,\eta\rangle,\,
                                  \langle\theta,\eta\rangle \le 1\}.
\]
A \textbf{face} $F$ of $\Al$ is $F' \cap (1\oplus\mathfrak t_\Q)$ where $F'$ is
a face of $\Chmb^{\mathrm{aff}}$.

Any $\eta\in \Chmb$ determines a fractional co-character $\C^\times\to T$ but
nevertheless a well-defined parabolic $P(\eta)$.  Any parabolic is conjugate
to some $P(\eta)$ and if $\eta,\eta'$ are in the interior of the same face
then $P(\eta)=P(\eta')$.  Similarly any $\eta\in \Al$ determines a parahoric
$P(\eta)\subset LG$.  Any parahoric is conjugate either to $P(\eta)$ or to
$P(-\eta)$.  Let
\[
\Al_e = \{\eta\in \Al \mid \langle\theta,\eta\rangle =1\}.
\]
If $\eta\in \Al_e$ the resulting parahoric is called \textbf{exotic}.
Alternatively, the inclusion \[\{\alpha_1,\dots,\alpha_r\}\subset
\{\alpha_0,\dots,\alpha_r\}\] defines a map from faces of $\Chmb$ to those of
$\Al$.  The faces missed by $\Chmb$ are exactly those contained in $\Al_e$.

The exotic parahorics give rise to moduli spaces of torsors on curves which
are not isomorphic with moduli spaces of $G$-bundles.  Informally then the
exotic parahorics can be viewed as geometry only visible to $LG$.

The ordered simple roots $\{\alpha_0,\alpha_1,\dots,\alpha_r\}$ determine
ordered vertices $\{\eta_0,\dots,\eta_r\}$ determined by the conditions
\[
\langle \eta_i,\alpha_j\rangle = 0 \text{ for } i\neq j
\quad\text{and}\quad
\langle \eta_0,\alpha_0\rangle =1.
\]
If we write $\theta = \sum_{i=1}^r n_i\alpha_i$ and set $n_0=1$ then one can
check these conditions can be expressed as
\begin{equation}
\label{eq:alpha-eta}
\langle \alpha_i,\eta_j\rangle = \frac{1}{n_i}\delta_{i,j}.
\end{equation}
Now for each $I\subset\{0,\dots,r\}$ we define
\[
\eta_I = \sum_{i\in I}\eta_i.
\]
The alcove $\Al$ is a compact convex polytope whose faces are in bijection
with conjugacy classes of parahoric subgroups of $LG$. For each
$I\subset\{0,\dots,r\}$ the cocharacter
\[
\eta_I := \sum_{i\in I}\eta_i
\]
lies in the relative interior of the face of $\Al$ corresponding to the
complement of $I$, meaning that $\langle \alpha_j,\eta_I\rangle =0$ for $j\notin I$
and hence determines a parahoric subgroup
\[
P(\eta_I)\subset LG.
\]
Equivalently, $\eta_I:\C^\times\to\C^\times\times T$ is a one--parameter
subgroup whose conjugation action is used to define $P(\eta_I)$ as the subgroup
on which the limit $\lim_{t\to0}\eta_I(t)g\eta_I(t)^{-1}$ exists. Note that if $I=\varnothing$ we take $\eta_I$ to be the trivial co-character.
Finally, we set
\begin{equation}
\label{eq:PI-UI-LI}
\begin{aligned}
P_I &= P(\eta_I), &\qquad P_{\bar I} &= P(-\eta_I),\\
U_I &= U(\eta_I), &\qquad U_{\bar I} &= U(-\eta_I),\\
L_I &= L(\eta_I) = L(-\eta_I).
\end{aligned}
\end{equation}
    \begin{example}
        We work out the case $G=\SL_2$ in detail. Take the standard maximal torus
    \[
    T = \left\{
    \begin{pmatrix}
    t & 0 \\ 0 & t^{-1}
    \end{pmatrix}
    \;\middle|\; t\in\C^\times
    \right\} \cong \C^\times.
    \]
    A cocharacter $\eta\in\hom(\C^\times,T)$ is determined by an integer $m$:
    \[
    \eta_m:\C^\times\to T,\qquad
    \eta_m(t)=\begin{pmatrix} t^m & 0 \\ 0 & t^{-m}\end{pmatrix}.
    \]
    So $\hom(\C^\times,T)\cong\Z$. A character $\chi\in\hom(T,\C^\times)$ is also determined by an integer $k$:
    \[
    \chi_k\begin{pmatrix} t & 0 \\ 0 & t^{-1}\end{pmatrix} = t^k.
    \]

    The pairing $\langle\eta_m,\chi_k\rangle = \chi_k\circ\eta_m$ is
    \[
    \langle\eta_m,\chi_k\rangle = km \in \Z.
    \]
    The simple (finite) root $\alpha$ corresponds to the character $\chi_2$, so if we identify $\mathfrak t_\Q \cong \Q$ using the basis $\eta_1$, then
    \[
    \langle \alpha,\eta_m\rangle = 2m.
    \]
    We can (and usually do) renormalize so that $\langle\alpha,\eta_1\rangle = 1$, but the picture is the same: $\mathfrak t_\Q$ is a line and the Weyl chamber is the half-line $m\ge 0$.


    Take $\eta(t)=\eta_1(t)=\operatorname{diag}(t,t^{-1})$. For $g=\begin{pmatrix} a & b \\ c & d\end{pmatrix}\in \SL_2$, we have
    \[
    \eta(t)g\eta(t)^{-1} = \begin{pmatrix} t & 0\\ 0 & t^{-1}\end{pmatrix}
    \begin{pmatrix} a & b \\ c & d\end{pmatrix}
    \begin{pmatrix} t^{-1} & 0\\ 0 & t\end{pmatrix} = \begin{pmatrix}
    a & t^2 b\\ t^{-2} c & d
    \end{pmatrix}.
    \]
    As $t\to0$, this has a limit if and only if $c=0$. So
    \[
    P(\eta) = \left\{
    \begin{pmatrix} a & b \\ 0 & d\end{pmatrix}\in \SL_2
    \right\} = \text{upper Borel } B.
    \]

    The Levi and unipotent parts are $L(\eta)=$ diagonal torus (the copy of $T$), and $U(\eta)=$ strictly upper triangular unipotent matrices. Similarly $P(-\eta)$ is the lower Borel.

    The Weyl chamber $\Ch$ is $\{\eta_m \mid m\ge 0\}\subset \Q$. All nonzero $m>0$ lie in the interior of the same cone, so $P(\eta_m)$ is always conjugate to $B$. The ``faces'' of the cone are: the origin $\{0\}$ and the open half-line $\{m>0\}$. At $\eta=0$, $P(0)=G$; in the open face we get the Borel. This is the finite-type picture behind the general definition.

    For the affine root system $\widehat{\mathfrak{sl}}_2$ (type $A_1^{(1)}$): there are two simple affine roots $\alpha_0,\alpha_1$, the highest finite root is $\theta=\alpha_1$, and the extended Cartan is $\Q\oplus \mathfrak t_\Q$. Restricting to the slice $1\oplus\mathfrak t_\Q$ (the ``height 1'' slice) identifies the affine Weyl alcove
    \[
    \Al = \{\eta\in\mathfrak t_\Q \mid 0\le\langle\alpha_1,\eta\rangle,\,
    \langle\theta,\eta\rangle\le 1\}.
    \]
    For $\mathfrak{sl}_2$, $\theta=\alpha_1$, so this reduces to
    \[
    \Al = \{\eta\in\mathfrak t_\Q \mid 0\le \langle\alpha,\eta\rangle\le 1\}.
    \]
    Identifying $\mathfrak t_\Q\cong\Q$ so that $\langle\alpha,\eta\rangle$ is literally the coordinate, we get $\Al = [0,1] \subset \Q$. 
You can write a cocharacter as a pair
    \[
    \eta(t) = (t^m, \eta_T(t)),
    \]
    where $t^m \in \C^\times$ and $\eta_T(t) \in T$ where first component rescales the loop parameter $(t^m \cdot g)(z) = g(t^m z)$.
    The vertices $\eta_0,\eta_1$ are the endpoints $0$ and $1$. The interior $0<\langle\alpha,\eta\rangle<1$ corresponds to the "Iwahori" parahoric (the analogue of a Borel in the loop group).

    In the loop group $LG = \SL_2(\C((z)))$, the choice $\eta_0(t)=(t,1)$ rescales the loop variable and gives
    \[
    P(\eta_0) = \{ g(z)\in G(\C((z))) \mid \lim_{t\to0} g(tz)\ \text{exists in }G(\C((z)))\} =
    \SL_2(\C[[z]]) = L^+\SL_2,
    \]
    the standard maximal parahoric corresponding to the vertex $\eta_0$.  
    The other vertex $\eta_1$ corresponds to the cocharacter $\eta_1(t)=(t, \eta_T(t))$ where $\eta_T(t)=\operatorname{diag}(t,t^{-1})$. Then \[
(\eta_1(t)\cdot g)(z)
=
\begin{pmatrix}
a(tz) & t^{2} b(tz)\\
t^{-2} c(tz) & d(tz)
\end{pmatrix}\] so the limit as $t\to0$ exists if and only if $c(z)$ vanishes at $z=0$. Thus the parahoric subgroup is
    \[
    P(\eta_1) = \left\{
    \begin{pmatrix} a(z) & b(z) \\ c(z) & d(z)\end{pmatrix}
    \in \SL_2(\C((z))) \mid c(z) \in z^2\C[[z]], z^2b(z)\in\C[[z]], a(z),d(z)\in\C[[z]]
    \right\}.
    \]
    This parahoric subgroup is not conjugate to $L^+\SL_2$.
    
    Finally any choice of point $\eta$ in the interior of the interval $\Al=(0,1)$ gives conjugate parahoric subgroups, so choose $\eta=\tfrac12$. This gives \[(\eta(t)\cdot g)(z)
=
\begin{pmatrix}
a(tz) & t\, b(tz)\\
t^{-1} c(tz) & d(tz)
\end{pmatrix}\] so the limit as $t\to0$ exists if and only if $a(z),d(z)\in\C[[z]]$, $zb(z)\in \C[[z]]$, and $c(z)\in z\C[[z]]$. 

After intersecting with the positive loop group $\SL_2(\C[[z]])$, we get the Iwahori subgroup
\[
I = \{ g(z)\in \SL_2(\C[[z]]) \mid g(0)\in B\},
\]
where $B$ is the upper Borel subgroup of $\SL_2$.
\end{example}

\subsection{Twisted curves and admissible bundles}
Generally we work over $\Spec \C$ and a scheme will mean a scheme over
$\Spec \C$.  Let $S$ be a scheme.  We denote a flat family of curves
$C \to S$ as $C_S$.  If $B$ is an $S$--scheme then
$C_B := C_S \times_S B$.  For affine schemes $\Spec R \to S$ we write
$C_R$ for $C_{\Spec R}$.

Generally we work with a fixed curve over $\Spec \C$ or with a family of
curves over $S = \Spec \C[[s]]$.  Set $S^* = \Spec \C((s))$ and
$S_0 = \Spec \C = \Spec \C[[s]]/(s)$ the closed point.  Then $C_S$ always
denotes a curve with generic fiber $C_{S^*}$ smooth and special fiber
$C_0 := C_{S_0}$ nodal with unique node $p$.  We write $C_S - p$ for the
open subscheme $C_S\setminus\{p\}$.  We also assume $C_S$ is a regular
surface as a scheme over $\Spec \C$.

For any closed point $p$ in a scheme $Z$ we denote by $\widehat{\mathcal O}_{Z,p}$
the completion of $\mathcal O_{Z,p}$ with respect to the maximal ideal.
We often use $D$ to denote a formal neighborhood of a point in a curve.
The cases that will arise are:
\begin{itemize}
\item If $p\in C$ is a smooth curve,
      $\widehat{\mathcal O}_{C,p}\cong \C[[z]]$ and we set
      $D=\Spec \C[[z]]$.

\item If $p\in C_0$ is the node,
      $\widehat{\mathcal O}_{C,p} \cong \C[[x,y]]/(xy)$ and we set
      $D_0 = \Spec \C[[x,y]]/(xy)$.

\item If $p\in C_S$ is the node,
      $\widehat{\mathcal O}_{C_S,p} \cong \C[[x,y,s]]/(xy - s)$
      and we set $D_S = \Spec \C[[x,y,s]]/(xy - s)$.
\end{itemize}

For $k\ge 2$ and $k$th roots $u,v$ of $x,y$ we set
\[
D^{\frac1k}_S = \Spec \C[[u,v]].
\]

The last case arises as follows.  We first notice that if we base change
$D_S$ under $s\mapsto s^k$ then $D_S$ becomes
\[
\Spec \C[[x,y,s]]/(xy - s^k).
\]
If we let $\mu_k$ denote the $k$th roots of unity then
\[
\Spec \C[[x,y,s]]/(xy - s^k) \;=\; D_S^{\frac1k}/\mu_k
\]
where $\zeta\in\mu_k$ acts by $(u,v) \mapsto (\zeta u,\zeta^{-1}v)$.
A basic strategy we employ is to replace the curve
$\Spec \C[[x,y,s]]/(xy - s)$ with the orbifold or twisted curve
\[
D_S^{\frac1k}/\mu_k.
\]

\begin{remark}
    Recall that a course moduli space of a stack $\mathcal{X}$ is an algebraic space $X$ with a morphism $\pi:\mathcal{X}\to X$ satisfying the following:
\begin{enumerate}
    \item For any algebraically closed field $k$, the map $\pi$ induces a bijection between isomorphism classes of $k$-points of $\mathcal{X}$ and $k$-points of $X$.
    \item The map $\pi$ is universal for maps from $\mathcal{X}$ to algebraic spaces.
\end{enumerate}
For example $[pt/G]$ has coarse moduli space \(pt = \Spec k\). For 

\end{remark}


We recall the definition of a twisted curve (with no marked points) in
characteristic~$0$.

\begin{definition}
A \textbf{twisted nodal curve} $\cC \to S$ is a proper Deligne--Mumford stack
such that
\begin{enumerate}[(i)]
\item the geometric fibers of $\cC \to S$ are connected of dimension~$1$ and
the coarse moduli space $C$ of $\cC$ is a nodal curve over~$S$;

\item if $U \subset \cC$ denotes the complement of the singular locus of
$\cC \to S$, then $U \to C$ is an open immersion;
\item let $p : \Spec k \to C$ be a geometric point mapping to a node and
let $s \in S$ denote the image of $\Spec k$ under $C \to S$, and let
$\mathfrak m_{S,s}$ denote the maximal ideal of the local ring
$\mathcal O_{S,s}$.  Then there is an integer $k$ and an element
$t \in \mathfrak m_{S,s}$ such that
\[
\Spec \mathcal O_{C,p} \times_C \cC \;\cong\; [D^{sh} / \mu_k],
\]
where $D^{sh}$ denotes the strict henselization of
\[
D := \Spec \mathcal O_{S,s}[u,v]/(uv-t)
\]
at the point $(\mathfrak m_{S,s},u,v)$, and $\zeta\in\mu_k$ acts by
\[
\zeta\cdot (u,v) = (\zeta u,\zeta^{-1}v).
\]
\end{enumerate}
\end{definition}

We did not mention markings because largely we will not make use of them
except for one exception.  If $C$ is a smooth curve we can twist at a
marked point $p$ as described below.  Let $p\in C$ and
$D = \Spec \C[[z]]$ as in the first bullet point above, and fix a positive
integer $k$ and a $k$th root $w$ of $z$.  We have
$\Spec \C((w))/\mu_k = \Spec \C((z))$, so let $C_{[k]}$ denote
\[
C_{[k]}
:= C \setminus \{p\} \mathop{\cup}_{\Spec \C((z))}
        [\Spec \C[[w]] / \mu_k].
\]
Then $C_{[k]}$ is a twisted curve whose coarse moduli space is~$C$.

In a similar fashion, with $C_0$, $C_S$ as in the bullet points above, we
can construct twisted curves $C_{0,[k]}$ and $C_{S,[k]}$ with coarse moduli
spaces $C_0$, $C_S$ and such that the fiber of the node is $[pt/\mu_k]$.

The motivation to consider these objects comes from the valuative criterion for completenss. Specifically it comes from the following local calculation that Pablo does in Section 4.

\subsection{G-bundles on Twisted Chains}

In the previous section we saw that associated to the singleton sets
$\{i\}\subset \{0,r+1\}$ there is a moduli space parametrizing $G$-bundles
on a twisted nodal curve and further the moduli space can be identified
with an orbit of the wonderful embedding of the loop group.  In this
section we introduce a more general moduli problem which we show is
isomorphic to the orbit $O_I$ in the wonderful embedding for any
$I\subset \{0,\dots,r+1\}$.

Let $R_n$ denote the rational chain of projective lines with $n$--components. There is an
action of $\C^\times$ on $R_n$ which scales each component.  Let
$p_0,\dots,p_n$ denote the fixed points of this action.

Recall $u,v$ are $k$--th roots of $x,y$ which are our coordinates near a node.
Let $p',p''$ be the closed points of $\Spec \C[[u]]$, $\Spec \C[[v]]$ and
finally let $D^{\frac1k}_n$ be the curve obtained from
\[\Spec \C[[u]] \amalg R_n \amalg \Spec \C[[v]]\] by identifying $p'$ with $p_0$
and $p''$ with $p_n$.

The group $\mu_k$ acts on $D^{\frac1k}_n$ through its usual action on $u,v$ and
through the inclusion $\mu_k \subset \C^\times$ on the chain $R_n$.
For an $n$--tuple $(\beta_0,\dots,\beta_n)\in \hom(\C^\times,T)^n$, we can
speak about the equivariant $G$--bundles on $D^{\frac1k}_n$ with equivariant
structure at $p_i$ determined by $\beta_i$.  We refer to this equivalently as
a $G$--bundle on $[D^{\frac1k}_n/\mu_k]$ of type $(\beta_1,\dots,\beta_n)$.

\begin{remark}
Suppose have a space $B$ with an action of a group $\Pi$. We have a principal $H$-bundle $P \to B$. We want $\Pi$ to also act on $P$, but in a way compatible with the projection $P \to B$. An $\Pi$-equivariant $H$-bundle over $B$ is equivalent to an $H$-bundle over the quotient stack $[B/\Pi]$. Over the point $b/\Pi$ of this stack, the automorphisms are exactly $\Pi_b$. Therefore an $H$-bundle over $[B/\Pi]$ must specify how $\Pi_b$ acts on the fiber, and that is a representation $\rho:\Pi_b\to H$.
\end{remark}

Further, we can also glue $[D^{\frac1k}_n/\mu_k]$ to $C_0 - p_0$ to obtain a
curve $C_{n,[k]}$.  Let $C_n$ denote the coarse moduli space of $C_{n,[k]}$.


We call $C_n$ a \textbf{modification} of $C_0$ and $C_{n,[k]}$ a \textbf{twisted
modification} of $C_0$.


Recall the specific co-characters $\eta_0,\dots,\eta_r$. 

\begin{definition}
    For $I=\{i_1,\dots,i_n\}\subset\{0,\dots,r\}$ let
$T_{G,I}([D^{\frac1k}_n/\mu_k])$ denote the moduli space of pairs $(P,\tau)$
where $P$ is a $G$--bundle on $[D^{\frac1k}_n/\mu_k]$ of type
$(\eta_{i_1},\dots,\eta_{i_n})$ and $\tau$ is a trivialization on
$[\Spec \C((u))\times C((v))]/\mu_k$. 
\end{definition}

Let $H = \Aut(P)$ then restriction to
$\Spec \C[[u]]$ and $\Spec \C[[v]]$ realizes
\[
H \subset (L_u G)^{\mu_k} \times (L_v G)^{\mu_k}.
\]
\begin{theorem}
Let $I\subset\{0,\dots,r\}$ and $T_{G,I}([D^{\frac1k}_n/\mu_k])$ be as above.
Then there is an isomorphism
\[
T_{G,I}(C_{0,[k]}) \xrightarrow{\ \Psi^{\eta_I}\ }
(L_u G)^{\mu_k} \times (L_v G)^{\mu_k} \sslash H
\xrightarrow{\eta_I^(\cdot)\eta_I^{-1}}
\frac{L_{\mathrm{poly}}G \times L_{\mathrm{poly}}G}
     {Z(L_1)\times Z(L_1)\cdot P^{A,\pm}_I},
\]
where $\Psi^{\eta_I}$ is induced by the double coset construction and $\eta_I^{(\cdot)}\eta_I^{-1}$ is the map $LG \times LG/\cP^\Delta \to (L_uG \times L_vG)^{\mu_k}/(G_{u,v}^\Delta)^{\mu_k}$ given by \begin{align*}
    g(z)\cP \mapsto (\eta_I(w)g(w^k)\eta_I^{-1}(w))L^+_wG
\end{align*} on each factor.


Let $i : [D^{\frac1k}_n/\mu_k]\to C_{0,[k]}$ be the natural
map.  Then $i^\ast : T_{G,I}(C_{0,[k]}) \to T_{G,I}([D^{\frac1k}_n/\mu_k])$ is
an isomorphism.  In particular,
$T_{G,I}(C_{0,[k]})$ and $T_{G,I}([D^{\frac1k}_n/\mu_k])$ are isomorphic to an orbit in the wonderful embedding of $L^{\times}_{\mathrm{poly}}G$.
\end{theorem}

In this section we begin with a curve $C_S$ as in Section~2.3 and construct an algebraic
$S$--stack $\mathcal X_G(C_S)$ such that $\mathcal M_G(C_S)\subset \mathcal X_G(C_S)$ is a
dense open substack and the boundary is a divisor with normal crossings.  Further we show
the morphism $\mathcal X_G(C_S)\to S$ is complete.

For the remainder of this section we fix a simple group $G$ and further
fix an integer $k = k_G$ defined as follows. Let $\eta_i$ be the vertices of $\Al$. Define $k_i$ as the minimum integer such that $k_i \cdot \eta_i \in \hom(\C^\times, T)$ and set $k_G = \text{lcm}(k_i)$. The $\eta_i$ correspond to the maximal parahorics $P_i$ of $LG$ and further any parahoric $P$ is conjugate to a subgroup of some $P_i$. It follows readily that $k = k_G$ is the minimum value of $k$ for which the statement of Corollary~4.3 holds for any particular parahoric $P$.


We recall some of the notation from~2.3.  Namely,
$S = \Spec\C[[s]]$, $S^* = \Spec\C((s))$, $S_0 = \Spec\C[[s]]/(s) = \Spec\C$,
$C_0 = C_{S_0}$.  For $B$ an $S$--scheme we set $B^* = B\times_S S^*$,
$B_0 = B\times_S S_0$.  We also have $D_S = \Spec\C[[x,y]]$ considered as an $S$--scheme
via $s\mapsto xy$ and $D_0 = \Spec\C[[x,y]]/(xy)$.  Further, we set
$D_S^{1/k} := \C[[u,v]]$ where $u^k = x$ and $v^k = y$.  Then
$D_{S,[k]} = [D_S^{1/k}/\mu_k]$; the coarse moduli space of $D_{S,[k]}$ is
$\Spec\C[[x,y,s]]/(xy - s^k)$.  We further fix $p\in C_S$ to be the node.

To define $\mathcal X_G(C_S)$ we need to define twisted modifications of $C_S$; this is a
relative version of~(11).  Then in Subsection~5.2 we define $\mathcal X_G(C_S)$ to be the
moduli stack parametrizing $G$--bundles on twisted modifications.  There we prove the main
theorem which shows that $\mathcal X_G(C_S)$ satisfies the valuative criterion for
completeness.

\subsection{Twisted modifications}

Let $C_S$ be a nodal curve.  A \textbf{modification of length $\le n$ of $C_S$ over $B$} is
a curve $C'_B$ over $B$ with a morphism $C'_B \xrightarrow{\pi} C_B$ such that
\begin{itemize}
\item $C'_B$ is flat over $B$ and $\pi$ is finitely presented and projective;
\item $C'_B \xrightarrow{\pi} C_B^*$ is an isomorphism;
\item for $b\in B_0$ the map of curves $C'_b \xrightarrow{\pi} C_b$ is a modification; that
  is, the fiber $\pi^{-1}(p_b)$ over the unique node $p_b\in C_b$ is a rational chain of
  $\mathbb P^1$s with at most $n$ components and there is $b\in B_0$ such that
  $\pi^{-1}(p_b)$ has exactly $n$ components.
\end{itemize}


Let $(g_1,\dots,g_n)\in (\C^\times)^n$ act on $\C[[t_1,\dots,t_{n+1}]]$ by
\[
(t_1,\dots,t_n) \xmapsto{(g_1,\dots,g_n)}
(g_1 t_1,\tfrac{g_2}{g_1} t_2,\dots,\tfrac{g_n}{g_{n-1}} t_n,\tfrac{1}{g_n} t_{n-1}).
\]
This action extends to $C'_{[[t_1,\dots,t_{n+1}]]}$ such that for every closed point
$q\in \Spec\C[[t_1,\dots,t_{n+1}]]$ the stabilizer of $q$ in $(\C^\times)^n$ coincides with
$\Aut(C'_q/C_q)$.  We set
\[
\mathsf{Mdf}_n = [\C[[t_1,\dots,t_{n+1}]] / (\C^\times)^n].
\]
This is an algebraic $S$--stack that comes equipped with a curve
$[C'_S / (\C^\times)^n]$ and the modifications of $C_S$ over $B$ that arise from
$S$--maps $B\to \mathsf{Mdf}_n$ we call \textbf{local modifications of length $\le n$}.

A \textbf{twisted modification of length $\le n$ of $C_S$ over $B$} is a twisted curve
$C'_B$ such that its coarse moduli space $\overline{C'_B}$ is a modification of
length $\le n$ of $C_S$ over $B$.  A twisted modification is \textbf{of order $k$} if the
order of the stabilizer group of every twisted point has order exactly $k$.  Similarly, a
twisted modification is \textbf{of order $\le k$} if the order of the stabilizer of every
twisted point has order $\le k$.  A \textbf{local twisted modification} $C'_B$ is a twisted
modification whose coarse moduli space $\overline{C'_B}$ is a local modification. 

Let $\mathsf{Mdf}_n^{\mathrm{tw}}$ denote the functor that assigns to $B\to S$ the
groupoid of twisted local modifications of $C_S$ over $B$ of length $\le n$.  Let
$\mathsf{Mdf}_n^{\mathrm{tw},k} \subset \mathsf{Mdf}_n^{\mathrm{tw},\le k}$ be the functors
of twisted local modifications of order $k$ and order $\le k$, respectively.


\subsection{Construction of the algebraic stack}
Let $r = \operatorname{rk}(G)$. If $C'_B$ is a twisted modification of length $\le r$, then a 
$G$--bundle on $C'_B$ is called \textbf{admissible} if the co--characters determining the 
equivariant structure at all nodes are linearly independent over $\mathbb{Q}$ and are given by 
a subset of $\{\eta_0,\dots,\eta_r\}$.

Let $B$ be an $S$--scheme.  Define a groupoid $\mathcal{X}_G(C_S)$ over $S$--schemes by the assignment
\[
\mathcal{X}(C_S)(B)
=
\left\langle
\xymatrix{
  P_B \ar[r] & C'_B \ar[r] & C_B 
}
\right\rangle
\]
where $C'_B$ is a twisted local modification of $C_B$ and $P_B$ is an admissible $G$--bundle on 
$C_B$.  Isomorphisms are commutative diagrams
\[
\xymatrix{
  P_B \ar[d] \ar[r]^{\cong} & Q_B \ar[d] \\
  C'_B \ar[r]^{\cong} \ar[dr] & C''_B \ar[d] \\
  & C_B
}
\]

For notational convenience we abbreviate $\mathcal{X}_G(C_S)(B)$ as $\mathcal{X}_G(B)$.

\begin{theorem}
The functor $\mathcal{X}_G = \mathcal{X}_G(C_S)$ is an algebraic stack locally of finite type.  
It contains $\mathcal{M}_G(C_S)$ and $\mathcal{M}_G(C_{S^\ast})$ as dense open substacks, and the 
complement of $\mathcal{M}_G(C_{S^\ast})$ is a divisor with normal crossings.
\end{theorem}

\begin{theorem}
Let $R = \C[[s]]$ and $K = \C((s))$; for a finite extension $K \to K'$, let $R'$ denote the 
integral closure of $R$ in $K'$.  Given the right commutative square below, there is a finite 
extension $K \to K'$ and a dotted arrow making the entire diagram commute:
\[
\xymatrix{
\Spec K' \ar[r] \ar[d] & \Spec K \ar[r]^{h^*} \ar[d] & \mathcal{X}_G(C_S) \ar[d] \\
\Spec R' \ar[r] \ar@{-->}[ur]^{h} & \Spec R \ar[r]_{f} & S
}
\]
In particular, $\mathcal{X}_G(C_S)$ is complete over $S$.
\end{theorem}

\section{Finiteness for Fixed Curves}

Let $G$ be a reductive, connected complex Lie group and $\mathcal{M}$ the moduli stack of algebraic $G$-bundles over a smooth projective curve $\Sigma$ of genus $g$. We recall the finiteness theorem for this moduli stack.
We recall the finiteness theorem for the moduli stack of principal bundles on a fixed smooth curve.

\subsection{Admissible classes}

Given a representation $V$ of $G$, call $E^{*}V$ the vector bundle over 
$\Sigma \times \mathcal{M}$ associated to the universal $G$-bundle.
Call $\pi$ the projection along $\Sigma$, the relative canonical bundle $K$ of $\Sigma \times \mathcal{M} \to \mathcal{M}$ (so that $K|_{\Sigma} = K_{\Sigma}$), $\sqrt{K}$ its square root, $[C]$ the topological $K_{1}$-homology
class of a $1$-cycle $C$ on $\Sigma$.
Consider the following classes in the topological $K$-theory of $\mathcal{M}$:

\begin{enumerate}[(i)]
    \item The restriction $E_{x}^{*}V \in K^{0}(\mathcal{M})$ of $E^{*}V$ to a point $x \in \Sigma$;
    \item The slant product $E^{*}_{C}V := E^{*}V/[C] \in K^{-1}(\mathcal{M})$ of $E^{*}V$ with $[C]$;
    \item The Dirac index bundle $E^{*}_{\Sigma}V := R\pi_{*}(E^{*}V \otimes \sqrt{K}) 
        \in K^{0}(\mathcal{M})$ of $E^{*}V$ along $\Sigma$;
    \item The inverse determinant of cohomology,
        \[
            D_{\Sigma}V := \inv{\det} E^{*}_{\Sigma}V.
        \]
\end{enumerate}

We call the classes (i)-(iii) the \textbf{Atiyah-Bott generators}; 
they are introduced in \cite[§2]{AB}, along with their counterparts 
in cohomology, and can also be described from the K\"unneth decomposition of
$E^{*}V$ in
\[
    K^{0}(\Sigma \times \mathcal{M})
    \;\cong\; K^{0}(\Sigma) \otimes K^{0}(\mathcal{M})
    \,\oplus\, K^{1}(\Sigma) \otimes K^{1}(\mathcal{M}),
\]
by contraction with the various classes in $\Sigma$.
Classes (i) and (iv) are represented by algebraic vector bundles, while (iii)
can be realised as a perfect complex of $\mathcal{O}$-modules.
The class $E^{*}_{C}V$ in (ii) is not algebraic.
Note that
\[
    \det E^{*}_{\Sigma}V = \det R\pi_{*}(E^{*}V)
\]
when $\det V$ is trivial; an important example is the canonical bundle
\[
    \mathcal{K} = \det E^{*}_{\Sigma}\mathfrak{g}
\]
of $\mathcal{M}$, defined from the adjoint representation $\mathfrak{g}$.

\medskip


\begin{remark}
For a line bundle $\mathcal{L}$ on $\mathcal{M}=\mathrm{Bun}_G(\Sigma)$, one associates a 
\textbf{level} $\lambda(\mathcal{L})$, namely the invariant symmetric bilinear form on 
$\mathfrak g$ corresponding to the class 
$\lambda(\mathcal{L}) \in H^{4}(BG;\mathbb{Z})$.  
If $\mathcal{L}$ is a determinant line bundle $\det R\pi_{*}(E^{*}V)$ attached to a 
representation $V$ of $G$, then $\lambda(\mathcal{L})$ is the trace form 
$\operatorname{Tr}_{V}(xy)$ on $\mathfrak g$.  
When $G$ is not simply connected, such determinant bundles do not realise all possible integral 
levels. Passing from the simply connected cover $\widetilde G$ to
$G = \widetilde G/Z$ cuts down the lattice of integral invariant bilinear
forms by imposing congruence conditions along the finite central subgroup
$Z$, so that only a finite--index sublattice is realised by trace forms of actual $G$--representations.
\end{remark}

\begin{remark}[Smoothness and the relative canonical bundle]
Let $\mathcal M = \Bun_G(\Sigma)$ and let 
\[
\pi : \Sigma \times \mathcal M \longrightarrow \mathcal M
\]
be the projection.  Although the coarse moduli space of semistable $G$-bundles
may be singular, the \textbf{stack} $\mathcal M$ is a smooth Artin stack of
dimension $(g-1)\dim G$.  Indeed, for a bundle $P$ one has
\[T_{[P]}\mathcal M \simeq H^1(\Sigma,\Ad P)\] and $H^2(\Sigma,\Ad P)=0$
because $\dim\Sigma=1$, so deformations are unobstructed.

The relative canonical bundle $K := K_{\Sigma\times\mathcal M/\mathcal M}$
is defined purely from the morphism $\pi$, which is smooth of relative
dimension~$1$; no smoothness of the base is required.  In fact,
\[
K_{\Sigma\times\mathcal M/\mathcal M}
\;\cong\;
\pr_\Sigma^* K_\Sigma,
\]
the pullback of the ordinary canonical bundle of the curve.
\end{remark}

\begin{remark}
By contrast, the "canonical bundle" of the moduli stack itself is
\[
\mathcal K := \det R\pi_*(E^*\mathfrak g),
\]
the determinant of the cotangent complex of $\mathcal M$, and
Laszlo--Sorger construct a canonical Pfaffian square root
$\mathcal K^{1/2}$ of this line bundle. In particular, for semi-simple, not necessarily simply connected $G$ and for every theta characteristic $K^{1/2}_\Sigma$ on $\Sigma$, one has a square root
\[\mathcal K^{1/2} := \det R\pi_*(E^*\mathfrak g \otimes \pr_\Sigma^* K^{1/2}_\Sigma).\]


This gives rise to a natural "reference level"  $\lambda(\mathcal{K}^{1/2}) = \tfrac12\,\lambda(\mathcal{K})$.
We call a line bundle $\mathcal{L}$ on $\mathcal{M}$ \textbf{admissible} if its level exceeds 
that of $\mathcal{K}^{1/2}$, in the sense that 
$\lambda(\mathcal{L}) - \lambda(\mathcal{K}^{1/2})$ is positive definite on every simple  factor of $\mathfrak g$.  


Such positivity plays the role of an ampleness condition, and admissible line bundles provide 
the appropriate class of twistings needed for the K--theoretic index and Verlinde formulas. Products of an admissible line bundle and any number of Atiyah-Bott generators span the ring of \textbf{admissible classes}.
\end{remark}

\begin{remark}
We have defined a level by an integral invariant symmetric bilinear form on $\mathfrak g$ and simultaneously identified with central extensions of the loop group $LG$. The latter is completely determined by the action of the central scalar, which is to be an integer by the integrality condition. Abstractly, the Chern-Weil homomorphism identifies the cohomology ring $H^{*}(BG;\mathbb{R})$ of the classifying space $BG$ with the ring of invariant polynomials on the Lie algebra $\mathfrak{g}$ of $G$:
\[
    H^{*}(BG;\mathbb{R}) \cong \text{Inv}(\mathfrak{g}) := \text{Sym}(\mathfrak{g}^{*})^{G}
    \] and in degree four, we have
    \[    H^{4}(BG;\mathbb{R}) \cong \text{Inv}^{2}(\mathfrak{g}) \]
    the space of invariant symmetric bilinear forms on $\mathfrak{g}$. In particular $H^4(BG;\mathbb{R}) \cong H^3(\mf g)$ via the isomorphism we have just discussed. There is a transgression map arising from the fibration $G \to EG \to BG$:
    \[    \tau : H^{4}(BG;\mathbb{R}) \to H^{3}(G;\mathbb{R}) \]
    which is an isomorphism when $G$ is compact, simple, and simply connected. Thus we have the chain of isomorphisms
    \[    H^{4}(BG;\mathbb{R}) \;\cong\; H^{3}(\mathfrak g) \;\cong\; H^{3}(G;\mathbb{R}) \;\cong\; H^{2}(L\mathfrak g) \]
    which identifies the level defined via $H^4(BG;\mathbb{R})$ with the level defined via central extensions of the loop group $LG$, all of which are classified by invariant symmetric bilinear forms on $\mathfrak g$.

In particular, central extensions of the loop algebra $L\mathfrak{g}$ are classified by invariant symmetric bilinear forms on $\mathfrak{g}$, which are classified by $H^3(\mf g)$ defined by the Chevalley-Eilenberg complex. Given such a form $\langle\ ,\ \rangle$, the associated 3-cocycle is
    \[    \sigma(\xi,\eta,\zeta) = \langle [\xi,\eta], \zeta\rangle. \]
    Conversely, given a 3-cocycle $\sigma$ on $\mathfrak g$, one can define an invariant symmetric bilinear form by
    \[    \langle \xi,\eta\rangle := \sigma(\xi,[\eta_1,\eta_2]), \]
    where $\eta_1,\eta_2$ are any elements satisfying $\eta=[\eta_1,\eta_2]$ (such elements exist since $\mathfrak g$ is semisimple, and the definition is independent of the choice because $\sigma$ is a cocycle). We have seen that invariant symmetric bilinear forms on $\mathfrak g$ classify central extensions of the loop algebra $L\mathfrak g$ via the construction which takes $\langle\ ,\ \rangle$ to the cocycle
    \[
        \omega(\xi,\eta) = \frac{1}{2\pi} \int_0^{2\pi} \langle \xi(\theta), \eta'(\theta)\rangle\,d\theta.
    \] Moreover we have seen that any such cocycle $\omega$ arises from such a bilinear form. Thus we have an isomorphism 
    \[
        H^3(\mathfrak g) \;\xrightarrow{\cong}\; H^2(L\mathfrak g)
    \]
    On the other hand, if $G$ is compact, then the de Rham cohomology $H^3(G)$ is isomorphic to the Lie algebra cohomology $H^3(\mathfrak g)$. This is because every de Rham cohomology class has a unique left invariant representative form given by averaging, and therefore the cohomology of $G$ can be calculated from the cochain complex of the Lie algebra $\mf g$.
\end{remark}

\subsection{Levels of Line Bundles}
To certain line bundles on $\mathcal M$ we now associate a \textbf{level}, a
quadratic form on the Lie algebra $\mathfrak g$.  Briefly, for any
representation $V$, the level of $\det E^*_\Sigma V$ is the trace form
$\xi,\eta\mapsto \operatorname{Tr}_V(\xi\eta)$, and we wish to extend this
definition by linearity in the first Chern class of the line bundle.

Riemann--Roch along $\Sigma$ expresses $c_1(E^*_\Sigma V)$ as the image of
$\ch_2(V)=\tfrac12 c_1(V)^2 - c_2(V)$ under \textbf{transgression along $\Sigma$},
\[
\tau : H^4(BG;\mathbb Q)\ \longrightarrow\ H^2(\mathcal M;\mathbb Q)
\qquad\text{(construction (1.1.iii) in cohomology)}.
\]
It is important that $\tau$ is injective (Remark~4.11).  We now identify
$H^4(BG;\mathbb R)$ with the space of invariant symmetric bilinear forms on
$\mathfrak g_\kappa$ so that $\operatorname{Tr}_V$ corresponds to $\ch_2(V)$.
We say that the line bundle $\mathcal L$ \textbf{has a level} if its Chern class
$c_1(\mathcal L)$ agrees with some $\tau(h)$ in $H^2(\mathcal M;\mathbb Q)$;
the form $h$, called the \textbf{level} of $\mathcal L$, is then unique.

For $\mathrm{SL}_n$, the level of the positive generator of $\Pic(\mathcal M)$
is $-\operatorname{Tr}_{\mathbb C^n}$ in the standard representation; the
calculation is due to Quillen.  For another example, the level of
$\mathcal K^{-1/2}$ is $c := -\tfrac12\operatorname{Tr}_{\mathfrak g}$.
Positivity of a level refers to the quadratic form on $\mathfrak g_\kappa$;
thus $D_\Sigma V$ has positive level iff $V$ is $\mathfrak g$--faithful.
Finally, $\mathcal L$, with level $h$, is \textbf{admissible} iff
$h > -c$ as a quadratic form.

\begin{remark}[Properties of levels]
\begin{enumerate}[(i)]
\item
When $G$ is simply connected, the map
$\tau : H^4(BG;\mathbb Z) \to H^2(\mathcal M;\mathbb Z)$
is an isomorphism, but this fails (even rationally) as soon as
$\pi_1(G)\neq 0$.  Line bundles with a level satisfy a prescribed
relation between their Chern classes over the different components of
$\mathcal M$; cf.~(4.8).

\item
The trace forms span the negative semi--definite cone in
$H^4(BG;\mathbb R)$; so $\mathcal L$ has positive level iff
$c_1(\mathcal L)$ lies in the $\mathbb Q_+$--span of the
$c_1(D_\Sigma V)$’s for $\mathfrak g$--faithful $V$.

\item
For semi--simple $G$, the line bundle $\mathcal K$ has negative level,
and so $\mathcal O$ is admissible.  This fails for a torus, but
positive--level line bundles are admissible for any $G$.

\item
For $g>1$ and simply connected $G$, positivity of the level is
equivalent to ampleness on the moduli space.  (It suffices to check this
for simple $G$: recall then that $\Pic(\mathcal M)=\mathbb Z$ and that
$\mathcal K^{-1}$ is ample.)  When $\pi_1(G)\neq 0$, the positive--level
condition is much more restrictive.
\end{enumerate}
\end{remark}



\end{document}