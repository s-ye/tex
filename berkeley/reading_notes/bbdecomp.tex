\documentclass[12pt]{article}
\usepackage[english]{babel}
\usepackage[utf8x]{inputenc}
\usepackage[T1]{fontenc}
\usepackage{listings}
\usepackage{bookmark}
\usepackage{tikz}
\usepackage{/Users/songye03/Desktop/math_tex/style/quiver}
\usepackage{/Users/songye03/Desktop/math_tex/style/scribe}
\usepackage{fancyhdr}

\usepackage{parskip} % Automatically respects blank lines
\setlength{\parskip}{1em} % Adds more space between paragraphs
\setlength{\parindent}{0pt} % Removes paragraph indentation

\begin{document}


\lhead{Songyu Ye}
\rhead{\today}
\cfoot{\thepage}

\title{Bialynicki-Birula Decomposition}

\author{Songyu Ye}
\date{\today}
\maketitle


\begin{abstract}
We follow the original paper by Bialynicki-Birula and add details wherever I found them helpful.
\end{abstract}

\tableofcontents

\section{Preliminaries}
Let $\eta : G \times X \to X$ be an action of an algebraic group scheme $G$ on an algebraic scheme $X$ over an algebraically closed field $k$. The action $\eta$ is said to be \textbf{effective} if it satisfies the following condition:  
if $\eta$ is induced from an action 
\[
   \eta_1 : G_1 \times X \longrightarrow X
\]
(where $G_1$ is another algebraic group scheme) by a surjective homomorphism 
\[
   \varphi : G \longrightarrow G_1,
\] 
then $\varphi$ is an isomorphism. Equivalently, an action of a group $G$ on a scheme $X$ is effective (or faithful) if no nontrivial element of $G$ acts as the identity on all of $X$.

An algebraic group scheme $G$ is said to be linearly reductive if every 
$G$-module is semisimple. If $G$ is an algebraic torus (i.e.\ $G \cong \mathbb{G}_m \times \cdots \times \mathbb{G}_m$) 
then any $G$-module can be presented as a direct sum of one-dimensional 
$G$-modules; i.e., for any $G$-module $V$ and a fixed isomorphism 
$G \cong \mathbb{G}_m \times \cdots \times \mathbb{G}_m = \mathbb{G}_m^n$, 
where $n$ is an integer, there exists a basis $\{v_i\}_{i\in I}$ of $V$ such that 
for any $(\lambda_1,\dots,\lambda_n) \in \mathbb{G}_m(k) \times \cdots \times \mathbb{G}_m(k)$ 
the image of $v_i$ under $(\lambda_1,\dots,\lambda_n)$ equals 
$\lambda_1^{m_{i1}} \cdots \lambda_n^{m_{in}} v_i$, where $m_{ij}$ are integers.  

We say (again for a fixed $G \cong \mathbb{G}_m \times \cdots \times \mathbb{G}_m$) 
that the module is \textbf{positive} (resp.\ \textbf{negative}) if we have
\begin{enumerate}[(a)]
\item $m_{ij} \ge 0$ (resp.\ $m_{ij} \le 0$), for all $i,j$.
\item For every index $i \in I$, there exists $j$ such that $m_{ij} \ne 0$.
\end{enumerate}

The module is said to be \textbf{non-negative} (resp.\ \textbf{non-positive}) 
if (a) is satisfied. We say that the module is \textbf{fully definite} 
(resp.\ \textbf{definite}) if there exists an isomorphism 
$G \cong \mathbb{G}_m \times \cdots \times \mathbb{G}_m$ for which the 
module is positive (resp.\ non-negative). Equivalently, a module is fully definite (resp.\ definite) if its set of weights lies entirely in some positive (resp.\ non-negative) half-space of $X^*(G)\otimes_\mathbb{Z} \mathbb{R}$.

\medskip

Let $\eta: G \times X \to X$ be an action of a torus $G$ on $X$ and let $a$ 
be a closed point from $X^G$. We say that the action $\eta$ is 
\textbf{fully definite} (resp.\ \textbf{definite}) at a closed point 
$a \in X^G$ if the $G$-module $T_a(X)$ is fully definite (resp.\ definite).  

It is easy to see that if $X$ is irreducible quasi-affine and the action 
is definite at a closed point $a \in X^G$ then the $G$-module $k[X]$ 
is definite. Notice also that (cf.\ Theorem 2.1) if the action is fully definite 
at $a$ then $a$ is an isolated fixed point of $\eta$, and if $X$ is irreducible 
and affine and $G = \mathbb{G}_m$ then it is fully definite.

\section{Local theorems}
We prepare the proof of Theorem 2.5, which is an analog of the slice theorem for the smooth action of a compact Lie group on a manifold. 
\begin{lemma}[2.1]
Let $b$ be a non-singular closed point of an algebraic scheme.  
If $N$ is a $k$-vector space contained in $\mathfrak{m}_b$ such that the canonical map  
\[
   \mathfrak{m}_b \;\longrightarrow\; \mathfrak{m}_b / \mathfrak{m}_b^2
\]
maps $N$ isomorphically onto $(N + \mathfrak{m}_b^2)/\mathfrak{m}_b^2$, then the local ring 
$\mathcal{O}_b / N\mathcal{O}_b$ is regular and
\[
   (N\mathcal{O}_b + \mathfrak{m}_b^2)/\mathfrak{m}_b^2 \;=\; (N + \mathfrak{m}_b^2)/\mathfrak{m}_b^2.
\]

Moreover, if $N$ is contained in an ideal $\mathfrak{n}$ such that $\mathcal{O}_b/\mathfrak{n}$ 
is regular and 
\[
   (N\mathcal{O}_b + \mathfrak{m}_b^2)/\mathfrak{m}_b^2 \;=\; (\mathfrak{n} + \mathfrak{m}_b^2)/\mathfrak{m}_b^2,
\]
then $N\mathcal{O}_b = \mathfrak{n}$.
\end{lemma}

\begin{remark}[Reminder on Zariski tangent space]
    Recall for a $k$-point $a\in X$ with local ring $\mathcal{O}_{X,a}$ and maximal ideal $\mathfrak{m}_a$,
\[T_a(X)\;=\;\mathrm{Der}_k(\mathcal{O}_{X,a},k)\;\cong\;\mathrm{Hom}_k\!\big(\mathfrak{m}_a/\mathfrak{m}_a^2,\;k\big)\]
The $G$-action induces a representation on $\mathfrak{m}_a/\mathfrak{m}_a^2$ and then $T_a(X)$ carries the dual representation. If $X$ is smooth at $a$, then $\dim_k T_a(X)=\dim_k(\mathfrak{m}_a/\mathfrak{m}_a^2)$ and both have dimension $\dim X$.
\end{remark}

\begin{theorem}[2.1]
Let $G$ be linearly reductive. Suppose that $U_0 \subset U_1$ are $G$-submodules of $T_a(X)$.  
Let $X_0$ be a $G$-invariant closed irreducible subscheme containing $a$, such that $a$ is non-singular on $X_0$ and $T_a(X_0) = U_0$.  
Then one may find a closed irreducible $G$-invariant subscheme $X_1$ of $X$ such that $X_0$ is a closed subscheme of $X_1$, $a$ is non-singular on $X_1$, and $T_a(X_1) = U_1$.  
Moreover, if $X_0$ is reduced then $X_1$ is reduced.
\end{theorem}

\begin{proof}
Let $\mathfrak{m}, \mathfrak{n}_0$ be the ideals of $k[X]$ corresponding to $a$ and $X_0$, respectively.  
Then $\mathfrak{n}_0 \subset \mathfrak{m}$. Moreover, $\mathfrak{m}, \mathfrak{n}_0$ are $G$-submodules of $k[X]$ (recall that $G$ acts on $k[X]$ by $g\cdot f(x) = f(g^{-1}x)$, in particular contragredient action).

Also $\mathfrak{m}\mathcal{O}_a = \mathfrak{m}_a$ (This is just saying the maximal ideal in the local ring is obtained by extending the maximal ideal from the coordinate ring under localization).

We may consider the $G$-submodules $U_0^\perp, U_1^\perp \subset \mathfrak{m}_a / \mathfrak{m}_a^2$. Note that $U_1^\perp \subset U_0^\perp$ since $U_0 \subset U_1$.

The canonical map
\[
   \varphi_a : \mathfrak{m}_a \longrightarrow \mathfrak{m}_a / \mathfrak{m}_a^2
\]
induces a $G$-homomorphism $\mathfrak{m} \to \mathfrak{m}_a / \mathfrak{m}_a^2$, and $\varphi_a$ maps $\mathfrak{m}$ onto $\mathfrak{m}_a / \mathfrak{m}_a^2$ since $\mathfrak{m}$ generates $\mathfrak{m}_a$.  
Moreover, $\varphi_a$ sends $\mathfrak{n}_0$ onto $U_0^\perp$.  

Since $G$ is linearly reductive, we can lift the $G$-complement of $U_1^\perp$. Find a $G$-submodule $N_1$ of $\mathfrak{m}$ which satisfies the following property: $N_1 \subset \mathfrak{m}$, and $\varphi_a$ maps $N_1$ isomorphically onto $U_1^\perp$.  

Let $\mathfrak{n}_1$ be the ideal of $k[X]$ defined as the intersection of the radical of $N_1k[X]$ and $\mathfrak{n}_0$.  
Let $X_1$ be the irreducible component containing the element $a$ of the closed subscheme corresponding to the ideal $\mathfrak{n}_1$.  
\begin{remark}[What is happening here?]
A linear subspace $U_1 \subset T_a(X)$ corresponds to its annihilator $U_1^\perp \subset \mathfrak{m}_a/\mathfrak{m}_a^2$. So specifying a subspace of tangent directions is equivalent to specifying which cotangent linear forms vanish on it. $N_1 \subset \mathfrak{m} \subset k[X]$ is chosen so that its image in $\mathfrak{m}_a/\mathfrak{m}_a^2$ is $U_1^\perp$. In other words, elements of $N_1$ are global functions vanishing at $a$ whose differentials at $a$ kill exactly the tangent vectors in $U_1$. $N_1$ gives equations that cut out, to first order at $a$, exactly those tangent directions outside $U_1$.

The point is that we want a closed subscheme of $X$ defined by an ideal $\mathfrak{n}_1 \subset k[X]$ such that locally at $a$, the generators of $\mathfrak{n}_1$ are exactly $N_1$. If you only used $N_1$, you’d get an ideal describing a subvariety that locally has tangent space $U_1$. But globally, $N_1 k[X]$ might define something nonreduced or too small. So we take the radical, and intersects with $\mathfrak{n}_0$, to ensure that globally the closed set is reduced, $G$-stable, and still contains $X_0$.
\end{remark}




Then $X_1$ satisfies all desired properties.  

In fact, the ideal $\mathfrak{n}_1 \mathcal{O}_a$ is equal to $N_1 \mathcal{O}_a$ and is contained in the ideal $\mathfrak{n}_0 \mathcal{O}_a$. To see this, we start by localizing \begin{align*}
    \mathfrak n_1 \mathcal O_a
= \big(\sqrt{N_1 k[X]} \cap \mathfrak n_0\big)\mathcal O_a
= \sqrt{N_1 k[X]}\mathcal O_a \;\cap\; \mathfrak n_0 \mathcal O_a
\end{align*}

But since $a$ is in $X_0$, we have $\mathfrak{n}_0 \subset \mathfrak{m}$. After localizing at $a$, $\mathfrak{n}_0\mathcal{O}_a$ is contained in $\mathfrak{m}_a$. Meanwhile, $N_1\subset \mathfrak{m}$, so $N_1\mathcal{O}_a \subset \mathfrak{m}_a$ as well. So we only need to understand $\sqrt{N_1 k[X]}\mathcal{O}_a$.

The general fact $\sqrt{I}\mathcal{O}_a = \sqrt{\,I\mathcal{O}_a\,}$ for any ideal $I \subset k[X]$ implies that $\sqrt{N_1k[X]} \mathcal{O}_a = \sqrt{N_1 \mathcal{O}_a}$. It remains to show that $\sqrt{N_1\mathcal{O}_a} = N_1\mathcal{O}_a$ is already radical.

\begin{remark}[Why is $N_1\mathcal{O}_a$ radical?]
    We appeal to some standard commutative algebra. Let $(R,\mathfrak{m})$ be a regular local ring of dimension $d$.
    Choose elements $x_1,\dots,x_d\in\mathfrak{m}$ whose classes $\bar{x}_i$ form a $k$-basis of $\mathfrak{m}/\mathfrak{m}^2$; this is a regular system of parameters.

    For any $0\leq r\leq d$, set $I:=(x_{r+1},\dots,x_d)\subset R$. Then:
    \begin{enumerate}
        \item $x_{r+1},\dots,x_d$ is a regular sequence; hence $\operatorname{ht} I = d-r$.
        \item $R/I$ is regular local of dimension $r$ (indeed $R/I \cong k[[x_1,\dots,x_r]]$ after completion).
        \item In particular, $R/I$ is a domain, so $I$ is prime; therefore $I$ is radical.
    \end{enumerate}
Apply this fact to $N_1$. We have a subspace $U_1^\perp \subset \mathfrak{m}_a/\mathfrak{m}_a^2$. Choose lifts $f_{r+1},\dots,f_d \in \mathfrak{m}_a$ whose classes form a $k$-basis of $U_1^\perp$, and then extend to a basis of $\mathfrak{m}_a/\mathfrak{m}_a^2$ by adding $f_1,\dots,f_r$. Because $R=\mathcal{O}_{X,a}$ is regular, $f_1,\dots,f_d$ is a regular system of parameters. 
Choosing generators $f_{r+1},\dots,f_d\in N_1$ that map to a basis of $U_1^\perp$, we can write that ideal as $(f_{r+1},\dots,f_d)$, so $N_1\mathcal{O}_a = (f_{r+1},\dots,f_d)$. By the lemma above, $R/(N_1\mathcal{O}_a)$ is regular (hence reduced, hence a domain), so $N_1\mathcal{O}_a$ is prime and radical.
\end{remark}

Therefore:
\[\mathfrak{n}_1\mathcal{O}_a
= \sqrt{N_1k[X]}\mathcal{O}_a \cap \mathfrak{n}_0\mathcal{O}_a
= \sqrt{N_1\mathcal{O}_a}\;\cap\; \mathfrak{n}_0\mathcal{O}_a
= N_1\mathcal{O}_a \;\cap\; \mathfrak{n}_0\mathcal{O}_a\]

But since $N_1\subset \mathfrak{m}$ was chosen so that $\varphi_a(N_1)=U_1^\perp$ extends $\varphi_a(\mathfrak{n}_0)$, the containment
$N_1\mathcal{O}_a \subseteq \mathfrak{n}_0\mathcal{O}_a$
holds. So the intersection just gives
$\mathfrak{n}_1\mathcal{O}_a = N_1\mathcal{O}_a$. This gives $X_0 \subset X_1$.

Since $N_1$ maps isomorphically onto its image in $\mathfrak{m}_a/\mathfrak{m}_a^2$, $\mathcal{O}_a / N_1\mathcal{O}_a$ is regular. Geometrically: the subscheme $X_1$ defined by $\mathfrak{n}_1$ is smooth at $a$. Therefore the point $a$ is non-singular on $X_1$. Moreover (again by Lemma~2.1),
\[
   T_a(X_1) \;=\; \big( (N_1 + \mathfrak{m}_a^2)/\mathfrak{m}_a^2 \big)^\perp 
   \;=\; \varphi_a(N_1)^\perp 
   \;=\; U_1^\perp 
   \;=\; U_1.
\]
Since $N_1$ is a $G$-submodule of $k[X]$, $X_1$ is $G$-invariant.  
Moreover, if $X_0$ is reduced, then the radical of $\mathfrak{n}_0$ is equal to $\mathfrak{n}_0$, hence the radical of $\mathfrak{n}_1$ is equal to $\mathfrak{n}_1$, so $X_1$ is reduced.
\end{proof}

\begin{theorem}[2.2]
Let $G, U_0, U_1, X_0$ be as in Theorem~2.1. Moreover assume that 
$U_0 = \{0\}$ (i.e.\ $X_0 = \{a\}$). Then there exists exactly one 
$G$-invariant reduced and irreducible closed subscheme $X_1$ such that 
$a$ is non-singular on $X_1$ and $T_a(X_1) = U_1$ if and only if there exists 
no non-zero $G$-homomorphism
\[
   S^r(U_1) \;\longrightarrow\; T_a(X)/U_1,
\]
for any integer $r \ge 1$.
\end{theorem}

\begin{proof}
Let $X_1$ be a closed, reduced and irreducible subscheme of $X$ and let 
$\mathfrak n_1$ be the ideal of $X_1$ in $k[X]$. The subscheme $X_1$ satisfies 
the conditions given in the theorem if and only if there exists a 
$G$-submodule $N_1 \subset \mathfrak m$ such that $N_1$ is mapped 
isomorphically onto $U_1^\perp$ under the map induced by the canonical
\[
   \varphi_a : \mathfrak m_a \;\longrightarrow\; \mathfrak m_a / \mathfrak m_a^2,
   \qquad \text{and } N_1 \mathcal O_a = \mathfrak n_1 \mathcal O_a.
\]
In fact if such $N_1$ exists then $X_1$ satisfies the conditions of the theorem 
($X_1$ is $G$-invariant since $N_1 k[X]$ is a $G$-submodule, $a$ is non-singular 
on $X_1$ and $T_a(X_1)=U_1$ by Lemma~2.1). On the other hand if $X_1$ satisfies 
the conditions then $\mathfrak n_1$ is sent by $\varphi_a$ onto $U_1^\perp$, hence 
we may find a $G$-submodule $N_1 \subset \mathfrak m$ which is mapped 
isomorphically onto $U_1^\perp$. Then by Lemma~2.1,
\[
   N_1 \mathcal O_a = \mathfrak n_1 \mathcal O_a.
\]

Therefore one may find two different subschemes satisfying the conditions of 
the theorem if and only if one may find two subspaces $N_1, N_2 \subset \mathfrak m$ 
satisfying the following conditions:
\begin{enumerate}
   \item[(0)] $N_1, N_2$ are $G$-submodules of $k[X]$.
   \item[(1)] $N_1, N_2$ are mapped by $\varphi_a$ isomorphically onto $U_1^\perp$.
   \item[(2)] $N_2 \mathcal O_a \neq N_1 \mathcal O_a$.
\end{enumerate}
Put $\mathfrak n_1 = N_1 \mathcal O_a$. Then (2) is equivalent to:
\begin{enumerate}[(2$'$)]
   \item There exists an integer $n$ such that
   \[
      (N_2 + \mathfrak m_a^n)/\mathfrak m_a^n \;\not\subset\; 
      (\mathfrak n_1 + \mathfrak m_a^n)/\mathfrak m_a^n.
   \]
\end{enumerate}

On the other hand we have the following exact sequence of 
$\mathcal O_a$-modules and $G$-modules
\[
0 \;\longrightarrow\; (\mathfrak n_1 + \mathfrak m_a^n)/\mathfrak m_a^n
   \;\longrightarrow\; \mathfrak m_a/\mathfrak m_a^n 
   \xrightarrow{\;\psi\;} \mathfrak m_a/(\mathfrak n_1 + \mathfrak m_a^n)
   \;\longrightarrow\; 0,
\]
and $N_1,N_2$ satisfying (0), (1) satisfy (2$'$) if and only if
\begin{enumerate}[(2$''$)]
   \item $\psi\big( (N_2+\mathfrak m_a^n)/\mathfrak m_a^n \big) \neq 0$,
   for some integer $n \ge 1$.
\end{enumerate}

Moreover we have the following isomorphisms of $G$-modules:
\[
   \mathfrak m_a/(\mathfrak n_1 + \mathfrak m_a^n) 
   \;\cong\; \bigoplus_{r=1}^{n-1} S^r\!\big(\mathfrak m_a/(\mathfrak n_1 + \mathfrak m_a^2)\big)
   \;\cong\; \bigoplus_{r=1}^{n-1} S^r\!\big(\mathfrak m_a/\mathfrak m_a^2\big) / U_1^\perp,
\]
and
\[
   (N_2+\mathfrak m_a^n)/\mathfrak m_a^n \;\cong\; U_1^\perp.
\]

Hence if (0), (1), (2$''$) are satisfied by some $N_1, N_2$ then there exists a 
non-zero $G$-homomorphism
\[
   \tau : U_1^\perp \;\longrightarrow\; \mathfrak m_a/(\mathfrak n_1 + \mathfrak m_a^r),
\]
for some integer $r \ge 1$, or equivalently (by duality), there exists a 
non-zero $G$-homomorphism
\[
   S^r(U_1) \;\longrightarrow\; T_a(X)/U_1.
\]

On the other hand, if such a homomorphism $\tau$ exists and $N_1$ satisfying 
(0), (1) is chosen, then we may find a non-zero $G$-homomorphism
\[
   U_1^\perp \;\longrightarrow\; \mathfrak m_a/(\mathfrak n_1 + \mathfrak m_a^{r+1})
   \qquad \text{(where $\mathfrak n_1 = N_1 \mathcal O_a$).}
\]
Hence we may find a $G$-submodule $N_2'$ of $\mathfrak m_a/\mathfrak m_a^{r+1}$ 
mapped isomorphically onto $U_1^\perp$ (by the canonical map 
$\mathfrak m_a/\mathfrak m_a^{r+1} \to \mathfrak m_a/\mathfrak m_a^2$) but not 
contained in $(\mathfrak n_1 + \mathfrak m_a^{r+1})/\mathfrak m_a^{r+1}$.  
Then we may find a $G$-submodule $N_2$ of $\mathfrak m$ mapped isomorphically 
onto $N_2'$ by the canonical map $\mathfrak m \to \mathfrak m_a/\mathfrak m_a^{r+1}$ 
(since $\mathfrak m$ is mapped onto $\mathfrak m_a/\mathfrak m_a^{r+1}$).  
Then $N_1, N_2$ satisfy (0), (1), (2$''$), and hence (0), (1), (2).  
Thus the proof is complete.
\end{proof}

\begin{remark}
   [Interpreting the Hom condition] The uniqueness part of this theorem says that there is exactly one such $G$-invariant smooth subvariety if and only if there are no $G$-invariant polynomial equations of positive degree on $U_1$ valued in the quotient $T_a(X)/U_1$.

   This is because elements of \(\Hom_G(ST^r(U_1), T_a(X)/U_1)\) identify with G-equivariant homogeneous polynomial maps of degree r $P : U_1 \longrightarrow T_a(X)/U_1$. The existence of such a polynomial map implies the existence of a $G$-equivariant polynomial deformation of the “flat” slice $U_1$ inside $T_a(X)$.
   If one of those spaces is nonzero, you can modify the subvariety by adding a degree-r term, giving a different G-invariant subvariety tangent to $U_1$ at $a$. Thus, the uniqueness condition is equivalent to the nonexistence of such polynomial maps.
\end{remark}

\begin{corollary}
Let $G = \mathbb{G}_m$. If $U_1$ is equal to one of the following 
subspaces of $T_a(X)$: 
\[
   T_a(X)^0, \quad T_a(X)^+, \quad T_a(X)^-, \quad 
   T_a(X)^0 \oplus T_a(X)^+, \quad T_a(X)^0 \oplus T_a(X)^-,
\]
then there exists exactly one closed, irreducible and reduced subscheme 
$X_1$ through $a$ such that $X_1$ is $G$-invariant, $a$ is non-singular 
on $X_1$ and $T_a(X_1) = U_1$.
\end{corollary}

\begin{proof}
Look at the candidate $U_1$'s listed in the corollary. For each of these choices, $S^r(U_1)$ consists only of weight-0, weight positive, or weight negative parts. The quotient $T_a(X)/U_1$ consists of the complementary weights, and there is no nonzero homomorphism of $\mathbb{G}_m$-modules from a pure-weight representation to one of a different weight. So all the Hom spaces vanish and the uniqueness condition in Theorem 2.2 is satisfied.
\end{proof}

\begin{theorem}[2.3]
Let $X$ be irreducible and reduced. Let $G$ be an algebraic torus. 
If the action of $G$ on $X$ is definite at $a \in X^G$ then $X^G$ is irreducible.
\end{theorem}

\begin{proof}
Since the action is definite at $a$, $k[X]$ as a $G$-module is definite.  
Consider the subspace $k[X]^0 \subset k[X]$. Then there exists a 
$G$-homomorphism of algebras (the map that “throws away” all positive-weight components and keeps only the weight-zero part)
\[
   \gamma : k[X] \longrightarrow k[X]^0
\]
and $\gamma$ is a homomorphism of the $k$-algebra $k[X]$ onto the 
$k$-algebra $k[X]^0$ (since $k[X]$ as a $G$-module is definite). It is an algebra homomorphism precisely because the positive-weight part is an ideal (thanks to definiteness).

The kernel of this homomorphism is the complement of $k[X]^0$ in $k[X]$ 
and this is exactly the ideal defining $X^G$ (Any $f$ in the complement transforms by a nonzero weight, so it can’t take a nonzero constant value on a fixed point of $G$, so it must vanish on $X^G$).

Hence 
\[
   k[X]^0 = k[X^G]
\]
is an integral domain (because it is a subring of $k[X]$) and $X^G = \mathrm{Spec}(k[X]^0)$ is irreducible.
\end{proof}

\begin{corollary}
Let $G$ and $X$ be as in Theorem~2.3. If $\dim X = n$, $\dim G = m$,  $\dim X^G = n-m$, then $X^G$ is irreducible.
\end{corollary}

\begin{proof}
Take $a \in X^G$. Then the tangent space at $a$ splits as a $G$-module:
$T_a(X) = T_a(X^G) \oplus N_a$, where $T_a(X^G)$ is the zero-weight subspace (directions fixed by $G$) and $N_a$ is the "moving" part, sum of nonzero weight spaces.

Since $\dim X^G = n-m$, $\dim N_a = m = \dim G$. I claim that the weight spaces appearing in $N_a$ span the character group $X^*(G)\otimes_\mathbb{Z} \mathbb{R}$. If not, then there exists a nontrivial one-parameter subgroup $\lambda: \mathbb{G}_m \to G$ that kills all the weights appearing in $N_a$. Then $\lambda$ acts trivially on $N_a$, hence on $T_a(X)$. If a whole 1-dimensional subtorus acts trivially on $T_a(X)$, then infinitesimally the fixed locus of $G$ near $a$ has codimension at most $(m-1)$, not $m$. This contradicts the assumption that $\dim X^G = n-m$. So the weights appearing in $N_a$ span $X^*(G)\otimes_\mathbb{Z} \mathbb{R}$.

By changing coordinates on $G \cong (\mathbb{G}_m)^m$, we can assume those weights are the standard coordinate characters $(1,0,\dots,0)$, $(0,1,0,\dots,0)$, $\dots$. This puts us in the definite case. By Theorem 2.3, $X^G$ is irreducible.\end{proof}

Now we state a gluing theorem for local models which says that if two G-schemes look the same infinitesimally at fixed points (same tangent representation), and you’ve already matched up some invariant subschemes through a G-isomorphism, then you can find a third scheme $X_0$ mapping étale-equivariantly into both, such that everything matches on the invariant subschemes. In other words, there is a “common local étale neighborhood” $X_0$ of $a_1$, $a_2$ that identifies the situations.

\begin{theorem}[2.4]
Let $G$ be linearly reductive and let $G$ act on algebraic schemes $X_1, X_2$.  
Let $a_i \in X_i^G$ be closed and non-singular on $X_i$, for $i=1,2$.  
Suppose that the induced actions of $G(k)$ on $T_{a_i}(X_i)$, for $i=1,2$, are isomorphic and assume that there exist $G$-invariant closed subschemes $Y_i \subset X_i$, $i=1,2$, such that
\begin{enumerate}[(i)]
  \item $a_i \in Y_i$ and $a_i$ is non-singular on $Y_i$, for $i=1,2$,
  \item there exists a $G$-isomorphism $\alpha : Y_1 \to Y_2$ such that $\alpha(a_1) = a_2$.
\end{enumerate}
Then there exist a scheme $X_0$, an action of $G$ on $X_0$, a $G$-invariant subscheme $Y_0$ of $X_0$, morphisms 
\[
   \beta_i : X_0 \to X_i, \quad i=1,2,
\] 
and a closed point $a_0 \in X_0$ such that:
\begin{enumerate}[(a)]
  \item $\beta_1, \beta_2$ are \'etale,
  \item $\beta_i^{-1}(Y_i) = Y_0$ and $\beta_i$ maps $Y_0$ isomorphically onto an open subscheme of $Y_i$, for $i=1,2$,
  \item $\beta_1, \beta_2$ are $G$-morphisms,
  \item $a_0$ is fixed under the action of $G$ on $X_0$ and $\beta_i(a_0) = a_i$, for $i=1,2$.
\end{enumerate}
\end{theorem}

\begin{proof}
We may assume that the $X_i, Y_i$ are non-singular, for $i=1,2$.  
Consider $X_1 \times X_2$ with the action of $G$ induced by the action on factors.  
Then $(a_1,a_2)$ is fixed. The tangent space
\[
   T_{(a_1,a_2)}(X_1 \times X_2)
\]
can be identified with $T_{a_1}(X_1) \oplus T_{a_2}(X_2)$.  

Fix a $G$-isomorphism $\alpha : Y_1 \to Y_2$ such that $\alpha(a_1)=a_2$.  
Then $\alpha$ induces a $G$-isomorphism 
\[
   \alpha^* : T_{a_1}(Y_1) \to T_{a_2}(Y_2).
\]
Fix a $G$-isomorphism 
\[
   \psi : T_{a_1}(X_1) \to T_{a_2}(X_2)
\]
such that $\psi|_{T_{a_1}(Y_1)} = \alpha^*$ (the condition can be satisfied since $G$ is linearly reductive). Then consider the ``diagonal'' 
\[
   \Delta \subset T_{a_1}(X_1)\oplus T_{a_2}(X_2),
\]
i.e.\ the subset composed of all vectors of the form $(v,\psi(v))$.  

The subscheme $Y_1$ can be $G$-isomorphically immersed into $X_1 \times X_2$ by the map 
\[
   i_1 \times i_2 \alpha,
\]
where $i_1, i_2$ are the closed immersions $i_1 : Y_1 \to X_1$, $i_2 : Y_2 \to X_2$.  
Let $Y_0'$ be the obtained closed subscheme of $X_1 \times X_2$. Then $(a_1,a_2)\in Y_0'$, $Y_0'$ is $G$-invariant, and
\[
   T_{(a_1,a_2)}(Y_0') \subset \Delta.
\]
It follows from Theorem~2.1 that there exists a $G$-invariant closed irreducible subscheme $X'$ of $X_1 \times X_2$ such that $(a_1,a_2)$ is a non-singular point of $X'$, 
\[
   T_{(a_1,a_2)}(X') = \Delta, \quad X'\supset Y_0'.
\]
Let $\beta_i' : X' \to X_i$ be the projection of $X'$ into the factor $X_i$, for $i=1,2$.  
Then $\beta_i'$ induces an isomorphism 
\[
   T_{(a_1,a_2)}(X') = \Delta \;\to\; T_{a_i}(X_i),
\]
and hence $\beta_i'$ is \'etale at $(a_1,a_2)$, for $i=1,2$. Moreover $\beta_i'$ is $G$-invariant.  

\begin{remark}
    [What do we mean by etale here? What makes $\beta_i'$ etale at $(a_1,a_2)$?]
    For a morphism of schemes $f: X \to Y$ and a point $x \in X$ with $y=f(x)$, we say $f$ is étale at $x$ if:
    \begin{enumerate}
        \item $f$ is flat at $x$, and
        \item the induced map on residue fields $\kappa(y) \to \kappa(x)$ is a finite separable extension,
    \end{enumerate}
    equivalently: if $f$ is smooth of relative dimension $0$. If $x \in X$ is a closed point, this boils down to: $f$ is étale at $x$ if and only if the induced map of local rings
    $\mathcal{O}_{Y,y} \to \mathcal{O}_{X,x}$
    is a local isomorphism up to completion, or equivalently
    $T_x(X) \cong T_y(Y)$
    (isomorphism of tangent spaces at the residue field level) and $\dim X = \dim Y$ near those points. So tangent-space isomorphism is the infinitesimal characterization of being étale.
\end{remark}

Hence the subset of all points of $X'$ at which $\beta_1', \beta_2'$ are \'etale is non-empty (hence dense) and $G$-invariant. Let $X''$ be the open subscheme determined by the subset, let $\beta_i'' = \beta_i'|_{X''}$, for $i=1,2$, and $a_0=(a_1,a_2)$. Then conditions (a)--(d) are satisfied.

\medskip

Let $Y_i' := (\beta_i'')^{-1}(Y_i)$ and $Y' = Y_1' \cup Y_2'$, then the morphism 
\[
   (\beta_i''|_{Y'}) : Y_i' \to Y_i
\]
is \'etale ($i=1,2$). Since $Y_i$ is non-singular, $Y_i'$ is non-singular (etale maps are smooth of relative dimension 0). The intersection $Y_1' \cap Y_2'$ is open and dense in $Y_0'$ and since $\dim Y_i' = \dim Y_1 = \dim Y_0'$, we see that $Y_i’ \cap Y_0’$ is full dimensional in $Y_i'$ and closed. \red{It follows that $Y_i'\cap Y_0'$ is the connected component of $Y_i'$ containing $a$.}

Moreover, since $a$ is non-singular on $Y_i$, $Y_i'\cap Y_0'$ is the only irreducible component of $Y_i'$ containing $a$. Take
\[
   X_0 := X'' - (Y' - Y_0'), \quad Y_0 := Y' \cap Y_0', \quad \beta_i := \beta_i''|_{X_0}, \quad \beta_i = \beta_i''|_{X_0},
\]
then conditions (a)--(d) are satisfied.
\end{proof}

Now we come to the main result of this section, which is an equivariant local product decomposition theorem. It roughly says near a fixed point a, a definite torus action looks locally like a product of the fixed locus $X^G$ and a linear representation $V$. 


\begin{theorem}[2.5]
Let $G$ be an algebraic torus. Let the action of $G$ on $X$ be definite at $a$. Then there exists an open $G$-invariant neighbourhood $U$ of $a$ which is $G$-isomorphic to $(U \cap X^G) \times V$, where $V$ is a finite-dimensional (fully definite) $G$-module and the action of $G$ on $(U \cap X^G) \times V$ is induced by the trivial action of $G$ on $U \cap X^G$ and the linear action on $V$ (determined by the given structure of a $G$-module).
\end{theorem}

\begin{remark}
   [Analog with differential geometry] In differential geometry, there is a slice theorem for compact Lie group acting smoothly on manifolds \cite{Palais}. For any $x_0 \in M$, the slice theorem gives you a $G_{x_0}$-invariant submanifold $S \subset M$, called a slice, such that a neighborhood of the orbit $G\cdot x_0$ is diffeomorphic to $G \times_{G_{x_0}} S$, where $G_{x_0}$ is the stabilizer of $x_0$.Moreover, $S$ is transverse to the orbit $G\cdot x_0$, in the sense that \[T_{x_0}M = T_{x_0}(G\cdot x_0) \oplus T_{x_0}S\]

In the special case when $x_0$ is a fixed point, $G_{x_0} = G$ and the slice theorem says that a neighborhood of $x_0$ is diffeomorphic to $G \times_G S \cong S$, where $T_aS$ is the tangent space of $M$ at $x_0$. Hence you can take $S$ to be a small exponential image of $T_aM$.

Choose a $G$-invariant Riemannian metric (possible because $G$ is compact), so the exponential map $\exp_a : T_aM \to M$ is $G$-equivariant near the origin (since $G$ acts by isometries fixing $a$). Therefore the exponential map identifies a $G$-invariant neighborhood of $0\in T_aM$
with a $G$-invariant neighborhood of $a\in M$.

\end{remark}

In the proof of the theorem we shall use the following lemmas:
The first lemma says that once $U_1$ contains a slice over the zero vector in $V$, then it contains the whole cylinder above it.
\begin{lemma}[2.2]
Let $G$ be a torus and $V$ a fully definite finite-dimensional $G$-module.  
For an algebraic scheme $X$ consider the action of $G$ on $X \times V$ induced by the trivial action of $G$ on $X$ and by the action of $G$ on $V$ determined by the $G$-module structure.  
Then any open $G$-invariant subscheme $U_1$ of $X \times V$ contains $(U_1 \cap (X \times \{0\})) \times V$ (where we identify $X$ and $X \times \{0\}$).
\end{lemma}
\begin{proof}
Assume failure. Suppose $((U_1 \cap (X \times \{0\})) \times V) - U_1 \neq \varnothing$.
Pick a closed point $c = (x,v)$ in that difference.
So:
\begin{itemize}
\item $x \in U_1 \cap (X \times \{0\})$ (so the basepoint $(x,0)\in U_1$),
\item but $(x,v)\notin U_1$ for some $v\in V$.
\end{itemize}

Use $G$-invariance.
The set $((U_1 \cap (X \times \{0\})) \times V) - U_1$ is closed and $G$-invariant.
Therefore the whole orbit closure $\overline{G(k)\cdot c}$ is contained in this difference.
In particular, every limit point of the orbit of $(x,v)$ stays outside $U_1$.

Use definiteness of $V$.
Because $V$ is a fully definite torus representation, every nonzero vector has an orbit whose closure meets the origin $0\in V$.
(Geometrically: all weights are positive or all are negative, so scaling by $\lambda\in \mathbb{G}_m$ drives any $v\neq 0$ toward $0$.)
Thus $\overline{G(k)\cdot c}$ contains some point of the form $(x,0)$.

Contradiction.
But $(x,0)\in U_1$ by assumption ($x\in U_1 \cap (X\times\{0\})$).
So we found a point $(x,0)$ that lies both in $U_1$ and in the closed complement of $U_1$. Contradiction.

Therefore the assumption was wrong, and the whole cylinder is contained in $U_1$.
\end{proof}

\begin{lemma}[2.3]
Let $G, X, a \in X^G$ be as in Theorem~2.5. Moreover, assume that $X$ is reduced and irreducible.  
Then there exists a one-dimensional connected reduced group subscheme $G_0 \subset G$ (hence $G_0 \cong \mathbb{G}_m$) such that
\begin{enumerate}[(a)]
  \item $X^{G_0} = X^G$,
  \item the induced action of $G_0$ on $X$ is definite at $a$,
  \item if the action of $G$ on $X$ is effective then the action of $G_0$ on $X$ is also effective.
\end{enumerate}
\end{lemma}

\begin{proof}
Fix an isomorphism $G \cong \mathbb{G}_m \times \cdots \times \mathbb{G}_m = \mathbb{G}_m^n$ and a basis $v_1, \dots, v_r$ of $T_a(X)$ such that for $g=(\lambda_1, \dots, \lambda_n)\in \mathbb{G}_m^n(k)$,
\[
   g v_i = \lambda_1^{m_{i1}} \cdots \lambda_n^{m_{in}} v_i, \qquad m_{ij} \geq 0.
\]
Let $G_0$ be the diagonal of $G \cong \mathbb{G}_m^n$.  
Then for any $g=(\lambda, \dots, \lambda)\in G_0(k)$,
\[
   g v_i = \lambda^{\sum_j m_{ij}} v_i, \quad i=1,\dots,r.
\]
Hence the induced action of $G_0$ on $X$ is definite at $a$ and by Theorem~2.3, $X^{G_0}$ is irreducible.  
Since $X^{G_0} \subset X^G$ and $\dim X^G = \dim X^{G_0}$ (because $T_a(X)^{G_0} = T_a(X)^G$), it follows that $X^{G_0} = X^G$.

\begin{remark}
    [Importance of definiteness] For $G$: $v_i$ is invariant iff all $m_{ij}=0$. For $G_0$: $v_i$ is invariant iff $\sum_j m_{ij}=0$. But since all $m_{ij}\ge 0$ (definiteness assumption), $\sum_j m_{ij}=0$ if and only if $m_{ij}=0$ for all $j$. Thus the two conditions are identical and $T_a(X)^G = T_a(X)^{G_0}$.
\end{remark}

If the action of $G$ is effective then by Lemma~2.4 (below) the induced action on $T_a(X)$ is effective and the sequences $(m_{i1}, \dots, m_{in})$ span $\mathbb{Z} \times \cdots \times \mathbb{Z} = \mathbb{Z}^n$.  
In particular there exist integers $l_i$, $i=1,\dots,r$, such that
\[
   \sum_i l_i (m_{i1}, \dots, m_{in}) = (1,0,\dots,0).
\]
Then
\[
   \sum_i l_i \Big( \sum_j m_{ij} \Big) = 1
\]
and hence the $\gcd$ of $\sum_j m_{ij}$, for $i$ running through $1,\dots,r$, is equal to $1$.  
Thus the action of $G_0$ is effective.
\end{proof}

Let $\Omega$ be a fixed universal domain for $k$ (i.e. $\Omega$ is a field extension of $k$ which is algebraically closed and of infinite transcendence degree over $k$).

\begin{remark}
    [Why do we introduce $\Omega$?] We want to talk about the generic point of $X$ and its stabilizer under $G$. But the generic point is not an ordinary $k$-point of $X$, its local ring is the function field $\kappa(X)$ of $X$ as opposed to a residue field of a closed point, which is a finite extension of $k$. 

    So how do we talk about the stabilizer "at" this point in concrete terms? Instead of working directly over $\kappa(X)$, we enlarge the base field to a very large algebraically closed field $\Omega$ with infinite transcendence degree over $k$.

    We do this because every finitely generated field over $k$ (in particular, $\kappa(X)$) can be embedded into $\Omega$, and that means the generic point of $X$ (which lives over $\kappa(X)$) becomes an $\Omega$-point of $X_\Omega$.

    So now we can think of the "geometric generic point" as an actual point $t \in X(\Omega)$. Once we have a genuine $\Omega$-point $t$, we can talk about its stabilizer subgroup scheme $S_t \subseteq G_\Omega$. This stabilizer is the same as the generic stabilizer, just realized inside a big algebraically closed field.

The stabilizer at the generic point measures the global kernel of the group action. Suppose $g \in G$ fixes the generic point $\eta$. That means $g$ fixes every rational function in $\kappa(X)$. But if $g$ acts trivially on the function field, then $g$ acts trivially on a dense open subset of $X$. Since group actions are continuous (scheme-theoretically regular), this forces $g$ to act trivially on all of $X$.

Thus the stabilizer at the generic point is exactly the subgroup scheme of elements of $G$ that act trivially on all of $X$.
This is why in the condition in the following lemma is equivalent to “the action is effective.”
\end{remark}

\begin{lemma}[2.4]
Let $G$ be linearly reductive and $X$ be irreducible and reduced.  
Let $a \in X^G$ be a closed point. Then the following conditions are equivalent:
\begin{enumerate}
   \item the action of $G$ on $X$ is effective,
   \item the induced action of $G(k)$ on $k[X]$ is effective,
   \item the induced action of $G(k)$ on $T_a(X)$ is effective,
   \item the stabilizer group $S_t \subset G$ at a generic point $t \in X(\Omega)$ is trivial.
\end{enumerate}
\end{lemma}
\begin{proof}
$(1) \Rightarrow (2)$ is obvious.
$(3) \Rightarrow (4)$ follows from the fact that if an element of $G$ fixes the generic point, it fixes all of $X$, hence also $a$. So the generic stabilizer is always contained in the stabilizer at any closed point — and in particular at $a$.
$(4) \Rightarrow (1)$ see the above remark.

$(2) \Rightarrow (3)$ is the nontrivial part. See the remark following this proof for a sketch of why this is true.
\end{proof}

\begin{remark}[Elaboration on (2) implies (3)]
We will sketch why if the action on $T_a(X)$ is not effective, then the action on $k[X]$ is not effective.

Assume some nontrivial subgroup $H\subset G$ acts trivially on $T_a(X)$. Then $H$ acts trivially on $\mathfrak m_a/\mathfrak m_a^2$. \red{Because $G$ is linearly reductive, the functor "H-invariants" is exact.} The $\mathfrak m_a$-adic filtration on the local ring gives graded pieces $\mathrm{gr}_{\mathfrak m_a}(\mathcal O_{X,a}) = \bigoplus_{n\ge 0} \mathfrak m_a^n/\mathfrak m_a^{n+1}$, and each $\mathfrak m_a^n/\mathfrak m_a^{n+1}$ is a quotient of the symmetric power $S^n(\mathfrak m_a/\mathfrak m_a^2)$. Since $H$ acts trivially on $\mathfrak m_a/\mathfrak m_a^2$, it acts trivially on all $S^n(\mathfrak m_a/\mathfrak m_a^2)$, hence on every $\mathfrak m_a^n/\mathfrak m_a^{n+1}$, hence on each finite jet ring $\mathcal O_{X,a}/\mathfrak m_a^n$. 

By exactness, the successive extensions split $H$-equivariantly, so $H$ acts trivially on the completed local ring $\widehat{\mathcal O}_{X,a}$. The Krull intersection theorem guarantees that the natural map $\mathcal O_{X,a} \to \widehat{\mathcal O}_{X,a}$ is injective. Therefore the action of $H$ is trivial on $\mathcal O_{X,a}$.

Now suppose $X$ is affine near $a$: $X = \Spec A$, with $a$ corresponding to maximal ideal $\mathfrak m\subset A$. Then $\mathcal O_{X,a} = A_\mathfrak m$. Since $A$ is finitely generated over $k$, the local ring $A_\mathfrak m$ is generated as a $k$-algebra by finitely many elements of $A$. Pick generators $f_1,\dots,f_r\in A$ whose images generate $A_\mathfrak m$. If $g^*=\mathrm{id}$ on $\mathcal O_{X,a}$, then in particular $g^*(f_i)=f_i$ in the localization for each generator $f_i$.

That means: there exists some neighborhood $U_i$ of $a$ (where denominators used to localize don't vanish) such that $g^*(f_i)=f_i$ as functions on $U_i$. Let $U = \bigcap_{i=1}^r U_i$. This is still a neighborhood of $a$. On $B=\mathcal O_X(U)$, the automorphism $g^*$ and the identity coincide after localization. But localizing at $a$ is injective, so they must already coincide on $B$. Hence $g$ is the identity on $U$.

Therefore, trivial action on $\widehat{\mathcal O}_{X,a}$ forces $H$ to act trivially on a Zariski neighborhood of $a$. If a group element acts as the identity on a nonempty open subset of an irreducible scheme, it acts as the identity everywhere. Thus every $h\in H$ acts trivially on all of $X$, i.e. trivially on $k[X]$.
\end{remark}

\begin{remark}[Using Luna's étale slice theorem]
    In the above remark, one can also appeal to Luna's slice theorem, which we state here without proof. However this theorem came after Bialynicki-Birula wrote his paper, and is overkill for our situation.
    \begin{adjustwidth}{2em}{0pt}
\vspace{-\parskip}
\vspace{-16pt}
\begin{theorem}[Luna's étale slice theorem]
    Let $G$ be a reductive algebraic group acting on an affine variety $X$ over an algebraically closed field. Pick a point $x \in X$ with stabilizer $H \subseteq G$. Then there exists: a finite-dimensional $H$-representation $V$ (the slice representation, essentially the tangent representation $T_x(X)/T_x(G\cdot x)$), and an étale, $H$-equivariant morphism $(G \times^H V) \longrightarrow X$ sending $[e,0]$ to $x$, such that étale-locally near $x$, the $G$-variety $X$ looks like the homogeneous fiber bundle $G \times^H V$.
\end{theorem}
\end{adjustwidth}
Since $G$ is reductive and $a$ is a fixed point, Luna's étale slice theorem gives a $G$-equivariant étale map from a linear model to $X$: $\phi:(V,0)\longrightarrow (X,a)$, where $V$ is a finite-dimensional $G$-representation whose linear action is the isotropy/tangent representation at $a$; moreover, $\phi$ is étale at $0$ and $G$-equivariant: $\phi\circ g_V = g_X\circ \phi$.
Since G is reductive and a is a fixed point, Luna’s étale slice theorem (or, for tori, Sumihiro linearization) gives a G-equivariant étale map from a linear model to X, where $V$ is a finite-dimensional $G$-representation whose linear action is the isotropy/tangent representation at $a$; moreover, $\phi$ is étale at $0$ and $G$-equivariant.

Because $\phi$ is étale at $0$, it induces an isomorphism of completed local rings \[\widehat{\mathcal O}_{X,a}\xleftarrow[\cong]{\phi^\#} \widehat{\mathcal O}_{V,0}\] 

By the same argument as above we again see that $g$ acts trivially on a Zariski neighborhood of $0$ in $V$, which we identify with a neighborhood of $a$ in $X$ via $\phi$.
\end{remark}

\begin{definition}
    Let $Y$ be a scheme over $k$. We say $Y$ is \textbf{geometrically unibranched} if for every point $y \in Y$: (1) the local ring $\mathcal{O}_{Y,y}$ is reduced and has a unique minimal prime (so the germ of $Y$ at $y$ is irreducible), and (2) the normalization of $\mathcal{O}_{Y,y}$ is still local (i.e. has only one maximal ideal).
\end{definition}
$Y$ may have singularities, but at each point there is only one "branch" of the variety passing through that point. One can't "split" the normalization into multiple components over that point. Contrast with something like a node (e.g. $xy=0$ in $\mathbb{A}^2$), where two branches meet: that is not unibranched. See the appendix for more details.

\begin{lemma}[2.5]
Let $G = \mathbb{G}_m$ and $t \in X(\Omega)$.  
Then the algebraic scheme $\overline{G(\Omega)\cdot t}$ (over $\Omega$) defined as the closure of $G(\Omega)\cdot t$ in $X_\Omega$ is geometrically unibranched.
\end{lemma}

\begin{proof}
If $t$ is fixed for the action of $G$ on $X$ then the lemma is obvious.  
Suppose that $t$ is not fixed. Since $X$ is quasi-affine, $\overline{G(\Omega)\cdot t}$ is also quasi-affine. This is because if $X$ is quasi-affine and $G$ acts on $X$, then every $G$-orbit is quasi-affine (classical fact: a quasi-affine open subset of an affine variety remains quasi-affine, and orbits are locally closed). Thus the normalization of $\overline{G(\Omega)\cdot t}$ is quasi-affine (General fact: the normalization of a quasi-affine scheme is again quasi-affine.)

Since the orbit $G_\circ\Omega t$ is open in $\overline{G(\Omega)\cdot t}$ and is isomorphic to $\Spec \Omega[x,1/x]$, the normalization of $\overline{G(\Omega)\cdot t}$ is either equal to $\overline{G(\Omega)\cdot t}$ (in case $G_\Omega t = G(\Omega)\cdot t$) or is isomorphic to $\Spec \Omega[x]$ (in case $G_\Omega t \neq G(\Omega)\cdot t$).  
In both cases the map of the normalization onto $\overline{G(\Omega)\cdot t}$ is one-to-one.

In other words, we are starting with orbit $G_\Omega \cdot t$, which is isomorphic to $\Spec \Omega[x,1/x]$. Its closure in a quasi-affine space must be quasi-affine, 1-dimensional, and normalizable to something quasi-affine. There are only two possibilities: stay $\mathbb G_m$ (if the orbit is closed), or compactify to $\mathbb A^1$ (if closure adds one point). \red{Hence the lemma is proved.}
\end{proof}

\begin{proof}
[Proof of Theorem~2.5]
Replacing $X$ by a $G$-invariant reduced and irreducible neighbourhood of~$a$,
we may restrict considerations to the case where $X$ is reduced and irreducible.
Moreover, we may assume that the action is effective.
Let $V$ be the $G$-submodule complement of $T_a(X^G)$ in $T_a(X)$,
i.e.\ let $T_a(X^G)\oplus V = T_a(X)$.
Then the $G$-module $V$ is fully definite.
Apply Theorem~2.4 to the case where
\[
   X_1 = X,\qquad 
   X_2 = X^G\times V
\]
(with the action of $G$ induced by the trivial action on $X^G$
and the action on $V$ determined by the $G$-module structure of~$V$),
\[
   Y_1 = X^G \subset X = X_1, \qquad
   Y_2 = X^G\times\{0\}\subset X_2 = X^G\times V, \qquad
   a_1=a,\ a_2=(a,0).
\]
The assumptions of the theorem are satisfied and hence we may find
and fix $X_0$, $Y_0$, $\beta_1$, $\beta_2$, $a_0\in X_0$
satisfying conditions~(a)–(d).
We shall show that $\beta_1,\beta_2$ are open immersions.

First let us check that this will prove the theorem. If $\beta_2: X_0 \hookrightarrow X_2 = X^G\times V$ is an open immersion, so $\beta_2(X_0)$ is an open subset of $X^G\times V$. Because $\beta_2$ is étale and $G$-equivariant, the set $U_0 := \beta_2(X_0) \cap X_2^G$ is just an open neighbourhood of $a_2$ in $X_2^G$ since $X_2^G = X^G \times \{0\}$. So $U_0 \subset X^G$ is an open neighbourhood of $a$. Here we are invoking the fact that \red{étale maps are open}.

Since $\beta_2(X_0)$ is $G$-invariant (because $\beta_2$ is a $G$-map) and intersects the fixed locus in $U_0$, the product structure of $X_2 = X^G\times V$ forces: $\beta_2(X_0) \supset U_0\times V$. Here we use Lemma~2.2.

Because $\beta_2$ is an isomorphism onto its image, we can identify $X_0' := \beta_2^{-1}(U_0\times V) \subset X_0$. Then $X_0'$ is $G$-isomorphic to $U_0\times V$. Composing with the other open immersion $\beta_1$, define $U := \beta_1(X_0') \subset X$. Then $U$ is an open $G$-invariant neighbourhood of $a$, and $U \simeq_G (U_0\times V)$

Since $\beta_1,\beta_2$ are \'etale it suffices to show
that they are birational (\red{since an \'etale and birational morphism of algebraic schemes is an open immersion}).
To prove this we may (and will) assume in the sequel that $G=\mathbb G_m$. In fact, let $G_0\subset G$ be as in Lemma~2.3, then $X^{G_0}=X^G$ and hence $X_0,\beta_1,\beta_2,a_0$ fixed above satisfy conditions~(a)–(d) for $G$ replaced by~$G_0$.

The set $\beta_2(X_0)$ is open and, because it is also $G$-invariant,
by Lemma~2.2 it contains the non-empty and open subscheme
\[
  (\beta_2(X_0)\cap X_2^G)\times V .
\]
Let $t\in [\beta_2(X_0)](\Omega)$ be generic over~$k$
(then $t\in (\beta_2(X_0)\cap X_2^G)\times V(\Omega)$). In particular, this means that $t$ corresponds to the generic point of the open subset $\beta_2(X_0)$ of $X_2$.

The set $\beta_2^{-1}(t)$ is non-empty and finite. It is nonempty because $t$ was chosen inside the image $\beta_2(X_0)$. It is finite because $\beta_2$ is \'etale. \red{Hence it is relative dimension $0$ and its fibers are finite type, zero-dimensional, and reduced.}


Write $t = (x,v)$ with $x \in X^G(\Omega)$ and $v \in V(\Omega)$. Because $G$ acts trivially on $X^G$, the $G$-orbit of $t$ is \[G(\Omega)\cdot t = \{(x, g\cdot v) : g\in G(\Omega)\} \subset X_2(\Omega)\] 
Since $G = \mathbb{G}_m$ acting linearly on $V$ with weights of definite sign, we know that $\lim_{\lambda\to 0} \lambda\cdot v = 0$. That is, all points in $V$ flow to $0$ under the torus action. Therefore, the orbit closure in $V$ is $\overline{G(\Omega)\cdot v} = (G(\Omega)\cdot v) \cup \{0\}$. Lifting this to $X^G \times V$, we get $\overline{G(\Omega)\cdot t} = \big( G(\Omega)\cdot t \big) \cup \big( X^G(\Omega)\times \{0\} \big)$.

Because $V$ is fully definite, all nontrivial one-parameter subgroups of $G$ all contract $V$ toward $0$. Hence $X_2(\Omega) \cap \overline{G(\Omega)\cdot t} = (G(\Omega)\cdot t) \cup \{b\}$.

Consider the map
\[
   \beta_2^t := \beta_2^{-1}(\overline{G(\Omega)\cdot t})
      \longrightarrow \overline{G(\Omega)\cdot t}
\]
Notice that it follows from the equivariance of $\beta_2$ that $\beta_2^{-1}(b)$ is composed of exactly one point. This is because multiple disjoint fixed points mapping to the same one would violate local injectivity of an étale morphism at a fixed point. Denote it by~$b'$.
Moreover
\begin{align*}
\beta_2^{-1}(X_2(\Omega)\cap \overline{G(\Omega)\cdot t})
   &= \beta_2^{-1}(G(\Omega)\cdot t)\cup \beta_2^{-1}(b) \\
   &= \bigcup_{t_i\in\beta_2^{-1}(t)}(X_0(\Omega)\cap {G(\Omega)\cdot t_i}) \cup\{b'\} \\
   &= \bigcup_{t_i\in\beta_2^{-1}(t)}(X_0(\Omega)\cap \overline{G(\Omega)\cdot t_i})
      \cup\{b'\}.
\end{align*}
The last equality follows from the fact that $\beta_2^{-1}(\overline{G(\Omega)\cdot t})$
is closed and so we can replace each orbit by its closure. Hence $b'\in\bigcup_i\overline{G(\Omega)\cdot t_i}$.
Otherwise $\{b'\}$ would be open because it's the complement of a closed set. So $\{b\}$ would be open in
$\overline{G(\Omega)\cdot t}$ because étale morphisms are open. This is a contradiction because $b$ is a limit point of $G(\Omega)\cdot t$ and cannot be open.

Notice that $G(\Omega)\cdot t_i\cap G(\Omega)\cdot t_j=\varnothing$
for $t_i\neq t_j$ since the isotropy group $S_t$ of the point~$t$
is trivial.  In fact, if $G(\Omega)\cdot t_i\cap G(\Omega)\cdot t_j\neq\varnothing$
then $G(\Omega)\cdot t_i=G(\Omega)\cdot t_j$.
Hence there exists $g\in G(\Omega)$ such that $g(t_i)=t_j$
and thus
\[
   t = \beta_2(g(t_i)) = g(\beta_2(t_i)) = g(t) ,
\]
and $g\in S_t(\Omega)$.
If $t_i\neq t_j$ then $g\neq e\in G(\Omega)$ and hence
$S_t$ is non-trivial, which by Lemma~2.4 contradicts the assumption
that the action is effective.
Hence $b'\in \overline{G(\Omega)\cdot t_i}$ for exactly one~$t_i$ (by Lemma~2.5). In particular the target curve is unibranched and etale maps preserve the number of branches (we appeal to the appendix), only one of those preimage branches can approach the special point $b'$. Otherwise, the image $\overline{G(\Omega)\cdot t}$ would have multiple branches meeting at $b'$, contradicting geometric unibranchness.

Since $t\in X_2(\Omega)$ is generic (over~$k$) and the property
$\overline{G(\Omega)\cdot t_0}\cap X_2^G(\Omega)\neq\varnothing$
holds for any generic (over~$k$) $t_0\in X_2(\Omega)$,
in particular for any $t_j\in\beta_2^{-1}(t)$,
$\overline{G(\Omega)\cdot t_j}\cap X_2^G(\Omega)\neq\varnothing$.
But
\[
\overline{G(\Omega)\cdot t_j}\cap X_2^G(\Omega)
  = (G(\Omega)\cdot t_j\cup\{b'\})\cap X_2^G(\Omega)
  = \{b'\},
\]
hence $t_j=t_i$ for all $t_j\in\beta_2^{-1}(t)$. This is because already proved there is exactly one orbit closure among the $\overline{G(\Omega)\cdot t_i}$ that meets $b'$. \red{Thus $\beta_2^{-1}(t)$ is a one-element set and therefore
$\beta_2$ is birational (since $\beta_2$ is also \'etale).}

\begin{remark}
   [Degree of an extension of function fields] Let $f : X \to Y$ be a dominant morphism of integral, finite-type $k$-schemes.

   Then $f$ induces an injective field extension $k(Y) \hookrightarrow k(X)$, where $k(X)$ and $k(Y)$ are the function fields (i.e. residue fields at their generic points). The generic fiber $X_\eta := X \times_Y \Spec k(Y)$ has coordinate ring $\mathcal{O}(X_\eta) = \Gamma(X, \mathcal{O}_X) \otimes_{\Gamma(Y,\mathcal{O}_Y)} k(Y)$. If $f$ is finite and dominant, then \[\deg(f) := \text{length}_{k(Y)} (\mathcal{O}(X_\eta)) = [k(X) : k(Y)]\]

   Every algebraic field extension $L/K$ factors as $L/K^{\mathrm{sep}}/K$ where $L/K^{\mathrm{sep}}$ is purely inseparable and $K^{\mathrm{sep}}/K$ is separable.
   
   A morphism $f : X \to Y$ is generically separable if $k(X)/k(Y)$ is separable. If the extension is separable, the generic fiber is reduced, and the number of geometric points in it equals $[k(X):k(Y)]$. If the extension is purely inseparable: the generic fiber is non-reduced (fat point), and set-theoretically it may be a single point, but its coordinate ring has nilpotents.

Étale implies unramified + flat implies separable extension of degree equal to fiber size. Hence any étale, generically one-to-one morphism (one-point generic fiber) is birational. 

In characteristic 0, every finite extension is separable, so there's no issue. In characteristic $p>0$, you can have purely inseparable extensions where the generic fiber has one point but the field extension has degree $p^r>1$. For example the extension $k(t)/k(t^p)$, coming from $f:\mathbb{A}^1\to\mathbb{A}^1$, $t\mapsto t^p$. One sees that $[k(t):k(t^p)] = p$, and the generic fiber has one point but is nonreduced.
\end{remark}

Now consider $\beta_1$.
Take $u\in X_1(\Omega)$, generic for $X_1$ over~$k$.
Then $\beta_1^{-1}(u)$ is non-empty and finite.
Let $u_1,u_2\in\beta_1^{-1}(u)$.
Then
\[
  X_1(\Omega)\cap \overline{G(\Omega)\cdot u_1}
     = G(\Omega)\cdot u_1\cup\{s_1\},
  \qquad
  X_1(\Omega)\cap \overline{G(\Omega)\cdot u_2}
     = G(\Omega)\cdot u_2\cup\{s_2\},
\]
where $s_1,s_2\in X_0^G(\Omega)$
(because we have shown already that $X_0$ contains an open
$G$-invariant subscheme which is $G$-isomorphic to
$U_0\times V$, where $U_0$ is an open neighbourhood of~$a$ in~$X^G$
and the $G$-module $V$ is fully definite).
\[
  G(\Omega)\cdot u_i \subset \beta_1^{-1}(G(\Omega)\cdot u),
  \quad i=1,2,
\]
hence $\beta_1(s_i)=\beta_2(s_i)$ and
\[
  X_1(\Omega)\cap \overline{G(\Omega)\cdot u}
    = G(\Omega)\cdot u\cup\{s_i\}.
\]
Moreover, since $\beta_1|_{X_1^{G^\circ}}$ is one-one
(condition~(b) of Theorem~2.4),
$s_1=s_2$.
Therefore $u_1=u_2$ (by an argument as in the first part of the proof,
where we have shown $t_j=t_i$ for all $t_j\in\beta_2^{-1}(t)$).
Since $\beta_1$ is \'etale this shows that $\beta_1$ is birational. \end{proof}

\begin{corollary}
Let $G$ be an $n$-dimensional torus and let $\dim X = n$.
If the action of $G$ on $X$ is effective and there exists a point $a \in X^G$, 
then $X$ contains an open $G$-invariant neighbourhood of $a$ which is 
$G$-isomorphic to a $k$-vector space $V$ with a linear action of $G$.
\end{corollary}

\begin{proof}
Since the action is effective and $\dim G = \dim X$, for any closed point 
$a \in X$ the action is fully definite at $a$. Hence $X$ contains an open 
$G$-invariant neighbourhood of $a$ which is $G$-isomorphic to a $k$-vector 
space $V$ with the action of $G$ induced by a $G$-module structure.
\end{proof}

\begin{corollary}
Any effective action of an $n$-torus $G$ on an 
$n$-dimensional vector space $V$ is equivalent to a linear action; 
i.e.\ $V$ with the action of $G$ is $G$-isomorphic to $V$ with an action 
of $G$ determined by a $G$-module structure on $V$.
\end{corollary}

\begin{proof}
\red{Every algebraic torus acting effectively on affine space $\mathbb A^n$ has a fixed point. }Then from the above corollary we obtain that there exists an open 
$G$-invariant subscheme $U$ which is isomorphic to $V$ with an action of $G$ 
determined by a $G$-module structure on $V$. No proper open subscheme of $V$ 
can be isomorphic to $V$, hence $U = V$.
\end{proof}

\begin{remark}
   Bialynicki-Birula gets the existence of a fixed point using an argument from one of his earlier papers, \red{which I didn't hunt down}. But the sort of result he needs reminded me of the Borel fixed point theorem.
\begin{adjustwidth}{2em}{0pt}
   \vspace{-\parskip} % cancel the paragraph skip
   \vspace{-16pt}
\begin{theorem}[Borel fixed point theorem]
If $G$ is a connected solvable linear algebraic group acting regularly on a non-empty, complete algebraic variety $V$ over an algebraically closed field $k$, then $G$ has a fixed point on $V$.
\end{theorem}
\end{adjustwidth}
However $V$ is not complete in our situation, so we cannot use this theorem to guarantee the existence of a fixed point.
\end{remark}
\section{$\alpha$-fibrations}
Let $G$ be an algebraic group scheme over $k$ and let $Y$ be an algebraic scheme over $k$. In this section we recall the notion of an $\alpha$-fibration over $Y$ and prove some basic properties of $\alpha$-fibrations.
\subsection{Definition and basic properties}

Recall that a group scheme $G$ over $k$ is a scheme $G \to \Spec k$ equipped with maps $m : G \times_k G \to G$, $e : \Spec k \to G$, $i : G \to G$ satisfying the group axioms.

Now fix any k-scheme $Y$. We can form the base change of $G$ along $Y \to \Spec k$: $G_Y := G \times_k Y$. This is a scheme over $Y$, and it inherits a group structure relative to $Y$ by base change of the structure morphisms: $m_Y : G_Y \times_Y G_Y \to G_Y$, $e_Y : Y \to G_Y$, $i_Y : G_Y \to G_Y$. Thus, $G_Y$ is a group object in the category of $Y$-schemes — meaning it is a group scheme over $Y$.

Let $X$ be any scheme over $Y$. To give an action of $G_Y = G \times_k Y$ on $X$ (as a $Y$-scheme) means to give a morphism $\mu : G_Y \times_Y X \to X$ satisfying the usual axioms: $\mu(e_Y \times \id_X) = \id_X$, $\mu(m_Y \times \id_X) = \mu(\id_{G_Y} \times \mu)$. All maps are morphisms of $Y$-schemes, i.e. they commute with projection to $Y$.

For a geometric point $y \in Y$ (say with residue field $k(y)$), form the fiber product: $G_y = G \times_k k(y)$, $X_y = X \times_Y \Spec k(y)$. Then the map $\mu$ restricts fiberwise to a map $\mu_y : G_y \times_{k(y)} X_y \to X_y$. So giving a $G_Y$-action on $X$ is the same as giving for each fiber $X_y$ an action of the group scheme $G_y$ (which is just $G$ after base change to $k(y)$), in a way that varies algebraically with $y$.



\begin{example}
   Consider $G = \mathbb{G}_m = \Spec k[t, t^{-1}]$, $Y = \Spec k[s]$ (so $Y = \mathbb{A}^1_k$). Then \[G_Y = \mathbb{G}_m \times_k \mathbb{A}^1 = \Spec k[s,t,t^{-1}]\] This is a group scheme over $Y$ via the projection $\Spec k[s,t,t^{-1}] \to \Spec k[s]$. Now take \[X = \mathbb{A}^1_Y = \Spec k[s,x]\]

   Then an action of $G_Y$ on $X$ over $Y$ is a morphism $\mu : \Spec k[s,t,t^{-1},x] \to \Spec k[s,x]$ over $\Spec k[s]$. Equivalently, it's a comorphism $k[s,x] \to k[s,t,t^{-1},x]$ that fixes $s$ and satisfies the group law property. For example, the "weight-2" action $t \cdot x = t^2 x$ corresponds to the comorphism $x \mapsto t^2 x$.
\end{example} 

Let $V$ be a finite-dimensional $k$-vector space (viewed as an algebraic 
scheme over $k$) and let 
\[
  \alpha : G \longrightarrow \GL(V)
\]
be a group homomorphism. Then $\alpha$ determines an action of $G$ on $V$ 
and a structure of a $G$-module on $V$. 

\begin{definition}[$\alpha$-fibration]
   A \textbf{trivial $\alpha$-fibration over $Y$} is a $Y$-scheme $V \times Y$ 
equipped with an action of $G \times Y$ induced by the action of $G$ on $V$ 
determined by $\alpha$.

We say that a $Y$-scheme $X$ with an action of $G \times Y$ is an 
\textbf{$\alpha$-fibration over $Y$} if there exists an open covering 
$\{Y_i\}$ of $Y$ such that $X \times_Y Y_i$ is $Y_i$-isomorphic to a trivial $\alpha$-fibration over $Y_i$ for each $i$.
\end{definition}

The following proposition roughly says that for an $\alpha$-fibration over the fixed point locus, the local linear model of the G-action near a fixed point is unique.

In particular the fiber representation (the local G-module describing the directions normal to the fixed locus) is uniquely determined. Moreover, if the base is smooth and the fiber representation has no invariants (e.g. a definite torus representation), then any two possible “projections” $X \to X^G$ defining the fibration actually coincide.
\begin{proposition}[3.1]
Let $X$ be a quasi-affine algebraic scheme and let $G \times X \to X$ be an action of a linearly reductive algebraic group scheme $G$. Suppose that there exist morphisms $\gamma_i : X \to X^G$ $(i = 1, 2)$ such that $X$ with $\gamma_1$ (resp.\ with $\gamma_2$) and the given action is an $\alpha_1$-fibration (resp.\ $\alpha_2$-fibration), for some homomorphisms $\alpha_i : G \to \GL(V_i)$ $(i = 1, 2)$. Then $\alpha_1$ is equivalent to $\alpha_2$. Moreover, if $X$ is non-singular and there is no non-zero $G$-homomorphism
\[
   ST^r(V_1) \longrightarrow k_{\mathrm{triv}}
\]
into a trivial one-dimensional $G$-module, where $r$ is a positive integer, then $\gamma_1 = \gamma_2$.
\end{proposition}

\begin{proof}
Let $a$ be a closed point of $X^G$. Then $V_1, V_2$ as $G$-modules are isomorphic to $T_a(X)/T_a(X^G)$ and hence $\alpha_1$ is equivalent to $\alpha_2$ (for this part the assumption that $G$ is linearly reductive was not necessary).

To prove the second part, note first that the direct sum complement $U$ of $T_a(X^G)$ in $T_a(X)$ is determined uniquely (since there is no non-zero $G$-homomorphism $ST^r(V_1) \to k_{\mathrm{triv}}$). Hence if there is no non-zero $G$-homomorphism of $ST^r(V_1)$ into a trivial one-dimensional $G$-module, for any positive integer $r$, then by Theorem~2.2 there exists exactly one closed $G$-invariant subscheme $X_a$ of $X$ such that $a$ is non-singular on $X_a$ and $T_a(X_a) \oplus T_a(X^G) = T_a(X)$. Since $\gamma_1^{-1}(a), \gamma_2^{-1}(a)$ both satisfy the condition, $\gamma_1^{-1}(a) = \gamma_2^{-1}(a)$ for any closed point $a \in X^G$. This implies $\gamma_1 = \gamma_2$.
\end{proof}

\begin{corollary}
Let $X$, $G$, and $\gamma_1, \gamma_2$ be as in Proposition~3.1. Moreover, assume that $X$ is non-singular, $G$ is a torus, and that $T_a(X)$ as a $G$-module (with the structure determined by the action $G \times X \to X$) is definite for some closed point $a \in X^G$. Then $\gamma_1 = \gamma_2$.
\end{corollary}

\begin{proof}
Since $T_a(X)$ as a $G$-module is definite, the $G$-module $T_a(X)/T_a(X^G)$ is fully definite, and the assumptions of the second part of Proposition~3.1 are satisfied.
\end{proof}

\begin{definition}
[$G$-fibration]A $Y$-scheme $X$ with an action of $G \times Y$ is said to be a \textbf{$G$-fibration} if there exists an open covering $\{Y_i\}$ of $Y$ such that $X \times_Y Y_i$ is an $\alpha_i$-fibration over $Y_i$, where for every $i$, $\alpha_i$ is a representation
\[
   \alpha_i : G \longrightarrow \GL(V_i).
\]
If $\dim V_i = n$ for all $i$, then we say that the $G$-fibration is $n$-dimensional.
\end{definition}

\begin{proposition}[3.2]
Let a $Y$-scheme $X$ with an action of $G \times Y$ be a $G$-fibration. Then for any connected component $Y'$ of $Y$, the $Y'$-scheme $X \times_Y Y'$ with the induced action of $G \times Y'$ is an $\alpha$-fibration for some homomorphism
\[
   \alpha : G \longrightarrow \GL(V).
\]
In particular, if $Y$ is connected, then the dimension (and representation type) of any $G$-fibration is well-defined.
\end{proposition}

\begin{proof}
   By definition of G-fibration there is an open cover $\{Y_i\}_{i\in I}$ of $Y$ and, for each $i$, a representation 
   \[\alpha_i: G \longrightarrow \mathrm{GL}(V_i)\]
   together with a $G\times Y_i$-equivariant isomorphism over $Y_i$,
   $\phi_i: X\times_Y Y_i \xrightarrow{\sim} Y_i\times V_i$,
   where the action on the right is induced by $\alpha_i$.

   Fix a connected component $Y'\subset Y$ and refine the cover so that each $U_i:=Y_i\cap Y'$ is open and $Y'=\bigcup_i U_i$ with the index graph connected (i.e. for any $i,j$ there is a chain $i=i_0,i_1,\dots,i_m=j$ with $U_{i_\ell}\cap U_{i_{\ell+1}}\neq\varnothing$).

   For any pair $i,j$ with $U_{ij}:=U_i\cap U_j\neq\varnothing$, we have two $\alpha$-fibration structures on the same $G\times U_{ij}$-scheme $X\times_YY_{ij}$:
   $X\times_YY_{ij}\cong U_{ij}\times V_i$
   and
   $X\times_YY_{ij}\cong U_{ij}\times V_j$,
   compatible with the given $G\times U_{ij}$-action. Apply Proposition 3.1 to the $Y$-scheme $X\times_YU_{ij}$ with these two structures ($\gamma_i,\gamma_j$ the projections to $U_{ij}$). Proposition 3.1 (first part) yields:
   $\alpha_i$ and $\alpha_j$ are equivalent representations of $G$,
   i.e. there exists a $G$-linear isomorphism of $k$-vector spaces $V_i \xrightarrow{\sim} V_j$. In particular,
   $\dim V_i = \dim V_j$ for all overlapping $i,j$.

   Because $Y'$ is connected and the index graph is connected, by chaining overlaps we deduce that all $\alpha_i$ (with $U_i\neq\varnothing$) are mutually equivalent; in particular all $\dim V_i$ are equal. Fix one index $i_0$ and set
   $V := V_{i_0}$, $\alpha := \alpha_{i_0}$.
   For each $i$ choose a $G$-linear isomorphism $\psi_i: V \xrightarrow{\sim} V_i$ (existence follows from equivalence just proved). Then
   $X\times_Y U_i \xrightarrow{\phi_i} U_i\times V_i
   \xrightarrow{\mathrm{id}\times \psi_i^{-1}} U_i\times V$
   exhibits $X\times_Y U_i$ as a trivial $\alpha$-fibration over $U_i$. Hence $X' = X\times_Y Y'$ is an $\alpha$-fibration over $Y'$.

   Finally, since all $\dim V_i$ coincide on a connected component, the dimension of the $G$-fibration is constant on $Y'$; in particular, if $Y$ is connected, the dimension is well defined globally. The same applies to the representation type.
   \end{proof}

   \section{Decompositions determined by the action of $\mathbb{G}_m$}
In this section we assume that $k$ is algebraically closed, $G$ is an algebraic one-dimensional torus, i.e. $G = \mathbb{G}_m$, and all considered algebraic schemes are reduced and defined over $k$. Moreover, we assume that $X$ is a non-singular algebraic scheme, $\eta$ is a fixed action of $G$ on $X$, and $X$ can be covered by $G$-invariant quasi-affine open subschemes. If $X$ is projective then the last assumption is a consequence of the preceding ones.

\begin{theorem}[and definition of $X_i^+$, $X_i^-$, $\gamma_i^+$, $\gamma_i^-$]
Let $X^G = \bigsqcup_{i=1}^r (X^G)_i$ be the decomposition of $X^G$ into connected components. Then, for any $i = 1, \dots, r$, there exists a (unique) locally closed non-singular and $G$-invariant subscheme $X_i^+$ (resp.\ $X_i^-$) of $X$ and a (unique) morphism $\gamma_i^+ : X_i^+ \to (X^G)_i$ (resp.\ $\gamma_i^- : X_i^- \to (X^G)_i$) such that:
\begin{enumerate}[(a)]
  \item $(X^G)_i$ is a closed subscheme of $X_i^+$ (resp.\ $X_i^-$), and $\gamma_i^+|_{(X^G)_i}$ (resp.\ $\gamma_i^-|_{(X^G)_i}$) is the identity.
  \item $X_i^+$ (resp.\ $X_i^-$) with the action of $G$ (induced by the action of $G$ on $X$) and with $\gamma_i^+$ (resp.\ $\gamma_i^-$) is a $G$-fibration over $(X^G)_i$.
  \item For any closed $a \in (X^G)_i$, $T_a(X_i^+) = T_a(X)^0 \oplus T_a(X)^+$ (resp.\ $T_a(X_i^-) = T_a(X)^0 \oplus T_a(X)^-$).
\end{enumerate}
The dimension of the fibration defined in (b) equals $\dim T_a(X)^+$ (resp.\ $\dim T_a(X)^-$) for any closed $a \in (X^G)_i$.
\end{theorem}

\begin{proof}
It follows from our assumptions and Theorem~2.1 that for any closed $a \in X^G$ we may find a locally closed irreducible $G$-invariant subscheme $Y_a'$ of $X$ such that $a \in Y_a'$, $a$ is non-singular on $Y_a'$, and $T_a(Y_a') = T_a(X)^0 \oplus T_a(X)^+$ (resp.\ $T_a(Y_a') = T_a(X)^0 \oplus T_a(X)^-$). It follows from Theorem~2.5 that there exist an open subscheme $Y_a$ of $Y_a'$ and a morphism $\gamma_a : Y_a \to Y_a \cap X^G$ such that $Y_a$ is non-singular, $a \in Y_a$, $Y_a$ is $G$-invariant, and $\gamma_a : Y_a \to Y_a \cap X^G$ is isomorphic to a trivial $G$-fibration. Moreover, if $a,b \in X^G$ are closed points, then, by the corollary of Theorem~2.2, $\gamma_a^{-1}(Y_a \cap Y_b \cap X^G) = \gamma_b^{-1}(Y_a \cap Y_b \cap X^G)$, and (by the corollary of Proposition~3.1) $\gamma_a|_{Y_a \cap Y_b} = \gamma_b|_{Y_a \cap Y_b}$. Moreover, for any closed point $c \in Y_a \cap Y_b \cap X^G$ we have $Y_a \cap Y_b \supset \gamma_c^{-1}(Y_a \cap Y_b \cap X^G)$.

Since $(X^G)_i$ is noetherian, we may find a finite set $\{a_1, \dots, a_s\}$ of closed points of $(X^G)_i$ such that
\[
   (X^G)_i = \bigcup_{s=1}^i (Y_{a_i} \cap (X^G)_i).
\]
Then $X_i^+ = \bigcup_{i=1}^s Y_{a_i}$ (resp.\ $X_i^- = \bigcup_{i=1}^s Y_{a_i}$) is $G$-invariant, locally closed, and there exists $\gamma_i^+ : X_i^+ \to (X^G)_i$ (resp.\ $\gamma_i^- : X_i^- \to (X^G)_i$) determined uniquely by the equalities $\gamma_i^+|_{Y_{a_i}} = \gamma_{a_i}$ (resp.\ $\gamma_i^-|_{Y_{a_i}} = \gamma_{a_i}$). Then $\gamma_i^+$ (resp.\ $\gamma_i^-$) satisfies conditions (a), (b), (c). Uniqueness is evident.
\end{proof}

\begin{theorem}[4.2]
Let $X$ be quasi-affine.  
Let 
\[
    X^G = \bigsqcup_{i=1}^r (X^G)_i
\]
be the decomposition of $X^G$ into connected components.  
If $t \in X(k)$, then the $k$-closure of $\overline{G(k)\cdot t}$ is not disjoint from $(X^G)_i$, where $i$ is an integer $1 \le i \le r$, if and only if $t \in X_i^+$ or $t \in X_i^-$.  
Moreover,
\[
    X_i^+ \cap X_i^- = (X^G)_i, \qquad \text{for } i = 1, \dots, r.
\]
\end{theorem}

\begin{proof}
If $t \in X_i^+ \cup X_i^-$ then $\overline{G(k)\cdot t} \cap (X^G)_i \neq \varnothing$.

Suppose now that $a \in \overline{G(k)\cdot t} \cap (X^G)_i$. It follows from Theorem~2.4 that there exist an algebraic scheme $X_0$, an action of $G$ on $X_0$, and $G$-morphisms
\[
  \beta_i : X_0 \longrightarrow X, \quad \beta_2 : X_0 \longrightarrow (X^G)_i \times (T_a(X)^+ \oplus T_a(X)^-),
\]
with a point $a_0 \in X_0$ such that the conditions (a)--(d) of the theorem are satisfied for $a_1 = a$, $a_2 = (a,0)$, $Y_1 = (X^G)_i$, $Y_2 = (X^G)_i \times \{0\}$. Let $\beta_1^{-1}(t) = \{t_1, \ldots, t_s\}$. Then
\[
  \beta_1^{-1}(\overline{G(k)\cdot t}) = \bigcup_{i=1}^s \overline{G(k)\cdot t_i}.
\]

As in the proof of Theorem~2.4, we infer that exactly one orbit $G(k)\cdot t_i$ has $a_0$ in its $k$-closure. Suppose that $\overline{G(k)\cdot t_1} \supset G(k)\cdot t_1 \cup \{a_0\}$. Then $G(k)\cdot \beta_2(t_1)$ is not closed (since $\beta_2$ is $G$-invariant and continuous, hence preserves closures). Hence \[\beta_2(t_1) \in ((X^G)_i \times T_a(X)^+) \cup ((X^G)_i \times T_a(X)^-)\] because any point $(a,v^+ + v^-)$ with $v^+ \neq 0$ and $v^- \neq 0$ is closed because it has no finite limit points under the $G(k)$-action. Let $Z_i^+$ (resp. $Z_i^-$) be the irreducible component of $\beta_2^{-1}((X^G)_i \times T_a(X)^+)$ (resp. $\beta_2^{-1}((X^G)_i \times T_a(X)^-)$) containing $\beta_1^{-1}((X^G)_i)$. Then, by Theorem~2.2 and conditions (a)--(d), we have
\[
  \beta_1(Z_i^+) = X_i^+ \cap \beta_1(X_0), \qquad \beta_1(Z_i^-) = X_i^- \cap \beta_1(X_0).
\]
Hence $t \in X_i^+ \cup X_i^-$ (note that $t \in Z_i^+ \cup Z_i^-$ since $\beta_2(t_1) \in (X^G)_i \times (T_a(X)^+ \cup T_a(X)^-)$ and $G(k)\cdot t$ contains the point $a_0$). Moreover, since $(X^G)_i \times T_a(X)^+ \cap T_a(X)^- = (X^G)_i \times \{0\}$, we get
\[
  X_i^+ \cap X_i^- = (X^G)_i.
\] as desired.
\end{proof}

This lemma guarantees that under completeness assumptions, every non-fixed point lies in a unique attracting and a unique repelling set associated to different connected components of the fixed locus.
\begin{lemma}[4.1]
Let 
\[
  X^G = \bigsqcup_{i=1}^r (X^G)_i
\]
be the decomposition of $X^G$ into connected components.  
Then for any $t \in X(k) - X^G(k)$ such that $\overline{G(k)\cdot t}$ is complete, 
there exist two different integers $1 \le i,j \le r$ such that 
\[
  \overline{G(k)\cdot t} \cap (X^G)_i \ne \varnothing, 
  \qquad 
  \overline{G(k)\cdot t} \cap (X^G)_j \ne \varnothing.
\]
Thus $t \in X_i^+ \cup X_i^-$ and $t \in X_j^+ \cup X_j^-$.  
Moreover, if $t \in X_i^+$ then $t \in X_j^-$, and vice versa.
\end{lemma}

\begin{proof}
Since $\overline{G(k)\cdot t}$ is complete, it contains at least one closed point from $X^G$.  
Hence \red{how do we get into the quasiaffine situation}, by Theorem~4.2, there exists an integer $i$ such that 
$t \in X_i^+ \cup X_i^-$.  
Let $t \in X_i^+$.  
Then $\overline{G(k)\cdot t}$ is not contained in $X_i^+$ since $t \notin X^G$, $\overline{G(k)\cdot t}$ is complete.

Any closed complete $G$-invariant subscheme of $X_i^+$ is contained in $(X^G)_i$. This is because of the following facts: If $Y$ is proper over $k$ and $V$ is affine (of finite type), then every morphism $f: Y \to V$ is proper, hence its image $f(Y)$ is a proper closed subscheme of the affine variety $V$. A scheme that is both proper and affine over $k$ is finite over $k$ (zero-dimensional).

In particular, using the local model we can consider the projection $X_i^+ \to V$ which by the fact is proper and therefore has image a proper closed subscheme of $V$. Since $V$ is a vector space, the only proper closed subscheme that is also $G$-invariant is the origin. Therefore $p(Z)=\{0\}$ and
$Z \subset (X^G)_i \times \{0\} = (X^G)_i$.

Hence $\overline{G(k)\cdot t}$ contains a point from $(X^G)_j$, where $j \ne i$, 
and by Theorem~4.2 we have $t \in X_j^+ \cup X_j^-$.  
One can easily show that in this case in fact $t \in X_j^-$ (this corresponds to the limit at infinity).  Analogously, if $t \in X_i^-$ then $t \in X_j^+$.
\end{proof}

\begin{definition}[Locally closed decomposition]
   For any algebraic scheme $X$, a presentation of $X$ as a union $X = \bigcup_{i=1}^r X_i$ is called a \textbf{locally closed decomposition} of $X$ if the $X_i$ are locally closed reduced subschemes of $X$ and $X_i \cap X_j = \varnothing$ for $i \neq j$. The decomposition is called $G$-invariant if each $X_i$ is $G$-invariant.
\end{definition}

\begin{theorem}[Theorem 4.3]
Let the scheme $X$ be complete. Let $X^G = \bigsqcup_{i=1}^r (X^G)_i$ be the decomposition into connected components. Then there exists a unique locally closed $G$-invariant decomposition $X = \bigsqcup_{i=1}^r X_i$ (resp. $X = \bigsqcup_{i=1}^r X_i'$) and morphisms $\gamma_i : X_i \to (X^G)_i$ (resp. $\gamma_i' : X_i' \to (X^G)_i$) for $i = 1, \dots, r$ such that:

\begin{enumerate}[(i)]
\item $(X_i)^G = (X^G)_i$ (resp. $(X_i')^G = (X^G)_i$), for $i = 1, \dots, r$;
\item Each $X_i$ with $\gamma_i$ (resp. $X_i'$ with $\gamma_i'$) is a $G$-fibration over $(X^G)_i$, for $i = 1, \dots, r$;
\item For any closed point $a \in (X^G)_i$, one has \[T_a(X_i) = T_a((X^G)_i) \oplus T_a(X)^+\] (resp. $T_a(X_i') = T_a((X^G)_i) \oplus T_a(X)^-$) for $i = 1, \dots, r$.
\end{enumerate}
\end{theorem}

\begin{proof}
Take $X_i = X_i^+$ (resp. $X_i’ = X_i^-$), $\gamma_i = \gamma_i^+$ (resp. $\gamma_i’ = \gamma_i^-$). Then it follows from the properties of $X_i^+$ and $\gamma_i^+$ (resp. $X_i^-$ and $\gamma_i^-$) contained in Theorem 4.1 and from Lemma 4.1 that the conditions of Theorem~4.3 are satisfied. In particular every closed subscheme of a complete subscheme is complete.
\end{proof}


\begin{definition}
   The decomposition obtained in Theorem~4.3 using $X_i^+$ is called the \textbf{$(+)$-decomposition} of $X$. The one obtained using $X_i^-$ is called the \textbf{$(-)$-decomposition}.

For any connected component $(X^G)_i$ of $X^G$, define \[N^+((X^G)_i) := \dim T_a(X)^+\]\[N^-((X^G)_i) := \dim T_a(X)^-\] for any closed point $a \in (X^G)_i$. These numbers represent the dimensions of the fibers of the $G$-fibrations $\gamma_i : X_i^+ \to (X^G)_i$ and $\gamma_i' : X_i^- \to (X^G)_i$ respectively.

\end{definition}

\begin{corollary}[Corollary 4.1]
Assume that the scheme $X$ is complete and irreducible. Then there exists exactly one connected component $(X^G)^+$ (resp. $(X^G)^-$) of $X^G$ such that $N^-((X^G)^+) = 0$ (resp. $N^+((X^G)^-) = 0$).
\end{corollary}

\begin{proof}
Take the $(+)$-decomposition (resp. the $(-)$-decomposition) \[X = \bigsqcup_{i=1}^r X_i^+\] (resp. $X = \bigsqcup_{j=1}^r X_j^-$). Since the decomposition is locally closed, there exists exactly one $i$ such that $X_i^+$ (resp. exactly one $j$ such that $X_j^-$) is open in $X$. This is because $X$ is a union of these cells, so one of them must be of dimension $\dim X$ and being locally closed, it is open in its closure which is $X$ since $X$ is irreducible. Uniqueness is clear since irreducible implies that any two non-empty open subsets intersect.

Then $N^-((X^G)_i) = 0$ (resp. $N^+((X^G)_j) = 0$). Hence it suffices to take $(X^G)^+ = (X^G)_i$ (resp. $(X^G)^- = (X^G)_j$).
\end{proof}

Notice that the connected components $(X^G)^+$ and $(X^G)^-$ of $X^G$ coincide, in the case where $X$ is non-singular, with the irreducible components of $X^G$.

\section{Appendix: Geometrically unibranch}
We add some remarks on the notion of geometrically unibranch from the Stacks Project \cite{Stacks}.
\begin{definition}[Branches for local rings, 15.107.6]
Let $A$ be a local ring with henselization $A^h$ and strict henselization $A^{sh}$.
The \textbf{number of branches} of $A$ is the number of minimal primes of $A^h$ if finite
and $\infty$ otherwise.
The \textbf{number of geometric branches} of $A$ is the number of minimal primes of
$A^{sh}$ if finite and $\infty$ otherwise.
\end{definition}

\begin{definition}[Branches for schemes, 28.15.1]
Let $X$ be a scheme. Let $x \in X$.  
The \textbf{number of branches of $X$ at $x$} is the number of branches of the local ring
$\mathcal{O}_{X,x}$. The \textbf{number of geometric branches of $X$ at $x$} is the number of geometric branches
of the local ring $\mathcal{O}_{X,x}$.
\end{definition}

\begin{lemma}[33.40.1]
Let $X$ be a scheme. Assume every quasi-compact open of $X$ has finitely many irreducible components.
Let $\nu : X^\nu \to X$ be the normalization of $X$. Let $x \in X$.
\begin{enumerate}
    \item The number of branches of $X$ at $x$ is the number of inverse images of $x$ in $X^\nu$.
    \item The number of geometric branches of $X$ at $x$ is
    \[
        \sum_{\nu(x^\nu) = x} [\kappa(x^\nu) : \kappa(x)]_s.
    \]
\end{enumerate}
\end{lemma}

\begin{remark}
   From the lemma, it becomes clear that for schemes over an algebraically closed field of characteristic 0, the number of geometric branches at a point is just the number of preimages in the normalization.
\end{remark}

\begin{lemma}[37.36.2]
Let $X$ be a scheme and $x \in X$ a point. Then
\begin{enumerate}
    \item the number of branches of $X$ at $x$ is equal to the supremum of the number of irreducible components of $U$ passing through $u$, taken over elementary \'etale neighbourhoods $(U, u) \to (X, x)$
    \item the number of geometric branches of $X$ at $x$ is equal to the supremum of the number of irreducible components of $U$ passing through $u$, taken over \'etale neighbourhoods $(U, u) \to (X, x)$
    \item $X$ is unibranch at $x$ if and only if for every elementary \'etale neighbourhood $(U, u) \to (X, x)$
    there is exactly one irreducible component of $U$ passing through $u$
    \item $X$ is geometrically unibranch at $x$ if and only if for every \'etale neighbourhood $(U, u) \to (X, x)$
    there is exactly one irreducible component of $U$ passing through $u$.
\end{enumerate}
\end{lemma}

\section{References}
\begin{enumerate}
\bibitem{BB} Bialynicki-Birula, A. "Some theorems on actions of algebraic groups." \textit{Annals of Mathematics} 98 (1973), no. 3, 480--497.
\bibitem{Palais} Palais, R. S. "On the existence of slices for actions of non-compact Lie groups." \textit{Annals of Mathematics} 73 (1961), no. 2, 295--323.
\bibitem{Stacks} The Stacks Project Authors. "The Stacks Project." \url{https://stacks.math.columbia.edu}, 2023.
\end{enumerate}
\end{document}