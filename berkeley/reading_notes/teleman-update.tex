\documentclass[12pt]{article}
\usepackage[english]{babel}
\usepackage[utf8x]{inputenc}
\usepackage[T1]{fontenc}
\usepackage{listings}
\usepackage{bookmark}
\usepackage{tikz}
\usepackage[all]{xy}
\makeatletter
\def\input@path{{../../style/}}
\makeatother

\usepackage{../../style/quiver}
\makeatletter
\def\input@path{{../../style/}}
\makeatother

\usepackage{../../style/scribe}
\usepackage{fancyhdr}
\DeclareMathOperator{\Al}{Al}
\DeclareMathOperator{\Chmb}{Ch}
\DeclareMathOperator{\pt}{pt}
\DeclareMathOperator{\defect}{defect}
\DeclareMathOperator{\Sing}{Sing}

\usepackage{parskip} % Automatically respects blank lines
\setlength{\parskip}{1em} % Adds more space between paragraphs
\setlength{\parindent}{0pt} % Removes paragraph indentation

% ---------- Title ----------
\title{Extending FTT to $G$-bundles}
\author{Songyu Ye}
\date{\today}

\begin{document}
\maketitle

\section{Goal}
The goal of this project is to extend the results of Frenkel-Teleman-Tolland \cite{ftt} on the existence of derived pushforward of admissible classes on the moduli stack of $\C^\times$-bundles over stable curves to the case of principal $G$-bundles for a general reductive group $G$.

\section{Results from Frenkel-Teleman-Tolland}

\begin{definition}[Gieseker bundle]\label{def:gieseker-bundle}
Let $(\Sigma,\sigma_i)$ be a stable marked curve.  A
\textbf{Gieseker $\C^\times$--bundle} on $(\Sigma,\sigma_i)$ is a pair
$(m,\mathcal P)$ consisting of
\begin{enumerate}
\item a modification $m : (C,\sigma_i) \to (\Sigma,\sigma_i)$ (in the sense that $m$ is an isomorphism away from the nodes of $\Sigma$, and the preimage of each node is either a node or a $\mathbb P^1$ meeting the two branches transversely), and
\item a principal $\C^\times$--bundle $p : \mathcal P \to C$,
\end{enumerate}
which satisfy the \textbf{Gieseker condition}:
the restriction of $\mathcal P$ to every unstable $\mathbb P^1$ has degree $1$.
\end{definition}



Let $\widetilde{\mathcal{M}}_{g,I}([pt/\C^*])$ be the moduli stack of Gieseker $\C^*$-bundles over stable curves of genus $g$ with $I$ marked points, as constructed in Frnekel-Teleman-Tolland \cite{ftt}. It carries a universal principal $\C^*$-bundle $\mathcal{P} \to \mathcal{C}$ over the universal curve $\pi:\mathcal{C} \to \widetilde{\mathcal{M}}_{g,I}([pt/\C^*])$. 

Let $F : \widetilde{\mathcal{M}}_{g,I}([pt/\C^*]) \to \overline{\mathcal{M}_{g,I}}$ be the bundle-forgetting map. Let $\phi: C \to [pt/\C^*]$ be the classifying map of the universal bundle $\mathcal{P}$. Let $\pi:\mathcal{C} \to \widetilde{\mathcal{M}}_{g,I}([pt/\C^*])$ be the universal curve.

A line bundle $\mathcal{L}$ on
$\overline{\mathcal{M}}_{g,I}([\mathrm{pt}/\C^\times])$ is \emph{admissible} if
\[
\mathcal{L} \simeq \bigl(\det R\pi_*\phi^*\C_1\bigr)^{\otimes(-q)},
\]
where $\C_1$ is the standard representation and $q$ is a positive rational number.
An \emph{admissible complex} is a sum of complexes of the form
\[
\mathcal{L}\;\boxtimes\;\bigotimes_a \bigl(R\pi_*\phi^*V_a\bigr)
\;\boxtimes\;\bigotimes_i \bigl(\mathrm{ev}_i^*V_i \otimes T_i^{\otimes n_i}\bigr).
\]
The subring of $K\!\left(\overline{\mathcal{M}}_{g,I}([\mathrm{pt}/\C^\times])\right)$
generated by such products is called the \emph{ring of admissible classes}.
The main result of \cite{ftt} is the following theorem.


\begin{theorem}
The derived pushforward $RF_* \alpha$ of an admissible complex $\alpha$ along the bundle-forgetting map $F : \widetilde{\mathcal{M}}_{g,I}([pt/G]) \to \overline{\mathcal{M}_{g,I}}$ is a bounded complex of coherent sheaves.
\end{theorem}
\section{General idea}

Let $S = \C[[s]], S^* = \C((s))$ and $B$ be an $S$-scheme.  Let $C_S \to S$ be a projective flat family of curves with generic fiber $\C_{S^*}$ smooth and special fiber $C_0$ nodal with unique node $p$. Let $C_B = C_S \times_S B$.

Solis \cite{solis} defines the $S$-stack $\mathcal{X}_G(C_S)$ whose points evaluated at a test scheme $B/S$ are given by elements $(C'_B,P_B)$ where $C'_B$ is a twisted modification of $C_B$ and $P_B$ is an admissible $G$-bundle on $C'_B$. This stack is over a fixed curve $C_S$ and Solis shows that it is algebraic, locally of finite type, and complete over $S$. It contains $M_G(C_S)$ and $M_G(C_{S^*})$ as dense open substacks, and the complement of $M_G(C_{S^*})$ is a divisor with normal crossings.

In this section, we discuss how to generalize Solis' construction to families of curves by working over the universal curve over the moduli stack of stable curves $\overline{\mathfrak{M}}_{g,I}$. Let $\pi:\overline{\mathcal{C}_{g,I}} \to \overline{\mathfrak{M}}_{g,I}$ be the universal curve over the moduli stack of stable curves of genus $g$ with $I$ marked points.

Let $\pi:C\to B$ be a prestable family of nodal curves. Let
\[
\Sigma := \mathrm{Sing}(C/B)\subset C
\]
be the relative singular locus. It is finite étale over $B$ after restricting to the locus where the number of nodes is constant; globally it is at least finite unramified in good situations.
\begin{definition}
    

A \textbf{modification of $C/B$} is a proper morphism $m:C'\to C$ over $B$ such that:
\begin{enumerate}
\item $C'\to B$ is flat prestable curve, and $m$ is finitely presented and projective.

\item $m$ is an isomorphism away from the nodes:
\[
m:\; C'\setminus m^{-1}(\Sigma)\ \xrightarrow{\sim}\ C\setminus \Sigma.
\]

\item For every geometric point $b\to B$ and every node $p\in \Sigma_b\subset C_b$, the fiber of $m_b:C'_b\to C_b$ over $p$ is either a point (no modification at that node) or a chain of $\mathbb P^1$'s meeting the two branches in the standard way, and $m_b$ contracts that chain to $p$ and is an isomorphism elsewhere.
\end{enumerate}
\end{definition}

A \textbf{length $\le n$ condition} can be stated as:
\begin{itemize}
\item for every $b$ and every node $p\in \Sigma_b$, the chain over $p$ has at most $n$ components.
\end{itemize}

\begin{definition}[Twisted nodal curves over a base]\label{def:twisted-curve}
Let $B$ be a scheme over $\C$.
A \textbf{twisted nodal curve over $B$} is a proper Deligne--Mumford stack
\[
\pi:\mathcal C \longrightarrow B
\]
such that:

\begin{enumerate}
\item
The geometric fibers of $\pi$ are connected, one--dimensional, and the
coarse moduli space $\overline{\mathcal C}$ is a nodal curve.

\item
Let $\mathcal U \subset \mathcal C$ be the complement of the relative
singular locus $\mathrm{Sing}(\mathcal C/B)$.  
Then the restriction
\[
\mathcal U \hookrightarrow \mathcal C
\]
is an open immersion.

\item
For any geometric point $p:\Spec k\to \mathcal C$ mapping to a node of the
fiber over $b\in B$, there exists an integer $k\ge1$ and an element
$t\in \mathfrak m_{B,b}$ such that, étale-locally on $B$ at $b$ and strictly
henselian locally on $\mathcal C$ at $p$, there is an isomorphism
\[
\Spec \mathcal O_{\mathcal C,p}^{sh}
\;\cong\;
\Bigl[\,\Spec\bigl(\mathcal O_{B,b}^{sh}[u,v]/(uv-t)\bigr)\ \big/\ \mu_k\,\Bigr],
\]
where $\zeta\in\mu_k$ acts by
\[
(u,v)\longmapsto(\zeta u,\zeta^{-1}v).
\]
\end{enumerate}
\end{definition}

\begin{definition}
    A \textbf{twisted modification of $C/B$} is a twisted nodal curve $\mathcal{C} \to B$ whose coarse moduli space $\overline{\mathcal{C}}$ is a modification of $C/B$.
\end{definition}
Let $r = \operatorname{rk}(G)$. The ordered simple roots $\{\alpha_0,\alpha_1,\dots,\alpha_r\}$ determine
ordered vertices $\{\eta_0,\dots,\eta_r\}$ determined by the conditions
\[
\langle \eta_i,\alpha_j\rangle = 0 \text{ for } i\neq j
\quad\text{and}\quad
\langle \eta_0,\alpha_0\rangle =1.
\]
If we write $\theta = \sum_{i=1}^r n_i\alpha_i$ and set $n_0=1$ then one can
check these conditions can be expressed as
\begin{equation}
\label{eq:alpha-eta}
\langle \alpha_i,\eta_j\rangle = \frac{1}{n_i}\delta_{i,j}.
\end{equation}

Following \cite{solis}, if $C'_B$ is a twisted modification of length $\le r$, then a 
$G$--bundle on $C'_B$ is called \textbf{admissible} if the co--characters determining the 
equivariant structure at all nodes are linearly independent over $\mathbb{Q}$ and are given by 
a subset of $\{\eta_0,\dots,\eta_r\}$.

\begin{definition}
    We define a stack $\mathcal{X}_{G,g,I}$ over $\overline{\mathfrak{M}}_{g,I}$ whose points over a test scheme $B \to \overline{\mathfrak{M}}_{g,I}$ are given by pairs $(C'_B,P_B)$ where $C'_B$ is a twisted modification of the pullback $C_B$ of the universal curve $\overline{\mathfrak{C}_{g,I}}$ to $B$, and $P_B$ is an admissible $G$-bundle on $C'_B$. 
\end{definition}

Let $\Sigma_0, \sigma_{0,i}$ be a fixed stable curve of genus $g$ with $I$ marked points. Let $B$ be an affine etale neighborhood of the point $[\Sigma_0,\sigma_{0,i}]$ in $\overline{\mathcal{M}}_{g,I}$. Let $\mathcal{X}_{G,g,I}\vert_B$ be the fiber of $F$ over the map $B \to \overline{\mathcal{M}}_{g,I}$, where $F:\mathcal{X}_{G,g,I} \to \overline{\mathcal{M}}_{g,I}$ is the natural projection.

We need to produce a local chart $A$ for the stack $\mathcal{X}_{G,g,I}\vert_B$. This stack $A$ will be a category fibered over $B$. We need to show that the stack $A$ is represented by an algebraic space and then display $\mathcal{X}_{G,g,I}\vert_B$ as a quotient stack $[A/H]$ for some reductive group $H$ acting on $A$.

Following the strategy of proof carried out for $\C^*$-bundles in FTT \cite{ftt}, we aim to carry out the following:
\begin{enumerate}
\item $\mathcal X_{G,g,I}$ is algebraic and locally of finite type over $\overline{\mathcal M}_{g,I}$.

\item After \'etale localization to $B\to\overline{\mathcal M}_{g,I}$, we have
\[
\mathcal X_{G,g,I}\vert_B\;\simeq\; [A/H]
\]
where $A$ is a smooth algebraic space and $H=G^V$ is reductive.

\item There exists an open subspace $A^\circ\subset A$ (defined by a numerical stability condition generalizing the genus bounds) such that the quotient $[A^\circ/H]$ factors as
\[
[A^\circ/H] \;\simeq\; [\pt/Z(G)_0] \times Q
\]
where $Q\to B$ is a proper moduli space.

\item The complement $A\setminus A^\circ$ admits a stratification by closed $H$-invariant pieces $Z_\delta(\pi)$ and $W_\delta(\pi)$, each an affine bundle over a fixed-point locus $F_\delta(\pi)$ under a subgroup $H(\pi)\cong (\C^\times)^2$ of $H$.

\item For admissible $K$-theory classes $\alpha$ on $\mathcal X_{G,g,I}$, the weights of the fibers of $\alpha$ over the fixed loci $F_\delta(\pi)$ are bounded below by linear functions of the discrete defect parameters $(n_+,n_-)$ in a way that forces vanishing of $H(\pi)$-invariants for sufficiently large defect.
\end{enumerate}

These properties, combined with the local cohomology argument, imply that for any admissible class $\alpha$, the $H$-invariant part of $R\Gamma(A,\alpha)$ is a finitely generated $\mathcal O_B$-module that vanishes in high degrees. Since this holds étale-locally over $B$ and the stack structure descends, we conclude that the derived pushforward
\[
RF_* \alpha : \mathcal X_{G,g,I} \longrightarrow \overline{\mathcal M}_{g,I}
\]
is a bounded complex of coherent sheaves, where $F$ is the forgetful map forgetting the bundle and the modification.

\subsection{Algebraicity and finite type of $\mathcal X_{G,g,I}$}

I have sketched a proof of the following proposition, establishing that
$\mathcal X_{G,g,I}$ is an algebraic stack, locally of finite type over
$\overline{\mathcal M}_{g,I}$. 

\begin{proposition}\label{prop:X-algebraic}
The projection
\[
F:\mathcal X_{G,g,I}\to \overline{\mathcal M}_{g,I}
\]
is algebraic and locally of finite type.
\end{proposition}


\begin{proof}
We decompose the definition of $\mathcal X_{G,g,I}$ into three pieces:
twisted curves and their modifications, principal $G$--bundles, and the
admissibility condition.

\medskip
\noindent\textbf{Step 1: Twisted curves.}
Let $\mathfrak M^{tw}_{g,I}$ denote the stack of twisted stable curves of genus
$g$ with $I$ markings.
By Abramovich--Vistoli \cite{abramovich-vistoli}
$\mathfrak M^{tw}_{g,I}$ is a Deligne--Mumford
stack, locally of finite type over $\C$, equipped with a representable morphism
\[
\mathfrak M^{tw}_{g,I}\to \overline{\mathcal M}_{g,I}
\]
sending a twisted curve to its coarse stable curve.

\medskip
\noindent\textbf{Step 2: Twisted modifications.}
Let $\mathsf{TwMdf}_{\le r,g,I}$ be the stack over $\overline{\mathcal M}_{g,I}$
whose objects over a scheme $B\to \overline{\mathcal M}_{g,I}$ are twisted
modifications of length $\le r$ of the pulled--back universal curve $\Sigma_B\to B$.

Étale--locally on $\overline{\mathcal M}_{g,I}$, the universal curve is
simultaneously versal at all nodes. Twisted modifications of bounded length
are obtained by inserting chains of $\P^1$'s and taking balanced root stacks at the nodes. Standard results on expanded degenerations and twisted curves imply that $\mathsf{TwMdf}_{\le r,g,I}$ is an algebraic stack, locally of finite type over $\overline{\mathcal M}_{g,I}$. In particular, we have the following result from \cite{ACFW}. They do not explicitly state their results in this manner, but the following proposition follows from their constructions.

\begin{proposition}[twisted expansions are algebraic and lft]
\label{prop:ACFW-twisted-expansions-algebraic}
There exist algebraic stacks
\[
\mathcal T^{\mathrm{tw}}
\qquad\text{and}\qquad
\mathfrak T^{\mathrm{tw}}
\]
called the stacks of \textbf{twisted expanded pairs} and \textbf{twisted expanded
degenerations}, with the following properties.

\begin{enumerate}[(1)]
\item (\textbf{Moduli interpretation})
For any scheme $S$, objects of $\mathcal T^{\mathrm{tw}}(S)$ (resp.\
$\mathfrak T^{\mathrm{tw}}(S)$) are flat families over $S$ whose geometric
fibers are \textbf{standard $r$--twisted expansions} of the universal pair
$(\A^1,0)$ (resp.\ of the universal degeneration $\A^2\to \A^1$), in the sense
of \cite[Def.\ 2.4.2]{ACFW}.
In particular, these families are obtained from the basic local node
$xy=t$ by inserting chains of $\P^1$'s and equipping the resulting nodes with
balanced stabilizers (equivalently, by iterated balanced root constructions at
the boundary).

\item (\textbf{Algebraicity and finiteness})
The stacks $\mathcal T^{\mathrm{tw}}$ and $\mathfrak T^{\mathrm{tw}}$ are algebraic
(indeed Deligne--Mumford) and locally of finite type (equivalently, locally of
finite presentation) over $\Spec\Z$.
Moreover, the locus of expansions of length $\le r$ is an open substack
\[
\mathcal T^{\mathrm{tw}}_{\le r}\subset \mathcal T^{\mathrm{tw}},
\qquad
\mathfrak T^{\mathrm{tw}}_{\le r}\subset \mathfrak T^{\mathrm{tw}},
\]
and is locally of finite type.
This follows from \cite[Lem.\ 3.1.3]{ACFW} (and the discussion surrounding it).

\item (\textbf{Base change})
For any morphism $S'\to S$, formation of twisted expansions commutes with base
change: the pullback of a twisted expansion family over $S$ is a twisted
expansion family over $S'$.
Equivalently, $\mathcal T^{\mathrm{tw}}$ and $\mathfrak T^{\mathrm{tw}}$ define
categories fibered in groupoids over schemes.
\end{enumerate}
\end{proposition}


\noindent\textbf{Step 3: $G$--bundles on twisted curves.}
Let $\Bun_G^{tw}\to \mathfrak M^{tw}_{g,I}$ be the stack assigning to a twisted
curve $\mathcal C\to B$ the groupoid of principal $G$--bundles on $\mathcal C$. For reductive $G$, this stack is algebraic and locally of finite presentation over
$\mathfrak M^{tw}_{g,I}$.

\medskip
\noindent\textbf{Step 4: The ambient stack.}
Form the fiber product
\[
\mathcal Y :=
\mathsf{TwMdf}_{\le r,g,I}
\times_{\mathfrak M^{tw}_{g,I}}
\Bun_G^{tw}.
\]
An object of $\mathcal Y(B)$ is a twisted modification
$\mathcal C'_B\to \Sigma_B$ together with a principal $G$--bundle on
$\mathcal C'_B$. By Steps~2 and~3, $\mathcal Y$ is an algebraic stack, locally of
finite type over $\overline{\mathcal M}_{g,I}$.

\medskip
\noindent\textbf{Step 5: Admissibility.}
The admissibility condition on $G$-bundles is a restriction on the local monodromy homomorphisms
$\mu_k\to G$ at the twisted nodes, requiring the associated rational cocharacters
to lie in a fixed finite set and satisfy a linear independence condition.
After restricting to strata where the set of nodes is locally constant, these
local types are discrete invariants and are locally constant in families.
Since only finitely many types are allowed, the admissible locus defines an
open-and-closed substack
\[
\mathcal X_{G,g,I}\subset \mathcal Y.
\]

Being an open-and-closed substack of the algebraic stack $\mathcal Y$,
$\mathcal X_{G,g,I}$ is algebraic and locally of finite type over
$\overline{\mathcal M}_{g,I}$. This proves the proposition.
\end{proof}


\subsection{Etale presentation by an algebraic space}
Fix a geometric point $[\Sigma_0,\sigma_{0,i}] \in \overline{\mathcal M}_{g,I}$ and let
$B\to \overline{\mathcal M}_{g,I}$ be an affine \'etale neighborhood. Let
\[
\Sigma := \overline{\mathfrak C}_{g,I}\times_{\overline{\mathcal M}_{g,I}} B \to B
\]
be the pulled-back universal curve. Let $V$ be the set of stable components of $\Sigma_0$
(equivalently, the stable vertices of its dual graph). After possibly refining $B$
\'etale-locally, choose sections
\[
\sigma_v : B \to \Sigma \qquad (v\in V)
\]
landing in the smooth locus of $\Sigma\to B$ such that every stable component of every
geometric fiber meets at least one $\sigma_v$.

Let $\mathcal X|_B := \mathcal X_{G,g,I}\times_{\overline{\mathcal M}_{g,I}} B$.

\begin{definition}[Framed chart]
\label{def:framed-chart}
Define a category fibered in groupoids $A\to(\Sch/B)$ by the following assignment.
For a $B$--scheme $T\to B$, an object of $A(T)$ is a triple
\[
(\mathcal C'_T \xrightarrow{m} \Sigma_T,\ \mathcal P_T,\ (t_v)_{v\in V})
\]
where:
\begin{enumerate}[(i)]
\item $\Sigma_T := \Sigma\times_B T$;
\item $m:\mathcal C'_T\to \Sigma_T$ is a twisted modification (with the chosen length bound);
\item $\mathcal P_T$ is an admissible principal $G$--bundle on $\mathcal C'_T$;
\item for each $v\in V$, a \textbf{framing} is an isomorphism of $G$--torsors
\[
t_v : (\sigma_{v,T})^*\mathcal P_T \xrightarrow{\ \sim\ } G_T.
\]
\end{enumerate}
Morphisms are isomorphisms of $(\mathcal C'_T,m,\mathcal P_T)$ compatible with all $t_v$.
\end{definition}

Let
\[
H := G^V
\]
act on $A$ by changing framings: for $h=(h_v)\in H(T)$ send $t_v$ to $h_v\circ t_v$.
There is a forgetful morphism
\[
\pi: A \longrightarrow \mathcal X|_B
\]
forgetting the framings.


However unlike the case of line bundles, framings do not necessarily rigidify $G$--bundles, as the following example shows.
\begin{example}[Why a single framing does not rigidify $G$--bundles]\label{ex:SL2-framing-fails}
Let $C=\P^1$ and $G=\SL_2$.  Consider the rank--$2$ vector bundle
\[
E \;=\; \mathcal O_{\P^1}(1)\oplus \mathcal O_{\P^1}(-1),
\qquad \det(E)\cong \mathcal O_{\P^1},
\]
and let $\mathcal P$ be the associated principal $\SL_2$--bundle.

Automorphisms of $\mathcal P$ are the same as determinant--$1$
automorphisms of $E$.  There is a unipotent subgroup
\[
U
\;=\;
\left\{
\begin{pmatrix}
1 & s \\
0 & 1
\end{pmatrix}
\;\middle|\;
s\in H^0\!\bigl(\P^1,\Hom(\mathcal O(-1),\mathcal O(1))\bigr)
\right\}
\;\subset\;
\Aut(\mathcal P).
\]
Since $\Hom(\mathcal O(-1),\mathcal O(1))\cong \mathcal O(2)$ and
$H^0(\P^1,\mathcal O(2))\cong \C^3$, we obtain
\[
U \cong \Ga^3.
\]
In particular, $\Aut(\mathcal P)$ is positive--dimensional and far larger
than the center of $\SL_2$.

Now fix a point $p\in\P^1$ and choose a framing
$t:\mathcal P|_p \cong \SL_2$, equivalently a basis of $E_p$.
An element of $U$ preserves this framing if and only if it acts trivially
on the fiber $E_p$, i.e.\ if $s(p)=0$.  The subspace of sections vanishing at
$p$ has dimension $2$:
\[
\{s\in H^0(\P^1,\mathcal O(2)) \mid s(p)=0\}\cong \C^2.
\]
Therefore,
\[
\Aut(\mathcal P,t) \;\supset\; \Ga^2,
\]
and the framed bundle still has a positive--dimensional automorphism group.

This shows that, unlike the case $G=\C^\times$, a single trivialization
point does \textbf{not} rigidify a $G$--bundle for general reductive $G$.
Non--central infinitesimal automorphisms coming from
$H^0(C,\ad(\mathcal P))$ may vanish at a point while remaining nonzero
globally.
\end{example}

In the $\C^\times$ case, after étale localization $B \to \overline{\mathcal M}_{g,I}$, one presents
\[
\widetilde{\mathcal M}\vert_B \simeq [A/\mathcal G], \qquad \mathcal G \cong (\C^\times)^V,
\]
where $A$ is an algebraic space obtained by rigidifying bundles with trivializations at points $\sigma_v$ on each stable component. The key input is that once you trivialize at one point per stable component, every automorphism dies, so $A$ has no stabilizers.

For general reductive $G$, this step fails on the full stack $\mathcal X_{G,g,I}$. There are non-central automorphisms of principal $G$-bundles on projective curves which vanish at a point but are nontrivial globally.

What seems to fix this is to restrict to a “regularly stable” locus: require that on every stable component the restricted bundle satisfies
\[
\Aut(P|_X) = Z(G).
\]
On this open locus, any automorphism is central, and a single framing point forces it to be trivial. Then the same construction gives a chart $A^{\mathrm{rs}}$ which is an algebraic space, with
\[
\mathcal X^{\mathrm{rs}}\vert_B \simeq [A^{\mathrm{rs}}/G^V].
\]

So the difficulty is that on the full stack $\mathcal X$ there is no way to uniformly kill stabilizers with finitely many framings, whereas on the regularly stable locus the FTT mechanism goes through unchanged.

I wanted to ask your thoughts on the best way to proceed conceptually: either (1) work on $\mathcal X^{\mathrm{rs}}$ and then try to extend results from this open substack, or (2) attempt a genuinely stacky version of the local chart and redo the local cohomology argument with stabilizers present.


\section{References}
\begin{enumerate}
    \bibitem{ACFW} D. Abramovich, C. Cadman, B. Fantechi, and J. Wise, \textit{Expanded degenerations and pairs}, preprint, arXiv:1208.6322.
    \bibitem{abramovich-vistoli} D. Abramovich and A. Vistoli, \textit{Compactifying the space of stable maps}, J. Amer. Math. Soc. 15 (2002), no. 1, 27–75.
    \bibitem{behrend-fantechi} K. Behrend and B. Fantechi, \textit{The intrinsic normal cone}, Invent. Math. 128 (1997), no. 1, 45–88.
    \bibitem{ftt} Edward Frenkel, Constantin Teleman, and A. J. Tolland, \textit{Gromov-Witten Gauge Theory I}.
    \bibitem{solis} P. Solis, \textit{Wonderful Loop Group Embeddings and Applications to the Moduli of $G$-bundles on Curves}, PhD thesis, ProQuest Dissertations and Theses (2014).
\end{enumerate}
\end{document}