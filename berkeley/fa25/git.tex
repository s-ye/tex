\documentclass[12pt]{article}
\usepackage[english]{babel}
\usepackage[utf8x]{inputenc}
\usepackage[T1]{fontenc}
\usepackage{listings}
\usepackage{bookmark}
\usepackage{tikz}

\makeatletter
\def\input@path{{../../style/}}
\makeatother

\usepackage{../../style/quiver}
\usepackage{../../style/scribe}
\usepackage{fancyhdr}

\usepackage{parskip} % Automatically respects blank lines
\usepackage{booktabs} % For \addlinespace command
\setlength{\parskip}{1em} % Adds more space between paragraphs
\setlength{\parindent}{0pt} % Removes paragraph indentation

\begin{document}


\lhead{Songyu Ye}
\rhead{\today}
\cfoot{\thepage}

\title{Geometric invariant theory}

\author{Songyu Ye}
\date{\today}
\maketitle


\begin{abstract}

These are reading notes for \emph{Geometric Invariant Theory} by Mumford, Fogarty and Kirwan.
\end{abstract}

\section{Example}
Consider the action of $G = \PGL(n+1)$ on $X = (\mathbb{P}^n)^{m+1}$ using the line bundle
\[
L = \mathcal{O}_{\mathbb{P}^n}(1)^{\boxtimes (m+1)} = \mathcal{O}(1, \ldots, 1) = \pi_1^* \mathcal{O}_{\mathbb{P}^n}(1) \otimes \cdots \otimes \pi_{m+1}^* \mathcal{O}_{\mathbb{P}^n}(1)
\] where $\pi_i : X \to \mathbb{P}^n$ is the projection to the $i$-th factor.
We need to lift the geometric action of $G$ on $X$ to a linear action on $L$. The natural group that acts linearly on $\mathcal{O}_{\mathbb{P}^n}(1)$ is $\GL(n+1)$. There is no canonical way to make an element of $\PGL(n+1)$ act linearly on the fibers of $\mathcal{O}(1)$, because it is only defined up to scalar. The exact sequence is
\[
1 \to \mathbb{G}_m \to \GL(n+1) \to \PGL(n+1) \to 1
\]
Why don't we stay with $\GL(n+1)$ instead of $\PGL(n+1)$? Because the center $\mathbb{G}_m$ acts trivially on $X$ and this introduces a useless symmetry which breaks stability.

We can restrict to the subgroup $\SL(n+1) \subset \GL(n+1)$, which kills most of the scalars except for the finite center $\mu_{n+1}$. We want the linearization to descend to $\PGL(n+1)$, so we need the center $\mu_{n+1} = \ker(\SL(n+1) \to \PGL(n+1)
)$ to act trivially on the fibers of $L$. 

On $\mathcal O_{\mathbb P^n}(1)$, a scalar $\zeta I \in \mu_{n+1} \subset SL(n+1)$ acts as multiplication by $\zeta$ on each fiber.

On 
\[
\mathcal L \;=\; \mathcal O_{\mathbb P^n}(1)^{\boxtimes m},
\]
it therefore acts as multiplication by $\zeta^m$.

Hence the $SL(n+1)$--linearization of $\mathcal L$ factors through $PGL(n+1)$ if and only if every $\zeta \in \mu_{n+1}$ acts trivially on $\mathcal L$, i.e.
\[
\zeta^m = 1 \quad \text{for all } \zeta \text{ with } \zeta^{n+1} = 1.
\]
This holds if and only if
\[
n+1 \mid m.
\] More generally, if we consider the line bundle
\[  L_i = \mathcal O_{\mathbb P^n}(a_i) \] on the $i$-th factor, then the same argument shows that the $SL(n+1)$--linearization of
\[ \mathcal L = \bigotimes_{i=1}^{m+1} \pi_i^* L_i = \mathcal O_{\mathbb P^n}(a_1, \ldots, a_{m+1}) \] descends to $PGL(n+1)$ if and only if
\[n+1 \mid \sum_{i=1}^{m+1} a_i.\]

In any case, by means of these linearizations we can define invariant sections of all the sheaves $\mathcal L_{\boldsymbol\alpha}$. To construct such invariant sections, let $X_0,\ldots,X_n$ be the canonical sections of $\mathcal O_{\mathbb P^n}(1)$ on $\mathbb P^n$. Let
\[
X_i^{(j)} = \pi_j^*(X_i)
\]
be the induced sections of $L_j$.


\begin{definition}
For all sequences $\boldsymbol\alpha = (\alpha_0,\ldots,\alpha_n)$ of integers such that $0 \le \alpha_i \le m$, let
\[
D_{\alpha_0,\ldots,\alpha_n}
=
\det\bigl( X_i^{(\alpha_j)} \bigr)_{0 \le i,j \le n}
\]
be the section of $L_{\alpha_0} \otimes \cdots \otimes L_{\alpha_n}$ obtained by addition and tensor product as in the determinant.
\end{definition}

 It is evident that $D_{\alpha_0,\ldots,\alpha_n}$ is an invariant section of
$L_{\alpha_0} \otimes \cdots \otimes L_{\alpha_n}$. The non-vanishing of suitable $D$'s defines the open sets we are looking for. Explicitly, a point of $(\mathbb{P}^n)^{m+1}$ is a tuple
\[
(p_0,\ldots,p_m), \quad p_j = [v_j], \quad v_j \in k^{n+1} \setminus \{0\}.
\]

Choose homogeneous lifts $v_j$. Put them as columns of a matrix
\[
M = [v_0 \mid v_1 \mid \cdots \mid v_m] \in \text{Mat}_{n+1,m+1}.
\]
Then 
\[
D_{\alpha_0,\ldots,\alpha_n} = \det(v_{\alpha_0}, \ldots, v_{\alpha_n}).
\]
The fact that $D_{\alpha_0,\ldots,\alpha_n}$ is well defined as a section of $L_{\alpha_0} \otimes \cdots \otimes L_{\alpha_n}$ follows from the following properties:
\begin{itemize}
\item $D_{\alpha_0,\ldots,\alpha_n} = 0$ iff the points $p_{\alpha_0}, \ldots, p_{\alpha_n}$ lie in a hyperplane.
\item Under $g \in \GL(n+1)$, all minors are multiplied by $\det(g)$.
\item Rescaling columns rescales the corresponding minors.
\end{itemize}

\begin{definition}
An $R$-partition of $\{0,1,\ldots,n\}$ is an ordered set of subsets $S_1,\ldots,S_\nu$ of $\{0,1,\ldots,n\}$ such that
\begin{enumerate}
\item[(i)] $S_i \cap (S_1 \cup \cdots \cup S_{i-1})$ consists of exactly one integer for $i=2,\ldots,
\nu$
\item[(ii)] $\bigcup_i S_i = \{0,1,\ldots,n\}$.
\end{enumerate}
\end{definition}

\begin{definition}
Given an $R$-partition $R=\{S_1,\ldots,S_\nu\}$, let $U_R \subset (\mathbb P^n)^{m+1}$ be the open subset defined by
\begin{enumerate}
\item[(i)] $D_{0,1,\ldots,n} \neq 0$,
\item[(ii)] for all $k$ between $1$ and $\nu$, and for all $i \in S_k$,
\[
D_{0,\ldots,i-1,i+1,\ldots,n,n+k} \neq 0
\]
\end{enumerate}
\end{definition}

Not only is $U_R$ affine, but the whole structure of the action of $PGL(n+1)$ on $U_R$ can be described explicitly. On each open set $U_R$, a configuration of points in $(\mathbb{P}^n)^{m+1}$ is uniquely the same thing as
\begin{enumerate}
\item a projective frame, and
\item a collection of free affine parameters
\end{enumerate}


\begin{proposition}
Let $R=\{S_1,\ldots,S_\nu\}$ be an $R$-partition of $\{0,1,\ldots,n\}$. Let $\PGL(n+1)$ act on $PGL(n+1)\times \mathbb A^{n\nu - n}$ by the product of left translation on itself and the trivial action on the affine space. Then there is a $\PGL(n+1)$-linear isomorphism:
\[
U_R \cong \PGL(n+1) \times \mathbb A^{n\nu - n}.
\]

Hence $U_R$ is a globally trivial principal fibre bundle with respect to the action of $\PGL(n+1)$, with base space $\mathbb A^{n\nu - n}$.
\end{proposition}
\begin{proof}

Fix an $R$-partition and the associated open set $U_R \subset X$.
On $U_R$, define sections $\lambda_j$ by
\[
\lambda_{\mu(1)} := 1, \qquad
\lambda_j := \lambda_{\mu(x(j))}\,
\frac{D_{0,1,\dots,\widehat{\mu(x(j))},\dots,n,j}}
     {D_{0,1,\dots,n}}
\quad (0 \le j \le n).
\]

These satisfy $\lambda_j \in \Gamma(U_R, L_j \otimes L_{\mu(1)}^{-1})$.

We define
\[
\phi = (\phi_1,\phi_2) : U_R \longrightarrow PGL(n+1) \times A_R
\]
as follows.

Identifying $PGL(n+1)$ with the open subset $\{\det \neq 0\} \subset \mathbb P^{(n+1)^2-1}$
with homogeneous coordinates $a_{ij}$, define $\phi_1$ by
\[
(\phi_1)^*(a_{ij}) = (-1)^j\, X_i^{(j)} \otimes \lambda_j^{-1}.
\]

Define $\phi_2$ by, for $k \ge 1$,
\[
(\phi_2)^*(x_i^{(n+k)}) =
\frac{D_{0,1,\dots,\widehat{i},\dots,n,n+k}}
     {D_{0,1,\dots,n}}\,
\frac{\lambda_i}{\lambda_{\mu(k)}}.
\]

Then $\phi$ is a $PGL(n+1)$-equivariant isomorphism
\[
U_R \;\cong\; PGL(n+1) \times A_R.
\]
Unraveling over a field, a point of $U_R$ corresponds to a tuple of points $(p_0,\ldots,p_m)$ in $(\mathbb P^n)^{m+1}$ satisfying the conditions defining $U_R$. Writing $p_j = [v_j]$ with $v_j \in k^{n+1} \setminus \{0\}$ with $\det [v_0 \mid v_1 \mid \cdots \mid v_n] \neq 0$, the map $\phi$ sends this point to
\[\inv{A}, \inv{A} v_{n+1}, \ldots, \inv{A} v_m\]
where $A = [v_0 \mid v_1 \mid \cdots \mid v_n]$.
\end{proof}

What is this R-partition formalism really saying? When $\nu = 1$, then we are forced to take $R = \set{S_1}$ and $S_1 = \set{0,1,\ldots,n}$. The definition of $U_R$ reads $D_{0,1,\ldots,n} \neq 0$, so  $U_R$ is those tuples of points $(p_0,\ldots,p_{n+1})$ for which the points $p_0,\ldots,p_n$ are not colinear and $p_{n+1}$ is not in any of the coordinate hyperplanes determined by them, i.e. $n+2$ points in $\P^n$ in general position. Then there exists a unique projective transformation $g \in PGL(n+1)$ sending $p_i$ to $e_i$ for $i=0,\ldots,n$ and sending $p_{n+1}$ to $[1:1:\cdots:1]$. So in particular $U_R = \PGL(n+1)$ in this case.

We can also study the case when $n=2$, $\nu = 2$ and $S_1 = \set{0,1}$ and $S_2 = \set{1,2}$. We are then thinking about $m = n+\nu+1=5$ points in $\P^2$, say $(p_0,p_1,p_2,p_3,p_4)$. The set $U_R$ is then cut out by the nonvanishing of the determinants $D_{0,1,2} $, $D_{1,2,3}$, $D_{0,2,3}$, $D_{0,2,4}$, and $D_{0,1,4}$. Geometrically this means that $p_0,p_1,p_2$ are not colinear, $p_3$ does not lie on the lines $p_1p_2$ and $p_0p_2$, and $p_4$ does not lie on the lines $p_0p_1$ and $p_0p_2$. Then we can uniquely normalize $p_0,p_1,p_2$ to be the coordinate points $[1:0:0]$, $[0:1:0]$, $[0:0:1]$, and normalize $p_3$ and $p_4$ to be $[1,1,a]$ and $[b,1,1]$.


\begin{definition}
    For fixed $m$ let $U_{reg} \subset (\mathbb P^n)^{m+1}$ be union of all $U_R$ where $R$ runs through all R-partitions of $\{0,1,\ldots,n\}$ with $\nu = m - n$.
\end{definition}

\begin{proposition}[Proposition 3.3]
Let $x = (x^{(0)},x^{(1)},\dots,x^{(m)})$ be a geometric point of $(\mathbb P^n)^{m+1}$. Then the following are equivalent:
\begin{enumerate}
\item The stabilizer $S(x)$ is $0$-dimensional.
\item There do not exist disjoint proper linear subspaces $L'$ and $L''$ of $\mathbb P^n$ such that every $x^{(i)}$ lies in either $L'$ or $L''$.
\item $x$ is a geometric point of $U_{\mathrm{reg}}$.
\end{enumerate}
\end{proposition}

\begin{proof}
Let $k$ be the algebraically closed field over which $x$ is defined. For simplicity, we shall write $\mathbb P$ for $\mathbb P^n_k$, and $U_R$ for $U_R \times_k \Spec k$, etc., in the course of this proof.

First, the implication $(1)\Rightarrow(2)$ is clear; for in suitable homogeneous coordinates $\{X_i\}$, one may assume that
\[
L' \subset \{X_0 = X_1 = \cdots = X_r = 0\}, \qquad
L'' \subset \{X_{r+1} = \cdots = X_n = 0\}.
\]
Then the subgroup of transformations
\[
\begin{pmatrix}
\alpha I_{r+1} & 0 \\
0 & \beta I_{n-r}
\end{pmatrix}
\subset PGL(n+1)
\]
leaves $x$ fixed.

Secondly, $(3)\Rightarrow(1)$ is an immediate consequence of the equivariant trivialization $U_R \cong PGL(n+1) \times \mathbb A^{n\nu - n}$ and the fact that $PGL(n+1)$ acts freely on itself by left translation.

Thirdly, we will prove that $(2)\Rightarrow(3)$. By virtue of $(2)$, all the points $x^{(i)}$ cannot lie in one hyperplane, hence we can choose $n+1$ of the $x^{(i)}$ which are not in one hyperplane, say $x^{(0)},x^{(1)},\dots,x^{(n)}$. Without loss of generality, we may assume that these have homogeneous coordinates $x^{(j)}_i = \delta_{ij}$.

Now for each $n+k$ between $n+1$ and $m$, let $S_k$ be the set of integers $i$ such that
\[
D_{0,1,\dots,\widehat{i},\dots,n,n+k} \neq 0,
\]
i.e.\ $x^{(n+k)}$ is not in the hyperplane spanned by
$x^{(0)},\dots,\widehat{x^{(i)}},\dots,x^{(n)}$.

Then I claim that there is no partition of the set $\{0,1,\dots,n\}$ into two disjoint subsets $T'$ and $T''$ such that every $S_k$ is contained in either $T'$ or $T''$. For if there were, and if one let $L'$ (resp.\ $L''$) be the linear subspace defined by $X_i = 0$ for all $i \in T'$ (resp.\ $i \in T''$), then every point $x^{(k)}$ would lie in $L' \cup L''$, contradicting~(2).

It follows immediately from a combinatorial argument that a suitable set of subsets $S_i \subset S_j$ is an $R$-partition $R$ and that $x \in U_R$.
\end{proof}
It thus follows that $U_{reg}$ is the locus of prestable points in the following sense. Let $G$ be a reductive algebraic group
acting via $\sigma$, on $X$ scheme of finite type over a field $k$. Now suppose that $L$ is an invertible sheaf on $X$ and that
$\phi$ is a $G$-linearization of $L$.
The key concepts are the following.

\begin{definition}[Mumford, Definition 1.7]
Let $x$ be a geometric point of $X$.
\begin{enumerate}
\item[(a)]
$x$ is \emph{pre-stable} (with respect to $\sigma$) if there exists
an invariant affine open subset $U \subset X$ such that
$x \in U$ and every $G$-orbit in $U$ is closed in $U$ 



\item[(b)]
$x$ is \emph{semi-stable} (with respect to $\sigma,L,\phi$) if there exists
a section $s \in H^0(X,L^{\otimes n})$ for some $n>0$ such that
$s(x)\neq 0$, the open subset $X_s$ is affine, and $s$ is invariant.
Equivalently, if $\phi_n : \sigma^*(L^{\otimes n}) \to p_2^*(L^{\otimes n})$
is the induced linearization, then
\[
\phi_n(\sigma^* s) = p_2^*(s).
\]

\item[(c)]
$x$ is \emph{stable} (with respect to $\sigma,L,\phi$) if there exists
a section $s \in H^0(X,L^{\otimes n})$ for some $n>0$ such that
$s(x)\neq 0$, the open subset $X_s$ is affine, $s$ is invariant,
and the action of $G$ on $X_s$ is closed.
\end{enumerate}
\end{definition}

A single closed orbit can sit inside a region where nearby orbits are wildly non-closed. That gives bad quotient behavior. You cannot form a reasonable local quotient around such a point.

For example, consider the action of $\Gm$ on $\A^2$ by
\[t \cdot (x,y) = (tx, t^{-1}y).\] The origin is a closed orbit, but any neighborhood of the origin contains points whose orbits are not closed (e.g. points on the hyperbolas $xy = c \neq 0$). Thus the origin is not prestable.

Prestable points are exactly those that sit inside a region where no orbit collapses onto another. This is the weakest hypothesis under which local quotients look like honest orbit spaces.

\begin{definition}[Mumford, Definition 0.6]
Given an action of $G/S$ on $X/S$, a pair $(Y,\phi)$ consisting of a pre-scheme $Y$
over $S$ and an $S$-morphism $\phi : X \to Y$ is called a \emph{geometric quotient}
(of $X$ by $G$) if the following conditions are satisfied:

\begin{enumerate}
\item[(i)]
$\phi \circ \sigma = \phi \circ p_2$ (as in Definition~0.5).

\item[(ii)]
$\phi$ is surjective, and the image of $\Psi$ is $X \times_Y X$ (cf.\ Definition~0.4).
Equivalently, the geometric fibres of $\phi$ are precisely the orbits of the geometric
points of $X$ (for geometric points over an algebraically closed field of sufficiently
high transcendence degree).

\item[(iii)]
$\phi$ is submersive, i.e.\ a subset $U \subset Y$ is open if and only if
$\phi^{-1}(U)$ is open in $X$. Likewise, $U' \subset Y'$ is open if and only if
$\phi^{-1}(U')$ is open in $X'$.

\item[(iv)]
The fundamental sheaf $\mathcal O_Y$ is the subsheaf of $\phi_*(\mathcal O_X)$ consisting
of invariant functions. That is, if $f \in \Gamma(U,\phi_*(\mathcal O_X))
= \Gamma(\phi^{-1}(U),\mathcal O_X)$, then $f \in \Gamma(U,\mathcal O_Y)$ if and only if
the diagram
\[
\begin{tikzcd}
G \times \phi^{-1}(U) \ar[r,"\sigma"] \ar[d,"p_2"'] &
\phi^{-1}(U) \ar[d,"F"] \\
\phi^{-1}(U) \ar[r,"F"] &
\mathbb A^1
\end{tikzcd}
\]
commutes, where $F$ is the morphism defined by $f$.
\end{enumerate}
\end{definition}


Our next step is to construct a geometric quotient of $U_{\mathrm{reg}}$ by $PGL(n+1)$.
Let $U_1,\dots,U_N$ be the open subsets $U_R$ of $U_{\mathrm{reg}}$ and the subsets obtained
from these by permuting the coordinates. Let $(Z_i,\phi_i)$ be the geometric quotient
of $U_i$ by $PGL(n+1)$. For all pairs $i,j$, $U_i\cap U_j$ is an invariant open subset
of $U_i$ and $U_j$. Therefore, by Corollary~3.2, if $\sigma_i : Z_i \to U_i$ is the global
section of $\phi_i$, we know:
\[
PGL(n+1)\times \sigma_i^{-1}(U_i\cap U_j) \;\cong\;
U_i\cap U_j \;\cong\;
PGL(n+1)\times \sigma_j^{-1}(U_i\cap U_j).
\]

In other words, both $\sigma_i^{-1}(U_i\cap U_j)$ and $\sigma_j^{-1}(U_i\cap U_j)$ are
geometric quotients of $U_i\cap U_j$ by $PGL(n+1)$; therefore, they are canonically
isomorphic. We use this isomorphism to glue together $Z_i$ and $Z_j$. For any three of
the quotients $Z_i,Z_j,Z_k$, these identifications are obviously compatible. Therefore,
we have defined a pre-scheme $Z$ and a morphism $\phi : U_{\mathrm{reg}} \to Z$. Clearly,
$U_{\mathrm{reg}}$ is a locally trivial principal fibre bundle over $Z$; a fortiori,
$(Z,\phi)$ is a geometric quotient of $U_{\mathrm{reg}}$ by $PGL(n+1)$. However, $Z$ is
very far from being a scheme, let alone being quasi-projective.

Although $Z$ is not quasi-projective, it carries various invertible sheaves. To
investigate these, we make use of the theory of descent: by SGA~8, \S1, the set of
invertible sheaves on $Z$ is isomorphic to the set of invertible sheaves on
$U_{\mathrm{reg}}$ plus descent data for $\phi$. But $\phi$-descent data is precisely the
same as a $PGL(n+1)$-linearization, since:
\[
U_{\mathrm{reg}} \times_Z U_{\mathrm{reg}} \;\cong\; PGL(n+1)\times U_{\mathrm{reg}}.
\]

But $L_i^{\,n+1}$ admits a $PGL(n+1)$-linearization. Therefore, there is an invertible
sheaf $M_i$ on $Z$ such that
\[
L_i^{\,n+1} \;\cong\; \phi^*(M_i).
\]

Moreover, the section
\[
(D_{\alpha_0,\dots,\alpha_n})^{n+1} \in
\Gamma\!\left(U_{\mathrm{reg}}, L_{\alpha_0}^{\,n+1}\otimes\cdots\otimes L_{\alpha_n}^{\,n+1}\right)
\]
is invariant in the $SL(n+1)$-linearization of this sheaf, hence in the
$PGL(n+1)$-linearization of this sheaf. Therefore, according to SGA~8, \S1, there is a
section $E_{\alpha_0,\dots,\alpha_n}$ of
\[
M_{\alpha_0}\otimes\cdots\otimes M_{\alpha_n}
\]
such that
\[
(D_{\alpha_0,\dots,\alpha_n})^{n+1} = \phi^*(E_{\alpha_0,\dots,\alpha_n}).
\]

\end{document}