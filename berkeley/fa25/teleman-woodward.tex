\documentclass[12pt]{article}
\usepackage[english]{babel}
\usepackage[utf8x]{inputenc}
\usepackage[T1]{fontenc}
\usepackage{listings}
\usepackage{bookmark}
\usepackage{tikz}

\makeatletter
\def\input@path{{../../style/}}
\makeatother

\usepackage{../../style/quiver}
\makeatletter
\def\input@path{{../../style/}}
\makeatother

\usepackage{../../style/scribe}
\usepackage{fancyhdr}

\usepackage{parskip} % Automatically respects blank lines
\setlength{\parskip}{1em} % Adds more space between paragraphs
\setlength{\parindent}{0pt} % Removes paragraph indentation
\DeclareMathOperator{\Eul}{Eul}
\DeclareMathOperator{\Ind}{Ind}

\begin{document}


\lhead{Songyu Ye}
\rhead{\today}
\cfoot{\thepage}

\title{Teleman Woodward}

\author{Songyu Ye}
\date{\today}
\maketitle


\begin{abstract}
Abstract
\end{abstract}

\tableofcontents
\section{Overview of the finiteness theorem from Teleman Woodward}
Let $G$ be a connected reductive group and $\mf M = \Bun_G(C)$ the moduli stack of $G$-bundles on a smooth projective curve $C$. The goal of Teleman-Woodward is to compute the index of certain $K$-theory classes on $\mf M$, generalizing the Verlinde formula for line bundles.

However, our goal is to merely establish the finiteness of the index in the case of nodal curves. Abstracting from the paper, the finiteness theorem has the following structure:
\begin{enumerate}
    \item Stratify the full stack $\mathfrak{M}$ by Harder–Narasimhan type $\mathfrak{M} = \bigsqcup_\xi \mathfrak{M}_\xi$. \red{Include some explanation of HN type here.}
    \item Filter $R\Gamma(\mathfrak{M},E)$ by local cohomology along the closed unions $\bigcup_{\xi' \ge \xi} \mathfrak{M}_{\xi'}$.
    \item Show that admissibility forces the local terms for $\xi$ sufficiently large to vanish.
    \item Deduce that only finitely many strata contribute, hence the index is finite.
\end{enumerate}

\subsection*{Admissible classes (Teleman--Woodward)}

Let $\Sigma$ be a smooth projective curve, $G$ a connected reductive group, and
$\mf M=\Bun_G(\Sigma)$ the moduli stack of $G$--bundles.  Let $\mathcal E$ denote
the universal $G$--bundle on $\Sigma\times\mf M$, and for a finite--dimensional
representation $V$ of $G$ write $\mathcal E^*V$ for the associated vector bundle.

Teleman--Woodward single out the following natural $K$--theory classes on $\mf M$:

\begin{enumerate}[(i)]
\item $E_x^*V\in K^0(\mf M)$, the restriction of $\mathcal E^*V$ to $\{x\}\times\mf M$,
for a point $x\in\Sigma$;

\item $E_C^*V := \mathcal E^*V/[C]\in K^{-1}(\mf M)$, the slant product of $\mathcal E^*V$
with a $1$--cycle $C$ on $\Sigma$;

\item $E_\Sigma^*V := R\pi_*(\mathcal E^*V\otimes\sqrt K)\in K^0(\mf M)$, the Dirac index
bundle along $\Sigma$, where $\pi:\Sigma\times\mf M\to\mf M$ and $\sqrt K$ is a square
root of the canonical bundle of $\Sigma$;

\item $D_\Sigma V := \det^{-1}E_\Sigma^*V$, the inverse determinant of cohomology.
\end{enumerate}

The classes in (i)--(iii) are called the \emph{Atiyah--Bott generators}.  The classes
in (iv) are determinant line bundles on $\mf M$.

The first Chern class of any determinant line bundle
$\mathcal L$ defines an invariant quadratic form
\[
h_\mathcal L\in H^4(BG;\R)\cong \Sym^2(\mf g^*)^G \cong
\{\text{invariant symmetric bilinear forms on }\mf g\},
\]
called the \emph{level} of $\mathcal L$.  Let $c$ denote the distinguished quadratic
form corresponding to the canonical bundle
$\mathcal K=\det E_\Sigma^*\mf g$.

\begin{definition}[Teleman--Woodward]
A line bundle $\mathcal L$ on $\mf M$ is called \emph{admissible} if the shifted
quadratic form
\[
h_\mathcal L + c
\]
is positive definite on $\mf g$. An \emph{admissible class} in $K^*(\mf M)$ is any finite product of an admissible line
bundle with Atiyah--Bott generators.
\end{definition}

\subsection*{Shatz stratification}

Recall that any $G$--bundle over $\Sigma$ admits a canonical reduction of structure
group to a standard parabolic subgroup $P\subset G$, for which the associated
bundle with Levi structure group is semistable. 

\begin{remark}
    For a principal $G$-bundle $P$ on a smooth curve $\Sigma$, there is a Harder–Narasimhan (HN) theory generalizing the usual HN filtration of vector bundles. The outcome is a canonical reduction of $P$ to a parabolic subgroup $P\subset G$. “Canonical” means: determined functorially by $P$ (up to unique isomorphism), not a choice. The defining property is that if you pass from $P$ (the parabolic) to its Levi quotient $L=P/R_u(P)$, the induced $L$-bundle is semistable.
“Standard parabolic” means: a parabolic containing a fixed Borel $B$ (chosen once), so parabolics are indexed by subsets of simple roots.

Intuition: this parabolic reduction packages “the most destabilizing” subobject(s) of the bundle.
\end{remark}


Topologically, this reduction
is classified by a coweight of $P/[P,P]$; we identify this with a (possibly
fractional) dominant coweight $\xi$ of $\mf g$, called the \emph{instability type}
of the original bundle.  Then $P$ is the standard parabolic subgroup defined by
$\xi$; we denote it by $P_\xi$ and its Levi subgroup by $G_\xi$.  If $\mf M_\xi$
denotes the stack of $G$--bundles of type $\xi$, we have an algebraic
stratification 
\[
\mf M = \bigsqcup_\xi \mf M_\xi .
\]
Sending a $P_\xi$--bundle to its associated Levi bundle defines a morphism
from $\mf M_\xi$ to the stack $\mf M^{ss}_{G_\xi,\xi}$ of semistable principal
$G_\xi$--bundles of type $\xi$; the fibres are quotient stacks of affine spaces
by unipotent groups (equivalently the corresponding Lie algebra is nilpotent).  The virtual normal bundle for the morphism
$\mf M^{ss}_{G_\xi,\xi} \to \mf M$ is the complex
\[
\nu_\xi = R\pi_*\mathcal E^*(\mf g/\mf g_\xi)[1].
\]

Its $K$--theory Euler class should be the alternating sum of exterior powers
\[
\lambda_{-1}(\nu_\xi^\vee) := \sum_p (-1)^p \wedge^p(\nu_\xi^\vee),
\]
but for now this infinite sum is only a formal expression, whose meaning is to
be spelled out.

\subsection*{Local cohomology}

Finite open unions of Shatz strata
\[
\mf M_{\le \xi} = \bigcup_{\mu\le \xi} \mf M_\mu
\]
can be presented as quotient stacks of smooth quasi--projective varieties by
reductive groups.  The cohomology with supports over $\mf M_\xi$ of a vector
bundle $\mathcal E$ is
\[
H^\bullet_{\mf M_\xi}(\mf M_{\le \xi},\mathcal E_{\le \xi})
= H^{\bullet + d_\xi}(\mf M_\xi, \mathcal R_\xi \mathcal E),
\tag{1.9}
\]
where $d_\xi$ is the codimension of $\mf M_\xi$ and $\mathcal R_\xi \mathcal E$
denotes the sheaf of "$\mathcal E$--valued residues along $\mf M_\xi$." In particular \[\mathcal R_\xi \mathcal E \;:=\; i_\xi^! (\mathcal E_{\le \xi})[-d_\xi]\]

where \(i_\xi:\mf M_\xi\hookrightarrow \mf M_{\le \xi}\) is the inclusion and $i^!$ is the extraordinary pullback (local duality functor).

This is a stacky derived version of the fact that for a smooth closed subvariety $Z \subset X$, local cohomology along $Z$ equals cohomology on $Z$ twisted by the normal bundle and shifted by codimension.

\red{Basically I think we need to find the right stratification and Dan HL has a machine that produces such stratifications, known as $\theta$-stratifications. }

\subsection*{Role of the Shatz stratification in Teleman--Woodward}

The proof of the finiteness theorem in \cite{TelemanWoodward} is organized around
the Harder--Narasimhan (Shatz) stratification of the moduli stack
\[
\mf M=\Bun_G(\Sigma)=\bigsqcup_{\xi}\mf M_\xi,
\]
indexed by dominant rational coweights $\xi$.  This stratification plays the role
of a Morse stratification for the Yang--Mills functional, and replaces compactness
in the non--finite--type stack $\mf M$.

\medskip
\noindent\textbf{(1) Filtration by supports.}
The partial order on instability types defines an increasing filtration by open
substacks
\[
\mf M_{\le \xi} := \bigcup_{\mu\le\xi}\mf M_\mu.
\]
For any sheaf or complex $\mathcal E$ on \(\mf M\), this filtration produces a filtration on derived global sections
$R\Gamma(\mf M,\mathcal E)$
by the subcomplexes
$R\Gamma_{\mf M_{\le \xi_k}}(\mf M,\mathcal E)$.

\begin{remark}[General mechanism of local cohomology filtration]
    Suppose you have a space/stack $X$ and an increasing sequence of open substacks
$\emptyset = U_0 \subset U_1 \subset U_2 \subset \cdots \subset X$
with closed complements
$Z_k := X \setminus U_k$. In our situation \(X = \mf M\), \(U_k = \mf M_{\le \xi_k}\), \(Z_k = \bigcup_{\mu > \xi_k} \mf M_\mu\). For any sheaf or complex $\mathcal E$ on $X$, there is a canonical exact triangle\begin{align*}
R\Gamma_{Z_k}(X,\mathcal E)
\;\to\;
R\Gamma(X,\mathcal E)
\;\to\;
R\Gamma(U_k,\mathcal E|_{U_k})
\;\to\;
\end{align*}
This triangle is the definition of local cohomology with supports in $Z_k$.
\end{remark}


The decreasing family of closed substacks $Z_k=X\setminus U_k$ induces a
decreasing filtration $F^k:=R\Gamma_{Z_k}(X,\mathcal E)$ of $R\Gamma(X,\mathcal E)$.
Its successive graded pieces are
\[
\gr^k F \simeq R\Gamma_{Z_k\setminus Z_{k+1}}(U_{k+1},\mathcal E|_{U_{k+1}}).
\]
In the Shatz situation (refining the indexing so that $U_k=U_{k-1}\sqcup \mf M_{\xi_k}$),
this becomes
\[
\gr^{k-1}F \simeq R\Gamma_{\mf M_{\xi_k}}(\mf M_{\le\xi_k},\mathcal E_{\le\xi_k}).
\]
Equivalently there is a spectral sequence with
\[
E_1^{\xi,*} = R\Gamma_{\mf M_\xi}(\mf M_{\le\xi},\mathcal E_{\le\xi})
\quad\Longrightarrow\quad
R\Gamma(\mf M,\mathcal E).
\]

\medskip
\noindent\textbf{(2) Reduction to semistable Levi moduli.}
Each stratum $\mf M_\xi$ carries a canonical morphism
\[
\mf M_\xi \longrightarrow \mf M^{ss}_{G_\xi,\xi}
\]
to the moduli stack of semistable principal $G_\xi$--bundles of fixed topological
type, whose fibres are quotient stacks of affine spaces by unipotent groups.
This identifies $\mf M_\xi$ as a bundle of unstable directions over a semistable
core.

\medskip
\noindent\textbf{(3) Virtual normal complex.}
The stratification provides a uniform description of the virtual normal complex
of $\mf M^{ss}_{G_\xi,\xi}$ in $\mf M$:
\[
\nu_\xi = R\pi_*\mathcal E^*(\mf g/\mf g_\xi)[1].
\]
Consequently, local cohomology along $\mf M_\xi$ may be expressed formally as
\[
R\Gamma_{\mf M_\xi}(\mf M_{\le\xi},\mathcal E)
\simeq
R\Gamma\bigl(\mf M_\xi,
\mathcal E\otimes \lambda_{-1}(\nu_\xi^\vee)^{-1}\bigr),
\]
where $\lambda_{-1}(\nu_\xi^\vee)$ is the $K$--theoretic Euler class of the normal
complex.

\medskip
\noindent\textbf{(4) Weight decomposition and admissibility.}
The representation $\mf g/\mf g_\xi$ decomposes into positive $\xi$--weight
spaces.  This induces a natural grading on $\nu_\xi$ and hence on
$\lambda_{-1}(\nu_\xi^\vee)^{-1}$.  Admissibility of the twisting line bundle
forces all sufficiently unstable types $\xi$ to contribute only strictly negative
weights, so that the formal inverse Euler class becomes summable and the local
contributions vanish for $\xi$ sufficiently large.

\medskip
\noindent\textbf{(5) Finiteness.}
Since only finitely many instability types can contribute nontrivially, the
local--cohomology filtration terminates after finitely many steps.  This yields
the finiteness of the index.

\medskip
In summary, the Shatz stratification supplies a canonical filtration, a reduction
to semistable Levi moduli, and a uniform normal complex whose weight decomposition
is controlled by admissibility.  All finiteness statements in
\cite{TelemanWoodward} are ultimately consequences of this structure.


\section{Toy model: $G=\GL_2$ on a smooth curve}

Let $\Sigma$ be a smooth projective curve of genus $g\ge 2$ over $\C$, and let
\[
\mf M=\Bun_{\GL_2}(\Sigma)
\]
be the moduli stack of rank $2$ vector bundles on $\Sigma$.
We explain explicitly the Shatz stratification, the Levi description of strata,
the virtual normal complex, and the weight bookkeeping behind the
Teleman--Woodward finiteness mechanism in this case.

\subsection{Harder--Narasimhan type and Shatz strata}

Every $E\in \mf M$ admits a unique Harder--Narasimhan filtration
\[
0\subset L \subset E,
\qquad M:=E/L,
\]
where $L$ is a line subbundle of maximal slope.  Write
\[
\deg(L)=d_1,\qquad \deg(M)=d_2,\qquad m:=d_1-d_2\ge 0.
\]
Then $E$ is semistable iff $m=0$ (equivalently $d_1=d_2$).

The Shatz (HN) stratum of type $(d_1,d_2)$ is the locally closed substack
\[
\mf M_{d_1,d_2}\subset \mf M
\]
parametrizing bundles whose HN filtration has graded pieces $(L,M)$ of degrees
$(d_1,d_2)$ (so $m>0$ on unstable strata).  One has the stratification
\[
\mf M=\bigsqcup_{d_1\ge d_2}\mf M_{d_1,d_2}.
\]

\subsection{Parabolic and Levi}

Fix the standard Borel $B\subset \GL_2$ of upper triangular matrices.
For $m>0$, the destabilizing reduction is to the standard parabolic
\[
P=\left\{\begin{pmatrix} * & * \\ 0 & * \end{pmatrix}\right\},
\]
with Levi subgroup
\[
G_\xi \cong \GL_1\times \GL_1
\quad\text{(diagonal matrices).}
\]
Equivalently, the associated dominant coweight (instability type) may be taken as
\[
\xi=\left(\frac m2,-\frac m2\right)\in \mf t_\Q,
\]
so that the positive root $\alpha$ satisfies $\alpha(\xi)=m$.

\subsection{Semistable Levi core and structure of the stratum}

A principal $G_\xi$--bundle is the same as a pair of line bundles $(L,M)$, hence
the moduli stack of semistable $G_\xi$--bundles of type $(d_1,d_2)$ is
\[
\mf M^{ss}_{G_\xi,\xi}\;\cong\;\Pic^{d_1}(\Sigma)\times \Pic^{d_2}(\Sigma).
\]
There is a canonical morphism
\[
q:\mf M_{d_1,d_2}\longrightarrow \Pic^{d_1}(\Sigma)\times \Pic^{d_2}(\Sigma),
\qquad E\mapsto (L,E/L).
\]
Fixing $(L,M)$, the fibre of $q$ over $(L,M)$ classifies extensions
\[
0\to L\to E\to M\to 0,
\]
hence is governed by
\[
\Ext^1(M,L)\cong H^1(\Sigma,L\otimes M^{-1}).
\]
Automorphisms of a given extension (fixing $(L,M)$) come from
\[
\Hom(M,L)\cong H^0(\Sigma,L\otimes M^{-1}),
\]
which is a unipotent group (additively) acting on the affine space
$H^1(\Sigma,L\otimes M^{-1})$ by the usual change-of-splitting.
Thus the fibre is a quotient stack
$\Bigl[H^1(\Sigma,L\otimes M^{-1})\Big/\;H^0(\Sigma,L\otimes M^{-1})\Bigr]$
making $\mf M_{d_1,d_2}$ a (stacky) affine fibration over the semistable Levi core.

\subsection{Virtual normal complex}

The tangent complex of $\Bun_G$ at a $G$--bundle $E$ is
\[
T_{\mf M,E}\simeq R\Gamma(\Sigma,\End(E))[1].
\]
Over the Levi core $(L,M)$ the adjoint representation decomposes as
\[
\End(L\oplus M)
\;=\;
\underbrace{\End(L)\oplus\End(M)}_{\mf g_\xi}
\;\oplus\;
\underbrace{\Hom(M,L)\oplus \Hom(L,M)}_{\mf g/\mf g_\xi}.
\]
Along the stratum $\mf M_{d_1,d_2}$, the relevant (unstable) normal directions
are governed by the positive--weight root space $\Hom(M,L)$, and the virtual normal
complex for the inclusion of the Levi moduli into $\mf M$ is
\[
\nu_\xi \;\simeq\; R\Gamma(\Sigma, L\otimes M^{-1})[1].
\]
Equivalently, $\nu_\xi$ has cohomology
\[
H^{-1}(\nu_\xi)\cong H^0(\Sigma,L\otimes M^{-1}),
\qquad
H^0(\nu_\xi)\cong H^1(\Sigma,L\otimes M^{-1}).
\]

\subsection{$K$--theoretic Euler class and its formal inverse}

Formally, the $K$--theory Euler class of the dual normal complex is
\[
\lambda_{-1}(\nu_\xi^\vee)
=\sum_{p\ge 0}(-1)^p\wedge^p(\nu_\xi^\vee).
\]
Because $\nu_\xi$ is a shifted cohomology complex, its inverse Euler class
expands into exterior powers of the $H^0$--piece and symmetric powers of the
$H^1$--piece.  Schematically one may think of
\[
\lambda_{-1}(\nu_\xi^\vee)^{-1}
\sim
\frac{\Sym^\bullet\bigl(H^1(\Sigma,L\otimes M^{-1})^\vee\bigr)}
{\Lambda^\bullet\bigl(H^0(\Sigma,L\otimes M^{-1})^\vee\bigr)}
\]
an infinite sum in ordinary $K$--theory which is made meaningful in
Teleman--Woodward by working in a suitable completion determined by $\xi$--weights.

\subsection{Weight bookkeeping: linear vs.\ quadratic growth}

The one--parameter subgroup $\xi$ acts on $\Hom(M,L)$ with weight $\alpha(\xi)=m$.
Consequently, $\xi$ acts on $H^i(\Sigma,L\otimes M^{-1})$ with weight $m$, and hence
on the graded summand
\[
\Sym^p\bigl(H^1(\Sigma,L\otimes M^{-1})^\vee\bigr)
\]
with weight $p\,m$.  This is the \emph{linear} growth in the instability parameter $m$.

On the other hand, a determinant line bundle $\mathcal L$ on $\mf M$ has a
level $h_{\mathcal L}\in \Sym^2(\mf g^*)^G$, and Teleman--Woodward introduce the
canonical correction $c$ coming from $\mathcal K=\det E_\Sigma^*\mf g$.
For an \emph{admissible} $\mathcal L$, the form $h_{\mathcal L}+c$ is positive
definite, so
\[
(h_{\mathcal L}+c)(\xi,\xi)\to +\infty \quad\text{as } \|\xi\|\to\infty.
\]
In the $\GL_2$ normalization $\xi=(m/2,-m/2)$ and the standard invariant form
$(X,Y)=\tr(XY)$ on diagonal matrices gives
\[
(\xi,\xi)=\frac{m^2}{2},
\]
so $(h_{\mathcal L}+c)(\xi,\xi)$ grows like a positive constant times $m^2$.
In Teleman--Woodward's local cohomology calculation, twisting by $\mathcal L$
shifts the $\xi$--weight spectrum by a \emph{negative} amount with leading term
\[
-(h_{\mathcal L}+c)(\xi,\xi)\sim -\kappa m^2 \qquad (\kappa>0).
\]

Thus, on the $\xi$--stratum, the inverse Euler class contributes graded pieces with
weights growing at most \emph{linearly} in $m$ (e.g.\ $p\,m$), while an admissible
twist shifts weights by a \emph{quadratic} negative term $\sim-\kappa m^2$.
This is the mechanism behind the eventual vanishing of sufficiently unstable
strata in the Teleman--Woodward finiteness theorem.

\subsection*{Why $\xi$--invariants control finiteness of the index}

Let $\Sigma$ be a smooth projective curve and $\mf M=\Bun_G(\Sigma)$.
For a class $\mathcal E\in K^*(\mf M)$ one defines its index by the Euler
characteristic
\[
\Ind(\mf M,\mathcal E)\;:=\;\chi(\mf M,\mathcal E)
\;=\;\sum_i(-1)^i\dim H^i(\mf M,\mathcal E),
\]
whenever the right-hand side is finite.

Since $\mf M$ is not of finite type, finiteness is proved by filtering
$\mf M$ by finite-type open substacks using the Shatz stratification
\[
\mf M=\bigsqcup_{\xi}\mf M_\xi,
\qquad
\mf M_{\le \xi}:=\bigcup_{\mu\le\xi}\mf M_\mu.
\]
The open substacks $\mf M_{\le\xi}$ form an increasing filtration of $\mf M$,
and for any complex $\mathcal E$ on $\mf M$ this induces a filtration of
$R\Gamma(\mf M,\mathcal E)$ by local cohomology with supports in the complements.
Equivalently, there is a spectral sequence whose $E_1$--page is built from the
local cohomology complexes
\[
E_1^{\xi,*}\;=\;R\Gamma_{\mf M_\xi}(\mf M_{\le\xi},\mathcal E_{\le\xi})
\quad\Longrightarrow\quad
R\Gamma(\mf M,\mathcal E).
\]
In particular, finiteness of $\Ind(\mf M,\mathcal E)$ follows once one knows:
\begin{enumerate}[(a)]
\item for each $\xi$, the contribution of $R\Gamma_{\mf M_\xi}(\mf M_{\le\xi},\mathcal E_{\le\xi})$
to Euler characteristic is finite-dimensional; and
\item all but finitely many $\xi$ contribute trivially.
\end{enumerate}

Teleman--Woodward identify each local term by a purity/local-duality statement:
\[
R\Gamma_{\mf M_\xi}(\mf M_{\le\xi},\mathcal E_{\le\xi})
\simeq
R\Gamma\bigl(\mf M_\xi,\mathcal R_\xi\mathcal E\bigr)[d_\xi],
\qquad
\mathcal R_\xi\mathcal E:= i_\xi^!(\mathcal E_{\le\xi})[-d_\xi],
\]
where $d_\xi=\codim(\mf M_\xi,\mf M_{\le\xi})$ and $i_\xi:\mf M_\xi\hookrightarrow\mf M_{\le\xi}$
is the inclusion.


Moreover, the residue object $\mathcal R_\xi\mathcal E$ may be expressed
formally in terms of the virtual normal complex
\[
\nu_\xi=R\pi_*\mathcal E^*(\mf g/\mf g_\xi)[1]
\]
as
\[
\mathcal R_\xi\mathcal E\;\sim\;
\mathcal E|_{\mf M_\xi}\otimes \Eul(\nu_\xi)^{-1}.
\]

\begin{remark}
Let $X$ be an algebraic stack, let $i:Z\hookrightarrow X$ be a closed immersion,
and let $\mathcal E$ be a (bounded) complex of coherent sheaves on $X$.
Recall that the \emph{local cohomology} of $\mathcal E$ with supports in $Z$ is
defined by the exact triangle
\[
R\Gamma_Z(X,\mathcal E)\;\longrightarrow\;R\Gamma(X,\mathcal E)
\;\longrightarrow\;R\Gamma(X\setminus Z,\mathcal E|_{X\setminus Z})\;\longrightarrow,
\]
or, equivalently, by the functor of sections with supports
$R\Gamma_Z(X,-)=R\Gamma(X,R\Gamma_Z(-))$.
Local duality packages these groups as ordinary cohomology on $Z$:
one defines the \emph{residue object} along $Z$ by
\[
\mathcal R_Z(\mathcal E)\;:=\; i^!(\mathcal E)[-\codim(Z)],
\]
so that (under standard hypotheses, e.g.\ $i$ a local complete intersection)
\[
R\Gamma_Z(X,\mathcal E)\;\simeq\;R\Gamma\bigl(Z,\mathcal R_Z(\mathcal E)\bigr)[\codim(Z)].
\]
In Teleman--Woodward's setting one takes $X=\mf M_{\le\xi}$ and $Z=\mf M_\xi$, so
\[
\mathcal R_\xi\mathcal E := i_\xi^!(\mathcal E_{\le\xi})[-d_\xi],
\qquad d_\xi=\codim(\mf M_\xi,\mf M_{\le\xi}).
\]

\smallskip
\noindent\textbf{Why an Euler factor appears.}
If $i:Z\hookrightarrow X$ is a \emph{regular embedding} between smooth schemes
with normal bundle $N_{Z/X}$, then $i^!$ is controlled by the normal directions.
At the level of $K$--theory one has the standard identity
\begin{equation}
\label{eq:regular-embed-euler}
\bigl[i^!(\mathcal E)\bigr]
=
\bigl[i^*(\mathcal E)\bigr]\cdot \lambda_{-1}(N_{Z/X}^\vee)^{-1}
\qquad\text{in }K^0(Z),
\end{equation}
where
\[
\lambda_{-1}(W):=\sum_{p\ge 0}(-1)^p[\wedge^p W]
\]
is the $K$--theoretic Euler class.
Heuristically, local cohomology measures ``principal parts along $Z$'',
and principal parts are obtained by expanding in the normal directions; the
inverse Euler class $\lambda_{-1}(N_{Z/X}^\vee)^{-1}$ is the $K$--theoretic avatar
of this expansion.

\smallskip
\noindent\textbf{Virtual normal complex.}
In the Shatz situation, $\mf M_\xi\hookrightarrow \mf M_{\le\xi}$ is not presented
globally as a single regular embedding into a smooth ambient space.  Instead,
Teleman--Woodward use the fact that $\mf M_\xi$ maps to a finite--type semistable
Levi stack $\mf M^{ss}_{G_\xi,\xi}$ with fibres quotient stacks of affine spaces
by unipotent groups, and there is a canonical \emph{virtual normal complex}
(perfect complex playing the role of $N_{Z/X}$)
\[
\nu_\xi \;=\; R\pi_*\mathcal E^*(\mf g/\mf g_\xi)[1]
\qquad\text{on }\mf M^{ss}_{G_\xi,\xi},
\]
whose pullback to $\mf M_\xi$ controls the unstable directions transverse to the
Levi moduli.  Consequently, the $K$--theory Euler class is defined by
\[
\Eul(\nu_\xi^\vee)\;:=\;\lambda_{-1}(\nu_\xi^\vee),
\]
and the same formal identity as \eqref{eq:regular-embed-euler} holds with
$N_{Z/X}$ replaced by $\nu_\xi$:
\begin{equation}
\label{eq:virtual-euler}
\bigl[\mathcal R_\xi\mathcal E\bigr]
=
\bigl[\mathcal E|_{\mf M_\xi}\bigr]\cdot \lambda_{-1}(\nu_\xi^\vee)^{-1}.
\end{equation}
\end{remark}



\subsubsection{The polarized inverse Euler class and the formula for $\Eul(\nu_\xi)^{-1}_+$}

We explain the origin of Teleman--Woodward's formula
\[
\Eul(\nu_\xi)^{-1}_+
:=
\Sym\Bigl(
R\pi_*\mathcal E^*(\mf p_\xi/\mf g_\xi)[1]^\vee
\ \oplus\
R\pi_*\mathcal E^*(\mf g/\mf p_\xi)[1]
\Bigr)\ \otimes\
\det\Bigl(R\pi_*\mathcal E^*(\mf g/\mf p_\xi)[1]\Bigr)[d_\xi],
\tag{$\ast$}
\]
It is a formal inverse to the Euler class with a weight decomposition satisfing the key properties that only weights $\le 0$ occur and each weight space is finite.

\medskip
\noindent\textbf{(1) $\xi$--weights and the parabolic splitting.}
Fix a maximal torus $T\subset G$ and a Borel $B\supset T$.
For a dominant rational coweight $\xi\in X_*(T)\otimes\Q$, let
$P_\xi$ be the associated \emph{standard} parabolic subgroup (so $B\subset P_\xi$),
and let $G_\xi$ be its Levi subgroup.  At the Lie algebra level one has a canonical
$\xi$--weight decomposition
\[
\mf g = \mf g_\xi \oplus \mf n_\xi \oplus \mf n_\xi^-,
\qquad
\mf n_\xi=\bigoplus_{\langle\alpha,\xi\rangle>0}\mf g_\alpha,
\quad
\mf n_\xi^-=\bigoplus_{\langle\alpha,\xi\rangle<0}\mf g_\alpha.
\]
Equivalently,
\[
\mf p_\xi=\mf g_\xi\oplus \mf n_\xi,
\qquad
\mf g/\mf g_\xi \cong (\mf p_\xi/\mf g_\xi)\oplus (\mf g/\mf p_\xi),
\]
where $\mf p_\xi/\mf g_\xi\cong \mf n_\xi$ carries strictly \emph{positive} $\xi$--weights
and $\mf g/\mf p_\xi\cong \mf n_\xi^-$ carries strictly \emph{negative} $\xi$--weights.

\medskip
\noindent\textbf{(2) The virtual normal complex and its $\xi$--grading.}
Let $\mf M_\xi$ be the Shatz stratum of instability type $\xi$ and let
$\mf M^{ss}_{G_\xi,\xi}$ be the semistable Levi moduli.

Sending a $P_\xi$--bundle to its associated Levi bundle defines a morphism
$q_\xi:\mf M_\xi\to \mf M^{ss}_{G_\xi,\xi}$.
The fibres are quotient stacks of affine spaces by unipotent groups.
The deformation theory transverse to the Levi directions is governed by the
perfect complex on $\mf M^{ss}_{G_\xi,\xi}$
\[
\nu_\xi:= R\pi_*\mathcal E^*(\mf g/\mf g_\xi)[1],
\]
whose pullback along $q_\xi$ controls the virtual normal directions along the stratum and 
where \[\pi:\Sigma\times \mf M^{ss}_{G_\xi,\xi}\to \mf M^{ss}_{G_\xi,\xi}\] and
$\mathcal E$ is the universal bundle.
Using the splitting above,
\[
\nu_\xi \simeq \nu_\xi^+ \oplus \nu_\xi^-,
\qquad
\nu_\xi^+ := R\pi_*\mathcal E^*(\mf p_\xi/\mf g_\xi)[1],
\quad
\nu_\xi^- := R\pi_*\mathcal E^*(\mf g/\mf p_\xi)[1].
\]
Thus $\nu_\xi$ carries a canonical $\Z$--grading by $\xi$--weights: $\nu_\xi^+$ has
strictly positive weights and $\nu_\xi^-$ has strictly negative weights.

\medskip
\noindent\textbf{(3) Why an ``inverse Euler class'' is not a genuine $K$--class.}
For a vector bundle $W$, the $K$--theoretic Euler class is
\[
\lambda_{-1}(W^\vee)=\sum_{p\ge 0}(-1)^p[\wedge^p W^\vee].
\]
Even for a line bundle $L$, the inverse of $1-L^\vee$ is \emph{not} a finite $K$--class:
\[
(1-L^\vee)^{-1} = \sum_{n\ge 0}(L^\vee)^n
\qquad\text{(a formal geometric series).}
\]
Teleman--Woodward therefore work in a \emph{completion} of equivariant $K$--theory
determined by the $\xi$--weights.  In such a completion one is allowed to expand
$1-L^\vee$ as a geometric series \emph{in whichever direction is convergent in the chosen
completion}.  This is the meaning of the phrase ``prefers $\xi$--negative eigenvalues.''

\medskip
\noindent\textbf{(4) The basic one--dimensional identity and the determinant correction.}
Let $\Gm$ act on a one--dimensional representation of weight $w\neq 0$, so the character is $t^w$.
Then
\[
(1-t^w)^{-1}
=
\sum_{n\ge 0} t^{nw}
\qquad\text{as a formal series in the direction of weights }w,w,2w,\dots.
\]
If we instead want an expansion which involves only \emph{nonpositive} weights (i.e.\ which
``prefers negative eigenvalues''), we rewrite
\[
(1-t^w)^{-1} = -t^{-w}\,(1-t^{-w})^{-1}
= -t^{-w}\sum_{n\ge 0} t^{-nw}.
\]
Compared to the naive geometric series, this introduces a prefactor $-t^{-w}$.
In higher rank, multiplying these prefactors over all negative--weight lines produces a
\emph{determinant factor}. 

\medskip
\noindent\textbf{(5) From weights to a polarized inverse for a split complex.}
Suppose a perfect complex $K$ carries a $\xi$--grading and splits as
\[
K \simeq K^+ \oplus K^-,
\]
where all $\xi$--weights in $K^+$ are $>0$ and all $\xi$--weights in $K^-$ are $<0$.
Then \[\lambda_{-1}(K^\vee)=\lambda_{-1}((K^+)^\vee)\cdot \lambda_{-1}((K^-)^\vee)\]
and one defines a \emph{polarized inverse} $\lambda_{-1}(K^\vee)^{-1}_+$ by inverting
each factor in the completion which expands in the direction of \emph{$\xi$--negative weights}.
The outcome is the standard schematic identity
\begin{equation}
\label{eq:polarized-euler}
\lambda_{-1}(K^\vee)^{-1}_+
\;=\;
\Sym\bigl((K^+)^\vee \oplus K^-\bigr)\ \otimes\ \det(K^-)\,[\mathrm{shift}],
\end{equation}
where:
\begin{itemize}
\item $\Sym(-)$ denotes the total symmetric algebra
$\Sym^\bullet(-)=\bigoplus_{n\ge 0}\Sym^n(-)$, interpreted in $K$--theory (or in the
corresponding completed $K$--group) as a formal power series;
\item $\det(K^-)$ is the determinant line of the perfect complex $K^-$, which precisely
packages the product of the one--dimensional prefactors in \textbf{(4)};
\item $[\mathrm{shift}]$ is the cohomological degree shift dictated by local duality/purity
(and in the Shatz situation becomes $[d_\xi]$).
\end{itemize}


\begin{remark}[Origin of the determinant factor in the polarized inverse]
\label{rem:det-factor}
Let $\Gm$ act with a $\Z$--grading, and let $W$ be a finite--rank $\Gm$--equivariant
vector bundle with \emph{strictly negative} weights.
Write the $K$--theoretic Euler class as
\[
\lambda_{-1}(W^\vee)=\prod_{i}(1-L_i^\vee),
\]
after splitting $W=\bigoplus_i L_i$ into $\Gm$--eigenlines (locally on the base).
Formally,
\[
(1-L_i^\vee)^{-1}=\sum_{n\ge 0}(L_i^\vee)^n
\]
is the geometric expansion in nonnegative powers of $L_i^\vee$.
However, if $L_i$ has \emph{negative} $\xi$--weight, then $L_i^\vee$ has \emph{positive}
weight, so this expansion lives in the completion which prefers \emph{positive} weights.
To invert in the opposite completion (the one preferring negative weights), we rewrite
\[
(1-L_i^\vee)^{-1}
= -\,L_i\cdot (1-L_i)^{-1}
= -\,L_i\sum_{n\ge 0} L_i^{n},
\]
which is now a series in nonnegative powers of $L_i$ (hence in nonpositive weights).
The price paid for using this expansion is the prefactor $(-L_i)$.

Multiplying over $i$ gives
\[
\lambda_{-1}(W^\vee)^{-1}\Big|_{\text{prefer negative}}
=
\Bigl(\prod_i (-L_i)\Bigr)\cdot \prod_i (1-L_i)^{-1}
=
(-1)^{\rank W}\,\det(W)\cdot \Sym(W),
\]
where $\Sym(W):=\bigoplus_{n\ge 0}\Sym^n(W)$.
Up to the harmless sign $(-1)^{\rank W}$ (often suppressed in $K$--theory conventions),
this explains the appearance of the factor $\det(W)$ in the polarized inverse.

For a perfect complex $K^-$ of strictly negative weights, the same argument applied
to any local splitting into graded line bundles (together with the multiplicativity
of $\lambda_{-1}$ in $K$--theory) produces the factor $\det(K^-)$ in the polarized inverse
$\lambda_{-1}(K^\vee)^{-1}_+$.
\end{remark}

\noindent\textbf{(6) Specialization to $\nu_\xi$.}
Apply \eqref{eq:polarized-euler} to $K=\nu_\xi$ and the splitting
$\nu_\xi\simeq \nu_\xi^+\oplus \nu_\xi^-$ from \textbf{(2)}.
Then $(K^+)^\vee=(\nu_\xi^+)^\vee$ and $K^-=\nu_\xi^-$, and we obtain
\[
\Eul(\nu_\xi)^{-1}_+
:=
\lambda_{-1}(\nu_\xi^\vee)^{-1}_+
=
\Sym\bigl((\nu_\xi^+)^\vee \oplus \nu_\xi^-\bigr)\ \otimes\ \det(\nu_\xi^-)\,[d_\xi].
\]
Unwinding the definitions of $\nu_\xi^\pm$ gives exactly the formula $(\ast)$ above.

\subsubsection*{Weight bookkeeping and the finiteness mechanism}

Fix an instability type $\xi$ and consider the local contribution supported on the Shatz stratum
$\mf M_\xi$.  Teleman--Woodward control this contribution by analysing the
$\xi$--weight decomposition of the \emph{polarized inverse Euler factor}
$\Eul(\nu_\xi)^{-1}_+$ and its tensor product with an admissible class
$\mathcal E$.

\medskip
\noindent\textbf{(A) The determinant weight and the quadratic form $c(\xi,\xi)$.}
Recall the polarized inverse Euler factor (cf.\ \cite[\S1.10--1.11]{TelemanWoodward})
\begin{equation}
\label{eq:polarized-eul}
\Eul(\nu_\xi)^{-1}_+
:=
\Sym\Bigl(
R\pi_*\mathcal E^*(\mf p_\xi/\mf g_\xi)[1]^\vee
\ \oplus\
R\pi_*\mathcal E^*(\mf g/\mf p_\xi)[1]
\Bigr)\ \otimes\
\det\Bigl(R\pi_*\mathcal E^*(\mf g/\mf p_\xi)[1]\Bigr)[d_\xi].
\tag{$\ast$}
\end{equation}
The second tensor factor is a determinant line bundle
\[
\det\Bigl(R\pi_*\mathcal E^*(\mf g/\mf p_\xi)[1]\Bigr).
\]
Because $\mf g/\mf p_\xi$ is a direct sum of root spaces $\mf g_\alpha$ with
$\langle\alpha,\xi\rangle<0$, the one--parameter subgroup determined by $\xi$
acts on $\mf g/\mf p_\xi$ with strictly \emph{negative} weights.  Consequently,
the induced $\Gm$--action on the determinant line above has a well-defined
$\xi$--weight which may be computed as a signed sum of these negative integers,
counted with the appropriate cohomological multiplicities coming from
$R\pi_*$.

Teleman--Woodward package this total determinant weight by a distinguished
invariant quadratic form
\[
c\in \Sym^2(\mf g^*)^G,
\]
namely the quadratic form attached (via Grothendieck--Riemann--Roch) to the
canonical bundle
\[
\mathcal K := \det(E_\Sigma^*\mf g)
\qquad\text{on }\mf M.
\]
With this notation, the determinant factor in \eqref{eq:polarized-eul} has
$\xi$--weight
\begin{equation}
\label{eq:weight-is-c}
\wt_\xi\Bigl(\det(R\pi_*\mathcal E^*(\mf g/\mf p_\xi)[1])\Bigr)=c(\xi,\xi).
\end{equation}
In Teleman--Woodward's conventions, $c(\xi,\xi)$ is \emph{negative} when $\xi$
is viewed in the compact real form $i\mf t_k$; equivalently, $c$ is negative
definite on $i\mf t_k$.


\begin{proof}[Justification of \eqref{eq:weight-is-c}]
Fix $\xi$ and let $\lambda_\xi:\Gm\to G$ be the canonical one--parameter subgroup.
Consider the determinant line
\[
\mathcal D_\xi = \det \Bigl(R\pi_*\mathcal E^*(\mf g/\mf p_\xi)[1]\Bigr)
\]
which appears in the polarized inverse Euler class $\Eul(\nu_\xi)^{-1}_+$.

As a $T$--module (and hence as a $\lambda_\xi$--module),
\[
\mf g/\mf p_\xi\;\cong\;\bigoplus_{\langle\alpha,\xi\rangle<0}\mf g_\alpha,
\]
a direct sum of root spaces on which $\lambda_\xi$ acts with weights
$\langle\alpha,\xi\rangle<0$.
The corresponding trace form is
\[
\Tr_{\mf g/\mf p_\xi}(\eta,\eta)
\;=\;\sum_{\langle\alpha,\xi\rangle<0}\langle\alpha,\eta\rangle^2
\qquad (\eta\in \mf t),
\]
because $\eta$ acts on $\mf g_\alpha$ by the scalar $\langle\alpha,\eta\rangle$.

Now compare with the adjoint trace form:
\[
\Tr_{\mf g}(\eta,\eta)\;=\;\sum_{\alpha\in \Phi}\langle\alpha,\eta\rangle^2,
\]
since $\eta$ acts trivially on $\mf t$ and by $\langle\alpha,\eta\rangle$ on $\mf g_\alpha$.
The root system is symmetric $\alpha\leftrightarrow -\alpha$, so the sum over
$\{\alpha:\langle\alpha,\xi\rangle<0\}$ is exactly half the sum over all roots:
\[
\Tr_{\mf g/\mf p_\xi}(\eta,\eta)\;=\;\frac12\,\Tr_{\mf g}(\eta,\eta).
\]
Therefore the level of $\mathcal D_\xi$ is
\[
\mathrm{lev}(\mathcal D_\xi)
\;=\;\Tr_{\mf g/\mf p_\xi}
\;=\;\frac12\,\Tr_{\mf g}.
\]

There is the classical indentification of the level of the Pfaffian square root $\mathcal K^{-1/2}$
\[
c\;:=\;-\frac12\,\Tr_{\mf g},
\]
where $\mathcal K=\det(E_\Sigma^*\mf g)$ is the canonical bundle on $\mf M$.
Combining with the computation above gives
\[
\mathrm{lev}(\mathcal D_\xi) = -c.
\]
However the shift $[1]$ in the determinant line $\mathcal D_\xi$: \begin{align*}
    \det \Bigl(R\pi_*\mathcal E^*(\mf g/\mf p_\xi)[1]\Bigr) = \det \Bigl(R\pi_*\mathcal E^*(\mf g/\mf p_\xi)\Bigr)^{-1}
\end{align*} leaves us with $c$ as desired.
\end{proof}

\medskip
\noindent\textbf{(B) Tensoring by an admissible class $\mathcal E$.}
Lemma \cite[\S1.11]{TelemanWoodward} concerns the $\xi$--invariant part of
\[
\mathcal E\otimes \Eul(\nu_\xi)^{-1}_+,
\qquad
\bigl(\mathcal E\otimes \Eul(\nu_\xi)^{-1}_+\bigr)^{\xi\text{-inv}},
\]
i.e.\ the weight--$0$ subobject for the $\Gm$--action defined by $\xi$.

Write $\mathcal E$ as a product
\[
\mathcal E \;=\; \mathcal L\ \otimes\ (\text{Atiyah--Bott generators}),
\]
where $\mathcal L$ is a determinant line bundle and the remaining factors are
built from the Atiyah--Bott generators $E_x^*V$, $E_C^*V$, $E_\Sigma^*V$.

\medskip
\noindent\textbf{(B1) Quadratic shift from $\mathcal L$.}
By Grothendieck--Riemann--Roch, the first Chern class of $\mathcal L$ determines
an invariant quadratic form
\[
h=h_{\mathcal L}\in \Sym^2(\mf g^*)^G,
\]
called the \emph{level} of $\mathcal L$.  Teleman--Woodward's GRR calculation
shows that, on the $\xi$--stratum, the $\xi$--weight contributed by $\mathcal L$
has leading behaviour controlled by this level:
\begin{equation}
\label{eq:L-weight}
\wt_\xi(\mathcal L)\sim h(\xi,\xi),
\qquad\text{quadratic in }\xi.
\end{equation}
Combining \eqref{eq:L-weight} with the determinant contribution
\eqref{eq:weight-is-c} coming from $\Eul(\nu_\xi)^{-1}_+$, the \emph{net} quadratic
behaviour is governed by
\begin{equation}
\label{eq:h-plus-c}
(h+c)(\xi,\xi).
\end{equation}

Recall that $\mathcal L$ is \emph{admissible} precisely when $h+c$ is positive
definite on $\mf g$, equivalently when
\[
(h+c)(\xi,\xi)\to +\infty\qquad\text{as }\|\xi\|\to\infty.
\]

\medskip
\noindent\textbf{(B2) Linear perturbation from Atiyah--Bott factors.}
The remaining factors in $\mathcal E$ are Atiyah--Bott generators attached to
representations $V$ of $G$.  Their $\xi$--weights are governed by ordinary
representation theory: if $\lambda$ is a weight of $V$, then $\xi$ acts with
weight $\langle\lambda,\xi\rangle$.  In particular, these contributions are at
most \emph{linear} in $\xi$:
\begin{equation}
\label{eq:AB-linear}
\wt_\xi(\text{Atiyah--Bott factors}) = O(\|\xi\|).
\end{equation}

\medskip
\noindent\textbf{(C) Finite-dimensionality of the $\xi$--invariant part.}
The polarized inverse Euler factor $\Eul(\nu_\xi)^{-1}_+$ has a $\xi$--weight
decomposition with two crucial properties:
\begin{enumerate}[(i)]
\item only weights $\le 0$ occur; and
\item each weight space has finite multiplicity.
\end{enumerate}
These follow from the fact that in \eqref{eq:polarized-eul} the symmetric algebra
is generated by strictly negative $\xi$--weight summands.

Fix $\xi$.  The weight--$0$ piece of
$\mathcal E\otimes \Eul(\nu_\xi)^{-1}_+$
is obtained by summing those weight spaces of $\Eul(\nu_\xi)^{-1}_+$ whose weights
cancel the (finite) set of weights appearing in $\mathcal E$.  Since each weight
space of $\Eul(\nu_\xi)^{-1}_+$ has finite multiplicity, it follows that
\begin{equation}
\label{eq:finite-invariants}
\bigl(\mathcal E\otimes \Eul(\nu_\xi)^{-1}_+\bigr)^{\xi\text{-inv}}
\ \text{is finite-dimensional.}
\end{equation}
This is the first conclusion of \cite[\S1.11]{TelemanWoodward}.

\medskip
\noindent\textbf{(D) Eventual vanishing for $\|\xi\|\gg 0$.}
Now let $\xi$ vary in the dominant cone.  The symmetric algebra part of
$\Eul(\nu_\xi)^{-1}_+$ produces weights by taking symmetric powers of
negative--weight generators.  The possible weights contributed in this way move
away from $0$ in steps controlled by the individual $\xi$--weights of the
generators; these steps scale \emph{linearly} in $\xi$ (because root weights
$\langle\alpha,\xi\rangle$ are linear in $\xi$).

On the other hand, twisting by an admissible line bundle $\mathcal L$ produces
the quadratic shift \eqref{eq:h-plus-c}.  Combining with the linear perturbation
\eqref{eq:AB-linear} from Atiyah--Bott generators, the net effect is that the set
of $\xi$--weights appearing in
$\mathcal E\otimes \Eul(\nu_\xi)^{-1}_+$
is translated by a term which grows like $(h+c)(\xi,\xi)$, up to linear error.
Since $(h+c)(\xi,\xi)$ grows quadratically while all available ``corrections''
coming from symmetric powers grow at most linearly, it follows that for $\|\xi\|$
sufficiently large the total $\xi$--weight $0$ cannot occur.  Equivalently, there
exists $B>0$ such that
\begin{equation}
\label{eq:eventual-vanishing}
\|\xi\|>B \quad\Longrightarrow\quad
\bigl(\mathcal E\otimes \Eul(\nu_\xi)^{-1}_+\bigr)^{\xi\text{-inv}}=0.
\end{equation}
This is the second conclusion of \cite[\S1.11]{TelemanWoodward}.

\medskip
\noindent\textbf{(E) Consequence for finiteness of the index.}
The local cohomology filtration of $R\Gamma(\mf M,\mathcal E)$ by Shatz supports
has graded pieces controlled by the strata $\mf M_\xi$.
Identifying the residue contribution along $\mf M_\xi$ with the $\xi$--invariant
part of $\mathcal E\otimes \Eul(\nu_\xi)^{-1}_+$, the finiteness statement
\eqref{eq:finite-invariants} gives finite-dimensionality of each stratum
contribution, while the vanishing \eqref{eq:eventual-vanishing} shows that only
finitely many $\xi$ contribute.  Therefore the local-to-global spectral sequence
has only finitely many nonzero columns, and the index $\Ind(\mf M,\mathcal E)$ is
finite.



\section{References}
\begin{enumerate}
    \bibitem{TelemanWoodward} Teleman, C., \& Woodward, C. (2012). The index formula on the moduli of G-bundles. Annals of Mathematics, 176(2), 601-77.
\end{enumerate}
\end{document}