\documentclass[12pt]{article}  % or any other class
\usepackage{/Users/songye03/Desktop/Math_tex/style/psetconfig}         % loads your custom style
\title{Homework 1}
\author{Songyu Ye}
\date{\today}

\begin{document}
\psettitle

\begin{problem}{1}
Check that localization preserves products and colons. Explain how this implies that the following proposition from the textbook. 
\begin{proposition}[S, Prop 1.3.5]
    Let $A$ be a Dedekind domain and $I$ a non-zero fractional ideal. Then $I$ is invertible.       
\end{proposition}
\end{problem}

\begin{solution}

\begin{enumerate}[(i)]
  \item $(a \cdot b)_{\mathfrak{p}} = a_{\mathfrak{p}} \cdot b_{\mathfrak{p}}$.

  \begin{proof}
  Every element of $a_{\mathfrak{p}} \cdot b_{\mathfrak{p}}$ is a finite sum of products
  \[
    \frac{a}{s} \cdot \frac{b}{t} = \frac{ab}{st} \qquad (a \in a,\, b \in b,\, s,t \in S),
  \]
  hence lies in $S^{-1}(ab) = (ab)_{\mathfrak{p}}$.

  Conversely, $(ab)_{\mathfrak{p}}$ is generated (as an ideal of $A_{\mathfrak{p}}$) 
  by the fractions $\tfrac{ab}{s}$ with $a \in a$, $b \in b$, $s \in S$, and
  \[
    \frac{ab}{s} = \frac{a}{1} \cdot \frac{b}{s} \in a_{\mathfrak{p}} \cdot b_{\mathfrak{p}}.
  \]
  So the two ideals are equal.
  \end{proof}

  \item $(a : b)_{\mathfrak{p}} = (a_{\mathfrak{p}} : b_{\mathfrak{p}})$.

  \begin{proof}
  Recall
  \[
    (a : b) = \{x \in A \mid x b \subset a \}, 
    \qquad
    (a_{\mathfrak{p}} : b_{\mathfrak{p}}) = \{ z \in A_{\mathfrak{p}} \mid z b_{\mathfrak{p}} \subset a_{\mathfrak{p}} \}.
  \]

  ($\subseteq$): If $\tfrac{x}{s} \in S^{-1}(a : b)$, then $x b \subset a$, so for any $\tfrac{b}{t} \in b_{\mathfrak{p}}$ we have
  \[
    \frac{x}{s} \cdot \frac{b}{t} = \frac{xb}{st} \in S^{-1} a = a_{\mathfrak{p}}.
  \]
  Hence $\tfrac{x}{s} \in (a_{\mathfrak{p}} : b_{\mathfrak{p}})$.

  ($\supseteq$): Assume $\tfrac{x}{s} \in (a_{\mathfrak{p}} : b_{\mathfrak{p}})$.  
  Choose generators $b = (b_1, \dots, b_n)$. This is where we use that $b$ is finitely generated. For each $i$, $\tfrac{x}{s} \cdot \tfrac{b_i}{1} \in a_{\mathfrak{p}}$, so there exists $s_i \in S$ with $s_i x b_i \in a$. Let $t = \prod_i s_i \in S$. Then $t x b_i \in a$ for all $i$, hence $t x b \subset a$, i.e.\ $t x \in (a : b)$. Therefore
  \[
    \frac{x}{s} = \frac{tx}{ts} \in S^{-1}(a : b) = (a : b)_{\mathfrak{p}}.
  \]
  \end{proof}
\end{enumerate}
Now let $I$ be a nonzero fractional ideal in the Dedekind domain $A$. We want to show $I$ is invertible, i.e.\ there exists a fractional ideal $J$ such that $IJ = A$.
Pick a nonzero prime $\mathfrak{p} \subset A$. Localize at $\mathfrak{p}$: in the DVR $A_{\mathfrak{p}}$, every nonzero fractional ideal is of the form 
  \[
    I_{\mathfrak{p}} = \mathfrak{p}_{\mathfrak{p}}^{n} \quad (n \in \mathbb{Z}).
  \] Its inverse in $A_{\mathfrak{p}}$ is then 
  \[
    J_{\mathfrak{p}} = \mathfrak{p}_{\mathfrak{p}}^{-n},
  \]
  so that $I_{\mathfrak{p}} J_{\mathfrak{p}} = A_{\mathfrak{p}}$. Thus,locally at every prime, $I$ has an inverse.
Define globally
\[
  J := \prod_{\mathfrak{p}} \mathfrak{p}^{-n_{\mathfrak{p}}},
\]
where $I_{\mathfrak{p}} = \mathfrak{p}_{\mathfrak{p}}^{\,n_{\mathfrak{p}}}$.  
Only finitely many exponents $n_{\mathfrak{p}}$ are nonzero, so this makes sense.

Now for each prime $\mathfrak{q}$,
  \[
    (IJ)_{\mathfrak{q}} = I_{\mathfrak{q}} J_{\mathfrak{q}} = A_{\mathfrak{q}}.
  \]
Since the localizations are the unit ideal everywhere, we get globally
  \[
    IJ = A.
  \]
This is because if $IJ$ were proper, it would be contained in some maximal ideal $\mathfrak{m}$, hence $(IJ)_{\mathfrak{m}} \subseteq \mathfrak{m}_{\mathfrak{m}}$, contradicting the above. Thus $I$ is invertible with inverse $J$.
\end{solution}

\begin{problem}{2}
    Let $A$ be a Dedekind domain, $S$ a multiplicatively closed subset which is strictly smaller than the set of all nonzero elements.
\begin{enumerate}
    \item Show that the localization $S^{-1}A$ is a Dedekind domain.
    \item Show that the extension of ideals (and similarly for fractional ideals) induces a surjection from the ideal class group of $A$ to that of $S^{-1}A$.
\end{enumerate}
\end{problem}

\begin{solution}
    \begin{enumerate}
        \item Recall that a Noetherian domain $R$ is Dedekind iff for every nonzero prime $\mathfrak p\subset R$, the localization $R_{\mathfrak p}$ is a DVR.

Let $\mathfrak P$ be a nonzero prime of $S^{-1}A$. Then $\mathfrak P = S^{-1}\mathfrak p$ for a unique prime $\mathfrak p\subset A$ with $\mathfrak p\cap S=\varnothing$. Moreover,
$(S^{-1}A)_{\mathfrak P}\;\cong\;A_{\mathfrak p}$. Since $A$ is Dedekind, every $A_{\mathfrak p}$ (for $\mathfrak p\neq (0)$) is a DVR; hence every $(S^{-1}A)_{\mathfrak P}$ is a DVR. Therefore $S^{-1}A$ is Dedekind.

We also check that $S^{-1}A$ is Noetherian and an integral domain.  Let $I$ be an ideal of $S^{-1}A$. Consider its contraction in $A$: $I^c := \{a\in A \mid \tfrac{a}{1}\in I\}$. This is an ideal of $A$. Since $A$ is Noetherian, $I^c$ is finitely generated, say by $a_1,\dots,a_n$. Then $I = S^{-1}I^c = \left(\frac{a_1}{1},\dots,\frac{a_n}{1}\right)$, so $I$ is finitely generated in $S^{-1}A$. Thus every ideal of $S^{-1}A$ is finitely generated, i.e. $S^{-1}A$ is Noetherian.

Finally, $S^{-1}A$ is an integral domain because if $\tfrac{a}{s}\cdot \tfrac{b}{t} = 0$ in $S^{-1}A$, then there exists $u\in S$ with $uab = 0$ in $A$. Since $A$ is an integral domain and $u\neq 0$, we must have $a=0$ or $b=0$, hence $\tfrac{a}{s}=0$ or $\tfrac{b}{t}=0$ in $S^{-1}A$.

        \item Recall unique factorization of fractional ideals in Dedekind domains. In $A$, every nonzero fractional ideal has a unique factorization
\begin{align*}
    I=\prod_{\mathfrak p} \mathfrak p^{\,n_{\mathfrak p}}\qquad(n_{\mathfrak p}\in\mathbb Z,\ \text{all but finitely many }0).
\end{align*}	
In $S^{-1}A$: the nonzero prime ideals are exactly $S^{-1}\mathfrak p$ with $\mathfrak p\cap S=\varnothing$ (because if $\mathfrak p\cap S\neq\varnothing$, then $S^{-1}\mathfrak p$ contains a unit, hence equals $S^{-1}A$). So
\begin{align*}
    S^{-1}\!\left(\mathfrak p^{\,n}\right)=
\begin{cases}
(S^{-1}\mathfrak p)^{\,n}, & \mathfrak p\cap S=\varnothing,\\[2pt]
S^{-1}A, & \mathfrak p\cap S\neq\varnothing.
\end{cases}
\end{align*}
Hence for any fractional ideal $I=\prod_{\mathfrak p} \mathfrak p^{\,n_{\mathfrak p}}$,
$S^{-1}I=\prod_{\mathfrak p\cap S=\varnothing}(S^{-1}\mathfrak p)^{\,n_{\mathfrak p}}$.

Now take an arbitrary class $[J]\in\mathrm{Cl}(S^{-1}A)$. Factor $J$ in $S^{-1}A$: $J=\prod_{\mathfrak p\cap S=\varnothing}(S^{-1}\mathfrak p)^{\,m_{\mathfrak p}}$, with only finitely many nonzero $m_{\mathfrak p}\in\mathbb Z$. Define the fractional ideal of $A$ by $$I:=\prod_{\mathfrak p\cap S=\varnothing}\mathfrak p^{\,m_{\mathfrak p}}$$ Then by the argument above, $S^{-1}I=J$. Consequently $[J]=\Phi([I])$, proving that $\Phi$ is onto.
    \end{enumerate}
\end{solution}

\begin{problem}{3}
Let $A$ be a Dedekind domain. Consider a nonzero ideal $a$ in A. Show that every ideal in the quotient ring $A/a$ is principal. Deduce that every ideal of $A$ is generated by at most two elements.
\end{problem}

\begin{solution}
    Let $A$ be Dedekind and $0\neq\mathfrak a\lhd A$. Factor \begin{align*}
         \mathfrak a=\prod_{i=1}^r \mathfrak p_i^{e_i}\qquad(\mathfrak p_i \text{ distinct}).
    \end{align*}
By the Chinese remainder theorem, $A/\mathfrak a \;\cong\; \prod_{i=1}^r A/\mathfrak p_i^{e_i}$. So it suffices to show every ideal of $A/\mathfrak p^{e}$ is principal. This is every ideal in a product of two rings is a product of ideals in each ring.

For a fixed prime $\mathfrak p$, the natural map $A\to A_{\mathfrak p}$ induces an isomorphism $A/\mathfrak p^{e}\;\xrightarrow{\ \sim\ }\;A_{\mathfrak p}/\mathfrak p^{e}A_{\mathfrak p}$. This is because in general we have an isomorphism $A/B \cong (A/C) / (B/C)$ for ideals $C\subset B\subset A$.

But $A_{\mathfrak p}$ is a DVR (Dedekind $\Rightarrow$ localizations at nonzero primes are DVRs). In a DVR with uniformizer $\pi$, all ideals are powers of the maximal ideal, hence in the quotient $A_{\mathfrak p}/\mathfrak p^{e}A_{\mathfrak p}$ the ideals are exactly $\frac{\mathfrak p^{k}A_{\mathfrak p}}{\mathfrak p^{e}A_{\mathfrak p}}\quad(0\le k\le e)$, and each is generated by the class of $\pi^{k}$. Thus every ideal of $A_{\mathfrak p}/\mathfrak p^{e}A_{\mathfrak p}$, hence of $A/\mathfrak p^{e}$, is principal. 

Now let $I$ be an ideal of $A$. Pick some nonzero $a$ not in $I$, and consider the image of $I$ in $A/(a)$. It is principal, generated by some $\overline{b}\in A/a$. Pick a lift $b \in A$ of $\overline{b}$. Then $I$ contains the ideal $(a,b)$, and the image of $I$ in $A/(a,b)$ is zero. So $I=(a,b)$ is generated by two elements.
\end{solution}

\begin{problem}{4}
    Let $d$ be a square free integer (which is either positive or negative). Determine the integral closure of $\mathbb{Z}[\sqrt{d}]$ (a.k.a. the ring of integers) in $\mathbb{Q}(\sqrt{d})$. (For example, give an explicit $\mathbb{Z}$-basis for the integral closure.)
\end{problem}
\begin{solution}
We claim that if $K = \mathbb{Q}(\sqrt{d})$ with $d$ square-free, then the ring of integers is
\[
\mathcal{O}_K =
\begin{cases}
\mathbb{Z}[\sqrt{d}], & d \equiv 2,3 \pmod{4}, \\[6pt]
\mathbb{Z}\!\left[\frac{1+\sqrt{d}}{2}\right], & d \equiv 1 \pmod{4}.
\end{cases}
\]

Equivalently, an explicit $\mathbb{Z}$-basis is
\[
\{1, \sqrt{d}\} \quad \text{if } d \equiv 2,3 \pmod{4}, 
\qquad
\left\{1, \tfrac{1+\sqrt{d}}{2}\right\} \quad \text{if } d \equiv 1 \pmod{4}.
\]

Let $\alpha = a + b\sqrt{d}$ with $a,b \in \mathbb{Q}$. The minimal polynomial over $\mathbb{Q}$ is
\[
x^2 - 2ax + (a^2 - b^2 d),
\]
so $\alpha$ is an algebraic integer iff the coefficients are integers:
\[
2a \in \mathbb{Z}, 
\qquad 
a^2 - b^2 d \in \mathbb{Z}. 
\tag{$\ast$}
\]

Write $b=n/r$ in lowest terms with $n\in\mathbb Z, r>0$ and $m=2a$. Then
\[
a^2 - b^2 d = \frac{m^2}{4} - \frac{n^2 d}{r^2} \in \mathbb Z.
\]
Multiply by $4r^2$:
\[
m^2 r^2 - 4n^2 d \in 4r^2 \mathbb Z. 
\]
The left hand side is a multiple of $r^2$, and $d$ is squarefree, $n$ and $r$ are coprime, so $r^2 \mid 4$, so $r = 1$ or $2$. 

Suppose $r=1$. Then $b=n\in \mathbb Z, a=m/2$. If $m$ even, then $a\in\mathbb Z$. Condition (*) gives nothing new, so $\alpha\in \mathbb Z[\sqrt d]$. If $m$ odd, then $a^2=\tfrac{m^2}{4}\equiv \tfrac14\pmod 1$. For (*) to hold, we would need $b^2 d \equiv \tfrac14\pmod 1$, which only happens when $d\equiv 1\pmod 4$ (this overlaps with Case B below). So if $d\equiv 2,3\pmod 4$, the only integrals are with $a,b\in\mathbb Z$, i.e. $\cO_K \subset \mathbb Z[\sqrt d]$. The reverse inclusion is clear, so $\cO_K = \mathbb Z[\sqrt d]$.


Suppose $r=2$. Now $b=n/2$. Then
\[
\alpha = \frac{m}{2} + \frac{n}{2}\sqrt d
= \frac{m-n}{2} + n\cdot\frac{1+\sqrt d}{2}.
\]
So $\alpha\in \mathbb Z\left[\tfrac{1+\sqrt d}{2}\right]$ if $\tfrac{m-n}{2}\in\mathbb Z$.

If $n$ is even, then $n^2 d \equiv 0 \pmod 4$. Condition (*) then requires $m^2 \equiv 0 \pmod 4$, so $m$ is even. Suppose both $m,n$ even. Then $\alpha \in \mathbb Z[\sqrt d]$ (trivial case). If $n$ is odd, then $n^2 d \equiv d \pmod 4$. So condition (*) becomes $m^2 \equiv d \pmod 4$. Now $m^2$ is either $0$ or $1 \pmod 4$. If $d\equiv 1\pmod 4$, then we need $m^2\equiv 1$, so $m$ must be odd. Thus $m$ and $n$ have the same parity. If $d\equiv 2,3\pmod 4$, there is no solution (since $d\equiv 2,3$ cannot be a quadratic residue mod 4).


This implies that $m$ and $n$ have the same parity when $d\equiv 1\pmod 4$. If $d\equiv 1\pmod 4$, then indeed $\frac{1+\sqrt d}{2}$ is integral (it satisfies $x^2-x+\frac{1-d}{4}=0$ with integer constant term). Hence we get a larger order:
\[
\mathcal O_K = \mathbb Z\left[\tfrac{1+\sqrt d}{2}\right].
\]
If $d\equiv 2,3\pmod 4$, then the congruence condition forces $m,n$ both even, so the element reduces back to one in $\mathbb Z[\sqrt d]$. No genuinely new elements appear.

\end{solution}
\end{document}