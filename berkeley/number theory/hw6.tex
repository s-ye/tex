\documentclass[12pt]{article}  % or any other class
\usepackage{/Users/songye03/Desktop/Math_tex/style/psetconfig}         % loads your custom style
\title{Homework 6}
\author{Songyu Ye}
\date{\today}



\begin{document}
\psettitle



\begin{problem}[1]
    Prove Krasner's Lemma ([S, p.~30, II.2 Exercise 1]) but only assuming that $K$ is a non-Archimedean CVF.

Let $E/K$ be a finite Galois extension of a complete field $K$> Prolong the valuation of $K$ to $E$. Let $x\in E$ and let $\set{x=x_1,x_2,\ldots,x_n}$ be the Galois conjugates of $x$ over $K$, with $x=x_1$. Let $y\in E$ so that $|y-x|<|y-x_i|$ for $i\geq 2$. Show that $x$ belongs to the field $K(y)$. Note that if $x_i$ is conjugate to $x$ over $K(y)$, then $|y-x| = |y-x_i|$.

    \textit{Note:} We need not assume that the valuation is discrete since the unique extension of valuations (as covered in class; see \cite[N, II.4.8]) works without requiring discreteness.
\end{problem}


\begin{solution}
Let $E/K$ be a finite Galois extension of a complete non-Archimedean valued field $K$.  Prolong the valuation of $K$ to $E$.  Let $x \in E$ have Galois conjugates $\{x_1=x, x_2, \dots, x_n\}$ over $K$.  Suppose that $y \in E$ satisfies
\[
|y - x| < |y - x_i| \quad \text{for all } i \ge 2.
\]
We will show that $x \in K(y)$.

Let $f(T) = \prod_{i=1}^n (T - x_i) \in K[T]$ be the minimal polynomial of $x$ over $K$.  Then
\[
f(y) = \prod_{i=1}^n (y - x_i)
      = (y - x) \cdot \prod_{i \ge 2} (y - x_i).
\]
For each $i \ge 2$, since $|y - x| < |y - x_i| = |\alpha_i - \alpha|$, we have $|y - x_i| = |x_i - x|$ by the ultrametric inequality.  Therefore
\[
|f(y)| = |y - x| \cdot \prod_{i \ge 2} |x_i - x|
        = |y - x| \, |f'(x)|.
\]
Because $|y - x| < |x_i - x|$ for all $i \ge 2$, it follows that
\[
|f(y)| < |f'(x)|^2.
\]

\begin{lemma}[(Hensel's Lemma)]
Let $A$ be a complete non-Archimedean valuation ring (for instance, a complete DVR), and let $f \in A[x]$. 
Suppose $a_0 \in A$ satisfies
\[
|f(a_0)| < |f'(a_0)|^2.
\]
Then the sequence defined by Newton iteration
\[
a_{n+1} := a_n - \frac{f(a_n)}{f'(a_n)} \qquad (n \ge 0)
\]
is well-defined and converges to a unique root $a \in A$ of $f$, satisfying
\[
|a - a_0| \le \frac{|f(a_0)|}{|f'(a_0)|^2}.
\]
Moreover, this root is unique within the ball $\{z \in A : |z - a_0| < |f'(a_0)|\}$.
\end{lemma}


By Hensel's lemma (which does not require the valuation to be discrete), there exists a unique root $\tilde{x}$ of $f$ such that
\[
|\tilde{x} - y| \le \frac{|f(y)|}{|f'(x)|} < |f'(x)|.
\]
But the only conjugate of $x$ that lies within this neighborhood of $y$ is $x$ itself, so $\tilde{x} = x$.  Hence $x$ is obtained from $y$ by solving $f(T) = 0$ within $K(y)$, showing that $x \in K(y)$.
\end{solution}

\begin{problem}[2]
\begin{enumerate}
    \item Do [S, p.~30, Exercise 2 in Section II.2] but only assuming that $K$ is a non-Archimedean CVF (not necessarily discrete). Let $K$ be a complete field, and let $f(X) \in K[X]$ be a separable irreducible polynomial of degree $n$. Let $L/K$ be the extension of degree $n$ defined by $f$. Show that for every polynomial $h(X)$ of degree $n$ that is close enough to $f(X)$, $h(X)$ is irreducible and the extension $L_h/K$ defined by $h$ is isomorphic to $L/K$.

    \begin{itemize}
        \item Two polynomials 
        \[
            f(x) = \sum_{i=0}^n a_i x^i, 
            \qquad 
            g(x) = \sum_{i=0}^n b_i x^i
        \]
        are considered \emph{close} if 
        \[
            \sup_{0 \le i \le n} |a_i - b_i|
        \]
        is sufficiently small (i.e.\ less than some $\varepsilon > 0$ depending on the initial data of the problem).
    \end{itemize}

    \item Note that the $p$-adic valuation on $\Q_p$ extends uniquely to a valuation on $\overline{\Q}_p$.
    (We still refer to the latter as the $p$-adic valuation.)
    Let $C$ denote the completion of $\overline{\Q}_p$ with respect to the $p$-adic valuation.
    Use (i) to prove that $C$ is algebraically closed.
    (People often write $\C_p$ for this $C$.)
\end{enumerate}
\end{problem}

\begin{solution}
    \begin{enumerate}
        \item Let $f\in K[X]$ be separable irreducible of degree $n$ and let $L=K(\alpha)$ with $f(\alpha)=0$. Write the distinct $K$-embeddings of $L$ into a fixed algebraic closure as $\sigma_1=\mathrm{id},\sigma_2,\dots,\sigma_n$, and set $\alpha_i:=\sigma_i(\alpha)$. Since $f$ is separable, $f'(\alpha)\neq 0$ and the finite set $\{\alpha_i\}_{i=1}^n$ has a positive mutual separation
\[
\delta \;:=\; \min_{i\ge 2}|\alpha-\alpha_i| \;>\;0.
\]
Let $h(X)=\sum_{i=0}^n b_iX^i$ be a polynomial of degree $n$ with coefficients sufficiently close to those of $f(X)=\sum_{i=0}^n a_iX^i$ in the sense that $\sup_i|a_i-b_i|<\varepsilon$ for $\varepsilon$ to be chosen below.

By continuity of evaluation, if $\varepsilon$ is small then
\[
|h(\alpha)|=\Big|\sum_{i=0}^n (b_i-a_i)\alpha^i\Big| \quad\text{is arbitrarily small,}
\qquad\text{and}\qquad
|h'(\alpha)-f'(\alpha)| \text{ is small,}
\]
hence $|h'(\alpha)|=|f'(\alpha)|\neq 0$ for $\varepsilon$ small enough. Choose $\varepsilon$ so that
\[
|h(\alpha)| \;<\; |h'(\alpha)|^2
\qquad\text{and}\qquad
\frac{|h(\alpha)|}{|h'(\alpha)|} \;<\; \delta .
\]
Applying Hensel's lemma (Newton form) in the complete non-Archimedean field $K$ to the pair $(h,a_0=\alpha)$, we obtain a unique root $\beta$ of $h$ with
\[
|\beta-\alpha| \;\le\; \frac{|h(\alpha)|}{|h'(\alpha)|} \;<\; \delta.
\]
Therefore $|\beta-\alpha|<|\alpha-\alpha_i|$ for all $i\ge 2$. By Krasner's lemma, we conclude $K(\alpha)\subseteq K(\beta)$.
But $[K(\beta):K]\le \deg h=n=[K(\alpha):K]$, so necessarily $[K(\beta):K]=n$ and $K(\beta)=K(\alpha)$. In particular $h$ is irreducible over $K$ and the extension $L_h:=K(\beta)$ is $K$-isomorphic to $L$.

\item Let $K=\Q_p$, let $\overline{\Q}_p$ be its algebraic closure endowed with the unique extension of the $p$-adic valuation, and let $C$ be the completion of $\overline{\Q}_p$ (often denoted $\C_p$). We prove $C$ is algebraically closed.

Take any nonconstant $h\in C[X]$ of degree $n$. Approximate its coefficients by elements of $\overline{\Q}_p$ to obtain $f\in \overline{\Q}_p[X]$ of the same degree $n$ with coefficients sufficiently close so that the inequalities used in (i) hold for each simple root of $f$. Since characteristic is $0$, we may (and do) choose $f$ \emph{separable} (discriminant nonzero is an open condition on the coefficients). Fix a root $\alpha\in \overline{\Q}_p\subset C$ of $f$. By the same Hensel argument as in (i), there exists $\beta\in C$ with $h(\beta)=0$ and $|\beta-\alpha|$ arbitrarily small. Thus $h$ has at least one root in $C$. Dividing $h$ by $(X-\beta)$ and repeating by induction on the degree, we factor $h$ completely over $C$. Hence $C$ is algebraically closed.
\end{enumerate}
\end{solution}
\begin{problem}[3]
    Fix an integer $n \ge 2$ and an algebraic closure $\overline{\Q}_p$ of the field $\Q_p$ of $p$-adic numbers.
    Let $L_n$ be a degree $n$ extension of $\Q_p$ in $\overline{\Q}_p$ such that $(p) \subset \Z_p$ is unramified in $L_n$.
    Write $\mu(L_n)$ for the (multiplicative) torsion subgroup of $L_n^\times$, namely the group of all roots of unity in $L_n$,
    and $\mu_N$ for the subgroup of $N$-th roots of unity in $\overline{\Q}_p^\times$.
    \begin{enumerate}[(1)]
        \item Show that 
        \[
            \mu(L_n) = 
            \begin{cases}
                \mu_{p^n-1} & \text{if $p$ is odd,} \\[4pt]
                \mu_{2(p^n-1)} & \text{if $p$ is even (namely if $p=2$).}
            \end{cases}
        \]
        \emph{Hint:} Hensel's lemma can help to show $\supseteq$.

        \item Prove that 
        \[
            L_n = \Q_p(\mu_{p^n-1}).
        \]
    \end{enumerate}

    \noindent
    \textit{Note:} This implies that there exists a \emph{unique} degree $n$ unramified extension of $\Q_p$ in $\overline{\Q}_p$.
    It also follows that such an extension is Galois over $\Q_p$.
\end{problem}

\begin{solution}
\begin{enumerate}
    \item 
Let $\mathcal{O}_{L_n}$ denote the valuation ring of $L_n$, with maximal ideal $\mathfrak{p}_{L_n}$ and residue field $k_{L_n}=\mathcal{O}_{L_n}/\mathfrak{p}_{L_n}$. Since $L_n/\Q_p$ is unramified of degree $n$, we have $k_{L_n}\cong\mathbb{F}_{p^n}$ and $\mathfrak{p}_{L_n}=p\mathcal{O}_{L_n}$. The reduction map
\[
\mathcal{O}_{L_n}^\times \;\twoheadrightarrow\; k_{L_n}^\times=\mathbb{F}_{p^n}^\times
\]
has kernel $1+p\mathcal{O}_{L_n}$, which is a torsion-free. Therefore all torsion in $\mathcal{O}_{L_n}^\times$ comes from lifts of roots of unity in $\mathbb{F}_{p^n}^\times$. Since $\mathbb{F}_{p^n}^\times$ is cyclic of order $p^n-1$, we expect $\mu(L_n)$ to have the same order.

Let $\bar{\zeta}\in\mathbb{F}_{p^n}^\times$ be a generator. It satisfies $\bar{\zeta}^{p^n-1}=1$ and $(\bar{\zeta})^{p^n-1}-1=0$ in $\mathbb{F}_{p^n}$. Consider the polynomial
\[
f(X)=X^{p^n-1}-1\in\mathcal{O}_{L_n}[X].
\]
Its derivative $f'(X)=(p^n-1)X^{p^n-2}$ is nonzero mod $p$, since $p\nmid(p^n-1)$. Thus all roots of $f$ in the residue field are simple. By Hensel's lemma, each simple root in $\mathbb{F}_{p^n}$ lifts uniquely to a root in $\mathcal{O}_{L_n}$. Hence the reduction map induces an isomorphism
\[
\mu_{p^n-1}(\mathcal{O}_{L_n})\;\cong\;\mathbb{F}_{p^n}^\times,
\]
and we conclude that $\mu(L_n)=\mu_{p^n-1}$ when $p$ is odd. For $p=2$, we also have $-1\in L_n$ (of order $2$), so
\[
\mu(L_n)=\mu_{2(p^n-1)}.
\]

\item We now show that $L_n=\Q_p(\mu_{p^n-1})$. The polynomial $X^{p^n-1}-1$ splits completely over $L_n$ since all $(p^n-1)$-st roots of unity lie in $L_n$. Reducing mod $p$, $X^{p^n-1}-1$ also splits completely over $\mathbb{F}_{p^n}$, and not over any smaller field, because $\mathbb{F}_{p^n}^\times$ is the unique cyclic group of order $p^n-1$. Hence the minimal polynomial of a primitive $(p^n-1)$-st root of unity over $\Q_p$ has degree $n$. Therefore
\[
[\Q_p(\mu_{p^n-1}):\Q_p]=n.
\]
Since $p\nmid(p^n-1)$, the extension $\Q_p(\mu_{p^n-1})/\Q_p$ is unramified. But there exists a \emph{unique} unramified degree-$n$ extension of $\Q_p$ in $\overline{\Q}_p$, so we must have
\[
L_n=\Q_p(\mu_{p^n-1}).
\]

In summary, we have shown
\[
\mu(L_n)=
\begin{cases}
\mu_{p^n-1}, & p\text{ odd},\\[4pt]
\mu_{2(p^n-1)}, & p=2,
\end{cases}
\qquad\text{and}\qquad
L_n=\Q_p(\mu_{p^n-1}).
\]
\end{enumerate}
\end{solution}



\begin{problem}[4]
    Do [N, p.~134, Exercise 1 in Section II.4]: 
    Show that an infinite \emph{separable} algebraic extension $L$ of a non-Archimedean complete valued field $K$
    is never complete.
    (The separability condition is missing in that exercise but it is needed. 
    It is unnecessary, but feel free to assume that the valuation is discrete.)

    \textit{Hint:} 
    A possible idea is to construct a well-designed Cauchy sequence in $L$ that does not converge (so you get a contradiction if it converges).
    Krasner's lemma can help.

    \textbf{Examples:}
    When $K = \Q_p$, examples of naturally occurring infinite extensions (which are thus incomplete) are:
    \begin{itemize}
        \item the algebraic closure $\overline{\Q}_p$,
        \item the \emph{maximal unramified extension}
        \[
            \Q_p^{\mathrm{unr}} := \bigcup_{n \ge 1} L_n \quad (\text{where $L_n$ is as above}),
        \]
        \item the \emph{infinite $p$-cyclotomic extension}
        \[
            \Q_p(\mu_{p^\infty}) := \bigcup_{n \ge 1} \Q_p(\mu_{p^n}).
        \]
    \end{itemize}

    \noindent
    \textit{Note:} 
    The complete field $C$ is an infinite but non-algebraic extension of $\Q_p$.
    So it does not contradict the conclusion of Problem~4 above.
\end{problem}

\begin{solution}
Let $K$ be a complete non-Archimedean valued field and let $L/K$ be an infinite separable algebraic extension. 
Write $L=\bigcup_{n\ge1}L_n$ where $L_1\subset L_2\subset\cdots$ is an ascending tower of finite separable extensions with $[L_n:K]<\infty$ and $\bigcup_n L_n=L$. 
Each $L_n$ is complete because finite extensions of complete fields remain complete.

For each $n$, choose $\alpha_n\in L_{n+1}\setminus L_n$ and let $f_n(X)\in L_n[X]$ be its minimal polynomial. 
Since $f_n$ is separable, its distinct conjugates $\sigma(\alpha_n)$ satisfy
\[
\delta_n := \min_{\sigma\neq1}|\alpha_n-\sigma(\alpha_n)|>0.
\]
By the density of $L_n$ in $L_{n+1}$, we can choose $\beta_n\in L_{n+1}$ with 
$|\beta_n-\alpha_n|<\frac12\delta_n$. 
By Krasner's lemma}, $L_n(\alpha_n)=L_n(\beta_n)$, so replacing $\alpha_n$ by $\beta_n$ does not change the field extension, but we may assume $|\alpha_n|$ is as small as we wish.

Now define
\[
x_m := \alpha_1+\alpha_2+\cdots+\alpha_m \in L_m.
\]
By choosing each $\alpha_n$ small enough so that $|\alpha_{n+1}|<|\alpha_n|^2$, 
the sequence $(x_m)$ satisfies $|x_{m+1}-x_m|=|\alpha_{m+1}|\to0$. Hence $(x_m)$ is Cauchy in $L$. However, the limit of $(x_m)$ cannot lie in any finite stage $L_n$ since each $\alpha_{n+1}\notin L_n$; thus it has no limit in $L$. Therefore $L$ is not complete.

\end{solution}

\end{document}