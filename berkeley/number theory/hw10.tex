\documentclass[12pt]{article}  % or any other class
\usepackage{/Users/songye03/Desktop/Math_tex/style/psetconfig}         % loads your custom style
\usepackage{mathtools}
\title{Homework 10}
\author{Songyu Ye}
\date{\today}

\newtheorem{nonexample}{Non-Example}
\newtheorem{example}{Example}
\newtheorem{theorem}{Theorem}
\newtheorem{definition}{Definition}
\newtheorem{remark}{Remark}
\DeclareMathOperator{\vol}{vol}


\begin{document}
\psettitle

\begin{problem}[1]
Let
\[
L_i(x_1, \ldots, x_n) = \sum_{j=1}^n a_{ij} x_j, \qquad i = 1, \ldots, n,
\]
be real linear forms such that $\det(a_{ij}) \ne 0$, and let $c_1, \ldots, c_n$ be positive real numbers such that 
\[
c_1 \cdots c_n > |\det(a_{ij})|.
\]
Show that there exist integers $m_1, \ldots, m_n \in \mathbb{Z}$ such that
\[
|L_i(m_1, \ldots, m_n)| < c_i, \qquad i = 1, \ldots, n.
\]
\[
c_1 \cdots c_n > |\det(a_{ij})|.
\]
Show that there exist integers $m_1, \ldots, m_n \in \mathbb{Z}$ (not all zero) such that
\[
|L_i(m_1, \ldots, m_n)| < c_i, \qquad i = 1, \ldots, n.
\]

\textbf{Hint.} Use Minkowski's lattice point theorem.
\end{problem}

\begin{solution}
Let $A = (a_{ij})$ and define the linear map
\[
T : \mathbb{R}^n \to \mathbb{R}^n, \qquad
T(x_1,\dots,x_n) = (L_1(x),\dots,L_n(x)).
\]
Then $T$ is invertible since $\det(A) \neq 0$. The image
\[
\Gamma := T(\mathbb{Z}^n)
\]
is a lattice in $\mathbb{R}^n$ with
\[
\det(\Gamma) = |\det(A)| = |\det(a_{ij})|.
\]

Let
\[
D := \{ y = (y_1,\dots,y_n) \in \mathbb{R}^n : |y_i| < c_i \text{ for all } i \}.
\]
Then $D$ is convex, symmetric about the origin, and
\[
\operatorname{vol}(D) = \prod_{i=1}^n (2c_i) = 2^n c_1 \cdots c_n
> 2^n |\det(a_{ij})|
= 2^n \det(\Gamma).
\]
By Minkowski's lattice point theorem, $D$ contains a nonzero lattice point
\[
y = (y_1,\dots,y_n) \in \Gamma \cap D,\quad y \neq 0.
\]
Since $y \in \Gamma = T(\mathbb{Z}^n)$, there exists $m = (m_1,\dots,m_n) \in \mathbb{Z}^n$,
$m \neq 0$, such that
\[
y = T(m) = (L_1(m),\dots,L_n(m)).
\]
Because $y \in D$, we have
\[
|L_i(m_1,\dots,m_n)| = |y_i| < c_i
\quad \text{for } i = 1,\dots,n.
\]
This gives the desired integers $m_1,\dots,m_n$.
\end{solution}

\begin{problem}[2]
Give a proof of the following theorem (attributed to Dirichlet) by using Minkowski's theorem.

\begin{theorem}
Let $\alpha \in \mathbb{R}$ be a positive irrational number. Then there exist infinitely many pairs of 
$m,n \in \mathbb{Z}_{>0}$, which are coprime to each other, such that
\[
\left| \frac{n}{m} - \alpha \right| < \frac{1}{m^2}.
\]
\end{theorem}

\textbf{Hint.}
Try $\Gamma = \mathbb{Z}^2$, 
\[
D = \{ (x,y) \in \mathbb{R}^2 : x \in (-M,M),\; y - \alpha x \in (-\varepsilon,\varepsilon) \}
\]
for infinitely many values of $\varepsilon \to 0^+$ and suitably chosen $M>0$ (depending on $\varepsilon$).
\end{problem}

\begin{solution}
Fix $M>1$ and consider the lattice $\Gamma = \mathbb{Z}^2$.
For $\varepsilon>0$ define the symmetric convex set
\[
D_{M,\varepsilon}
= \{(x,y)\in\mathbb{R}^2 : |x|\le M,\ |y-\alpha x|\le 1/M + \varepsilon\}.
\]
Then
\[
\vol(D_{M,\varepsilon})
= (2M)\cdot 2\Big(\frac1M+\varepsilon\Big)
= 4(1+M\varepsilon)
> 2^2\det(\Gamma)=4.
\]
By Minkowski's theorem, there exists a nonzero lattice point
\((m,n)\in\Gamma\cap D_{M,\varepsilon}\).
So for each $\varepsilon>0$ there is $(m,n)\in\mathbb{Z}^2\setminus\{(0,0)\}$ with
\[
|m|\le M,
\qquad
|n-\alpha m|\le \frac1M+\varepsilon.
\tag{$*$}
\]

Now restrict to the finite set
\[
S_M=\{(m,n)\in\mathbb{Z}^2 : 0<|m|\le M\}.
\]
For $(m,n)\in S_M$ set
\[
f(m,n)=|n-\alpha m|-\frac1M.
\]
Suppose, for contradiction, that no lattice point lies in the strip
\(|x|\le M,\ |y-\alpha x|\le 1/M\), i.e.\ $f(m,n)>0$ for all $(m,n)\in S_M$.
Since $S_M$ is finite, we can define
\[
\delta := \min_{(m,n)\in S_M} f(m,n) > 0.
\]
Choose $\varepsilon$ with $0<\varepsilon<\delta$.
By Minkowski there exists $(m,n)\in S_M$ satisfying $(*)$, so
\[
f(m,n) = |n-\alpha m|-\frac1M \le \varepsilon < \delta,
\]
contradicting the definition of $\delta$.
Hence our assumption was false, and there exists a nonzero
\((m,n)\in\mathbb{Z}^2\) with
\[
|m|\le M,
\qquad
|n-\alpha m| \le \frac1M.
\]

For this pair we have
\[
\left|\frac{n}{m}-\alpha\right|
= \frac{|n-\alpha m|}{|m|}
\le \frac{1/M}{|m|}
\le \frac{1}{m^2}.
\]

Since we can choose $M$ arbitrarily large, this yields infinitely many such pairs
$(m,n)$ with $m>0$.
Finally, given such $(m,n)$ with $\gcd(m,n)=d>1$, write $m=dm'$, $n=dn'$.
Then
\[
\left|\frac{n'}{m'}-\alpha\right|
= \left|\frac{n}{m}-\alpha\right|
< \frac1{m^2}
\le \frac1{m'^2},
\]
so $(m',n')$ is coprime and also satisfies the inequality.
Thus there are infinitely many coprime $m,n\in\mathbb{Z}_{>0}$ with
\(
\left|\frac{n}{m}-\alpha\right| < \frac1{m^2}.
\)
\end{solution}

\begin{problem}[3]
The following subproblems are designed to let you apply Minkowski's theorem to prove Lagrange's theorem on four squares.

\begin{theorem}
For each $n \in \mathbb{Z}_{\ge 0}$, there exist $w,x,y,z \in \mathbb{Z}$ such that
\[
n = w^2 + x^2 + y^2 + z^2.
\]
\end{theorem}

\begin{enumerate}[(1)]
\item Reduce to the case where $n$ is square-free and (say) $n \ge 3$.

(If you like, you can further reduce to the case where $n$ is a prime: if two numbers are sums of four squares, 
then so is their product. This can be shown by considering quaternionic norms for $w + xi + yj + zk$, for example.)

\item Show that there exist $a,b \in \mathbb{Z}$ such that $a^2 + b^2 \equiv -1 \pmod{n}$.

(By CRT, it is enough to do this when $n$ is a prime. Then you can prove this by elementary number theory.)

\item Apply Minkowski's theorem to cleverly designed $\Gamma, D \subset \mathbb{R}^4$.

\textbf{Hint.} Try the following and notice that a nonzero element in the intersection gives a desired solution to the equation:
\[
\Gamma = \{ (w,x,y,z) \in \mathbb{Z}^4 :\; y \equiv a x + b y \pmod{n},\;
z \equiv b x - a y \pmod{n} \},
\]
\[
D = \{\, w^2 + x^2 + y^2 + z^2 < 2n \,\}.
\]
\end{enumerate}
\end{problem}

\begin{solution}
We prove that every $n\in\mathbb Z_{\ge0}$ is a sum of four squares. If $n=0$ or $1$ the statement is clear.
If $n=2$ then $2=1^2+1^2+0^2+0^2$.

Moreover, if $n_1$ and $n_2$ are sums of four squares, then so is $n_1n_2$:
this follows from the multiplicativity of the norm on Hamilton's quaternions. Thus it suffices to treat $n$ prime with $n\ge3$.


Fix an odd prime $p$. Among the residues mod $p$, there are $(p+1)/2$ quadratic residues
(including $0$). The set
\[
S = \{-1 - t^2 : t\in\mathbb F_p\}
\]
also has at most $(p+1)/2$ distinct values. Since $\mathbb F_p$ has $p$ elements, the
pigeonhole principle forces $S$ to contain a square. Thus there exist $a,b$ with
$a^2 \equiv -1 - b^2 \pmod p$, i.e.\ $a^2+b^2\equiv -1\pmod p$.
For $p=2$ one checks directly that $1^2+0^2\equiv -1\pmod2$.
Using CRT, we obtain $a,b\in\mathbb Z$ with
\[
a^2 + b^2 \equiv -1 \pmod n.
\]

Fix such $a,b$. Define a lattice $\Gamma\subset\mathbb Z^4$ by
\[
\Gamma
= \Bigl\{(w,x,y,z)\in\mathbb Z^4 :
\begin{array}{l}
y \equiv ax + bw \pmod n,\\[2pt]
z \equiv bx - aw \pmod n
\end{array}
\Bigr\}.
\]
Equivalently, $\Gamma$ is the kernel of the surjective homomorphism
\[
\phi:\mathbb Z^4 \to (\mathbb Z/n\mathbb Z)^2,\quad
\phi(w,x,y,z) = (\,y-ax-bw,\ z-bx+aw\,).
\]
Hence $|\operatorname{coker}\phi| = n^2$ and
\[
[\mathbb Z^4 : \Gamma] = n^2,
\]
so the determinant of $\Gamma$ is
\[
\det(\Gamma) = n^2.
\]
Let
\[
D = \{(w,x,y,z)\in\mathbb R^4 : w^2 + x^2 + y^2 + z^2 < 2n\}.
\]
Then $D$ is convex, symmetric, and
\[
\vol(D)
= \frac{\pi^2}{2} ( \sqrt{2n} )^4
= 2\pi^2 n^2
> 16 n^2
= 2^4 \det(\Gamma),
\]
since $\pi^2 > 8$.
By Minkowski's lattice point theorem, $D$ contains a nonzero point
\[
(w,x,y,z)\in\Gamma\cap D,\quad (w,x,y,z)\neq(0,0,0,0).
\]

Because $(w,x,y,z)\in\Gamma$, we have
\[
y \equiv ax + bw \pmod n,\qquad
z \equiv bx - aw \pmod n.
\]
Compute modulo $n$:
\begin{align*}
y^2 + z^2
&\equiv (ax + bw)^2 + (bx - aw)^2 \\
&= (a^2 + b^2)(x^2 + w^2)
\equiv - (x^2 + w^2) \pmod n.
\end{align*}
Thus
\[
w^2 + x^2 + y^2 + z^2 \equiv 0 \pmod n.
\]
Set
\[
Q := w^2 + x^2 + y^2 + z^2.
\]
We have $0 < Q < 2n$ (since $(w,x,y,z)\neq0$ and lies in $D$),
and $n \mid Q$, so necessarily $Q = n$.

Hence
\[
n = w^2 + x^2 + y^2 + z^2,
\]
as desired.
\end{solution}

\begin{problem}[4]
Show that in every ideal $\mathfrak{a} \neq 0$ of $\mathcal{O}_K$ there exists an element $a \neq 0$ such that
\[
|N_{K/\mathbb{Q}}(a)| \le M \, (\mathcal{O}_K : \mathfrak{a}),
\]
where
\[
M = \frac{n!}{n^n} \left( \frac{4}{\pi} \right)^s \sqrt{|d_K|}
\]
(the so-called \emph{Minkowski bound}).

\textbf{Hint.} Use exercise~2 to proceed as in (5.3), and make use of the inequality between arithmetic and geometric means,
\[
\frac{1}{n} \sum_\tau |z_\tau| \;\ge\; 
\Big( \prod_\tau |z_\tau| \Big)^{1/n}.
\]
\end{problem}

\begin{solution}
Let $K$ have degree $n=r+2s$ and discriminant $d_K$, and write
$K_{\mathbb R} = K\otimes_{\mathbb Q}\mathbb R \cong \mathbb R^r\times\mathbb C^s$.
Let
\[
\Phi:K\to K_{\mathbb R}
\]
be the Minkowski embedding. 



For a nonzero ideal $\mathfrak a\subset\cO_K$,
the image $\Lambda=\Phi(\mathfrak a)$ is a lattice with
\[
\det(\Lambda) = 2^{-s}\sqrt{|d_K|}\,(\cO_K:\mathfrak a).
\]

This is because in the standard identification $K_\mathbb{R} \cong \mathbb{R}^r \times \mathbb{C}^s \cong \mathbb{R}^n$, each complex embedding contributes a factor 2 in the Euclidean volume element. This gives
\[
\operatorname{vol}(\mathbb{R}^n / \Phi(\mathcal{O}_K)) = |\det A| \cdot 2^{-s} = 2^{-s}\sqrt{|\operatorname{disc}(\mathcal{O}_K)|} = 2^{-s}\sqrt{|d_K|}.
\]

and the fact that for lattices, determinant  scales by index:


For $t>0$ set
\[
X_t = \Bigl\{ z=(z_\tau)\in K_{\mathbb R} : \sum_{\tau} |z_\tau| < t \Bigr\}.
\]
This set is convex, centrally symmetric, and by Exercise~2 has volume
\[
\vol(X_t) = 2^r\pi^s\frac{t^n}{n!}.
\]

Choose $t$ so that
\[
\left(\frac{t}{n}\right)^n
= M\,(\cO_K:\mathfrak a),
\qquad
M = \frac{n!}{n^n}\left(\frac{4}{\pi}\right)^s\sqrt{|d_K|}.
\]
Equivalently,
\[
t^n = n!\left(\frac{4}{\pi}\right)^s\sqrt{|d_K|}\,(\cO_K:\mathfrak a).
\]
Then
\[
\vol(X_t)
= 2^r\pi^s\frac{t^n}{n!}
= 2^r\pi^s\left(\frac{4}{\pi}\right)^s\sqrt{|d_K|}\,(\cO_K:\mathfrak a)
= 2^{r+2s}\sqrt{|d_K|}\,(\cO_K:\mathfrak a)
= 2^n\sqrt{|d_K|}\,(\cO_K:\mathfrak a).
\]

Since
\[
2^n\det(\Lambda) = 2^n\cdot 2^{-s}\sqrt{|d_K|}\,(\cO_K:\mathfrak a)
= 2^{r+s}\sqrt{|d_K|}\,(\cO_K:\mathfrak a)
\le 2^n\sqrt{|d_K|}\,(\cO_K:\mathfrak a),
\]
we have $\vol(X_t)\ge 2^n\det(\Lambda)$, so (by Minkowski, enlarging $t$
slightly if a strict inequality is required) there exists a nonzero
$\alpha\in\mathfrak a$ with $\Phi(\alpha)\in X_t$. Hence
\[
\sum_{\tau} |\tau(\alpha)| < t.
\]

By the arithmetic--geometric mean inequality,
\[
\Bigl(\prod_{\tau} |\tau(\alpha)|\Bigr)^{1/n}
\le \frac{1}{n}\sum_{\tau} |\tau(\alpha)|
< \frac{t}{n},
\]
so
\[
|N_{K/\mathbb Q}(\alpha)|
= \prod_{\tau} |\tau(\alpha)|
\le \left(\frac{t}{n}\right)^n
= M\,(\cO_K:\mathfrak a).
\]
Thus every nonzero ideal $\mathfrak a$ contains a nonzero $\alpha$ with
$|N_{K/\mathbb Q}(\alpha)|\le M\,(\cO_K:\mathfrak a)$.
\end{solution}

\end{document}