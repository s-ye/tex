\documentclass[12pt]{article}  % or any other class
\usepackage{/Users/songye03/Desktop/Math_tex/style/psetconfig}         % loads your custom style
\usepackage{mathtools}
\title{Homework 8}
\author{Songyu Ye}
\date{\today}

\newtheorem{nonexample}{Non-Example}
\newtheorem{example}{Example}
\newtheorem{definition}{Definition}
\newtheorem{remark}{Remark}
\newcommand{\llbracket}{[\![}
\newcommand{\rrbracket}{]\!]}


\begin{document}
\psettitle


Fix an algebraic closure $\overline{\Q}_p$ of $\Q_p$ and choose a ``compatible''
system of $p$-power roots $p^{1/p},p^{1/p^2},\ldots$ in $\overline{\Q}_p$
such that $(p^{1/p^{i+1}})^p=p^{1/p^i}$ for all $i$.
Similarly, choose a compatible system of $p$-power roots of unity
$\zeta_p,\zeta_{p^2},\ldots$ such that $(\zeta_{p^{i+1}})^p=\zeta_{p^i}$ for all $i$.
Define
\[
  \Q_p(p^{1/p^\infty})\:= \bigcup_{i\ge1}\Q_p(p^{1/p^i}),
  \qquad
  \Q_p(\zeta_{p^\infty}):= \bigcup_{i\ge1}\Q_p(\zeta_{p^i}).
\]

\noindent\textbf{[Fact]} (you’re encouraged to verify, but you’re welcome to take for granted):
These are infinite extensions of $\Q_p$ which are totally ramified
in that every finite subextension is totally ramified, or equivalently
in that the residue fields of the two extensions are still $\F_p$.
Their completions, denoted by $\wh{\Q_p(p^{1/p^\infty})}$ and
$\wh{\Q_p(\zeta_{p^\infty})}$, are perfectoid fields.

\medskip

On the other hand, the valuation on the CDVF $\F_p((t))$ extends uniquely to
\[
  \F_p\bigl((t^{1/p^\infty})\bigr)=\bigcup_{i\ge1}\F_p\bigl((t^{1/p^i})\bigr),
\]
where $t^{1/p},t^{1/p^2},\ldots$ form a compatible system of $p$-power roots of $t$.
(Another way to think of $\F_p((t^{1/p^\infty}))$ is that it is the perfection of
$\F_p((t))$ in the same way $\F_p[X^{p^{-\infty}}]$ is the perfection of $\F_p[X]$.)
The completion $\wh{\F_p((t^{1/p^\infty}))}$ is a perfectoid field as well,
now of characteristic $p$.

You can identify the valuation rings in $\Q_p(p^{1/p^\infty})$,
$\Q_p(\zeta_{p^\infty})$, $\F_p((t^{1/p^\infty}))$ by applying \,[S] I.6 to finite
subextensions. (Actually you have inseparable extensions in the last case, but
[S] I.6(ii) still applies. Or, you can directly compute the valuation ring to be
$\F_p [[t^{1/p^\infty}]]$.) So you can find the valuation rings in
their completions by taking closures.

\begin{problem}[1]
\begin{enumerate}
  \item[(1)] Take $K\coloneqq \wh{\Q_p(p^{1/p^\infty})}$.
  Show that $K^\flat\cong \wh{\F_p((t^{1/p^\infty}))}$.

  \emph{[Hint]} We have the valuation rings $A,A^\flat$ for $K,K^\flat$ satisfying
  $A^\flat=\varprojlim_{x\mapsto x^p}A/pA$.
  Try to find an isomorphism $A/pA\cong \F_p[t^{1/p^\infty}](t)$,
  from which you should be able to understand $K^\flat$.
  You may want to choose $\varpi^\flat\in A^\flat$ to be the element given by
  $(p,p^{1/p},p^{1/p^2},\ldots)$ in the inverse limit.

  \item[(2)] For $K\coloneqq \wh{\Q_p(\zeta_{p^\infty})}$, show that
  $K^\flat\cong \wh{\F_p((t^{1/p^\infty}))}$.

  \emph{[Hint]} This time you may want to consider $(1-\zeta_p,1-\zeta_{p^2},\ldots)$
  in the inverse limit. \emph{(This time, this element need not correspond to
  $\varpi^\flat=t\in A^\flat$.)}

  \item[(3)] Show that the two fields $\wh{\Q_p(p^{1/p^\infty})}$ and
  $\wh{\Q_p(\zeta_{p^\infty})}$ are not isomorphic over $\Q_p$.
\end{enumerate}
\end{problem}

\begin{solution}
  \leavevmode
\begin{enumerate}
\item[(1)] Let $A=\mathcal O_K$ be the valuation ring. Then 
$A=\widehat{\Z_p[p^{1/p^\infty}]}$.
Reducing mod $p$ gives
\[
A/p \;\cong\; \widehat{\F_p[t^{1/p^\infty}]}\;=\;\F_p\llbracket t^{1/p^\infty}\rrbracket,
\]
where $t$ is the image of $p$. This is because $\Z_p[p^{1/p^n}]/p\cong \F_p[t^{1/p^n}]$, and completion commutes with filtered colimits here.

For a perfectoid field $K$, the tilt is
\[
K^\flat=\operatorname{Frac}\!\Big(\varprojlim_{x\mapsto x^p} A/p\Big)^\wedge.
\]
Since $A/p=\F_p\llbracket t^{1/p^\infty}\rrbracket$ is perfect, the inverse limit along Frobenius identifies canonically with the same ring, sending
$t \longleftrightarrow \varpi^\flat:=(p,p^{1/p},p^{1/p^2},\dots)\in A^\flat$.
Hence
\[
A^\flat \;\cong\; \F_p\llbracket t^{1/p^\infty}\rrbracket
\quad\Rightarrow\quad
K^\flat\;\cong\; \widehat{\F_p((t^{1/p^\infty}))}.
\]

\item[(2)] Now $A=\widehat{\Z_p[\zeta_{p^\infty}]}$. Put 
$\varpi^\flat:=(1-\zeta_p, 1-\zeta_{p^2}, 1-\zeta_{p^3},\dots)\in A^\flat$.
Cyclotomic $p$-adic estimates give
$v_p(1-\zeta_{p^{n+1}})=\frac{1}{p^n(p-1)}$
and $(1-\zeta_{p^{n+1}})^p=(1-\zeta_{p^n})\cdot u_n$
with $u_n\in A^\times$. Thus Frobenius sends the class of $1-\zeta_{p^{n+1}}$ to the class of $1-\zeta_{p^n}$, so $\varpi^\flat$ defines a pseudo-uniformizer in the tilt. Exactly the same reduction argument as above shows
\[
A/p \;\cong\; \F_p\llbracket t^{1/p^\infty}\rrbracket
\quad(t\leftrightarrow 1-\zeta_p),
\]
hence again $A^\flat\cong\F_p\llbracket t^{1/p^\infty}\rrbracket$ and
\[
K^\flat \;\cong\; \widehat{\F_p((t^{1/p^\infty}))}.
\]

\item[(3)] In $K_2:=\widehat{\Q_p(\zeta_{p^\infty})}$ we have $\mu_{p^\infty}\subset K_2^\times$ by construction.
In $K_1:=\widehat{\Q_p(p^{1/p^\infty})}$ there are no nontrivial $p$-power roots of unity. Indeed, any $\xi\in\mu_{p^\infty}$ satisfies $\xi\equiv1\pmod p$. But on $1+pA_1$ the $p$-adic logarithm is injective (for $p\ge3$; for $p=2$ one restricts to $1+4A_1$, and the same conclusion holds), hence the only $p$-power torsion is 1.

Therefore $\mu_{p^\infty}\subset K_2$ but $\mu_{p^\infty}\not\subset K_1$. 
\end{enumerate}
\end{solution}

\begin{remark}
    The conclusion is that non-isomorphic perfectoid fields may admit
isomorphic tilting. We say $\wh{\Q_p(p^{1/p^\infty})}$ and $\wh{\Q_p(\zeta_{p^\infty})}$
are (different) "untilts" of the perfectoid field $\wh{\F_p((t^{1/p^\infty}))}$.

\end{remark}

\begin{problem}[2]
Let $C,\,C^\flat$ be the perfectoid fields as in Problem Set~7, \#1,
i.e.\ $C$ is the completion of $\overline{\Q}_p$, and $C^\flat$ is the
(characteristic $p$) ``tilt'' of $C$.
Let $A,\,A^\flat$ denote the valuation rings of $C,\,C^\flat$.
(The standard notation in the literature for such $A,\,A^\flat$ is
$\mathcal O_C,\,\mathcal O_{C^\flat}$. Adopt it if you like.)
As in the preceding HW, we have a multiplicative map
$f^\sharp:A^\flat\to A$, sending $x\mapsto x^\sharp$.

Notice that $C^\flat$ is a perfect field of char~$p$, and $A^\flat$ is a perfect ring
of characteristic $p$. The functor $W$ gives us the strict $p$-ring equipped with
surjection (the mod $p$ quotient map)
\[
  W(A^\flat)\twoheadrightarrow A^\flat
\]
which admits a Teichmüller lift $[\;]:A^\flat\to W(A^\flat)$.

\begin{enumerate}
  \item[(1)] Show that the map $x\mapsto x^\sharp \bmod p$ induces a ring isomorphism
  \[
    A^\flat/\varpi^\flat A^\flat \;\cong\; A/pA,
  \]
  where $\varpi^\flat\in A^\flat$ is an element as in Problem Set~7 such that
  $\lvert(\varpi^\flat)^\sharp\rvert=\lvert p\rvert$ (we’re taking $\varpi=p$ here).

  \emph{[Tip]} Freely use results from the previous homework; then you can do (1)
  without much extra work.

  \item[(2)] Prove that there is a \emph{unique} ring homomorphism
  \[
    \theta: W(A^\flat)\longrightarrow A
  \]
  such that $\theta([x])=x^\sharp$ for all $x\in A^\flat$.

  \emph{[Remark]} Here it’s not hard to see how to define $\theta$ uniquely; the main
  problem is to check the homomorphism property.

  \emph{[Hint]} The idea is similar to [S] p.\,38, Prop.\,10 but the latter is not exactly
  applicable as $A$ is not a $p$-ring (since $A/pA$ is not a perfect ring). Instead,
  adapt to a variant: see Lemma~1.1.6 of \emph{this paper}.

  \item[(3)] Show that the map $\theta$ is surjective.

  \emph{Hint:} Use part (1).
\end{enumerate}
\end{problem}

\begin{solution}
Write $C$ for the completed algebraic closure of $\Q_p$, $C^\flat$ its tilt, and
    $A=\mathcal O_C$, $A^\flat=\mathcal O_{C^\flat}$.
    Recall $A^\flat=\varprojlim_{F} A/pA$ (with respect to Frobenius), elements $x\in A^\flat$ are sequences
    $x=(x^{(0)},x^{(1)},\ldots)$ with $(x^{(n+1)})^p=x^{(n)}$.
    The multiplicative map $x\mapsto x^\sharp\colon A^\flat\to A$ is
    $x^\sharp=\lim_{n\to\infty} \widetilde{x^{(n)}}^{\,p^n}\in A$, where $\widetilde{x^{(n)}}\in A$ is any lift of $x^{(n)}\in A/p$; the limit exists and is independent of choices. Fix $\varpi^\flat\in A^\flat$ with $|(\varpi^\flat)^\sharp|=|p|$.

\begin{enumerate}
    \item[(1)] For $x=(x^{(0)},x^{(1)},\ldots)\in A^\flat$,
    $x^\sharp \equiv x^{(0)} \pmod{pA}$.
    Indeed, choose lifts $\widetilde{x^{(n)}}\equiv x^{(n)}\ (\bmod\ p)$. Then
    $\widetilde{x^{(n)}}^{p^n}\equiv (x^{(n)})^{p^n}=x^{(0)}\ (\bmod\ p)$;
    taking the limit preserves the congruence. Hence the reduction
    $r: A^\flat\longrightarrow A/pA$, $x\longmapsto x^\sharp\bmod p$
    is just the projection $A^\flat=\varprojlim_F (A/p)\to A/p$ onto the 0-th coordinate.
    In particular, $r$ is surjective.

    (b) Kernel is $\varpi^\flat A^\flat$.
    If $x\in\ker r$, then $x^{(0)}=0$. By compatibility under Frobenius,
    $x^{(n)}=0$ for all $n$ in the valuation sense appropriate to $A/p$, and one checks
    (using that $\varpi^\flat=(p\!\!\mod p,\,p^{1/p}\!\!\mod p,\ldots)$ and that Frobenius on $A/p$
    is surjective for a perfectoid $A$) that this is equivalent to divisibility by
    $\varpi^\flat$: there exists $y\in A^\flat$ with $x=\varpi^\flat y$.
    Conversely, $\varpi^\flat y$ always maps to 0 mod $p$.
    Hence $\ker r=\varpi^\flat A^\flat$.

Therefore we get an isomorphism
\[A^\flat/\varpi^\flat A^\flat \;\xrightarrow{\ \sim\ }\; A/pA\]
via $x\mapsto x^\sharp\bmod p$.

    \item[(2)] Recall the two following facts about Witt vectors:
    \begin{itemize}
        \item Every Witt vector has a Teichmüller expansion
    $w=\sum_{n=0}^\infty p^n [x_n]$, $x_n\in A^\flat$,
    converging $p$-adically in $W(A^\flat)$ and unique.

    \item The Teichmüller map $[\,\cdot\,]:A^\flat\to W(A^\flat)$ is multiplicative, and the Witt
    construction is designed so that any continuous ring map out of $W(A^\flat)$ is determined
    by its values on $[x]$ (with a compatibility that we will meet).
    \end{itemize}
    Define $\theta$ on Teichmüller series by
    \[
\theta\!\left(\sum_{n=0}^\infty p^n [x_n]\right)
    \;:=\; \sum_{n=0}^\infty p^n \,(x_n^\sharp)^{p^n}
    \]
    The series on the right converges in $A$ because $|x_n^\sharp|\le1$, so
    $|p^n(x_n^\sharp)^{p^n}|\le |p|^n\to0$.
    A standard Witt–polynomial check shows that the above respects addition and multiplication and hence defines a continuous ring homomorphism with
    $\theta([x])=x^\sharp$. A continuous ring map is determined by its values on $[x]$. Since we prescribed
    $\theta([x])=x^\sharp$, $\theta$ is unique.

    \item[(3)] We reduce $\theta$ modulo $p$. We have
    $W(A^\flat)\xrightarrow{\ \theta\ } A \longrightarrow A/pA$
    is the map $w\mapsto$ (0th Witt component) $\in A/pA$, which is surjective.


    Now we can lift $p$-adically to show surjectivity. Given $a\in A$:
    \begin{itemize}
        \item First, choose $x_0\in A^\flat$ such that $x_0^\sharp\equiv a \pmod{p}$. Let $w_0=[x_0]$. Then $\theta(w_0)\equiv a \pmod{p}$.
        \item For each $n\ge 0$, suppose we have $w_n$ with $\theta(w_n)\equiv a \pmod{p^{n+1}}$. By part (1), the reduction of $\theta$ on $p^{n+1}/p^{n+2}$ is the identity map. Thus we can find $x_{n+1}\in A^\flat$ that corrects the error modulo $p^{n+2}$.
        \item Set $w_{n+1}=w_n+p^{n+1}[x_{n+1}]$ to get $\theta(w_{n+1})\equiv a \pmod{p^{n+2}}$.
    \end{itemize}

    The $w_n$ converge to $w\in W(A^\flat)$ with $\theta(w)=a$. Hence $\theta$ is surjective.
\end{enumerate}
\end{solution}

\noindent\textbf{[Note 1]} The ring $W(A^\flat)$ equipped with the map $\theta$ is often
called $A_{\mathrm{inf}}$ and plays a central role in $p$-adic Hodge theory, and related
topics, e.g., see standard references.

\noindent\textbf{[Note 2]} Nothing is really special about our choice of $C,\,C^\flat$.
These assertions are valid for any perfectoid field $C$ of characteristic $0$ and its tilt
$C^\flat$ of characteristic $p$. In fact, $W(A^\flat)$ turns out to encode ``untilts'' of a
given $C^\flat$. (In general there are many perfectoid fields whose tilts are isomorphic to
$C^\flat$.) See the paragraph below Lemma~2.2.3, p.\,17 of Weinstein’s notes for a discussion.

\bigskip

\begin{problem}[3]
Show that
\[
  \bigl\{\,1,\ \theta,\ \tfrac12\,(\theta+\theta^2)\,\bigr\}
\]
is an integral basis (i.e.\ a $\Z$-basis of the ring of integers) of $\Q(\theta)$,
where $\theta$ is a root of $\theta^3-\theta-4=0$.
\end{problem}

\begin{solution}
    It is clear that $x^3 - x - 4$ is irreducible because it has no rational roots. For $x^3+ax+b$ the discriminant is $\Delta=-4a^3-27b^2$.
    Here $a=-1$, $b=-4$, so
    \[
    \Delta(f)=-4(-1)^3-27(-4)^2=4-432=-428=-4\cdot 107.
    \]
    Thus $\mathrm{disc}(1,\theta,\theta^2)=-428$.
    Hence the index $[\mathcal{O}_K:\mathbb{Z}[\theta]]$ divides 2.
    Set $\alpha = \frac{\theta^2+\theta}{2}$.
    Using $\theta^3=\theta+4$:
    \[
    \theta^3=\theta\theta^2=\theta(2\alpha-\theta)=2\alpha\theta-(2\alpha-\theta)=2\alpha\theta-2\alpha+\theta
    \]
    so comparing with $\theta+4$ gives $2\alpha(\theta-1)=4$, i.e.
    $\theta=1+\frac{2}{\alpha}$.
    Substituting in $f(\theta)=0$:
    \[
    (1+2/\alpha)^3-(1+2/\alpha)-4=0
    \implies 4\alpha^3-4\alpha^2-12\alpha-8=0
    \implies \alpha^3-\alpha^2-3\alpha-2=0
    \]
Hence $\alpha\in\mathcal{O}_K$. Now we check that the discriminant drops by the expected factor.
    Express the new basis $B'=\{1,\theta,\alpha\}$ in terms of the old $B=\{1,\theta,\theta^2\}$:
    $\alpha=\frac{1}{2}(\theta+\theta^2)$ gives
    \[
    \begin{bmatrix}1\\ \theta\\ \alpha\end{bmatrix} = 
    \begin{bmatrix}
    1&0&0\\
    0&1&\frac{1}{2}\\
    0&0&\frac{1}{2}
    \end{bmatrix}
    \begin{bmatrix}1\\ \theta\\ \theta^2\end{bmatrix}
    \]
    Thus $\det M=\frac{1}{2}$, and discriminants transform by
    \[\mathrm{disc}(B')=\mathrm{disc}(B)\cdot(\det M)^2=(-428)\cdot \frac{1}{4}=-107\]
    Since all elements of $B'$ are integral and $\mathrm{disc}(B')$ is square-free ($-107$), the order $\mathbb{Z}[1,\theta,\alpha]$ equals $\mathcal{O}_K$.
\end{solution}
\end{document}