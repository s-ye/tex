
\documentclass[12pt]{article}  % or any other class
\usepackage{/Users/songye03/Desktop/Math_tex/style/psetconfig}         % loads your custom style
\title{Homework 4}
\author{Songyu Ye}
\date{\today}

\newcommand{\Tr}{\operatorname{Tr}}
\newcommand{\fp}{\mathfrak{p}}
\newcommand{\m}{\mathfrak{m}}
\newcommand{\OO}{\mathcal{O}}

\newtheorem{theorem}{Theorem}[section]
\newtheorem{proposition}[theorem]{Proposition}
\newtheorem{remark}[theorem]{Remark}

\begin{document}
\psettitle

\begin{problem}[Problem 1]
Let $p$ be a prime number, and $n$ a positive integer greater than $1$. Find an example for each of the following with brief justifications.
\begin{enumerate}
    \item[(1)] A degree $n$ extension of $\mathbb{Q}$ in which $p$ is inert (i.e.\ the ring of integers in the extension possesses a unique prime $\mathfrak{q}$ above $p$, and the inertial degree $f_{\mathfrak{q}/p}$ is equal to $n$).
    \item[(2)] A degree $n$ extension of $\mathbb{Q}$ in which $p$ is totally ramified.
\end{enumerate}
\textit{Hint:} You can apply results of Serre, I.6 after localizing at $p$.

\textit{Remark:} There’s nothing special about $\mathbb{Q}$. The same question can be answered similarly with any global field in place of $\mathbb{Q}$.
\end{problem}

\begin{solution}
    \begin{enumerate}
        \item[(1)] Pick any monic irreducible polynomial $m(x)\in\mathbb{F}_p[x]$ of degree $n$. Lift its coefficients to $\mathbb{Z}$ to get $f(x)\in\mathbb{Z}[x]$ with the same degree and reduction $\overline{f}=m$. Let $K=\mathbb{Q}(\alpha)$ with $f(\alpha)=0$.

              Since $\overline{f}$ is irreducible over $\mathbb{F}_p$, Gauss's lemma gives that $f$ is irreducible over $\mathbb{Q}$, so $[K:\mathbb{Q}]=n$. Over a finite field, every irreducible polynomial is separable; hence $\gcd(\overline{f},\overline{f}')=1$. In particular, $\overline{f}$ has distinct roots and so the discriminant $\operatorname{disc}(\overline{f})\neq 0$ in $\mathbb{F}_p$ (If one computes the discriminant over $\mathbb{Z}$ and then reduce mod p, one gets the discriminant of the reduced polynomial $\overline f$). This means $p\nmid\operatorname{disc}(f)$.



              The relationship between the discriminant of $f$ and that of $K$ is given by
              \[\operatorname{disc}(f) \;=\; \operatorname{disc}(K)\cdot [\mathcal{O}_K:\mathbb{Z}[\alpha]]^2.
              \] which implies that $p\nmid [\mathcal{O}_K:\mathbb{Z}[\alpha]]$. Therefore, Dedekind's theorem applies to $f$ and $p$. By Dedekind's theorem, the factorization of $(p)$ in $\mathcal{O}_K$ matches the factorization of $\overline{f}$ in $\mathbb{F}_p[x]$. Since $\overline{f}$ is irreducible of degree $n$, we get a single prime $\mathfrak{q}$ above $p$ with residue degree $f_{\mathfrak{q}/p}=n$. Hence $p$ is inert.

        \item[(2)] Recall the following proposition from Serre's Local Fields.
              \begin{proposition}[Serre Proposition 1.6.17]
                  Let $A$ be a local ring with residue field $k$. Let $f \in A[x]$ be a monic polynomial. Let $B_f = A[x]/(f)$ free and finite type $A$-algebra. Suppose $A$ is a DVR and $f$ has the form \begin{align*}
                      f(x) = x^n + a_{n-1}x^{n-1} + \cdots + a_1 x + a_0, \quad a_i \in \mathfrak{m}_A \text{ for all } i, \quad a_0 \notin \mathfrak{m}_A^2.
                  \end{align*} i.e. $f$ is Eisenstein. Then $B_f$ is a DVR with uniformizer the class of $x$ in $B_f$, and residue field of $B_f$ is $k$.
              \end{proposition}

              Apply this proposition to $A=\mathbb{Z}_{(p)}$, the localization of $\mathbb{Z}$ at the prime ideal $(p)$, with $f(x)=x^n - p$. Then $B_f = \mathbb{Z}_{(p)}[x]/(x^n - p)$ is a DVR with residue field $\mathbb{F}_p$. The corresponding field extension is $K = \mathrm{Frac}(B_f) = \mathbb{Q}(\sqrt[n]{p})$. The minimal polynomial $f(x) = x^n-p$ is Eisenstein at $p$, so $[K:\mathbb{Q}]=n$. Eisenstein at $p$ implies that $p$ is totally ramified in $K$ because the residue field extension is trivial and therefore the ramification index must be $n$.
    \end{enumerate}
\end{solution}

\bigskip

\begin{problem}[Problem 2]
Let $A$ be a Dedekind domain, $K = \mathrm{Frac}(A)$. Let $L/K$ be a finite separable extension with normal closure $M$ of $L$ so that $M$ is Galois over $K$. Let $\mathfrak{p}$ be a prime ideal of $A$. Fix a prime ideal $\mathfrak{t}$ of $M$ above $\mathfrak{p}$. (By convention, this means $\mathfrak{t}$ is a nonzero prime in the integral closure of $A$ in $M$ such that $\mathfrak{t}$ divides $\mathfrak{p}$.) Denote by $D_{\mathfrak{t}}(M/K)$ the decomposition group of $\mathfrak{t}$ in $M/K$.
\begin{enumerate}
    \item[(i)] Define a map
          \[
              \mathrm{Gal}(M/K) \to \{\text{primes of $L$ above $\mathfrak{p}$}\},
              \qquad \sigma \mapsto \sigma(\mathfrak{t}) \cap L.
          \]
          Show that this map induces a bijection
          \[
              \mathrm{Gal}(M/L)\backslash \mathrm{Gal}(M/K)/D_{\mathfrak{t}}(M/K) \;\;\tilde{\longrightarrow}\;\; \{\text{primes of $L$ above $\mathfrak{p}$}\}.
          \]
    \item[(ii)] Assume that $\mathrm{Gal}(M/K) \simeq S_3$, the symmetric group in 3 variables, that $D_{\mathfrak{t}}(M/K)$ and $\mathrm{Gal}(M/L)$ are order $2$ subgroups of $\mathrm{Gal}(M/K)$ which are equal (not just isomorphic). Use part (i) to verify that $\mathfrak{p}$ does \emph{not} split completely in $L$.
\end{enumerate}
\textit{Remark:} The point of (ii) is that when the decomposition group of $\mathfrak{t}$ is not normal in $\mathrm{Gal}(M/K)$, the prime $\mathfrak{t}$ need not split completely in the decomposition field, which is $L$ here. A concrete example for (ii) can be given when
\[
    K = \mathbb{Q}, \quad L = \mathbb{Q}(\sqrt[3]{2}), \quad M = \mathbb{Q}(\sqrt[3]{2}, \zeta_3).
\]
By the Chebotarev density theorem, or by explicit computation, you can find $\mathfrak{t}$ such that $(\mathfrak{t}, M/K)$ is the unique nontrivial element of $\mathrm{Gal}(M/L)$. Then all the conditions of (ii) are satisfied.
\end{problem}

\begin{solution}
    \begin{enumerate}
        \item[(i)] Let $G=\mathrm{Gal}(M/K)$, $H=\mathrm{Gal}(M/L)$, and fix a prime $\mathfrak{t}$ of $M$ above $\mathfrak{p}\subset A$.
              Define
              \[\Phi:G\longrightarrow\{\text{primes of }L\text{ above }\mathfrak{p}\},\qquad
                  \sigma\longmapsto \big(\sigma\mathfrak{t}\big)\cap L.\]

              Since $\sigma$ is a $K$-automorphism, it fixes $\mf p$ and therefore the contraction of $\sigma\mathfrak{t}$ to $L$ is a prime of $L$ above $\mathfrak{p}$. In particular, the target of $\Phi$ is correct.

              Moreover, the map $\Phi$ is right $D_{\mathfrak{t}}$-invariant and left $H$-invariant:

              If $d\in D_{\mathfrak{t}}(M/K)=\{g\in G:g\mathfrak{t}=\mathfrak{t}\}$, then \[\Phi(\sigma d)=\big(\sigma d\,\mathfrak{t}\big)\cap L=\big(\sigma\mathfrak{t}\big)\cap L=\Phi(\sigma)\]
              If $h\in H$ (so $h$ fixes $L$), then
              \[\Phi(h\sigma)=\big(h\sigma\mathfrak{t}\big)\cap L
                  =h\big((\sigma\mathfrak{t})\cap L\big)
                  =(\sigma\mathfrak{t})\cap L=\Phi(\sigma)\]
              Thus $\Phi$ is constant on double cosets $H\sigma D_{\mathfrak{t}}$.

              So $\Phi$ descends to a map
              \[\overline\Phi:\ H\backslash G/D_{\mathfrak{t}}\ \longrightarrow\ \{\text{primes of }L\text{ above }\mathfrak{p}\}.\]
              Now I claim that $\overline\Phi$ is surjective and injective.

              Let $\mathfrak{q}$ be a prime of $L$ above $\mathfrak{p}$. Choose a prime $\mathfrak{t}'$ of $M$ above $\mathfrak{q}$. Because $M/K$ is Galois, there exists $\sigma\in G$ with $\sigma\mathfrak{t}=\mathfrak{t}'$. Then
              $\overline\Phi(H\sigma D_{\mathfrak{t}})=(\sigma\mathfrak{t})\cap L=\mathfrak{q}$.

              Suppose $\overline\Phi(H\sigma_1 D_{\mathfrak{t}})=\overline\Phi(H\sigma_2 D_{\mathfrak{t}})$. Then
              $(\sigma_1\mathfrak{t})\cap L=(\sigma_2\mathfrak{t})\cap L=: \mathfrak{q}$.
              Primes of $M$ above the same $\mathfrak{q}$ form a single $H$-orbit (see remark), so there is $\tau\in H$ with
              $\tau\sigma_1\mathfrak{t}=\sigma_2\mathfrak{t}$. Hence $\sigma_2^{-1}\tau\sigma_1\in D_{\mathfrak{t}}$, i.e.
              $\sigma_2\in H\sigma_1 D_{\mathfrak{t}}$. Thus the double cosets coincide.

              Therefore $\overline\Phi$ is a bijection:
              \[H\backslash G / D_{\mathfrak{t}}\ \xrightarrow{\ \sim\ }\ \{\text{primes of }L\text{ above }\mathfrak{p}\}.\]
        \item[(ii)] Assume $G\simeq S_3$, $|G|=6$, and that both $H=\mathrm{Gal}(M/L)$ and $D_{\mathfrak t}(M/K)$ are order 2 subgroups and are equal. Then $[L:K]=|G|/|H|=3$.

              By (i), the primes of $L$ above $\mathfrak{p}$ are in bijection with the double cosets $H\backslash G / H$. Take $H=\langle (12)\rangle\le S_3$ for concreteness. There are two double cosets, $H$ and $H(13)H$. One can check that the latter has size $4$. Thus there are exactly two primes of $L$ above $\mathfrak{p}$. If $\mathfrak{p}$ split completely in $L$, there would be $[L:K]=3$ distinct primes over $\mathfrak{p}$. Therefore $\mathfrak{p}$ does not split completely in $L$.
    \end{enumerate}

    \begin{remark}[This remark is just for myself]
        Suppose $M/K$ is finite Galois (i.e. finite, separable, normal). For any intermediate field $K\subseteq L\subseteq M$:

        $M/L$ is separable: Take any $\alpha \in M$. Its minimal polynomial over $K$, say $m_\alpha(x)$, is separable (no repeated roots). The minimal polynomial of $\alpha$ over $L$ divides $m_\alpha(x)$ in $L[x]$. A factor of a separable polynomial is still separable, so the minimal polynomial of $\alpha$ over $L$ is separable.

        $M/L$ is normal: Recall that for finite extensions, normal means that the minimal polynomial of any element in the extension splits completely in the extension. Take any $\alpha \in M$. Its minimal polynomial over $K$, say $m_\alpha(x)$, splits completely in $M$ since $M/K$ is normal. The minimal polynomial of $\alpha$ over $L$ divides $m_\alpha(x)$ in $L[x]$. Since $m_\alpha(x)$ splits completely in $M$, so does its factor, the minimal polynomial of $\alpha$ over $L$. Hence $M/L$ is normal.

        So $M/L$ is both separable and normal $\Rightarrow$ Galois. Thus $H=\mathrm{Gal}(M/L)$ is indeed the full automorphism group of $M$ over $L$.
    \end{remark}
\end{solution}

\bigskip

\begin{problem}[Neukirch Ch.~I.9, Exercise 3]
Continue the general setup from Problem~2. Assume the following:
\begin{enumerate}
    \item[(i)] $L/K$ is solvable, meaning that $\mathrm{Gal}(M/K)$ is a solvable group. (We are not assuming $M=L$.) Recall that a group $G$ is solvable if there is a chain of subgroups
          \[\{1\} = G_0 \triangleleft G_1 \triangleleft \cdots \triangleleft G_n = G\]
          such that each $G_i$ is normal in $G_{i+1}$ and the quotient $G_{i+1}/G_i$ is abelian.
    \item[(ii)] $p = [L:K]$ is a prime number.
\end{enumerate}

Now let $\mathfrak{p}$ be a prime of $K$ unramified in $L$. If there are two primes $\mathfrak{q}$ and $\mathfrak{q}'$ of $L$ above $\mathfrak{p}$ such that the inertial degrees $f_{\mathfrak{q}}$ and $f_{\mathfrak{q}'}$ are equal to $1$, then show that $\mathfrak{p}$ splits completely in $L/K$.

\textit{Caveat:} The extension degree $p$ has nothing to do with the prime ideal $\mathfrak{p}$ in the problem.

\textit{Hint:} Let $S_p$ denote the symmetric group in $p$ letters acting on $\{1,2,\dots,p\}$. If $G$ is a solvable subgroup of $S_p$ acting transitively on $\{1,2,\dots,p\}$ then every nontrivial element of $G$ fixes at most one element in $\{1,2,\dots,p\}$. (A reference for this fact is given in Neukirch.)
\end{problem}

\begin{solution}
    Let $G=\mathrm{Gal}(M/K)$, $H=\mathrm{Gal}(M/L)$. Then $[L:K]=[G:H]=p$ is prime. Let $\mathfrak{p}$ be a prime of $K$ unramified in $L$. Fix $\mathfrak{t}\mid\mathfrak{p}$ in $M$. Let $D=D_{\mathfrak{t}}(M/K)$, $I=I_{\mathfrak{t}}(M/K)$.

    Let $X=H \backslash G$. The set $X$ has size $p$ and there is a transitive action of $G$ on $X$ by right multiplication. Right multiplication gives a homomorphism $\pi:G\hookrightarrow S_X\cong S_p$, whose image $G^\ast:=\pi(G)$ is transitive and solvable.

    Recall by the previous problem that $X/D_{\mathfrak{t}}$ is in bijection with the primes of $L$ above $\mathfrak{p}$. For the base point $\bar e\in X$, $\mathrm{Stab}_D(\bar e)=\{d\in D:H d=H\} = D\cap H$. Hence $|\text{orbit of }\bar e| \;=\; [D:\,D\cap H]$. Moreover, we have that
    \[D/I \cong \mathrm{Gal}\!\big(\kappa(\mathfrak t)/\kappa(\mathfrak p)\big)\] so \[|D/I|=f_{\mathfrak t/\mathfrak p}\]
    \[(D\cap H)/(I\cap H) \cong \mathrm{Gal}\!\big(\kappa(\mathfrak t)/\kappa(\mathfrak q)\big)\] so $|(D\cap H)/(I\cap H)|=f_{\mathfrak t/\mathfrak q}$.

    Therefore
    \[[D:\,D\cap H]
        =\frac{|D|}{|D\cap H|}
        =\frac{|D/I|}{|(D\cap H)/(I\cap H)|}
        =\frac{f_{\mathfrak t/\mathfrak p}}{f_{\mathfrak t/\mathfrak q}}
        = f_{\mathfrak q/\mathfrak p}\]

    Thus we see that the D-orbit size on $X$ for $\mathfrak q$ equals $f_{\mathfrak q/\mathfrak p}$.

    Restriction and reduction give a surjection $D \xrightarrow{\ \mathrm{res}\ } D_{\mathfrak q}(L/K) \;\twoheadrightarrow\; \mathrm{Gal}\!\big(\kappa(\mathfrak q)/\kappa(\mathfrak p)\big)$ and since $\mf p$ is unramified in $L$, the kernel of $D \to \mathrm{Gal}(\kappa(\mathfrak q)/\kappa(\mathfrak p))$ is precisely $D\cap H$. In particular the size of the orbit through $\bar e$ is $f_{\mathfrak q/\mathfrak p} = \ord \Frob_{\mathfrak p}$, where $\Frob_{\mathfrak p}$ is the Frobenius element in $\mathrm{Gal}(\kappa(\mathfrak q)/\kappa(\mathfrak p))$.

In particular, $f=1$ if and only if the corresponding point of $X$ is fixed by $\mathrm{Frob}_{\mathfrak{p}}$. We know there exist two primes $\mathfrak{q},\mathfrak{q}'$ of $L$ above $\mathfrak{p}$ with $f_{\mathfrak{q}/\mathfrak{p}}=f_{\mathfrak{q}'/\mathfrak{p}}=1$. Equivalently, the permutation $\mathrm{Frob}_{\mathfrak{p}}\in G^\ast\le S_p$ fixes two distinct points of $X$. Therefore, the hint implies that $\mathrm{Frob}_{\mathfrak{p}}$ must be the identity permutation.

If the Frobenius permutation is the identity, all its cycles have length 1; hence $f_{\mathfrak{q}/\mathfrak{p}}=1$ for every prime $\mathfrak{q}$ of $L$ over $\mathfrak{p}$. Since $\mathfrak{p}$ is unramified in $L$, also $e_{\mathfrak{q}/\mathfrak{p}}=1$ for all $\mathfrak{q}$. Now we use the identity \[[L:K]=\sum_{\mathfrak{q}\mid\mathfrak{p}} e_{\mathfrak{q}/\mathfrak{p}} f_{\mathfrak{q}/\mathfrak{p}} =\#\{\mathfrak{q}\mid\mathfrak{p}\}\] Because $[L:K]=p$, there are $p$ distinct primes above $\mathfrak{p}$, as desired.
\end{solution}



\end{document}