
\documentclass[12pt]{article}  % or any other class
\usepackage{/Users/songye03/Desktop/Math_tex/style/psetconfig}         % loads your custom style
\title{Homework 4}
\author{Songyu Ye}
\date{\today}

\newcommand{\Tr}{\operatorname{Tr}}
\newcommand{\fp}{\mathfrak{p}}
\newcommand{\m}{\mathfrak{m}}
\newcommand{\OO}{\mathcal{O}}

\newtheorem{theorem}{Theorem}[section]

\begin{document}
\psettitle

\begin{problem}
Let $p$ be a prime number, and $n$ a positive integer greater than $1$. Find an example for each of the following with brief justifications.
\begin{enumerate}
    \item[(1)] A degree $n$ extension of $\mathbb{Q}$ in which $p$ is inert (i.e.\ the ring of integers in the extension possesses a unique prime $\mathfrak{q}$ above $p$, and the inertial degree $f_{\mathfrak{q}/p}$ is equal to $n$).
    \item[(2)] A degree $n$ extension of $\mathbb{Q}$ in which $p$ is totally ramified.
\end{enumerate}
\textit{Hint:} You can apply results of Serre, I.6 after localizing at $p$.

\textit{Remark:} There’s nothing special about $\mathbb{Q}$. The same question can be answered similarly with any global field in place of $\mathbb{Q}$.
\end{problem}

\begin{solution}
% leave blank or insert solution later
\end{solution}

\bigskip

\begin{problem}
Let $A$ be a Dedekind domain, $K = \mathrm{Frac}(A)$. Let $L/K$ be a finite separable extension with normal closure $M$ of $L$ so that $M$ is Galois over $K$. Let $\mathfrak{p}$ be a prime ideal of $A$. Fix a prime ideal $\mathfrak{t}$ of $M$ above $\mathfrak{p}$. (By convention, this means $\mathfrak{t}$ is a nonzero prime in the integral closure of $A$ in $M$ such that $\mathfrak{t}$ divides $\mathfrak{p}$.) Denote by $D_{\mathfrak{t}}(M/K)$ the decomposition group of $\mathfrak{t}$ in $M/K$.
\begin{enumerate}
    \item[(i)] Define a map
    \[
        \mathrm{Gal}(M/K) \to \{\text{primes of $L$ above $\mathfrak{p}$}\}, 
        \qquad \sigma \mapsto \sigma(\mathfrak{t}) \cap L.
    \]
    Show that this map induces a bijection
    \[
        \mathrm{Gal}(M/L)\backslash \mathrm{Gal}(M/K)/D_{\mathfrak{t}}(M/K) \;\;\tilde{\longrightarrow}\;\; \{\text{primes of $L$ above $\mathfrak{p}$}\}.
    \]
    \item[(ii)] Assume that $\mathrm{Gal}(M/K) \simeq S_3$, the symmetric group in 3 variables, that $D_{\mathfrak{t}}(M/K)$ and $\mathrm{Gal}(M/L)$ are order $2$ subgroups of $\mathrm{Gal}(M/K)$ which are equal (not just isomorphic). Use part (i) to verify that $\mathfrak{p}$ does \emph{not} split completely in $L$.
\end{enumerate}
\textit{Remark:} The point of (ii) is that when the decomposition group of $\mathfrak{t}$ is not normal in $\mathrm{Gal}(M/K)$, the prime $\mathfrak{t}$ need not split completely in the decomposition field, which is $L$ here. A concrete example for (ii) can be given when
\[
    K = \mathbb{Q}, \quad L = \mathbb{Q}(\sqrt[3]{2}), \quad M = \mathbb{Q}(\sqrt[3]{2}, \zeta_3).
\]
By the Chebotarev density theorem, or by explicit computation, you can find $\mathfrak{t}$ such that $(\mathfrak{t}, M/K)$ is the unique nontrivial element of $\mathrm{Gal}(M/L)$. Then all the conditions of (ii) are satisfied.
\end{problem}

\begin{solution}
% leave blank or insert solution later
\end{solution}

\bigskip

\begin{problem}[Neukirch Ch.~I.9, Exercise 3]
Continue the general setup from Problem~2. Assume the following:
\begin{enumerate}
    \item[(i)] $L/K$ is solvable, meaning that $\mathrm{Gal}(M/K)$ is a solvable group. (We are not assuming $M=L$.)
    \item[(ii)] $p = [L:K]$ is a prime number.
\end{enumerate}

Now let $\mathfrak{p}$ be a prime of $K$ unramified in $L$. If there are two primes $\mathfrak{q}$ and $\mathfrak{q}'$ of $L$ above $\mathfrak{p}$ such that the inertial degrees $f_{\mathfrak{q}}$ and $f_{\mathfrak{q}'}$ are equal to $1$, then show that $\mathfrak{p}$ splits completely in $L/K$.

\textit{Caveat:} The extension degree $p$ has nothing to do with the prime ideal $\mathfrak{p}$ in the problem.

\textit{Hint:} Let $S_p$ denote the symmetric group in $p$ letters acting on $\{1,2,\dots,p\}$. If $G$ is a solvable subgroup of $S_p$ acting transitively on $\{1,2,\dots,p\}$ then every nontrivial element of $G$ fixes at most one element in $\{1,2,\dots,p\}$. (A reference for this fact is given in Neukirch.)
\end{problem}

\begin{solution}
% leave blank or insert solution later
\end{solution}



\end{document}