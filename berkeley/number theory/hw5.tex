\documentclass[12pt]{article}  % or any other class
\usepackage{/Users/songye03/Desktop/Math_tex/style/psetconfig}         % loads your custom style
\title{Homework 1}
\author{Songyu Ye}
\date{\today}


\begin{document}
\psettitle

\begin{problem}[Problem 1]
Let $\zeta_n$ denote a primitive $n$-th root of unity (so that powers of $\zeta_n$ give all $n$-th roots of unity). Consider 
\[
L = \mathbb{Q}(\zeta_n) \quad \text{over} \quad K = \mathbb{Q}.
\]
This is a Galois extension and there is an isomorphism (``canonical'')
\[
i : \mathrm{Gal}(\mathbb{Q}(\zeta_n)/\mathbb{Q}) \;\;\widetilde{\longrightarrow}\;\; (\mathbb{Z}/n\mathbb{Z})^{\times}
\]
characterized by the equation that $\sigma(\zeta_n) = \zeta_n^{i(\sigma)}$ for all $\sigma \in \mathrm{Gal}(\mathbb{Q}(\zeta_n)/\mathbb{Q})$. 

Now let $p$ be a prime number coprime to $n$. You may accept that $p$ is unramified in $\mathbb{Q}(\zeta_n)/\mathbb{Q}$.

\begin{enumerate}
    \item[(i)] Prove that the Frobenius element 
    \[
    (p, \mathbb{Q}(\zeta_n)/\mathbb{Q})
    \]
    maps to $p \in (\mathbb{Z}/n\mathbb{Z})^{\times}$ under the map $i$.
    
    \item[(ii)] Using (i) show that $p$ splits completely in $\mathbb{Q}(\zeta_n)$ if and only if $p \equiv 1 \pmod{n}$.
    
    \item[*] Bonus: Can you describe the condition for $p$ to be inert in $\mathbb{Q}(\zeta_n)$?
\end{enumerate}
\end{problem}

\begin{problem}[Problem 2]
Assume that $n = q$ is a prime such that $q \equiv 1 \pmod{4}$. Recall there is a canonical isomorphism
\[
i : \mathrm{Gal}(\mathbb{Q}(\zeta_q)/\mathbb{Q}) \;\;\widetilde{\longrightarrow}\;\; (\mathbb{Z}/q\mathbb{Z})^{\times}
\]
sending the Frobenius element $(p, \mathbb{Q}(\zeta_q)/\mathbb{Q})$ to $p \in (\mathbb{Z}/q\mathbb{Z})^{\times}$ for every $p \neq q$. 
Take on faith that $\mathbb{Q}(\sqrt{q}) \subset \mathbb{Q}(\zeta_q)$. Now fix an \textbf{odd} prime $p \neq q$. 

\begin{enumerate}
    \item[(i)] Verify that $p$ is a square modulo $q$ if and only if $(p, \mathbb{Q}(\zeta_q)/\mathbb{Q})$ fixes the subfield $\mathbb{Q}(\sqrt{q})$ elementwise.
    
    \item[(ii)] Check that $(p, \mathbb{Q}(\zeta_q)/\mathbb{Q})$ fixes the subfield $\mathbb{Q}(\sqrt{q})$ elementwise if and only if $p$ splits completely in $\mathbb{Q}(\sqrt{q})$.
    
    \item[(iii)] Deduce from (i), (ii), and Problem Set 03 \#3 that $p$ is a square modulo $q$ if and only if $q$ is a square modulo $p$, namely
    \[
    \left( \frac{p}{q} \right) \left( \frac{q}{p} \right) = 1.
    \]
\end{enumerate}

\noindent
\textit{Note: Please refrain from using quadratic reciprocity since the point is to give a Galois-theoretic proof of quadratic reciprocity.}

\medskip

\noindent
* Bonus: When $q \equiv 3 \pmod{4}$, a similar argument with $\mathbb{Q}(\sqrt{-q})$ in place of $\mathbb{Q}(\sqrt{q})$ shows that
\[
\left( \frac{p}{q} \right)\left( \frac{-q}{p} \right) = 1.
\]
\end{problem}

\begin{problem}[Problem 3]
Do Lang's Algebra, Exercises VI.46, VI.47, and VI.48, pp.\ 330--331. Submit your solutions only for VI.47 and VI.48.

\medskip
\noindent
(Please do VI.46 but it's a private exercise. Note: These exercises will prepare us for the Witt vectors section [S] VI.6.)
\end{problem}
\end{document}