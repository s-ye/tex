\documentclass[12pt]{article}  % or any other class
\usepackage{/Users/songye03/Desktop/Math_tex/style/psetconfig}         % loads your custom style
\title{Homework 2}
\author{Songyu Ye}
\date{\today}

\newenvironment{lemma}{\begin{mdframed}[backgroundcolor=gray!20] \textbf{Lemma}}{\end{mdframed}}

\begin{document}
\psettitle

\begin{problem}[1 (from RS1)]
$\Delta$ is the unit disk, $\Delta^\times = \Delta \setminus \{0\}$.
\begin{enumerate}
  \item Prove that a holomorphic map $f:\Delta^\times \to \C$ which has an \emph{essential} (non-pole) singularity at $0$ has dense image in $\C$.
  \item Use this to show that any map $f:\Delta^\times \to \P$ which is never more than $N$-to-$1$, for a fixed number $N$, extends holomorphically to $\Delta$.
  \item Generalize (b) to the case when the target is an arbitrary compact Riemann surface $R$, by invoking Riemann’s theorem which guarantees the existence of meromorphic functions on $R$.
\end{enumerate}
\emph{Remark.} A much stronger (and more difficult) version of (a) says that $f$ assumes every value infinitely often, possibly with a single exception (such as $0$, for $e^{1/z}$). This is the Great Picard Theorem.
\end{problem}

\begin{solution}
\begin{enumerate}
  \item Let $f:\Delta^\times\to\C$ have an essential (non-pole) singularity at 0. If the image is not dense, there is a disc $D(a,r)\subset\C$ that $f$ misses near 0. Then $g(z)=\frac{1}{f(z)-a}$ is holomorphic and $|g(z)|\le r^{-1}$ near 0, hence extends holomorphically to 0 (Riemann's removable singularity theorem). If $g(0)\ne0$, then $f=a+1/g$ extends holomorphically across 0 (removable singularity). If $g(0)=0$, then $1/g$ has a pole at 0, so $f$ has a pole. Either way, the singularity at 0 is not essential. Contradiction. Hence the image of every punctured neighborhood is dense in $\C$.

  \item Assume toward a contradiction that 0 is an essential singularity. Work in the affine chart $\C\subset\P^1$, and fix a regular value $a\in\C$ of $f$ (possible since the critical values are discrete). Set $g(z):=f(z)-a$.

  For $r>0$ small with $g$ having no zeros on $|z|=r$, define the index \[n(r):=\frac{1}{2\pi i}\int_{|z|=r}\frac{g'(z)}{g(z)}\,dz\] which equals the number of solutions of $g(z)=0$ in $|z|<r$, counted with multiplicity (by the argument principle). 

\begin{lemma}
For every $M\in\N$ there exists $r_M>0$ such that $n(r_M)\ge M$.
\end{lemma}
Because 0 is essential, Casorati-Weierstrass gives: for every $\varepsilon\in(0,1)$ and every $r_0>0$ there exists $0<r<r_0$ with $\min_{|z|=r}|g(z)|<\varepsilon$ and $\max_{|z|=r}|g(z)|>\varepsilon^{-1}$. (If not, then on all small circles $|g|$ stays in a compact annulus, and a standard maximum-minimum argument would force $g$ to be bounded away from 0 near 0, making $1/g$ holomorphic there—contradicting that 0 is essential for $g$.)

  Fix $\varepsilon\in(0,1)$ so small that the circle $\{|w|=\varepsilon\}$ contains no critical values of the map $g$ from $|z|=r$ (this is possible by discreteness). Using (*) with that $\varepsilon$, choose $r$ so that along the circle $|z|=r$ the continuous curve $w(t):=g(re^{it})$ intersects $|w|=\varepsilon$ transversely many times and also intersects $|w|=\varepsilon^{-1}$. By continuity, we can arrange $2M$ alternating crossings of $|w|=\varepsilon$ as $t$ runs from 0 to $2\pi$ (inside/outside alternate because $|g|$ attains both $<\varepsilon$ and $>\varepsilon^{-1}$ values on the same circle).

  Each such alternating pair forces the argument of $w(t)$ to increase by at least $2\pi$ around the origin (the curve must go from inside to outside and back, swinging around 0 once; regularity of the crossings and the fact $a$ is a regular value ensure positive orientation). Hence the total change of $\arg g(re^{it})$ over $t\in[0,2\pi]$ is at least $2\pi M$. Therefore the winding number of $g(|z|=r)$ about 0 is $\ge M$, i.e. $n(r)\ge M$. $\square$

  With the Lemma, fix $M:=N+1$. Choose $r$ with $n(r)\ge M$. Then $g(z)=0$ has at least $M=N+1$ solutions in $|z|<r$. That is, the single value $a$ has at least $N+1$ preimages in $\Delta^\times$, contradicting that $f$ is never more than $N$-to-1.

  Thus 0 cannot be essential. The remaining possibilities for a holomorphic map to $\P^1$ are: removable singularity or pole; in either case $f$ extends holomorphically across 0.
  \item Let $g:R \to \P^1$ be a nonconstant meromorphic function on the compact Riemann surface $R$. Let $f:\Delta^\times \to R$ be a holomorphic map which is never more than $N$-to-1. Then $h:=g\circ f:\Delta^\times \to \P^1$ is also never more than $Nd$-to-$1$, where $d$ is the degree of $g$. By (b), $h$ extends holomorphically to $\Delta$.
\end{enumerate}
\end{solution}

\begin{problem}[2 ]
Identify successive pairs of edges of a $2n$-gon, labelled $a,a,b,b,c,c,\dots$, by matching points on matching edge pairs in \emph{parametric order}. (Equivalently, identify the points $\theta$ and $\theta+\pi/n$ on the boundary of the unit disk.) 

Explain why the surface obtained is homeomorphic to the one obtained by sewing on $n$ M{\"o}bius strips to an $n$-holed sphere, along matching boundaries. 

Which of these gives a Klein bottle?
\end{problem}

\begin{solution}
The $2n$-gon with edges $aa\,bb\,cc\cdots$ gives $\#^n \mathbb{RP}^2$. Each $\mathbb{RP}^2$ is "sphere with 1 hole + Möbius band." Taking the connected sum of $n$ such surfaces glues the sphere pieces into a sphere with $n$ holes, and the Möbius bands remain attached.

The case $n=2$ gives a Klein bottle. The polygon for $\R\P^2 \# \R\P^2$ has sides $aabb$. The polygon for the Klein bottle has sides $aba^{-1}b$. We want to show they represent the same surface. By cutting and re-gluing along the diagonal, we can transform the $aabb$ polygon into the $aba^{-1}b$ polygon, showing they are homeomorphic.
\end{solution}

\begin{problem}[3 (from RS2)]
Show that any degree $2$ holomorphic map $f:\C/L \to \P$ is a “M{\"o}bius transform of a shifted $\wp$–function":
\[
  f(u) = \frac{a\wp(u-w)+b}{c\wp(u-w)+d}, \qquad a,b,c,d,w \in \C.
\]
\emph{Comment.} You may assume standard facts about M{\"o}bius transformations.
\end{problem}

\begin{solution}
  Because $\deg f=2$, for a generic value $y\in\P^1$ the fiber $f^{-1}(y)=\{u_1,u_2\}$. Define $\tau(u_1)=u_2$ and $\tau(u_2)=u_1$. Standard covering theory shows: $\tau:E\to E$ is a holomorphic involution ($\tau^2=\mathrm{id}$), $f\circ\tau=f$, and the branch points are the fixed points of $\tau$ (there are 4 of them).

  Lift $\tau$ to $\widetilde\tau:\C\to\C$ with $\widetilde\tau(z+L)\equiv\tau(z)+L$. Any holomorphic self-map of $\C$ that descends to the torus has the form $\widetilde\tau(z)=az+b$, where $aL\subseteq L$, $|a|=1$. Since $\tau^2=\mathrm{id}$, we have $a^2=1\Rightarrow a=\pm1$. A degree-2 branched covering must have fixed points, forcing $a=-1$. Hence $\widetilde\tau(z)=-z+t$ with $2t\in L$. Passing to $E$, $\tau$ is the map $u\mapsto -u+w$ where $2w\equiv0$ in $E$.

  Now translate the torus by $w$: define $T_w(u)=u-w$ and replace $f$ by $g:=f\circ T_w$. Then the deck involution becomes $u\mapsto -u$, so $g$ is even: $g(u)=g(-u)$.

  Let $\wp$ be the Weierstrass $\wp$-function for $L$. It is even, has a double pole at 0, and no other poles in a period parallelogram. Every even elliptic function $h$ is a rational function of $\wp$: $h(u)=R(\wp(u))$ for some rational $R\in\C(x)$. This is because the poles of an even elliptic function occur in $\{\pm a_j\}$ with even principal parts. Subtract a polynomial $P(\wp)$ that matches all principal parts at $\pm a_j$; the difference is then an even elliptic function with no poles, hence constant. So $h=P(\wp)+\text{const}=R(\wp)$. Thus, for our $g$ there is $R\in\C(x)$ with $g(u)=R(\wp(u))$.

  The map $\wp:E\to\P^1$ has degree 2 so $\deg(g)=\deg(R\circ\wp)=\deg(R)\cdot \deg(\wp)=\deg(R)\cdot 2$. But $\deg(g)=\deg(f)=2$. Therefore $\deg(R)=1$ and so $R$ is a Möbius transform: \[R(x)=\frac{ax+b}{cx+d}\] with $ad-bc\neq0$. Undoing the translation $T_w$, we get $f(u)=\frac{a\wp(u-w)+b}{c\wp(u-w)+d}$ where $a,b,c,d,w\in\C$, $ad-bc\neq0$.
\end{solution}

\begin{problem}[4 (from RS2)]
Prove that any two meromorphic functions $f,g$ on a compact Riemann surface are \emph{algebraically related}: $P(f,g)\equiv 0$ for some $2$-variable polynomial $P$.

\emph{Hint.} Recall that a meromorphic function without poles must be constant, and estimate, in terms of $N$, the dimension of the vector space spanned by the functions $f^m g^n$, for $0\le m,n\le N$, to conclude that a linear dependence relation must hold for large $N$.
\end{problem}

\begin{solution}
  Let the pole divisors of $f$ and $g$ be
\[
(f)_\infty=\sum_{i=1}^r a_i p_i,\qquad (g)_\infty=\sum_{i=1}^r b_i p_i,
\]
where $a_i,b_i\ge 0$ and the $p_i$ are distinct points of $X$ (allowing some $a_i$ or
$b_i$ to be $0$ if only the other function has a pole there).
Set $A=\sum_i a_i$ and $B=\sum_i b_i$.
If $A=0$ or $B=0$, the corresponding function is holomorphic on $X$ and hence
constant, so the conclusion is trivial. Thus assume $A,B>0$.

For $m,n\ge 0$ put $h_{m,n}:=f^m g^n$.
Then $h_{m,n}$ has poles only at the $p_i$, with
\[
\ord_{p_i}(h_{m,n})\ge -(ma_i+nb_i),\qquad
\deg (h_{m,n})_\infty=\sum_i\max\{0,-\ord_{p_i}(h_{m,n})\}\le mA+nB.
\]
Fix $N\in\mathbb N$ and consider the vector space
\[
V_N:=\operatorname{span}_\C\{\,h_{m,n}: 0\le m,n\le N\,\}.
\]
All functions in $V_N$ lie in the space
\[
L(D_N),\qquad D_N:=N\sum_{i=1}^r (a_i+b_i)\,p_i,
\]
i.e.\ meromorphic functions with poles only at the $p_i$ and of order at most
$N(a_i+b_i)$ at $p_i$.

\begin{lemma}
  $\dim L(D_N) \leq 1 + \deg D_N = 1 + N(A+B)$.
\end{lemma}
To see this, note that the principal parts up to order $k_i$ at each $p_i$; these give at most $\sum_i k_i$ linear parameters, and adding a constant gives +1.

But there are $(N+1)^2$ monomials $h_{m,n}$ with $0\le m,n\le N$.
For $N$ large we have $(N+1)^2>1+N(A+B)$, hence the family
$\{h_{m,n}\}_{0\le m,n\le N}$ is linearly dependent:
there exist coefficients $c_{m,n}$, not all zero, such that
\[
\sum_{m,n=0}^{N} c_{m,n}\, f^m g^n \equiv 0 \quad\text{on } X.
\]
\end{solution}

\begin{problem}[5]
\begin{enumerate}
  \item Specializing the period lattice to the limiting case $\omega_1 = \pi$, $\omega_2 \to i\cdot\infty$, show that
  \[
    \wp(u) \to \cot^2(u) + \tfrac{2}{3}, \qquad
    \zeta(u) \to \cot(u)+u, \qquad
    \sigma(u) \to \sin(u)\cdot \exp(u^2/2).
  \]
  \item Do the series expansions apply?
  \item Find and check the differential equation expressing $(\wp')^2$ in terms of $\wp$ in this limit.
  \item Describe the (singular) analytic set in $\C^2$ parametrized as $z=\wp(u), w=\wp'(u)$.
\end{enumerate}
\end{problem}

\begin{solution}
  \begin{enumerate}
    \item Recall that we defined the Weierstrass functions $\zeta$ function and $\sigma$ function by
  \[\wp(u)=-\zeta'(u), \quad \zeta(u)=-\zeta(-u)\]
  \[\sigma(u) = \exp\left(\int_{u_0}^u \zeta(t)\,dt\right), \quad \sigma'(0) = 1\]

Let the lattice be $L=\{m\omega_1+n\omega_2:m,n\in\mathbb Z\}$ with $\omega_1=\pi$ and $\omega_2=iT$ where $T\to\infty$. A generic lattice point is $\omega_{m,n}=m\pi+i nT$.

Fix a compact set $K\subset\C$ and write $R=\sup_{u\in K}|u|$.
For $n\neq 0$ and $T$ large, $|\omega_{m,n}|\ge |n|T-|m|\pi$, so in particular $|\omega_{m,n}| \ge \tfrac12 |n|T$ and also $|u-\omega_{m,n}|\ge \tfrac12|\omega_{m,n}|$ (since $|u|\le R$ is bounded while $|\omega_{m,n}|\to\infty$ with $T$). Consider one summand in the partial–fraction expansion:
\[
S_{m,n}(u) := \frac{1}{(u-\omega_{m,n})^2}-\frac{1}{\omega_{m,n}^2}
\]
We have the identity:
\[
\frac{1}{(u-\omega)^2}-\frac{1}{\omega^2} = \frac{(2\omega u-u^2)}{(u-\omega)^2\,\omega^2}
\]
Hence, for $u\in K$:
\[
|S_{m,n}(u)| \le \frac{2|\omega_{m,n}|\,|u|+|u|^2}{|u-\omega_{m,n}|^2\,|\omega_{m,n}|^2} \le \frac{2R|\omega_{m,n}|+R^2}{(\tfrac12|\omega_{m,n}|)^2\,|\omega_{m,n}|^2} \le \frac{C_R}{|\omega_{m,n}|^3}
\]
for a constant $C_R$ depending only on $R$.

Therefore:
\[
\sum_{\substack{(m,n)\in\mathbb Z^2\\ n\ne 0}} |S_{m,n}(u)| \le C_R \sum_{n\ne 0}\sum_{m\in\mathbb Z}\frac{1}{|m\pi+inT|^3} = C_R \sum_{n\ne 0}\sum_{m\in\mathbb Z}\frac{1}{\big((m\pi)^2+(nT)^2\big)^{3/2}}
\]
For fixed $n\neq 0$, the inner sum over $m$ is $O\big((nT)^{-2}\big)$ (compare with $\int_{\mathbb R}\frac{dx}{(x^2+(nT)^2)^{3/2}}=\frac{2}{(nT)^2}$). Thus:
\[
\sum_{m\in\mathbb Z}\frac{1}{\big((m\pi)^2+(nT)^2\big)^{3/2}} \le \frac{C}{(nT)^2}
\]
with $C$ independent of $n,T$. Summing over $n\neq 0$ gives:
\[
\sum_{n\ne 0}\sum_{m\in\mathbb Z}\frac{1}{\big((m\pi)^2+(nT)^2\big)^{3/2}} \le \frac{C}{T^2}\sum_{n\ne 0}\frac{1}{n^2} = \frac{C'}{T^2} \xrightarrow[T\to\infty]{} 0
\]
This convergence is uniform in $u\in K$ because our bound does not depend on $u$ beyond $R$. Hence the total contribution to $\wp(u)$ from all terms with $n\neq 0$ tends to 0 uniformly on compact sets. The only nonvanishing terms in the partial–fraction sum are those with $n=0$, i.e. $\omega=m\pi$ with $m\in\mathbb Z\setminus\{0\}$. 

Hence
  \[\wp(u)\longrightarrow
  \frac1{u^2}+\sum_{m\ne0}\Big(\frac{1}{(u-m\pi)^2}-\frac1{(m\pi)^2}\Big)\]
  Recall the classical partial fractions
  $\csc^2 u=\frac1{u^2}+\sum_{m\ne0}\frac{1}{(u-m\pi)^2}$, 
  $\sum_{m\ne0}\frac1{(m\pi)^2}=\frac13$,
  so \[\wp(u) \longrightarrow \csc^2 u-\frac13 = \cot^2 u+\frac23\]

  Recall Weierstrass's product for $\sigma$ (for the lattice $L=\langle 2\omega_1,2\omega_2\rangle$):
  \[
  \sigma(u)
  =u\prod_{\omega\in L\setminus\{0\}}
  \Big(1-\frac{u}{\omega}\Big)\,
  \exp\!\Big(\frac{u}{\omega}+\frac{u^2}{2\omega^2}\Big).
  \]

  Now take the trigonometric degeneration $\omega_1=\pi$ fixed and $\omega_2\to i\infty$. All lattice points with nonzero vertical component ($n\neq 0$) go off to infinity and their factors tend to 1. What's left is the product over the horizontal periods $\omega=m\pi$, $m\in\mathbb Z\setminus\{0\}$. Thus
  \[
  \sigma(u)\ \longrightarrow\
  u\prod_{m\neq 0}\Big(1-\frac{u}{m\pi}\Big)
  \exp\!\Big(\frac{u}{m\pi}+\frac{u^2}{2m^2\pi^2}\Big).
  \]

  Pair the terms for $m$ and $-m$. Using the standard product for $\sin u$,
  \[
  \sin u
  = u\prod_{m=1}^\infty\Big(1-\frac{u^2}{m^2\pi^2}\Big),
  \]
  and the elementary identity
  \[
  \prod_{m=1}^\infty
  \exp\!\Big(\frac{u^2}{m^2\pi^2}\Big)
  =\exp\!\Big(\frac{u^2}{2}\Big)
  \quad(\text{telescopes after pairing }m\text{ and }-m),
  \]
  you get (up to a nonzero constant fixed by $\sigma'(0)=1$):
  \[
  \sigma(u)\ \longrightarrow\ \sin u\;\exp\!\Big(\frac{u^2}{2}\Big).
  \]

  Now differentiate   $\log\sigma(u)$ to get $\zeta(u)$:
  \[
  \zeta(u)=(\log\sigma)' \ \longrightarrow\
  (\log\sin u)' + \Big(\frac{u^2}{2}\Big)'
  = \cot u + u.
  \]

    \item Yes. We showed that the series converge uniformly on compact sets in the limit, and they converge to the series expansions with no $n\neq 0$ terms. We saw that the resulting sums are exactly the series expansions of the respective trigonometric functions.

 \item Put $X=\wp(u)$ in the limit $X=\cot^2 u+\tfrac23$.
  Then we calculate 
  \[\wp'(u)=\frac{d}{du}(\cot^2 u)= -2\cot u\,\csc^2 u\]
  \[(\wp')^2=4\cot^2 u\,\csc^4 u\]
  Use $\csc^2 u=\cot^2 u+1$ to express in $X$:
  \begin{align*}
    \cot^2 u=X-\tfrac23, \quad \csc^2 u=X+\tfrac13,
  \end{align*}
  hence
  $(\wp')^2=4\,(X-\tfrac23)(X+\tfrac13)^2
  =4X^3-\frac{4}{3}X-\frac{8}{27}$.
  So we find that
  \[(\wp')^2=4\wp^3-\frac{4}{3}\wp-\frac{8}{27}\]
\item The analytic set in $\C^2$ parametrized by $z=\wp(u), w=\wp'(u)$ is given by the cubic equation
  \[w^2=4z^3-\frac{4}{3}z-\frac{8}{27}
  =4\,(z-\tfrac{2}{3})(z+\tfrac{1}{3})^2\]
  Its discriminant is $g_2^3-27g_3^2=0$, so the curve is singular.
  \end{enumerate}
\end{solution}
\end{document}