\documentclass[12pt]{article}  % or any other class
\usepackage{/Users/songye03/Desktop/Math_tex/style/psetconfig}         % loads your custom style
\title{Homework 3}
\author{Songyu Ye}
\date{\today}


\begin{document}
\psettitle


\begin{problem}[1]
Which of the following is a Galois cover of the complex $z$-plane?
\begin{enumerate}[(a)]
    \item $w^2 = 4z^3 - g_2z - g_3$;
    \item $w^n - z^n = 1$;
    \item $w^3 + z + z^2 = w^2 + wz$; \textit{Hint: look at the fiber over $0$.}
    \item $w^2 - 2zw + z^3 = 1$.
\end{enumerate}
\end{problem}

\begin{solution}
    \begin{enumerate}
        \item The Riemann surface is degree two over the $z$-plane, and one sees by inspection that $w\mapsto -w$ is a deck transformation. So the cover has Deck group at least the size of the degree, hence is Galois.  
	 \item The Riemann surface is degree $n$ over the $z$-plane, and one sees by inspection that $w\mapsto \zeta w$ ($\zeta^n=1$) is a deck transformation. So the cover has Deck group at least the size of the degree, hence is Galois.  
    \item Look at the fiber over $z=0$. The point $(0,0)$ has valency $2$ with respect to the projection and hence the cover must fix it. In particular any deck transformation must also fix $(1,0)$ which is a point of valency $1$. So it must be the identity and the cover is not Galois.
    \item Set $y = w - z$. Then the equation becomes $y^2 = 1 + z^2 - z^3$ is a degree two cover of the $z$-plane, and one sees by inspection that $y \mapsto -y$ is a deck transformation. So the cover has Deck group at least the size of the degree, hence is Galois.
    \end{enumerate}
\end{solution}

\begin{problem}[2]
Let $V$ be a rank $2$ (for simplicity) vector bundle over a Riemann surface $R$. Assume that $V$ has two meromorphic sections $s_1, s_2$ which, at some point, are holomorphic and span the fiber.
\begin{enumerate}[(a)]
    \item Show that this will be the case everywhere except at a set of isolated points.
    \item At an exceptional point, show that we can modify $V$ by a finite sequence of elementary transformations so that $s_1$ and $s_2$ form a holomorphic frame of the new bundle.
\end{enumerate}
\textit{Suggestion:} First make the sections holomorphic, then find some numerical measure for their failure to give a basis. Then find a way to reduce that number.\\
\textit{Remark:} The argument generalizes to any dimension. If $R$ is compact, it follows that we can trivialize $V$ by a finite number of elementary transformations. If $R$ is non-compact, one can show that every vector bundle is in fact trivial.
\end{problem}

\begin{solution}
    Let $s_1, s_2$ be two meromorphic sections of a rank 2 vector bundle $V$ over a Riemann surface $R$. Since $V$ is a holomorphic vector bundle, there exists a local trivialization of $V$ around $p$. \begin{align*}
        V|_U \cong \cO_U e_1 \oplus \cO_U e_2
    \end{align*} and we can write \begin{align*}
        s_1 = f_1 e_1 + f_2 e_2, \quad s_2 = g_1 e_1 + g_2 e_2
    \end{align*} where $f_i, g_i$ are meromorphic functions on $U$. The failure of $s_1, s_2$ to span the fiber at a point $q \in U$ is given by the vanishing of the determinant \begin{align*}
        D(q) = f_1(q)g_2(q) - f_2(q)g_1(q).
    \end{align*} which is a meromorphic function on $U$. The zeroes of a meromorphic function are isolated unless the function is identically zero. Since $s_1, s_2$ span the fiber at $p$, $D$ is not identically zero. Therefore, the set of points where $s_1, s_2$ fail to be holomorphic or fail to span the fiber is a discrete set of isolated points in $R$, because meromorphic functions can only have isolated singularities and the determinant $D$ is meromorphic.

    Let $D$ be the effective divisor of the poles of $s_1, s_2$. We can make $s_1, s_2$ holomorphic by twisting $V$ with the line bundle $\cO(D)$, i.e. consider the new vector bundle \begin{align*}
        V(D) = V \otimes \cO(D)\end{align*} Then $s_1, s_2$ are holomorphic sections of $V(D)$. Now consider a point $p$ where $s_1, s_2$ fail to span the fiber of $V(D)$. If $s_1(p)$ and $s_2(p)$ both vanish, then twist by an appropriate power of $\cO(-p)$ to make at least one of them non-vanishing at $p$, say $s_1(p) \neq 0$. In a chart near $V(D)$ we have a local trivialization $V(D)|_U \cong \cO_U e_1 \oplus \cO_U e_2$ so that $s_1 = e_1$ and $s_2 = f(z)e_1 + g(z)e_2$ for some holomorphic functions $f(z), g(z)$. Let $L = \C e_1 \subset V_p$. We can perform an elementary transformation of $V(D)$ at $p$ with respect to $L$ to obtain a new vector bundle $V'$ which fits into the short exact sequence of coherent sheaves \begin{align}
        0 \to V' \to V(D) \to (V(D)_p/L) \otimes \cO_p \to 0.
    \end{align} 
    The wedge product of the sections is given by \begin{align*}
        s_1 \wedge s_2 = g(z) e_1 \wedge e_2.
    \end{align*} Since $s_1, s_2$ fail to span the fiber at $p$, we have $g(0) = 0$, so we can write $g(z) = z^n h(z)$ for some $n \geq 1$ and unit $h(0) \neq 0$. After absorbing the unit $h(z)$ into $e_2$, we can assume $g(z) = z^n$. Then we have in local coordinates sections $s_1 = e_1$ and $s_2 = f(z) e_1 + z^n e_2$. 
    
    The elementary transformation $V'$ is locally generated by the sections $s_1' = e_1$ and $s_2' = z e_2$. This is because $V'(U)$ consists of sections of $V(D)(U)$ whose value at $p$ lies in $L = \C e_1$. Any section of $V(D)(U)$ can be written as $a(z)e_1 + b(z)e_2$ for some holomorphic functions $a(z), b(z)$. The condition that the value at $p$ lies in $L$ means that $b(0) = 0$, so we can write $b(z) = z c(z)$ for some holomorphic function $c(z)$. Therefore, sections of $V'(U)$ are of the form \begin{align*}
        a(z)e_1 + z c(z)e_2, \quad a(z), c(z) \in \cO_U
    \end{align*} which means $V'(U)$ is a $\cO_U$-module freely generated by $e_1$ and $z e_2$. In particular, the bundle $V'$ is locally trivialized by the sections $e_1$ and $e_2' = z e_2$. In the new bundle $V'$, the sections $s_1$ and $s_2$ have wedge product \begin{align*}
        s_1' \wedge s_2' = z^{n-1} e_1 \wedge e_2'.
    \end{align*} Thus, the order of vanishing of the wedge product at $p$ has decreased by 1. By repeating this process a finite number of times, we can obtain a vector bundle where $s_1, s_2$ span the fiber at $p$. By performing this procedure at each point where $s_1, s_2$ fail to span the fiber, we can obtain a vector bundle where $s_1, s_2$ form a holomorphic frame everywhere.
\end{solution}

\begin{problem}[3]
    \leavevmode
    \begin{enumerate}[(a)]
        \item Consider the vector bundle $V$ with sheaf of sections $\mathcal{O}(n_1) \oplus \cdots \oplus \mathcal{O}(n_k)$ over $\mathbb{P}^1$, with $n_1 \le \cdots \le n_k$. Show that the sequence of integers $n_i$ is uniquely determined by $V$.
        \item In contrast with (a), show that $\mathcal{O}(1)\oplus\mathcal{O}(-1)$ and $\mathcal{O}\oplus\mathcal{O}$ are isomorphic as topological vector bundles.
        \item Show that there is a holomorphic automorphism of $V$ which takes the vector $[1,0,\dots,0]$ in the fiber over $0$ to $[1,1,\dots,1]$.
        \item Assuming the fact that every rank $k$ holomorphic vector bundle on $\mathbb{P}^1$ can be constructed from $\mathcal{O}^{\oplus k}$ by elementary transformations, show that it must be isomorphic to one of the form in (a).
    \end{enumerate}
\end{problem}

\begin{solution}
    \begin{enumerate}[(a)]
        \item Suppose also $V\simeq \bigoplus_{j=1}^k \mathcal O(m_j)$ with $m_1\le\cdots\le m_k$. Recall \[\operatorname{Hom}(\mathcal O(a),\mathcal O(b)) \cong H^0(\mathcal O(b-a))\] which is $0$ if $b<a$ and nonzero if $b\ge a$.

        Consider the $r$ maps $\phi_r$ given by the composition
        \begin{align*}
            \cO(n_k) \hookrightarrow V \cong \bigoplus_{j=1}^k \cO(m_j) \twoheadrightarrow \cO(m_r)
        \end{align*}
        At least one composite
        $\mathcal O(n_k)\to \mathcal O(m_r)$ is nonzero (if not then the inclusion would be zero), forcing $m_r\ge n_k$. With the orderings this gives $m_k\ge n_k$. By symmetry (swap the roles of $n,m$) we also get $n_k\ge m_k$. Hence $n_k=m_k$. Cancel that summand and argue by induction on $k$. 
        \item Complex vector bundles of rank $2$ on $\P^1\cong S^2$ are topologically classified by homotopy classes of maps from $S^2$ to the classifying space $BU(2)$. Since \[\pi_2(BU(2))\cong \pi_1(U(2))\cong \Z\] the isomorphism classes of rank $2$ complex vector bundles on $S^2$ are classified by an integer, which is the first Chern class of the bundle. The first Chern class is additive under direct sum, and $c_1(\mathcal O(n)) = n$. Thus, both $\mathcal O(1)\oplus \mathcal O(-1)$ and $\mathcal O\oplus \mathcal O$ have first Chern class $0$, so they are isomorphic as topological vector bundles.
        \item Order $n_1\le\cdots\le n_k$. For $i>1$ pick the constant section $s_{i1}\in H^0\!\big(\mathcal O(n_i-n_1)\big)$ equal to 1 (exists since $n_i-n_1\ge 0$). View $s_{i1}$ as a morphism $\mathcal O(n_1)\to \mathcal O(n_i)$.

        Define an endomorphism of $V$ by the lower-triangular matrix with ones on the diagonal and $s_{i1}$ in the $(i,1)$-entry, zeros elsewhere. It is clearly an automorphism. On the fiber over 0 it sends $[1,0,\dots,0]$ to $(1,s_{21}(0),\dots,s_{k1}(0))=(1,1,\dots,1)$.
        \item Let $L$ be the span of the first basis vector in the fiber of $\mathcal O(n_1)$ at $p$. On sections near $p$,
        $0\to \ker\big(\mathcal O(n_1)\xrightarrow{\mathrm{ev}_p}\mathbb C_p\big)\cong \mathcal O(n_1-1)\to \mathcal O(n_1)\to \mathbb C_p\to 0$,
        while the other summands are untouched. Therefore
        $\mathrm{Elm}_{p,L}\!\left(\bigoplus_i \mathcal O(n_i)\right)\;\cong\; \mathcal O(n_1-1)\oplus \mathcal O(n_2)\oplus\cdots\oplus \mathcal O(n_k)$. Dually, the inverse elementary transform raises one chosen summand by $+1$.
        Hence any sequence of elementary transforms starting from $\mathcal O^{\oplus k}$ stays split and merely adjusts integers on the summands by $\pm1$. After reordering to $n_1\le\cdots\le n_k$ you land in a bundle of the form in (a).

    \end{enumerate}
\end{solution}

\begin{problem}[4]
Show that on a compact Riemann surface $R$ of genus $g$ and a line bundle $L$ of degree $> 2g-2$, we have $H^1(R; \mathcal{O}(L)) = 0$. Find a counterexample to this if $L$ is a vector bundle instead.\\
\textit{Remark:} For noncompact Riemann surfaces, $H^1$ vanishes for any vector bundle.
\end{problem}

\begin{solution}
    Serre duality gives $H^1(R; \mathcal{O}(L)) \cong H^0(R; \mathcal{O}(K \otimes L^{-1}))^*$ where $K$ is the canonical bundle of $R$. Since $\deg(K) = 2g-2$, we have $\deg(K \otimes L^{-1}) = 2g-2 - \deg(L) < 0$. A line bundle of negative degree has no nontrivial global sections, so $H^0(R; \mathcal{O}(K \otimes L^{-1})) = 0$. Therefore, $H^1(R; \mathcal{O}(L)) = 0$.

    Take any $R$ with $g\ge 1$. Pick a line bundle $A$ with $\deg A>2g-2$ (there are plenty). Set $E = \mathcal O \oplus A$. Then $\deg E=\deg A>2g-2$ and \[H^1(R,E) = H^1(R,\mathcal O) \oplus H^1(R,A)\] By the claim above $H^1(R,A)=0$, while $H^1(R,\mathcal O)\cong \mathbb C^{\,g}\neq 0$ for $g\ge 1$. Hence $H^1(R,E)\neq 0$ even though $\deg E>2g-2$.
\end{solution}

\begin{problem}[5]
Prove that every compact Riemann surface of genus $2$ is \textit{hyperelliptic}, meaning that it can be realized as a double (branched) cover of $\mathbb{P}^1$.\\
\textit{Hint:} Use differentials.
\end{problem}

\begin{solution}
    Being genus $2$ means that the space of holomorphic differentials $H^0(R, K)$ is $2$-dimensional. Pick a basis $\omega_1, \omega_2$. Since $K$ has degree $2g-2=2$, the differentials $\omega_i$ each have two zeroes (counted with multiplicity). If they had a common zero, then they would be linearly dependent (since they are sections of a line bundle with a common zero), contradicting the choice of basis. Therefore, $\omega_1$ and $\omega_2$ have no common zeroes.

    Define a meromorphic function $f = \omega_1/\omega_2$. Since $\omega_1$ and $\omega_2$ have no common zeroes, $f$ is well-defined and meromorphic on $R$. The poles of $f$ are the zeroes of $\omega_2$, and the zeroes of $f$ are the zeroes of $\omega_1$. Each differential has two zeroes, so $f$ has two poles and two zeroes, counting multiplicities. Therefore, $f$ is a degree $2$ meromorphic function on $R$, which defines a double cover of $\mathbb{P}^1$ (the Riemann sphere).
\end{solution}


\end{document}