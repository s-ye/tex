\documentclass[12pt]{article}
\usepackage{/Users/songye03/Desktop/Math_tex/style/psetconfig}

\DeclareMathOperator{\res}{res}
\title{Homework 4}
\author{Songyu Ye}
\date{\today}

\begin{document}
\psettitle

\begin{problem}[1]
Prove the Weierstrass gap theorem, which says that for a compact Riemann surface $X$ of genus $g$, at each point there are exactly $g$ gaps in the sequence of possible pole orders of meromorphic functions. (That is, there are exactly $g$ integers $n \geq 1$ such that there is no meromorphic function with a single pole of order $n$ at the given point and holomorphic elsewhere.)
\end{problem}

\begin{solution}
$g=0$ case is clear since we have explicit meromorphic functions on $\mathbb{P}^1$ with a single pole of any order at any point. So we assume $g\geq1$.

A positive integer $n$ is a non-gap at $p$ if there exists a meromorphic function whose only pole is at $p$ and has order exactly $n$. Equivalently, $l(np) = l((n-1)p)+1$. Otherwise $n$ is a gap, i.e. $l(np)=l((n-1)p)$.

For any divisor $D$ and point $p$, $0\leq l(D+p)-l(D)\leq1$, so in particular every $n\geq1$ is either a gap or a non-gap.

Also, for $g\geq1$, $1$ is always a gap: a function with a single simple pole at $p$ would give a degree-1 map $X\to\mathbb{P}^1$, so $X\cong\mathbb{P}^1$ (genus 0), contradiction.

Riemann-Roch and Serre duality immediately imply that every $m\geq 2g-1$ is a non-gap. Hence all gaps lie in $\{1,\dots,2g-1\}$.

Let $G(m)$ be the number of gaps in $\{1,\dots,m\}$. The number of non-gaps in that interval are is $m-G(m)$. Starting from $l(0)=1$ and using that each non-gap increases $l$ by 1 while each gap leaves it unchanged, for any $m\geq1$, which implies that $l(mp) = 1 + (m-G(m))$.

Now take $m\geq 2g$. From above: $l(mp) = m+1-g$. Equating: $m+1-g = 1 + (m-G(m))$, therefore $G(m)=g$ and so there are exactly $g$ gaps.
    \end{solution}

\begin{problem}[2]
When $d > 2g - 2$, show that the Abel-Jacobi map
\[
\operatorname{Sym}^d R \longrightarrow J^d
\]
is a bundle of projective spaces. Equivalently, the holomorphic sections over $R$ of a varying line bundle of degree $d$ are the fibers of a vector bundle over the Jacobian of degree $d$.

\textbf{Suggestion:} Check that the differential of the map is onto at every point, and use this to find local sections that give a local framing of the bundle.

\textbf{Suggestion:} Prove the vanishing of the sheaf cohomologies of $\mathcal M(V)$ and $\mathcal P(V)$ where $\cM(V)$ is the sheaf of meromorphic sections of a vector bundle $V$ and $\mathcal P(V)$ is the sheaf of principal parts of meromorphic sections of $V$.
\end{problem}

\begin{solution}
    $d>2g-2$ implies that for any line bundle $L$ of degree $d$, $H^1(R;L)=0$ by Riemann-Roch and Serre duality. Therefore, the fiber of the Abel-Jacobi map over $L$ is $\mathbb{P}(H^0(R;L))$, which has constant dimension $d-g$. So set theoretically, the Abel-Jacobi map is a bundle of projective spaces.

    It remains to show that the map is holomorphic and locally trivial. Let $D = p_1 + \cdots + p_d$ with distinct points.  
Then
\[
T_D(\Sym^d R) \cong \bigoplus_{i=1}^d T_{p_i} R,
\qquad
T_{[L]}(J^d) \cong H^0(K_R)^\vee
\]
The differential
\[
d(AJ_d) : \bigoplus_i T_{p_i}R \longrightarrow H^0(K_R)^\vee
\]
is given (dually, via Serre duality $H^1(\mathcal O_R)^\vee \cong H^0(K_R)$) by
\[
d(AJ_d)(v_1,\dots,v_d)(\omega) = \sum_{i=1}^d \omega_{p_i}(v_i)
\]
There is a dual map 
\[
d(AJ_d)^\vee : H^0(K_R) \longrightarrow \bigoplus_i K_{R,p_i}
\] which evaluates a holomorphic differential at the points $p_i$.
The map $d(AJ_d)$ is surjective if and only if the dual map $d(AJ_d)^\vee$ is injective, i.e., if and only if there is
no nonzero $\omega \in H^0(K_R)$ vanishes at all $p_i$. But since $\deg D = d > 2g-2 = \deg K_R$, by Riemann-Roch and Serre duality, we conclude that $d(AJ_d)$ is surjective at every point.

By the holomorphic implicit function theorem, a holomorphic map between complex manifolds with surjective differential at a point is locally a projection.  
Thus for each $D_0$, there are local holomorphic coordinates $(z_1,\dots,z_{d-g},w_1,\dots,w_g)$ on $\Sym^d R$ and $(w_1,\dots,w_g)$ on $J^d$ such that
\[
A_d(z,w) = w.
\]
Therefore $A_d$ is a holomorphic submersion whose fibers are compact complex submanifolds of dimension $d-g$.  
Hence $A_d : \Sym^d R \to J^d$ is a holomorphic fiber bundle with fiber $\mathbb P^{d-g}$.
\end{solution}



\begin{problem}[3]
Kodaira’s theorem implies that for any vector bundle $V \to R$ on a compact Riemann surface, there exists a (bundle-dependent) degree $d \in \mathbb{Z}$ such that
\[
H^1(R; \mathcal{O}(V \otimes L)) = 0
\]
for all line bundles $L$ of degree $> d$.

Use this fact, the long exact sequence for cohomology of
\[
\mathcal{O} \to \mathcal{M} \to \mathcal{P},
\]
and the result of Question~4 below to show that the ad hoc definition of cohomology of vector bundles via principal parts computes the genuine sheaf cohomology of vector bundles.
\end{problem}

\begin{solution}
For any $V$ there exists $d_0$ such that for every line bundle $L$ with $\deg L > d_0$, $H^1(R;\mathcal O(V\otimes L)) = 0$. In particular, choose $L = \mathcal O(D)$ for some effective divisor $D$ with $\deg D > d_0$.

For an effective divisor $D$, we have the short exact sequence of sheaves
\[0 \to \mathcal O(V) \to \mathcal O(V(D)) \to \mathcal P_D(V) \to 0,\]
where $\mathcal P_D(V)$ is the sheaf of principal parts of meromorphic sections of $V$ with poles bounded by $D$.
Taking the long exact sequence in cohomology, we have
\[0 \to H^0(R;\mathcal O(V)) \to H^0(R;\mathcal O(V(D))) \to H^0(R;\mathcal P_D(V)) \to H^1(R;\mathcal O(V)) \to H^1(R;\mathcal O(V(D))) \to \cdots\]
Since $\deg D > d_0$, we have $H^1(R;\mathcal O(V(D))) = 0$. Therefore,
\[H^1(R;\mathcal O(V)) \cong \frac{H^0(R;\mathcal P_D(V))}{\operatorname{Im}(H^0(R;\mathcal O(V(D))) \to H^0(R;\mathcal P_D(V)))}.\]
Taking the direct limit over all effective divisors $D$, we have
\[H^1(R;\mathcal O(V)) \cong \varinjlim_D \frac{H^0(R;\mathcal P_D(V))}{\operatorname{Im}(H^0(R;\mathcal O(V(D))) \to H^0(R;\mathcal P_D(V)))}.\]
By the result of Question 4, we have
\[\varinjlim_D H^0(R;\mathcal P_D(V)) \cong H^0(R;\mathcal P(V)),\]
where $\mathcal P(V)$ is the sheaf of principal parts of meromorphic sections of $V$. Therefore,
\[H^1(R;\mathcal O(V)) \cong \frac{H^0(R;\mathcal P(V))}{\operatorname{Im}(\varinjlim_D H^0(R;\mathcal O(V(D))) \to H^0(R;\mathcal P(V)))}.\]
But $\varinjlim_D H^0(R;\mathcal O(V(D)))$ is precisely the space of global meromorphic sections of $V$, i.e., $H^0(R;\mathcal O(V))$. Thus,
\[H^1(R;\mathcal O(V)) \cong \frac{H^0(R;\mathcal P(V))}{\operatorname{Im}(H^0(R;\mathcal M(V)) \to H^0(R;\mathcal P(V)))}.\]
This shows that the ad hoc definition of cohomology of vector bundles via principal parts computes the genuine sheaf cohomology of vector bundles.
\end{solution}

\begin{problem}[4]
Show that, on a compact Hausdorff space $X$, sheaf cohomology commutes with filtered colimits (formerly known as direct limits): if $\mathcal{S} = \varinjlim_{n \to \infty} \mathcal{S}_n$, then
\[
\varinjlim_{n \to \infty} H^q(X; \mathcal{S}_n) = H^q(X; \mathcal{S}).
\]
(Here the indexing set can be any filtered set, not just $\mathbb{N}$.)

Use this to prove that the sheaf of principal parts of meromorphic sections of a vector bundle on a Riemann surface has no $H^1$ or higher cohomology.
\end{problem}

\begin{solution}
    Let $X$ be compact Hausdorff and $\{\mathcal{S}_\alpha\}$ a filtered direct system of sheaves. Let $\mathcal{S} = \varinjlim_\alpha \mathcal{S}_\alpha$ be the colimit in sheaves.

    Take a finite open cover $\mathfrak{U} = \{U_i\}_{i=1}^m$ of $X$ (exists since $X$ is compact). For any $p$,
    \[
    C^p(\mathfrak{U},\mathcal{S}) = \prod_{i_0<\dots<i_p} \mathcal{S}(U_{i_0}\cap\cdots\cap U_{i_p}).
    \]
    We will use a couple of lemmas and justify them at the end of the solution.
    \begin{lemma}[1]
     Filtered colimits are computed stalkwise and commute with finite products.
    \end{lemma}
The lemma implies that the Čech complex for $\mathcal{S}$ is the filtered colimit of the Čech complexes for $\mathcal{S}_\alpha$ since we have the following isomorphisms which commute with the differentials:
    \[
    C^p(\mathfrak{U},\mathcal{S}) = \prod_{i_0<\dots<i_p} \mathcal{S}(U_{i_0}\cap\cdots\cap U_{i_p}) = \prod_{i_0<\dots<i_p} \varinjlim_\alpha \mathcal{S}_\alpha(U_{i_0}\cap\cdots\cap U_{i_p})
    \]
    \[
    \cong \varinjlim_\alpha \prod_{i_0<\dots<i_p} \mathcal{S}_\alpha(U_{i_0}\cap\cdots\cap U_{i_p}) = \varinjlim_\alpha C^p(\mathfrak{U},\mathcal{S}_\alpha).
    \]


    \begin{lemma}[2]
    Filtered colimits of abelian groups are exact, and in particular taking cohomology commutes with $\varinjlim$:
    \[
    \check{H}^q(\mathfrak{U};\mathcal{S}) \cong \varinjlim_\alpha \check{H}^q(\mathfrak{U};\mathcal{S}_\alpha).
    \]
    \end{lemma}
    On a compact Hausdorff space (in particular, on a Riemann surface) Čech cohomology with respect to a good cover computes sheaf cohomology. Hence
    \[
    H^q(X;\mathcal{S}) \cong \varinjlim_\alpha H^q(X;\mathcal{S}_\alpha)
    \]
    for all $q$. 

    For the second part, let $R$ be a Riemann surface and $V$ a holomorphic vector bundle. For each effective divisor $D$ on $R$, let $\mathcal{P}_D(V)$ be the sheaf of principal parts of meromorphic sections of $V$ with poles bounded by $D$.

Observe that $\mathcal{P}_D(V)$ is supported on the finite set $\operatorname{Supp}(D)$, so it is a finite direct sum of skyscraper sheaves. 

\begin{lemma}[3]
    A sheaf on $R$ with finite support has no higher cohomology. In particular, for $q\geq1$,
    \[
    H^q(R;\mathcal{P}_D(V)) = 0,
    \]
\end{lemma}    
    The full principal parts sheaf is the filtered colimit over all effective divisors:
    \[
    \mathcal{P}(V) = \varinjlim_D \mathcal{P}_D(V),
    \]
    directed by $D\leq D'$ if $D$ divides $D'$. Applying the result from the first part:
    \[
    H^q(R;\mathcal{P}(V)) \cong \varinjlim_D H^q(R;\mathcal{P}_D(V)).
    \]
    But each $H^q(R;\mathcal{P}_D(V))=0$ for $q\geq1$, so the colimit is 0. Therefore
    \[
    H^q(R;\mathcal{P}(V)) = 0 \quad \text{for all } q\geq1.
    \]

    Now we justify the lemmas used above.
    \begin{proof}[Proof of Lemma 1]
        Define a presheaf $\mathcal{F}(U) := \varinjlim_\alpha \mathcal{S}_\alpha(U)$. We claim $\mathcal{F}$ is already a sheaf. Then $\mathcal{F}$ is the colimit in the sheaf category, so for any open $U$, we have
        \[
        \mathcal{S} := \varinjlim_\alpha \mathcal{S}_\alpha \quad\text{satisfies}\quad \mathcal{S}(U) = \varinjlim_\alpha \mathcal{S}_\alpha(U)
        \]
        To show $\mathcal{F}$ is a sheaf, take an open cover $U = \bigcup_i U_i$. For separatedness, suppose $s \in \mathcal{F}(U)$ restricts to zero in every $\mathcal{F}(U_i)$. Represent $s$ by some $s_\alpha \in \mathcal{S}_\alpha(U)$. Its restriction to $\mathcal{F}(U_i)$ is the image of $s_\alpha|_{U_i}$. If that image is zero in the colimit, then for each $i$ there is some stage $\beta_i\geq\alpha$ with $s_{\beta_i}|_{U_i}=0$. Since the system is filtered, find $\gamma$ dominating all $\beta_i$. Then $(s_\gamma)|_{U_i}=0$ for all $i$, so by the sheaf property of $\mathcal{S}_\gamma$, $s_\gamma=0$, hence $s=0$ in the colimit.

        For gluing, suppose we have $s_i \in \mathcal{F}(U_i)$ that agree on overlaps. Represent each $s_i$ by some $s_{i,\alpha_i}\in \mathcal{S}_{\alpha_i}(U_i)$. Filteredness gives a stage $\beta$ dominating all $\alpha_i$. Push all $s_{i,\alpha_i}$ to $\mathcal{S}_\beta$; they still agree on overlaps, so they glue to some $s_\beta\in \mathcal{S}_\beta(U)$. Its class in $\mathcal{F}(U)$ glues the original $s_i$.

        Therefore the presheaf colimit is a sheaf, and $\mathcal{S}(U)=\varinjlim_\alpha \mathcal{S}_\alpha(U)$. Thus filtered colimits are computed stalkwise, so it is enough to show that in the category of abelian groups filtered colimits commute with finite products. 
    \end{proof}

    \begin{proof}
    [Proof of Lemma 2]
    Take a short exact sequence of direct systems
    \[
    0 \to A_\alpha \xrightarrow{f_\alpha} B_\alpha \xrightarrow{g_\alpha} C_\alpha \to 0,
    \]
    with all maps compatible. We need to show that
    \[
    0 \to \varinjlim A_\alpha \to \varinjlim B_\alpha \to \varinjlim C_\alpha \to 0
    \]
    is exact.

    Left-exactness (injectivity at $\varinjlim A_\alpha$) holds for any colimit: if an element becomes zero in $\varinjlim B_\alpha$, it is already zero after some stage, and then in $A_\alpha$ by exactness there.

    Surjectivity at $\varinjlim C_\alpha$ is where filteredness is used. Let $[c_\alpha] \in \varinjlim C_\alpha$. Pick representative $c_\alpha\in C_\alpha$. Exactness at $C_\alpha$ gives $b_\alpha\in B_\alpha$ with $g_\alpha(b_\alpha)=c_\alpha$. Let $[b_\alpha]$ be its class in $\varinjlim B_\alpha$; its image is $[c_\alpha]$. 

    We need to check this does not depend on choices: if we change stage or lift, filteredness gives a common stage where the choices agree, because kernels and images are compatible. That is standard and uses only that the system is filtered.

    Hence the right map is surjective and the colimit of a short exact sequence is exact.
    \end{proof}

    \begin{proof}[Proof of Lemma 3]
    Let $\mathcal{F}$ be a sheaf on $R$ with finite support $\{p_1,\dots,p_n\}$. Then
    \[
    \mathcal{F} \cong \bigoplus_{i=1}^n (i_{p_i})_* \mathcal{F}_{p_i},
    \]
    where $i_{p_i} : \{p_i\} \hookrightarrow R$ is the inclusion. Each $(i_{p_i})_* \mathcal{F}_{p_i}$ is a skyscraper sheaf supported at $p_i$. 
    
Skyscraper sheaves are flasque. For $\mathcal{G} = (i_x)_*A$, where $A$ is an abelian group and $x\in R$, we have for $V\subset U$, the restriction map $\mathcal{G}(U)\to \mathcal{G}(V)$ is either:
    \[
    \begin{cases}
    \mathrm{id}_A & \text{if both contain }x \\
    A\to0 & \text{if }x\in U, x\notin V \\
    0\to0 & \text{otherwise}
    \end{cases}
    \]

    In all cases it is surjective. Therefore every restriction map is surjective, so $\mathcal{G}$ is flasque.

Finite direct sums of flasque sheaves are flasque since surjectivity is preserved under finite products. Flasque sheaves have vanishing higher cohomology. Therefore, each summand has $H^q(R;(i_{p_i})_* \mathcal{F}_{p_i})=0$ for $q\geq1$. 
Since cohomology commutes with finite direct sums, we conclude that $H^q(R;\mathcal{F})=0$ for $q\geq1$.
\end{proof}
\end{solution}

\begin{problem}[5]
A \textbf{nodal Riemann surface} $S$ is one which is smooth except for finitely many singularities that look locally like
\[
\{(x, y) \mid xy = 0\} \subset \mathbb{C}^2.
\]
It is obtained from a smooth Riemann surface $\widetilde{S}$ by identifying pairs of points; $\widetilde{S} \to S$ is called the \textbf{normalization}. Define the canonical bundle $K_S$ of $S$ to be the sheaf of differentials holomorphic on $\widetilde{S}$, except for simple poles at the nodes, with opposite residues on the two branches of $S$.

Show that $K_S$ is a line bundle. When $S$ is compact, prove the residue theorem for a meromorphic section of $K_S$ that is holomorphic at the nodes. Prove Serre duality for vector bundles on $S$.

\textbf{Hint:} Reduce to $\widetilde S$. (Use the Čech definition to show that you can compute $H^1$ on $\widetilde S$ instead of $S$.)
\end{problem}

\begin{solution}
    Let $\nu:\widetilde S\to S$ be the normalization.
Each node of $S$ has preimage a pair $\{p_i^+,p_i^-\}\subset\widetilde S$.
Set $N:=\sum_i(p_i^++p_i^-)$.

Let $K_{\widetilde S}$ be the canonical bundle on the smooth Riemann surface
$\widetilde S$. Consider $K_{\widetilde S}(N)$, whose sections are meromorphic $1$--forms
with at worst simple poles at the points of $N$.

Define $K_S$ as the subsheaf of $\nu_*K_{\widetilde S}(N)$ consisting of
sections $\omega$ such that for each node with preimages $p_i^\pm$,
\[
\res_{p_i^+}(\omega)+\res_{p_i^-}(\omega)=0.
\]
Equivalently, we have an exact sequence
\[
0\longrightarrow K_S\longrightarrow \nu_*K_{\widetilde S}(N)
\xrightarrow{\ \mathrm{Res}\ } \bigoplus_i \underline{\C}_{\text{node }i}
\longrightarrow 0,
\tag{nodal}
\label{nodal}
\]
where $\underline{\C}_{\text{node }i}$ is the skyscraper sheaf at the $i$-th node and the residue map takes
$\omega$ to the collection of sums of residues at $p_i^\pm$, explicitly at the node $i$:
\[\mathrm{Res}_i(\omega) = \res_{p_i^+}(\omega) + \res_{p_i^-}(\omega)\]


We claim $K_S$ is a line bundle.
Over smooth points of $S$, $\nu$ is an isomorphism, and the residue condition
is vacuous, so $K_S$ agrees with $K_{\widetilde S}$; in particular it has
rank $1$ there.

Near a node, choose local coordinates so $\widetilde S$ is the disjoint
union of two discs with coordinates $z$ and $w$, and the node identifies
$z=0$ with $w=0$. A section of $K_{\widetilde S}(N)$ has local form
\[
\left(\frac{a}{z}+\text{hol}\right)dz
\quad\text{on the $z$--branch,}\qquad
\left(\frac{b}{w}+\text{hol}\right)dw
\quad\text{on the $w$--branch.}
\]
The condition $a+b=0$ is one linear relation on the two residues, so the
allowed principal parts form a $1$-dimensional space. Thus the stalk of
$K_S$ at the node is also $1$-dimensional.



Assume $S$ compact. Let $\omega$ be a meromorphic section of $K_S$ which is
holomorphic at the nodes. Viewed on $\widetilde S$, this is a meromorphic
$1$--form $\widetilde\omega$ which has the same poles away from the preimages of nodes and is holomorphic at each $p_i^\pm$ (since $\omega$ has no poles at nodes).

Apply the usual residue theorem on $\widetilde S$:
\[
\sum_{q\in\widetilde S} \res_q(\widetilde\omega)=0.
\]
There is no contribution from $p_i^\pm$, so this sum is exactly
$\sum_{p\in S}\res_p(\omega)$. Hence
\[
\sum_{p\in S} \res_p(\omega)=0,
\]
so we obtain the residue theorem on $S$.


Let $E$ be a holomorphic vector bundle on $S$.
We claim there is a natural perfect pairing
\[
H^1(S,E)\times H^0(S,K_S\otimes E^\vee)\longrightarrow\C,
\]
inducing an isomorphism
\[
H^1(S,E)^\vee \cong H^0(S,K_S\otimes E^\vee).
\]

Hence $H^1(S,E)$ identifies with a
subspace of $H^1(\widetilde S,\nu^*E)$ cut out by linear relations at the nodes. In particular, using the principal parts characterization of $H^1$, a class in $H^1(S,E)$ corresponds to a collection of principal parts of $E$ at points of $\widetilde S$ (the poles of the sections) such that at each node the principal parts at $p_i^\pm$ have equal values residues in a common trivialization of $E$ near the node. 

Similarly, a section of $K_S\otimes E^\vee$ corresponds to a section
\[
\phi\in H^0\big(\widetilde S,K_{\widetilde S}(N)\otimes\nu^*E^\vee\big)
\]
such that at each node the residues at $p_i^\pm$ are opposite, i.e.\ to a
subspace of $H^0\big(\widetilde S,K_{\widetilde S}(N)\otimes\nu^*E^\vee\big)$
defined by linear relations.

On the smooth curve $\widetilde S$ we have the usual Serre duality pairing
\[
H^1(\widetilde S,\nu^*E)\times
H^0\big(\widetilde S,K_{\widetilde S}\otimes(\nu^*E)^\vee\big)
\to\C,
\]
which extends to allow simple poles at $N$, giving a perfect pairing
\[
H^1(\widetilde S,\nu^*E)\times
H^0\big(\widetilde S,K_{\widetilde S}(N)\otimes(\nu^*E)^\vee\big)
\to\C.
\]
The node-compatibility conditions defining $H^1(S,E)$ and $H^0(S,K_S\otimes
E^\vee)$ are dual to each other with respect to this pairing. In particular, locally the principal part $(v^+,v^-)$ pairs with $(\ell^+,\ell^-)$ by
$\ell^+(v^+)-\ell^-(v^-)$.

Thus its restriction yields a perfect pairing
\[
H^1(S,E)\times H^0(S,K_S\otimes E^\vee)\to\C.
\]
\end{solution}

\begin{problem}[6]
Let $S$ be a chain of elliptic Riemann surfaces $E_1, \dots, E_g$, connected by nodes.
\begin{enumerate}[label={\textbullet}, leftmargin=*]
    \item Describe the space of holomorphic differentials (sections of $K_S$) and the period lattice for $S$.
    \item Show that $\operatorname{Pic}(S) \cong \mathbb{Z}^g \times E_1 \times \cdots \times E_g$.
    \item Collapse this to $\mathbb{Z} \times E_1 \times \cdots \times E_g$, “remembering only the total degree,” by allowing a divisor to skip from one component to another at the node.  
    That is, if $p', p''$ are the two points on $\widetilde S$ over a node, identify the line bundles corresponding to $\mathcal{O}(p') \times \mathcal{O}$ and $\mathcal{O} \times \mathcal{O}(p'')$ on adjacent elliptic curves.  
    Show that you can now define an Abel–Jacobi map
    \[
    \operatorname{Sym}^k(S) \longrightarrow \operatorname{Pic}.
    \]
    \item What is the image of $S$ in $\operatorname{Pic}^1$? What is the canonical Theta-divisor $W_{g-1}$ in $\operatorname{Pic}^{g-1}$?
    \item Check that the Riemann Theta function for $S$ is the product of the Jacobi $\theta_3$ functions for the $E_i$:
    \[
    \theta_3(u \mid \tau) = \sum_{n=0}^{\infty} \exp(2\pi i n u + \pi i n^2 \tau),
    \]
    and identify the Riemann Theta divisor. Compare with $W_{g-1}$.
\end{enumerate}

\textbf{Note:} $\theta_3$ has a unique simple zero at the center of the period parallelogram.  
\textbf{Caution:} The instructor has not checked any of this!
\end{problem}

\begin{solution}
Let $\nu:\widetilde S=\bigsqcup_{i=1}^g E_i\to S$ be the normalization.
The canonical bundle $K_S$ is the subsheaf of
$\nu_*K_{\widetilde S}(N)$ consisting of meromorphic $1$-forms on
$\widetilde S$ with at worst simple poles at the preimages of nodes and
opposite residues on the two branches. For a section of $K_S$ which is
holomorphic at the nodes, all residues are zero, so there is no coupling
condition between components.

Each elliptic curve $E_i$ has
$H^0(E_i,K_{E_i})\cong\C$; choose a nonzero form $\omega_i$.
Viewing $\omega_i$ as a form on $\widetilde S$ extended by $0$ on the
other components gives a global section of $K_S$.
Thus
\[
H^0(S,K_S)\;\cong\;\bigoplus_{i=1}^g H^0(E_i,K_{E_i})
\;\cong\;\C^g.
\]
So a basis of holomorphic differentials on $S$ is given by
$\{\omega_1,\dots,\omega_g\}$, where $\omega_i$ lives on $E_i$ and vanishes
on the others.

Fix symplectic bases of $H_1(E_i,\Z)$ with periods
$\int_{a_i}\omega_i=1$, $\int_{b_i}\omega_i=\tau_i$.
Then
\[
H_1(S,\Z)\;\cong\;\bigoplus_{i=1}^g H_1(E_i,\Z),
\]
and the period matrix of $S$ (for the above basis of $H^0(K_S)$) is block
diagonal:
\[
\Pi_S=\mathrm{diag}(\tau_1,\dots,\tau_g).
\]
The period lattice of $S$ is the direct sum of the lattices of the $E_i$.

Line bundles on $S$ are obtained by choosing line bundles on each component and gluing uniquely at the
nodes. In particular we have the standard exact sequence of sheaves of units:
\[
0\to\mathcal O_S^\times
\to \nu_*\mathcal O_{\widetilde S}^\times
\to \bigoplus_{\text{nodes}} \underline{\C^\times}
\to 0
\]
Here a unit on $S$ pulls back to a unit on each component; the quotient is given by the ratios of the two branch values at each node.

Take cohomology. Using $H^1(\underline{\C^\times})=0$ (skyscraper), we get:
\[
0\to H^0(\mathcal O_S^\times)
\to H^0(\nu_*\mathcal O_{\widetilde S}^\times)
\to \bigoplus_{\text{nodes}} \C^\times
\to \Pic(S)
\to \Pic(\widetilde S)
\to 0
\]

We compute that $H^0(\mathcal O_S^\times)=\C^\times$, $H^0(\nu_*\mathcal O_{\widetilde S}^\times) = \prod_{i=1}^g H^0(\mathcal O_{E_i}^\times) = (\C^\times)^g$, and \[\Pic(\widetilde S)=\prod_{i=1}^g\Pic(E_i)\]

The map $(\C^\times)^g\to(\C^\times)^{\text{nodes}}$ is: for each node joining $E_i$ and $E_{i+1}$, send $(\lambda_1,\dots,\lambda_g)$ to $\lambda_i/\lambda_{i+1}$. For a chain there are $g-1$ nodes, so this map is:
\[
(\C^\times)^g \longrightarrow (\C^\times)^{g-1},
\qquad
(\lambda_i)\mapsto (\lambda_i/\lambda_{i+1})_{i=1}^{g-1}
\]

This map is surjective (choose $\lambda_1=1$, then define $\lambda_{i+1}=\lambda_i/t_i$ for given $t_i$). Its kernel is exactly the diagonal $\C^\times$. Thus the map $\bigoplus_{\text{nodes}}\C^\times \to \Pic(S)$ has kernel equal to the image of $(\C^\times)^g$, so its cokernel is zero. Therefore the next map
\[
\Pic(S) \longrightarrow \Pic(\widetilde S) = \prod_{i=1}^g \Pic(E_i)
\]
is an isomorphism.

On each elliptic curve,
\[
\Pic(E_i) \;\cong\; \Z \times E_i,
\]
via $(\deg,\Pic^0)\cong (\deg,\text{point})$.
Hence
\[
\Pic(S)
\;\cong\;
\prod_{i=1}^g (\Z \times E_i)
\;\cong\;
\Z^g \times E_1 \times \cdots \times E_g.
\]

Define an equivalence relation on $\Pic(S)$ by allowing degree to move
across a node: if $p',p''$ are the two preimages of a node on
$E_i,E_{i+1}$, identify
\[
\mathcal O_{E_i}(p')\boxtimes\mathcal O_{E_{i+1}}
\;\sim\;
\mathcal O_{E_i}\boxtimes\mathcal O_{E_{i+1}}(p'').
\]
On multidegrees this identifies $(\dots,d_i,\dots,d_{i+1},\dots)$ with
$(\dots,d_i-1,\dots,d_{i+1}+1,\dots)$, so only the \emph{total} degree
$\sum_i d_i$ remains.

The quotient of $\Pic(S)$ by this relation is
\[
\Z \times E_1\times\cdots\times E_g.
\]
Call this (still) $\Pic$.
Given an effective divisor $D$ of degree $k$ on $S$, define
\[
AJ_k(D) := [\mathcal O_S(D)] \in \Pic
\]
(using total degree $k$ and the $\Pic^0$-coordinates on each $E_i$).
This gives the Abel--Jacobi map
\[
AJ_k:\Sym^k(S)\longrightarrow \Pic.
\]

Fix basepoints $q_i\in E_i$ and use them to identify $\Pic^1$ with
\[
\Pic^1 \;\cong\; \{1\}\times E_1\times\cdots\times E_g.
\]
A point $x\in E_j\subset S$ maps to the line bundle
\[
\mathcal O_S(x) \sim
\mathcal O_{E_j}(x-q_j)
\]
on the $j$-th factor, trivial elsewhere.
Thus the image of $S$ in $\Pic^1$ is the union of the $g$ copies of the
$E_j$ embedded as
\[
x\in E_j \;\longmapsto\; (1; 0,\dots,0, [x-q_j], 0,\dots,0).
\]

For degree $g-1$, line bundles in $\Pic^{g-1}$ with a nonzero section are
exactly those arising from effective divisors of degree $g-1$.
On the chain, such a divisor omits at least one component, so
$W_{g-1}\subset\Pic^{g-1}$ is the union of $g$ translates where one
coordinate (corresponding to a chosen component) is constrained by the
condition that the section vanishes entirely on that component. 


With the chosen basis of $H^0(K_S)$, the period matrix of $S$ is
$\tau=\mathrm{diag}(\tau_1,\dots,\tau_g)$.
The Riemann theta function for $S$ is
\[
\Theta(z\mid\tau)
=\sum_{n\in\Z^g}
\exp\big(\pi i\, n^T\tau n + 2\pi i\,n^T z\big),
\quad z=(z_1,\dots,z_g).
\]
For diagonal $\tau$ this factorizes:
\[
\Theta(z\mid\tau)
=\prod_{i=1}^g \theta_3(z_i\mid\tau_i),
\]
where
\[
\theta_3(u\mid\tau)
=\sum_{n\in\Z}\exp(2\pi i n u + \pi i n^2\tau).
\]

Thus the theta divisor of $S$ is
\[
\{\Theta(z\mid\tau)=0\}
=\bigcup_{i=1}^g \{\,\theta_3(z_i\mid\tau_i)=0\,\},
\]
a union of coordinate hypersurfaces in the product torus, matching the
description of $W_{g-1}$ above.
\end{solution}

\end{document}