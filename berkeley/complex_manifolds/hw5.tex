\documentclass[12pt]{article}
\usepackage{/Users/songye03/Desktop/Math_tex/style/psetconfig}

\DeclareMathOperator{\res}{res}
\title{Homework 5}
\author{Songyu Ye}
\date{\today}

\begin{document}
\psettitle

For Questions 1 and 2, you may use the correspondence indicated in class between the representation of $H^1$ classes by classes by principal parts versus Dolbeault distributions.
\begin{problem}[1]
For a compact Riemann surface $R$, verify that the Serre duality pairing
\[
H^1(R;\cO)\otimes H^0(R;\Omega^1)\longrightarrow \C
\]
defined by principal parts and residues agrees with the one given by integration
of Dolbeault representatives.

Using the relation to harmonic forms, explain how this relates to
Poincar\'e duality on $R$.
\end{problem}

\begin{solution} Choose a meromorphic function $f$ on $R$ whose principal part at each
$p_i$ with prescribed principal parts. Let $U_i$ be pairwise disjoint coordinate discs around $p_i$, and choose
$\chi\in C^\infty(R)$ such that $\chi\equiv 1$ on smaller discs $U_i'\subset U_i$
and $\chi\equiv 0$ outside $\bigcup_i U_i$.
Define a $(0,1)$-\emph{current}
\[
T_f := \bar\partial(\chi f).
\]

Since $\bar\partial^2=0$, $T_f$ is $\bar\partial$-closed.
If we replace $f$ by $f+g$ for a global meromorphic function $g$
(with poles in $D$) or change $\chi$ within the same constraints,
$T_f$ changes by a current of the form $\bar\partial u$, so the
class $[T_f]$ in
\[
H^{0,1}_{\bar\partial}(R) \cong H^1(R,\cO)
\]
depends only on the underlying principal parts.

Let $\omega\in H^0(R,\Omega^1)$ be a holomorphic $1$-form.
The \emph{Dolbeault} definition of the pairing is
\[
\langle\alpha,\omega\rangle_{Dol}
:= \frac{1}{2\pi i}\int_R T_f\wedge\omega
= \frac{1}{2\pi i}\int_R \bar\partial(\chi f)\wedge\omega.
\]
Since $\omega$ is of type $(1,0)$ and holomorphic,
$\bar\partial\omega=0$, hence
\[
\bar\partial(\chi f)\wedge\omega
= \bar\partial(\chi f\omega).
\]

Let $D_i\subset U_i'$ be small closed discs around $p_i$ and set
\[
R_\varepsilon := R\setminus\bigcup_i D_i(\varepsilon),
\]
where $D_i(\varepsilon)$ are concentric discs of radius $\varepsilon$.
On $R_\varepsilon$ the form $\chi f\omega$ is smooth with compact
support, so Stokes' theorem gives
\[
\int_{R_\varepsilon} \bar\partial(\chi f\omega)
= \int_{\partial R_\varepsilon} \chi f\omega
= -\sum_i \int_{\partial D_i(\varepsilon)} f\omega,
\]
the sign coming from the induced orientation on the boundary.

Letting $\varepsilon\to 0$ and using the residue theorem,
\[
\int_{\partial D_i(\varepsilon)} f\omega
\;\longrightarrow\; 2\pi i\,\Res_{p_i}(f\omega),
\]
we obtain
\[
\frac{1}{2\pi i}\int_R \bar\partial(\chi f)\wedge\omega
= \sum_i \Res_{p_i}(f\omega).
\]
This is precisely the \emph{principal parts} definition of the Serre pairing.

Now equip $R$ with any Hermitian (necessarily K\"ahler) metric. Hodge theory yields the decompositions
\[
H^1_{\dR}(R,\C)
\;\cong\;
\cH^1(R) \;\cong\;
H^{1,0}_{\bar\partial}(R)\oplus H^{0,1}_{\bar\partial}(R),
\]
and every class has a unique harmonic representative.  Moreover,
\[
H^0(R,\Omega^1) \cong H^{1,0}_{\bar\partial}(R)
\]
consists of harmonic $(1,0)$-forms, and
\[
H^1(R,\cO) \cong H^{0,1}_{\bar\partial}(R)
\]
is represented by harmonic $(0,1)$-forms.  Complex conjugation gives
an isomorphism \[\overline{H^{1,0}_{\bar\partial}(R)}
\cong H^{0,1}_{\bar\partial}(R)\]

Poincaré duality on $R$ is given by the nondegenerate pairing
\[
H^1_{\dR}(R,\C)\times H^1_{\dR}(R,\C)
\longrightarrow \C,
\qquad
([\alpha],[\beta])\mapsto\int_R \alpha\wedge\beta.
\]
It is clear that $\alpha \wedge \beta$ is nonzero only if
$\alpha$ and $\beta$ are of complementary types, i.e. their wedge is of type
$(1,1)$, since $(1,0)\wedge(1,0)$ and $(0,1)\wedge(0,1)$ necessarily vanish.
Thus the Poincaré pairing restricts to a nondegenerate pairing
\[
H^{0,1}_{\bar\partial}(R)\;\otimes\; H^{1,0}_{\bar\partial}(R)
\longrightarrow \C,
\qquad
(\eta,\omega)\mapsto\int_R \eta\wedge\omega,
\]
with $\eta,\omega$ harmonic representatives.

Under the identifications
\[
H^1(R,\cO)\cong H^{0,1}_{\bar\partial}(R),
\qquad
H^0(R,\Omega^1)\cong H^{1,0}_{\bar\partial}(R),
\]
the Serre pairing of $\alpha$ and $\omega$ is
\[
\langle\alpha,\omega\rangle
= \frac{1}{2\pi i}\int_R \eta\wedge\omega,
\]
where $\eta$ is the harmonic $(0,1)$-representative of $\alpha$. In particular, on a compact Riemann surface the Serre duality
\[
H^1(R,\cO)\;\cong\;H^0(R,\Omega^1)^\vee
\]
is nothing but Poincaré duality in degree~$1$ up to the 
constant factor $2\pi i$, expressed via the Hodge
decomposition of $H^1_{\dR}(R,\C)$.
\end{solution}



\begin{problem}[2]
For a compact Riemann surface $R$, verify that the map
\[
H^1(R;\Z)\longrightarrow H^1(R;\cO)
\]
corresponds to the period map
\[
H_1(R;\Z)\otimes H^0(R;\Omega^1)\longrightarrow \C
\]
under integral Poincar\'e duality and Serre duality on $R$.
\end{problem}

\begin{solution}
Let $i:H^1(R;\Z)\to H^1(R;\cO)$ be the given homomorphism. We need to show
for every $c\in H^1(R;\Z)$ and $\omega\in H^0(R,\Omega^1)$, the Serre
pairing
\(
\langle i(c),\omega\rangle_{\mathrm{Serre}}
\)
equals the period of $\omega$ along the $1$--cycle Poincar\'e dual to
$c$.

By Hodge theory, every class in $H^1(R;\R)$ has a unique harmonic
representative.  An element $c\in H^1(R;\Z)$ maps to a real class
$c_\R\in H^1(R;\R)$ whose harmonic representative we denote by
$\alpha$ so
\[
[\alpha]_{\dR} = c_\R\in H^1_{\dR}(R;\R).
\]
Decompose $\alpha$
\[
\alpha = \alpha^{1,0}+\alpha^{0,1},
\qquad
\alpha^{0,1} = \overline{\alpha^{1,0}},
\]
since $\alpha$ is real.  Under the Dolbeault isomorphism and Hodge
decomposition, we have
\[
H^1(R,\cO)\;\cong\; H^{0,1}_{\bar\partial}(R)
\]
and the image $i(c)\in H^1(R,\cO)$ is represented by the harmonic $(0,1)$--form $\alpha^{0,1}$.

We know that the Serre pairing can be described as
\[
\langle\beta,\omega\rangle_{\mathrm{Serre}
}
=\frac{1}{2\pi i}\int_R \eta^{0,1}\wedge\omega
\]
whenever $\beta\in H^1(R,\cO)$ is represented by a harmonic
$(0,1)$--form $\eta^{0,1}$ and $\omega\in H^0(R,\Omega^1)$ is a
holomorphic $1$--form.

Applying this to $\beta=i(c)$ and $\eta^{0,1}=\alpha^{0,1}$ gives
\[
\langle i(c),\omega\rangle_{\mathrm{Serre}}
= \frac{1}{2\pi i}\int_R \alpha^{0,1}\wedge\omega.
\]
Since $R$ has complex dimension $1$, a $(2,0)$–form vanishes, hence
$\alpha^{1,0}\wedge\omega=0$, and therefore
\[
\alpha^{0,1}\wedge\omega
= (\alpha^{1,0}+\alpha^{0,1})\wedge\omega
= \alpha\wedge\omega.
\]
Thus
\begin{equation}
\label{eq:Serre-alpha}
\langle\iota^*c,\omega\rangle_{\mathrm{Serre}}
= \frac{1}{2\pi i}\int_R \alpha\wedge\omega.
\end{equation}
Integral Poincar\'e duality gives a perfect pairing
\[
H^1(R;\Z)\times H_1(R;\Z)\longrightarrow\Z,
\]
and we denote by $\gamma_c\in H_1(R;\Z)$ the Poincar\'e dual of $c$.

The de~Rham realization of this pairing is as follows.  The class
$c_\R\in H^1(R;\R)$ is represented by the closed $1$--form $\alpha$ with integral periods, i.e.
\[
\int_\gamma \alpha \in \Z
\quad\text{for all } \gamma\in H_1(R;\Z).
\]
The Poincar\'e dual cycle $\gamma_c$ is then characterized by
\[
\int_{\gamma_c} \beta
= \int_R \alpha\wedge\beta
\quad\text{for all closed $1$--forms }\beta,
\]

Thus, if we identify
\[
H^1(R;\Z)\xrightarrow{\;\text{PD}\;}\; H_1(R;\Z)
\quad\text{and}\quad
H^1(R;\cO)\xrightarrow{\;\text{Serre}\;}
H^0(R,\Omega^1)^\vee,
\]
the class $c\in H^1(R;\Z)$ maps to the functional
\[
H^0(R,\Omega^1)\longrightarrow\C,
\qquad
\omega\longmapsto \frac{1}{2\pi i}\int_{\gamma_c}\omega.
\]

This is precisely the period map (up to the factor $1/(2\pi i)$)
\[
H_1(R;\Z)\otimes H^0(R,\Omega^1)\longrightarrow\C,
\qquad
(\gamma,\omega)\longmapsto\int_\gamma\omega,
\]
with $\gamma=\gamma_c$ the Poincar\'e dual of $c$.

\end{solution}

\begin{problem}[3]
Show that the period mapping gives an isomorphism
\[
H_1(R;\Z)\xrightarrow{\ \sim\ } H_1(J;\Z),
\]
which can be realized geometrically by the Abel--Jacobi map
\[
R\longrightarrow J_1.
\]
Show that under this correspondence, $c_1(\Theta)\in\Lambda^2 H_1(R)$
is the intersection pairing on $R$.

\textit{Hints for the second part:} You can deduce it from the periodicity
formulas of the Riemann $\Theta$-function. Alternatively, you can find this
by exploiting the facts that the Poincar\'e dual of $c_1(\Theta)$ in $J_{g-1}$
is the Theta divisor, the image of $\Sym^{g-1}(R)$. The maps
\[
\Sym^g(R)\longrightarrow J_g
\qquad\text{and}\qquad
\Sym^{g-1}(R)\longrightarrow \operatorname{div}(\Theta)
\]
have degree $1$.
\end{problem}

\begin{solution}
The presentation of the Jacobian $J$ as
\[
J \cong H^1(R;\cO) / H_1(R;\Z)
\]
makes it clear that $H_1(J;\Z)$ is naturally identified with
$H_1(R;\Z)$, since the universal cover of $J$ is the vector space
$H^1(R;\cO)$. The period mapping 
\[
H_1(R;\Z)\to H_1(J;\Z)
\] is injective because of the Riemann bilinear relations, and since both groups are free abelian of rank $2g$, it is an isomorphism. Pick a base point $p_0\in R$ and define the Abel--Jacobi map
\[
\varphi:R\to J,
\qquad
p\mapsto \left[\omega\mapsto \int_{p_0}^p \omega\right].
\]
precisely implements the lift of the period mapping to the universal cover and hence induces the same isomorphism on $H_1$.

Pick a symplectic basis $\{a_1,\dots,a_g,b_1,\dots,b_g\}$ of $H_1(R,\Z)$, i.e.
\[a_i\cdot a_j = 0,
\quad b_i\cdot b_j = 0,
\quad a_i\cdot b_j = \delta_{ij}.\]


Under the identification
$H_1(R,\Z)\xrightarrow{\sim}\Lambda\cong H_1(J,\Z)$ coming from the period map and the Abel--Jacobi embedding, a homology class $\gamma\in H_1(R,\Z)$ corresponds to an integral vector $(m,n)\in\Z^{2g}$. The intersection pairing on $H_1(R,\Z)$ is given in these coordinates by
\[
(m,n)\cdot(m',n') \;=\; m^{\mathrm T}n' - m'^{\mathrm T}n.
\]
The Riemann theta function with period matrix $\tau$ is
\[
\theta(z\mid\tau)
:= \sum_{k\in\Z^g}
\exp\bigl(\pi i\,k^{\mathsf T}\tau k + 2\pi i\,k^{\mathsf T}z\bigr),
\quad z\in\C^g.
\]
The Riemann theta function satisfies the quasi-periodicity property.
\[
\theta\bigl(z + m + \tau n \mid \tau\bigr)
=
\exp\bigl(-\pi i n^{\mathsf T}\tau n - 2\pi in^{\mathsf T}z\bigr)
\theta(z\mid\tau)
\]
In particular, the Riemann theta function defines a holomorphic section of the line bundle
$\mathcal{O}_J(\Theta)$.

Hence, identifying $H^{2}(U,\mathbb{Z})$ and $H^{2}(X,\mathbb{Z})$ by the above
isomorphism, the Chern class of $L$ is simply $\delta(\operatorname{cl}\{e_{u}\})$.
Write $e_{u}(z)=e^{2\pi i f_{u}(z)}$ with $f_{u}$ holomorphic in $V$.  Then by
definition, $\delta(\operatorname{cl}\{e_{u}\})\in H^{2}(U,\mathbb{Z})$ is
given by the $2$--cocycle $F(u_{1},u_{2})$ on $U$ with coefficients in
$\mathbb{Z}$ defined by
\[
  F(u_{1},u_{2})
  = f_{u_{2}}(z+u_{1}) - f_{u_{1}+u_{2}}(z) + f_{u_{1}}(z)
  \in \mathbb{Z}. \tag{$*$}
\]

\begin{lemma}[1 (Mumford)]
Let $U \subset V$ be a lattice in a complex vector space $V$. The map which associates to any map $F: U\times U \to \mathbb{Z}$ the
map $AF: U\times U\to \mathbb{Z}$ defined by
\[
  AF(u_{1},u_{2}) = F(u_{1},u_{2}) - F(u_{2},u_{1})
\]
maps the group of $2$--cocycles $Z^{2}(U,\mathbb{Z})$ into the space of
alternating linear maps $U\times U\to\mathbb{Z}$, and induces an
isomorphism
\[
  A:\;
  H^{2}(U,\mathbb{Z})
  \;\xrightarrow{\ \sim\ }\;
  \operatorname{Hom}(\Lambda^{2}U,\mathbb{Z})
  \;\cong\;
  \Lambda^{2}\operatorname{Hom}(U,\mathbb{Z}).
\]
Furthermore for $\xi,\eta\in\operatorname{Hom}(U,\mathbb{Z})=H^{1}(U,\mathbb{Z})$,
we have $A(\xi\smile\eta)=\xi\wedge\eta$.
\end{lemma}

\begin{proposition}[2 (Mumford)]
The Chern class of the line bundle corresponding to
$\{e_{u}\}\in Z^{1}(U,H^{*})$ is the alternating $2$--form on $U$ with values
in $\mathbb{Z}$ given by
\[
  E(u_{1},u_{2})
  = f_{u_{2}}(z+u_{1}) + f_{u_{1}}(z)
    - f_{u_{1}}(z+u_{2}) - f_{u_{2}}(z),
  \qquad (z\text{ arbitrary in }V),
  \tag{$**$}
\]
where
\[
  e_{u}(z) = e^{\,2\pi i\, f_{u}(z)}.
\]
\end{proposition}
Moreover if we extend $E$ $\mathbb{R}$--linearly to a map
$V\times V\to\mathbb{R}$, then $E$ satisfies the identity
\[
  E(ix,iy) = E(x,y) \qquad \text{for } x,y\in V.
\]
\end{solution}

\begin{problem}[4]
Prove the following generalized Cauchy formula for a smooth function $f$
defined in the unit disk $\Delta$:
\[
f(z,\bar z)
=
\frac{1}{2\pi i}\oint_{|\zeta-z|=r}
\frac{f(\zeta)}{\zeta-z}\,d\zeta
+
\frac{1}{2\pi i}\iint_{\Delta'}
\frac{\partial f}{\partial\bar\zeta}\,
\frac{d\zeta\wedge d\bar\zeta}{\zeta-z},
\]
where $\Delta'\subset\Delta$ is the subdisk of radius $r<1$.

\textit{Remark:} When $f$ is holomorphic, you recover Cauchy's formula.
\end{problem}

\begin{problem}[5]
Let $L\to X$ be a holomorphic line bundle on a complex manifold, and let
$\alpha\in\cE^{0,1}$ be a $\bar\partial$-closed form. Show that the
re-defined operator
\[
\tilde{\bar\partial} = \bar\partial + \alpha
\]
on sections of $L$ defines a new holomorphic structure $L'$ on the same
underlying bundle, where local holomorphic sections are defined as those
killed by $\tilde{\bar\partial}$. Show that $L\simeq L'$ if $\alpha$ is
$\bar\partial$-exact. Relate this to the exponential sequence.

\textit{Remark:} For vector bundles, the same applies with an
$\alpha\in\cE^{0,1}(\End(V))$ satisfying the non-linear equation
\[
\bar\partial\alpha + \alpha\wedge\alpha = 0.
\]
The new bundle is isomorphic to the old one if
$\alpha = a^{-1}\bar\partial a$, for some smooth section $a$ of $\Aut(V)$.
\end{problem}

\begin{problem}[6]
Let $V$ be a complex $g$-dimensional vector space and
$L\simeq\Z^{2g}\subset V$ a lattice. Let $A = V/L$.

\begin{enumerate}
\item Using harmonic theory, compute the Dolbeault cohomology $H^\ast(A;\cO)$.

\item Show that the moduli space of holomorphic line bundles on $A$ with zero
Chern class is naturally identified with
\[
A^\vee := V^\vee / L^\vee.
\]

\item Show that the moduli space of holomorphic line bundles on $A^\vee$ is
naturally identified with $A$.

\item Define a line bundle
\[
\mathcal{P}\longrightarrow A\times A^\vee
\]
from the trivial line bundle over $V\times V^\vee$ with connection
\[
\nabla = d + i(x\,d\xi + \xi\,dx),
\]
by quotienting out the $L\times L^\vee$-action as follows: identify the
fiber $\C$ over $(x,\xi)\in V\times V^\vee$ with that over
$(x+\ell,\xi+\lambda)$ by multiplication by
\[
\exp\bigl(2\pi i(\lambda(x)+\xi(\ell))\bigr).
\]
Show that $\mathcal{P}$ is holomorphic, that
$\mathcal{P}|_{A\times\{a^\vee\}}$ is the line bundle over $A$ classified
by $a^\vee\in A^\vee$, and prove the corresponding statement for
$\{a\}\times A^\vee$.
\end{enumerate}
\end{problem}

\begin{problem}[7]
Show that, in the case of the Jacobian $J$ of a Riemann surface $R$,
one has a natural isomorphism $J \simeq J^\vee$.

\textit{Hint:} Remember the natural Hilbert space structure on holomorphic
differentials.

\textit{Remark:} This self-duality is a property of principally polarized
Abelian varieties, those $A$ equipped with a positive line bundle having a
single holomorphic section (the $\Theta$-function).
\end{problem}

\begin{problem}[8]
Given a holomorphic line bundle $\mathcal{L}$ on a complex manifold and a
smooth real closed $2$-form $\omega$ in the cohomology class of
$c_1(\mathcal{L})$, prove that there exists a Hermitian metric on
$\mathcal{L}$ whose holomorphic connection has curvature $-2\pi i\,\omega$.

Conclude (from Kodaira vanishing) that the holomorphic line bundles on a
compact Riemann surface $R$ which carry metrics of positive curvature are
precisely those of positive degree.

Show also that for every holomorphic vector bundle $V$ on $R$, there exists
a $d$ so that the twisted bundle $V(D)$ has no $H^1$ for any $D>d$.
\end{problem}

\begin{problem}[9]
Show that isomorphism classes of \emph{flat unitary} line bundles on a
manifold $X$ are classified by $H^1(X;U(1))$, with the constant sheaf $U(1)$
associated to the unit circle group in $\C^\times$.

When $X$ is compact K\"ahler, compare the constant and holomorphic
exponential sequences to conclude that the map
\[
H^1(X;U(1))\longrightarrow H^1(X;\cO^\times)
\]
induces a bijection from isomorphism classes of flat unitary line bundles to
those of holomorphic line bundles with zero Chern class.

\textit{Remark:} You probably need the Hodge decomposition theorem for the
second part.
\end{problem}

\begin{problem}[10]
Prove the global $\partial\bar\partial$-Lemma on a compact K\"ahler
manifold $X$: for any $d$-exact form $\varphi\in\cE^{p,q}$, there exists
$\psi\in\cE^{p-1,q-1}$ with
\[
\partial\bar\partial\psi = \varphi.
\]

\textit{Hint:} Show that
\[
\varphi = \partial\bar\partial^\ast \Box \varphi
\]
and use this and similar identities to find $\psi$.
\end{problem}

\end{document}