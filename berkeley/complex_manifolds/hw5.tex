\documentclass[12pt]{article}
\usepackage{/Users/songye03/Desktop/Math_tex/style/psetconfig}

\DeclareMathOperator{\res}{res}
\title{Homework 5}
\author{Songyu Ye}
\date{\today}

\begin{document}
\psettitle

\begin{problem}[1]
For a compact Riemann surface $R$, verify that the Serre duality pairing
\[
H^1(R;\cO)\otimes H^0(R;\Omega^1)\longrightarrow \C
\]
defined by principal parts and residues agrees with the one given by integration
of Dolbeault representatives.

Using the relation to harmonic forms, explain how this relates to
Poincar\'e duality on $R$.
\end{problem}

\begin{problem}[2]
For a compact Riemann surface $R$, verify that the map
\[
H^1(R;\Z)\longrightarrow H^1(R;\cO)
\]
corresponds to the period map
\[
H_1(R;\Z)\otimes H^0(R;\Omega^1)\longrightarrow \C
\]
under integral Poincar\'e duality and Serre duality on $R$.
\end{problem}

\begin{problem}[3]
Show that the period mapping gives an isomorphism
\[
H_1(R;\Z)\xrightarrow{\ \sim\ } H_1(J;\Z),
\]
which can be realized geometrically by the Abel--Jacobi map
\[
R\longrightarrow J_1.
\]

Show that under this correspondence, $c_1(\Theta)\in\Lambda^2 H_1(R)$
is the intersection pairing on $R$.

\textit{Hints for the second part:} You can deduce it from the periodicity
formulas of the Riemann $\Theta$-function. Alternatively, you can find this
by exploiting the facts that the Poincar\'e dual of $c_1(\Theta)$ in $J_{g-1}$
is the Theta divisor, the image of $\Sym^{g-1}(R)$. The maps
\[
\Sym^g(R)\longrightarrow J_g
\qquad\text{and}\qquad
\Sym^{g-1}(R)\longrightarrow \operatorname{div}(\Theta)
\]
have degree $1$.
\end{problem}

\begin{problem}[4]
Prove the following generalized Cauchy formula for a smooth function $f$
defined in the unit disk $\Delta$:
\[
f(z,\bar z)
=
\frac{1}{2\pi i}\oint_{|\zeta-z|=r}
\frac{f(\zeta)}{\zeta-z}\,d\zeta
+
\frac{1}{2\pi i}\iint_{\Delta'}
\frac{\partial f}{\partial\bar\zeta}\,
\frac{d\zeta\wedge d\bar\zeta}{\zeta-z},
\]
where $\Delta'\subset\Delta$ is the subdisk of radius $r<1$.

\textit{Remark:} When $f$ is holomorphic, you recover Cauchy's formula.
\end{problem}

\begin{problem}[5]
Let $L\to X$ be a holomorphic line bundle on a complex manifold, and let
$\alpha\in\cE^{0,1}$ be a $\bar\partial$-closed form. Show that the
re-defined operator
\[
\tilde{\bar\partial} = \bar\partial + \alpha
\]
on sections of $L$ defines a new holomorphic structure $L'$ on the same
underlying bundle, where local holomorphic sections are defined as those
killed by $\tilde{\bar\partial}$. Show that $L\simeq L'$ if $\alpha$ is
$\bar\partial$-exact. Relate this to the exponential sequence.

\textit{Remark:} For vector bundles, the same applies with an
$\alpha\in\cE^{0,1}(\End(V))$ satisfying the non-linear equation
\[
\bar\partial\alpha + \alpha\wedge\alpha = 0.
\]
The new bundle is isomorphic to the old one if
$\alpha = a^{-1}\bar\partial a$, for some smooth section $a$ of $\Aut(V)$.
\end{problem}

\begin{problem}[6]
Let $V$ be a complex $g$-dimensional vector space and
$L\simeq\Z^{2g}\subset V$ a lattice. Let $A = V/L$.

\begin{enumerate}
\item Using harmonic theory, compute the Dolbeault cohomology $H^\ast(A;\cO)$.

\item Show that the moduli space of holomorphic line bundles on $A$ with zero
Chern class is naturally identified with
\[
A^\vee := V^\vee / L^\vee.
\]

\item Show that the moduli space of holomorphic line bundles on $A^\vee$ is
naturally identified with $A$.

\item Define a line bundle
\[
\mathcal{P}\longrightarrow A\times A^\vee
\]
from the trivial line bundle over $V\times V^\vee$ with connection
\[
\nabla = d + i(x\,d\xi + \xi\,dx),
\]
by quotienting out the $L\times L^\vee$-action as follows: identify the
fiber $\C$ over $(x,\xi)\in V\times V^\vee$ with that over
$(x+\ell,\xi+\lambda)$ by multiplication by
\[
\exp\bigl(2\pi i(\lambda(x)+\xi(\ell))\bigr).
\]
Show that $\mathcal{P}$ is holomorphic, that
$\mathcal{P}|_{A\times\{a^\vee\}}$ is the line bundle over $A$ classified
by $a^\vee\in A^\vee$, and prove the corresponding statement for
$\{a\}\times A^\vee$.
\end{enumerate}
\end{problem}

\begin{problem}[7]
Show that, in the case of the Jacobian $J$ of a Riemann surface $R$,
one has a natural isomorphism $J \simeq J^\vee$.

\textit{Hint:} Remember the natural Hilbert space structure on holomorphic
differentials.

\textit{Remark:} This self-duality is a property of principally polarized
Abelian varieties, those $A$ equipped with a positive line bundle having a
single holomorphic section (the $\Theta$-function).
\end{problem}

\begin{problem}[8]
Given a holomorphic line bundle $\mathcal{L}$ on a complex manifold and a
smooth real closed $2$-form $\omega$ in the cohomology class of
$c_1(\mathcal{L})$, prove that there exists a Hermitian metric on
$\mathcal{L}$ whose holomorphic connection has curvature $-2\pi i\,\omega$.

Conclude (from Kodaira vanishing) that the holomorphic line bundles on a
compact Riemann surface $R$ which carry metrics of positive curvature are
precisely those of positive degree.

Show also that for every holomorphic vector bundle $V$ on $R$, there exists
a $d$ so that the twisted bundle $V(D)$ has no $H^1$ for any $D>d$.
\end{problem}

\begin{problem}[9]
Show that isomorphism classes of \emph{flat unitary} line bundles on a
manifold $X$ are classified by $H^1(X;U(1))$, with the constant sheaf $U(1)$
associated to the unit circle group in $\C^\times$.

When $X$ is compact K\"ahler, compare the constant and holomorphic
exponential sequences to conclude that the map
\[
H^1(X;U(1))\longrightarrow H^1(X;\cO^\times)
\]
induces a bijection from isomorphism classes of flat unitary line bundles to
those of holomorphic line bundles with zero Chern class.

\textit{Remark:} You probably need the Hodge decomposition theorem for the
second part.
\end{problem}

\begin{problem}[10]
Prove the global $\partial\bar\partial$-Lemma on a compact K\"ahler
manifold $X$: for any $d$-exact form $\varphi\in\cE^{p,q}$, there exists
$\psi\in\cE^{p-1,q-1}$ with
\[
\partial\bar\partial\psi = \varphi.
\]

\textit{Hint:} Show that
\[
\varphi = \partial\bar\partial^\ast \Box \varphi
\]
and use this and similar identities to find $\psi$.
\end{problem}

\end{document}