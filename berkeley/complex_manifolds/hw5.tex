\documentclass[12pt]{article}
\usepackage{/Users/songye03/Desktop/Math_tex/style/psetconfig}

\DeclareMathOperator{\res}{res}
\title{Homework 5}
\author{Songyu Ye}
\date{\today}

\begin{document}
\psettitle

For Questions 1 and 2, you may use the correspondence indicated in class between the representation of $H^1$ classes by classes by principal parts versus Dolbeault distributions.
\begin{problem}[1]
For a compact Riemann surface $R$, verify that the Serre duality pairing
\[
H^1(R;\cO)\otimes H^0(R;\Omega^1)\longrightarrow \C
\]
defined by principal parts and residues agrees with the one given by integration
of Dolbeault representatives.

Using the relation to harmonic forms, explain how this relates to
Poincar\'e duality on $R$.
\end{problem}

\begin{solution} Choose a meromorphic function $f$ on $R$ whose principal part at each
$p_i$ with prescribed principal parts. Let $U_i$ be pairwise disjoint coordinate discs around $p_i$, and choose
$\chi\in C^\infty(R)$ such that $\chi\equiv 1$ on smaller discs $U_i'\subset U_i$
and $\chi\equiv 0$ outside $\bigcup_i U_i$.
Define a $(0,1)$-current
\[
T_f := \bar\partial(\chi f).
\]

Since $\bar\partial^2=0$, $T_f$ is $\bar\partial$-closed.
If we replace $f$ by $f+g$ for a global meromorphic function $g$
(with poles in $D$) or change $\chi$ within the same constraints,
$T_f$ changes by a current of the form $\bar\partial u$, so the
class $[T_f]$ in
\[
H^{0,1}_{\bar\partial}(R) \cong H^1(R,\cO)
\]
depends only on the underlying principal parts.

Let $\omega\in H^0(R,\Omega^1)$ be a holomorphic $1$-form.
The Dolbeault definition of the pairing is
\[
\langle\alpha,\omega\rangle_{Dol}
:= \frac{1}{2\pi i}\int_R T_f\wedge\omega
= \frac{1}{2\pi i}\int_R \bar\partial(\chi f)\wedge\omega.
\]
Since $\omega$ is of type $(1,0)$ and holomorphic,
$\bar\partial\omega=0$, hence
\[
\bar\partial(\chi f)\wedge\omega
= \bar\partial(\chi f\omega).
\]

Let $D_i\subset U_i'$ be small closed discs around $p_i$ and set
\[
R_\varepsilon := R\setminus\bigcup_i D_i(\varepsilon),
\]
where $D_i(\varepsilon)$ are concentric discs of radius $\varepsilon$.
On $R_\varepsilon$ the form $\chi f\omega$ is smooth with compact
support, so Stokes' theorem gives
\[
\int_{R_\varepsilon} \bar\partial(\chi f\omega)
= \int_{\partial R_\varepsilon} \chi f\omega
= -\sum_i \int_{\partial D_i(\varepsilon)} f\omega,
\]
the sign coming from the induced orientation on the boundary.

Letting $\varepsilon\to 0$ and using the residue theorem,
\[
\int_{\partial D_i(\varepsilon)} f\omega
\;\longrightarrow\; 2\pi i\,\Res_{p_i}(f\omega),
\]
we obtain
\[
\frac{1}{2\pi i}\int_R \bar\partial(\chi f)\wedge\omega
= \sum_i \Res_{p_i}(f\omega).
\]
This is precisely the principal parts definition of the Serre pairing.

Now equip $R$ with any Hermitian (necessarily K\"ahler) metric. Hodge theory yields the decompositions
\[
H^1_{\dR}(R,\C)
\;\cong\;
\cH^1(R) \;\cong\;
H^{1,0}_{\bar\partial}(R)\oplus H^{0,1}_{\bar\partial}(R),
\]
and every class has a unique harmonic representative.  Moreover,
\[
H^0(R,\Omega^1) \cong H^{1,0}_{\bar\partial}(R)
\]
consists of harmonic $(1,0)$-forms, and
\[
H^1(R,\cO) \cong H^{0,1}_{\bar\partial}(R)
\]
is represented by harmonic $(0,1)$-forms.  Complex conjugation gives
an isomorphism \[\overline{H^{1,0}_{\bar\partial}(R)}
\cong H^{0,1}_{\bar\partial}(R)\]

Poincaré duality on $R$ is given by the nondegenerate pairing
\[
H^1_{\dR}(R,\C)\times H^1_{\dR}(R,\C)
\longrightarrow \C,
\qquad
([\alpha],[\beta])\mapsto\int_R \alpha\wedge\beta.
\]
It is clear that $\alpha \wedge \beta$ is nonzero only if
$\alpha$ and $\beta$ are of complementary types, i.e. their wedge is of type
$(1,1)$, since $(1,0)\wedge(1,0)$ and $(0,1)\wedge(0,1)$ necessarily vanish.
Thus the Poincaré pairing restricts to a nondegenerate pairing
\[
H^{0,1}_{\bar\partial}(R)\;\otimes\; H^{1,0}_{\bar\partial}(R)
\longrightarrow \C,
\qquad
(\eta,\omega)\mapsto\int_R \eta\wedge\omega,
\]
with $\eta,\omega$ harmonic representatives.

Under the identifications
\[
H^1(R,\cO)\cong H^{0,1}_{\bar\partial}(R),
\qquad
H^0(R,\Omega^1)\cong H^{1,0}_{\bar\partial}(R),
\]
the Serre pairing of $\alpha$ and $\omega$ is
\[
\langle\alpha,\omega\rangle
= \frac{1}{2\pi i}\int_R \eta\wedge\omega,
\]
where $\eta$ is the harmonic $(0,1)$-representative of $\alpha$. In particular, on a compact Riemann surface the Serre duality
\[
H^1(R,\cO)\;\cong\;H^0(R,\Omega^1)^\vee
\]
is nothing but Poincaré duality in degree~$1$ up to the 
constant factor $2\pi i$, expressed via the Hodge
decomposition of $H^1_{\dR}(R,\C)$.
\end{solution}



\begin{problem}[2]
For a compact Riemann surface $R$, verify that the map
\[
H^1(R;\Z)\longrightarrow H^1(R;\cO)
\]
corresponds to the period map
\[
H_1(R;\Z)\otimes H^0(R;\Omega^1)\longrightarrow \C
\]
under integral Poincar\'e duality and Serre duality on $R$.
\end{problem}

\begin{solution}
Let $i:H^1(R;\Z)\to H^1(R;\cO)$ be the given homomorphism. We need to show
for every $c\in H^1(R;\Z)$ and $\omega\in H^0(R,\Omega^1)$, the Serre
pairing
\(
\langle i(c),\omega\rangle_{\mathrm{Serre}}
\)
equals the period of $\omega$ along the $1$-cycle Poincar\'e dual to
$c$.

By Hodge theory, every class in $H^1(R;\R)$ has a unique harmonic
representative.  An element $c\in H^1(R;\Z)$ maps to a real class
$c_\R\in H^1(R;\R)$ whose harmonic representative we denote by
$\alpha$ so
\[
[\alpha]_{\dR} = c_\R\in H^1_{\dR}(R;\R).
\]
Decompose $\alpha$
\[
\alpha = \alpha^{1,0}+\alpha^{0,1},
\qquad
\alpha^{0,1} = \overline{\alpha^{1,0}},
\]
since $\alpha$ is real.  Under the Dolbeault isomorphism and Hodge
decomposition, we have
\[
H^1(R,\cO)\;\cong\; H^{0,1}_{\bar\partial}(R)
\]
and the image $i(c)\in H^1(R,\cO)$ is represented by the harmonic $(0,1)$-form $\alpha^{0,1}$.

We know that the Serre pairing can be described as
\[
\langle\beta,\omega\rangle_{\mathrm{Serre}
}
=\frac{1}{2\pi i}\int_R \eta^{0,1}\wedge\omega
\]
whenever $\beta\in H^1(R,\cO)$ is represented by a harmonic
$(0,1)$-form $\eta^{0,1}$ and $\omega\in H^0(R,\Omega^1)$ is a
holomorphic $1$-form.

Applying this to $\beta=i(c)$ and $\eta^{0,1}=\alpha^{0,1}$ gives
\[
\langle i(c),\omega\rangle_{\mathrm{Serre}}
= \frac{1}{2\pi i}\int_R \alpha^{0,1}\wedge\omega.
\]
Since $R$ has complex dimension $1$, a $(2,0)$-form vanishes, hence
$\alpha^{1,0}\wedge\omega=0$, and therefore
\[
\alpha^{0,1}\wedge\omega
= (\alpha^{1,0}+\alpha^{0,1})\wedge\omega
= \alpha\wedge\omega.
\]
Thus
\begin{equation}
\label{eq:Serre-alpha}
\langle\iota^*c,\omega\rangle_{\mathrm{Serre}}
= \frac{1}{2\pi i}\int_R \alpha\wedge\omega.
\end{equation}
Integral Poincar\'e duality gives a perfect pairing
\[
H^1(R;\Z)\times H_1(R;\Z)\longrightarrow\Z,
\]
and we denote by $\gamma_c\in H_1(R;\Z)$ the Poincar\'e dual of $c$.

The de~Rham realization of this pairing is as follows.  The class
$c_\R\in H^1(R;\R)$ is represented by the closed $1$-form $\alpha$ with integral periods, i.e.
\[
\int_\gamma \alpha \in \Z
\quad\text{for all } \gamma\in H_1(R;\Z).
\]
The Poincar\'e dual cycle $\gamma_c$ is then characterized by
\[
\int_{\gamma_c} \beta
= \int_R \alpha\wedge\beta
\quad\text{for all closed $1$-forms }\beta,
\]

Thus, if we identify
\[
H^1(R;\Z)\xrightarrow{\;\text{PD}\;}\; H_1(R;\Z)
\quad\text{and}\quad
H^1(R;\cO)\xrightarrow{\;\text{Serre}\;}
H^0(R,\Omega^1)^\vee,
\]
the class $c\in H^1(R;\Z)$ maps to the functional
\[
H^0(R,\Omega^1)\longrightarrow\C,
\qquad
\omega\longmapsto \frac{1}{2\pi i}\int_{\gamma_c}\omega.
\]

This is precisely the period map (up to the factor $1/(2\pi i)$)
\[
H_1(R;\Z)\otimes H^0(R,\Omega^1)\longrightarrow\C,
\qquad
(\gamma,\omega)\longmapsto\int_\gamma\omega,
\]
with $\gamma=\gamma_c$ the Poincar\'e dual of $c$.
\end{solution}

\begin{problem}[3]
Show that the period mapping gives an isomorphism
\[
H_1(R;\Z)\xrightarrow{\ \sim\ } H_1(J;\Z),
\]
which can be realized geometrically by the Abel-Jacobi map
\[
R\longrightarrow J_1.
\]
Show that under this correspondence, $c_1(\Theta)\in\Lambda^2 H_1(R)$
is the intersection pairing on $R$.

\textit{Hints for the second part:} You can deduce it from the periodicity
formulas of the Riemann $\Theta$-function. Alternatively, you can find this
by exploiting the facts that the Poincar\'e dual of $c_1(\Theta)$ in $J_{g-1}$
is the Theta divisor, the image of $\Sym^{g-1}(R)$. The maps
\[
\Sym^g(R)\longrightarrow J_g
\qquad\text{and}\qquad
\Sym^{g-1}(R)\longrightarrow \operatorname{div}(\Theta)
\]
have degree $1$.
\end{problem}

\begin{solution}
The presentation of the Jacobian $J$ as
\[
J \cong H^1(R;\cO) / H_1(R;\Z)
\]
makes it clear that $H_1(J;\Z)$ is naturally identified with
$H_1(R;\Z)$, since the universal cover of $J$ is the vector space
$H^1(R;\cO)$. The period mapping 
\[
H_1(R;\Z)\to H_1(J;\Z)
\] is injective because of the Riemann bilinear relations, and since both groups are free abelian of rank $2g$, it is an isomorphism. Pick a base point $p_0\in R$ and define the Abel-Jacobi map
\[
\varphi:R\to J,
\qquad
p\mapsto \left[\omega\mapsto \int_{p_0}^p \omega\right].
\]
precisely implements the lift of the period mapping to the universal cover and hence induces the same isomorphism on $H_1$.

To identify $c_1(\Theta)$ with the intersection pairing on $H_1(R,\Z)$, we first note that by the universal coefficient theorem and the fact that $H^k(J,\Z) = \Alt^k(H_1(J,\Z),\Z)$
(the group law on $J$ induces a map $H_1(J,\Z) \otimes \dots \otimes H_1(J,\Z) \to H_k(J,\Z)$ as follows. For each $\alpha\in H_1(J,\mathbb Z)$ choose a loop
$\ell_\alpha: S^1\to J$
representing $\alpha$. For $\alpha_1,\dots, \alpha_k$, consider the map  $(S^1)^k \to J$ given by $(t_1,\dots,t_k) \mapsto \ell_{\alpha_1}(t_1) + \dots + \ell_{\alpha_k}(t_k)$. For orientation reasons, this map is alternating in the $\alpha_i$). We have
\[
\begin{aligned}
H^2(J,\Z)
&\cong \operatorname{Hom}(H_2(J,\Z),\Z)\\
&\cong \operatorname{Hom}(\Lambda^2 H_1(J,\Z),\Z)\\
&\cong \operatorname{Alt}^2(H_1(J,\Z),\Z) \xrightarrow{\iota^*} \operatorname{Alt}^2(H_1(R,\Z),\Z)
\end{aligned}
\] so indeed $c_1(\Theta)$ corresponds to an alternating bilinear form on $H_1(R,\Z)$. Pick a symplectic basis $\{a_1,\dots,a_g,b_1,\dots,b_g\}$ of $H_1(R,\Z)$, i.e.
\[a_i\cdot a_j = 0,
\quad b_i\cdot b_j = 0,
\quad a_i\cdot b_j = \delta_{ij}.\]


Under the identification
$H_1(R,\Z)\xrightarrow{\sim}\Lambda\cong H_1(J,\Z)$ coming from the period map and the Abel-Jacobi embedding, a homology class $\gamma\in H_1(R,\Z)$ corresponds to an integral vector $(m,n)\in\Z^{2g}$. The intersection pairing on $H_1(R,\Z)$ is given in these coordinates by
\[
(m,n)\cdot(m',n') \;=\; m^{\mathrm T}n' - m'^{\mathrm T}n.
\]
The Riemann theta function with period matrix $\tau$ is
\[
\theta(z\mid\tau)
:= \sum_{k\in\Z^g}
\exp\bigl(\pi i\,k^{\mathsf T}\tau k + 2\pi i\,k^{\mathsf T}z\bigr),
\quad z\in\C^g.
\]
The Riemann theta function satisfies the quasi-periodicity property:
\[
\theta\bigl(z + m + \tau n \mid \tau\bigr)
=
\exp\bigl(-\pi i n^{\mathsf T}\tau n - 2\pi in^{\mathsf T}z\bigr)
\theta(z\mid\tau).
\]
In particular, the Riemann theta function defines a holomorphic section of the line bundle $\mathcal{O}_J(\Theta)$.
Thus the factor of automorphy is
\[
e_u(z)
=\exp\bigl(-\pi i\,n^{\mathsf T}\tau n-2\pi i\,n^{\mathsf T}z\bigr)
=\exp\bigl(2\pi i f_u(z)\bigr)
\]
with
\[
f_u(z)=-\tfrac12\,n^{\mathsf T}\tau n - n^{\mathsf T}z,\qquad u=m+\tau n.
\]

The following lemma is standard.
\begin{lemma}[1 (Mumford)]
Let $U \subset V$ be a lattice in a complex vector space $V$. The map which associates to any map $F: U\times U \to \mathbb{Z}$ the
map $AF: U\times U\to \mathbb{Z}$ defined by
\[
  AF(u_{1},u_{2}) = F(u_{1},u_{2}) - F(u_{2},u_{1})
\]
maps the group of $2$-cocycles $Z^{2}(U,\mathbb{Z})$ into the space of
alternating linear maps $U\times U\to\mathbb{Z}$, and induces an
isomorphism
\[
  A:\;
  H^{2}(U,\mathbb{Z})
  \;\xrightarrow{\ \sim\ }\;
  \operatorname{Hom}(\Lambda^{2}U,\mathbb{Z})
  \;\cong\;
  \Lambda^{2}\operatorname{Hom}(U,\mathbb{Z}).
\]
Furthermore for $\xi,\eta\in\operatorname{Hom}(U,\mathbb{Z})=H^{1}(U,\mathbb{Z})$,
we have $A(\xi\smile\eta)=\xi\wedge\eta$.
\end{lemma} 
The proof of this lemma is given at the end of the solution.
\begin{lemma}[2 (Mumford)]
  If $e_u(z)=\exp(2\pi i f_u(z))$ are the factors of automorphy, then the Chern class
$c_1(L)\in H^2(J,\Z)\cong\Alt^2(\Lambda,\Z)$ corresponds to the form
\[
E(u_1,u_2)
= f_{u_2}(z+u_1)+f_{u_1}(z)-f_{u_1}(z+u_2)-f_{u_2}(z)
\]
which is bilinear, alternating, and independent of $z$.
\end{lemma}
Substituting into Mumford's formula and using the symmetry of $\tau$, the quadratic terms cancel, leaving
\[
E(u_1,u_2)= n_1^{\mathsf T}m_2 - n_2^{\mathsf T}m_1.
\]
recovering the intersection pairing on $H_1(R,\Z)$.

\begin{proof}
[Proof of Lemma 2]
A holomorphic line bundle on $J$ is equivalent to the trivial bundle
$V\times\C\to V$ endowed with the $\Lambda$--action
\[
u\cdot(z,\xi)=(z+u, e_u(z)\xi).
\]
Thus $L$ corresponds to the multiplicative $1$--cocycle $\{e_u\}$
satisfying the identity \[e_{u_1+u_2}(z)=e_{u_2}(z+u_1)e_{u_1}(z)\]

Consider the exponential sequence on $V$,
\[
0 \longrightarrow \Z
  \longrightarrow \cO_V
  \xrightarrow{\exp(2\pi i\,\cdot)} \cO_V^\times
  \longrightarrow 1.
\]
Passing to group cohomology of $\Lambda$ gives an exact sequence
\[
H^1(\Lambda,\cO^\times_V)
  \xrightarrow{\;\delta\;}
H^2(\Lambda,\Z),
\]
and it is standard that the Chern class $c_1(L)$ is precisely the
image under $\delta$ of the cocycle $\{e_u\}$.

Choose logarithms $f_u$ with $e_u=\exp(2\pi i f_u)$.  The formula for
$\delta$ on group cochains gives the $2$--cochain
\[
F(u_1,u_2)(z)
  := f_{u_2}(z+u_1)-f_{u_1+u_2}(z)+f_{u_1}(z).
\]
Because of the multiplicative cocycle identity taking
logarithms yields
\[
f_{u_1+u_2}(z)
  - f_{u_2}(z+u_1)
  - f_{u_1}(z)
  \in \Z.
\]
Hence $F(u_1,u_2)(z)$ takes values in $\Z$.  Being holomorphic and
integer--valued, it is constant in $z$.  Thus $F$ defines a function
\[
F:\Lambda\times\Lambda\longrightarrow\Z,
\]
and one checks directly that it satisfies the group cocycle condition
$\delta F=0$.  Therefore $F$ is a $2$--cocycle representing the Chern
class $c_1(L)$.

The previous lemma identifies
\[
H^2(\Lambda,\Z)\;\cong\;\Alt^2(\Lambda,\Z)
\]
via the alternation map
\[
AF(u_1,u_2)=F(u_1,u_2)-F(u_2,u_1).
\]
Applying this to the cocycle $F$ gives
\[
\begin{aligned}
E(u_1,u_2)
 &= F(u_1,u_2)-F(u_2,u_1) \\
 &= f_{u_2}(z+u_1)+f_{u_1}(z)
    -f_{u_1}(z+u_2)-f_{u_2}(z),
\end{aligned}
\]
which is therefore a representative of $c_1(L)$ in the alternating
form picture.  In particular $E$ is bilinear, alternating, and
independent of~$z$.
\end{proof}
\end{solution}


\begin{problem}[4]
Prove the following generalized Cauchy formula for a smooth function $f$
defined in the unit disk $\Delta$:
\[
f(z,\bar z)
=
\frac{1}{2\pi i}\oint_{|\zeta-z|=r}
\frac{f(\zeta)}{\zeta-z}\,d\zeta
+
\frac{1}{2\pi i}\iint_{\Delta'}
\frac{\partial f}{\partial\bar\zeta}\,
\frac{d\zeta\wedge d\bar\zeta}{\zeta-z},
\]
where $\Delta'\subset\Delta$ is the subdisk of radius $r<1$.
\end{problem}

\begin{solution}
We prove the two lemmas below at the end.
\begin{lemma}[1]
Let $X\subset\C$ be a bounded domain with smooth boundary $\partial X$.  Then
  \[
\frac{\partial \chi_X}{\partial \bar z}
= \frac{i}{2} \oint_{\partial X} dz,
\]
where the distribution on the right denotes contour integration along $\partial X$.
\end{lemma}

We can now deduce the generalized Cauchy integral formula.
Let
\[
u = \frac{\chi_X}{\pi(z-z_0)} \in L^1_{\mathrm{loc}}(X),
\qquad z_0\in X.
\]
where $L^1_{\mathrm{loc}}(X)$ is the space of locally integrable functions on $X$. We may apply the Leibniz rule to compute
\[
\frac{\partial u}{\partial \bar z}
= \frac{\partial}{\partial \bar z}\!\left(\frac{1}{\pi(z-z_0)}\right)\chi_X
+ \frac{1}{\pi(z-z_0)}\frac{\partial \chi_X}{\partial \bar z}.
\]
\begin{lemma}[2]
\[
\frac{\partial}{\partial \bar z}\!\left(\frac{1}{\pi(z-z_0)}\right)
= \delta_{z_0}
\]
\end{lemma}
It follows from the lemma and the identity above that
\[
\frac{\partial u}{\partial \bar z}
= \delta_{z_0} + \frac{1}{\pi(z-z_0)}\frac{\partial \chi_X}{\partial \bar z}.
\]
Applying both sides to $\varphi\in C^\infty_c(X)$ gives
\[
\Bigl\langle \frac{\partial u}{\partial \bar z},\,\varphi \Bigr\rangle
= \varphi(z_0)
+ \Bigl\langle \frac{\partial \chi_X}{\partial \bar z},
  \frac{\varphi}{\pi(z-z_0)}\Bigr\rangle
= \varphi(z_0)
+ \frac{i}{2}\oint_{\partial X}\frac{\varphi(z)}{\pi(z-z_0)}\,dz.
\]
Rearranging yields the desired generalized Cauchy formula:
\[
\varphi(z_0)
= \frac{1}{2\pi i}\oint_{\partial X}\frac{\varphi(z)}{z-z_0}\,dz
+ \frac{1}{2\pi i}\iint_X
   \frac{\partial \varphi}{\partial \bar z}(z)
   \frac{dx\wedge dy}{z-z_0}.
\] 
\end{solution}

\begin{proof}[Proof of Lemma 1]
For $\varphi \in C^\infty_c(X)$, compute
\[
\Bigl\langle \frac{\partial}{\partial \bar z} \chi_X,\, \varphi \Bigr\rangle
= - \int_X \frac{\partial \varphi}{\partial \bar z}\, dx\,dy
= -\frac{1}{2}\int_X (\partial_x \varphi + i\,\partial_y \varphi)\, dx\,dy.
\]
Let $\partial X$ be oriented counterclockwise, and parametrize it by arc length
$s\mapsto (x(s),y(s))$.  Denote by $\tau=(x'(s),y'(s))$ the unit tangent
vector and $\nu=(-y'(s),x'(s))$ the unit outward normal.  Define
$V=(\varphi,i\varphi)\in C^\infty_c(X)^2$, so that
\[
\operatorname{div} V = \partial_x \varphi + i\,\partial_y \varphi.
\]
By the divergence theorem,
\[
-\frac{1}{2}\int_X (\partial_x \varphi + i\,\partial_y \varphi)\,dx\,dy
= -\frac{1}{2}\int_{\partial X} V\cdot \nu\, ds
= -\frac{1}{2}\int_0^\ell (\varphi\nu_1 + i\varphi\nu_2)\,ds.
\]
Since $\nu=(-y',x')$, we get
\[
V\cdot \nu = \varphi(-y' + i x')
= i\,\varphi (x' + i y').
\]
Thus
\[
-\frac{1}{2}\int_X (\partial_x \varphi + i\,\partial_y \varphi)\,dx\,dy
= \frac{i}{2}\int_0^\ell \varphi(x(s),y(s))\,(x'(s)+i y'(s))\,ds
= \frac{i}{2}\oint_{\partial X} \varphi\,dz.
\]
Hence
\[
\frac{\partial \chi_X}{\partial \bar z}
= \frac{i}{2}\oint_{\partial X} dz.
\]
\end{proof}

\begin{proof}[Proof of Lemma 2]
Let $\varphi\in C^\infty_c(X)$ be a test function.  By definition,
\[
\Bigl\langle 
\frac{\partial}{\partial\bar z}\!\left(\frac{1}{\pi(z-z_0)}\right),
\,\varphi
\Bigr\rangle
:= 
-\,\frac{1}{\pi}\iint_{X\setminus\{z_0\}}
\frac{1}{z-z_0}\,
\frac{\partial\varphi}{\partial\bar z}\, dx\,dy.
\]
Remove a small disk $D_\varepsilon$ of radius $\varepsilon$ centered at
$z_0$, and write $A_\varepsilon=X\setminus D_\varepsilon$.  Then
\[
\Bigl\langle 
\frac{\partial}{\partial\bar z}\!\left(\frac{1}{\pi(z-z_0)}\right),
\varphi
\Bigr\rangle
=
-\frac{1}{\pi}\lim_{\varepsilon\to 0}
\iint_{A_\varepsilon}
\frac{1}{z-z_0}\,
\frac{\partial\varphi}{\partial\bar z}\, dx\,dy.
\]

We  know that \begin{align*}
  d\!\left(\frac{\varphi}{z-z_0} \, dz\right)
= d\!\left(\frac{\varphi}{z-z_0}\right)\wedge dz
= \frac{1}{z-z_0} d\varphi\wedge dz
+ \varphi\, d\!\left(\frac{1}{z-z_0}\right)\wedge dz.
\end{align*}
But 
\begin{align*}
  d\!\left(\frac{1}{z-z_0}\right)\wedge dz = g(z)\,dz\wedge dz = 0 \implies 
d\!\left(\frac{\varphi}{z-z_0} \, dz\right)
= \frac{1}{z-z_0} \, d(\varphi\,dz).
\end{align*}

This, together with the fact that
\[
\frac{\partial\varphi}{\partial\bar z}\,dx\,dy
=\frac{1}{2i}\, d(\varphi\,dz).
\]
gives upon application of Stokes' theorem,
\[
\iint_{A_\varepsilon}
\frac{1}{z-z_0}\,
\frac{\partial\varphi}{\partial\bar z}\, dx\,dy
=
\frac{1}{2i}
\int_{\partial A_\varepsilon}
\frac{\varphi(z)}{z-z_0}\,dz.
\]
The boundary $\partial A_\varepsilon$ consists of the small circle
$\partial D_\varepsilon$, oriented negatively.  Parametrizing
$\partial D_\varepsilon$ positively gives an extra minus sign, hence
\[
\iint_{A_\varepsilon}
\frac{1}{z-z_0}\,
\frac{\partial\varphi}{\partial\bar z}\, dx\,dy
=
-\frac{1}{2i}
\int_{|z-z_0|=\varepsilon}
\frac{\varphi(z)}{z-z_0}\,dz.
\]

Expand $\varphi$ near $z_0$:
\[
\varphi(z)=\varphi(z_0)+O(\varepsilon).
\]
Thus
\[
\int_{|z-z_0|=\varepsilon}
\frac{\varphi(z)}{z-z_0}\,dz
=
\varphi(z_0)\int_{|z-z_0|=\varepsilon}\frac{dz}{z-z_0}
+O(\varepsilon).
\]
But
\[
\int_{|z-z_0|=\varepsilon}\frac{dz}{z-z_0}=2\pi i.
\]
Hence letting $\varepsilon\to 0$,
\[
\lim_{\varepsilon\to0}
\int_{|z-z_0|=\varepsilon}
\frac{\varphi(z)}{z-z_0}\,dz
=2\pi i\,\varphi(z_0).
\]

Combining everything,
\[
\Bigl\langle 
\frac{\partial}{\partial\bar z}\!\left(\frac{1}{\pi(z-z_0)}\right),
\varphi
\Bigr\rangle
=
-\frac{1}{\pi}
\left(
-\frac{1}{2i}
\right)
(2\pi i)\,\varphi(z_0)
=
\varphi(z_0).
\]

This is precisely the defining property of the Dirac delta at $z_0$.
\end{proof}

\begin{problem}[5]
Let $L\to X$ be a holomorphic line bundle on a complex manifold, and let
$\alpha\in\cE^{0,1}$ be a $\bar\partial$-closed form. Show that the
re-defined operator
\[
\tilde{\bar\partial} = \bar\partial + \alpha
\]
on sections of $L$ defines a new holomorphic structure $L'$ on the same
underlying bundle, where local holomorphic sections are defined as those
killed by $\tilde{\bar\partial}$. Show that $L\simeq L'$ if $\alpha$ is
$\bar\partial$-exact. Relate this to the exponential sequence.

\textit{Remark:} For vector bundles, the same applies with an
$\alpha\in\cE^{0,1}(\End(V))$ satisfying the non-linear equation
\[
\bar\partial\alpha + \alpha\wedge\alpha = 0.
\]
The new bundle is isomorphic to the old one if
$\alpha = a^{-1}\bar\partial a$, for some smooth section $a$ of $\Aut(V)$.
\end{problem}

\begin{solution}
From the given holomorphic structure, we have a $\C$-linear map
\[
\bar\partial_L\colon \cE^0(L)\longrightarrow\cE^{0,1}(L)
\]
satisfying the Leibniz rule and the condition
$\bar\partial_L^2=0$.
We have the new operator
\[
\tilde{\bar\partial}s := \bar\partial s + \alpha\wedge s
\in\cE^{0,1}(L).
\]
First we check that $\tilde{\bar\partial}$ is a $\bar\partial$-operator.
For $f\in C^\infty(X)$ and $s\in\cE^0(L)$,
\[
\tilde{\bar\partial}(fs)
= \bar\partial(fs) + \alpha\,fs
= (\bar\partial f)\,s + f\,\bar\partial s + f\,\alpha s
= (\bar\partial f)\,s + f\,\tilde{\bar\partial}s,
\]
so the Leibniz rule holds.

Next, compute $\tilde{\bar\partial}^2$.  View $\bar\partial$ as a
derivation of degree $(0,1)$ on $\cE^{0,\bullet}(L)$; then for
$\beta\in\cE^{0,1}$ and $\eta\in\cE^{0,q}(L)$,
\[
\bar\partial(\beta\wedge\eta)
= (\bar\partial\beta)\wedge\eta - \beta\wedge\bar\partial\eta.
\]
Hence, for a section $s\in\cE^0(L)$,
\begin{align*}
\tilde{\bar\partial}^2 s
&= \tilde{\bar\partial}\bigl(\bar\partial s + \alpha \wedge s\bigr) \\
&= \bar\partial(\bar\partial s + \alpha \wedge s)
   + \alpha\wedge(\bar\partial s + \alpha \wedge s) \\
&= \bar\partial^2 s
   + \bar\partial(\alpha \wedge s)
   + \alpha\wedge\bar\partial s
   + \alpha\wedge\alpha \wedge s \\
&= 0 + (\bar\partial\alpha)\wedge s - \alpha\wedge\bar\partial s
   + \alpha\wedge\bar\partial s + \alpha\wedge\alpha \wedge s \\
&= (\bar\partial\alpha)\wedge s + \alpha\wedge\alpha \wedge s.
\end{align*}
By assumption $\bar\partial\alpha=0$, and since $\alpha$ is a $1$-form,
$\alpha\wedge\alpha=0$.  Thus $\tilde{\bar\partial}^2s=0$ for all $s$,
so $\tilde{\bar\partial}^2=0$ and $\tilde{\bar\partial}$ is a
$\bar\partial$-operator.  It therefore defines a new holomorphic
structure $L'$ on the same underlying smooth bundle, whose local
holomorphic sections are those killed by $\tilde{\bar\partial}$.

Now we check that if $\alpha$ is $\bar\partial$-exact, then $L'\simeq L$.
Suppose $\alpha=\bar\partial\phi$ for some smooth complex-valued
function $\phi$.  Define an automorphism of the \(C^\infty\) line bundle
$L$ by multiplication with $e^\phi$:
\[
F\colon L\longrightarrow L,\qquad s\longmapsto e^\phi s.
\]
We claim that $F$ is an isomorphism of holomorphic line bundles
$L'\to L$, i.e.
\[
\bar\partial(Fs) = F(\tilde{\bar\partial}s)
\quad\text{for all }s.
\]
Indeed,
\[
\bar\partial(Fs)
= \bar\partial(e^\phi s)
= e^\phi(\bar\partial\phi\wedge s + \bar\partial s)
= e^\phi(\alpha\wedge s + \bar\partial s)
= F(\tilde{\bar\partial}s).
\]
Thus $F$ is holomorphic with respect to $\tilde{\bar\partial}$ on the
domain and $\bar\partial$ on the target, so $L'\simeq L$.

The $(0,1)$-form $\alpha$ is $\bar\partial$-closed, so it defines a
Dolbeault cohomology class
\[
[\alpha]\in H^{0,1}_{\bar\partial}(X)
\cong H^1(X;\cO).
\]
Changing $\alpha$ by a $\bar\partial$-exact form does not change this
class, and by the computation above such a change yields an isomorphic
holomorphic structure.  Thus the isomorphism class of the new line
bundle $L'$ depends only on $[\alpha]$.

Recall the holomorphic exponential sequence
\[
0\longrightarrow \Z
\longrightarrow \cO
\xrightarrow{\ \exp(2\pi i \cdot)\ } \cO^\times
\longrightarrow 1,
\]
whose long exact cohomology sequence contains
\[
H^1(X;\cO) \xrightarrow{\ \exp\ } H^1(X;\cO^\times),
\]
and $H^1(X;\cO^\times)\cong\Pic(X)$ classifies holomorphic line bundles.
The class $[\alpha]\in H^1(X;\cO)$ maps under the exponential to the
class of the holomorphic line bundle $L'\otimes L^{-1}$. 
\end{solution}

\begin{problem}[6]
Let $V$ be a complex $g$-dimensional vector space and
$L\simeq\Z^{2g}\subset V$ a lattice. Let $A = V/L$.

\begin{enumerate}
\item Using harmonic theory, compute the Dolbeault cohomology $H^\ast(A;\cO)$.

\item Show that the moduli space of holomorphic line bundles on $A$ with zero
Chern class is naturally identified with
\[
A^\vee := \bar{V}^\vee / L^\vee.
\]

\item Show that the moduli space of holomorphic line bundles on $A^\vee$ with zero Chern class is naturally identified with $A$.

\item Define a line bundle
\[
\mathcal{P}\longrightarrow A\times A^\vee
\]
from the trivial line bundle over $V\times V^\vee$ with connection
\[
\nabla = d + i(x\,d\xi + \xi\,dx),
\]
by dividing the $L\times L^\vee$-action as follows: identify the
fiber $\C$ over $(x,\xi)\in V\times V^\vee$ with that over
$(x+\ell,\xi+\lambda)$ by multiplication by
\[
\exp\bigl(2\pi i(\lambda(x)+\xi(\ell))\bigr).
\]
Show that $\mathcal{P}$ is holomorphic, that
$\mathcal{P}|_{A\times\{a^\vee\}}$ is the line bundle over $A$ classified
by $a^\vee\in A^\vee$, and prove the corresponding statement for
$\{a\}\times A^\vee$.
\end{enumerate}
\end{problem}

\begin{solution}
  Write $g=\dim_\C V$.  Choose a Hermitian inner product on $V$ which is
$L$-invariant. By translation invariance and the locality of Kahler
geometry, this induces a flat Kahler metric on $A$.  
  \begin{enumerate}
    \item The Dolbeault Laplacian $\Box_{\bar\partial}$ on $(0,q)$-forms is
translation invariant.  On a flat torus, a $(0,q)$-form is harmonic iff it
has constant coefficients. This is because in a global parallel frame, the Laplacian on forms acts coefficientwise as the usual scalar Laplacian and the that any harmonic function on a compact manifold is constant by the maximum principle.

A translation-invariant $(0,1)$-form on $A$ is determined by its value at a single point (say $0$), and conversely any linear functional on $T_0^{0,1}A$ extends uniquely to a translation-invariant $(0,1)$-form.
Thus
\[
\cH^{0,q}(A)\cong \Lambda^q (T_0^{0,1}A)^\vee
\cong \Lambda^q \overline{V}^\vee,
\]
where $\overline{V}$ is $V$ with the conjugate complex structure.
By Hodge theory,
\[
H^{0,q}(A) \cong \cH^{0,q}(A),
\]
and since $H^q(A;\cO) \cong H^{0,q}(A)$ we obtain
\[
H^q(A;\cO) \;\cong\; \Lambda^q\overline{V}^\vee \cong \Lambda^q V^\vee,
\qquad 0\le q\le g.
\]
More generally, the same reasoning shows
\[
H^{p,q}(A) \;\cong\; \Lambda^p V^\vee \otimes \Lambda^q\overline{V}^\vee
\]


\item Isomorphism classes of holomorphic line bundles on $A$ are classified by
\[
\Pic(A) \cong H^1(A;\cO^\times).
\]
Consider the holomorphic exponential sequence
\[
0\longrightarrow \Z 
\longrightarrow \cO
\xrightarrow{\ \exp(2\pi i\cdot)\ }
\cO^\times
\longrightarrow 1,
\]
with long exact cohomology sequence
\[
H^1(A;\cO) \longrightarrow H^1(A;\cO^\times)
\xrightarrow{c_1} H^2(A;\Z).
\]
The subgroup
\[
\Pic^0(A)\;:=\;\ker\bigl(c_1:\Pic(A)\to H^2(A;\Z)\bigr)
\]
is precisely those bundles with zero Chern class. Exactness shows
\[
\Pic^0(A) \;\cong\; H^1(A;\cO)/\operatorname{im}\bigl(H^1(A;\Z)\bigr)
\]
We have already computed
\[
H^1(A;\cO)\cong H^{0,1}(A)\cong \overline{V}^\vee 
\]
On the other hand, $H^1(A;\Z)\cong \Hom_\Z(L,\Z)\cong L^\vee$, and this
sits inside $H^1(A;\cO)$ as a lattice (one way to see this is via
$H^1(A;\Z)\subset H^1(A;\R)\subset H^1(A;\C)$ and the Hodge decomposition).
Thus
\[
\Pic^0(A)
\cong 
H^1(A;\cO)/H^1(A;\Z)
\cong
\overline{V}^\vee / L^\vee
=
A^\vee.
\]
So the moduli space of holomorphic line bundles on $A$ with $c_1=0$ is
canonically identified with $A^\vee$.

\item Apply the same argument to $A^\vee = V^\vee/L^\vee$.
\item
Consider the trivial line bundle
\[
V\times V^\vee \times \C \longrightarrow V\times V^\vee
\]
equipped with the connection
\[
\nabla = d + i(x\,d\xi + \xi\,dx),
\]
where $x\in V$, $\xi\in V^\vee$, and $x\,d\xi+\xi\,dx$ denotes the
tautological pairing (so in coordinates it is linear in $x$ and $\xi$).

The group $L\times L^\vee$ acts on $V\times V^\vee\times\C$ by
\[
(\ell,\lambda)\cdot(x,\xi,z)
= \bigl(x+\ell,\;\xi+\lambda,\;
   e^{2\pi i(\lambda(x)+\xi(\ell))}\,z\bigr).
\]
The multiplier
\(
e^{2\pi i(\lambda(x)+\xi(\ell))}
\)
is holomorphic in $(x,\xi)$ (it is the exponential of a holomorphic linear
function), so this action is by holomorphic bundle automorphisms of the
trivial holomorphic line bundle $V\times V^\vee\times\C$.

The quotient
\[
\mathcal{P}
:= (V\times V^\vee\times\C)/(L\times L^\vee)
\longrightarrow (V/L)\times(V^\vee/L^\vee)
= A\times A^\vee
\]
is therefore a holomorphic line bundle, the Poincar\'e bundle.  The
connection $\nabla$ is invariant under the $L\times L^\vee$-action, hence
descends to a connection on $\mathcal{P}$.

Now fix $a^\vee\in A^\vee$ and choose a lift $\xi_0\in V^\vee$.
The restriction $\mathcal{P}|_{A\times\{a^\vee\}}$ is obtained as the
quotient of $V\times\C$ by the following $L$-action: an element
$\ell\in L$ sends $(x,z)$ to
\[
(x+\ell,\;e^{2\pi i\,\xi_0(\ell)}z),
\]
because along the slice $\xi=\xi_0$ the factor $\lambda(x)$ vanishes
(we take $\lambda=0$ to stay in the same fiber over $a^\vee$), while
$\xi(\ell)=\xi_0(\ell)$ is constant in $x$.  Thus the monodromy of the
resulting line bundle around $\ell\in L$ is exactly
\[
\chi_{\xi_0}(\ell) = e^{2\pi i\,\xi_0(\ell)}.
\]
By part (2), a holomorphic line bundle on $A$ with zero Chern class is
classified precisely by such a character $L\to U(1)$, and changing
$\xi_0$ by an element of $L^\vee$ does not change the character
$\chi_{\xi_0}$.  Hence the isomorphism class of
$\mathcal{P}|_{A\times\{a^\vee\}}$ depends only on the class
$a^\vee = [\xi_0]\in A^\vee$ and is exactly the line bundle over $A$
classified by $a^\vee$.

The argument for the restriction to $\{a\}\times A^\vee$ is symmetric.
Fix $a\in A$ with lift $x_0\in V$.  Then the effective $L^\vee$-action
on $V^\vee\times\C$ along the slice $x=x_0$ is
\[
\lambda\cdot(\xi,z)
= \bigl(\xi+\lambda,\; e^{2\pi i\,\lambda(x_0)} z\bigr),
\]
so the monodromy around $\lambda\in L^\vee$ is
\[
\chi_{x_0}(\lambda) = e^{2\pi i\,\lambda(x_0)}.
\]
This is precisely the character of $L^\vee$ corresponding to the point $a=[x_0]\in A$, and hence
$\mathcal{P}|_{\{a\}\times A^\vee}$ is the line bundle on $A^\vee$
classified by $a$.
  \end{enumerate}
\end{solution}


\begin{problem}[7]
Show that, in the case of the Jacobian $J$ of a Riemann surface $R$,
one has a natural isomorphism $J \simeq J^\vee$.

\textit{Hint:} Remember the natural Hilbert space structure on holomorphic
differentials.

\textit{Remark:} This self-duality is a property of principally polarized
Abelian varieties, those $A$ equipped with a positive line bundle having a
single holomorphic section (the $\Theta$-function).
\end{problem}

\begin{solution}
Let $R$ be a compact Riemann surface of genus $g$ and $J$ its Jacobian.
Recall that
\[
J \;\simeq\; V/\Lambda,
\qquad
V := H^0(R,\Omega^1)^\vee,
\]
where the lattice $\Lambda$ is the image of $H_1(R;\Z)$ under the period
map
\[
H_1(R;\Z)\longrightarrow H^0(R,\Omega^1)^\vee,
\qquad
\gamma\longmapsto\bigl(\omega\mapsto\textstyle\int_\gamma\omega\bigr).
\]
There is a natural Hermitian inner product on the space of holomorphic
differentials
\[
\langle\omega,\eta\rangle
:= \frac{i}{2}\int_R \omega\wedge\overline{\eta},
\qquad
\omega,\eta\in H^0(R,\Omega^1).
\]
This is positive definite and defines a Hilbert space structure on
$H^0(R,\Omega^1)$. This inner product gives a linear isomorphism
\[
\bar\rho\colon \overline{H^0(R,\Omega^1)} \xrightarrow{\ \sim\ } V.
\]
Dualizing, we obtain an anti\-linear isomorphism
\[
\overline V^\vee \;\xrightarrow{\ \sim\ }\; H^0(R,\Omega^1).
\]
The intersection pairing on $H_1(R;\Z)$ is unimodular, so it induces an
isomorphism
\[
H_1(R;\Z) \xrightarrow{\ \sim\ } H^1(R;\Z)
\cong \Hom(H_1(R;\Z),\Z).
\]
Translating this through the identification $\Lambda\cong H_1(R;\Z)$, we
obtain a canonical isomorphism of lattices
\[
\Lambda \xrightarrow{\ \sim\ } \Lambda^\vee.
\]
Recall that the first Chern class $c_1(\Theta)\in H^2(J;\Z)$ of the theta line
  bundle corresponds, under the isomorphism
  \(H^2(J;\Z)\cong\Alt^2(H_1(J;\Z),\Z)\), to the
  intersection form
  \[
  H_1(R;\Z)\times H_1(R;\Z)\longrightarrow\Z.
  \]
  This implies that the inner product on holomorphic differentials actually refines the intersection pairing on $H_1(R;\Z)$. Therefore the isomorphism $\overline V^\vee \;\xrightarrow{\ \sim\ }\; H^0(R,\Omega^1)$ actually descends to an isomorphism of complex tori
\[J^\vee = \overline V^\vee / \Lambda^\vee
\xrightarrow{\ \sim\ }
H^0(R,\Omega^1) / \Lambda
\xrightarrow{\ \sim\ }J.\]
\end{solution}

\begin{problem}[8]
\begin{enumerate}
  \item Given a holomorphic line bundle $\mathcal{L}$ on a complex manifold and a
smooth real closed $2$-form $\omega$ in the cohomology class of
$c_1(\mathcal{L})$, prove that there exists a Hermitian metric on
$\mathcal{L}$ whose holomorphic connection has curvature $-2\pi i\,\omega$.

\item Conclude (from Kodaira vanishing) that the holomorphic line bundles on a
compact Riemann surface $R$ which carry metrics of positive curvature are
precisely those of positive degree.

\item Show also that for every holomorphic vector bundle $V$ on $R$, there exists
a $d$ so that the twisted bundle $V(D)$ has no $H^1$ for any $D>d$.
\end{enumerate}
\end{problem}

\begin{solution}
  \begin{enumerate}
    \item 
 Let $L\to X$ be a holomorphic line bundle on a complex manifold and let
$\omega$ be a smooth real closed $2$--form representing $c_1(L)\in H^2(X;\R)$.
Choose any Hermitian metric $h_0$ on $L$ and let $F_0$ denote the curvature of
its Chern connection.  Then
$
\frac{i}{2\pi}F_0\in\Omega^{1,1}(X,\R)$ represents the class $c_1(L)$
Since $\omega$ also represents $c_1(L)$, the difference
\[
\omega-\frac{i}{2\pi}F_0
\]
is an exact real $(1,1)$--form.  By the $\partial\bar\partial$--lemma (which
holds on every complex curve and more generally on K\"ahler manifolds), there
exists a real-valued smooth function $\varphi$ such that
\[
\omega-\frac{i}{2\pi}F_0=\frac{i}{2\pi}\,\partial\bar\partial\varphi.
\]

Define a new Hermitian metric $h$ by $h=e^{-\varphi}h_0$. In a local holomorphic frame $l$, if $h(l,l)=e^{-\phi}$
the curvature is $F_h=\partial\bar\partial\phi$.
Thus
\[
F_h = F_0 + \partial\bar\partial\varphi
     = -2\pi i\,\omega.
\]
Hence $h$ is a Hermitian metric whose Chern connection has curvature $-2\pi i\,\omega$.

\item
For a Hermitian metric $h$ on a holomorphic line bundle $L\to R$, the degree is
\[
\deg(L)=\int_R c_1(L)
       = \int_R \frac{i}{2\pi}F_h.
\]
If $h$ has positive curvature, then the form $\frac{i}{2\pi}F_h$ is positive on
$R$, hence its integral is strictly positive and $\deg(L)>0$.

The hyperplane bundle $\cO_{\P^N}(1)$ carries the Fubini--Study Hermitian
metric $h_{\mathrm{FS}}$ whose curvature form $F_{\mathrm{FS}}$ is a
positive $(1,1)$-form.  Pulling back gives a metric $h_m = \Phi_m^*
h_{\mathrm{FS}}$ on $L^{\otimes m}$ with positive curvature
\[
F_{h_m} = \Phi_m^* F_{\mathrm{FS}} > 0.
\]

Now define a Hermitian metric $h$ on $L$ by taking an $m$th root locally:
in a local holomorphic frame $e$ of $L$, write
\[
h_m(e^{\otimes m}, e^{\otimes m}) = e^{-\phi_m}
\]
and set
\[
h(e,e) := e^{-\phi_m/m}.
\]
Then the curvature satisfies
\[
F_h = \frac{1}{m} F_{h_m},
\]
which is still a positive $(1,1)$-form.  Thus $L$ admits a Hermitian metric
of positive curvature. Therefore the holomorphic line bundles on $R$ which carry metrics of positivecurvature are precisely those of positive degree.

\item This follows immediately from Serre duality and the Riemann--Roch theorem.  For a holomorphic vector bundle $V$ on $R$ and a divisor $D$, Serre duality gives
\[H^1(R,V(D)) \cong H^0(R, K_R \otimes V^\vee \otimes \cO(-D))^\vee.\]
By Riemann--Roch, for $\deg(D)$ sufficiently large, the degree of the bundle $K_R \otimes V^\vee \otimes \cO(-D)$ becomes negative, and hence
\[H^0(R, K_R \otimes V^\vee \otimes \cO(-D)) = 0.\]
Thus, for such $D$, we have $H^1(R,V(D))=0$.  
\end{enumerate}
\end{solution}

\begin{problem}[9]
Show that isomorphism classes of flat unitary line bundles on a
manifold $X$ are classified by $H^1(X;U(1))$, with the constant sheaf $U(1)$
associated to the unit circle group in $\C^\times$.

When $X$ is compact K\"ahler, compare the constant and holomorphic
exponential sequences to conclude that the map
\[
H^1(X;U(1))\longrightarrow H^1(X;\cO^\times)
\]
induces a bijection from isomorphism classes of flat unitary line bundles to
those of holomorphic line bundles with zero Chern class.

\textit{Remark:} You probably need the Hodge decomposition theorem for the
second part.
\end{problem}

\begin{solution}
A flat unitary line bundle on $X$ is a complex line bundle with structure
group $U(1)$ and a flat unitary connection.  Choosing a good open cover
$\{U_i\}$, such a bundle is given by locally constant transition functions
$g_{ij}\colon U_{ij}\to U(1)$ satisfying the cocycle condition
$g_{ij}g_{jk}g_{ki}=1$ on $U_{ijk}$, and two such bundles are isomorphic if
their cocycles differ by a coboundary $g'_{ij}=h_i^{-1}g_{ij}h_j$ with
$h_i\colon U_i\to U(1)$ locally constant.  This is exactly the description
of Čech 1--cocycles and coboundaries for the constant sheaf $U(1)$, so
isomorphism classes of flat unitary line bundles are classified by
\[
H^1(X;U(1)).
\]
Now suppose $X$ is compact K\"ahler.  Consider the constant exponential
sequence
\[
0\longrightarrow \Z \longrightarrow \R
\xrightarrow{\ \exp(2\pi i\cdot)\ }
U(1)\longrightarrow 1
\]
and the holomorphic exponential sequence
\[
0\longrightarrow \Z \longrightarrow \cO
\xrightarrow{\ \exp(2\pi i\cdot)\ }
\cO^\times\longrightarrow 1.
\]
The inclusions $\R\hookrightarrow\cO$ and $U(1)\hookrightarrow\cO^\times$
give a morphism of short exact sequences and so we get a commutative diagram of long exact cohomology sequences:
\begin{align*}
\begin{array}{ccccccccc}
  H^1(X;\Z) &\to H^1(X;\R) &\to H^1(X;U(1)) &\xrightarrow{c_1^{\mathrm{top}}} &H^2(X;\Z)
  \\  
  \downarrow &\downarrow &\downarrow & & \downarrow
  \\
  H^1(X;\Z) &\to H^1(X;\cO) &\to H^1(X;\cO^\times) &\xrightarrow{c_1^{\mathrm{hol}}} &H^2(X;\Z)
\end{array}
\end{align*}Let
\[
\Pic^0(X):=\ker\bigl(c_1^{\mathrm{hol}}\bigr)
\]
be the subgroup of holomorphic line bundles with $c_1=0$.

There is a commutative diagram
\begin{align*}
\begin{array}{ccc}
H^1(X;\R)/H^1(X;\Z) &\cong & \ker c_1^{\mathrm{top}}
\\
\downarrow & &\downarrow
\\
H^1(X;\cO)/H^1(X;\Z) & = & \Pic^0(X)
\end{array}
\end{align*}
The left vertical map is an isomorphism by the Hodge decomposition theorem. Thus the right vertical map is also an isomorphism, and we conclude that every holomorphic line bundle with zero Chern class corresponds to a flat unitary line bundle with zero Chern class.
\end{solution} 

\begin{problem}[10]
Prove the global $\partial\bar\partial$-Lemma on a compact K\"ahler
manifold $X$: for any $d$-exact form $\varphi\in\cE^{p,q}$, there exists
$\psi\in\cE^{p-1,q-1}$ with
\[
\partial\bar\partial\psi = \varphi.
\]

\textit{Hint:} Show that
\[
\varphi = \partial\bar\partial^\ast \Box \varphi
\]
and use this and similar identities to find $\psi$.
\end{problem}

\begin{solution}
Fix a K\"ahler metric on $X$ and let $\Delta_d$, $\Delta_{\partial}$ and $\Delta_{\bar\partial}$ denote the $d$-, $\partial$- and $\bar\partial$-Laplacians. On a K\"ahler manifold we have the identities
\[
\Delta_d = 2\Delta_{\partial} = 2\Delta_{\bar\partial}
\]
acting on each $\cE^{p,q}$, and these Laplacians commute with $\partial$ and $\bar\partial$.


We define the Green operator $G_{\bar\partial}: \cE^{p,q}(X) \to \cE^{p,q}(X)$ as the inverse of $\Delta_{\bar\partial}$ on the orthogonal complement of the space of $\bar\partial$-harmonic forms, and zero on the harmonic forms.

Regard $\varphi$ as a $\bar\partial$-exact form. Since its Dolbeault class in $H^{p,q}_{\bar\partial}(X)$ is zero, its $\bar\partial$-harmonic part vanishes and we can write
\[
\varphi = \Delta_{\bar\partial} G_{\bar\partial}\varphi
       = (\bar\partial\bar\partial^* + \bar\partial^*\bar\partial)G_{\bar\partial}\varphi,
\]
Using that $\partial$ commutes with $\Delta_{\bar\partial}$ on a K\"ahler manifold, we have
\[
0 = \partial\varphi = \partial\Delta_{\bar\partial} G_{\bar\partial}\varphi = \Delta_{\bar\partial} \partial G_{\bar\partial}\varphi,
\]
so $\partial G_{\bar\partial}\varphi$ is $\bar\partial$-harmonic and $\bar\partial$-exact, hence $\partial G_{\bar\partial}\varphi=0$. It follows that $\bar\partial^*\bar\partial G_{\bar\partial}\varphi=0$ and
\[
\varphi = \bar\partial\bar\partial^* G_{\bar\partial}\varphi.
\]

Now use the K\"ahler identity $\bar\partial^* = -i[\Lambda,\partial]$, where $\Lambda$ is contraction with the K\"ahler form. Then
\[
\varphi = -i\,\bar\partial[\Lambda,\partial]G_{\bar\partial}\varphi
        = -i\bigl(\bar\partial\Lambda\partial - \bar\partial\partial\Lambda\bigr)G_{\bar\partial}\varphi.
\]
Since $\partial G_{\bar\partial}\varphi=0$, the first term vanishes, and using $\bar\partial\partial = -\partial\bar\partial$ we obtain
\[
\varphi = i\,\partial\bar\partial\bigl(\Lambda G_{\bar\partial}\varphi\bigr).
\]
Thus, if we set
\[
\psi := i\,\Lambda G_{\bar\partial}\varphi \in \cE^{p-1,q-1}(X),
\]
then
\[
\partial\bar\partial\psi = \varphi.
\]
\end{solution}

\end{document}