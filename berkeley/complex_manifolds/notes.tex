\documentclass[12pt]{article}
\usepackage[english]{babel}
\usepackage[utf8x]{inputenc}
\usepackage[T1]{fontenc}
\usepackage{listings}
\usepackage{bookmark}
\usepackage{tikz}
\usepackage{/Users/songye03/Desktop/math_tex/style/quiver}
\usepackage{/Users/songye03/Desktop/math_tex/style/scribe}
\usepackage{fancyhdr}

\usepackage{parskip} % Automatically respects blank lines
\setlength{\parskip}{1em} % Adds more space between paragraphs
\setlength{\parindent}{0pt} % Removes paragraph indentation

\begin{document}


\lhead{Songyu Ye}
\rhead{\today}
\cfoot{\thepage}

\title{Title}

\author{Songyu Ye}
\date{\today}
\maketitle


\begin{abstract}
Abstract
\end{abstract}

\tableofcontents

\section{Introduction}
\begin{theorem}
The following categories are equivalent:
\begin{itemize}
    \item Compact Riemann surfaces with nonconstant holomorphic maps
    \item Smooth proper (and hence projective) algebraic curves over $\mathbb{C}$ with nonconstant morphisms
    \item Field extensions of $\mathbb{C}$ of transcendence degree $1$, of finite degree over $\mathbb{C}(t)$ where $t$ is transcendental over $\mathbb{C}$, with field homomorphisms over $\mathbb{C}$
\end{itemize}

The correspondence in one direction is:
\begin{align*}
    \text{Riemann surface } S &\mapsto \text{ function field } \mathbb{C}(S) \\
    \text{Holomorphic map } f: S \to S' &\mapsto \text{ field homomorphism } f^*: \mathbb{C}(S') \to \mathbb{C}(S) \\
\end{align*}
\end{theorem}

\begin{remark}
    For curves, smooth and proper implies projective. This is false in higher dimensions.
\end{remark}

Common to both is the construction of nonconstant meromorphic functions. It suffices to find \begin{itemize}
    \item A map $f:R \to \P^1$ which realizes $R$ as a branched cover of $\P^1$ (the transcendental part of the function field) \begin{align*}
        f^*: \mathbb{C}(z) &\hookrightarrow \mathbb{C}(R) \\
        z &\mapsto f
    \end{align*}
    \item A nonconstant meromorphic function $g$ on $S$ which separates the sheets (the finite part of the function field)
\end{itemize}
Once you have these functions, consider the set of pairs $\{(f(p), g(p)) : p \in S\} \subset \mathbb{P}^1 \times \mathbb{P}^1$. This is an analytic curve. By a theorem of Riemann (or later by Chow's theorem), an analytic curve in projective space is algebraic. So there exists a nonzero polynomial $P(x,y)$ such that
\[
P(f,g) = 0 \quad \text{on } S.
\]
Thus, the image of $S$ under $(f,g)$ is contained in the algebraic curve $P(x,y)=0$. Moreover, because $g$ separates the sheets, $(f,g)$ is generically injective, so the map is birational. Hence $S$ and the curve $P(x,y)=0$ have the same function field. So you've now explicitly realized $\mathbb{C}(S) = \mathbb{C}(f,g)$.

\section{}
We state Riemann's theorem which allows us to pass from the analytic setting to the algebraic setting.
\begin{theorem}
    Let $R$ be a compact Riemann surface and $p\in R$. There exists a meromorphic function $f$ with poles of arbitrary order $n$ at $p$ and holomorphic elsewhere, provided that $n$ is sufficiently large.
\end{theorem}

The method of proof involves constructing holomorphic differentials with poles at $p$, and in fact one can get them to any order of pole $\geq 2$. Then if these differentials are exact, their integrals give a single valued function with pole only at $p$.
\end{document}