\documentclass[12pt]{article}
\usepackage[english]{babel}
\usepackage[utf8x]{inputenc}
\usepackage[T1]{fontenc}
\usepackage{listings}
\usepackage{bookmark}
\usepackage{tikz}
\usepackage{/Users/songye03/Desktop/math_tex/style/quiver}
\usepackage{/Users/songye03/Desktop/math_tex/style/scribe}
\usepackage{fancyhdr}

\usepackage{parskip} % Automatically respects blank lines
\setlength{\parskip}{1em} % Adds more space between paragraphs
\setlength{\parindent}{0pt} % Removes paragraph indentation

\begin{document}


\lhead{Songyu Ye}
\rhead{\today}
\cfoot{\thepage}

\title{Derived Categories and Bondal--Orlov Reconstruction}

\author{Songyu Ye}
\date{\today}
\maketitle


\begin{abstract}
These notes discuss the concept of derived categories in the context of algebraic geometry, particularly focusing on the Bondal--Orlov reconstruction theorem. We explore how derived categories encapsulate information about coherent sheaves on smooth projective varieties and how this framework leads to a powerful reconstruction result under certain positivity conditions on the canonical bundle.
\end{abstract}

\tableofcontents

\section{Algebraic vs analytic}
My talk will be about algebraic geometry, but the course is about complex manifolds, so let me start with a brief comparison of the two subjects.

We had the notion of a Hodge manifold: a compact Kahler manifold $X$ which admits an ample line bundle $\mathcal{L}$. By Kodaira's embedding theorem, such an $X$ is biholomorphic to a smooth projective variety. Conversely, any projective variety over $\mathbb{C}$ (by definition a closed subvariety of projective space $\mathbb{P}^n$) which is smooth is a Hodge manifold by the pull back of $\mathcal{O}_{\mathbb{P}^n}(1)$. We also defined coherent analytic sheaves on complex manifolds, and coherent sheaf cohomology $H^i(X,\mathcal{F})$ for coherent analytic sheaves $\mathcal{F}$.

This talk will focus on coherent algebraic sheaves on smooth projective varieties, and their derived categories. The two theories are closely related: by Serre's GAGA theorem, for a smooth projective variety $X$ over $\mathbb{C}$, the category of coherent algebraic sheaves on $X$ is equivalent to the category of coherent analytic sheaves on the associated complex manifold $X^{an}$. Moreover, the cohomology groups $H^i(X,\mathcal{F})$ computed in either category are isomorphic.

\section{Derived Categories}
We've spent a lot of time in this course on sheaf cohomology: $H^i(X, \mathcal{F})$, defined via resolutions: injective resolutions, Čech resolutions, Dolbeault resolutions on complex manifolds, etc.

A basic pattern is: whenever we have a left exact functor like
\[
\Gamma(X, -), \quad \mathcal{F} \longmapsto \mathcal{F} \otimes \mathcal{G}, \quad f_*, \ldots
\]
we get derived functors $R^i F$ measuring the failure of exactness.

Key observation: To compute $R^i F(\mathcal{F})$ you choose a resolution $\mathcal{F} \to I^\bullet$ and apply $F$ termwise. All choices of resolution give the same cohomology groups, but not canonically the same map between complexes.

This motivates keeping the whole resolution instead of just its cohomology.

\begin{itemize}
  \item A (cochain) complex in an abelian category $\mathcal A$ is
  \[
  \cdots \to A^{i-1} \xrightarrow{d^{i-1}} A^i \xrightarrow{d^i} A^{i+1} \to \cdots,
  \qquad d^{i}\circ d^{i-1}=0.
  \]
  \item A morphism of complexes is a collection of maps commuting with the differentials.
  \item Two morphisms $f,g:A^\bullet\to B^\bullet$ are \emph{homotopic} if
  \[
    f^i - g^i = d_B^{i-1}\circ h^i + h^{i+1}\circ d_A^i
  \]
  for some $h^i:A^i\to B^{i-1}$.
\end{itemize}

The \textbf{homotopy category} $K(\mathcal A)$ has:
\begin{itemize}
  \item objects = complexes in $\mathcal A$,
  \item morphisms = maps of complexes modulo homotopy.
\end{itemize}

This is natural because in a first course on algebraic topology, one of the key results is that homotopic maps induce the same map on cohomology.

A morphism of complexes $f:A^\bullet\to B^\bullet$ is a
\textbf{quasi-isomorphism} if it induces isomorphisms on all
cohomology sheaves:
\[
\mathcal H^i(A^\bullet) \xrightarrow{\;\cong\;} \mathcal H^i(B^\bullet)\quad\forall i.
\]

\textbf{Definition (very informally).}
The derived category $D(\mathcal A)$ is obtained from $K(\mathcal A)$
by \emph{localizing} at all quasi-isomorphisms:
\[
D(\mathcal A) := K(\mathcal A)[\text{quasi-isos}^{-1}].
\]
This means:
\begin{itemize}
  \item we freely adjoin formal inverses to all quasi-isomorphisms,
  \item two complexes that are quasi-isomorphic become isomorphic objects in $D(\mathcal A)$,
  \item morphisms can be represented by "roofs"
  \[
    A^\bullet \xleftarrow{\simeq} C^\bullet \to B^\bullet,
  \]
  with the left arrow a quasi-isomorphism.
\end{itemize}

For a smooth projective variety or complex manifold $X$, we write:
\[
D^b(X) := D^b_{\mathrm{coh}}(X)
\]
for the bounded derived category of coherent sheaves on $X$.
Objects are bounded complexes of coherent sheaves on $X$, up to quasi-isomorphism.

Major payoff: A fundamental fact is:
\[
\Ext^i_X(\mathcal F,\mathcal G)
\simeq
\Hom_{D^b(X)}(\mathcal F,\mathcal G[i]).
\]
In other words, all the $\Ext^i$-groups between two sheaves
are packaged as morphisms between shifts in $D^b(X)$.

Let $X$ be a smooth projective variety of dimension $n$ with canonical bundle $\omega_X$.
For us, Serre duality for vector bundles on a curve says that
\[
H^1(X, V)
\;\cong\;
H^0(X,V^*\otimes\omega_X)^*
\]
functorially in $V$. This extends to all coherent sheaves $\mathcal{F}$ and $\mathcal{G}$ on $X$:
\[
H^1(X,\mathcal F)
 \cong 
H^{0}(X,\mathcal F\otimes\omega_X)^*,
\]
because every coherent sheaf admits a finite resolution by vector bundles of length $\le 1$.

Let $\mathcal{F}$ be a coherent sheaf on a smooth projective curve $X$, and let $\mathcal{O}_X(1)$ be a very ample line bundle. For $m \gg 0$, the twisted sheaf $\mathcal{F}(m) := \mathcal{F} \otimes \mathcal{O}_X(m)$ is generated by global sections. So there is a surjection
\[
\mathcal{O}_X^{\oplus N} \twoheadrightarrow \mathcal{F}(m)
\]
for some $N$. Twisting back by $\mathcal{O}_X(-m)$, we get a surjection from a vector bundle onto $\mathcal{F}$:
\[
\mathcal{O}_X(-m)^{\oplus N} \twoheadrightarrow \mathcal{F}.
\]
Call this vector bundle $E_0 := \mathcal{O}_X(-m)^{\oplus N}$.

Let $E_1 := \ker(E_0 \twoheadrightarrow \mathcal{F})$. Then we have an exact sequence
\[
0 \longrightarrow E_1 \longrightarrow E_0 \longrightarrow \mathcal{F} \longrightarrow 0.
\]
Two key facts apply: a smooth projective curve is regular of dimension $1$, so its local rings $\mathcal{O}_{X,x}$ are discrete valuation rings (DVRs), and over a DVR, any finitely generated torsion-free module is free. The kernel $E_1$ is a coherent subsheaf of the vector bundle $E_0$, so it is torsion-free. A torsion-free coherent sheaf on a smooth curve is therefore locally free (since its stalk at each point is a torsion-free $\mathcal{O}_{X,x}$-module, hence free). So $E_1$ is also a vector bundle, giving us a short exact sequence of vector bundles resolving $\mathcal{F}$:
\[
0 \longrightarrow E_1 \longrightarrow E_0 \longrightarrow \mathcal{F} \longrightarrow 0.
\]

Because $\mathcal{F}$ has a finite locally free resolution, you can compute $\Ext^i_X(\mathcal{F}, \mathcal{G})$ by applying $\Hom(-, \mathcal{G})$ to that resolution. Since Serre duality is already known for vector bundles (the terms $E_0, E_1$), you can pass to $\mathcal{F}$ by homological algebra and the same duality formula extends to arbitrary coherent $\mathcal{F}$. 


The statement in fact generalizes to coherent sheaves on higher-dimensional varieties:
\[
\Ext^i_X(\mathcal F,\mathcal G)
\;\cong\;
\Ext^{n-i}_X(\mathcal G,\mathcal F\otimes\omega_X)^*
\]
functorially in $\mathcal F,\mathcal G$. This recovers the previous Serre duality since
\[H^i(X,\mathcal F) \cong \Ext^i_X(\mathcal O_X,\mathcal F).\]

In the language of derived categories, this duality is encoded by a
\textbf{Serre functor}:
\[
S_X(-) := -\otimes\omega_X[n] : D^b(X)\to D^b(X)
\]
such that
\[
\Hom_{D^b(X)}(A,B)
\;\cong\;
\Hom_{D^b(X)}(B,S_X A)^*
\]
naturally in $A,B$.

\begin{definition}
An object $P\in D^b(X)$ is a \emph{point object} if:
\begin{enumerate}
  \item $S_X(P)\simeq P[n]$ (it has the expected Serre duality of a codimension-$n$ object),
  \item $\Hom^{<0}(P,P)=0$,
  \item $\Hom^0(P,P)$ is a field (no nontrivial idempotents).
\end{enumerate}
\end{definition}

\begin{theorem}[Bondal--Orlov, special case]
If $\omega_X$ is ample or anti-ample, then the point objects in
$D^b(X)$ are exactly the shifts of skyscraper sheaves:
\[
P \text{ point object}
\quad\Longleftrightarrow\quad
P \simeq \mathcal O_x[m]
\quad\text{for some closed }x\in X,\ m\in\mathbb Z.
\]
\end{theorem}

\begin{definition}
An object $L\in D^b(X)$ is \emph{invertible} if for every point object $P$
there is an integer $s$ such that
\[
\Hom^s(L,P)\cong k,\qquad \Hom^i(L,P)=0\ \text{for }i\neq s.
\]
\end{definition}

\begin{theorem}
Under the same positivity assumption on $\omega_X$, the invertible objects
in $D^b(X)$ are exactly the shifts of line bundles:
\[
L \text{ invertible}
\quad\Longleftrightarrow\quad
L \simeq \mathcal L[t],\quad\mathcal L\in\Pic(X).
\]
\end{theorem}

\begin{theorem}[Bondal--Orlov reconstruction]
Let $X$ and $Y$ be smooth projective varieties over $k$ whose canonical
bundles $\omega_X,\omega_Y$ are ample or anti-ample. If there is a
triangulated equivalence
\[
F : D^b_{\mathrm{coh}}(X) \xrightarrow{\ \sim\ } D^b_{\mathrm{coh}}(Y),
\]
then $X$ and $Y$ are isomorphic as varieties.
\end{theorem}
\begin{proof}
An equivalence $F$ preserves Serre functors and Ext groups. Therefore $F$ sends point objects to point objects and invertible objects to invertible objects.

This induces a bijection between closed points $\{\mathcal{O}_x[m]\} \leftrightarrow \{\mathcal{O}_y[m]\}$ and an isomorphism of Picard groups (line bundles).

Using compatibility with the Serre functor, one checks that $F(\omega_X^k) \simeq \omega_Y^k$ for all $k$, and hence
\[
H^0(X,\omega_X^k) \cong H^0(Y,\omega_Y^k)
\]
as rings in $k$. If $\omega_X$ is ample or anti-ample, the canonical (or anti-canonical) ring
\[
R(X,\omega_X) := \bigoplus_{k \geq 0} H^0(X,\omega_X^k)
\]
determines $X$ as a projective variety:
\[
X \cong \Proj R(X,\omega_X), \qquad Y \cong \Proj R(Y,\omega_Y).
\]
Since the rings are isomorphic, $\Proj R(X,\omega_X) \cong \Proj R(Y,\omega_Y)$, so $X \cong Y$.
\end{proof}

\end{document}