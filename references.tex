\documentclass[12pt]{article}
\usepackage[english]{babel}
\usepackage[utf8x]{inputenc}
\usepackage[T1]{fontenc}
\usepackage{/Users/songye03/Desktop/math_tex/style/scribe}
\usepackage{listings}
\usepackage{tikz}
\usepackage{setspace}



\begin{document}

Songyu Ye 

sy459@cornell.edu

\today
\section{References}
This section is a complete collection of references that I have encountered in the wild. Please try to be detailed about title, author, year, and how/why/when you came across.
\begin{enumerate}[(a)]
    \item Serre Algebraic Coherent Sheaves, mentioned in Borcherds Schemes video
    \item Fulton Toric Varieties, mentioned by Allen
    \item Stackings and the W-cycles Conjecture, Louder and Wilton referenced in topology seminar by Daniel Wise, good potential read with Manning
    \item Ribbon knots are slice knots - Bernstein seminar, understand complimentary spheres and slice knots 
    \item Books on Stacks, Olsson \textit{Algebraic Spaces and Stacks}, DHL has a book, Vistoli Notes on Grothendieck topologies, fibered categories and descent theory
    \item Mumford Red book, Serre groupes algebriques et corps de classes
    \item 6ff, Olsson, Laslo, Zhang
    \item Hopf algebras and algebraic groups - books by Hochschild, Ferror Santos, Mahir Car
    \item Algebraic K-theory for squares categories, Jonathan Campbell, Josefien Kuijper, Mona Merling, Inna Zakharevich
    \item Algebraic K Theory of Spaces, Waldausen
    \item Hodge Theory Zian, Voison Hodge Theory, Kellar More Birational geometry of algebraic varieties
    \item Kac '85, Hodgkin 67, Brion '02, Brion '97 (for the classical approach), Goresky Kottwitz Macpherson, Braden Macpherson, Totaro 99, Heller Malagon Lopez 13, Kostant Kumar 86 90, 
 

    
\end{enumerate}

\section{Keywords}
This section is a list of buzzwords that I have heard thrown around by older mathematicians.

\begin{enumerate}[(a)]
    \item minimal model program, birational geometry, algebraic manifolds
    \item Reeb vector fields, contact forms, Hamiltonian vector fields, Seifert conjecture, Weinstein conjecture
    \item Ample line bundles, Derived Schur polynomials, Hodge-Riemann relations, complex Kahler manifold, Hodge index, Dually Lorentzian polynomials (June Huh)
    \item Algebraic Stacks
    \item Algebraic groups, Affine group schemes, reductive algebraic groups
    \item Linking number, Seifert surface, Reidemeister moves, virtual knots, Seifert's algorithm, Seifert genus, framing coefficient, blackboard framing, zero framing, 2 handlebodies, linking pairing
    \item 6 functor formalism, closed symmetric monoidal categories, derived category, quasicategory, the $\infty$-cat $\text{Corr(Var)}$, Nagata compactification, $\text{Cat}^L_\infty$ the infinity category of infinity categories, infinity category of correspondences, 1 category (ex. Span), triangulated category, stable infinity category, adjoint triples, abstract blowup, recollement, fiber sequences in the stable infinity category, infinity functors, cartesian fibration over Span category. "Lower shreek" Yuri knows alot about this stuff. As did Dan
    \item Gromov Witten invariants, sympletic topology and algebraic geometry, Deligne-Mumford moduli space of curves 
    \item Toric surfaces, K3 surfaces
    \item SW categories (dual to Waldhausen categories), admissible morphisms, subtractive sequences (functorial on fiber squares), S-construction for exact categories (arrows go other way for Waldhausen categories), K-theory of Waldhausen categories, the Grothendieck topology on a category, loop spaces, realizations of categories, Thomason model construction, \red{motivic homotopy theory }, K theory of stable infinity category, complete variety, hypersheaf, scissors congruence group, abstract blowup squares, Cd-structures, Bithner's presentation of the Grothendieck Ring of Varieties, K-theory of spectra, $K^\square_0(Var) = K^\square_0(Compl) = K^\square_0(SmComp)$, Sites
    \item Pamleve 6 equations, spherical braid on $n$ strands, Schksinger equation 1912, isomonodromy deformation, monodromy representation, Artin stack, character stack, Deligne Mumford curve, hypergeometric local systems, local convolution, derived pushforward, derived Algebraic Geometry, $d$-diml moduli space of $K3$ suraces, double covers of $\P^2$, finite complex reflection groups, Shimura varieties.
    \item commutative variety of a lie algebra, commutative multiplicative variety of a lie group, Knutson complete intersection of $\mf{gl}_n\times\mf{gl}_n$, DG Algebra,
    \item Chow ring, Borel Hopf Theory, generalized oriented cohomology theory, classifying space, fibration sequence, Borel map, Hopf algebra structure on cohomology, Formal Group Laws, Lazard Ring, formal structure algebra, algerbaic cobordism, Quillen formula for tensor of line bundles, universal formal group law,
            Two parameter Todd genus of Hirzebruch, duoidal categories, bimonoid, twisted coproduct, virtual ring, schubert class, tensor hat, Demazure formula, Peterson algebra, Schubert class, affine Hecke algebra, spherical Hecke algebra, Bach-table, $\Omega K$, $LK/T$.
    \item GKM graph, Homogeneous spaces, Equivariant cohomology
    \item homological algebra and representation theory and applies them to problems arising in algebraic geometry and topology. 
    \item McKay correspondence and K theory (Sept 6 2024)
    \item Okounkov and the theory of correspondences introduced by Alan Weinstein (Sept 6 2024)
\end{enumerate}

\section{Talks}
This section is a list of talks that I have listened to. Please include information about the title, who, when, where, and any link to materials for the talks.
    
\begin{enumerate}[(a)]
    \item Mike Hutchins (UC Berkeley) spoke about Reeb vector fields and periodic orbits, quantitative closing lemmas, Irie's Lemma at the 2023 Chelluri Lecture at Cornell University (https://pi.math.cornell.edu/m/node/11271) (Sept 2023) 
    \item Julius Ross (University of Illinois, Chicago) spoke about Hodge-Riemann property for Schur classes (https://pi.math.cornell.edu/m/node/11205) (Sept 2023) 
    \item MurphyKate Montee Carelton College spoke about Random Quotients of Free Products at Cornell Topology seminar (Sept 2023)
    \item Josefien Kuijper Stockholm University Six-functor formalisms are compactly supported spoke at AG seminar (Oct 2023)
    \item Daniel Litt (UT): Let $X_n$ be the set of conjugacy classes of n-tuples of 2x2 matrices whose product is the identity matrix. There is a natural braid group action on $X_n$, whose study goes back to work of Markoff in the late 19th century. The most basic question one can ask about this action, which dates to work of Painlevé, Fuchs, Schlesinger, and Garnier in the beginning of the 20th century, is: what are the finite orbits of this action? I'll explain the history of this question, as well as some recent work, joint with Lam and Landesman, in which we give a complete classification of such finite orbits, by algebro-geometric methods, when at least one of the matrices in question has infinite order. Time permitting, I'll discuss other variants of this question, whose answer relies on non-abelian Hodge theory and the Langlands program, and resolves conjectures of Esnault-Kerz, Budur-Wang, Kisin, and Whang (Oct 2023)
    \item Michael Brown (Auburn) - The classical Bernstein-Gel’fand-Gel’fand (BGG) correspondence is a derived equivalence between a polynomial ring and an exterior algebra. I will discuss a geometric version of the BGG correspondence involving the derived category of a toric variety. This is joint work with Daniel Erman.
    \item BUGCAT 2023 talks (pdf included in folder)
    \item Kirill Zaynullin - Oriented cohomology of a linear algebraic group vs localization in duoidal categories
    \item Yuri Berest - representation homology, Manning's PhD thesis involving representation variety, relationship to $\A^1$-homotopy theory, higher Hoschchilds homology, Kan Loop Group
    \item Keller Vande Bogert (Notre Dame): spoke about Borel Weil Bott in positive characterstic
    \item Victoria Hoskins - Motives of moduli spaces of vector bundles over curves
    \item Andres Puentes - Tropical Methods in A1 enumerative geometry

\end{enumerate}
\section{Dinner}
\begin{enumerate}
    \item Mia with Allen and AG speaker and Trevor Jones (5th year PhD of Dan'
    s)
    \item Simeon's with Allen and Daniel Stern and speaker
    \item Dan's house with Ritvik and speaker Daniel Litt
    \item Mia with Allen and Yuri and his wife and Daniel Stern. I had a nice steak and a cocktail
    and a wonton soup.
    \item Viva Taqueria with Keller, Ritvik, Allen, and Trevor. I had a nice burrito 
    and everybody else got a drink with their food. I was very tempted to order one but I had six shots of 
    tequila before I arrived. The dinner wasn't paid for by the department but Allen just
    paid for everybody.
\end{enumerate}

\section{Things to learn}
\subsection{Algebraic geometry}
\begin{enumerate}
    \item algebraic curves, in particular working hands on with algebraic varieties
    \item Mumford Lectures on Curves on an Algebraic Surface
\end{enumerate}
\subsection{Algebraic topology}

\section{Quotations}
\begin{enumerate}
    \item Yuri Berest: singular homology you know, that homology theory that Allen Hatcher wrote about in his book
    \item Yuri Berest: Marcelo studied coalgebras half of his life, why he didn't study cogroups? They are all trivial!
\end{enumerate}

\section{Potential advisors}
\begin{enumerate}[(a)]
    \item Kirsten Wickelgren (Duke University): Algebraic topology, algebraic geometry, and number theory. Homotopy theory and arithmetic geometry. A1-homotopy theory, motives, K-theory, equivariant homotopy theory, Grothendieck’s anabelian program.
    \item Tom Bachhman (Germany): Motivic homotopy theory, motivic cohomology, algebraic geometry Algebraic and hermitian K-theory, higher category theory, tt-categories
    \item Isabel Vogt (Brown): Brill Noether theory
    \item Frances Kirwan (Oxford): Intersection homology, Geometric invariant theory, moduli spaces, symplectic geometry, algebraic geometry
\end{enumerate}

\section{Journaling}
\subsection{Dec 3 2023}
\doublespacing
Many things happened this past week that have been quite overwhelming. I went to the Oliver Club this past Thursday.
I had a meltdown in Dan's office on Friday morning. I had a second meltdown in Birgit's office that same afternoon.
I also found out that I was not going to fail 3410. I met my groupmates for the 3410 final project and I honestly had a great time being with them.

\hfill

The Oliver Club happened this past Thursday and the speaker spoke largely about these interesting quantum combinatorial formulas and the homological,
derived AG machinery which stood behind them. Before the talk began, I went to get some cheese and out of nowhere, Tara asked me: "What are you doing this summer? Do you want to stay here?
Do you need money?" I felt very good about that. It felt better than getting into an REU.

\hfill

The talk started and Allen ended up sitting next to me. It was the second or third time this has happened and without fail, he pulled out his phone during the talk
and started showing me pictures and writing on his notepad and angling it toward me. This time, he showed me a picture of Freeman Dyson and wrote down an extension of a theorem that 
the speaker had mentioned, along with an example computed out. He saw me writing down the theorem in response to what the speaker had said. At some point during the talk, the speaker 
writes down $H^*$ for cohomology of a complex of simplicial abelian groups and carries on. In a delayed fashion as always, Hubbard stops the speaker and asks, "that $H^*$ you have there, 
what space are you taking the singular cohomology of?" The speaker is confused because he does not care about singular cohomology and has no idea why Hubbard brought it up. Allen speaks up and 
says "John that's not singular cohomology, it's a different cohomology" to which Hubbard says with a big goofy smile on his face "Well I thought I knew about cohomology but I guess I was wrong!"

\hfill

It was quite interesting having dinner with Allen this past Thursday. I learned alot of things about him
and the department. He has an ex wife. His oldest son wants to study physics at Cornell. He named Joel 
as his most successful student and I was not suprised when I learned that this was 
during the time he taught at UC Berkeley. I am not sure why he made that remark.

\hfill

A problem that the Cornell math department has been facing is that there is a lack of money. 
I do not understand the exact finances but the want for money can clearly be felt thoughout the department.
For example I emailed the library to inquire if the library could purchase a book. I received a reply saying sorry, that 
isn't going to be possible. The curious thing is that I later saw that the print copy was not even available for sale until 2024.
What this means is that they didn't even check. Also Cornell is the only "top" math department not accepting graduate students this year.
I was told there was only funding for ten students over the course of the next two years (owing ultimately to the fact that
some of the elder graduate students (think 6th and 7th years) are graduating exceptionally delayed). I think this is quite absurd, as in there are no 
other departments which are suffering from this circumstance.

\hfill

Allen also made some remarks about job talks and told a story about an individual who came to deliver a job talk in a number thoery seminar at Cornell.
By the second sentence of the talk, he was already using etale cohomology. A few more sentences in and Hubbard interjects: "What is that $H^*$ there?" to which
the speaker replies "Oh, it's just etale cohomology" and then proceeds. Hubbard walked out of the talk and that was the only time Allen seen Hubbard walk out of a talk in 20 years.
Going out to dinner with Allen was a lot of fun. He's an entertertaining man with many stories. He has an enthusiasm for life which many of his younger collegues seem to have already lost.
I like the way he dresses. I like the stupid ass treadmill in his office. I like how he was a world record juggler. I liked hanging out with him at dinner and I'm glad that he seems to be fond of me.

\hfill

I met with Dan on Friday and we came to the topic of what sort of document Dan was expecting us to hand in. Holden suggested the idea that we write something together and turn it in.
The problem with that is Holden has not done anything for this project this semester. Every opportunity Dan provided for us to do math, I was the only person that ever did anything at all.
Naturally I do not want to write something with Holden. However I did not know how quite to say this. I ended up getting quite frustrated and noticing something, Dan insisted that we speak privately.
I sort of told him that I was just upset with the way progress had been going. Holden came back and we did math for the rest of the hour, and then after Holden left, Dan and I talked for nearly another hour
 about continuing next semester. In particular he asked if I was not content with the topic or if I was unhappy that we didn't prove a shiny theorem. I told him that I had a hard time understanding
 the progress which I had made this semester (this is the importance of taking notes) and I was unsatisfied with that. He responded that I don't need to worry about offending him or 
 disappointing him, and that he was in fact very happy with the progress that we had made thus far. I would like to tell him that I would like to work with him individually, because I do 
 not like the burden of working with other people. I do not know how to tell him that. Also it does not seem logistically possible the way
  Curious how this is the way I feel toward math, but I am very content working in groups on the 3410 coding assignment.

  \hfill

Birgit is retiring this semester. Everytime I talk with her I feel so stupid. I very much look up to her as well. She was the first female math professor at Cornell. She did her 
undergraduate in Germany, where she tells me that there were no such things as textbooks. Students wrote and rewrote their notes and come two years,
you would really need to study your notes because there was a big cumulative exam. When she went to MIT in the 1970s, she was shocked that the students had textbooks.
This past meeting I was deeply disappointed by my perfomance. I usually am every week, but this week I couldn't even fill up the board before I got terribly confused and betrayed 
my lack of understanding. She was very nice about it but it's clear that I really am not doing good enough.
\section{Saturday Dec 9 2023}
Yesterday I met with Birgit for the last time. I prepared some math to talk about but when I entered her office, she asked me to tell her what I learned this semester. I blurted out
some nonsense about the classification of complex semisimple Lie Algebras, starting with the example of $\sl(2,\C)$ and introducing the theory of positive roots (already highly nontrivial as she remarked)
and the theory of the highest weight. We also touched on Lie algbera cohomology, infinite dimensional representations, noncompact groups (her favorite), and we even talked 
a little bit about machine learning. She told me some ludicrous stories.

\hfill

She began with her life in Germany. Since she was in high school, she was the only girl in all of her classes. Even at that time, there were no girls in the math and science academies, she remarked.
But she had the fortune of meeting all of these math physics people at conferences. When it came time for her to go to graduate school, she was 24 years old. She was decided between going to
MIT to work with Bertram Kostant, whom she had become acquainted with either directly at a conference or via a friend, or to stay in Germany, as it was very common at the time in Europe
to spend your entire academic career at the same institution.

\hfill

She decided to go to MIT. \textbf{ It was the only graduate school she applied to.} Upon her arrival, she was greeted by Kostant, who immediately brought her to an office,
inside sitting a stout, serious man. Kostant greeted him, "Irving, this is Birgit" and then he closed the door. And so Birgit met Irving Segal for the first time. 
And she went on, describing her affairs at MIT. She described the weekly Lie groups seminar which met at Kostant's house, regular attendees including Segal, David Vogan, etc. Again she was the only woman. 

\hfill

The expectation at MIT was that you would just figure everything out by yourself. Birgit tells me that she only saw Kostant every couple months. She would have her chance to speak with him if 
she were lucky enough to catch him in the mathematics library. In my mind's eye, I see a young woman meeting with her professor in the dim light of the MIT mathematics library, in the room 
where they kept the journals, locked up behind gates and only mathematics students could access. Birgit remarks that Kostant was a very distant advisor. When she handed in her thesis, 
Kostant immediately handed it to David Vogan and told him to check if it was correct. Vogan said it was and so Birgit graduated. It was only at her PhD defense that Kostant finally heard of 
what Birgit had been working on for the past four years. 

\hfill

We talked about how she felt being a woman in mathematics, in particular being the only woman in the room for the better part of her career (she remarked that things began to change at some point)
She said it didn't bother her. In particular she said that she had never at any point felt that being a woman had hurt her career in any way, and in fact she felt that
it had benefitted her in her ability to stand out. The way she saw it, she worked closely with many men who had these nasty perspectives about women in math, but they never were nasty toward her.
In fact, by being the only woman in a field of men, and a respected mathematician at that, she felt that it helped her stand out, particularly in the early years of her career before entering MIT.
She didn't say it, but my impression was that they were satisfied with her mathematics and so with her individual as well. What a curious perspective. It really makes you rethink
modern feminism.

\hfill

We talked at length about how the state of affairs of mathematics has changed over the last fourty years. There is a lot more administration, a lot more overhead.
When she was at MIT, professors would maybe teach one course to a small group of math graduate students, but otherwise they were wholly devoted to their research.
I get the impression from the way Birgit spoke that many of these men were irritated by anything
that was not mathematics. And it seemed to me that she felt the same. Not that she disliked students or teaching, but of course she became a mathematician to 
do mathematics. One climbs so high that it becomes a chore to look down and help the others.

\hfill

And so my takeaway is that mathematics is not about who one's advisor is, the trajectory of one's career lies solely in their own hands. It is your responsibility to make something happen.
It doesn't matter how brilliant your advisor is because he/she is their own individual, their own mathematician, and it does not matter how close you are to he or her, because their duty is to themself.
They may help you, they may guide you, but they cannot think for you. They cannot create for you. And it is your job as the student to create things to interest them, because if you
can even interest them for a moment, these are individuals whose lives are guided by their interests, so if you can invoke a genuine interest in them, then they will do their best to help you.

\hfill

As another remark on the teaching, the teaching load has significantly increased as well as other things that just make the career path less pure 
than it used to be. I use pure in the sense that being a mathematician should ultimately be about doing mathematics, not teaching or attending meethings.
Another good example to look at is postdoctoral fellowships. Birgit remarks that doing a postdoc was not a thing. After you graduated from your PhD,
you became faculty and then all of a sudden, you were on your own. The thought of that scares me. When there is nothing above me to look to, only 
the hands of my companions as we feel the way through the dark together.


\hfill

She also made a remark about young people. She said that in recent years there has been increasing amount of support for young people, in particular graduate students, in mathematics. She pointed the example
of how nowadays, conference funding is generally reserved for young people and senior faculty have to pay out of their salaries (she reminded me that not all mathematicians are so fortunate to have grant money).
\red{I should really consider how I use Tara's grant money this spring}
Now this made a lot of sense to me. Senior faculty have more access to resources. Why shouldn't confernece funding go to young people? But the way Birgit spoke on the matter made be realize that this idea which
was so natural to me was the complete opposite for her, because for her entire career senior faculty had always been prioritized. And this sort ties into what I said earlier about the teaching responsibility.
Mathematics has recently taken this perspective that young people deserve attention, time, and money from universities and mathematicians. 
Mathematics used to alot more of the gatekept attitude that one must figure it out by themselves, but sort of it has become more open and welcoming now.
This is of course good for many number of reasons. However you do run into a lot of issues. Of course there is this burden on faculty. 
There is also this attitude amongst the graduate students here at Cornell, in line with sort of attitude
\subsection{Dec 11 2023}
Treat this trip as though if you do not see these places now, then there will not be 
another chance before you die.
\subsection{Dec 12 2023}
Today was a wonderful day for poker. I got to the Guitar Hotel around 7:30 AM and sat down at 8.
I played for about 2 hours and won 1500. Afterwards I snuck into the pool (hotel guests only) by hopping a rope.
Once I was in the pool section, I tried to get in but the lifeguard stopped me, telling me
that I needed to get a wristband from the front in order to get in the water. I went up to these girls
working the front desk and told them that I lost my wristband, and without checking my reservation
they just put one on my wrist and I was in. The pool was okay, very beautiful but in terms of the diversity of the pool, not so good.
The deepest the water got was 4 ft, pretty disappointing. The slide was a lot of fun, I went down three times.

\hfill

While I was sitting beside the pool, eating my overpriced burger and drinking my overpriced cocktail (50 dollars total holy fuck)
I thought about two things. Confidence works, and I can't believe that I was here in paradise when
two days ago, I was miserable studying for the 3410 exam. As a follow up, I thought about how to approach beautiful women with confidence. Believing in 
the magic of confidence makes me more confident.

\hfill 

Afterwards, I went back to the poker room. This time, it really felt like I was sunrunning.
My bluffs were getting through, the runouts were perfect, I was connecting with flops everytime I raised preflop. 
However, on the last hand my luck ran out (hence it was my last hand, been practicing discipline (or maybe I'm just a pussy)).
In particular, I bet 300 into 500 on the river with AAKx, with the K(diamond) on xx(diamond)(diamond)(diamond) (straighty flushy board)
and the guy called with some junk two pair. It was a terrible call and I think my bluff was pretty standard,
but I need to work on my live tells. Hence I just got up, in the green for 855 that session. Afterwards, I went to Hollywood Beach
and walked up the boardwalk, stopping at a couple restaurants and bars. I talked to a bartender who
told me traveling solo was definitely the move and that people can only hold you back. She told me
to go check out the bars in Fort Lauderdale and to get on an airboat tour.

\hfill

Finally we went to the Big Easy Casino. I played the 1/2 5 card for about 30 minutes and won 35 dollars.
I racked up for two reasons. First I wasn't having fun, and the second, I was too short to really play 5 card
(my opinion is that it is a game that really only works when everybody is super deep) and I wasn't interested in
buying in for more. I won't go back and I don't think I will play more 5 card, but it was good to try.

\subsection{Dec 16 2023}
Landed in Cancun this morning. Immediately greeted by people trying to get my money.
Hopped on an ADO bus striaght to Tulum. Went to the ruins and saw so many beautiful sights and 
beautiful women. Went to the beach and saw the ruins. Checked in at hostel for 300 pesos/night. 
Walked around and ate some tacos and tortas, bought ridiculously sized beer bottle. Saw a night market, ate an elote.

\subsection{Dec 17 2023}
Bought a guided tour for 1500 mexican pesos last night while I was drunk wandering the streets. Overall it was good but 
at first I got upset because I thought I could've done things for cheaper and by myself, but 
after completing the trip I wasn't so convinced. I don't think I will do another one though, I realized
the value of controlling your own itenerary and not having to wait around for people (that was the whole point of traveling solo!!)

\hfill

We first went to eat a lunch buffet at some Mayan local village. It was ok, although this is when I first got
upset because we were wasting time seeing things that I wasn't interested in. My friend Sue reminded me that
we are here to support the locals and that you should just enjoy it. I lightened up after that and the rest of 
the tour was great. We then went onto Chitzen Itza and saw the temples and the ruins of the city. I really
wish we had more time there but 2 hours was enough to see the vast majority of the interesting stuff I think. We then went
to a cenote (sinkhole swimming pool) and swam there. It was really beautiful. Finally we stopped in Valladolid
and I haggled with the merchants. I almost secured a pretty good deal (1600 to 300 pesos) for 4 tall shot glasses but then 
when I went to pay, I realized I miscounted the cash I had on me and the deal was off.

\hfill

I got dinner with Sue when we got back. She was a very talkitive and friendly lady. Her husband died during COVID and her father
had died the week before I met her. She was in Tulum to go diving. She spend 1200 USD on 5 days of diving (+ certification)
and said she had the time of her life. She urged me to do the same and I considered it, I think I will decide against it and move onto
the next city. Tulum was fun but a little more touristy than I had hoped. 

\subsection{Dec 18 2023}
I will head to Balacar unless I can find a collectivo heading to Sian Kaan, in which case I may spend another night in Tulum.

Or at least that's what I thought. I showed up at the bus station and I changed my mind last minute. I got some ceviche and grilled octopus for 585 pesos.
I ended up going to Merida to the west. I got a hostel for 250 pesos/night. I got on a tour bus and saw some lovely cruches and parks and homes. 
At night I saw some live music. I didn't understand the words but you don't need to in order to feel the music. I recognized one of the melodies 
and I thought it was so funny how connected strangers can be. At the concert, a cop approached me and told me I needed to pour out my beer.
I didn't want any trouble so of course I did what he said. On my way home, I saw some nice Mayan women dancing in the park, with crowds surrounding them.
Also I ate a simple hamburger, the meat did not taste like meat and somehow the whole thing tasted like tacos. Also all of these hostels have
bed bugs, I need to clean my stuff when I get home.

\subsection{Dec 19 2023}
The day started by me looking for a rental car. This was absolutely ridiculous given that I don't have a driver's license with me. That was no obstacle however. 
The first place I went to, everything was going smoothly until he asked for my license. 
When I told him I didn’t have one, he put his hands to his head and looked absolutely rattled. 
He made a couple phone calls and next thing you know, he was driving me to another car rental place where I can rent “sin licensia.” 
He asked me, what are you going to do if the police pull you over, and I told him “my friend said you can just pay the police” and
 he just laughed and took me to the car rental. It was insane because I only understood half of his sentence, my Spanish was so broken, and every other minute we would have to pass google translate back and forth. It was so fucking awesome. I didn’t end up renting a car, but that’s because I couldn’t put down the 40k pesos deposit. I am positive that I would have gotten the car if my deposit had gone through.

\hfill

Afterwards I figured out how to take a bus to Uxmal. I bought the ticket and in the meantime, I checked out some of the nice villas and museums in the city center (zocalo). I got on the bus and I met two really cool women, Joanna and Seohee. Joanna was an Austrian-born woman in her late 20s early 30s living in Paris. She was in Merida traveling and we spoke a lot of solo travel and our experiences thus far. We met because we were both the last ones on the bus and there were no seats, and so she sat in a pocket right next to the windshield (cracked by the way) and I sat on the steps that people use to board the bus. That shit would not fly in the United States, but in Mexico it was completely standard.

\hfill

Seohee was a Korean college student and she was 21 years old. I met her while waiting for Joanna to use the restroom. She is studying abroad in Mexico City. She is majoring in Spanish language and culture and she wants to do research in Spanish for graduate school. She was traveling, flying into Cancun and going to Valladolid and then Merida. 

\hfill

The three of us walked around the ruins and took some photos. After we got back to Merida, Joanna parted ways with us and Seohee and I went to a museum, hung out for 15 minutes, and then hugged goodbye. She was really cute and I entertained the idea of making some sort of move, but decided against it because it was her last day of travel and her life is so far removed from mine. But it was nice spending some time with her, and even though we will never see each other again, I think that the time we spent together still meant something. That is the wonderful thing about traveling solo. You meet all of these wonderful people and even though they won’t be a part of your lives, even if you only know them for a couple of hours, the imprints which you leave on one another are meaningful and last much longer than the brief time you were in each others company. It was such a lovely experience.

\hfill

Afterwards I went and got some Yucatecan food, I had some chicken called Pollo Pibil and a taco-resembling dish called Salbut, the meat inside was (very Yucatecan famous) cochinita pobil.
Throughout the day I had some Tres leches cake, some limon peanuts, a Mexican chocolate treat, a piece of chicken that Joanna gave me, some tripe soup Mondongo that was too intense for me, and a Modelo.
\end{document}